\documentclass{lehramt-informatik-examen}

\begin{document}
% 46110

% 46111

% 46112

% 46113

% 46114
% „Aufgabe 2“
% „Aufgabe 3“

% 46115
% „5. Datenstrukturen und Algorithmen: Binäre Suchbäume und AVL-Bäume“
% „Aufgabe 1“
% „Aufgabe 4“
% „„Streuspeicherung““
% „Aufgabe 10: Graphen I“
% „Aufgabe 6“
% „Aufgabe 7 Heap und binärer Suchbaum und AVL Baum“
% „Frühjahr 2014 (46115) - Thema 1 Aufgabe 8“
% „Frühjahr 2014 (46115) - Thema 2 Aufgabe 3“
% „Aufgabe 4“
% „Aufgabe 3“
% „Aufgabe 8: Hashing“
% „Aufgabe 4“
% „Aufgabe 5“
% „Aufgabe 8“
% „Aufgabe 2“
% „Aufgabe 3“
% „Aufgabe 1: Reguläre Sprachen“
% „Aufgabe 4“
% „Aufgabe 2“
% „Aufgabe 7: Dynamische Programmierung“
% „Aufgabe 3“
% „Aufgabe 6:“
% „Aufgabe 7“
% „Aufgabe zum Mergesort“
% „Aufgabe“
% „Aufgabe 12: Graphen III“
% „Aufgabe 1“
% „Aufgabe 2“
% „Aufgabe 4“
% „Aufgabe 6 (Stacks)“
% „Aufgabe 1“
% „Aufgabe 2 (Rechtslineare Grammatik)“
% „Aufgabe 7 (Heapify)“
% „Aufgabe 8“
% „Aufgabe 1“
% „Aufgabe 6“
% „Aufgabe 7“
% „Aufgabe 1“
% „Aufgabe 2“
% „Aufgabe 3“
% „Aufgabe 4“
% „Aufgabe“
% „Aufgabe 2“
% „Aufgabe 3“
% „Aufgabe 5“
% „Aufgabe 1“
% „Aufgabe 2“
% „Aufgabe 3“
% „Aufgabe 1“
% „Aufgabe 2“
% „Aufgabe 3“
% „Aufgabe 4“

% 46116
% „Aufgabe 3“
% „Hotel-Verwaltung“
% „2. modernen Softwaretechnologie“
% „Aufgabe“
% „2. Anfragen“
% „Aufgabe 1“
% „Aufgabe 1: „Formale Verifikation““
% „Aufgabe 2: Relationale Algebra“
% „Aufgabe 3: SQL“
% „Aufgabe 1: Allgemeine SWT, Vorgehensmodelle und Requirements“
% „Aufgabe 2: Vorgehensmodelle“
% „Aufgabe 3: UML Diagramme in der Anwendung“
% „3. Projektmanagement“
% „1. Relationale Anfragesprachen“
% „Aufgabe 3“
% „3. Projektmanagement“
% „Aufgabe 1“
% „Aufgabe 4: SQL (Check-Up)“
% „Aufgabe 2: „Testen““
% „Aufgabe 3“
% „Aufgabe 5“
% „Aufgabe 2“
% „Aufgabe 2: Modellierung von Interaktionen durch Sequenzdiagramme“
% „Aufgabe 4:“
% „Aufgabe 2“
% „Aufgabe 3“
% „Aufgabe 5: Projektmanagement“
% „Aufgabe 3: Physische Datenorganisation, TID“
% „Aufgabe 4: Qualitätssicherung“
% „Aufgabe 3: Entwurfsmuster“
% „Aufgabe 4“
% „Aufgabe 3 (Verhaltens-Modellierung mit Zustandsdiagrammen)“
% „Aufgabe 4“
% „Aufgabe 2: ER-Diagramm“
% „Aufgabe 4“
% „Aufgabe 2“
% „Aufgabe 3“
% „Aufgabe 1: ER-Modell“

% 46118

% 46119

% 46121

% 66110

% 66111

% 66112
% „Aufgabe 5“
% „Aufgabe 8“
% „Aufgabe 14: Listen“
% „Aufgabe 3: Hashing“

% 66113
% „Aufgabe 5: SQL“
% „Aufgabe 3: SQL“

% 66114

% 66115
% „Aufgabe 7“
% „Aufgabe 1“
% „Aufgabe 3: Hashing“
% „Aufgabe 8“
% „Aufgabe 3“
% „Aufgabe 4“
% „Aufgabe 7“
% „Aufgabe 1“
% „Aufgabe 4“
% „Aufgabe 4: Komplexität“
% „Aufgabe 7“
% „Aufgabe 8“
% „Aufgabe 2“
% „Aufgabe 6“
% „Aufgabe 5“
% „Aufgabe“
% „Aufgabe 3“
% „7. Aufgabe: Heap und binärer Suchbaum“
% „8. Aufgabe: AVL-Bäume“
% „9. Aufgabe: Dijkstra“
% „Aufgabe 1: „Rekursion und Induktion““
% „Aufgabe 2“
% „Aufgabe 3“
% „Aufgabe 5“
% „Aufgabe 7“
% „Aufgabe 5“
% „Aufgabe 6“
% „Aufgabe 1“
% „Aufgabe 4“
% „Aufgabe 4“
% „Aufgabe 7“
% „Aufgabe 1“
% „Aufgabe 2“
% „4. Turingmaschinen“
% „Komplexität“
% „Aufgabe 6“
% „Verständnis formale Sprachen“
% „Verständnis Berechenbarkeitstheorie“
% „Verständnis Komplexitätstheorie“
% „4. Hashing“
% „Dijkstra Algorithmus“
% „Verständnis Suchbäume“
% „Aufgabe“
% „Aufgabe 2“
% „Teilaufgabe IV“
% „Aufgabe 3“
% „Aufgabe 7“
% „Aufgabe 1 (Graphalgorithmen)“
% „Aufgabe 2“
% „Aufgaben 3“
% „Aufgabe 4: Vollständige Induktion“
% „Aufgabe 5“
% „Aufgabe 6“
% „Aufgabe 2“
% „Aufgabe 3“
% „Aufgabe 2“
% „Aufgabe 3“
% „Aufgabe 8: Greedy-Färben von Intervallen“
% „Aufgabe 5“
% „Staatsexamenaufgabe zu Binärbaum, Halde, AVL“
% „Aufgabe 1“
% „Aufgabe 3“
% „Aufgabe 6“
% „Aufgabe 7“
% „Aufgabe 8“
% „Aufgabe 9:“
% „Aufgabe 10:“
% „Aufgabe 11“
% „Aufgabe 3“
% „Aufgabe 5 (AVL Bäume)“
% „Aufgabe 4“
% „Aufgabe 6 (Backtracking)“
% „Aufgabe 8“
% „Aufgabe 1“
% „Aufgabe 2“
% „Aufgabe 3“
% „Aufgabe 4“
% „Aufgabe 5“
% „Aufgabe 6 (Algorithmen und Datenstrukturen)“
% „Aufgabe 1“
% „Aufgabe 4“
% „Aufgabe 5 (Sortierverfahren)“
% „Aufgabe 6 (O-Notation)“
% „Aufgabe 7 (Bäume)“
% „Aufgabe 1“
% „Aufgabe 6“
% „Aufgabe 7 (AVL-Bäume)“
% „Aufgabe 8 (Minimaler Spannbaum)“
% „Aufgabe 9 (Hashing)“
% „Aufgabe 2“
% „Aufgabe 3“
% „Aufgabe 3“
% „Aufgabe 4“
% „Aufgabe 8“
% „Aufgabe 10“
% „Aufgabe 1“
% „Aufgabe 2“
% „Aufgabe 3“
% „Aufgabe 4“
% „Aufgabe 5“
% „Aufgabe 1“
% „Aufgabe 2“
% „Aufgabe 3“
% „Aufgabe 4“
% „Aufgabe 5“
% „Aufgabe 1“
% „Aufgabe 2“
% „Aufgabe 1“
% „Aufgabe 2“
% „Aufgabe 3“
% „Aufgabe 4“
% „Aufgabe 5“
% „Aufgabe 1“
% „Aufgabe 2“
% „Aufgabe 1“
% „Aufgabe 2“
% „Aufgabe 3“
% „Aufgabe 4“
% „Aufgabe 1“
% „Aufgabe 2“
% „Aufgabe 3“
% „Aufgabe 4“
% „Aufgabe 2“
% „Aufgabe 3“
% „Aufgabe 5“

% 66116
% „Entity-Relationship-Modell“
% „Normalisierung“
% „SQL“
% „1 Objektorientierung“
% „Aufgabe 2“
% „Aufgabe 4“
% „Aufgabe 2: Kontrollflussorientiertes Testen“
% „Aufgabe 5“
% „Aufgabe“
% „Aufgabe 2“
% „Aufgabe 3“
% „Aufgabe 2“
% „Aufgabe 3“
% „Aufgabe 3“
% „Aufgabe 2“
% „Aufgabe 3“
% „1. Modellierung“
% „2. Normalformen“
% „Aufgabe 3“
% „Aufgabe 6: B-Baum“
% „Aufgabe 3“
% „Forstverwaltung“
% „Aufgabe 2“
% „Aufgabe 2“
% „Projektmanagement“
% „Aufgabe 1: SQL“
% „Aufgabe 5“
% „Aufgabe 1“
% „Aufgabe 2“
% „Aufgabe 3“
% „Aufgabe 2“
% „Aufgabe 2“
% „Aufgabe 2“
% „Aufgabe 5“
% „Aufgabe 1“
% „Aufgabe 4“
% „5. Schemadefinition“
% „6. Relationale Anfragen in SQL“
% „Aufgabe 3“
% „Aufgabe 1“
% „Aufgabe 2“
% „Aufgabe 1“
% „Aufgabe 4“
% „Aufgabe 6“
% „Aufgabe 1“
% „Aufgabe 2“
% „Aufgabe 2“
% „Aufgabe 4“
% „Aufgabe 3“
% „Aufgabe 2“
% „Aufgabe 3“
% „Aufgabe 1“
% „Aufgabe 2“
% „Aufgabe 3“
% „Aufgabe 4“
% „Aufgabe 5“
% „Aufgabe 1“
% „Aufgabe 3“
% „Aufgabe 5“
% „Aufgabe 6“
% „Aufgabe 1“
% „Aufgabe 2“
% „Aufgabe 3“
% „Aufgabe 1“
% „Aufgabe 2“
% „Aufgabe 3“
% „Aufgabe 4: (Entwurfsmuster)“
% „Aufgabe 1“
% „Aufgabe 2: (ER-Modellierung)“
% „Aufgabe 3“
% „Aufgabe 4“
% „Aufgabe 5“
% „Aufgabe 1“
% „Aufgabe 2“
% „Aufgabe 3“
% „Aufgabe 1“
% „Aufgabe 2“
% „Aufgabe 3“
% „Aufgabe 4“
% „Aufgabe 5“
% „Aufgabe 6“
% „Aufgabe 7“
% „Aufgabe 1“
% „Aufgabe 1“
% „Aufgabe 2“
% „Aufgabe 3“
% „Aufgabe 4“
% „Aufgabe 5“
% „Aufgabe 6“
% „Aufgabe 7“
% „Aufgabe 8“
% „Aufgabe 4“
% „Aufgabe 5“
% „Aufgabe 1“
% „Aufgabe 2“
% „Aufgabe 3“
% „Aufgabe 4“
% „Aufgabe 1“
% „Aufgabe 2“
% „Aufgabe 3“
% „Aufgabe 4“
% „Aufgabe 5“
% „Aufgabe 1“
% „Aufgabe 2“
% „Aufgabe 3“
% „Aufgabe 4“
% „Aufgabe 5“
% „Aufgabe 6“
% „Aufgabe 2“
% „Aufgabe 3“
% „Aufgabe 4“
% „Aufgabe 1“
% „Aufgabe 2“
% „Aufgabe 3“
% „Aufgabe 4“
% „Aufgabe 5“
% „Aufgabe 6“
% „Aufgabe“
% „Aufgabe 8“
% „Aufgabe 9“
% „Aufgabe 10“
% „Aufgabe 11“
% „Aufgabe 12“
% „Aufgabe 1“
% „Aufgabe 2“
% „Aufgabe 3“
% „Aufgabe 4“
% „Aufgabe 5“
% „Aufgabe 6“
% „Aufgabe 1“
% „Aufgabe 2“
% „Aufgabe 4“
% „Aufgabe 5“
% „Aufgabe 1“
% „Aufgabe 2“
% „Aufgabe 3“
% „Aufgabe 4“
% „Aufgabe 5“
% „Aufgabe 6“
% „Aufgabe 7“

% 66118
% „Thema Nr. 2“
% „Thema Nr. 3“
\end{document}
