\documentclass{lehramt-informatik-aufgabe}
\liLadePakete{komplexitaetstheorie}
\begin{document}

\section{Post‘sches Korrespondenzproblem (PCP) und Modifiziertes Post‘sches Korrespondenzproblem (MPCP)
\index{Entscheidbarkeit}
\footcite[Seite 46-48]{theo:fs:4}}

Post‘sches Korrespondenzproblem\footcite[Seite 326-330]{hoffmann}

\liProblemBeschreibung{PCP}
{$(x_1, y_1), \dots, (x_n, y_n)$ mit $x_1, y_1 \in \Sigma +$}
{Gibt es eine Folge $i_1, i_2, \dots, i_k \in \mathbb{N}$ mit der Eigenschaft
$x_{i_1} x_{i_2} \dots x_{i_k} = y_{i_1} y_{i_2} \dots y_{i_k}$}

Modifiziertes Post‘sches Korrespondenzproblem

\liProblemBeschreibung{MPCP}
{$(x_1, y_1), \dots, (x_n, y_n)$ mit $x_1, y_1 \in \Sigma +$}
{Gibt es eine Folge $i_1, i_2, \dots, i_k \in \mathbb{N}$ mit der Eigenschaft
$x_{i_1} x_{i_2} \dots x_{i_k} = y_{i_1} y_{i_2} \dots y_{i_k}$}

\end{document}
