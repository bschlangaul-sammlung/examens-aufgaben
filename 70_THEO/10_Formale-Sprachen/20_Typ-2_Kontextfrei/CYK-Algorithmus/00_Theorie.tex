\documentclass{lehramt-informatik-haupt}
\liLadePakete{syntax}
\begin{document}
\def\t#1#2{$t_{#1,#2}$}

\chapter{CYK-Algorithmus}
\index{CYK-Algorithmus}

\begin{liQuellen}
\item \cite[Seite 45-75]{theo:fs:2}
\item \cite[Seite 186-188]{hoffmann}
\item \cite{wiki:cyk}
\end{liQuellen}

\section{Online-Tools}

\begin{itemize}
\item \url{https://www.xarg.org/tools/cyk-algorithm/}
\end{itemize}

\noindent
Der Algorithmus ist nach Cocke, Younger, Kasami benannt. Er dient dazu
das \memph{Wortproblem} für eine Sprache zu entscheiden, die durch eine
CNF-Grammatik (\memph{Chomsky-Normalform}) gegeben ist.

Bilde zu einem Wort $w$ alle Teilwörter der Länge $1, 2, 3, \dots$ und
bestimme die Ableitbarkeit von Variablen. Ist das „Teilwort“ der Länge
$|w|$ vom Startsymbol ableitbar, so gehört $w$ zur Sprache.

\begin{center}
\begin{tabular}{|c||c|c|c|c|c|}
\hline
    & a    & b    & a    & b    & a \\
\hline
i/j & 1    & 2    & 3    & 4    & 5 \\\hline\hline
1   & \t11 & \t12 & \t13 & \t14 & \t15 \\\cline{1-6}
2   & \t21 & \t22 & \t23 & \t24 \\\cline{1-5}
3   & \t31 & \t32 & \t33 \\\cline{1-4}
4   & \t41 & \t42 \\\cline{1-3}
5   & \t51 \\\cline{1-2}
\end{tabular}
\end{center}

\begin{itemize}

\item Erste Tabellenzeile aus den Produktionsregeln ableiten.

\item Ab der zweiten Tabellenzeile, aus den bereits aufgefüllten Zellen
ableiten

\item Variablen kombinieren (Reihenfolge beachten!)

Zum Beispiel: \t11 (AB) \t12 (CD): AC AD BC BD

\item Mehrere Zellenpaare durchprobieren

\begin{itemize}
\item Zum Beispiel: \t21

\begin{enumerate}
\item \t11 und \t12
\end{enumerate}

\item Zum Beispiel: \t51

\begin{enumerate}
\item \t11 und \t42
\item \t21 und \t33
\item \t31 und \t24
\item \t41 und \t15
\end{enumerate}
\end{itemize}

\item Steht in der letzten Zelle die Start-Variable, dann ist das Wort
ableitbar.

\end{itemize}

\liJavaDatei{formale_sprachen/CYKAlgorithmus}

\literatur

\end{document}
