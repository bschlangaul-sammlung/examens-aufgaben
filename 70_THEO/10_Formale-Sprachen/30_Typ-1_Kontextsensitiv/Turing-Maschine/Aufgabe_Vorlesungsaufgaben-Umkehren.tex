\documentclass{lehramt-informatik-aufgabe}
\liLadePakete{formale-sprachen,mathe,automaten}
\begin{document}
\def\z#1{\,z_#1\,}
\def\p{&\rightarrow}
\def\l{\,\liTuringLeerzeichen\,}

\liAufgabenTitel{Vorlesungsaufgaben ab-Wörter umkehren}
\section{Wörter umkehren auf dem Band der Turingmaschine
\index{Turing-Maschine}
\footcite[Seite 27]{theo:fs:3}}

% Info_2021-04-23-2021-04-23_13.17.40 3h10

\begin{enumerate}
\item Geben Sie eine Turingmaschine an, die die Eingabe über dem
Alphabet \liAlphabet{a, b} umkehrt.

Beispiele:

\begin{itemize}
\item $abb \rightarrow bba$
\item $aaaaba \rightarrow abaaaa$
\item $aaa \rightarrow aaa$
\end{itemize}

Tipp:
Fügen Sie ein extra Zeichen ein, welches das Eingabewort von deinem
umgedrehten Wort trennt.
Das Ergebniswort muss nicht an derselben Stelle wie das
Eingabewort stehen.

\begin{liAntwort}
\begin{center}
\begin{tikzpicture}[li turingmaschine]
  \node[state,initial] (z0) at (7.57cm,-1.71cm) {$z_0$};
  \node[state] (z1) at (5.71cm,-3.14cm) {$z_1$};
  \node[state] (z2) at (5.71cm,-5.29cm) {$z_2$};
  \node[state] (z3) at (8.71cm,-4.57cm) {$z_3$};
  \node[state] (z4) at (8.43cm,-7.43cm) {$z_4$};
  \node[state] (z5) at (12.57cm,-5.57cm) {$z_5$};
  \node[state] (z6) at (8.14cm,-8.86cm) {$z_6$};
  \node[state] (z7) at (10.71cm,-9.43cm) {$z_7$};
  \node[state,accepting] (z8) at (13cm,-9.43cm) {$z_8$};

  \liTuringKante[above left,pos=0.7]{z0}{z1}{
    a, a, L;
    b, b, L;
  }

  \liTuringKante[above,loop above]{z0}{z0}{
    LEER, LEER, N;
  }

  \liTuringKante[right]{z1}{z2}{
    LEER, *, R;
  }

  \liTuringKante[above]{z2}{z3}{
    a, *, L;
  }

  \liTuringKante[below left]{z2}{z4}{
    b, *, L;
  }

  \liTuringKante[above,loop above]{z3}{z3}{
    b, b, L;
    a, a, L;
    *, *, L;
  }

  \liTuringKante[above]{z3}{z5}{
    LEER, a, R;
  }

  \liTuringKante[above left,pos=0.8 ]{z4}{z5}{
    LEER, b, R;
  }

  \liTuringKante[above,loop above]{z4}{z4}{
    *, *, L;
    a, a, L;
    b, b, L;
  }

  \liTuringKante[right]{z5}{z6}{
    *, *, R;
  }

  \liTuringKante[above,loop above]{z5}{z5}{
    a, a, R;
    b, b, R;
  }

  \liTuringKante[above,loop below]{z6}{z6}{
    *, *, R;
  }

  \liTuringKante[left,bend left]{z6}{z2}{
    a, a, N;
    b, b, N;
  }

  \liTuringKante[above]{z6}{z7}{
    LEER, LEER, L;
  }

  \liTuringKante[above,loop above]{z7}{z7}{
    *, LEER, L;
  }

  \liTuringKante[above]{z7}{z8}{
    b, b, N;
    a, a, N;
  }
\end{tikzpicture}
\end{center}
\liFlaci{Af75rdjbc}
\end{liAntwort}

\item Geben Sie anschließend eine Konfigurationsfolge ihrer TM für $ab$
an.

\begin{liAntwort}
\begin{align*}
\z0 a b \p \z1 \l  ab\\
\p \z2 * a b \\
\p * \z3 * b \\
\p \z3 * * b \\
\p \z3 \l * * b \\
\p a \z5 * * b \\
\p a * \z6 * b \\
\p a * * \z6 b \\
\p a * * \z2 b \\
\p a * \z4 * * \\
\p a \z4 * * * \\
\p \z4 a * * * \\
\p \z4 \l a * * * \\
\p b \z5 a * * * \\
\p b a \z6 * * * \\
\p b a * \z6 * * \\
\p b a * * \z6 * \\
\p b a * * * \z6 \l\\
\p b a * * \z7 \l\\
\p b a * \z7 \l\\
\p b a \z7 \l\\
\p b  \z8  a
\end{align*}

\end{liAntwort}
\end{enumerate}

\end{document}
