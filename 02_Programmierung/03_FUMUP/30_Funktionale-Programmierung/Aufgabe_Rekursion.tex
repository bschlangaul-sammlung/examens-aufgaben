\documentclass{lehramt-informatik-aufgabe}
\liLadePakete{syntax}
\begin{document}
\liAufgabenTitel{Rekursion}
\section{Übungen zur Rekursion
\index{Funktionale Programmierung mit Haskell}
\footcite{fumup:ab:3}}

Implementiere in der Dateilists.hs eine Funktion
\liHaskellCode{mylength(liste)}, die die Länge der übergebenen Liste
berechnet. Die Funktion \liHaskellCode{myconcat(liste1, liste2)} soll
als Ergebnis die Konkatenation der beiden Listen liefern. Mit
\liHaskellCode{myappend(liste, elem)} soll das Element an das Ende der
Liste angehängt werden. Die Funktion \liHaskellCode{listSum(liste)} soll
die Summe aller Werte in der Liste zurückliefern. Verwenden Sie in
dieser Aufgabe keine spezialisierten Listenfunktionen (wie z.B. den
\liHaskellCode{++}-Operator) außer dem \liHaskellCode{:}-Operator.

\begin{minted}{haskell}
\end{minted}

\end{document}
