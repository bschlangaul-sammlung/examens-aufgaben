\documentclass{lehramt-informatik-aufgabe}
\liLadePakete{syntax}
\begin{document}
\liAufgabenTitel{Damenproblem}

\section{Ein Beispiel: Das Damenproblem
\index{Backtracking}
\footcite[Seite 18 - 19 (PDF 15-19)]{aud:fs:3}}

Acht Damen sollen auf einem Schachbrett so aufgestellt werden, dass
keine zwei Damen einander gemäß ihren in den Schachregeln definierten
Zugmöglichkeiten schlagen können. Für Damen heißt dies konkret: Es
dürfen keine zwei Damen auf derselben Reihe, Linie oder Diagonale
stehen.

Es gibt 92 mögliche Lösungen für das 8x8 Feld – aber wie findet man
diese?

\liJavaDatei[firstline=3]{aufgaben/aud/pu_3/damenproblem/Damenproblem}
\liJavaDatei[firstline=3]{aufgaben/aud/pu_3/damenproblem/Ausgabe}
\end{document}
