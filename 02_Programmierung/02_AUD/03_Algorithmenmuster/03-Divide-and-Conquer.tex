\documentclass{lehramt-informatik-haupt}
\usepackage{tikz}
\usepackage{xcolor}
% https://kleinco.de/latex-snippets/7/tikz-towers-of-hanoi-illustration-for-latex-documents/
\def\mset #1[#2]=#3{%
	\expandafter\xdef\csname #1#2\endcsname{#3}
}
\def\mget #1[#2]{%
	\csname #1#2\endcsname
}
\def\minc #1[#2]+=#3{%
	\pgfmathparse{\mget #1[#2]+#3}%
	\mset #1[#2]=\pgfmathresult
}

\begin{document}

%%%%%%%%%%%%%%%%%%%%%%%%%%%%%%%%%%%%%%%%%%%%%%%%%%%%%%%%%%%%%%%%%%%%%%%%
% Theorie-Teil
%%%%%%%%%%%%%%%%%%%%%%%%%%%%%%%%%%%%%%%%%%%%%%%%%%%%%%%%%%%%%%%%%%%%%%%%

\chapter{Divide-and-Conquer (Teile und herrsche)}

\begin{quellen}
\item \cite[Seite 7-9]{aud:fs:3}
\item \cite[Seite 218-222 (PDF 236-240)]{saake}
\item \cite{wiki:teile-und-herrsche-verfahren}
\end{quellen}

Das Prinzip „Divide-and-conquer“, deutsch „Teile und herrsche,“ basiert
darauf, in einem Schritt eine \memph{große Aufgabe} in mehrere
\memph{kleinere Aufgaben zu teilen} und diese rekursiv zu bearbeiten –
also ein klassischer Einsatz des \memph{Rekursionsprinzips}.
\footcite[Seite 219 (PDF 237)]{saake}

%%%%%%%%%%%%%%%%%%%%%%%%%%%%%%%%%%%%%%%%%%%%%%%%%%%%%%%%%%%%%%%%%%%%%%%%
% Aufgaben
%%%%%%%%%%%%%%%%%%%%%%%%%%%%%%%%%%%%%%%%%%%%%%%%%%%%%%%%%%%%%%%%%%%%%%%%

\chapter{Aufgaben}

% https://www.yumpu.com/de/document/read/17936712/ubungen-zum-prasenzmodul-algorithmen-und-datenstrukturen
\section{Türme von Hanoi
\footcite[Seite 1, Aufgabe 2: Türme von Hanoi]{aud:ab:7}
}

Betrachten wir das folgende Spiel (Türme von Hanoi), das aus drei Stäben
1, 2 und 3 besteht, die senkrecht im Boden befestigt sind. Weiter gibt
es n kreisförmige Scheiben mit einem Loch im Mittelpunkt, so dass man
sie auf die Stäbe stecken kann. Dabei haben die Scheiben verschiedene
Radien, alle sind unterschiedlich groß. Zu Beginn stecken alle Scheiben
auf dem Stab 1, wobei immer eine kleinere auf einer größeren liegt. Das
Ziel des Spiels ist es nun, die Scheiben so umzuordnen, dass sie in der
gleichen Reihenfolge auf dem Stab 3 liegen. Dabei darf immer nur eine
Scheibe bewegt werden und es darf nie eine größere auf einer kleineren
Scheibe liegen. Stab 2 darf dabei als Hilfsstab verwendet werden.

Ein Beispiel für 4 Scheiben finden Sie in folgendem Bild:

% https://kleinco.de/latex-snippets/7/tikz-towers-of-hanoi-illustration-for-latex-documents/
% ---- set number and configuration of discs HERE ----
\edef\numdiscs{4}
\def\sequence{4/1,3/1,2/3,1/2}

\begin{tikzpicture}[line width=4mm,brown!40,line cap=round,xscale=3]
  % init colors
  \foreach[count=\j] \c in {red,green!80!black,blue,orange,violet,gray,yellow!80!black,purple,cyan,black}
  \mset col[\j]={\c};
  % draw poles and init pole counters
  \foreach \j in {1,2,3}{
    \mset pos[\j]=0
    \draw (\j,-.5) -- +(0, .5 + .5*\numdiscs);
  }
  % draw base
  \draw (.5,-.5) -- +(3,0);
  % draw discs
  \foreach[count=\k] \i/\j in \sequence{
    \draw[draw={\mget col[\i]}] (\j,\mget pos[\j]) +(-.4*\i/\numdiscs,0) -- +(.4*\i/\numdiscs,0) node[pos=.5,white,font=\sffamily] {\i};
    \minc pos[\j]+={.5}
  }
\end{tikzpicture}

Entwerfen Sie mit Hilfe der Vorlage eine varibale Simulation der Türme von Hanoi.

\begin{enumerate}
\item Ein ELEMENT hat immer einen Wert (Integer) und kennt das
Nachfolgende Element, wobei immer nur das jeweilige Element auf seinen
Wert und seinen Nachfolger zugreifen darf

\item Ein Turm ist einem Stack (Kellerspeicher) nachempfunden und kennt
somit nur das erste Element. Hinweis: Beachten Sie, dass nur kleinere
Elemente auf den bisherigen Stack gelegt werden können

\item In der Klasse HANOI müssen Sie nur die Methode \java{public void
hanoi (int n, TURM quelle, TURM ziel, TURM hilfe)} implementieren. Die
anderen Methoden sind zur Veranschaulichung des Spiels! Entwerfen Sie
eine rekursive Methode die einen Turm der Höhe n vom Stab quelle auf den
Stab ziel transportiert und den Stab hilfe als Hilfsstab verwendet.

\end{enumerate}

\literatur

\end{document}
