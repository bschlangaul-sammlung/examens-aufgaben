\documentclass{lehramt-informatik-haupt}
\liLadePakete{syntax}

\begin{document}

%%%%%%%%%%%%%%%%%%%%%%%%%%%%%%%%%%%%%%%%%%%%%%%%%%%%%%%%%%%%%%%%%%%%%%%%
% Theorie-Teil
%%%%%%%%%%%%%%%%%%%%%%%%%%%%%%%%%%%%%%%%%%%%%%%%%%%%%%%%%%%%%%%%%%%%%%%%

\chapter{Queue}

\begin{quellen}
\item \cite[Kapitel 6.2.1.5, Seite 183]{schneider}
\item \cite{wiki:warteschlange}
\end{quellen}

\noindent
Eine Queue, auch \memph{Warteschlange} oder \memph{Puffer} genannt, ist
eine Datenstruktur, bei der die Elemente – ähnlich wie bei einem Stack –
\memph{als Folge „organisiert“} sind.
%
Ein neues Element kann \memph{nur auf einer Seite} der Elementfolge
\memph{eingefügt} werden.
%
Das \memph{Entfernen} eines Elements ist – im Gegensatz zum Stack – nur
\memph{auf der anderen Seite} der Elementfolge möglich (\memph{FIFO}:
\memph{First In – First Out}). Das Element, das als erstes in eine Queue
eingefügt wird, muss somit als erstes Element wieder entfernt werden.
%
Die Operation zum Hinzufügen eines Elements wird oft \java{insert()}
bzw. \java{enqueue()}, die Operation zum Entfernen eines Elements oft
\java{remove()} bzw. \java{dequeue()} genannt.
\footcite[Seite 24 (PDF 21)]{aud:fs:4}

%%
%
%%

\section{Suchen in Queues}

Auch in einer Queue muss man bei der \memph{Suche} nach einem bestimmten
Element im schlimmsten Fall \memph{bis zum Ende} der Queue suchen.
%
Hierbei kann allerdings \memph{ohne Zerstörung der Datenstruktur}, also
auch ohne „Hilfsstruktur“, vorgegangen werden.
%
Wegen des \memph{möglichen linearen Komplettdurchlaufs} durch die
gesamte Datenstruktur ist auch bei Queues die Suche eines Elements nicht
optimal.
\footcite[Seite 25 (PDF 22)]{aud:fs:4}

%-----------------------------------------------------------------------
%
%-----------------------------------------------------------------------

\section{Aufgabe aus dem Informatik-Biber: Tellerstapel - Biberschlagen}

Im Restaurant der Biberschule gibt es normalerweise zwei Warteschlangen:
In der einen holen sich die kleinen Biber ihre hohen grünen Teller, in
der anderen holen sich die großen Biber ihre flachen braunen Teller.
Wegen Bauarbeiten kann es heute nur eine Warteschlange für alle Biber
geben. Die Küchenbiber müssen deshalb einen Tellerstapel vorbereiten,
der zur Schlange passt: Sie müssen die grünen und braunen Teller so
stapeln, dass jeder Biber in der Schlange den passenden Teller bekommt.
Schau dir zum Beispiel diese Warteschlange an. Für diese Warteschlange
müssen die Teller so gestapelt sein.

Daten, die mit Computerprogrammen verarbeitet werden sollen, müssen
passend organisiert sein. Informatiker beschäftigen sich deshalb
intensiv mit Datenstrukturen. Zwei einfache Datenstrukturen sind
„Schlange” (Queue) und „Stapel” (Stack). Bei einer „Schlange” kann man
nur auf die zuerst eingereihten Daten zugreifen (nach dem Prinzip FIFO:
„first in, first out”). Bei einem „Stapel” kann man nur auf die zuletzt
eingereihten Daten zugreifen (nach dem Prinzip LIFO: „last in, first
out”). Die Datenstruktur der wartenden Biber ist eine „Schlange”. Die
Datenstruktur der Teller ist ein „Stapel”.
\footcite[Seite 35]{net:pdf:informatik-biber-2010}

\literatur

\end{document}
