\documentclass{lehramt-informatik-aufgabe}
\liLadePakete{syntax,uml}
\begin{document}
\liAufgabenTitel{Playlist}

\section{Musikliste
\footcite[Seite 1-2, Aufgaben 1-3]{aud:ab:4}}

\subsection{Aufgabe 1}

\begin{center}
\begin{tikzpicture}[scale=0.8, transform shape]
\tiny
\umlclass
{MusikListe}
{
  int anzahl
}
{
  void setzeErsten(Knoten e)\\
  int aktualisiereAnzahl()\\
  String gibMusikstückListe()\\
  Knoten entnimmOben()\\
  Knoten gibKnoten()
}

\umlclass[x=6]
{Knoten}
{
}
{
  Knoten gibNächsten()\\
  void setzeNächsten(Knoten k)\\
  MusikStück gibMusikstück()
}

\umlclass[x=13]
{MusikStueck}
{
  String titel
}
{
  String gibTitel()
}

\umlaggreg[mult1=1,mult2=1,arg={verwaltet \liLeserichtungRechts}]{Knoten}{MusikStueck}

% \umlaggreg[
%   angle1=10,
%   angle2=90,
%   loopsize=3cm,
%   mult1=1,
%   mult2=0..1,
%   pos1=0.05,
%   pos2=0.95,
%   arg1={nächster \liLeserichtungRechts},
% ]{Knoten}{Knoten}

\umlassoc[arg1=erster \liLeserichtungRechts, pos1=0.4]{MusikListe}{Knoten}

\end{tikzpicture}
\end{center}

Das Klassendiagramm zeigt den Aufbau einer Playlist.

\begin{enumerate}
\item Implementieren Sie das Klassendiagramm.

\begin{liAntwort}
\liPseudoUeberschrift{Klasse „MusikListe“}
\liJavaDatei[firstline=3,lastline=66]{aufgaben/aud/ab_4/musikliste/MusikListe}

\liPseudoUeberschrift{Klasse „Knoten“}
\liJavaDatei[firstline=3,lastline=24]{aufgaben/aud/ab_4/musikliste/Knoten}

\liPseudoUeberschrift{Klasse „MusikStueck“}
\liJavaDatei[firstline=3]{aufgaben/aud/ab_4/musikliste/MusikStueck}
\end{liAntwort}

\item Schreiben Sie eine Testklasse, in der Sie eine Playlist mit
mindestens vier Liedern erstellen.

\begin{liAntwort}
\liJavaDatei[firstline=3,lastline=16]{aufgaben/aud/ab_4/musikliste/TestKlasse}
\end{liAntwort}
\end{enumerate}

\subsection{Aufgabe 2}

Die Playlist aus Aufgabe 1 soll nun erweitert werden. Aktualisieren Sie
Ihren Code entsprechend!
\index{Einfach-verkettete Liste}

\begin{enumerate}

%%
%
%%

\item Bisher wurde das erste Element der Musikliste in einer öffentlich
sichtbaren Variable gespeichert, dies ist jedoch nicht sinnvoll.
Erstellen Sie eine Methode \liJavaCode{setzeErsten()}, mit der anstatt dessen
die Liste der erstellten Musikstücke angesprochen werden kann.
\index{Implementierung in Java}

\begin{liAntwort}
\liJavaDatei[firstline=12,lastline=13]{aufgaben/aud/ab_4/musikliste/MusikListe}
\end{liAntwort}

%%
%
%%

\item Außerdem wird ein Attribut \liJavaCode{anzahl} benötigt, dass mit Hilfe
der Methode \liJavaCode{aktualisiereAnzahl()} auf dem aktuellen Stand
gehalten werden kann.

\begin{liAntwort}
\liJavaDatei[firstline=14,lastline=15]{aufgaben/aud/ab_4/musikliste/MusikListe}
\end{liAntwort}

%%
%
%%

\item Eine weitere Methode \liJavaCode{gibMusikstückListe()} soll die Titel
aller Lieder in der Liste als \liJavaCode{String} zurückgeben.

\begin{liAntwort}
\liJavaDatei[firstline=32,lastline=43]{aufgaben/aud/ab_4/musikliste/MusikListe}
\end{liAntwort}

%%
%
%%

\item Mit \liJavaCode{entnimmOben()} soll der erste Titel aus der Liste
entnommen werden können.

\begin{liAntwort}
\liJavaDatei[firstline=45,lastline=53]{aufgaben/aud/ab_4/musikliste/MusikListe}
\end{liAntwort}

%%
%
%%

\item Es soll der Titel des Musikstücks ermittelt werden, das an einer
bestimmten Position in der Musikliste abgespeichert ist. Implementieren
Sie dazu die Methode \liJavaCode{gibKnoten()}.

\begin{liAntwort}
\liJavaDatei[firstline=55,lastline=66]{aufgaben/aud/ab_4/musikliste/MusikListe}
\end{liAntwort}

%%
%
%%

\item Die Musikliste soll nun in eine doppelt verkettete Liste umgebaut
werden. Fügen Sie entsprechende Attribute, getter- und setter-Methoden
hinzu.
\index{doppelt-verkettete Liste}

\begin{liAntwort}
\liJavaDatei[firstline=5,lastline=5]{aufgaben/aud/ab_4/musikliste/Knoten}
\liJavaDatei[firstline=22,lastline=28]{aufgaben/aud/ab_4/musikliste/Knoten}
\end{liAntwort}

%%
%
%%

\item Testen Sie die Funktionalität der neuen Methoden in Ihrer
Testklasse.

\begin{liAntwort}
\liJavaDatei[firstline=17,lastline=29]{aufgaben/aud/ab_4/musikliste/TestKlasse}
\end{liAntwort}

\end{enumerate}

\subsection{Rekursion}

Die Anzahl der Titel in der Musikliste aus Aufgabe 1 kann auch unter
Verwendung einer rekursiven Methode ermittelt werden. Implementieren Sie
eine Methode \liJavaCode{zaehleEintraege()}, die analog zu
\liJavaCode{aktualisiereAnzahl()} angibt, wie viele Titel in der Musikliste
gespeichert sind, dies aber rekursiv ermittelt! Testen Sie diese Methode
in der Testklasse. Hinweis: Um für die gesamte Musikliste aufgerufen
werden zu können, muss diese Methode in der Musikliste selbst und auch
in der Klasse Knoten existieren!
\index{Rekursion}

\begin{liAntwort}
\liPseudoUeberschrift{Klasse „MusikListe“}
\liJavaDatei[firstline=68,lastline=74]{aufgaben/aud/ab_4/musikliste/MusikListe}

\liPseudoUeberschrift{Klasse „Knoten“}
\liJavaDatei[firstline=34,lastline=40]{aufgaben/aud/ab_4/musikliste/Knoten}

\liPseudoUeberschrift{Klasse „TestKlasse“}
\liJavaDatei[firstline=30,lastline=20]{aufgaben/aud/ab_4/musikliste/TestKlasse}
\end{liAntwort}
\end{document}
