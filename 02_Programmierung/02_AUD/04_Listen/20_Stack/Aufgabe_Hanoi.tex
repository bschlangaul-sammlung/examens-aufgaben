\documentclass{lehramt-informatik-aufgabe}
\liLadePakete{syntax,hanoi}
\begin{document}
\liAufgabenTitel{Hanoi}

\section{Aufgabe 2: Türme von Hanoi
\index{Stapel (Stack)}
\footcite{aud:ab:7}}

Betrachten wir das folgende Spiel (Türme von Hanoi), das aus drei Stäben
1, 2 und 3 besteht, die senkrecht im Boden befestigt sind. Weiter gibt
es $n$ kreisförmige Scheiben mit einem Loch im Mittelpunkt, so dass man
sie auf die Stäbe stecken kann. Dabei haben die Scheiben verschiedene
Radien, alle sind unterschiedlich groß. Zu Beginn stecken alle Scheiben
auf dem Stab 1, wobei immer eine kleinere auf einer größeren liegt. Das
Ziel des Spiels ist es nun, die Scheiben so umzuordnen, dass sie in der
gleichen Reihenfolge auf dem Stab 3 liegen. Dabei darf immer nur eine
Scheibe bewegt werden und es darf nie eine größere auf einer kleineren
Scheibe liegen. Stab 2 darf dabei als Hilfsstab verwendet werden.

Ein Beispiel für 4 Scheiben finden Sie in folgendem Bild:

\begin{center}
\liHanoi{4}{4/1,3/1,2/1,1/1}
\end{center}

\begin{center}
\liHanoi{4}{4/3,3/3,2/3,1/3}
\end{center}

\begin{enumerate}

%%
% 1.
%%

\item Ein \java{Element} hat immer einen Wert (Integer) und kennt das
Nachfolgende \java{Element}, wobei immer nur das jeweilige
\java{Element} auf seinen Wert und seinen Nachfolger zugreifen darf.

\begin{antwort}
\inputcode[firstline=6]{aufgaben/aud/ab_7/hanoi/Element}
\end{antwort}

%%
% 2.
%%

\item Ein \java{Turm} ist einem Stack (Kellerspeicher) nachempfunden und
kennt somit nur das erste Element. Hinweis: Beachten Sie, dass nur
kleinere Elemente auf den bisherigen Stack gelegt werden können

\begin{antwort}
\inputcode[firstline=6]{aufgaben/aud/ab_7/hanoi/Turm}
\end{antwort}

%%
% 3.
%%

\item In der Klasse \java{Hanoi} müssen Sie nur die Methode \java{public
void hanoi (int n, Turm quelle, Turm ziel, Turm hilfe)} implementieren.
Die anderen Methoden sind zur Veranschaulichung des Spiels! Entwerfen
Sie eine rekursive Methode die einen Turm der Höhe $n$ vom Stab
\java{quelle} auf den Stab \java{ziel} transportiert und den Stab
\java{hilfe} als Hilfsstab verwendet.

\begin{antwort}
\inputcode[firstline=6]{aufgaben/aud/ab_7/hanoi/Hanoi}
\end{antwort}
\end{enumerate}
\end{document}
