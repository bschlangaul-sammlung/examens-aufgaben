\documentclass{lehramt-informatik-haupt}
\liLadePakete{sortieren,syntax}

\begin{document}

%%%%%%%%%%%%%%%%%%%%%%%%%%%%%%%%%%%%%%%%%%%%%%%%%%%%%%%%%%%%%%%%%%%%%%%%
% Theorie-Teil
%%%%%%%%%%%%%%%%%%%%%%%%%%%%%%%%%%%%%%%%%%%%%%%%%%%%%%%%%%%%%%%%%%%%%%%%

\chapter{BubbleSort: Blasensortierung}

\begin{quellen}
\item \cite[Seite 43]{aud:fs:tafeluebung-11}
\item \cite[Seite 129-131 (PDF 147-149)]{saake}
\item \cite{wiki:bubblesort}
\end{quellen}

\liVertauschen{3 2 5 1 4}

1. Durchgang

\liVertauschen{>3 <2 5 1 4}

\liVertauschen{2 3 >5 <1 4}

\liVertauschen{2 3 1 >5 <4}

2. Durchgang

\liVertauschen{2 >3 <1 4 5}

\liVertauschen{2 1 3 4 5}

3. Durchgang

\liVertauschen{>2 <1 3 4 5}

fertig

\liVertauschen{1 2 3 4 5}

\section{Funktionsweise}

\begin{itemize}
\item solange zu sortierende Liste nicht vollständig sortiert ist:
\item iteriere von vorne über die Elemente
\item falls zwei aufeinanderfolgende Elemente nicht sortiert sind: vertauschen
\item fertig, wenn während einer Runde keine Vertauschung passiert ist
\end{itemize}

\section{Eigenschaften}

\begin{itemize}
\item Laufzeitkomplexität:

\begin{itemize}
\item $\mathcal{O}(n)$ (im Best-Case)
\item $\mathcal{O}(n^2)$ (im Average- und Worst-Case)
\end{itemize}

\item stabil
\item in-situ
\end{itemize}

\liJavaDatei[firstline=8,lastline=21]{sortier/BubbleSort}

\footcite[Seite 130-131 (PDF 148-149)]{saake}

\literatur

\end{document}
