\documentclass{lehramt-informatik-haupt}
\liLadePakete{syntax,mathe}

\begin{document}

%%%%%%%%%%%%%%%%%%%%%%%%%%%%%%%%%%%%%%%%%%%%%%%%%%%%%%%%%%%%%%%%%%%%%%%%
% Theorie-Teil
%%%%%%%%%%%%%%%%%%%%%%%%%%%%%%%%%%%%%%%%%%%%%%%%%%%%%%%%%%%%%%%%%%%%%%%%

\chapter{Selectionsort: Sortieren durch Auswählen\footcite[Seite 39]{aud:fs:tafeluebung-11}}

\begin{quellen}
\item \cite[Seite 39]{aud:fs:tafeluebung-11}
\item \cite{wiki:selectionsort}
\item \cite[Seite 127-129 (PDF 145-147)]{saake}
\item \cite[6.4.1 Naive Sortierverfahren, Seite 191]{schneider}
\end{quellen}

\begin{itemize}
\item Funktionsweise

\begin{itemize}
\item solange zu sortierende Liste mehr als ein Element beinhaltet:

\begin{itemize}
\item lösche das \emph{Maximum / Minimum} aus der Liste
\item füge es ans Ende der Ergebnisliste
\item wiederhole, bis Eingangsliste leer
\end{itemize}

\end{itemize}

\item \emph{Eigenschaften} von Selectionsort:

\begin{itemize}
\item Laufzeitkomplexität:
$\mathcal{O}(n^2)$ (im \emph{Best}-, \emph{Average}- und
\emph{Worst-Case})

\item Stabilität leicht erreichbar
\item bei Zahlen \emph{in-situ}
\end{itemize}

\end{itemize}

%%
%
%%

\section{Iterativ}

\inputcode[firstline=5,lastline=29]{sortier/SelectionSort}
\footcite[Seite 128 (PDF 146)]{saake}

\section{Halb Rekursiv}

\inputcode[firstline=31,lastline=52]{sortier/SelectionSort}

\section{Rekursiv}

\inputcode[firstline=54,lastline=77]{sortier/SelectionSort}

%%%%%%%%%%%%%%%%%%%%%%%%%%%%%%%%%%%%%%%%%%%%%%%%%%%%%%%%%%%%%%%%%%%%%%%%
% Aufgaben
%%%%%%%%%%%%%%%%%%%%%%%%%%%%%%%%%%%%%%%%%%%%%%%%%%%%%%%%%%%%%%%%%%%%%%%%

\chapter{Aufgaben}

%-----------------------------------------------------------------------
%
%-----------------------------------------------------------------------

\ExamensAufgabeTA 66115 / 2014 / 09 : Thema 2 Aufgabe 6

%-----------------------------------------------------------------------
%
%-----------------------------------------------------------------------

\section{Sortieren\footcite[Aufgabe 2]{aud:e-klausur}}

In dieser Aufgabe soll ein gegebenes Integer Array mit Hilfe von
\textbf{Selection Sort} sortiert werden. Es soll eine iterative und eine
rekursive Methode geschrieben werden.

Verwenden Sie zur Implementierung jeweils die Methodenköpfe
\java{selectionSortIterativ()} und \java{selectionSortRekursiv()}. Eine
\java{swap}-Methode, die für ein gegebenes Array und zwei Indizes die
Einträge an den jeweiligen Indizes des Arrays vertauscht, ist gegeben
und muss nicht implementiert werden.

Es müssen keine weiteren Methoden geschrieben werden!

\begin{antwort}
\liPseudoUeberschrift{iterativ}

\inputcode[firstline=11,lastline=21]{aufgaben/aud/e_klausur/SelectionSort}

\liPseudoUeberschrift{rekursiv}

\inputcode[firstline=23,lastline=35]{aufgaben/aud/e_klausur/SelectionSort}
\end{antwort}


\literatur

\end{document}
