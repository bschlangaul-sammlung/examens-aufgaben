\documentclass{lehramt-informatik-aufgabe}
\liLadePakete{syntax,mathe}
\begin{document}
\liAufgabenTitel{Implementierung der merge-Methode}

\section{Implementierung der \liJavaCode{merge}-Methode, Berechnung der
Zeitkomplexität
\index{Mergesort}
\footcite[Seite 2, Aufgabe 3: Mergesort]{aud:ab:7}
}

Das Sortierverfahren \emph{Mergesort}, das nach der Strategie
\emph{Divide-and-Conquer} arbeitet, sortiert eine Sequenz, indem die
Sequenz in zwei Teile zerlegt wird, die dann einzeln sortiert und wieder
zu einer sortierten Sequenz zusammengemischt werden (to merge =
zusammenmischen, verschmelzen).

\begin{enumerate}

%%
% (a)
%%

\item Gegeben seien folgende Methoden:

\liJavaDatei[firstline=13,lastline=31]{aufgaben/aud/ab_7/mergesort/Mergesort}

Schreiben Sie die Methode \liJavaCode{public int[] merge (int[] s, int[] r)},
die die beiden aufsteigend sortierten Sequenzen \liJavaCode{s} und \liJavaCode{r} zu
einer aufsteigend sortierten Sequenz zusammenmischt.

\begin{antwort}
\liJavaDatei[firstline=33,lastline=60]{aufgaben/aud/ab_7/mergesort/Mergesort}
\end{antwort}

%%
% (b)
%%

\item Analysieren Sie die Zeitkomplexität von \liJavaCode{mergesort}.

\begin{antwort}
$O(n \cdot \log n)$

\liPseudoUeberschrift{Erklärung}

Mergesort ist ein stabiles Sortierverfahren, vorausgesetzt der
Merge-Schritt ist korrekt implementiert. Seine Komplexität beträgt im
Worst-, Best- und Average-Case in Landau-Notation ausgedrückt stets $O(n
\cdot \log n)$. Für die Laufzeit $T(n)$ von Mergesort bei $n$ zu
sortierenden Elementen gilt die Rekursionsformel

\begin{align*}
T(n) & = \\
     & T\left(\left\lfloor\frac{n}{2}\right\rfloor\right) + && \text{Aufwand, 1. Teil zu sortieren}\\
     & T\left(\left\lceil\frac{n}{2}\right\rceil\right) + && \text{Aufwand, 2. Teil zu sortieren}\\
     & \mathcal{O}(n) && \text{Aufwand, beide Teile zu verschmelzen}\\
\end{align*}

mit dem Rekursionsanfang $T(1) = 1$.

Nach dem Master-Theorem kann die Rekursionsformel durch

\begin{displaymath}
2T\left(\left\lfloor\frac{n}{2}\right\rfloor\right) + n
\end{displaymath}

bzw.

\begin{displaymath}
2T\left(\left\lceil\frac{n}{2}\right\rceil\right) + n
\end{displaymath}

approximiert werden mit jeweils der Lösung $T(n) = O(n \cdot \log n)$.
\end{antwort}

\end{enumerate}
\end{document}
