\documentclass{lehramt-informatik-haupt}
\liLadePakete{syntax,sortieren,mathe}

\begin{document}

%%%%%%%%%%%%%%%%%%%%%%%%%%%%%%%%%%%%%%%%%%%%%%%%%%%%%%%%%%%%%%%%%%%%%%%%
% Theorie-Teil
%%%%%%%%%%%%%%%%%%%%%%%%%%%%%%%%%%%%%%%%%%%%%%%%%%%%%%%%%%%%%%%%%%%%%%%%

\chapter{MergeSort: Sortieren durch Verschmelzen}

\begin{quellen}
\item \cite[Seite 50]{aud:fs:tafeluebung-11}
\item \cite[Seite 131-135(PDF 149-153)]{saake}
\item \cite{wiki:mergesort}
\item \cite[6.4.2 Effiziente Sortierverfahren, Seite 192]{schneider}
\end{quellen}

\begin{center}
\def\myNodes{}
\begin{forest}
  /tikz/arrows=->, /tikz/>=latex, /tikz/nodes={draw},
  for tree={delay={sort}}, sort level=2
[38 27 43 3 9 82 10
  [38 27 43 3
    [38 27 [39][27]]
    [43 3 [43][3]]
  ]
  [9 82 10
    [9 82 [9] [82]]
    [10 [10]]
  ]
]
%
\coordinate (m) at (!|-!\forestOnes);
\myNodes
\end{forest}
\end{center}

\begin{itemize}
\item Funtionsweise:

\begin{itemize}
\item Listen mit maximal \emph{einem} Element sind \emph{trivialerweise sortiert}
\item falls zu sortierende Liste mehr als ein Element beinhaltet:

\begin{itemize}
\item \emph{teile} Liste in zwei kleinere Listen auf
\item verfahre \emph{rekursiv} mit den beiden Teillisten
\item \emph{verschmelze} die zwei rekursiv sortierten Listen
\item dabei Sortierung in verschmolzener Liste beibehalten
\end{itemize}
\end{itemize}

$\rightarrow$ Teile-Und-Herrsche
\item Eigenschaften von MergeSort:

\begin{itemize}
\item Laufzeitkomplexität:
$\mathcal{O}(n \cdot log(n))$ (im Best-, Average- und Worst-Case)
\item bei geeigneter „Verschmelzung“ \emph{stabile} Sortierung
\item durch Rekursion wachsender Aufrufstapel
$\rightarrow$ \emph{out-of-place}
\end{itemize}

\end{itemize}

\liJavaDatei[firstline=14,lastline=61]{sortier/MergeSort}
\footcite[Seite 134 (PDF 152)]{saake}

\literatur

\end{document}
