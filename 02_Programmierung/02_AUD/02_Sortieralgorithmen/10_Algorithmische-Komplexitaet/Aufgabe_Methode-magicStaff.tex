\documentclass{lehramt-informatik-aufgabe}
\liLadePakete{syntax,mathe}
\begin{document}
\liAufgabenTitel{Methode „magicStaff()“}

\section{Komplexität
\index{Komplexität}
\footcite[Aufgabe 5]{aud:e-klausur}
}

Welche Komplexität hat das Programmfragment?

\liJavaDatei[firstline=5,lastline=17]{aufgaben/aud/e_klausur/Komplexitaet}

\noindent
Bestimmen Sie in Abhängigkeit von $n$ die Komplexität des
Programmabschnitts im

\begin{enumerate}
\item Best-Case.

\begin{antwort}
$\mathcal{O}(1)$: Wenn die erste Zahl im Feld \liJavaCode{array} ohne Rest
durch 3 teilbar ist, wird sofort aus der for-Schleife ausgestiegen
(wegen der \liJavaCode{break} Anweisung).
\end{antwort}

\item Worst-Case.
\begin{antwort}
$\mathcal{O}(n^2)$: Wenn keine Zahl aus \liJavaCode{array} ohne Rest durch 3
teilbar ist, werden zwei Schleifen (\liJavaCode{for} und \liJavaCode{do while}) über
die Anzahl \liJavaCode{n} der Elemente des Felds durchlaufen.
\end{antwort}
\end{enumerate}

\end{document}
