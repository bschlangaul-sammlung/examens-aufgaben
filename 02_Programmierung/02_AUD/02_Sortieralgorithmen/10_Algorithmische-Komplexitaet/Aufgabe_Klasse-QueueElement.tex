\documentclass{lehramt-informatik-aufgabe}
\liLadePakete{syntax}
\begin{document}
\liAufgabenTitel{Klasse-QueueElement}

\section{Aufgabe zur Komplexität
\index{Komplexität}
\footcite[Seite 3]{aud:pu:2}}

\noindent
Der Konstruktor \liJavaCode{QueueElement(...)} und die Methode
\liJavaCode{setNext(...)} sowie \liJavaCode{getNext(...)} haben $\mathcal{O}(1)$.
Gib die Zeitkomplexität der Methode \liJavaCode{append(int content)} an, die
einer Schlange ein neues Element anhängt.

\begin{minted}{java}
public void append(int contents) {
QueueElement newElement = new QueueElement(contents) ;
  if (first == 0) {
    first = newElement;
    last = newElement;
  } else {
    last.setNext(newElement); // neues Element hinten anhaengen
    last = last.getNext(); // angehaengtes Element ist Letztes
  }
}
\end{minted}

\begin{antwort}
Das Anhängen eines neuen Elements in die gegebene Warteschlange hat
konstanten Rechenzeitbedarf $\mathcal{O}(1)$, egal wie lange die
Schlange ist, da man das letzte Element direkt ansprechen kann.
\end{antwort}

\end{document}
