\documentclass{lehramt-informatik-aufgabe}
\liLadePakete{syntax,spalten}
\begin{document}
\index{Binäre Suche}
\liAufgabenTitel{Methode „sucheBinaer()“}

\section{Aufgabe 3: Suchen\footcite{aud:ab:2}}

Für diese Aufgabe wird die Vorlage Suchalgorithmen benötigt, die auf dem
Beiblatt genauer erklärt wird.\footcite{Diese Aufgabe stammt aus dem
Übungsblatt 1 zu Algorithmen und Datenstrukturen von Prof. Dr. Martin
Hennecke und Rainer Gall an der Universität Würzburg und wurde
dankenswerterweise zur Verwendung in diesem Aufgabenblatt zur Verfügung
gestellt.}

Vervollständigen Sie die Methode \liJavaCode{sucheBinaer()}. Die fertige
Methode soll in der Lage sein, beliebige Werte in beliebigen sortierten
Arrays zu suchen. Im gelben Textfeld des Eingabefensters soll dabei
ausführlich und nachvollziehbar angezeigt werden, wie die Methode
vorgeht. Beispielsweise so:

{
\tiny
\begin{multicols}{2}
\begin{verbatim}
Führe die Methode sucheSequenziell() aus:
Suche in diesem Feld: 3, 5, 7, 17, 42, 23
Überprüfen des Werts an der Position 0
Überprüfen des Werts an der Position 1
Überprüfen des Werts an der Position 2
Überprüfen des Werts an der Position 3
Fertig :-)
\end{verbatim}

\begin{verbatim}
Führe die Methode sucheBinaer() aus:
Suche in diesem Feld: 3, 5, 7, 17, 42, 23
Suchbereich: 0 bis 5
Mitte: 2, also neuer Bereich: 3 bis 5
Mitte: 4, also neuer Bereich: 3 bis 3
Mitte: 3, Treffer : -)
\end{verbatim}
\end{multicols}
}

\begin{enumerate}
\item Sequenzielle Suche
\begin{antwort}
\liJavaDatei[firstline=22,lastline=44]{aufgaben/aud/ab_2/Suchalgorithmen}
\end{antwort}

\item Binäre Suche

\begin{antwort}
\liJavaDatei[firstline=46,lastline=76]{aufgaben/aud/ab_2/Suchalgorithmen}
\end{antwort}

\end{enumerate}
\end{document}
