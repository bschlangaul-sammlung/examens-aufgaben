\documentclass{lehramt-informatik-aufgabe}
\liLadePakete{syntax}
\begin{document}
\liAufgabenTitel{iterativ-rekursiv}

\section{Sortieren
\index{Selectionsort}
\footcite[Aufgabe 2]{aud:e-klausur}}

In dieser Aufgabe soll ein gegebenes Integer Array mit Hilfe von
\textbf{Selection Sort} sortiert werden. Es soll eine
iterative\index{Iterative Realisation} und eine
rekursive\index{Rekursion} Methode geschrieben werden.

Verwenden Sie zur Implementierung jeweils die Methodenköpfe
\java{selectionSortIterativ()} und \java{selectionSortRekursiv()}. Eine
\java{swap}-Methode, die für ein gegebenes Array und zwei Indizes die
Einträge an den jeweiligen Indizes des Arrays vertauscht, ist gegeben
und muss nicht implementiert werden.

Es müssen keine weiteren Methoden geschrieben werden!

\begin{antwort}
\liPseudoUeberschrift{iterativ}

\inputcode[firstline=11,lastline=21]{aufgaben/aud/e_klausur/SelectionSort}

\liPseudoUeberschrift{rekursiv}

\inputcode[firstline=23,lastline=35]{aufgaben/aud/e_klausur/SelectionSort}
\end{antwort}

\end{document}
