\documentclass{lehramt-informatik-aufgabe}
\liLadePakete{graph}
\begin{document}
\liAufgabenTitel{Graph A-I}

\section{Dijkstra-Algorithmus
\index{Algorithmus von Dijkstra}
\footcite[Aufgabe 4]{aud:e-klausur}
}

Führen Sie auf dem gegebenen Graphen die Suche nach der kürzesten
Distanz aller Knoten zum Startknoten A mit dem Algorithmus von Dijkstra
durch. Tragen Sie die Abarbeitungsreihenfolge, den unmittelbaren
Vorgängerknoten, sowie die ermittelte kürzeste Distanz für jeden Knoten
ein! Bei gleichen Distanzen arbeiten Sie die Knoten in lexikalischer
Reihenfolge ab.

\graph knoten {
  \knoten{A}(1,4)
  \knoten{B}(3,5)
  \knoten{C}(3,3)
  \knoten{D}(0,2)
  \knoten{E}(5,5)
  \knoten{F}(5,1)
  \knoten{G}(3,0)
  \knoten{H}(6,3)
  \knoten{I}(8,4)
} kanten {
  \kante(A-B){2}
  \kante(A-C){5}
  \kante(A-D){2}
  \kante(B-C){3}
  \kante(B-E){1}
  \kante(C-D){3}
  \kante(C-E){1}
  \kante(C-F){1}
  \kante(C-H){1}
  \kante(D-G){2}
  \kante(E-I){7}
  \kante(F-G){2}
  \kante(F-H){3}
  \kante(F-H){3}
  \kante(H-I){1}
}

\begin{liGraphenFormat}
A: 1 4
B: 3 5
C: 3 3
D: 0 2
E: 5 5
F: 5 1
G: 3 0
H: 6 3
I: 8 4
A -- B: 2
A -- C: 5
A -- D: 2
B -- C: 3
B -- E
C -- D: 3
C -- E
C -- F
C -- H
D -- G: 2
E -- I: 7
F -- G: 2
F -- H: 3
H -- I
\end{liGraphenFormat}

Als TikZ-Umgebung

\begin{tikzpicture}
\node[li graph knoten] (A) at (1,4) {A};
\node[li graph knoten] (B) at (3,5) {B};
\node[li graph knoten] (C) at (3,3) {C};
\node[li graph knoten] (D) at (0,2) {D};
\node[li graph knoten] (E) at (5,5) {E};
\node[li graph knoten] (F) at (5,1) {F};
\node[li graph knoten] (G) at (3,0) {G};
\node[li graph knoten] (H) at (6,3) {H};
\node[li graph knoten] (I) at (8,4) {I};

\path[li graph kante] (A) edge node {2} (B);
\path[li graph kante] (A) edge node {5} (C);
\path[li graph kante] (A) edge node {2} (D);
\path[li graph kante] (B) edge node {3} (C);
\path[li graph kante] (B) edge (E);
\path[li graph kante] (C) edge node {3} (D);
\path[li graph kante] (C) edge (E);
\path[li graph kante] (C) edge (F);
\path[li graph kante] (C) edge (H);
\path[li graph kante] (D) edge node {2} (G);
\path[li graph kante] (E) edge node {7} (I);
\path[li graph kante] (F) edge node {2} (G);
\path[li graph kante] (F) edge node {3} (H);
\path[li graph kante] (H) edge (I);
\end{tikzpicture}

\end{document}
