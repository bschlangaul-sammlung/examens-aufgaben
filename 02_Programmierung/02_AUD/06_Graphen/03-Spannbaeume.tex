\documentclass{lehramt-informatik-haupt}
\liLadePakete{graph,pseudo}

\usetikzlibrary{arrows.meta}

\begin{document}

%%%%%%%%%%%%%%%%%%%%%%%%%%%%%%%%%%%%%%%%%%%%%%%%%%%%%%%%%%%%%%%%%%%%%%%%
% Theorie-Teil
%%%%%%%%%%%%%%%%%%%%%%%%%%%%%%%%%%%%%%%%%%%%%%%%%%%%%%%%%%%%%%%%%%%%%%%%

\chapter{Spannbäume}

%-----------------------------------------------------------------------
%
%-----------------------------------------------------------------------

\section{Definition}

Es sei $G$ ein zusammenhängender Graph und $S$ ein zusammenhängender
Teilgraph von G. $S$ ist ein Spannbaum von $G$, falls $S$ \memph{alle
Knoten von $G$ enthält} und $S$ \memph{zyklenfrei} ist.

%-----------------------------------------------------------------------
%
%-----------------------------------------------------------------------

\section{Minimaler Spannbaum}

$S$ ist ein minimaler Spannbaum, falls $S$ ein Spannbaum von $G$ ist,
und die Summe der \memph{Kantengewichte} in $S$ \memph{kleiner oder
gleich der aller anderen möglichen Spannbäume} $S‘$ von $G$ ist
\footcite[Seite 29 (PDF 23)]{aud:fs:6}.

%-----------------------------------------------------------------------
%
%-----------------------------------------------------------------------

\section{Algorithmus von Kruskal}

Durch den Algorithmus von Kruskal wird ein \memph{minimaler Spannbaum}
eines ungerichteten, zusammenhängenden und kantengewichteten Graphen
bestimmt. Sollen bespielsweise Rohrleitungen so verlegt werden, dass
jeder Punkt erreicht wird, kann dieser Algorithmus zum Einsatz kommen.
Wie viele Meter Rohr benötigt man mindestens? Auf diese Frage liefert
der Algorithmus eine Antwort.
\footcite[Seite 30 (PDF 24)]{aud:fs:6}

Der Algorithmus von Kruskal nutzt die Kreiseigenschaft minimaler
Spannbäume. Dazu werden die Kanten in der \memph{ersten Phase}
\memph{aufsteigend nach ihrem Gewicht sortiert}. In der zweiten Phase
wird \memph{über die sortierten Kanten iteriert}. Wenn eine
\memph{Kante} zwei Knoten verbindet, die \memph{noch nicht} durch einen
Pfad vorheriger Kanten \memph{verbunden} sind, wird diese Kante zum
minimalen Spannbaum \memph{hinzugenommen}.
\footcite{wiki:kruskal}

\begin{algorithm}[H]
\KwData{$G = (V,E,w)$: ein zusammenhängender, ungerichteter, kantengewichteter Graph
kruskal(G)}

$E'\leftarrow \emptyset $\;
$L\leftarrow E$\;

Sortiere die Kanten in L aufsteigend nach ihrem Kantengewicht.\;
\While{$L \neq \emptyset $}{
  wähle eine Kante $e\in L$ mit kleinstem Kantengewicht\;
  entferne die Kante e aus L\;

  \If{der Graph $(V, E' \cup \lbrace e\rbrace)$ keinen Kreis enthält}{
    $E'\leftarrow E'\cup \lbrace e\rbrace $\;
  }
}

\KwResult{$M = (V,E')$ ist ein minimaler Spannbaum von G.}
\caption{Minimaler Spannbaum nach Kruskal\footcite{wiki:kruskal}}
\end{algorithm}

\graph knoten {
  \knoten{A}(0,5)
  \knoten{B}(2,4)
  \knoten{C}(5,5)
  \knoten{D}(0,2)
  \knoten{E}(4,2)
  \knoten{F}(2,1)
  \knoten{G}(5,0)
} kanten {
  \kante(A-B){7}
  \kante(A-D){5}
  \kante(B-C){8}
  \kante(B-D){9}
  \kante(B-E){7}
  \kante(C-E){5}
  \kante(D-E){15}
  \kante(D-F){6}
  \kante(E-F){8}
  \kante(E-G){9}
  \kante(F-G){11}
}

\def\TmpGraph#1{
  \graph knoten {
    \knoten{A}(0,5)
    \knoten{B}(2,4)
    \knoten{C}(5,5)
    \knoten{D}(0,2)
    \knoten{E}(4,2)
    \knoten{F}(2,1)
    \knoten{G}(5,0)
  } kanten {
    #1
  }
}

\TmpGraph{
  \kante(A-B){7}
  \kante(A-D){5}
  \kante(B-C){8}
  \kante(B-D){9}
  \kante(B-E){7}
  \kante(C-E){5}
  \kante(D-E){15}
  \kante(D-F){6}
  \kante(E-F){8}
  \kante(E-G){9}
  \kante(F-G){11}
}

\begin{minipage}{7cm}
\TmpGraph{
  \KANTE(A-B){7}
  \KANTE(A-D){5}
  \kante(B-C){8}
  \kante(B-D){9}
  \KANTE(B-E){7}
  \KANTE(C-E){5}
  \kante(D-E){15}
  \KANTE(D-F){6}
  \kante(E-F){8}
  \KANTE(E-G){9}
  \kante(F-G){11}
}
\end{minipage}
\begin{minipage}{4cm}
\begin{center}
\begin{tabular}{|l|l|r|}
\hline
Kante & & Gewicht\\\hline\hline
AD, CE & $2 \times 5$ & $10$\\
DF     &              & $6$\\
AB, BE & $2 \times 7$ & $14$\\
EG     &              & $9$\\\hline
       &              & $39$\\\hline
\end{tabular}
\end{center}
\end{minipage}

%-----------------------------------------------------------------------
%
%-----------------------------------------------------------------------

\section{Algorithmus von Prim\footcite{wiki:prim}}

Der Algorithmus von Prim dient ebenfalls der Bestimmung eines minimalen
Spannbaums, erreicht jedoch sein Ziel durch eine unterschiedliche
Vorgehensweise.

Der \memph{Teilgraph} $T$ wird \memph{schrittweise vergrößert}, bis
dieser ein Spannbaum ist. Der Algorithmus benötigt einen
\memph{konkreten Startpunkt}. Es wird die \memph{günstigste} von
einem Teilgraphen $T$ \memph{ausgehende Kante ausgewählt} und zu diesem
Spannbaum hinzugefügt. Der Algorithmus fügt jede Kante und deren
Endknoten zur Lösung hinzu, die mit allen zuvor gewählten Kanten keinen
Kreis bildet.
\footcite[Seite 32, (PDF 26)]{aud:fs:6}

\begin{minipage}{7cm}
\TmpGraph{
  \KANTE(A-B){7}
  \KANTE(A-D){5}
  \kante(B-C){8}
  \kante(B-D){9}
  \KANTE(B-E){7}
  \KANTE(C-E){5}
  \kante(D-E){15}
  \KANTE(D-F){6}
  \kante(E-F){8}
  \KANTE(E-G){9}
  \kante(F-G){11}
}
\end{minipage}
\begin{minipage}{4cm}
\begin{center}
\begin{tabular}{|l|r|}
\hline
Kante & Gewicht\\\hline\hline
AD & $5$\\
DF & $6$\\
AB & $7$\\
BE & $7$\\
EC & $5$\\
EG & $9$\\\hline
   & $39$\\\hline
\end{tabular}
\end{center}
\end{minipage}

\literatur

\end{document}
