\documentclass{lehramt-informatik-aufgabe}
\liLadePakete{graph}
\begin{document}
\liAufgabenTitel{Studiflix}
\section{Algorithmus von Prim
\index{Algorithmus von Prim}
}

\liFussnoteUrl{https://studyflix.de/informatik/prim-algorithmus-1293}

\begin{liGraphenFormat}
B: 3 4
A: 0 5
C: 5 5
D: 2 2
E: 5 2
F: 0 0
G: 3 0
A -- B: 14
A -- D: 10
B -- D: 18
B -- E: 13
C -- B: 16
C -- E: 9
D -- E: 30
D -- F: 17
D -- G: 12
E -- G: 16
F -- G: 22
\end{liGraphenFormat}

\begin{tikzpicture}[li graph]
\node (A) at (0,5) {A};
\node (B) at (3,4) {B};
\node (C) at (5,5) {C};
\node (D) at (2,2) {D};
\node (E) at (5,2) {E};
\node (F) at (0,0) {F};
\node (G) at (3,0) {G};

\path[li graph kante] (A) edge node {14} (B);
\path[li graph kante] (A) edge node {10} (D);
\path[li graph kante] (B) edge node {18} (D);
\path[li graph kante] (B) edge node {13} (E);
\path[li graph kante] (C) edge node {16} (B);
\path[li graph kante] (C) edge node {9} (E);
\path[li graph kante] (D) edge node {30} (E);
\path[li graph kante] (D) edge node {17} (F);
\path[li graph kante] (D) edge node {12} (G);
\path[li graph kante] (E) edge node {16} (G);
\path[li graph kante] (F) edge node {22} (G);
\end{tikzpicture}

\begin{liAntwort}
\begin{tikzpicture}[li graph]
\node (A) at (0,5) {A};
\node (B) at (3,4) {B};
\node (C) at (5,5) {C};
\node (D) at (2,2) {D};
\node (E) at (5,2) {E};
\node (F) at (0,0) {F};
\node (G) at (3,0) {G};

\path[li graph kante,li markierung] (A) edge node {14} (B);
\path[li graph kante,li markierung] (A) edge node {10} (D);
\path[li graph kante] (B) edge node {18} (D);
\path[li graph kante,li markierung] (B) edge node {13} (E);
\path[li graph kante] (C) edge node {16} (B);
\path[li graph kante,li markierung] (C) edge node {9} (E);
\path[li graph kante] (D) edge node {30} (E);
\path[li graph kante,li markierung] (D) edge node {17} (F);
\path[li graph kante,li markierung] (D) edge node {12} (G);
\path[li graph kante] (E) edge node {16} (G);
\path[li graph kante] (F) edge node {22} (G);
\end{tikzpicture}
\end{liAntwort}

\end{document}
