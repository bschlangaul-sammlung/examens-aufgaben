\documentclass{lehramt-informatik-aufgabe}
\liLadePakete{graph}
\begin{document}
\liAufgabenTitel{TUM}
\section{Standardbeispiel TUM
\index{Algorithmus von Prim}
}

% \footcite{\url{https://www-m9.ma.tum.de/graph-algorithms/mst-prim/index_de.html}}

Standardbeispiel

\begin{liGraphenFormat}
0: 0 3
1: 2 0
2: 3 7
3: 5 0
4: 5 5
5: 8 3
0 -- 1: 10
0 -- 2: 20
1 -- 3: 50
1 -- 4: 10
2 -- 3: 20
2 -- 4: 33
3 -- 4: 20
3 -- 5: 2
4 -- 5: 1
\end{liGraphenFormat}

\begin{tikzpicture}[li graph]
\node (0) at (0,3) {0};
\node (1) at (2,0) {1};
\node (2) at (3,7) {2};
\node (3) at (5,0) {3};
\node (4) at (5,5) {4};
\node (5) at (8,3) {5};

\path[li graph kante] (0) edge node {10} (1);
\path[li graph kante] (0) edge node {20} (2);
\path[li graph kante] (1) edge node {50} (3);
\path[li graph kante] (1) edge node {10} (4);
\path[li graph kante] (2) edge node {20} (3);
\path[li graph kante] (2) edge node {33} (4);
\path[li graph kante] (3) edge node {20} (4);
\path[li graph kante] (3) edge node {2} (5);
\path[li graph kante] (4) edge (5);
\end{tikzpicture}

\end{document}
