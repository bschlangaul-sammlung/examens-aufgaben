\documentclass{lehramt-informatik-aufgabe}
\liLadePakete{syntax}
\begin{document}
\liAufgabenTitel{Feld-Invertierer}

\section{Feld-Invertierer
\index{Rekursion}
\footcite[Aufgabe 1: Arrays und Rekursion, Diese Aufgabe stammt
aus der Vorlesung Konzepte der Programmierung von Prof. Bernhard
Westfechtel der Universität Bayreuth, WS 2017/18, Übungsblatt 8 und
wurde dankenswerterweise zur Verwendung in diesem Aufgabenblatt zur
Verfügung gestellt.]{aud:ab:2}}

\begin{enumerate}

%%
%
%%

\item Erstellen Sie eine neue Klasse \liJavaCode{ArrayInvertierer} mit einer
rekursiven Methode, die den Inhalt eines ihr übergebenen 1D-Arrays
gefüllt mit Strings invertiert. Auf diese Weise kann z.B. ein deutscher
Satz im Array gespeichert und dann verkehrt herum ausgegeben werden.

\emph{Wichtig:} Nicht das übergebene Array soll verändert werden,
sondern ein Neues erstellt und von der Methode zurückgegeben werden.

\emph{Tipp:} Sie dürfen dafür gerne auch rekursive Hilfsmethoden
benutzen.

%%
%
%%

\item Implementieren Sie dann eine main-Methode, in der Sie zwei
verschieden lange \liJavaCode{String}-Arrays erzeugen und die Wortreihenfolge
umkehren lassen. Das Ergebnis soll auf der Konsole ausgegeben werden
und könnte z.B. wie folgt aussehen.

\bigskip

{
\ttfamily
Den Satz\\
Ich find dich einfach klasse!\\
wuerde Meister Yoda so aussprechen:\\
klasse! einfach dich find Ich\\

Den Satz\\
Das war super einfach/schwer\\
wuerde Meister Yoda so aussprechen:\\
einfach/schwer super war Das
}

\bigskip

[optional] Wenn das ursprüngliche \liJavaCode{String}-Array selbst verändert
werden soll, braucht die rekursive Methode keine Rückgabe. Versuchen
Sie, diese Aufgabe ohne das Nutzen einer Hilfsmethode zu lösen.

\begin{antwort}
\liJavaDatei[firstline=3]{aufgaben/aud/ab_2/ArrayInvertierer}
\end{antwort}
\end{enumerate}

\end{document}
