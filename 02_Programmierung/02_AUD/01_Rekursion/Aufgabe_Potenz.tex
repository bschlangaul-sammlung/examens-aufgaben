\documentclass{lehramt-informatik-aufgabe}
\liLadePakete{syntax,mathe}
\begin{document}
\liAufgabenTitel{Potenz}

\section{Potenz
\index{Rekursion}
\footcite[Aufgabe 2: Rekursion, Seite 2]{aud:ab:1}}

Gegeben ist folgende Methode.

\liJavaDatei[firstline=5,lastline=12]{aufgaben/aud/ab_1/Rekursion}
\begin{enumerate}

%%
% (a)
%%

\item Beschreiben Sie kurz, woran man erkennt, dass es sich bei der
gegebenen Methode um eine rekursive Methode handelt. Gehen Sie dabei auf
wichtige Bestandteile der rekursiven Methode ein.

\begin{antwort}
Die Methode mit dem Namen \liJavaCode{function} ruft sich in der letzten
Code-Zeile selbst auf. Außerdem gibt es eine Abbruchbedingung
(\liJavaCode{if (e == 1) { return b * 1; }}), womit verhindert wird, dass die
Rekursion unendlich weiter läuft.
\end{antwort}

%%
% (b)
%%

\item Geben Sie die Rekursionsvorschrift für die Methode \liJavaCode{function}
an. Denken Sie dabei an die Angabe der Zahlenbereiche!

\begin{antwort}
\begin{equation*}
\text{int function(int b, int e)} =
\begin{cases}
\text{return b*1}, & \text{falls e = 1}.\\
\text{return b*function(b,e-1)}, & \text{falls e > 1}.
\end{cases}
\end{equation*}
\end{antwort}

%%
% (c)
%%

\item Erklären Sie kurz, was die Methode \liJavaCode{function} berechnet.

\begin{antwort}
Die Methode \liJavaCode{function} berechnet die Potenz $b^e$.
\end{antwort}
\end{enumerate}
\end{document}
