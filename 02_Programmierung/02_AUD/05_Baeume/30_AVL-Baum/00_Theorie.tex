\documentclass{lehramt-informatik-haupt}
\liLadePakete{mathe,baum,syntax}
\usepackage{xparse}

\begin{document}

%%%%%%%%%%%%%%%%%%%%%%%%%%%%%%%%%%%%%%%%%%%%%%%%%%%%%%%%%%%%%%%%%%%%%%%%
% Theorie-Teil
%%%%%%%%%%%%%%%%%%%%%%%%%%%%%%%%%%%%%%%%%%%%%%%%%%%%%%%%%%%%%%%%%%%%%%%%

\section{AVL-Bäume}

\begin{quellen}
\item \cite{wiki:avl-baum}
\item \cite[Kapitel 14.4.2, Seite 378-386 (PDF 394-402)]{saake}
\item \cite[bst]{net:html:visualgo}
\end{quellen}

%-----------------------------------------------------------------------
%
%-----------------------------------------------------------------------

\section{Visualisierungstools}

\begin{itemize}
\item \url{https://www.cs.usfca.edu/~galles/visualization/AVLtree.html}
\item \url{https://visualgo.net/bn/bst} (oben in den Tabs umschalten auf AVL)
\end{itemize}

\noindent
Ein AVL-Baum ist ein binärer Suchbaum, der \memph{höhenbalanciert} ist,
d.\,h. für jeden Knoten gilt:

\begin{center}
$|h(\text{rechterTeilbaum}) - h(\text{linkerTeilbaum})| \leq 1$
\end{center}

\noindent
Die \memph{„Entartung“} des Baums wird so vermieden. Die Höhe eines
AVL-Baums ist $h \in \mathcal{O}(\log n)$. Beim Einfügen und Löschen von
Knoten muss die AVL-Eigenschaft durch \memph{Rotationen}
wiederhergestellt werden.
\footcite[Seite 22 (PDF 16)]{aud:fs:5}

%-----------------------------------------------------------------------
%
%-----------------------------------------------------------------------

\section{Rotationsregeln}

\begin{center}
Balance-Faktor $=$ Knotenzahl rechter Teilbaum $-$ Knotenzahl linker Teilbaum
\end{center}

Rotationsregeln:
bO= „oberer“ Wurzelknoten, bU = Kindknoten von bO

\begin{itemize}
\item bO = -2, bU = -1  Rechtsrotation (Rechtsdrehung um bO)
\item bO = +2, bU = +1  Linksrotation (Linksdrehung um bO)
\item bO = -2, bU = +1  Links-Rechts-Rotation (Linksdrehung um bU, dann Rechtsdrehung um bO)
\item bO = +2, bU = -1  Rechts-Links-Rotation (Rechtsdrehung um bU, dann Linksdrehung um bO)
\end{itemize}

%%
%
%%

\subsection{Linksrotation}

\liJavaDatei[firstline=147,lastline=152]{baum/AVLBaum}

%%
%
%%

\subsection{Rechtsrotation}

\liJavaDatei[firstline=160,lastline=165]{baum/AVLBaum}

Kindknoten, die nach der Drehung „im Weg stehen“, werden „umgeklappt“,
also vom linken zum rechten Kind der nächsten Ebene und umgekehrt

\literatur

\end{document}
