\documentclass{lehramt-informatik-haupt}
\liLadePakete{mathe,baum,syntax,spalten}
\usepackage{xparse}

\begin{document}

%%%%%%%%%%%%%%%%%%%%%%%%%%%%%%%%%%%%%%%%%%%%%%%%%%%%%%%%%%%%%%%%%%%%%%%%
% Theorie-Teil
%%%%%%%%%%%%%%%%%%%%%%%%%%%%%%%%%%%%%%%%%%%%%%%%%%%%%%%%%%%%%%%%%%%%%%%%

\section{AVL-Bäume}

\begin{quellen}
\item \cite{wiki:avl-baum}
\item \cite[Kapitel 14.4.2, Seite 378-386 (PDF 394-402)]{saake}
\item \cite[bst]{net:html:visualgo}
\end{quellen}

%-----------------------------------------------------------------------
%
%-----------------------------------------------------------------------

\section{Visualisierungstools}

\begin{itemize}
\item \url{https://www.cs.usfca.edu/~galles/visualization/AVLtree.html}
\item \url{https://visualgo.net/bn/bst} (oben in den Tabs umschalten auf AVL)
\end{itemize}

\noindent
Ein AVL-Baum ist ein binärer Suchbaum, der \memph{höhenbalanciert} ist,
d.\,h. für jeden Knoten gilt:

\begin{center}
$|h(\text{rechterTeilbaum}) - h(\text{linkerTeilbaum})| \leq 1$
\end{center}

\noindent
Die \memph{„Entartung“} des Baums wird so vermieden. Die Höhe eines
AVL-Baums ist $h \in \mathcal{O}(\log n)$. Beim Einfügen und Löschen von
Knoten muss die AVL-Eigenschaft durch \memph{Rotationen}
wiederhergestellt werden.
\footcite[Seite 22 (PDF 16)]{aud:fs:5}

\begin{liProjektSprache}{Baum}
baum binär (
  setze: 1 2 3 4 5;
  drucke;
)

baum avl (
  setze: 1 2 3 4 5;
  drucke;
)
\end{liProjektSprache}

\begin{multicols}{2}
\liPseudoUeberschrift{Binärer Suchbaum}

\begin{tikzpicture}[li binaer baum]
\Tree
[.1
  \edge[blank]; \node[blank]{};
  [.2
    \edge[blank]; \node[blank]{};
    [.3
      \edge[blank]; \node[blank]{};
      [.4
        \edge[blank]; \node[blank]{};
        [.5 ]
      ]
    ]
  ]
]
\end{tikzpicture}

\columnbreak

\liPseudoUeberschrift{AVL-Baum}

\begin{tikzpicture}[li binaer baum]
\Tree
[.\node[label=1]{2};
  [.\node[label=0]{1}; ]
  [.\node[label=0]{4};
    [.\node[label=0]{3}; ]
    [.\node[label=0]{5}; ]
  ]
]
\end{tikzpicture}

\end{multicols}

%-----------------------------------------------------------------------
%
%-----------------------------------------------------------------------

\section{Rotationsregeln}

\begin{center}
$b$ (Balance-Faktor) $=$ Knotenzahl rechter Teilbaum $-$ Knotenzahl linker Teilbaum
\end{center}

\begin{description}
\item[$bO =$] Balance-Faktor des „oberen“ Wurzelknotens
\item[$bU =$] Balance-Faktor des Kindknoten von $bO$
\end{description}

\noindent
\begin{tabular}{l|l}
$bO = -2$, $bU = -1$ &
Rechtsrotation (Rechtsdrehung um $bO$) \\

$bO = +2$, $bU = +1$ &
Linksrotation (Linksdrehung um $bO$) \\

$bO = -2$, $bU = +1$ &
Links-Rechts-Rotation (Linksdrehung um $bU$, dann Rechtsdrehung um $bO$) \\

$bO = +2$, $bU = -1$ &
Rechts-Links-Rotation (Rechtsdrehung um $bU$, dann Linksdrehung um $bO$) \\
\end{tabular}

%%
%
%%

\subsection{Linksrotation}

\liJavaDatei[firstline=70,lastline=78]{baum/AVLBaum}

%%
%
%%

\subsection{Rechtsrotation}

\liJavaDatei[firstline=60,lastline=68]{baum/AVLBaum}

Kindknoten, die nach der Drehung „im Weg stehen“, werden „umgeklappt“,
also vom linken zum rechten Kind der nächsten Ebene und umgekehrt

\liJavaDatei[firstline=3]{baum/AVLBaum}

\literatur

\end{document}
