\documentclass{lehramt-informatik-haupt}
\usepackage{multicol}
\liLadePakete{baum}

\begin{document}

%%%%%%%%%%%%%%%%%%%%%%%%%%%%%%%%%%%%%%%%%%%%%%%%%%%%%%%%%%%%%%%%%%%%%%%%
% Theorie-Teil
%%%%%%%%%%%%%%%%%%%%%%%%%%%%%%%%%%%%%%%%%%%%%%%%%%%%%%%%%%%%%%%%%%%%%%%%

\chapter{B-Bäume}

\begin{quellen}
\item \cite{wiki:bbaum}
\item \cite[Kapitel 14.4.3, Seite 386-399 (PDF 402-415)]{saake}
\item \cite[Kapitel 13.5.4.2 Balancierte Mehrwegbäume, Seite 464, wird
nur erwähnt, nicht beschrieben]{schneider}
\end{quellen}

\url{https://www.cs.usfca.edu/~galles/visualization/BTree.html}

%-----------------------------------------------------------------------
%
%-----------------------------------------------------------------------

\section{Definition}

Ein Baum heißt genau dann B-Baum, wenn gilt:
%

\begin{itemize}
\item Jeder Knoten außer der Wurzel enthält zwischen \memph{$k$ und $2k$
Elemente} (Schlüsselwerte), $k$ wird als \memph{Ordnung} des B-Baums
bezeichnet.
%
\item Jeder \memph{Knoten ist entweder ein Blatt (ohne Kinder)} oder hat
\memph{mindestens $k + 1$ und höchstens $2k + 1$ Kind-Knoten}.
%
\item Der Wurzelknoten ist \memph{entweder ein Blatt oder hat mindestens
2 Nachfolger}.
%
\item Alle Blätter haben die \memph{gleiche Tiefe}, d.\,h. alle Wege von
der Wurzel bis zu den Blättern sind gleich lang. Pfade haben die Länge
$h-1$, wobei $h$ die Höhe des gesamten Baums ist.
\end{itemize}

\footcite[Seite 32 (PDF 26)]{aud:fs:5}

%-----------------------------------------------------------------------
%
%-----------------------------------------------------------------------

\section{Einfügen}

Das Einfügen in einen B-Baum erfolgt \memph{nur in den Blattknoten}.
Wenn in einem Blattknoten die \memph{maximale Anzahl} von Elementen
($2k$) erreicht ist, findet ein \memph{Split} statt, d.\,h. die
Elemente werden aufgeteilt und ein neuer Knoten entsteht. Das
\memph{mittlere Element} des ursprünglichen Knotens wird dabei \memph{in
den Elternknoten integriert}.
\footcite[Seite 32 (PDF 26)]{aud:fs:5}

%-----------------------------------------------------------------------
%
%-----------------------------------------------------------------------

\section{Suchen}

\begin{enumerate}
\item Beginnend mit Wurzelknoten wird Knoten jeweils \memph{von links
nach rechts} durchsucht:

\item \memph{Stimmt} Element mit gesuchtem Schlüsselwert
\memph{überein}, ist der Satz \memph{gefunden}.

\item Ist das \memph{Element größer} als der gesuchte Wert, wird die
Suche im \memph{links} hängenden Unterbaum \memph{fortgesetzt}.

\item Ist das \memph{Element kleiner} als der gesuchte Wert, wird der
Vergleich mit dem \memph{nächsten Element der Wurzel wiederholt}.

\item Ist auch das letztes Element der Wurzel noch kleiner als der
gesuchte Wert, dann wird die Suche im rechten Unterbaum des Elements
fortgesetzt.

\item Falls ein weiterer Abstieg in den Unterbaum nicht möglich ist
(d.\,h. Blattknoten), wird die Suche abgebrechen. Dann ist kein Satz mit
dem gewünschten Schlüsselwert vorhanden.
\end{enumerate}

\footcite[Seite 37 (PDF 31)]{aud:fs:5}

%-----------------------------------------------------------------------
%
%-----------------------------------------------------------------------

\section{Löschen}

Suche den Knoten, in dem das zu löschende Element E liegt.

\begin{itemize}
\item Falls das Element $E$ im Blattknoten liegt, dann lösche $E$ dort
und behandle ggf. entstehenden Unterlauf durch Mischen.

\item Falls das Element $E$ in einem inneren Knoten liegt, dann
untersuche den linken und rechten Unterbaum von $E$:

\begin{itemize}
\item Betrachte den Blattknoten mit dem direkten Vorgänger $E'$ von $E$
und den Blattknoten mit direktem Nachfolger $E''$ von $E$.

\item Wähle den Blattknoten aus, der mehr Elemente hat. Falls beide
Blattknoten gleich viele Elemente haben, wähle zufällig einen der beiden
aus.

\item Ersetze das zu löschende Element $E$ durch $E'$ bzw. $E''$ aus dem
gewählten Blattknoten.

\item Lösche $E'$ bzw. $E''$ im gewählten Blattknoten und behandle ggf.
entstehenden Unterlauf in diesem Blattknoten.
\end{itemize}
\end{itemize}

\footcite[Seite 39 (PDF 32)]{aud:fs:5}

\subsection{Hinweise zum Unterlauf:}

Ein Unterlauf entsteht auf Blattebene. Der Unterlauf wird durch Mischen
des Unterlaufknotens mit seinem Nachbarknoten und dem darüberliegenden
Element beseitig.
\footcite[Seite 40 (PDF 33)]{aud:fs:5}

\literatur

\end{document}
