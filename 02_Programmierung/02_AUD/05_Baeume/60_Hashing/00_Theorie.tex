\documentclass{lehramt-informatik-haupt}
\liLadePakete{syntax,mathe}
\usepackage{amsmath}

\begin{document}

%%%%%%%%%%%%%%%%%%%%%%%%%%%%%%%%%%%%%%%%%%%%%%%%%%%%%%%%%%%%%%%%%%%%%%%%
% Theorie-Teil
%%%%%%%%%%%%%%%%%%%%%%%%%%%%%%%%%%%%%%%%%%%%%%%%%%%%%%%%%%%%%%%%%%%%%%%%

\chapter{Hashing}

\begin{liQuellen}
\item \cite[Kapitel 6.3.3 Streutabellen (Hashing), Seite
188-189]{schneider}
\item \cite[Seite 22-39]{aud:fs:tafeluebung-10}
\item \cite[Seite 419-439]{saake}
\item \cite{wiki:hashtabelle}
\end{liQuellen}

%-----------------------------------------------------------------------
%
%-----------------------------------------------------------------------

\section{Grundlagen}
Das Ziel von Hashing ist einerseits einen \memph{extrem großen
Schlüsselraum} auf einen vernünftig \memph{kleinen Bereich von ganzen
Zahlen} abzubilden und andererseits dass \memph{zwei Schlüssel auf die
selbe Zahl} abgebildet werden, soll möglichst \memph{unwahrscheinlich}
sein.\footnote{Seite 4
\url{https://moves.rwth-aachen.de/wp-content/uploads/SS15/dsal/lec13.pdf}}

Daten werden in einer \memph{Streu(wert)tabelle} (hash table) abgelegt.
Aufgrund des wahlfreien Zugriffs eignen sich Felder zum Abspeichern der
Daten.
%
Eine \memph{Hashfunktion} $h$ bildet ein Datenelement auf einen
\memph{Hashwert} ab. Das Datenelement benötigt dazu einen
\memph{Schlüssel} (key), der das Element eindeutig identifiziert. Der
Hashwert wird als Index in dem Feld verwendet. Das Datenelement wird
im entsprechenden \memph{Bucket} der Tabelle gespeichert.
%
Bei der \memph{Suche} nach einem Element mit \memph{bekanntem Schlüssel}
wird der Index mittels der Hashfunktion bestimmt. Dies geschieht mit
\memph{konstantem Aufwand}.
%
Der Aufwand des Nachschlagens an entsprechender Stelle ist abhängig von
der \emph{Organisationsform}.
\footcite[Seite 25]{aud:fs:tafeluebung-10}

%-----------------------------------------------------------------------
%
%-----------------------------------------------------------------------

\section{Kollisionen}

Da Hashfunktionen im Allgemeinen  \memph{nicht eindeutig (injektiv)}
sind, können zwei unterschiedliche Schlüssel zum selben Hash-Wert, also
zum selben Feld in der Tabelle, führen. Dieses Ereignis wird als
\memph{Kollision} bezeichnet. In diesem Fall muss die Hashtabelle
mehrere Werte in demselben Bucket aufnehmen.

Eine Kollision benötigt bei der Suche eine spezielle Behandlung durch
das Verfahren: Zunächst wird aus einem Suchschlüssel wieder ein Hashwert
berechnet, der den Bucket des gesuchten Datenobjektes bestimmt; dann
muss noch durch direkten Vergleich des Suchschlüssels mit den Objekten
im Bucket das gesuchte Ziel bestimmt werden.

Zur Behandlung von Kollisionen werden kollidierte Daten nach einer
\memph{Ausweichstrategie in alternativen Feldern} oder in einer
\memph{Liste} gespeichert. Schlimmstenfalls können Kollisionen zu einer
\memph{Entartung der Hashtabelle} führen, wenn wenige Hashwerte sehr
vielen Objekten zugewiesen wurden, während andere Hashwerte unbenutzt
bleiben.\footcite{wiki:hashtabelle}

\section{Kollisionsauflösungsstrategien}

Um das Kollisions-Problem zu handhaben, gibt es diverse
Kollisionsauflösungsstrategien.

%%
%
%%

\subsection{geschlossenes Hashing mit offener Adressierung}

Wenn dabei ein Eintrag an einer schon belegten Stelle in der
Tabelle abgelegt werden soll, wird stattdessen eine \memph{andere freie
Stelle genommen}. Häufig werden drei Varianten unterschieden:

\subsubsection{lineares Sondieren}

es wird um ein \memph{konstantes Intervall} verschoben nach einer freien
Stelle gesucht. Meistens wird die Intervallgröße auf 1 festgelegt.

\subsubsection{quadratisches Sondieren}

Nach jedem erfolglosen Suchschritt wird das \memph{Intervall quadriert}.

\liPseudoUeberschrift{Beispiele}

\begin{itemize}

%%
%
%%

\item \liPseudoUeberschrift{nach Foliensatz der TU Braunschweig}
\footnote{Seite 25\url{https://www.ibr.cs.tu-bs.de/courses/ws0708/aud/skript/hash.np.pdf}}

\liPseudoUeberschrift{Formel}

$h(k, i) := h'(k) + (-1)^{i+1} \cdot \left\lfloor \frac{i+1}{2}\right\rfloor ^2 \mod m$

$k$, $k+1^2$, $k-1^2$, $k+2^2$, $k-2^2$,
$\ldots$
$k+(\frac{m-1}{2})^2$, $k-(\frac{m-1}{2})^2 \bmod m$

\liPseudoUeberschrift{Werte}

$m=19$, d. h. das Feld (die Tabelle) hat die Index-Nummern 0 bis 18.
$k = h(x) = 7$

\liPseudoUeberschrift{Sondierungsfolgen}

\def\tmp#1{{\tiny#1}}

\begin{tabular}{|l|l|l|l|}
i & Rechnung & Ergebnis & Index in der Tabelle\\\hline\hline
0 & $7 + 0^2$ & 7 & 7\\
1 & $7 + 1^2$ & 8 & 8\\
1 & $7 - 1^2$ & 6 & 6\\
2 & $7 + 2^2$ & 11 & 11\\
2 & $7 - 2^2$ & 3 & 2 \\
3 & $7 + 3^2 = 7 + 9$  & 16 & 16 \\

3 & $7 - 3^2 = 7 - 9$  & -2 &
17 \tmp{($19-2=10$) oder ($0 \rightarrow 0$, $-1 \rightarrow 18$, $-2 \rightarrow 17$)}
\\

4 & $7 + 4^2 = 7 + 16$  & 23 &
4 \tmp{($23-19=4$) oder ($19 \rightarrow 0$, $20 \rightarrow 1$, $21 \rightarrow 2$, $22 \rightarrow 3$, $23 \rightarrow 4$)}
\\

4 & $7 - 4^2 = 7 - 16$  & -9 &
10 \tmp{($19-9=10$) oder ($0 \rightarrow 0$, $-1 \rightarrow 18$, $-2 \rightarrow 17$, $\cdots$, $-9 \rightarrow 10$)}
\\

5 & $7 + 5^2 = 7 + 25$  & 32 & 13 \tmp{($32-19=13$)} \\
5 & $7 - 5^2 = 7 - 25$  & -18 & 1 \tmp{($19-18=1$)}\\
\end{tabular}

%%
%
%%

\item \liPseudoUeberschrift{nach Foliensatz der RWTH Aachen}
\footnote{Seite 19 \url{https://moves.rwth-aachen.de/wp-content/uploads/SS15/dsal/lec13.pdf}}

$h'(k) = k \mod 11$

$h(k, i) = (h'(k) + i + 3i^2) \mod 11$

$h'(17) = 17 \mod 11 = 6$

\liPseudoUeberschrift{Sondierungsfolgen}

$h(17, 0) = (17 + 0 + 3 \cdot 0^2) \mod 11 = 6$

$h(17, 1) = (17 + 1 + 3 \cdot 1^2) \mod 11 = 21 \mod 11 = 10$

$h(17, 2) = (17 + 2 + 3 \cdot 2^2) \mod 11 = 31 \mod 11 = 31 - 2 \cdot 11 = 9$

$h(17, 3) = (17 + 3 + 3 \cdot 3^2) \mod 11 = 47 \mod 11 = 47 - 4 \cdot 11 = 3$

\end{itemize}

\subsubsection{doppeltes Hashen}

eine \memph{weitere Hash-Funktion} liefert das Intervall.

%%
%
%%

\subsection{offenes Hashing mit geschlossener Adressierung}

Anstelle der gesuchten Daten enthält die Hashtabelle hier
\memph{Behälter} (englisch \memph{Buckets}), die alle Daten mit gleichem
Hash-Wert aufnehmen. Es müssen die \memph{Elemente im Behälter
durchsucht werden}. Oft wird die Verkettung durch eine lineare Liste pro
Behälter realisiert.
\footcite{wiki:hashtabelle}

%-----------------------------------------------------------------------
%
%-----------------------------------------------------------------------

\section{Die Divisionsrestmethode\footcite{wiki:divisionsrestmethode}}

Die Divisionsrestmethode - auch Modulo genannt liefert eine
Hashfunktion. Die Funktion lautet:  $h(k)=k{\bmod {m}}$. $m$ ist die
Größe der Hashtabelle. Die Hash-Funktion kann sehr schnell berechnet
werden. Die Wahl der Tabellengröße $m$ beeinflusst die
Kollisionswahrscheinlichkeit der Funktionswerte von $h$. Für
praxisrelevante Anwendungen liefert die Wahl einer Primzahl für $m$.

\section{Belegungsfaktor\footcite[Seite 29]{aud:fs:tafeluebung-10}}

$\text{Belegungsfaktor} =
\frac{\text{Anzahl tatsächlich eingetragenen Schlüssel}}
{\text{Anzahl Hashwerte}}$

\literatur

\end{document}
