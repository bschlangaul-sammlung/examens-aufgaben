\documentclass{lehramt-informatik-haupt}
\liLadePakete{baum}

\begin{document}

%%%%%%%%%%%%%%%%%%%%%%%%%%%%%%%%%%%%%%%%%%%%%%%%%%%%%%%%%%%%%%%%%%%%%%%%
% Theorie-Teil
%%%%%%%%%%%%%%%%%%%%%%%%%%%%%%%%%%%%%%%%%%%%%%%%%%%%%%%%%%%%%%%%%%%%%%%%

\chapter{Bäume}

\section{Grundlagen}

\begin{quellen}
\item \cite[Kapitel 6.2.2.3, Seite 186]{schneider}
\item \cite[Seite 345-419]{saake}
\end{quellen}

Bäume sind eine der \memph{wichtigsten dynamischen Datenstrukturen} in
der Informatik. Es können nicht nur Daten, sondern \memph{auch
Beziehungen }(z. B. Ordnungen) der Daten gespeichert werden. Bäume sind
aus Knoten aufgebaut, die durch (gerichtete) Kanten verbunden sind. Die
Daten werden in der Regel in den Knoten gespeichert. Die Wurzel eines
Baums besitzt nur auslaufende Kanten. Blätter sind Knoten mit nur einer
einlaufenden Kante.
\footcite[Seite 2]{aud:fs:5}

%%
%
%%

\subsection{Definition (Baum - rekursiv)}

Ein Baum ist leer oder er besteht aus einer Wurzel und einer leeren oder
nichtleeren endlichen Menge disjunkter Bäume (sogenannte Teilbäume).

%%
%
%%

\subsection{Definition (Binärbaum)}

Ein Binärbaum ist ein Baum, bei dem jeder Knoten genau zwei
Verzweigungsmöglichkeiten besitzt.

Diese Variante des Baumes wird sehr häufig verwendet. Es ist eine
Hierarchische Datenstruktur. Jedes Baumelement besitzt einen linken und
einen rechten Teilbaum. Diese Datenstruktur ist gut geeignet zum
Sortieren.
\footcite[Seite 3]{aud:fs:5}

\literatur

\end{document}
