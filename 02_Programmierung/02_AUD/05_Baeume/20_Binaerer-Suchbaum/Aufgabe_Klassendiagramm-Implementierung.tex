\documentclass{lehramt-informatik-aufgabe}
\liLadePakete{uml,syntax}
\begin{document}
\liAufgabenTitel{Klassendiagramm und Implementierung}

\section{Klassendiagramm und Implementierung
\index{Binärbaum}
\footcite[Aufgabe 1: Binärbaum, Seite 1]{aud:pu:5}}

\begin{enumerate}

%%
% (a)
%%

\item Erstellen Sie ein Klassendiagramm für einen Binärbaum.

\begin{antwort}
\liPseudoUeberschrift{einfacher Binärbaum}

\begin{tikzpicture}
\umlsimpleclass{Binärbaum}
\umlclass[x=4]{Knoten}{wert}{}
\umlaggreg[mult=0..1,arg=wurzel,pos=0.6]{Binärbaum}{Knoten}
\umlaggreg[mult=0..2,pos=0.8]{Knoten}{Knoten}
\end{tikzpicture}

\liPseudoUeberschrift{Binärbaum mit Kompositum}

\begin{tikzpicture}
\umlsimpleclass[x=-1,y=5]{Binärbaum}
\umlsimpleclass[x=3,y=5,type=abstrakt]{Baumelement}
\umlsimpleclass[x=1,y=3]{Abschluss}
\umlsimpleclass[x=5,y=3]{Datenknoten}
\umlsimpleclass[x=5,y=1]{Inhalt}

\umlassoc[mult2=1,stereo=wurzel \LeserichtungRechts]{Binärbaum}{Baumelement}
\umlVHVinherit{Abschluss}{Baumelement}
\umlVHVinherit{Datenknoten}{Baumelement}

\umlaggreg[mult=1]{Datenknoten}{Inhalt}
\umlVHaggreg[mult2=2,pos2=1.8,anchor1=50]{Datenknoten}{Baumelement}
\end{tikzpicture}
\end{antwort}

%%
% (b)
%%

\item Entwerfen Sie eine mögliche Implementierung zur Erzeugung eines
binären Baumes in Java.

\begin{antwort}
\liPseudoUeberschrift{einfacher Binärbaum}

\liJavaDatei[firstline=3]{baum/einfach/Baum}
\liJavaDatei[firstline=3]{baum/einfach/Knoten}

\liPseudoUeberschrift{Binärbaum mit Kompositum}

\liJavaDatei[firstline=3]{baum/kompositum/Ahnenbaum}
\liJavaDatei[firstline=3]{baum/kompositum/Baumelement}
\liJavaDatei[firstline=3]{baum/kompositum/Abschluss}
\liJavaDatei[firstline=3]{baum/kompositum/Datenknoten}
\liJavaDatei[firstline=3]{baum/kompositum/Person}
\end{antwort}
\end{enumerate}

\end{document}
