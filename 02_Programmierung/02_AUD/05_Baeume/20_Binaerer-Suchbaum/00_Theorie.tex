\documentclass{lehramt-informatik-haupt}
\liLadePakete{syntax,uml,baum}
\usepackage{amsmath}
\usepackage{tabularx}

\begin{document}

%%%%%%%%%%%%%%%%%%%%%%%%%%%%%%%%%%%%%%%%%%%%%%%%%%%%%%%%%%%%%%%%%%%%%%%%
% Theorie-Teil
%%%%%%%%%%%%%%%%%%%%%%%%%%%%%%%%%%%%%%%%%%%%%%%%%%%%%%%%%%%%%%%%%%%%%%%%

\chapter{Binärer Suchbaum}

\begin{liQuellen}
\item \cite{wiki:binaerbaum}
\item \cite{wiki:binaerer-suchbaum}
\item \cite[Kapitel 14.2.1, Seite 348-358 (PDF 364-374)]{saake}
\item \cite[Seite 4-14]{aud:fs:5}
\end{liQuellen}

%-----------------------------------------------------------------------
%
%-----------------------------------------------------------------------

\section{Visualisierungstools}

\begin{itemize}
\item \url{http://btv.melezinek.cz/binary-search-tree.html} Auch für Postorder etc
\item \url{https://visualgo.net/bn/bst}
\end{itemize}

\noindent
Ein binärer Suchbaum - häufig abgekürzt als \textbf{BST} (von englisch
\emph{Binary Search Tree}) - ist ein binärer Baum, bei dem die Knoten
\emph{„Schlüssel“} tragen, und die Schlüssel des \memph{linken
Teilbaums} eines Knotens nur \memph{kleiner (oder gleich)} und die des
\memph{rechten} Teilbaums nur \memph{größer}\footcite[Seite 5]{aud:fs:5}
als der Schlüssel des Knotens selbst sind.
\footcite{wiki:binaerer-suchbaum}

\begin{center}
\begin{tabular}{ll}
$S$ &
Wert eines Knotens \\

$S_\text{links}$ &
Wert der Wurzel des linken Teilbaumes \\

$S_\text{rechts}$ &
Wert der Wurzel des rechten Teilbaumes \\
\end{tabular}
\end{center}

Bei einem binären Suchbaum gilt:
$\text{S}_\text{links} \leq \text{S} < \text{S}_\text{recht}$

\section{Beispiel}

Anordnung von \verb|7, 3, 6, 9, 2, 9, 5, 1| in einem int-Baum
\footcite[Seite 5]{aud:fs:5}

\begin{center}
\begin{tikzpicture}[li binaer baum]
\Tree
[.7
  [.3
    [.2
      \edge[]; {1}
      \edge[blank]; \node[blank]{};
    ]
    [.6
      \edge[]; {5}
      \edge[blank]; \node[blank]{};
    ]
  ]
  [.9
    \edge[]; {9}
    \edge[blank]; \node[blank]{};
  ]
]
\end{tikzpicture}
\end{center}

%%
%
%%

\subsection{Traversierung eines Baums}

Es gibt verschiedene Möglichkeiten einen binären Baum zu
traversieren:

\begin{itemize}
\item \texttt{preorder}:

besuche die Wurzel, dann den linken Unterbaum, dann den rechten
Unterbaum; auch: WLR

\liJavaDatei[firstline=54,lastline=68]{baum/BinaerBaum}

\item \texttt{inorder}:

besuche den linken Unterbaum, dann die Wurzel, dann den rechten
Unterbaum;
auch: LWR

\liJavaDatei[firstline=36,lastline=52]{baum/BinaerBaum}

\item \texttt{postorder}:

besuche den linken Unterbaum, dann den rechten, dann die Wurzel;
auch: LRW

\liJavaDatei[firstline=70,lastline=84]{baum/BinaerBaum}

\item \texttt{levelorder}:

Beginnend bei der Baumwurzel werden die
Ebenen von links nach rechts durchlaufen.

\liJavaDatei[firstline=86,lastline=107]{baum/BinaerBaum}

\end{itemize}

\begin{center}
\begin{tikzpicture}[li binaer baum]
\Tree
[.7
  [.3
    [.2
      [.1 ]
      \edge[blank]; \node[blank]{};
    ]
    [.6
      [.5 ]
      \edge[blank]; \node[blank]{};
    ]
  ]
  [.9
    [.8 ]
    \edge[blank]; \node[blank]{};
  ]
]
\end{tikzpicture}
\end{center}

\begin{center}
\begin{tabular}{ll}
\texttt{preorder} & \texttt{7 3 2 1 6 5 9 8} \\
\texttt{inorder} & \texttt{1 2 3 5 6 7 8 9} \\
\texttt{postorder} & \texttt{1 2 5 6 3 8 9 7} \\
\texttt{levelorder} & \texttt{7 3 9 2 6 8 1 5} \\
\end{tabular}
\end{center}

%-----------------------------------------------------------------------
%
%-----------------------------------------------------------------------

\section{Löschen eines Knotens}

Die komplizierteste Operation im binären Suchbaum ist das Löschen. Dies
liegt daran, dass beim Entfernen eines inneren Knotens einer der beiden
Teilbäume des Knotens „hochgezogen“ und gegebenenfalls umgeordnet werden
muss.

Grundsätzlich müssen beim Löschen eines Knotens drei Fälle
berücksichtigt werden:

\begin{enumerate}
\item Der Knoten ist ein \memph{Blatt}. Dieser Fall ist der einfachste,
da hier nur der Elternknoten zu bestimmen ist und dessen Verweis auf den
Knoten entfernt werden muss.

\item Der Knoten besitzt nur \memph{einen Kindknoten}. In diesem Fall
ist der Verweis vom Elternknoten auf den Kindknoten des zulöschenden
Knotens umzulenken.

\item Der Knoten ist ein innerer Knoten mit \memph{zwei Kindknoten}.
Hierbei muss der Knoten durch den am weitesten links stehenden Knoten
des rechten Teilbaumes ersetzt werden, da dieser in der
Sortierreihenfolge der nächste Knoten ist. Alternativ kann auch der am
weitestens rechts stehende Knoten des linken Teilbaumes verwendet
werden.\footcite[364]{saake}
\end{enumerate}

%%
%
%%

\subsection{Knoten ohne Nachfolger\footcite[Seite 10]{aud:fs:5}}

\begin{compactenum}
\item Suche den zu löschenden Knoten
\item Lösche die Referenz vom Vorgängerkonten auf den zu löschenden Knoten
\end{compactenum}

\begin{minipage}{0.5\linewidth}
\begin{tikzpicture}[li binaer baum]
\Tree
[.7
  [.3
    [.2
      [.1 ]
      \edge[blank]; \node[blank]{};
    ]
    [.6
      [.5 ]
      \edge[blank]; \node[blank]{};
    ]
  ]
  [.11
    \edge[dotted]; \node[dotted]{9};
    \edge[blank]; \node[blank]{};
  ]
]
\end{tikzpicture}
\end{minipage}
\begin{minipage}{0.5\linewidth}
\begin{tikzpicture}[li binaer baum]
\Tree
[.7
  [.3
    [.2
      [.1 ]
      \edge[blank]; \node[blank]{};
    ]
    [.6
      [.5 ]
      \edge[blank]; \node[blank]{};
    ]
  ]
  [.11 ]
]
\end{tikzpicture}
\end{minipage}

%%
%
%%

\subsubsection{Knoten mit genau einem Nachfolger\footcite[Seite 11]{aud:fs:5}}

\begin{compactenum}
\item Suche den zu löschenden Knoten
\item Referenz von Vorgängerkonten auf den (einzigen) Nachfolgerknoten
des zu löschenden Knotens
\end{compactenum}

\begin{minipage}{0.5\linewidth}
\begin{tikzpicture}[li binaer baum]
\Tree
[.7
  [.\node(3){3};
    [.2
      [.1 ]
      \edge[blank]; \node[blank]{};
    ]
    [.\node[dotted]{6};
      [.\node(5){5}; ]
      \edge[blank]; \node[blank]{};
    ]
  ]
  [.11
    [.9 ]
    \edge[blank]; \node[blank]{};
  ]
]
\draw (3) -- (5);
\end{tikzpicture}
\end{minipage}
\begin{minipage}{0.5\linewidth}
\begin{tikzpicture}[li binaer baum]
\Tree
[.7
  [.3
    [.2
      [.1 ]
      \edge[blank]; \node[blank]{};
    ]
    [.5 ]
  ]
  [.11
    [.9 ]
    \edge[blank]; \node[blank]{};
  ]
]
\end{tikzpicture}
\end{minipage}
%%
%
%%

\subsection{Knoten mit zwei Nachfolgern\footcite[Seite 12]{aud:fs:5}}

\begin{compactenum}
\item Suche den zu löschenden Knoten
\item Suche den „kleinsten“ Knoten im rechten Teilbaum
\item Ersetze den zu löschenden Knoten durch den „kleinsten“ Knoten
\end{compactenum}

\begin{minipage}{0.5\linewidth}
\begin{tikzpicture}[li binaer baum]
\Tree
[.\node(7){7};
  [.\node[dotted](3){3};
    [.\node(2){2};
      [.1 ]
      \edge[blank]; \node[blank]{};
    ]
    [.\node(6){6};
      [.\node(5){5}; ]
      \edge[blank]; \node[blank]{};
    ]
  ]
  [.11
    [.9 ]
    \edge[blank]; \node[blank]{};
  ]
]
\draw[dashed] (7) ..controls +(south east:5.1).. (5);
\draw[dashed] (5) ..controls +(east:0.7).. (6);
\draw[dashed] (5) ..controls +(south west:0.7).. (2);
\end{tikzpicture}
\end{minipage}
\begin{minipage}{0.5\linewidth}
\begin{tikzpicture}[li binaer baum]
\Tree
[.7
  [.5
    [.2
      [.1 ]
      \edge[blank]; \node[blank]{};
    ]
    [.6 ]
  ]
  [.11
    [.9 ]
    \edge[blank]; \node[blank]{};
  ]
]
\end{tikzpicture}
\end{minipage}

\literatur

\end{document}
