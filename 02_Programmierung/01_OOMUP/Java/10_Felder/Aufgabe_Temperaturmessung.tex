\documentclass{lehramt-informatik-aufgabe}
\liLadePakete{syntax}
\begin{document}
\liAufgabenTitel{Temperaturmessung}

\section{Temperaturmessung
\index{Feld (Array)}
\footcite[Seite 61, Klett, Informatik 3, S. 68]{oomup:fs:3}}

Steffi will ein Jahr lang jeden Tag um 15 Uhr die Temperatur auf ihrem
Balkon messen und die Ergebnisse auswerten. Dazu definiert sie eine
Klasse \java{Tempmessung}.

\begin{enumerate}

%%
% a)
%%

\item Lege ein Feld \java{temperatur} an, welches die reellen Werte für
jeden Tag eines Jahres aufnehmen kann. Definiere eine Methode, um das
Feld mit zufälligen Temperaturwerten zu belegen.

%%
% b)
%%

\item Nach genau einem Jahr sollen mithilfe dreier Methoden der Tag mit
dem höchsten Temperaturwert, die niedrigste gemessene Temperatur und der
Durchschnittswert aller Messwerte bestimmt werden. Implementiere
geeignete Methoden.\index{Implementierung in Java}

\begin{antwort}
\inputcode{aufgaben/oomup/pu_3/Tempmessung}
\end{antwort}

\end{enumerate}

\end{document}
