\documentclass{lehramt-informatik-haupt}
\liLadePakete{syntax}
\begin{document}

%%%%%%%%%%%%%%%%%%%%%%%%%%%%%%%%%%%%%%%%%%%%%%%%%%%%%%%%%%%%%%%%%%%%%%%%
% Theorie-Teil
%%%%%%%%%%%%%%%%%%%%%%%%%%%%%%%%%%%%%%%%%%%%%%%%%%%%%%%%%%%%%%%%%%%%%%%%

\chapter{Felder}

\section{Initialsierung
\footcite[Seite 46]{oomup:fs:3}}

Oft sind die zu speichernden Werte erst zur Laufzeit bekannt. In diesem
Fall kann man Platz für eine festgelegte Anzahl an Elementen
reservieren.

\texttt{<Datentyp>[] <Bezeichner> = new <Datentype>[<Anzahl>];}

Die Anzahl kann ein beliebiger Ausdruck sein, der zu einem \java{int}
ausgewerte wird. So angelegte Felder werden mit \memph{0-wertigen Einträgen}
gefüllt. Zum Beispiel:

\begin{minted}{java}
boolean[] korrigiert = new boolean[23];
// -> {false, false, ..., false}
\end{minted}

\begin{minted}{java}
boolean[] noten = new int[6];
// -> {0, 0, 0, 0, 0, 0}
\end{minted}

Die Größe eines einmal angelegten Feldes kann nicht mehr verändert
werden. Wieviele Elemente in einem Feld Platz haben, kann mit dem
Attribut \texttt{.length} abgefragt werden.

\literatur

\end{document}
