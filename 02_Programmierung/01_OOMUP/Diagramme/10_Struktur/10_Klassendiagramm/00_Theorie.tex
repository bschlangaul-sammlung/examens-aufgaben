\documentclass{lehramt-informatik-haupt}
\liLadePakete{uml,syntax}

\begin{document}

%%%%%%%%%%%%%%%%%%%%%%%%%%%%%%%%%%%%%%%%%%%%%%%%%%%%%%%%%%%%%%%%%%%%%%%%
% Theorie-Teil
%%%%%%%%%%%%%%%%%%%%%%%%%%%%%%%%%%%%%%%%%%%%%%%%%%%%%%%%%%%%%%%%%%%%%%%%

\chapter{Klassendiagramm}

\begin{tikzpicture}
\begin{umlpackage}{p}

\begin{umlpackage}{sp1}
\umlclass[template=T]{A}{
  n : uint \\ t : float
}{}
\umlclass[y=-3]{B}{
  d : double
}{
  \umlvirt{setB(b : B) : void} \\ getB() : B}
\end{umlpackage}
\begin{umlpackage}[x=10,y=-6]{sp2}
\umlinterface{C}{
  n : uint \\ s : string
}{}
\end{umlpackage}
\umlclass[x=2,y=-10]{D}{
  n : uint
  }{}
\umlsimpleinterface{E}
\end{umlpackage}

\umlassoc[geometry=-|-, arg1=tata, mult1=*, pos1=0.1, align1=right, arg2=toto, mult2=1, pos2=2.9, align2=left]{C}{B}
\umlunicompo[geometry=-|, arg=titi, mult=*, pos=1.7, stereo=vector]{D}{C}
\umlimport[geometry=|-, anchors=90 and 50, name=import]{sp2}{sp1}
\umlaggreg[arg=tutu, mult=1, pos=0.8, angle1=30, angle2=60, loopsize=2cm]{D}{D}
\umlinherit[geometry=-|]{D}{B}
\umlnote[x=2.5,y=-6, width=3cm]{B}{Je suis une note qui concerne la classe B}
\umlnote[x=7.5,y=-2]{import-2}{note}
\end{tikzpicture}

\begin{liQuellen}
\item \cite[Seite 108-169]{rupp}
\item \cite[Zusammengefügtes PDF Seite 30-32 / Kapitel „Klassen“ 8-10]{brinda}
\item \cite{wiki:klassendiagramm}
\end{liQuellen}

Klassendiagramme (class diagrams) beschreiben die Entitäten eines
Systems und welche Beziehungen sie untereinander eingehen können
(Struktur der Daten). Neben Paketdiagrammen werden Klassendiagramme bzw.
deren Notation wahrscheinlich am häufigsten eingesetzt.
\footcite[Seite 166]{schatten}

\begin{description}

%%
%
%%

\item[Klasse]
Klassen werden durch Rechtecke dargestellt, die entweder nur den Namen
der Klasse (fett gedruckt) tragen oder zusätzlich auch Attribute und
Methoden spezifiziert haben.
Dabei werden diese drei Rubriken (engl. compartment) – Klassenname,
Attribute, Methoden – jeweils durch eine horizontale
Linie getrennt.
\liLatexCode{\umlclass{Name}{Attribute}{Methoden}}
\footcite{wiki:klassendiagramm}
\footcite[Seite 49-50 (PDF 65-66)]{uml}

\begin{tikzpicture}
\umlclass{Name}{Attribute}{Methoden}
\umlsimpleclass[x=3]{Lehrer}
\umlclass[x=8]{Lehrer}{
- alter: int\\
+ name: String\\
+ guteLaune: Boolean = false
}{
+ frageAb()
}
\end{tikzpicture}

%%
%
%%

\item[Abstrakte Klassen]
Kursiv oder mit Untertitle \{abstract\} oder mit <<>>
\liLatexCode{\umlsimpleclass[type=abstract]{Name}}
\liLatexCode{\umlsimpleclass[tags=abstract]{Name}}
\liLatexCode{\umlsimpleclass[type=abstrakt]{Name}}

\begin{tikzpicture}
\umlsimpleclass[type=abstract,x=0]{Lehrer}
\umlsimpleclass[tags=abstract,x=3]{Lehrer}
\umlsimpleclass[type=abstrakt,x=6]{Lehrer}
\end{tikzpicture}

%%
%
%%

\item[Sichtbarkeit]
$-$: private, $+$: public, $\#$: protected, \~{}: package
\footcite[Seite 141-142 (PDF 158-159)]{uml}

\begin{tikzpicture}
\umlclass{Lehrer}{
  - geburtstag \\
  + name \\
  \# spitzname \\
  \~{} hobby \\
}{}
\end{tikzpicture}

\item[Klassenattribute]
Klassenattribute (statische Attribute) werden unterstrichen \liLatexCode{\umlstatic{- gehalt}}
\footcite[Seite 118,121]{rupp}

\begin{center}
\begin{tikzpicture}
\umlclass{Lehrer}{
  \umlstatic{- gehalt}
}{}
\end{tikzpicture}
\end{center}

%%
%
%%

\item[Multiplizität]
\zB 0..1, 1..1, 1, 0..*, *, n..m
\footcite[Seite 98 (PDF 115)]{uml}
\liLatexCode{\umlcompo[mult1=1,mult2=1..*]{A}{B}}

%%
%
%%

\item[Rollen (Assoziationsenden)]
\footcite[Seite 145]{rupp}
\liLatexCode{\umlassoc[arg1=faecher,arg2=lehrer]{Fach}{Lehrer}}

\begin{center}
\begin{tikzpicture}
\umlsimpleclass{Fach}
\umlsimpleclass[x=4]{Lehrer}
\umlassoc[arg1=faecher,arg2=lehrer]{Fach}{Lehrer}
\end{tikzpicture}
\end{center}

\item[Leserichtung]

Oftmals ist der gewählte Assoziationsname nicht für beide
Assoziationsrichtungen gleicher­ maßen treffend. Ist dies der Fall, so
kann eine bevorzugte Leserichtung für eine Assoziation angegeben werden.
Die bevorzugte Leserichtung wird durch ein ausgefülltes Dreieck
darge­stellt, das nach dem Assoziationsnamen platziert wird und dessen
Spitze in die gewünschte Leserichtung weist.
\footcite[Seite 144]{rupp}
\liLatexCode{
  \umlassoc[name=fach-lehrer]{Fach}{Lehrer}
  \node[below=0cm of fach-lehrer-middle] {unterrichtet \liLeserichtungRechts};
}

\begin{center}
\begin{tikzpicture}
\umlsimpleclass{Fach}
\umlsimpleclass[x=5]{Lehrer}
\umlassoc[name=fach-lehrer]{Fach}{Lehrer}

\node[below=0cm of fach-lehrer-middle] {unterrichtet \liLeserichtungRechts};
\end{tikzpicture}
\end{center}

%%
%
%%

\item[Methoden/Operationsdeklaration]
\zB \verb|-div(divident : Int, divisor : Int) : double|

%%
%
%%

\item[Assoziation] Eine Assoziation beschreibt eine Beziehung zwischen
zwei oder mehr Klassen. An den Enden von Assoziationen sind häufig
Multiplizitäten vermerkt. \liLatexCode{\umlassoc[mult1=1..*,mult2=1]{Fach}{Lehrer}}

\begin{center}
\begin{tikzpicture}
\umlsimpleclass{Fach}
\umlsimpleclass[x=4]{Lehrer}
\umlassoc[mult1=1..*,mult2=1]{Fach}{Lehrer}
\end{tikzpicture}
\end{center}

%%
%
%%

\item[Generalisierung (Inheritance)]

Eine Generalisierung in der UML ist
eine \emph{gerichtete} Beziehung zwischen einer \emph{generelleren} und
einer \emph{spezielleren} Klasse. Der Pfeil zeigt von der spezielleren
Klasse zur generellen Klasse.
\footcite[Kapitel 6.4.6 Generalisierung, Seite 135]{rupp}
\liLatexCode{\umlinherit{A}{B}}

\begin{center}
\begin{tikzpicture}
\umlsimpleclass[y=1.4,tags=generelle Klasse]{Mensch}
\umlsimpleclass[tags=spezielle Klasse]{Lehrer}
\umlinherit{Lehrer}{Mensch}
\end{tikzpicture}
\end{center}

\item[Realisierung (Realization)] Gestrichelte Linie mit nicht
ausgefüllten Dreieck als Pfeilspitze. \liLatexCode{\umlreal{A}{B}}
\footcite[Kapitel 6.4.13, Seite 164]{rupp}

\begin{center}
\begin{tikzpicture}
\umlsimpleclass[]{Implementierung}
\umlsimpleclass[x=4cm]{Spezifikation}
\umlreal{Implementierung}{Spezifikation}
\end{tikzpicture}
\end{center}

%%
%
%%

\item[Aggregation (Aggregation)]

Teil-Ganzes-Beziehung, nicht-ausgefüllte Raute am „Ganzen“-Objekt,
in Java keine entsprechung
\liLatexCode{\umlaggreg{A}{B}}

\begin{center}
\begin{tikzpicture}
\umlsimpleclass{Unterrichtsstunde}
\umlsimpleclass[x=6]{Schüler}
\umlaggreg[mult1=20..*,mult2=12..*]{Unterrichtsstunde}{Schüler}
\end{tikzpicture}
\end{center}

%%
%
%%

\item[Komposition (Composition)]

starke Aggregation, ausgefüllte Raute, Teilobjekte existenzabhängig von
Aggregationsobjekt: \zB Bank, Konto
\liLatexCode{\umlcompo{A}{B}}

\begin{center}
\begin{tikzpicture}
\umlsimpleclass{Schule}
\umlsimpleclass[x=4]{Klassenzimmer}
\umlcompo[mult1=1,mult2=1..*]{Schule}{Klassenzimmer}
\end{tikzpicture}
\end{center}

\item[Assoziationsklasse]
beschreibt Assoziation zwischen anderen Klassen beschreibt, wird
aufgelöst: entweder Verteilung der Attribute und Methoden oder
eigenständige Klasse. Als „Kochrezept“: „Was bei den Multiplizitäten
vorher bei den einzelnen Klassen steht, hüpft jeweils auf die andere
Seite der aufgelösten Assoziationsklasse.“

\item[Abhängigkeitsbeziehung (Dependency)]

Gestrichelte Line mit offenen Pfeil. Der Pfeil zeigt \emph{vom
abhängigen auf das unabhängige} Modellelement. \liLatexCode{\umldep{A}{B}}
\footcite[Kapitel 6.4.10 Abhängigkeitsbeziehung, Seite 159]{rupp}

\begin{center}
\begin{tikzpicture}
\umlsimpleclass[tags=abhängig]{Schüler}
\umlsimpleclass[x=4,tags=unabhängig]{Lehrer}
\umldep{Schüler}{Lehrer}
\end{tikzpicture}
\end{center}

\begin{description}
\item[Verwendungsbeziehung]

Eine Usage-Beziehung wird als Abhängigkeitsbeziehung  mit dem
Schlüsselwort «use» modelliert. Der Pfeil zeigt vom „unvollständigen“
Element auf das benötigte Element.
\footcite[Kapitel 6.4.11 Verwendungsbeziehung, Seite 161]{rupp}
\liLatexCode{\umldep[stereo=use]{A}{B}}

\begin{center}
\begin{tikzpicture}
\umlsimpleclass[tags=unvollständig]{Schüler}
\umlsimpleclass[x=4,tags=benötigt]{Lehrer}
\umldep[stereo=use]{Schüler}{Lehrer}
\end{tikzpicture}
\end{center}

\end{description}

\end{description}

%-----------------------------------------------------------------------
%
%-----------------------------------------------------------------------

\section{Arten von Assoziationen}

\begin{description}
\item[Ungerichtete Assoziation]
Linie mit Beschriftung

\begin{center}
\begin{tikzpicture}
\umlsimpleclass{A}
\umlsimpleclass[x=4]{B}
\umlassoc{A}{B}
\end{tikzpicture}
\end{center}

Assoziation mit sich selbst.

\begin{center}
\begin{tikzpicture}
\umlsimpleclass{A}
\umlassoc[angle1=-90,angle2=0,loopsize=3cm,arg1=next,arg2=previous,mult1=1,mult2=1]{A}{A}
\end{tikzpicture}
\end{center}

\item[Gerichtete Assoziation]
Linie mit Beschriftung und Leserichtungspfeil

\begin{center}
\begin{tikzpicture}
\umlsimpleclass{A}
\umlsimpleclass[x=4]{B}
\umluniassoc{A}{B}
\end{tikzpicture}
\end{center}

\item[Reflexive Assoziation]
Pfeil mit Beschriftung

%-----------------------------------------------------------------------
%
%-----------------------------------------------------------------------

\item[Navigierbare Assoziation]

Die Navigierbarkeit von Assoziationen wird durch eine Pfeilspitze am
Ende einer Assoziation ausgedrückt. Die Pfeilrichtung zeigt an, dass die
Instanzen der Klasse A die Instanzen der Klasse B in Pfeilrichtung
„kennen“.
\footcite[Seite 150]{rupp}

%%
%
%%

\begin{description}
\item[Unspezifizierte Navigierbarkeit] \liLatexCode{\umlassoc{A}{B}}

\begin{center}
\begin{tikzpicture}
\umlsimpleclass{A}
\umlsimpleclass[x=4]{B}
\umlassoc{A}{B}
\end{tikzpicture}
\end{center}

%%
%
%%

\item[Unidirektionale Navigierbarkeit] \liLatexCode{\umluniassoc{A}{B}}

\begin{center}
\begin{tikzpicture}
\umlsimpleclass{A}
\umlsimpleclass[x=4]{B}
\umluniassoc{A}{B}
\end{tikzpicture}
\end{center}
\end{description}
\end{description}

\literatur

\end{document}
