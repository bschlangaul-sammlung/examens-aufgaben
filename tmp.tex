\documentclass{lehramt-informatik-aufgabe}
\usepackage{tikz}
\usetikzlibrary{shapes.multipart,positioning,fit}

\begin{document}

\def\pfeil#1#2{
  \draw[-latex] ([xshift=1mm]reihe.#1 north) -- ++(0,0.25) -| ([xshift=-1mm]reihe.#2 north);
}

\def\pfeilUnten#1#2{
  \draw[-latex] ([xshift=1mm]reihe.#1 south) -- ++(0,-0.25) -| ([xshift=-1mm]reihe.#2 south);
}

\def\markierung#1#2{\node[
  draw,
  very thick,
  fit=(reihe.#1) (reihe.#2),
  inner sep=0pt
] {};
}

\tikzset{li sortierung zahlenreihe/.style={
  draw,thin,
  font=\large,
  rectangle split horizontal,
  rectangle split,
}}

\tikz[
  rectangle split parts=5,
]{

  \node[li sortierung zahlenreihe] (reihe) {\nodepart{one} 2 \nodepart{two} 1 \nodepart{three} 3};
  \pfeil{one}{two}
  \pfeil{two}{three}
  \markierung{two split south}{three split north}
  \pfeilUnten{three}{one}
}

\end{document}
