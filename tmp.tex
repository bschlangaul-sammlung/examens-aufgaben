\documentclass{article}
\usepackage{hyperref}
\begin{document}

\input /etc/lehramt-informatik.config.tex

\ExplSyntaxOn

\clist_new:N \g_auf_schluessel_clist
\clist_set:Nn \g_auf_schluessel_clist {
  titel,
  thematik,
  fussnote,
  fussnote_seite,
  relativer_pfad,
  examen_nummer,
  examen_jahr,
  examen_monat,
  examen_thema_nr,
  examen_teilaufgabe_nr,
  examen_aufgabe_nr,
}

\clist_map_inline:Nn \g_auf_schluessel_clist {
  \tl_new:c {g_auf_#1_tl}
}

\keys_define:nn { aufgabentitel }
{
  Titel .tl_gset:N = \g_auf_titel_tl,
  Thematik .tl_gset:N = \g_auf_thematik_tl,
  Fussnote .tl_gset:N = \g_auf_fussnote_tl,
  FussnoteSeite .tl_gset:N = \g_auf_fussnote_seite_tl,
  RelativerPfad .tl_gset:N = \g_auf_relativer_pfad_tl,
  ExamenNummer .tl_gset:N = \g_auf_examen_nummer_tl,
  ExamenJahr .tl_gset:N = \g_auf_examen_jahr_tl,
  ExamenMonat .tl_gset:N = \g_auf_examen_monat_tl,
  ExamenThemaNr .tl_gset:N = \g_auf_examen_thema_nr_tl,
  ExamenTeilaufgabeNr .tl_gset:N = \g_auf_examen_teilaufgabe_nr_tl,
  ExamenAufgabeNr .tl_gset:N = \g_auf_examen_aufgabe_nr_tl,
}

\cs_gset:Npn \_setze_relativen_pfad: {
  \tl_if_empty:NTF \g_auf_relativer_pfad_tl
  {
    \bool_if:nTF
    {
      ! \tl_if_empty_p:N \g_auf_examen_nummer_tl &&
      ! \tl_if_empty_p:N \g_auf_examen_jahr_tl &&
      ! \tl_if_empty_p:N \g_auf_examen_monat_tl
    }
    {
      \tl_gset:Nn \g_auf_relativer_pfad_tl {
        Staatsexamen /
        \g_auf_examen_nummer_tl /
        \g_auf_examen_jahr_tl /
        \g_auf_examen_monat_tl /
        \tl_if_empty:NTF \g_auf_examen_thema_nr_tl {} { Thema - \g_auf_examen_thema_nr_tl / }
        \tl_if_empty:NTF \g_auf_examen_teilaufgabe_nr_tl {} { Teilaufgabe - \g_auf_examen_teilaufgabe_nr_tl / }
        \tl_if_empty:NTF \g_auf_examen_aufgabe_nr_tl {} { Aufgabe - \g_auf_examen_aufgabe_nr_tl .tex }
      }
    }
    {}
  }
  {}
}

\cs_gset:Npn \_gib_examen_titel: {
  \bool_if:nTF
  {
    ! \tl_if_empty_p:N \g_auf_examen_nummer_tl &&
    ! \tl_if_empty_p:N \g_auf_examen_jahr_tl &&
    ! \tl_if_empty_p:N \g_auf_examen_monat_tl &&
    ! \tl_if_empty_p:N \g_auf_examen_aufgabe_nr_tl
  }
  {
    \noindent
    \par
    Staatsexamen ~
    \g_auf_examen_nummer_tl ~ / ~
    \g_auf_examen_jahr_tl ~ / ~
    \g_auf_examen_monat_tl ~ / ~
    \tl_if_empty:NTF \g_auf_examen_thema_nr_tl {} { Thema ~ Nr. ~ \g_auf_examen_thema_nr_tl ~ / ~ }
    \tl_if_empty:NTF \g_auf_examen_teilaufgabe_nr_tl {} { Teilaufgabe ~ Nr. ~ \g_auf_examen_teilaufgabe_nr_tl ~ / ~ }
    \tl_if_empty:NTF \g_auf_examen_aufgabe_nr_tl {} { Aufgabe ~ Nr. ~ \g_auf_examen_aufgabe_nr_tl }
    \par
  }
}

% https://raw.githubusercontent.com/Josef-Friedrich/check_zpool_scrub/master/README.md
\cs_new:Npn \_gib_github_url: {
  \tl_if_empty:NTF \g_auf_relativer_pfad_tl {} {
    Github :~\href{
      \LehramtInformatikGithubRawDomain /
      \LehramtInformatikGithubTexRepo /
      \LehramtInformatikGitBranch /
      \g_auf_relativer_pfad_tl
    }{
      \g_auf_relativer_pfad_tl
    }
  }
}

\cs_new:Npn \_gib_aufgaben_titel: {
  \g_auf_titel_tl

  \tl_if_empty:NTF \g_auf_thematik_tl
  {}
  {
   \, ~ [
    \g_auf_thematik_tl
   ]
  }
}

\cs_new:Npn \_setze_variablen_zurueck: {
  \clist_map_inline:Nn \g_auf_schluessel_clist {
    \tl_clear:c {g_auf_##1_tl}
  }
}

\msg_new:nnn { aufgabentitel } { kein-titel }
  { Der~Schlüssel~ist~zwingend~notwendig }

\def\liSetzeAufgabenTitel#1{
  \_setze_variablen_zurueck:

  \tl_clear:N \g_auf_examen_teilaufgabe_nr_tl

  \keys_set:nn { aufgabentitel } {
    #1
  }

  \_setze_relativen_pfad:

  \tl_if_empty:NTF \g_auf_titel_tl
  {
    \msg_fatal:nn { aufgabentitel } { kein-titel }
  }
  {
  }

  \section{\_gib_aufgaben_titel:}

  \par
  \_gib_github_url:
}

\ExplSyntaxOff

\liSetzeAufgabenTitel{
  Titel = Titel,
  Thematik = Thematik,
  ExamenAufgabeNr = 3,
}

\liSetzeAufgabenTitel{
  Titel = Titel 1,
  ExamenTeilaufgabeNr = 4,
}

\liSetzeAufgabenTitel{
  Titel = Titel 1,
  ExamenNummer = 66116,
  ExamenJahr = 2021,
  ExamenMonat = 09,
  ExamenThemaNr = 1,
  ExamenTeilaufgabeNr = 2,
  ExamenAufgabeNr = 3,
}

\liSetzeAufgabenTitel{
  Titel = Lol,
  Thematik = Troll,
  RelativerPfad = SOSY/Aufgabe-1.tex
}

\end{document}
