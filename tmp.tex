\documentclass{lehramt-informatik-aufgabe}
\usepackage{tikz}
\usetikzlibrary{shapes.multipart,positioning,fit}

\begin{document}

\def\pfeil#1#2{
  \draw[-latex] ([xshift=1mm]reihe.#1 north) -- ++(0,0.25) -| ([xshift=-1mm]reihe.#2 north);
}

\def\pfeilUnten#1#2{
  \draw[-latex] ([xshift=1mm]reihe.#1 south) -- ++(0,-0.25) -| ([xshift=-1mm]reihe.#2 south);
}

\tikzset{li sortierung zahlenreihe/.style={
  draw,thin,
  font=\large,
  rectangle split horizontal,
  rectangle split,
}}

\begin{tikzpicture}[
  rectangle split parts=5,
]
\node [li sortierung zahlenreihe] (reihe) {\nodepart{one} 2 \nodepart{two} 1 \nodepart{three} 3};

\node[
  draw,
  very thick,
  fit=(reihe.two split south) (reihe.three split north),
  inner sep=0pt
] {};

% \node at (a.two split north){n};
% \node at (a.one split south){s};

\pfeil{one}{two}
\pfeil{two}{three}

\pfeilUnten{three}{one}
\end{tikzpicture}

\end{document}
