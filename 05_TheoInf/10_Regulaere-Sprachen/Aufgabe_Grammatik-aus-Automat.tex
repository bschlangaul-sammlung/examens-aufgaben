\documentclass{lehramt-informatik-aufgabe}
\liLadePakete{automaten}
\begin{document}
\liAufgabenTitel{Grammatik aus Automat}
\section{Grammatik aus Automat
\index{Reguläre Sprache}
\footcite{theoinf:ab:1}}

Sei $M = (\{q_0, q_1, q_2 \}, \{a, b\}, \sigma, q_0, \{q 2\})$ ein
endlicher Automat. Die Übergangsfunktion sei wie in dem unten
abgebildeten Diagramm definiert.

\begin{center}
\begin{tikzpicture}[->]
\node[state] (0) at (0,0) {$q_0$};
\node[state] (1) at (2,2) {$q_1$};
\node[state,accepting] (2) at (4,0) {$q_2$};

\path (0) edge[above] node{a} (2);
\path (2) edge[above,bend left] node{b} (0);
\path (0) edge[above] node{b} (1);
\path (1) edge[above] node{b} (2);
\path (2) edge[above,bend left] node{a} (1);
\path (1) edge[loop,above] node{a} (1);
\end{tikzpicture}
\end{center}

\begin{enumerate}
\item Gebe eine reguläre Grammatik $G$ an, sodass $L(G) = L(M)$ gilt.

\item ̈Überlege dir ein systematisches Verfahren, um einen
deterministischen endlichen Automaten in eine reguläre Grammatik
umzuwandeln.
\end{enumerate}

\end{document}
