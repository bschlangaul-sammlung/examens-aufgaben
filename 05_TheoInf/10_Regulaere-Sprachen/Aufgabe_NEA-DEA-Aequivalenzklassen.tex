\documentclass{lehramt-informatik-aufgabe}
\liLadePakete{}
\begin{document}
\liAufgabenTitel{NEA-DEA-Aequivalenzklassen}
\section{NEA-DEA-Äquivalenzklassen
\index{Reguläre Sprache}
\footcite{theoinf:ab:1}}

Gegeben ist der deterministische endliche Automat $A = (Q, \{0, 1\},
\delta, q_0 , F)$, wobei $Q = \{A, B, C, D, E\}$, $q_0 = A$, $F = {E}$ und

\begin{center}
\begin{tabular}{l||l|l}
δ & 0 & 1 \\\hline\hline
A & B & C \\\hline
B & E & C \\\hline
C & D & C \\\hline
D & E & A \\\hline
E & E & E \\\hline
\end{tabular}
\end{center}

\begin{enumerate}
\item Minimieren Sie den Automaten mit dem bekannten
Minimierungsalgorithmus. Dokumentieren Sie die Schritte geeignet.

\item Geben Sie einen regulären Ausdruck für die erkannte Sprache an.

\item Geben Sie die Äquivalenzklassen der Myhill-Nerode-Äquivalenz der
Sprache durch reguläre Ausdrücke an.
\end{enumerate}

\end{document}
