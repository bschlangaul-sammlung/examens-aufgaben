\documentclass{lehramt-informatik-haupt}

\begin{document}

%%%%%%%%%%%%%%%%%%%%%%%%%%%%%%%%%%%%%%%%%%%%%%%%%%%%%%%%%%%%%%%%%%%%%%%%
% Theorie-Teil
%%%%%%%%%%%%%%%%%%%%%%%%%%%%%%%%%%%%%%%%%%%%%%%%%%%%%%%%%%%%%%%%%%%%%%%%

\chapter{Reguläre Sprachen}

\section{Chomsky-Hierarchie}

Die Sprachen lassen sich folgendermaßen einteilen:

\begin{description}
\item[Typ 0] Phrasenstrukturgrammatik
\item[Typ 1] kontextsensitive Grammatik
\item[Typ 2] kontextfreie Grammatik
\item[Typ 3] reguläre Grammatik\footcite[Seite 14]{theoinf:fs:1}
\end{description}

\begin{itemize}
\item $L_3 = \{(ab)^n \, | \, n \in \mathbb{N}\}$ ist regulär (Typ 3 Sprache)

\item $L_2 = \{a^n b^n \, | \, n \in \mathbb{N}\}$ ist kontextfrei (Typ 2 Sprache), aber nicht regulär

\item $L_1 = \{a^n b^n c^n \, | \, n \in \mathbb{N}\}$ ist kontextsensitiv (Typ 1 Sprache), aber nicht kontextfrei

\item $L_0 = \{a^{(2^n)}\, | \, n \in \mathbb{N}\}$ ist eine Typ 0 Sprache\footcite[Seite 15]{theoinf:fs:1}

\end{itemize}

\literatur

\end{document}
