\documentclass{lehramt-informatik-aufgabe}
\liLadePakete{mathe}
\begin{document}
\liAufgabenTitel{Vorlesungsaufgaben}
\section{Vorlesungsaufgaben
\index{Reguläre Sprache}}

\section{Übung zu regulären Grammatiken\footcite[Seite 21]{theoinf:fs:1}}

Gegeben ist eine Sprache $L \subset \Sigma^*$ mit $\Sigma = \{a, b\}$. Zu
der Sprache $L$ gehören alle Wörter, die die Zeichenfolge $abba$
beinhalten.

\begin{enumerate}
\item Gib eine Grammatik an, die diese Sprache erzeugt.

% https://flaci.com/Gib2h94ar
\begin{liAntwort}
S $\rightarrow$ aS | aB | bS | bA

A $\rightarrow$ aB

B $\rightarrow$ bC

C $\rightarrow$ bD

D $\rightarrow$ aE

E $\rightarrow$ aE | bE | a | b | $\epsilon$
\end{liAntwort}

\item Gib eine Ableitung/Syntaxbaum zu deiner Grammatik für das Wort
\texttt{aabbab} an.

\begin{liAntwort}
S $\rightarrow$
aS $\rightarrow$
aaB $\rightarrow$
aabC $\rightarrow$
aabbD $\rightarrow$
aabbaE $\rightarrow$
aabbab
\end{liAntwort}
\end{enumerate}

%-----------------------------------------------------------------------
%
%-----------------------------------------------------------------------

\section{Übungen zu regulären Ausdrücken\footcite[Seite 21]{theoinf:fs:1}}

\begin{enumerate}
\item Gegeben ist eine Sprache $L \subset \Sigma^*$ mit $\Sigma =
\{a,b\}$. Zu der Sprache $L$ gehören alle Wörter, die die Zeichenfolge
\texttt{abba} beinhalten.

Gib einen regulären Ausdruck für diese Sprache an.

Gebe möglichst einfache reguläre Ausdrücke für die folgenden
Sprachen $L \subset \Sigma^*$ mit $\Sigma = \{a,b\}$ und x ∈ {1,2,3}.

\begin{description}
\item[$L_1$] = $\{ x | x \text{ beinhaltet eine gerade Anzahl von } a \}$

\item[$L_2$] = $\{ x | x \text{ beinhaltet eine ungerade Anzahl von } b \}$

\item[$L_3$] = $\{ x | x \text{ beinhaltet an seinen geradzahligen Positionen ausschließlich } a \}$
\end{description}

\item Gib einen regulären Ausdruck der eine syntaktisch gültige
E-Mail-Adresse erkennt. (mindestens 1 Zeichen (Groß-/Kleinbuchstabe oder
Zahl) vor dem @; mindestens 1 Zeichen (Groß-/Kleinbuchstabe oder Zahl)
nach dem @; alle E-Mail-Adressen sollen auf .de oder .com enden
\end{enumerate}

\end{document}
