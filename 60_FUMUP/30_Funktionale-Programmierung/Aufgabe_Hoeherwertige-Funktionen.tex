\documentclass{lehramt-informatik-aufgabe}
\liLadePakete{syntax}
\begin{document}
\liAufgabenTitel{Höherwertige Funktionen}
\section{Höherwertige Funktionen
\index{Funktionale Programmierung mit Haskell}
\footcite{fumup:ab:3}}

Implementiere in der Datei switch.hs. Die Funktion
\liHaskellCode{switch} bekommt eine einwertige Entscheidungsfunktion
\liHaskellCode{s}, sowie ein Feld aus einwertigen Funktionen übergeben.
Ihr Ergebnis ist eine einwertige Funktion (mit Parameter
\liHaskellCode{x}). Diese Funktion interpretiert den Rückgabewert der
Entscheidungsfunktion (\liHaskellCode{s x}) als Index, mit dem eine
Funktion aus dem Feld auswählt wird. Diese gewählte Funktion wird dann
für \liHaskellCode{x} auswertet. Wenn der ermittelte Funktionsindex
nicht im Feld liegt, soll \liHaskellCode{x} unverändert zurückgegeben
werden.

\begin{liAntwort}
\liHaskellDatei{switch.hs}
\end{liAntwort}

\end{document}
