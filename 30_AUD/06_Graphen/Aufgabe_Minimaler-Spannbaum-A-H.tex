\documentclass{lehramt-informatik-aufgabe}
\liLadePakete{graph}
\begin{document}
\liAufgabenTitel{Minimaler Spannbaum A-H}

\section{Spannbaum
\index{Minimaler Spannbaum}
\footcite[Aufgabe 6]{aud:e-klausur}}

Ermitteln Sie einen minimalen Spannbaum des vorliegenden Graphen. Nutzen
Sie den \emph{Knoten A als Startknoten} in ihrem Algorithmus.

\graph knoten {
  \knoten{A}(0,5)
  \knoten{B}(8,5)
  \knoten{C}(4,4)
  \knoten{D}(2,3)
  \knoten{E}(6,3)
  \knoten{F}(4,1)
  \knoten{G}(0,0)
  \knoten{H}(8,0)
} kanten {
  \kante(A-B){5}
  \kante(A-C){8}
  \kante(A-D){7}
  \kante(A-G){1}
  \kante(B-C){2}
  \kante(B-E){7}
  \kante(B-H){2}
  \kante(C-D){1}
  \kante(C-E){3}
  \kante(C-F){6}
  \kante(D-F){2}
  \kante(D-G){8}
  \kante(E-H){6}
  \kante(F-E){5}
  \kante(F-H){3}
  \kante(G-F){5}
  \kante(G-H){4}
}

\begin{enumerate}
\item Welches Gewicht hat der Spannbaum insgesamt?

\begin{liAntwort}
\graph knoten {
  \knoten{A}(0,5)
  \knoten{B}(8,5)
  \knoten{C}(4,4)
  \knoten{D}(2,3)
  \knoten{E}(6,3)
  \knoten{F}(4,1)
  \knoten{G}(0,0)
  \knoten{H}(8,0)
} kanten {
  \kante(A-B){5}
  \kante(A-C){8}
  \kante(A-D){7}
  \KANTE(A-G){1}
  \KANTE(B-C){2}
  \kante(B-E){7}
  \KANTE(B-H){2}
  \KANTE(C-D){1}
  \KANTE(C-E){3}
  \kante(C-F){6}
  \KANTE(D-F){2}
  \kante(D-G){8}
  \kante(E-H){6}
  \kante(F-E){5}
  \kante(F-H){3}
  \kante(G-F){5}
  \KANTE(G-H){4}
}

\begin{center}
\begin{tabular}{|l|l|r|}
\hline
Kante & & Gewicht\\\hline\hline
AG, CD     & $2 \times 1$ & $2$\\
BD, BH, DF & $3 \times 2$ & $6$\\
CE         & $1 \times 3$ & $3$\\
GH         & $1 \times 4$ & $4$\\\hline
           &              & $15$\\\hline
\end{tabular}
\end{center}
\end{liAntwort}

\item Welchen Algorithmus haben Sie zur Ermittlung eingesetzt?
\index{Algorithmus von Kruskal}
\begin{liAntwort}
Kruskal
\end{liAntwort}
\end{enumerate}
\end{document}
