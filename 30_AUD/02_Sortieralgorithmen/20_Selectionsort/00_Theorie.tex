\documentclass{lehramt-informatik-haupt}
\liLadePakete{syntax,mathe}

\begin{document}

%%%%%%%%%%%%%%%%%%%%%%%%%%%%%%%%%%%%%%%%%%%%%%%%%%%%%%%%%%%%%%%%%%%%%%%%
% Theorie-Teil
%%%%%%%%%%%%%%%%%%%%%%%%%%%%%%%%%%%%%%%%%%%%%%%%%%%%%%%%%%%%%%%%%%%%%%%%

\chapter{Selectionsort: Sortieren durch Auswählen\footcite[Seite 39]{aud:fs:tafeluebung-11}}

\begin{liQuellen}
\item \cite[Seite 39]{aud:fs:tafeluebung-11}
\item \cite{wiki:selectionsort}
\item \cite[Seite 127-129 (PDF 145-147)]{saake}
\item \cite[6.4.1 Naive Sortierverfahren, Seite 191]{schneider}
\end{liQuellen}

\begin{itemize}
\item Funktionsweise

\begin{itemize}
\item solange zu sortierende Liste mehr als ein Element beinhaltet:

\begin{itemize}
\item lösche das \emph{Maximum / Minimum} aus der Liste
\item füge es ans Ende der Ergebnisliste
\item wiederhole, bis Eingangsliste leer
\end{itemize}

\end{itemize}

\item \emph{Eigenschaften} von Selectionsort:

\begin{itemize}
\item Laufzeitkomplexität:
$\mathcal{O}(n^2)$ (im \emph{Best}-, \emph{Average}- und
\emph{Worst-Case})

\item Stabilität leicht erreichbar
\item bei Zahlen \emph{in-situ}
\end{itemize}

\end{itemize}

%%
%
%%

\section{Iterativ}

\liJavaDatei[firstline=5,lastline=29]{sortier/SelectionSort}
\footcite[Seite 128 (PDF 146)]{saake}

\section{Halb Rekursiv}

\liJavaDatei[firstline=31,lastline=52]{sortier/SelectionSort}

\section{Rekursiv}

\liJavaDatei[firstline=54,lastline=77]{sortier/SelectionSort}

\literatur

\end{document}
