\documentclass{lehramt-informatik-haupt}
\liLadePakete{mathe,master-theorem}

\begin{document}
\let\O=\liO
\let\o=\liOmega
\let\T=\liT
\let\t=\liTheta

%%%%%%%%%%%%%%%%%%%%%%%%%%%%%%%%%%%%%%%%%%%%%%%%%%%%%%%%%%%%%%%%%%%%%%%%
% Theorie-Teil
%%%%%%%%%%%%%%%%%%%%%%%%%%%%%%%%%%%%%%%%%%%%%%%%%%%%%%%%%%%%%%%%%%%%%%%%

\section{Master-Theorem}

Der Hauptsatz der Laufzeitfunktionen – oder oft auch aus dem Englischen
als Master-Theorem entlehnt – bietet eine \memph{schnelle Lösung} für
die Frage, \memph{in welcher Laufzeitklasse} eine gegebene
\memph{rekursiv definierte Funktion} liegt. Mit dem Master-Theorem kann
allerdings \memph{nicht jede rekursiv definierte Funktion} gelöst
werden. Lässt sich keiner der \memph{drei möglichen Fälle} des
Master-Theorems auf die Funktion $T$ anwenden, so muss man die
Komplexitätsklasse der Funktion anderweitig berechnen.

\begin{displaymath}
T(n) = \T{a}{b} + f(n)
\end{displaymath}

\begin{itemize}
\item[$a =$]
Anzahl der Unterprobleme in der Rekursion

\item[$\textstyle{\frac{1}{b}} =$]
Teil des Originalproblems, welches wiederum durch alle Unterprobleme
repräsentiert wird

\item[$f(n) =$]
Kosten (Aufwand, Nebenkosten), die durch die Division des Problems und
die Kombination der Teillösungen entstehen
\end{itemize}
\footcite{wiki:master-theorem}
\footcite[Seite 19-35 (PDF 11-24)]{aud:fs:2}

\begin{description}
\item[1. Fall:]
$T(n) \in \t{n^{\log_{b}a}}$

\hfill falls \liBedingungEins
für $\varepsilon > 0$

\item[2. Fall:]
$T(n) \in \t{n^{\log_{b}a} \cdot \log n}$

\hfill falls \liBedingungZwei

\item[3. Fall:]
$T(n) \in \t{f(n)}$

\hfill falls \liBedingungDrei
für $\varepsilon > 0$
und ebenfalls für ein $c$ mit $0 < c < 1$ und alle hinreichend großen $n$
gilt:
$a \cdot f(\textstyle {\frac {n}{b}})\leq c \cdot f(n)$
\end{description}

\literatur
\end{document}
