\documentclass{lehramt-informatik-haupt}
\liLadePakete{baum,syntax,spalten}

\begin{document}

%%%%%%%%%%%%%%%%%%%%%%%%%%%%%%%%%%%%%%%%%%%%%%%%%%%%%%%%%%%%%%%%%%%%%%%%
% Theorie-Teil
%%%%%%%%%%%%%%%%%%%%%%%%%%%%%%%%%%%%%%%%%%%%%%%%%%%%%%%%%%%%%%%%%%%%%%%%

% AUD_5_Teil1.m4v 43min
\chapter{B-Bäume}

\begin{liQuellen}
\item \cite{wiki:bbaum}
\item \cite[Kapitel 14.4.3, Seite 386-399 (PDF 402-415)]{saake}
\item \cite[Kapitel 13.5.4.2 Balancierte Mehrwegbäume, Seite 464, wird
nur erwähnt, nicht beschrieben]{schneider}
\item \cite[7.8 B-Bäume Seite 224-228]{kemper}
\end{liQuellen}

\url{https://www.cs.usfca.edu/~galles/visualization/BTree.html}
Max. Degree = 5 entspricht k = 2

\noindent
Eine ausgeglichene Baumstruktur ist der von R. Bayer und E. McCreight
entwickelte B-Baum. Hierbei \memph{steht der Name „B“ für balanciert},
breit, buschig oder auch Bayer, nicht jedoch für binär. Die Grundidee
des B-Baumes ist es gerade, dass der Verzweigungsgrad variiert, während
die Baumhöhe vollständig ausgeglichen ist.
\footcite[Seite 386]{saake}

%-----------------------------------------------------------------------
%
%-----------------------------------------------------------------------

\section{Definition}

Ein Baum heißt genau dann B-Baum, wenn gilt:
%

\begin{enumerate}
\item Jeder Knoten außer der Wurzel enthält zwischen \memph{$k$ und $2k$
Elemente} (Schlüsselwerte), $k$ wird als \memph{Ordnung} des B-Baums
bezeichnet.
%
\item Jeder \memph{Knoten ist entweder ein Blatt (ohne Kinder)} oder hat
\memph{mindestens $k + 1$ und höchstens $2k + 1$ Kind-Knoten}.
%
\item Der Wurzelknoten ist \memph{entweder ein Blatt oder hat mindestens
2 Nachfolger}.
%
\item Alle Blätter haben die \memph{gleiche Tiefe}, \dh alle Wege von
der Wurzel bis zu den Blättern sind gleich lang. Pfade haben die Länge
$h - 1$, wobei $h$ die Höhe des gesamten Baums ist.
\end{enumerate}
\footcite[Seite 32]{aud:fs:5}

%-----------------------------------------------------------------------
%
%-----------------------------------------------------------------------

\section{Einfügen}

Das Einfügen in einen B-Baum erfolgt \memph{nur in den Blattknoten}.
Wenn in einem Blattknoten die \memph{maximale Anzahl} von Elementen
($2k$) erreicht ist, findet ein \memph{Split} statt, \dh die
Elemente werden aufgeteilt und ein neuer Knoten entsteht. Das
\memph{mittlere Element} des ursprünglichen Knotens wird dabei \memph{in
den Elternknoten integriert}.
\footcite[Seite 32]{aud:fs:5}

%-----------------------------------------------------------------------
%
%-----------------------------------------------------------------------

\section{Suchen}

Beginnend mit dem Wurzelknoten werden die Knoten jeweils \memph{von
links nach rechts} durchsucht:
%
\memph{Stimmt} ein Element mit dem gesuchtem Schlüsselwert
\memph{überein}, ist der Satz \memph{gefunden}.
%
Ist das \memph{Element größer} als der gesuchte Wert, wird die
Suche im \memph{links} hängenden Unterbaum \memph{fortgesetzt}.
%
Ist das \memph{Element kleiner} als der gesuchte Wert, wird der
Vergleich mit dem \memph{nächsten Element der Wurzel wiederholt}.
%
Ist auch das letztes Element der Wurzel noch kleiner als der gesuchte
Wert, dann wird die Suche im \memph{rechten Unterbaum} des Elements
fortgesetzt.
%
Falls ein weiterer Abstieg in den Unterbaum nicht möglich ist
(\dh Blattknoten), wird die Suche abgebrechen. Dann ist kein Satz mit
dem gewünschten Schlüsselwert vorhanden.
\footcite[Seite 37]{aud:fs:5}

%-----------------------------------------------------------------------
%
%-----------------------------------------------------------------------

% AUD_5_Teil1.m4v 55min
\section{Löschen}

Suche den Knoten, in dem das zu löschende Element $E$ liegt.

\begin{itemize}
\item Falls das Element $E$ im Blattknoten liegt, dann lösche $E$ dort
und behandle ggf. entstehenden Unterlauf durch Mischen.

\item Falls das Element $E$ in einem inneren Knoten liegt, dann
untersuche den linken und rechten Unterbaum von $E$:

\begin{itemize}
\item Betrachte den Blattknoten mit dem direkten Vorgänger $E'$ von $E$
und den Blattknoten mit direktem Nachfolger $E''$ von $E$.

\item Wähle den Blattknoten aus, der mehr Elemente hat. Falls beide
Blattknoten gleich viele Elemente haben, wähle zufällig einen der beiden
aus.

\item Ersetze das zu löschende Element $E$ durch $E'$ bzw. $E''$ aus dem
gewählten Blattknoten.

\item Lösche $E'$ bzw. $E''$ im gewählten Blattknoten und behandle ggf.
entstehenden Unterlauf in diesem Blattknoten.
\end{itemize}
\end{itemize}
\footcite[Seite 39]{aud:fs:5}

\subsection{Hinweise zum Unterlauf:}

Ein Unterlauf entsteht auf Blattebene. Der Unterlauf wird durch Mischen
des Unterlaufknotens mit seinem Nachbarknoten und dem darüberliegenden
Elemen  durchgeführt, dabei wird sozusagen ein Splitt rückwärts
durchgeführt. Wurde einmal mit dem Mischen auf Blattebene begonnen, muss
das evtl. nach oben fortgesetzt werden. Mischen wird so lange
fortgesetzt, bis kein Unterlauf mehr existiert oder die Wurzel erreicht
ist. Wird die Wurzel erreicht, kann der Baum in der Höhe um 1
schrumpfen. Beim Mischen kann auch wieder ein Überlauf entstehen, d.h.
es muss wieder gesplittet werde.
\footcite[Seite 40]{aud:fs:5}

\liJavaDatei{baum/BBaum}

\literatur

\end{document}
