\documentclass{lehramt-informatik-aufgabe}
\liLadePakete{baum}
\begin{document}
\liAufgabenTitel{Einfügen und Löschen in B-Bäumen}

% https://vsis-www.informatik.uni-hamburg.de/oldServer/teaching/ws-17.18/gdb/aufgaben/Aufgabenblatt06Lsg.pdf

\section{Einfügen und Löschen in B-Bäumen
\index{B-Baum}
}
\begin{enumerate}

\item Gegeben ist der unten vereinfacht dargestellte B-Baum der Klasse τ (2,
h) . Fügen Sie die (Datensätze mit den) Schlüsselwerte(n) 67, 57, 61,
75, 5, 13, 2, 91, 9, 17, 10 und 8 ein. Geben Sie in jedem Einfügeschritt
die verwendete Maÿnahme (einfaches Einfügen in einen Knoten, Splitten)
an und zeichnen Sie den Baum nach jedem Knotensplit neu. Als Splitfaktor
wird dabei m = 1 gewählt.

\begin{tikzpicture}[
  li bbaum,
  level 1/.style={level distance=15mm,sibling distance=20mm},
  level 2/.style={level distance=10mm,sibling distance=20mm},
]
\node {15 \nodepart{two} 63} [->]
  child {
    node {7 \nodepart{two} 11}
  }
  child {
    node {21 \nodepart{two} 33 \nodepart{three} 54}
  }
  child {
    node {71 \nodepart{two} 78}
  };
\end{tikzpicture}

\begin{liAntwort}

%%
%
%%

Schlüsselwert 67 (einfaches Einfügen)
Schlüsselwert 57 (einfaches Einfügen)
Schlüsselwert 61 (Splitten)

\begin{tikzpicture}[
  li bbaum,
  level 1/.style={level distance=15mm,sibling distance=22mm},
  level 2/.style={level distance=10mm,sibling distance=20mm},
]
\node {15 \nodepart{two} 54 \nodepart{three} 63} [->]
  child {
    node {7 \nodepart{two} 11}
  }
  child {
    node {21 \nodepart{two} 33}
  }
  child {
    node {57  \nodepart{two} 61}
  }
  child {
    node {67 \nodepart{two} 71 \nodepart{three} 78}
  };
\end{tikzpicture}

%%
%
%%

Schlüsselwert 75 (einfaches Einfügen)
Schlüsselwert 5 (einfaches Einfügen)
Schlüsselwert 13 (einfaches Einfügen)
Schlüsselwert 2 (Splitten)

\begin{tikzpicture}[
  li bbaum,
  level 1/.style={level distance=15mm,sibling distance=22mm},
  level 2/.style={level distance=10mm,sibling distance=20mm},
]
\node {7 \nodepart{two} 15 \nodepart{three} 54 \nodepart{four} 63} [->]
  child {
    node {2 \nodepart{two} 5}
  }
  child {
    node {11 \nodepart{two} 13}
  }
  child {
    node {21 \nodepart{two} 33}
  }
  child {
    node {57  \nodepart{two} 61}
  }
  child {
    node {67 \nodepart{two} 71 \nodepart{three} 75 \nodepart{four} 78}
  };
\end{tikzpicture}
\end{liAntwort}

\item Gegeben ist der unten dargestellte B-Baum der Klasse τ (2, h) .
Löschen Sie die (Datensätze mit den) Schlüsselwerte(n) 24, 51, 87, 34,
71, 19, 31 und 8 . Geben Sie in jedem Löschschritt die verwendete
Maÿnahme (einfaches Löschen, Mischen, Ausgleichen) an und zeichnen Sie
den Baum nach jeder Veränderung der Knotenstruktur (Mischen,
Ausgleichen) neu. Für Ausgleichsoperationen sollen nur unmittelbare
Nachbarknoten herangezogen werden.

\begin{tikzpicture}[
  li bbaum,
  level 1/.style={level distance=15mm,sibling distance=64mm},
  level 2/.style={level distance=10mm,sibling distance=20mm},
  level 3/.style={level distance=10mm,sibling distance=20mm},
]
\node {36} [->]
  child {
    node {13 \nodepart{two} 22}
    child {
      node {1 \nodepart{two} 8}
    }
    child {
      node {15 \nodepart{two} 19}
    }
    child {
      node {24 \nodepart{two} 31 \nodepart{three} 34}
    }
  }
  child {
    node {61 \nodepart{two} 82}
    child {
      node {37 \nodepart{two} 51 \nodepart{three} 54 \nodepart{four} 59}
    }
    child {
      node {63 \nodepart{two} 71}
    }
    child {
      node {87 \nodepart{two} 89}
    }
  };
\end{tikzpicture}

\end{enumerate}

\end{document}
