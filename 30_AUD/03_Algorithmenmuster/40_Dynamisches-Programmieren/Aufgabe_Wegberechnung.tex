\documentclass{lehramt-informatik-aufgabe}
\liLadePakete{mathe,syntax}
\begin{document}
\liAufgabenTitel{Wegberechnung im Gitter}

\section{Wegberechnung im Gitter
\footcite[Seite 1, Aufgabe 2: Dynamische Programmierung]{aud:ab:3}}

Betrachten Sie das folgende Gitter mit $m + 1$ Zeilen und $n + 1$
Spalten ($m \geq 1$ und $n \geq 1$):
\footnote{Quelle möglicherweise von \url{https://www.yumpu.com/de/document/read/17936760/ubungen-zum-prasenzmodul-algorithmen-und-datenstrukturen}}
geeksforgeeks
\footnote{\url{https://www.geeksforgeeks.org/count-possible-paths-top-left-bottom-right-nxm-matrix/}}

Angenommen, Sie befinden sich zu Beginn am Punkt $(0, 0)$ und wollen zum
Punkt $(m, n)$.

Für die Anzahl $A(i, j)$ aller verschiedenen Wege vom Punkt $(0, 0)$ zum
Punkt $(i, j)$ lassen sich folgende drei Fälle unterscheiden (es geht
jeweils um die kürzesten Wege
ohne Umweg!):

\begin{itemize}
\item $1 \leq i \leq m$ und $j = 0$:\\\\
Es gibt genau einen Weg von $(0, 0)$ nach $(i, 0)$ für
$1 \leq i \leq m$.

\item $i = 0$ und $1 \leq j \leq n$:\\\\
Es gibt genau einen Weg von $(0, 0)$ nach $(0, j)$ für
$1 \leq j \leq n$.

\item $1 \leq i \leq m$ und $1 \leq j \leq n$:\\\\
auf dem Weg zu $(i, j)$
muss als vorletzter Punkt entweder $(i-1, j)$ oder $(i, j-1)$ besucht
worden sein.
\end{itemize}

\noindent
Daraus ergibt sich folgende Rekursionsgleichung:

\begin{equation*}
A(i, j) =
\begin{cases}
1 &
\text{falls }
(1 \leq i \leq m \text{ und } j = 0) \text{ oder }
(i = 0 \text{ und } 1 \leq j \leq n) \\

A(i - 1, j) + A(i, j - 1) &
\text{falls }
1 \leq i \leq m \text{ und }
1 \leq j \leq n \\
\end{cases}
\end{equation*}

\noindent
Implementieren Sie die Java-Klasse \liJavaCode{Gitter} mit der Methode

\begin{center}
\liJavaCode{public int berechneAnzahlWege()},
\end{center}

\noindent
die ausgehend von der Rekursionsgleichung durch dynamische
Programmierung die Anzahl aller Wege vom Punkt $(0, 0)$ zum Punkt $(m,
n)$ berechnet.
Die Überprüfung, ob $m \leq 1$ und $n \leq 1$ gilt, können Sie der
Einfachheit halber weglassen.

\begin{liAntwort}
\liJavaDatei[firstline=32,lastline=46]{aufgaben/aud/ab_3/Gitter}
\end{liAntwort}
\end{document}
