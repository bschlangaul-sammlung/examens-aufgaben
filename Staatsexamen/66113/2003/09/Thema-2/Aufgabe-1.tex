\documentclass{lehramt-informatik-aufgabe}
\liLadePakete{}
\begin{document}
\liAufgabenTitel{Nachteile vollständige Normalisierung}

\section{Aufgabe 1
\index{Normalformen}
\footcite[Thema 2 Aufgabe 1]{66113:2003:09}}

Erläutern Sie, inwiefern sich eine vollständige Normalisierung
nachteilig auf die Geschwindigkeit der Anfragebearbeitung auswirken kann
und wie darauf reagiert werden kann!
\footcite[Aufgabe 10: Nachteile der Normalisierung]{db:ab:klausurvorbereitung}

\begin{antwort}
Durch die vielen Tabellen sind schon bei einfacheren Anfragen schnell
Joins notwendig, was bei komplexeren Anfragen und großen Datenmengen zu
einigem Rechenaufwand führen kann. Hier ist es sinnvoll, zuerst eine
Selektion zu treffen, anstatt in einem einfachen Kreuzprodukt
auch sämtliche sinnlose Tupel miteinander zu verknüpfen.
\end{antwort}

\end{document}
