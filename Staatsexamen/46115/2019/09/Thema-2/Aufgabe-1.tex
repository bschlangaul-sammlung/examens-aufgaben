\documentclass{lehramt-informatik-aufgabe}
\liLadePakete{formale-sprachen,automaten}
\begin{document}
\let\m=\liMenge

\liAufgabenTitel{Komplemetieren eines NEA}
\section{Aufgabe 1 (Komplemetieren eines NEA)
\index{Potenzmengenalgorithmus}
\footcite{46115:2019:09}}

Es sei der nichtdeterministische endliche Automat $A = (\m{a, b},\, \m{1,
2, 3, 4, 5},\, \delta,\, 1,\, \m{4, 5})$ gegeben, wobei $\delta$ durch
folgenden Zeichnung beschrieben ist.
\footcite[Seite 17, Aufgabe 11]{theo:ab:1}
\index{Nichtdeterministisch endlicher Automat (NEA)}

\begin{liAntwort}
\begin{center}
\begin{tikzpicture}[->,node distance=2cm]
\node[state,initial] (1) {1};
\node[state,above right of=1] (2) {2};
\node[state,below right of=1] (3) {3};
\node[state,right of=2] (4) {4};
\node[state,right of=3] (5) {5};

\path (1) edge[above] node{a} (2);
\end{tikzpicture}
\end{center}
\end{liAntwort}

Konstruieren Sie nachvollziehbar einen deterministischen endlichen
Automaten $A'$ , der das Komplement von $L(A)$ akzeptiert!
\index{Deterministisch endlicher Automat (DEA)}
\end{document}
