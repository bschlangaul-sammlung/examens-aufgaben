\documentclass{lehramt-informatik-aufgabe}
\liLadePakete{syntax}
\begin{document}

\let\j=\liJavaCode
\liAufgabenTitel{Mystery-Stacks}
\section{Aufgabe 6 (Stacks)
\index{Stapel (Stack)}
\footcite{46115:2019:09}}

Gegeben sei die Implementierung eines Stacks ganzer Zahlen mit folgender
Schnittstelle:

\liJavaDatei[firstline=3]{examen/examen_46115/jahr_2019/herbst/mystery_stack/IntStack}

Betrachten Sie nun die Realisierung der folgenden Datenstruktur
\j{Mystery}, die zwei Stacks benutzt.

\liJavaDatei[firstline=3,lastline=18]{examen/examen_46115/jahr_2019/herbst/mystery_stack/Mystery}

\begin{enumerate}

%%
% (a)
%%

\item Skizzieren Sie nach jedem Methodenaufruf der im folgenden
angegebenen Befehlssequenz den Zustand der beiden Stacks eines Objekts
\j{m} der Klasse \j{Mystery}. Geben Sie zudem bei jedem Aufruf der
Methode \j{bar} an, welchen Wert diese zurückliefert.

\liJavaDatei[firstline=21,lastline=30]{examen/examen_46115/jahr_2019/herbst/mystery_stack/Mystery}

%%
% (b)
%%

\item Sei $n$ die Anzahl der in einem Objekt der Klasse \j{Mystery}
gespeicherten Werte. Im folgenden wird gefragt, wieviele Aufrufe von
Operationen der Klasse \j{IntStack} einzelne Aufrufe von Methoden der
Klasse \j{Mystery} verursachen. Begründen Sie jeweils Ihre Antwort.

\begin{enumerate}

%%
% (i)
%%

\item Wie viele Aufrufe von Operationen der Klasse \j{IntStack}
verursacht die Methode \j{foo(x)} im besten Fall?

%%
% (ii)
%%

\item Wie viele Aufrufe von Operationen der Klasse \j{IntStack}
verursacht die Methode \j{foo(x)} im schlechtesten Fall?

%%
% (ii)
%%

\item Wie viele Aufrufe von Operationen der Klasse \j{IntStack}
verursacht die Methode \j{bar()} im besten Fall?

%%
% (iv)
%%

\item Wie viele Aufrufe von Operationen der Klasse \j{IntStack}
verursacht die Methode \j{bar()} im schlechtesten Fall?
\end{enumerate}

%%
% (c)
%%

\item Welche allgemeinen Eigenschaften werden durch die Methoden \j{foo}
und \j{bar} realisiert? Unter welchem Namen ist diese Datenstruktur
allgemein bekannt?
\end{enumerate}

\end{document}
