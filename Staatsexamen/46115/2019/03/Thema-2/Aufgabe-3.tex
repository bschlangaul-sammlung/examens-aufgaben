\documentclass{bschlangaul-aufgabe}

\begin{document}
\bAufgabenMetadaten{
  Titel = {Aufgabe 3},
  Thematik = {},
  RelativerPfad = Staatsexamen/46115/2019/03/Thema-2/Aufgabe-3.tex,
  ZitatSchluessel = examen:46115:2019:03,
  BearbeitungsStand = unbekannt,
  Korrektheit = unbekannt,
  Stichwoerter = {AVL-Baum},
  ExamenNummer = 46115,
  ExamenJahr = 2019,
  ExamenMonat = 03,
  ExamenThemaNr = 2,
  ExamenAufgabeNr = 3,
}
\begin{enumerate}

%%
% (a)
%%

\item Zeigen oder widerlegen Sie die folgende Aussage: Wird ein Element
in einen AVL-Baum eingefügt und unmittelbar danach wieder gelöscht, so
befindet sich der AVL-Baum wieder in seinem
Ursprungszustand.\index{AVL-Baum}
\footcite{examen:46115:2019:03}

%%
% (b)
%%

\item Fügen Sie in den gegebenen Baum den Schlüssel 11 ein.

Rebalancieren Sie anschließend den Baum so, dass die AVL-Eigenschaft
wieder erreicht wird. Zeichnen Sie den Baum nach jeder Einfach- und
Doppelrotation und benennen Sie die Art der Rotation (Links-, Rechts-,
Links-Rechts-, oder Rechts-Links-Rotation). Argumentieren Sie jeweils
über die Höhenbalancen der Teilbäume.

Tipp: Zeichnen Sie nach jedem Schritt die Höhenbalancen in den Baum ein.
\end{enumerate}
\end{document}
