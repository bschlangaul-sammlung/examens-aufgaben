\documentclass{bschlangaul-aufgabe}
\liLadePakete{syntax}
\begin{document}
\liAufgabenTitel{händisch sortieren, implementieren, Komplexität}
\section{Aufgabe 4
\index{Bubblesort}
\footcite{examen:46115:2017:03}}

Bei Bubblesort wird eine unsortierte Folge von Elementen $a_1,
a_2,\dots, a_n$, von links nach rechts durchlaufen, wobei zwei
benachbarte Elemente $a_i$ und $a_{i + 1}$ getauscht werden, falls sie
nicht in der richtigen Reihenfolge stehen. Dies wird so lange
wiederholt, bis die Folge sortiert ist.

\begin{enumerate}
%%
% a)
%%

\item Sortieren Sie die folgende Zahlenfolge mit Bubblesort. Geben Sie
die neue Zahlenfolge nach jedem (Tausch-)Schritt an: $3$, $2$, $4$, $1$

\begin{liAntwort}
\begin{verbatim}
 3  2  4  1  Eingabe
 3  2  4  1  Durchlauf Nr. 1
>3  2< 4  1  vertausche (i 0<>1)
 2  3 >4  1< vertausche (i 2<>3)
 2  3  1  4  Durchlauf Nr. 2
 2 >3  1< 4  vertausche (i 1<>2)
 2  1  3  4  Durchlauf Nr. 3
>2  1< 3  4  vertausche (i 0<>1)
 1  2  3  4  Durchlauf Nr. 4
 1  2  3  4  Ausgabe
\end{verbatim}
\end{liAntwort}

%%
% b)
%%

\item Geben Sie den Bubblesort-Algorithmus für ein Array von natürlichen
Zahlen in einer Programmiersprache Ihrer Wahl an. Die Funktion
\liJavaCode{swap (index1, index2)} kann verwendet werden, um zwei
Elemente des Arrays zu vertauschen.

\begin{liAntwort}
\liJavaDatei{examen/examen_46115/jahr_2017/fruehjahr/BubbleSort}

\liPseudoUeberschrift{Test}

\liJavaTestDatei{examen/examen_46115/jahr_2017/fruehjahr/BubbleSortTest}
\end{liAntwort}

%%
% c)
%%

\item Geben Sie eine obere Schranke für die Laufzeit an. Beschreiben Sie
mögliche Eingabedaten, mit denen diese Schranke erreicht wird.

\begin{liAntwort}
$\mathcal{O}(n^2)$

Diese obere Schranke wird erreicht, wenn die Zahlenfolgen in der
umgekehrten Reihenfolge bereits sortiert ist, \zB 4, 3, 2, 1.
\end{liAntwort}

\end{enumerate}
\end{document}
