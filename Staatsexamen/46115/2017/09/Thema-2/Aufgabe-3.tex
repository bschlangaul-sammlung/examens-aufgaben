\documentclass{lehramt-informatik-aufgabe}
\liLadePakete{syntax}
\begin{document}
\liAufgabenTitel{Primzahl}
\section{Aufgabe 3
\index{Dynamische Programmierung}
\footcite{examen:46115:2017:09}}

Die Methode \liJavaCode{pKR} berechnet die $n$-te Primzahl ($n \geq 1$)
kaskadenartig rekursiv und äußerst ineffizient:

\liJavaExamen[firstline=32,lastline=44]{46115}{2017}{09}{PrimzahlDP}

\noindent
Überführen Sie \liJavaCode{pKR} mittels \emph{dynamischer
Programmierung} (hier also \emph{Memoization}) und mit möglichst
\emph{wenigen Änderungen} so in die \emph{linear} rekursive Methode
\liJavaCode{pLR}, dass \liJavaCode{pLR(n, new long[n + 1])} ebenfalls
die $n$-te Primzahl ermittelt:

\begin{minted}{java}
private long pLR(int n, long[] ps) {
  ps[1] = 2;
  // ...
}
\end{minted}

\begin{liAntwort}
\begin{liExkurs}[Kaskadenartig rekursiv]
Kaskadenförmige Rekursion bezeichnet den Fall, in dem mehrere rekursive
Aufrufe nebeneinander stehen.
\end{liExkurs}

\begin{liExkurs}[Linear rekursiv]
Die häufigste Rekursionsform ist die lineare Rekursion, bei der in jedem
Fall der rekursiven Definition höchstens ein rekursiver Aufruf vorkommen
darf.
\end{liExkurs}

\liJavaExamen[firstline=55,lastline=74]{46115}{2017}{09}{PrimzahlDP}

\liPseudoUeberschrift{Der komplette Quellcode}

\liJavaExamen{46115}{2017}{09}{PrimzahlDP}
\end{liAntwort}

\end{document}
