\documentclass{lehramt-informatik-aufgabe}
\liLadePakete{}
\begin{document}
\liAufgabenTitel{VertexCover}
\section{Aufgabe 4
\index{Polynomialzeitreduktion}
\footcite{46115:2016:09}}

Betrachten Sie die beiden folgenden Probleme:

\liPseudoUeberschrift{VERTEX COVER}

\begin{description}
\item[Gegeben:]

Ein ungerichteter Graph $G = (V, E)$ und eine Zahl $k \in \{ 1, 2, 3,
\dots \}$.

\item[Frage:]

Gibt es eine Menge $C \subseteq V$ mit $|C| \leq k$, so dass für jede
Kante $(u, v)$ aus $E$ mindestens einer der Knoten $u$ und $v$ in $C$
ist?
\end{description}

\liPseudoUeberschrift{VERTEX COVER 3}

\begin{description}
\item[Gegeben:]

Ein ungerichteter Graph $G = (V, E)$ und eine Zahl $k \in \{ 3, 4, 5
\dots \}$.

\item[Frage:]

Gibt es eine Menge $C \subseteq V$ mit $|C| \leq k$, so dass für jede
Kante $(u, v)$ aus $E$ mindestens einer der Knoten $u$ und $v$ in $C$
ist?

Geben Sie eine polynomielle Reduktion von \texttt{VERTEX COVER} auf
\texttt{VERTEX COVER 3} an und begründe anschließend, dass die
Reduktion korrekt ist.
\end{description}
\footcite[Seite 12, Aufgabe 13]{theo:ab:4}

\begin{liAntwort}

$\texttt{VERTEX COVER} \leq_p \texttt{VERTEX COVER 3}$

$f$ fügt vier neue Knoten hinzu, von denen jeweils ein Paar verbunden
ist. Außerdem erhöht $f$ $k$ um $2$.

\begin{description}
\item[Total:]

Jeder Graph kann durch $f$ so verändert werden.

\item[Korrektheit:]

Wenn VC für k in G existiert, dann existiert auch VC mit $k + 2$ Knoten
in G 0 , da für den eingefügten Teilgraphen ein VC mit $k = 2$
existiert.

\item[In Polyzeit berechenbar:]

für Adjazenzmatrix müssen lediglich 4 neue Spalten/Zeilen ein-
gefügt werden und k+2 berechnet werden.
\end{description}

\end{liAntwort}

\end{document}
