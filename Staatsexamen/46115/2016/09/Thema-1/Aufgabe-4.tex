\documentclass{lehramt-informatik-aufgabe}
\liLadePakete{}
\begin{document}
\liAufgabenTitel{VertexCover}
\section{Aufgabe 4
\index{Polynomialzeitreduktion}
\footcite{46115:2016:09}}

Betrachte die beiden folgenden Probleme:

V ERT EXCOV ER

\begin{description}
\item[Gegeben:]

Ein ungerichteter Graph G = (v, E) und eine Zahl k ∈ {1, 2, 3, ...}.
\item[Frage:]

Gibt es eine Menge C ⊆ V mit |C| ≤ k, so dass für jede Kante (u, v) aus E
mindestens einer der Knoten u und v in C ist?
V ERT EXCOV ER3
\end{description}

\begin{description}
\item[Gegeben:]

Ein ungerichteter Graph G = (v, E) und eine Zahl k ∈ {3, 4, 5, ...}.

\item[Frage:]

Gibt es eine Menge C ⊆ V mit |C| ≤ k, so dass für jede Kante (u, v) aus E
mindestens einer der Knoten u und v in C ist?
Gib eine polynomielle Reduktion von V ERT EXCOV ER auf V ERT EXCOV ER3 an
und begründe anschließend, dass die Reduktion korrekt ist.
\end{description}
\footcite[Aufgabe 13, Seite 12]{theo:ab:4}

\begin{liAntwort}

V ERT EXCOV ER ≤ p V ERT EXCOV ER3

f fügt vier neue Knoten, von denen jeweils ein Paar verbunden ist. Außerdem erhöht f k um 2.

Total: Jeder Graph kann durch f so verändert werden. Korrektheit: Wenn VC für k in G exis-
tiert, dann existiert auch VC mit k + 2 Knoten in G 0 , da für den eingefügten Teilgraphen ein VC
mit k = 2 existiert.
In Polyzeit berechenbar: für Adjazenzmatrix müssen lediglich 4 neue Spalten/Zeilen ein-
gefügt werden und k+2 berechnet werden.

\end{liAntwort}

\end{document}
