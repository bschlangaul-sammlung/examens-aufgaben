\documentclass{lehramt-informatik-aufgabe}
\liLadePakete{syntax,mathe}
\begin{document}

\section{Aufgabe 7: Dynamische Programmierung
\index{Dynamische Programmierung}
\footcite{aud:pu:7}}

Mittels Dynamischer Programmierung (auch Memoization genannt) kann man
insbesondere rekursive Lösungen auf Kosten des Speicherbedarf
beschleunigen, indem man Zwischenergebnisse „abspeichert“ und bei
(wiederkehrendem) Bedarf „abruft“, ohne sie erneut berechnen zu müssen.
\footcite[Staatsexamen AuD/TI, 46115, Herbst 2016, Thema 2 Aufgabe 4]{examen:46115:2016:09}

Gegeben sei folgende geschachtelt-rekursive Funktion für $n, m \geq 0$:

\begin{equation*}
a(n, m) =
\begin{cases}
n + \lfloor \frac{n}{2} \rfloor &
\text{falls}\ m = 0\\

a(1, m-1), &
\text{falls}\ n = 0 \land m \neq 0 \\

a(n + \lfloor \sqrt{a(n-1,m)} \rfloor, m - 1), &
sonst \\
\end{cases}
\end{equation*}

\begin{enumerate}
\item Implementieren Sie die obige Funktion \java{a(n,m)} zunächst ohne
weitere Optimierungen als Prozedur/Methode in einer Programmiersprache
Ihrer Wahl.

\begin{antwort}
\inputcode[firstline=4,lastline=12]{aufgaben/aud/examen_46115_2016_09/DynamischeProgrammierung}
\end{antwort}

\item Geben Sie nun eine DP-Implementierung der Funktion \java{a(n,m)}
an, die \java{a(n,m)} für $0 \geq n \geq 100000$ und $0 \geq m \geq 25$
höchstens einmal gemäß obiger rekursiver Definition berechnet. Beachten
Sie, dass Ihre Prozedur trotz/dem auch weiterhin mit $n > 100000$ und $m
> 25$ aufgerufen werden können soll.

\begin{antwort}
\inputcode[firstline=14,lastline=34]{aufgaben/aud/examen_46115_2016_09/DynamischeProgrammierung}
\end{antwort}
\end{enumerate}
\end{document}
