\documentclass{lehramt-informatik-aufgabe}
\InformatikPakete{syntax,mathe}
\begin{document}

\section{Aufgabe 5: Komplexität \index{Komplexität}
\footcite[entnommen aus Algorithmen und Datenstrukturen, Übungsblatt 3, Universität Würzburg]{aud:pu:7}}

Geben Sie jeweils die kleinste, gerade noch passende Laufzeitkomplexität
folgender Java-Methoden im O-Kalkül (Landau-Notation) in Abhängigkeit
von n und ggf. der Länge der Arrays an. \footcite[Staatsexamen
Theoretische Informatik, Algorithmen und Datenstrukturen, Realschulen,
Herbst 2016, Thema 2 Aufgabe 2 (Auszug)]{examen:46115:2016:09}

\begin{enumerate}
\item \strut\bigskip

\inputcode[firstline=5,lastline=13]{aufgaben/aud/examen_46115_2016_09/Komplexitaet}

\begin{antwort}
Die Laufzeit liegt in $\mathcal{O}(n^2)$.

Begründung (nicht verlangt): Die äußere Schleife wird $n$-mal
durchlaufen. Die innere Schleife wird dann jeweils wieder $n$-mal
durchlaufen. Die Größe des Arrays spielt hier übrigens keine Rolle, da
die Schleifen ohnehin immer nur bis zum Wert $n$ ausgeführt werden.
\end{antwort}

\item \strut\bigskip

\inputcode[firstline=15,lastline=25]{aufgaben/aud/examen_46115_2016_09/Komplexitaet}

\begin{antwort}
Die Laufzeit liegt in $\mathcal{O}(\log(\text{keys.length}))$. Dabei ist
keys.length die Größe des Arrays bezüglich seiner ersten Dimension.

Begründung (nicht verlangt): Der Grund für diese Laufzeit ist derselbe
wie bei der binären Suche\footcite[Seite 122 (PDF 140)]{saake}. Die
Größe des Arrays bezüglich seiner zweiten Dimension spielt hier übrigens
keine Rolle, da diese Dimension hier ja nur einen einzigen festen Wert
annimmt.
\end{antwort}
\end{enumerate}

\end{document}
