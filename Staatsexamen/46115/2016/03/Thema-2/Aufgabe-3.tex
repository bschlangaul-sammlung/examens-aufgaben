\documentclass{lehramt-informatik-aufgabe}
\liLadePakete{}
\begin{document}
\liAufgabenTitel{Bruchsicherheit von Smartphones}
\section{Aufgabe 3
\index{Algorithmische Komplexität (O-Notation)}
\footcite{46115:2016:03}}

Sie sollen mithilfe von Falltests eine neue Serie von Smartphones auf
Bruchsicherheit testen.

Dazu wird eine Leiter mit $n$ Sprossen verwendet; die höchste Sprosse,
von der ein Smartphone heruntergeworfen werden kann ohne zu zerbrechen,
heiße \emph{„höchste sichere Sprosse“}. Das Ziel ist, die höchste
sichere Sprosse zu ermitteln. Man kann davon ausgehen, dass die höchste
sichere Sprosse nicht von der Art des Wurfs abhängt und dass alle
verwendeten Smartphones sich gleich verhalten. Eine Möglichkeit, die
höchste sichere Sprosse zu ermitteln, besteht darin, ein Gerät erst von
Sprosse $1$, dann von Sprosse $2$, etc. abzuwerfen, bis es schließlich
beim Wurf von Sprosse $k$ beschädigt wird (oder Sie oben angelangt
sind). Sprosse $k - 1$ (bzw. $n$) ist dann die höchste sichere Sprosse.
Bei diesem Verfahren wird maximal ein Smartphone zerstört, aber der
Zeitaufwand ist ungünstig.

\begin{enumerate}

%%
% a)
%%

\item Bestimmen Sie die Zahl der Würfe bei diesem Verfahren im
schlechtesten Fall.
\index{Lineare Suche}

\begin{liAntwort}
Die Zahl der Würfe im schlechtesten Fall ist $\mathcal{O} (k)$, wobei
$k$ die Anzahl der Sprossen ist. Geht das Smartphone erst bei der
höchsten Sprosse kaputt, muss es $k$ mal heruntergeworfen werden. Die
Komplexität entspricht der der linearen Suche.
\end{liAntwort}

%%
% b)
%%

\item Geben Sie nun ein Verfahren zur Ermittlung der höchsten sicheren
Sprosse an, welches nur $\mathcal{O}(\log n)$ Würfe benötigt, dafür
aber möglicherweise mehr Smartphones verbraucht.

\begin{liAntwort}
Man startet bei Sprosse $\frac{n}{2}$. Wenn das Smartphone kaputt geht,
macht man weiter mit der Sprosse in der Mitte der unteren Hälfte,
ansonsten mit der Sprosse in der Mitte der oberen Hälfte. Das Ganze
rekursiv.
\end{liAntwort}

%%
% c)
%%

\item Es gibt eine Strategie zur Ermittlung der höchsten sicheren
Sprosse mit $\mathcal{O}\left(\sqrt{n}\right)$ Würfen, bei dessen Anwendung
höchstens zwei Smartphones kaputtgehen. Finden Sie diese Strategie und
begründen Sie Ihre Funktionsweise und Wurfzahl.

Tipp: der erste Testwurf erfolgt von Sprosse $\left\lceil\sqrt{n}\,\right\rceil$.

\begin{liExkurs}[Interpolationssuche]
Die Interpolationssuche, auch Intervallsuche genannt, ist ein von der
binären Suche abgeleitetes Suchverfahren, das auf Listen und Feldern zum
Einsatz kommt.

Während der Algorithmus der binären Suche stets das mittlere Element des
Suchraums überprüft, versucht der Algorithmus der Interpolationssuche im
Suchraum einen günstigeren Teilungspunkt als die Mitte zu erraten. Die
Arbeitsweise ist mit der eines Menschen vergleichbar, der ein Wort in
einem Wörterbuch sucht: Die Suche nach Zylinder wird üblicherweise am
Ende des Wörterbuches begonnen, während die Suche nach Aal im vorderen
Bereich begonnen werden dürfte.
\liFussnoteUrl{https://de.wikipedia.org/wiki/Quadratische_Binärsuche}
\end{liExkurs}

\begin{liExkurs}[Quadratische Binärsuche]
Quadratische Binärsuche ist ein Suchalgorithmus ähnlich der Binärsuche
oder Interpolationssuche. Es versucht durch Reduzierung des Intervalls
in jedem Rekursionsschritt die Nachteile der Interpolationssuche zu
vermeiden.

Nach dem Muster der Interpolationssuche wird zunächst in jedem
rekursiven Schritt die vermutete Position $k$ interpoliert. Anschließend
wird – um die Nachteile der Interpolationssuche zu vermeiden – das
Intervall der Länge $\sqrt{n}$ gesucht, in
dem sich der gesuchte Wert befindet. Auf dieses Intervall wird der
nächste rekursive Aufruf der Suche angewendet.

Auf diese Weise verkleinert sich der Suchraum bei gegebener Liste der
Länge $n$ bei jedem rekursiven Schritt auf eine Liste der Länge $n$
$\sqrt n$.
\liFussnoteUrl{https://de.wikipedia.org/wiki/Quadratische_Binärsuche}
\end{liExkurs}

\begin{liAntwort}
Das Vorgehen ist folgendermaßen: Man beginnt auf Stufe 0 und falls das
Handy nicht kaputt geht, addiert man jeweils Wurzel n. Falls das Handy
kaputt geht, geht man linear in Einerschritten das Intervall von der
unteren Grenze (\dh von der Stufe vor der letzten Addition) bis zur
Kaputtstufe ab.
\liFussnoteUrl{http://www.inf.fu-berlin.de/lehre/WS06/HA/skript/vorlesung6.pdf}

\end{liAntwort}

\end{enumerate}
\end{document}
