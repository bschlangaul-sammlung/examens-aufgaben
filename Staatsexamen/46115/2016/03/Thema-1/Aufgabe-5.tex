\documentclass{lehramt-informatik-aufgabe}
\liLadePakete{}
\begin{document}
\liAufgabenTitel{NP}
\section{Aufgabe 5
\index{Komplexitätstheorie}
\footcite{46115:2016:03}}

Beschreiben Sie, was es heißt, dass ein Problem (Sprache) NP-vollständig
ist. Geben Sie ein NP-vollständiges Problem Ihrer Wahl an und erläuteren
Sie, dass (bzw.) warum es in NP liegt.
\footcite[Aufgabe 11, Seite 15]{theo:ab:4}

\begin{liAntwort}
NP-vollständig: NP-schwer und in NP

[Beliebiges Problem] liegt in NP, da der entsprechende Algorithmus
dieses Problem nicht-deterministisch in Polyzeit löst → Algorithmus rät
nichtdeterministisch Lösung, prüft sie in Polyzeit
\end{liAntwort}

\end{document}
