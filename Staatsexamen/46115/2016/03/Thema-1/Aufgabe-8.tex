\documentclass{bschlangaul-aufgabe}

\begin{document}
\liAufgabenTitel{Bubble- und Quicksort bei 25,1,12,27,30,9,33,34,18,16}
\section{Aufgabe 8
\index{Sortieralgorithmen}
\footcite{examen:46115:2016:03}}

\begin{enumerate}

%%
% a)
%%

\item Sortieren Sie das Array mit den Integer Zahlen

\begin{center}
25, 1, 12, 27, 30, 9, 33, 34, 18, 16
\end{center}

\begin{enumerate}

%%
% i)
%%

\item mit \emph{BubbleSort}
\index{Bubblesort}

\begin{liAntwort}
\footnotesize
\begin{verbatim}
 25  1   12  27  30  9   33  34  18  16  Eingabe
 25  1   12  27  30  9   33  34  18  16  Durchlauf Nr. 1
>25  1<  12  27  30  9   33  34  18  16  vertausche (i 0<>1)
 1  >25  12< 27  30  9   33  34  18  16  vertausche (i 1<>2)
 1   12  25  27 >30  9<  33  34  18  16  vertausche (i 4<>5)
 1   12  25  27  9   30  33 >34  18< 16  vertausche (i 7<>8)
 1   12  25  27  9   30  33  18 >34  16< vertausche (i 8<>9)
 1   12  25  27  9   30  33  18  16  34  Durchlauf Nr. 2
 1   12  25 >27  9<  30  33  18  16  34  vertausche (i 3<>4)
 1   12  25  9   27  30 >33  18< 16  34  vertausche (i 6<>7)
 1   12  25  9   27  30  18 >33  16< 34  vertausche (i 7<>8)
 1   12  25  9   27  30  18  16  33  34  Durchlauf Nr. 3
 1   12 >25  9<  27  30  18  16  33  34  vertausche (i 2<>3)
 1   12  9   25  27 >30  18< 16  33  34  vertausche (i 5<>6)
 1   12  9   25  27  18 >30  16< 33  34  vertausche (i 6<>7)
 1   12  9   25  27  18  16  30  33  34  Durchlauf Nr. 4
 1  >12  9<  25  27  18  16  30  33  34  vertausche (i 1<>2)
 1   9   12  25 >27  18< 16  30  33  34  vertausche (i 4<>5)
 1   9   12  25  18 >27  16< 30  33  34  vertausche (i 5<>6)
 1   9   12  25  18  16  27  30  33  34  Durchlauf Nr. 5
 1   9   12 >25  18< 16  27  30  33  34  vertausche (i 3<>4)
 1   9   12  18 >25  16< 27  30  33  34  vertausche (i 4<>5)
 1   9   12  18  16  25  27  30  33  34  Durchlauf Nr. 6
 1   9   12 >18  16< 25  27  30  33  34  vertausche (i 3<>4)
 1   9   12  16  18  25  27  30  33  34  Durchlauf Nr. 7
 1   9   12  16  18  25  27  30  33  34  Ausgabe
\end{verbatim}
\end{liAntwort}

%%
% ii)
%%

\item mit \emph{Quicksort}, wenn als Pivotelement das jeweils erste
Element gewählt wird.\index{Quicksort}

\end{enumerate}

Beschreiben Sie die Abläufe der Sortierverfahren

\begin{enumerate}

%%
% i)
%%

\item bei \emph{BubbleSort} durch eine Angabe der Zwischenergebnisse
nach jedem Durchlauf

%%
% ii)
%%

\item bei \emph{Quicksort} durch die Angabe der Zwischenergebnisse nach
den rekursiven Aufrufen.

\end{enumerate}
%%
% b)
%%

\item Welche Laufzeit (asymptotisch, in O-Notation) hat BubbleSort bei
beliebig großen Arrays mit n Elementen. Begründen Sie Ihre Antwort.

\end{enumerate}
\end{document}
