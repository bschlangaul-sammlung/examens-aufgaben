\documentclass{lehramt-informatik-aufgabe}
\liLadePakete{syntax}
\begin{document}
\liAufgabenTitel{Schreibtischlauf Haldensortierung}

\section{Aufgabe 6
\footcite[Thema 2 Aufgabe 6]{examen:46115:2013:03}}

\begin{enumerate}

%%
% a)
%%

\item Vervollständigen Sie die folgende Sortierung mit MergeSort
(Sortieren durch Mischen) — beginnen Sie dabei Ihren
„rekursiven Abstieg“ immer im linken Teilfeld:
\index{Mergesort}

D | 40 5 89 95 85 84 || 14 25 20 52 7 71 |

Notation: Markieren Sie Zeilen mit D(ivide), in denen das Array zerlegt
wird, und mit M(erge), in denen Teilarrays zusammengeführt werden.
Beispiel:

D | 82 || 89 44 |

D 82 | 89 || 44 |

M 82 | 44 89 |

M | 44 82 89 |

\begin{liAntwort}
D | 40 5 89 95 85 84 || 14 25 20 52 7 71 |
\end{liAntwort}

%%
% b)
%%

\item Sortieren Sie mittels HeapSort (Haldensortierung) die folgende
Liste weiter: Notation: Markieren Sie die Zeilen wie
folgt:\index{Heapsort}

\footcite[Aufgabe 4: Sortieren II entnommen aus Algorithmen und
Datenstrukturen, Übungsblatt 4, Universität Würzburg]{aud:pu:7}

\begin{description}
\item[I:] Initiale Heap-Eigenschaft hergestellt (größtes Element am
Anfang der Liste).

\item[R:] Erstes und letztes Element getauscht und letztes „gedanklich
entfernt“.

\item[S:] Erstes Element nach unten „versickert“ (Heap-Eigenschaft
wiederhergestellt).
\end{description}

\end{enumerate}

\end{document}
