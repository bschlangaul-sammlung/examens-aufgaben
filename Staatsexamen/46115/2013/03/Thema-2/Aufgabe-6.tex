\documentclass{lehramt-informatik-aufgabe}
\InformatikPakete{syntax}
\begin{document}
\section{Aufgabe 6
\index{Heapsort}
\footcite[Thema 2 Aufgabe 6]{examen:46115:2013:03}}

\begin{enumerate}

%%
% b)
%%

\item Sortieren Sie mittels HeapSort (Haldensortierung) die folgende
Liste weiter: Notation: Markieren Sie die Zeilen wie
folgt:

\footcite[Aufgabe 4: Sortieren II entnommen aus Algorithmen und
Datenstrukturen, Übungsblatt 4, Universität Würzburg]{aud:pu:7}

\begin{description}
\item[I:] Initiale Heap-Eigenschaft hergestellt (größtes Element am
Anfang der Liste).

\item[R:] Erstes und letztes Element getauscht und letztes „gedanklich
entfernt“.

\item[S:] Erstes Element nach unten „versickert“ (Heap-Eigenschaft
wiederhergestellt).
\end{description}

\end{enumerate}

\end{document}
