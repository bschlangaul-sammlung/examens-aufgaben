\documentclass{lehramt-informatik-minimal}
\InformatikPakete{syntax}
\begin{document}
\section{Aufgabe 4: Sortieren II\footcite[entnommen aus Algorithmen und
Datenstrukturen, Übungsblatt 4, Universität Würzburg]{aud:pu:7}}

\begin{enumerate}

%%
% b)
%%

\item Sortieren Sie mittels HeapSort (Haldensortierung) die folgende
Liste weiter: Notation: Markieren Sie die Zeilen wie
folgt:\footcite[Staatsexamen Theoretische Informatik, Algorithmen und
Datenstrukturen, Realschulen, Frühjahr 2013, Thema 2 Aufgabe 6
(Auszug)]{examen:46115:2013:03}

\begin{description}
\item[I:] Initiale Heap-Eigenschaft hergestellt (größtes Element am
Anfang der Liste).

\item[R:] Erstes und letztes Element getauscht und letztes „gedanklich
entfernt“.

\item[S:] Erstes Element nach unten „versickert“ (Heap-Eigenschaft
wiederhergestellt).
\end{description}

\end{enumerate}

\end{document}
