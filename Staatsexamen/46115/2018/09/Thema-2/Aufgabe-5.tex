\documentclass{lehramt-informatik-aufgabe}
\liLadePakete{}
\begin{document}
\liAufgabenTitel{Springerproblem beim Schach}

\section{Aufgabe 5 (Backtracking)
\index{Backtracking}
\footcite[Thema 2 Aufgabe 5 Seite 11-12]{examen:46115:2018:09}}

Das Springerproblem ist ein kombinatorisches Problem, das darin besteht,
für einen Springer auf einem leeren Schachbrett eine Route von einem
gegebenen Startfeld aus zu finden, auf der dieser jedes Feld des
Schachbretts genau einmal besucht.

Ein Schachbrett besteht aus 8x8 Feldern. Ein Springer kann bei einem Zug
von einem Ausgangsfeld aus eines von maximal 8 Folgefelder betreten, wie
dies in der folgenden Abbildung dargestellt ist. Der Springer darf
selbstverstndlich nicht über den Rand des Schachbretts hinausspringen.

Eine Lösung des Springerproblems mit Startfeld h1 sieht wie folgt aus.
Die Felder sind in ihrer Besuchsreihenfolge durchnummeriert. Der
Springer bewegt sich also von hi nach £2, dann von f2 nach h3 usw.

\begin{tabular}{llllllll}
41&10&29&26&49&12&31&16\\
28&25&40&11&30&15&50&13\\
9&42&27&56&61&48&17&32\\
24&39&58&47&64&53&14&51\\
43&8&55&62&57&60&33&18\\
38&23&46&59&54&63&52&3\\
7&44&21&36&5&2&19&34\\
22&37&6&45&20&35&4&1\\
\end{tabular}

Formulieren Sie einen rekursiven Algorithmus zur Lösung des
Springerproblems von einem vorgegebenen Startfeld aus. Es sollen dabei
alle möglichen Lösungen des Springerproblems gefunden werden. Die
Lösungen sollen durch Backtracking gefunden werden. Hierbei werden alle
möglichen Teilrouten systematisch durchprobiert, und Teilrouten, die
nicht zu einer Lösung des Springerproblems führen können, werden nicht
weiterverfolgt. Dies ist durch rekursiven Aufruf einer Lésungsfunktion
huepf(z, y, z) zu realisieren, wobei

\begin{itemize}

\item x und y die Koordinaten des als nächstes anzuspringenden Feldes
sind, und

\item z die aktuelle Rekursionstiefe enthält. Wenn die Rekursionstiefe
64 erreicht und das betreffende Feld noch unbesucht ist, ist eine Lösung
des Springerproblems gefunden.
\end{itemize}

Der initiale Aufruf Ihres Algorithmus kann beispielsweise über den Aufruf
huepf(1,8,1)

erfolgen.

Wählen Sie geeignete Datenstrukturen zur Verwaltung der unbesuchten
Felder und zum Speichern gefundener (Teil)Lösungen. Der Algorithmus soll
eine gefundene Lösung in der oben angegebenen Form ausdrucken, also als
Matrix mit der Besuchsreihenfolge pro Feld.

\end{document}
