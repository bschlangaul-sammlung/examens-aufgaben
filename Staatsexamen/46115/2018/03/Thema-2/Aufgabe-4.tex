\documentclass{lehramt-informatik-aufgabe}
\liLadePakete{graph}
\begin{document}

\section{Aufgabe 12: Graphen III
\index{minimaler Spannbaum}
\index{Algorithmus von Prim}
\footcite[(entnommen aus Algorithmen und Datenstrukturen, Übungsblatt 7, Universität Würzburg)]{aud:pu:7}}

Sei G der folgende Graph.
\footcite[Thema 2 Aufgabe 4 (gekürzt)]{examen:46115:2018:03}

\graph knoten {
  \knoten{a}(0,4)
  \knoten{b}(2,4)
  \knoten{c}(0,2)
  \knoten{d}(2,2)
  \knoten{e}(0,0)
  \knoten{f}(2,0)
} kanten {
  \kante(a-b){10}
  \kante(a-c){8}
  \kante(b-c){4}
  \kante(b-d){2}
  \kante(c-d){12}
  \kante(c-e){3}
  \kante(c-f){5}
  \kante(d-f){2}
  \kante(e-f){8}
}

\begin{enumerate}

%%
% (a)
%%

\item Der Algorithmus von Prim ist ein Algorithmus zur Bestimmung des
minimalen Spannbaums in einem Graphen. Geben Sie einen anderen
Algorithmus zur Bestimmung des minimalen Spannbaums an.

\begin{antwort}
Zum Beispiel der Algorithmus von Kruskal
\end{antwort}

%%
% (b)
%%

\item Führen Sie den Algorithmus von Prim schrittweise auf $G$ aus.
Ausgangsknoten soll der Knoten $a$ sein. Ihre Tabelle sollte wie folgt
beginnen:

\begin{tabular}{|l|l|l|l|l|l|l|}
\hline
a &
b &
c &
d &
e &
f &
Warteschlange\\\hline
\end{tabular}

Die Einträge der Tabelle geben an, wie weit der angegebene Knoten
vom aktuellen Baum entfernt ist.

\begin{antwort}
\graph knoten {
  \knoten{a}(0,4)
  \knoten{b}(2,4)
  \knoten{c}(0,2)
  \knoten{d}(2,2)
  \knoten{e}(0,0)
  \knoten{f}(2,0)
} kanten {
  \kante(a-b){10}
  \KANTE(a-c){8}
  \KANTE(b-c){4}
  \KANTE(b-d){2}
  \kante(c-d){12}
  \KANTE(c-e){3}
  \kante(c-f){5}
  \KANTE(d-f){2}
  \kante(e-f){8}
}

\begin{tabular}{|l|l|l|l|l|l|l|}
\hline
a &
b &
c &
d &
e &
f &
Warteschlange\\\hline\hline

0 & % a
$\infty$ & % b
$\infty$ & % c
$\infty$ & % d
$\infty$ & % e
$\infty$ & % f
a \\\hline % Warteschlange

0 & % a
10 & % b
8 & % c
$\infty$ & % d
$\infty$ & % e
$\infty$ & % f
c, b\\\hline % Warteschlange

0 & % a
4 & % b
0 & % c
12 & % d
3 & % e
5 & % f
e, b, f, d\\\hline % Warteschlange

0 & % a
4 & % b
0 & % c
12 & % d
0 & % e
5 & % f
b, f, d \\\hline % Warteschlange

0 & % a
0 & % b
0 & % c
2 & % d
0 & % e
5 & % f
d, f \\\hline % Warteschlange

0 & % a
0 & % b
0 & % c
0 & % d
0 & % e
2 & % f
f \\\hline % Warteschlange

0 & % a
0 & % b
0 & % c
0 & % d
0 & % e
0 & % f
\\\hline % Warteschlange
\end{tabular}

\end{antwort}

%%
% (c)
%%

\item Erklären Sie, warum der Kürzeste-Wege-Baum (also das gezeichnete
Ergebnis des Dijkstra-Algorithmus) und der minimale Spannbaum nicht
notwendigerweise identisch sind.

\begin{antwort}
Die Wahl der nächsten Kante erfolgt nach völlig verschiedenen Kriterien:

\begin{itemize}
\item Beim Kürzeste-Wege-Baum orientiert sie sich an der Entfernung der
einzelnen Knoten vom Startknoten.

\item Beim Spannbaum orientiert sie sich an der Entfernung der einzelnen
Knoten vom bereits erschlossenen Teil des Spannbaums.
\end{itemize}
\end{antwort}
\end{enumerate}
\end{document}
