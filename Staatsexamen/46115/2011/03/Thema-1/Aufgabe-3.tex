\documentclass{lehramt-informatik-aufgabe}
\liLadePakete{baum,spalten}

\begin{document}
\liAufgabenTitel{2-3-4-Baum}

\section{Aufgabe 9: Bäume
\index{B-Baum}\index{AVL-Baum}
\footcite[entnommen aus Algorithmen und
Datenstrukturen, Übungsblatt 6, Universität Würzburg]{aud:pu:7}}

\begin{enumerate}

%%
% (a)
%%

\item Fügen Sie in einen anfangs leeren 2-3-4-Baum (B-Baum der Ordnung
4)\footnote{ein Baum, für den folgendes gilt: Er besitzt in einem Knoten
max. 3 Schlüssel-Einträge und 4 Kindknoten und minimal einen Schlüssel
und 2 Nachfolger} der Reihe nach die folgenden Schlüssel ein:

\centerline{$1$, $2$, $3$, $5$, $7$, $8$, $9$, $4$, $11$, $12$, $13$, $6$.}

Dokumentieren Sie die Zwischenschritte so,
dass die Entstehung des Baumes und nicht nur das Endergebnis
nachvollziehbar ist. \footcite[Staatsexamen Theoretische Informatik,
Algorithmen und Datenstrukturen, Realschulen, Frühjahr 2011, Thema 1
Aufgabe 3]{examen:46115:2011:03}

\begin{liAntwort}
\begin{multicols}{2}
\begin{enumerate}

%%
%
%%

\item 1, 2, 3 einfügen:

\begin{tikzpicture}[
  scale=0.8,
  transform shape,
  bbaum,
  level 1/.style={level distance=10mm,sibling distance=32mm},
  level 2/.style={level distance=10mm,sibling distance=20mm},
]
\node {1 \nodepart{two} 2 \nodepart{three} 3};
\end{tikzpicture}

%%
%
%%

\item 5 einfügen:

\begin{tikzpicture}[
  scale=0.8,
  transform shape,
  bbaum,
  level 1/.style={level distance=8mm,sibling distance=25mm},
]
\node {2} [->]
  child {node {1}}
  child {node {3 \nodepart{two} 5}}
;
\end{tikzpicture}

%%
%
%%

\item 7 einfügen:

\begin{tikzpicture}[
  scale=0.8,
  transform shape,
  bbaum,
  level 1/.style={level distance=8mm,sibling distance=25mm},
]
\node {2} [->]
  child {node {1}}
  child {node {3 \nodepart{two} 5 \nodepart{three} 7}}
;
\end{tikzpicture}

%%
%
%%

\item 8 einfügen:

\begin{tikzpicture}[
  scale=0.8,
  transform shape,
  bbaum,
  level 1/.style={level distance=10mm,sibling distance=15mm},
]
\node {2 \nodepart{two} 5} [->]
  child {node {1}}
  child {node {3}}
  child {node {7 \nodepart{two} 8}}
;
\end{tikzpicture}

%%
%
%%

\item 9 und 4 einfügen:

\begin{tikzpicture}[
  scale=0.8,
  transform shape,
  bbaum,
  level 1/.style={level distance=10mm,sibling distance=15mm},
]
\node {2 \nodepart{two} 5} [->]
  child {node {1}}
  child {node {3 \nodepart{two} 4}}
  child {node {7 \nodepart{two} 8 \nodepart{three} 9}}
;
\end{tikzpicture}

%%
%
%%

\item 11 einfügen:

\begin{tikzpicture}[
  scale=0.8,
  transform shape,
  bbaum,
  level 1/.style={level distance=10mm,sibling distance=15mm},
]
\node {2 \nodepart{two} 5 \nodepart{three} 8} [->]
  child {node {1}}
  child {node {3 \nodepart{two} 4}}
  child {node {7}}
  child {node {9 \nodepart{two} 11}}
;
\end{tikzpicture}

%%
%
%%

\item 12 einfügen:

\begin{tikzpicture}[
  scale=0.8,
  transform shape,
  bbaum,
  level 1/.style={level distance=10mm,sibling distance=15mm},
]
\node {2 \nodepart{two} 5 \nodepart{three} 8} [->]
  child {node {1}}
  child {node {3 \nodepart{two} 4}}
  child {node {7}}
  child {node {9 \nodepart{two} 11 \nodepart{three} 12}}
;
\end{tikzpicture}

%%
%
%%

\item 13 einfügen (zwei Splits):

\begin{tikzpicture}[
  scale=0.8,
  transform shape,
  bbaum,
  level 1/.style={level distance=7mm,sibling distance=31mm},
  level 2/.style={level distance=10mm,sibling distance=12mm},
]
\node {5} [->]
  child {node {2}
    child {node {1}}
    child {node {3 \nodepart{two} 4}}
  }
  child {node {8 \nodepart{two} 11}
    child {node {7}}
    child {node {9}}
    child {node {12 \nodepart{two} 13}}
  }
;
\end{tikzpicture}

%%
%
%%

\item 6 einfügen:

\begin{tikzpicture}[
  scale=0.8,
  transform shape,
  bbaum,
  level 1/.style={level distance=7mm,sibling distance=31mm},
  level 2/.style={level distance=10mm,sibling distance=12mm},
]
\node {5} [->]
  child {node {2}
    child {node {1}}
    child {node {3 \nodepart{two} 4}}
  }
  child {node {8 \nodepart{two} 11}
    child {node {6 \nodepart{two} 7}}
    child {node {9}}
    child {node {12 \nodepart{two} 13}}
  }
;
\end{tikzpicture}
\end{enumerate}
\end{multicols}
\end{liAntwort}

%%
% (b)
%%

\item Zeichnen Sie einen Rot-Schwarz-Baum oder einen AVL-Baum, der
dieselben Einträge enthält.

%%
% (c)
%%

\item Geben Sie eine möglichst gute untere Schranke (in
$\Omega$-Notation) für die Anzahl der Schlüssel in einem 2-3-4-Baum der
Höhe h an.

Hinweis: Überlegen Sie sich, wie ein 2-3-4-Baum mit Höhe $h$ und
möglichst wenigen Schlüsseln aussieht.

\begin{liAntwort}
Ein 2-3-4-Baum mit möglichst wenigen Schlüsseln sieht aus wie ein
Binärbaum:

\begin{itemize}
\item Ein Baum der Höhe $1$ hat $1$ Schlüssel.
\item Ein Baum der Höhe $2$ hat $3$ Schlüssel.
\item Ein Baum der Höhe $3$ hat $7$ Schlüssel.
\item $\cdots$
\item Ein Baum der Höhe $h$ hat $2^h - 1$ Schlüssel.
\end{itemize}

Also liegt die Untergrenze für die Anzahl der Schlüssel in
$\Omega(2^h)$.
\end{liAntwort}

%%
% (d)
%%

\item Geben Sie eine möglichst gute obere Schranke (in
$\mathcal{O}$-Notation) für die Anzahl der Schlüssel in einem 2-3-4-Baum
der Höhe h an.

\begin{liAntwort}
Ein 2-3-4-Baum mit möglichst vielen Schlüsseln hat in jedem Knoten drei
Schlüssel. Und jeder Knoten, der kein Blatt ist, hat vier Kinder:

\begin{itemize}
\item Ein Baum der Höhe $1$ hat $3$ Schlüssel.
\item Ein Baum der Höhe $2$ hat $15$ Schlüssel.
\item Ein Baum der Höhe $3$ hat $63$ Schlüssel.
\item $\cdots$
\item Ein Baum der Höhe $h$ hat $4^h - 1$ Schlüssel.
\end{itemize}

Also liegt die Obergrenze für die Anzahl der Schlüssel in
$\mathcal{O}(4^h)$.
\end{liAntwort}

\end{enumerate}
\end{document}
