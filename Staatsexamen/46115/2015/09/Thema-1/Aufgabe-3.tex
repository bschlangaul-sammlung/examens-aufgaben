\documentclass{lehramt-informatik-aufgabe}
\liLadePakete{syntax}
\begin{document}
\liAufgabenTitel{Unimodale Zahlenfolge}
\section{Aufgabe 3
\index{Teile-und-Herrsche (Divide-and-Conquer)}
\footcite{46115:2015:09}}

Eine Folge von Zahlen $a_1, \dots, a_n$ heiße unimodal, wenn sie bis zu
einem bestimmten Punkt echt ansteigt und dann echt fällt. Zum Beispiel
ist die Folge $1,3,5,6,5,2,1$ unimodal, die Folgen $1,3,5,4,7,2,1$ und
$1,2,3,3,4,3,2,1$ aber nicht.

\begin{liExkurs}[Unimodale Abbildung]
Eine unimodale Abbildung oder unimodale Funktion ist in der Mathematik
eine Funktion mit einem eindeutigen (lokalen und globalen) Maximum wie
zum Beispiel $f(x)=-x^{2}$.
\liFussnoteUrl{https://de.wikipedia.org/wiki/Unimodale_Abbildung}
\end{liExkurs}

\begin{enumerate}

%%
% 1.
%%

\item Entwerfen Sie einen Algorithmus, der zu (als Array) gegebener
unimodaler Folge $a_1, \dots, a_n$ in Zeit $\mathcal{O}(\log n)$ das
Maximum $\max a_i$ berechnet. Ist die Folge nicht unimodal, so kann Ihr
Algorithmus ein beliebiges Ergebnis liefern. Größenvergleiche,
arithmetische Operationen und Arrayzugriffe können wie üblich in
konstanter Zeit ($\mathcal{O}(1)$) getätigt werden. Hinweise: binäre
Suche, divide-and-conquer.
\index{Binäre Suche}

\begin{liAntwort}
\liJavaExamen{46115}{2015}{09}{UnimodalFinder}
\end{liAntwort}

%%
% 2.
%%

\item Begründen Sie, dass Ihr Algorithmus tatsächlich in Zeit
$\mathcal{O}(\log n)$ läuft.

%%
% 3.
%%

\item Schreiben Sie Ihren Algorithmus in Pseudocode oder in einer
Programmiersprache Ihrer Wahl, \zB Java, auf. Sie dürfen voraussetzen,
dass die Eingabe in Form eines Arrays der Größe $n$ vorliegt.
\index{Implementierung in Java}

%%
% 4.
%%

\item Beschreiben Sie in Worten ein Verfahren, welches in Zeit
$\mathcal{O}(n)$ feststellt, ob eine vorgelegte Folge unimodal ist oder
nicht.

%%
% 5.
%%

\item Begründen Sie, dass es kein solches Verfahren (Test auf
Unimodalität) geben kann, welches in Zeit $\mathcal{O}(\log n)$ läuft.
\end{enumerate}
\end{document}
