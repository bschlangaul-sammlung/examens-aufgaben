\documentclass{lehramt-informatik-aufgabe}
\liLadePakete{mathe,syntax,vollstaendige-induktion}
\begin{document}
% m = markierung
\def\m#1{\textcolor{blue}{#1}}
\liAufgabenTitel{Methode function: Formale Verifikation - Induktionsbeweis}
\section{Aufgabe 4
\index{Vollständige Induktion}
\footcite{46115:2015:09}}

Gegeben sei die folgende Methode \liJavaCode{function}:

\liJavaExamen[firstline=4,lastline=9]{46115}{2015}{09}{Induktion}

\noindent
Beweisen Sie folgenden Zusammenhang mittels vollständiger Induktion:

\begin{displaymath}
\forall n \geq 1 \colon \text{function}(n) = f(n)\text{ mit }
f(n) := 1 - \frac{1}{n + 1}
\end{displaymath}

\noindent
Hinweis: Eventuelle Rechenungenauigkeiten, wie z. B. in Java, bei der
Behandlung von Fließkommazahlen (z. B. double) sollen beim Beweis nicht
berücksichtigt werden - Sie dürfen also annehmen, Fließkommazahlen
würden mathematische Genauigkeit aufweisen.

\begin{liAntwort}
\liInduktionAnfang

$f(1) := 1 - \frac{1}{1 + 1} = 1 - \frac{1}{2} = \frac{1}{2}$

\liInduktionVoraussetzung

$f(n) := 1 - \frac{1}{n + 1}$

\liInduktionSchritt

\liPseudoUeberschrift{zu zeigen:}

$f(n + 1) := 1 - \frac{1}{(n + 1) + 1} = f(n)$

\liPseudoUeberschrift{Vorarbeiten (Java in Mathe umwandeln):}

$\text{function}(n) = \frac{1}{n \cdot (n + 1)} + f(n - 1)$

\begin{align*}
f(n + 1) & := \frac{1}{\m{(n + 1)} \cdot (\m{(n + 1) }+ 1)} + f(\m{(n + 1)} - 1)& n + 1 \text{ einsetzen}\\
& = \frac{1}{(n + 1) \cdot (\m{n + 2})} + f(\text{n})
&  \text{ vereinfachen}\\
%
& = \frac{1}{(n + 1) \cdot (n + 2)} + \m{1 - \frac{1}{n + 1}}
& \text{für } f(n) \text{ Formel einsetzen}\\
%
& = 1 + \m{\frac{1}{(n + 1) \cdot (n + 2)}} - \frac{1}{n + 1}
& \text{1. Bruch an 2. Stelle geschrieben}\\
%
& = 1 + \frac{1}{(n + 1) \cdot (n + 2)} - \frac{1 \cdot \m{(n + 2)}}{(n + 1) \cdot \m{(n + 2)}}
& \text{2. Bruch mit }(n + 2) \text{ erweitert}\\
%
& = 1 + \frac{1 - (n + 2)}{(n + 1) \cdot (n + 2)}
& \text{die 2 Brüche subtrahiert}\\
%
& = 1 + \frac{1 - n \m{-} 2}{(n + 1) \cdot (n + 2)}
& \text{minus plus ist minus}\\
%
& = 1 + \frac{\m{-1} - n}{(n + 1) \cdot (n + 2)}
& \text{eins minus zwei ist minus eins}\\
%
& = 1 + \frac{\m{-1 \cdot (1 + n)}}{(n + 1) \cdot (n + 2)}
& (n + 1) \text{ ausklammern}\\
%
& = 1 + \left(\m{-1 \cdot} \frac{(1 + n)}{(n + 1) \cdot (n + 2)}\right)
& \text{minus vor den Bruch bringen}\\
%
& = 1 \m{-} \frac{(1 + n)}{(n + 1) \cdot (n + 2)}
& \text{plus minus ist minus}\\
%
& = 1 - \m{\frac{1}{n + 2}}
& (n + 1) \text{ gekürzt}\\
%
& = 1 - \frac{1}{\m{(n + 1)} + 1}
& \text{Umformen zur Verdeutlichung}\\
\end{align*}
\end{liAntwort}
\end{document}
