\documentclass{lehramt-informatik-aufgabe}
\liLadePakete{formale-sprachen}
\begin{document}
\liAufgabenMetadaten{
  Titel = {Aufgabe 3},
  Thematik = {L1, L2, L3 regulär oder kontextfrei},
  RelativerPfad = Staatsexamen/46115/2021/03/Thema-2/Teilaufgabe-1/Aufgabe-3.tex,
  ZitatSchluessel = examen:46115:2021:03,
  BearbeitungsStand = unbekannt,
  Korrektheit = unbekannt,
  Stichwoerter = {Reguläre Sprache, Kontextfreie Sprache},
  ExamenNummer = 46115,
  ExamenJahr = 2021,
  ExamenMonat = 03,
  ExamenThemaNr = 2,
  ExamenTeilaufgabeNr = 1,
  ExamenAufgabeNr = 3,
}

\let\m=\liMenge
\def\l#1{$L_#1$}

Sei $\mathbb{N}_0 = \m{0,1,2,\dots}$ die Menge aller natürlichen Zahlen
mit $0$. Betrachten Sie die folgenden Sprachen.
\index{Reguläre Sprache}
\index{Kontextfreie Sprache}
\footcite{examen:46115:2021:03}
\begin{enumerate}

%%
% a)
%%

\item \liAusdruck[L_1]{a^{3n} b^{2n} a^n}{n \in \mathbb{N}_0}

\begin{liAntwort}
nicht kontextfrei
\end{liAntwort}

%%
% b)
%%

\item \liAusdruck[L_2]{a^{3n} a^{2n} b^n}{n \in \mathbb{N}_0}

\begin{liAntwort}
kontextfrei.

Der Ausdruck lässt umformen in: \liAusdruck[L_2]{a^{5n} b^{n}}{n \in
\mathbb{N}_0}

\begin{liProduktionsRegeln}
S -> aaaaa S b | EPSILON
\end{liProduktionsRegeln}
\end{liAntwort}

%%
% c)
%%

\item \liAusdruck[L_3]{(ab)^{n} a (ba)^n b (ab)^n}{n \in \mathbb{N}_0}

\begin{liAntwort}
nicht kontextfrei
\end{liAntwort}

\end{enumerate}

\noindent
Geben Sie jeweils an, ob \l1, \l2 und \l3 kontextfrei und ob \l1, \l2
und \l3 regulär sind. Beweisen Sie Ihre Behauptung und ordnen Sie jede
Sprache in die kleinstmögliche Klasse (regulär, kontextfrei, nicht
kontextfrei) ein. Für eine Einordnung in kontextfrei zeigen Sie also,
dass die Sprache kontextfrei und nicht regulär ist.

Erfolgt ein Beweis durch Angabe eines Automaten, so ist eine klare
Beschreibung der Funktionsweise des Automaten und der Bedeutung der
Zustände erforderlich. Erfolgt der Beweis durch Angabe eines regulären
Ausdruckes, so ist eine intuitive Beschreibung erforderlich. Wird der
Beweis durch die Angabe einer Grammatik geführt, so ist die Bedeutung
der Variablen zu erläutern.
\end{document}
