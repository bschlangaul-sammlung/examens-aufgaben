\documentclass{lehramt-informatik-aufgabe}
\liLadePakete{}
\begin{document}
\liAufgabenMetadaten{
  Titel = {Aufgabe 1},
  Thematik = {O-Notation a(), b(), c(), d(), e(n)},
  RelativerPfad = Staatsexamen/46115/2021/03/Thema-2/Teilaufgabe-2/Aufgabe-1.tex,
  ZitatSchluessel = examen:46115:2021:03,
  BearbeitungsStand = unbekannt,
  Korrektheit = unbekannt,
  Stichwoerter = {Algorithmische Komplexität (O-Notation)},
  ExamenNummer = 46115,
  ExamenJahr = 2021,
  ExamenMonat = 03,
  ExamenThemaNr = 2,
  ExamenTeilaufgabeNr = 2,
  ExamenAufgabeNr = 1,
}

Sortieren Sie die unten angegebenen Funktionen der O-Klassen O(a), O(b),
O(c), O(d) und O(e) bezüglich ihrer Teilmengenbeziehungen. Nutzen Sie
ausschließlich die echte Teilmenge € sowie die Gleichheit = für die
Beziehung zwischen den Mengen. Folgendes Beispiel illustriert diese
Schreib- weise für einige Funktionen fı bis fs. (Diese haben nichts mit
den unten angegebenen Funktionen zu tun.)\index{Algorithmische Komplexität (O-Notation)}
\footcite{examen:46115:2021:03}

 O(fa) \& O(fs) = O(fs)  O(f:) = O(a)
Die angegebenen Beziehungen miissen weder bewiesen noch begründet werden.

\begin{itemize}
\item $a(n) = \sqrt{n^5} + 4n - 5$

\item $b(n) = log (log,(n))$
\item $c(n) = 2^n$

\item $d(n) = n? log(n) + 2n$

\item $e(n) = \frac{4^n}{\log_2n}$
\end{itemize}

\end{document}
