\documentclass{lehramt-informatik-aufgabe}
\liLadePakete{}
\begin{document}
\liAufgabenMetadaten{
  Titel = {Aufgabe 4},
  Thematik = {Graph a-i},
  RelativerPfad = Staatsexamen/46115/2021/03/Thema-2/Teilaufgabe-2/Aufgabe-4.tex,
  ZitatSchluessel = examen:46115:2021:03,
  BearbeitungsStand = unbekannt,
  Korrektheit = unbekannt,
  Stichwoerter = {Algorithmus von Dijkstra},
  ExamenNummer = 46115,
  ExamenJahr = 2021,
  ExamenMonat = 03,
  ExamenThemaNr = 2,
  ExamenTeilaufgabeNr = 2,
  ExamenAufgabeNr = 4,
}
\begin{enumerate}

%%
% a)
%%

\item Berechnen Sie im gegebenen gerichteten und gewichteten Graph G =
(V,E,w) mit Kantenlängen w : E — R mittels des Dijkstra-Algorithmus die
kürzesten (gerichteten) Pfade ausgehend vom Startknoten a.

Knoten, deren Entfernung von a bereits f+eststeht, seien als schwarz
bezeichnet und Knoten, bei denen lediglich eine obere Schranke \# oo für
ihre Entfernung von a bekannt ist, seien als grau
bezeichnet.\index{Algorithmus von Dijkstra}
\footcite{examen:46115:2021:03}

%%
% i)
%%

\item Geben Sie als Lösung eine Tabelle an. Fügen Sie jedes mal, wenn
der Algorithmus einen Knoten schwarz färbt, eine Zeile zu der Tabelle
hinzu. Die Tabelle soll dabei zwei Spalten beinhalten: die linke Spalte
zur Angabe des aktuell schwarz gewordenen Knotens und die rechte Spalte
mit der bereits aktualisierten Menge grauer Knoten. Jeder
Tabelleneintrag soll anstelle des nackten Knotennamens v ein Tripel (v,
v.d, v.r) sein. Dabei steht v.d für die aktuell bekannte kürzeste
Distanz zwischen a und v.

v.r ist der direkte Vorgänger von v auf dem zugehörigen kürzesten Weg von a.

%%
% ii)
%%

\item Zeichnen Sie zudem den entstandenen Kürzeste-Pfade-Baum.

%%
% b)
%%

\item Warum berechnet der Dijkstra-Algorithmus auf einem gerichteten
Eingabegraphen mit po- tentiell auch negativen Kantengewichten w : \# —
R nicht immer einen korrekten Kürzesten- Wege-Baum von einem gewählten
Startknoten aus? Geben Sie ein Beispiel an, für das der Algorithmus die
falsche Antwort liefert.

%%
% c)
%%

\item Begründen Sie, warum das Problem nicht gelöst werden kann, indem
der Betrag des niedrigs- ten (also des betragsmäßig größten negativen)
Kantengewichts im Graphen zu allen Kanten addiert wird.

\end{enumerate}
\end{document}
