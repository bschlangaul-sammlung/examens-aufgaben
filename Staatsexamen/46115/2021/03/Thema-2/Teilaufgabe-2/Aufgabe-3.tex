\documentclass{lehramt-informatik-aufgabe}
\liLadePakete{}
\begin{document}
\liAufgabenMetadaten{
  Titel = {Aufgabe 3},
  Thematik = {Lineare und Binäre Suchverfahren},
  RelativerPfad = Staatsexamen/46115/2021/03/Thema-2/Teilaufgabe-2/Aufgabe-3.tex,
  ZitatSchluessel = examen:46115:2021:03,
  BearbeitungsStand = unbekannt,
  Korrektheit = unbekannt,
  Stichwoerter = {Binäre Suche},
  ExamenNummer = 46115,
  ExamenJahr = 2021,
  ExamenMonat = 03,
  ExamenThemaNr = 2,
  ExamenTeilaufgabeNr = 2,
  ExamenAufgabeNr = 3,
}
\index{Binäre Suche}
\footcite{examen:46115:2021:03}

Gegeben ist ein aufsteigend sortiertes Array A von n ganzen Zahlen und eine ganze Zahl r.
Es wird der Algorithmus BinarySearch betrachtet, der A effizient nach dem Wert x absucht.
Ergebnis ist der Index i mit x = Ali] oder NIL, falls © \& A.

ı int BinarySearch(int[] A, int r)
2 f=1
3 r= A.length
4 while r > é do
5 m = | 45]
6 if 2 < Alm] then
7 r=om—l
8 else if 2 = Alm] then
9 | return m
10 else
11 | €=m+l1
12 return NIL
\begin{liAntwort}

%%
% a)
%%

\item Durchsuchen Sie das folgende Feld jeweils nach den in (i) bis
(iii) angegebenen Werten mittels binärer Suche. Geben Sie für jede
Iteration die Werte /,r,m und den betretenen if-Zweig an. Geben Sie
zudem den Ergebnis-Index bzw. NIL an.

Index

i]s] «| «| 2] 4] off

wen [ilsfol7] io] w]u]al ale!
%%
% (i)
%%

\item 10
%%
% (ii)
%%

\item 13
%%
% (iii)
%%

\item 22

%%
% b)
%%

\item Betrachten Sie auf das Array aus Teilaufgabe a). Für welche Werte
durchläuft der Algorith- mus nie den letzten else-Teil in Zeile 11?
Hinweis: Unterscheiden Sie auch zwischen enthaltenen und
nicht-enthaltenen Werten.

%%
% c)
%%

\item Wie ändert sich das Ergebnis der binären Suche, wenn im sortierten
Eingabefeld zwei auf-
einanderfolgende, unterschiedliche Werte vertauscht wurden? Betrachten Sie hierbei die be-
troffenen Werte, die anderen Feldelemente und nicht enthaltene Werte in Abhängigkeit vom
Ort der Vertauschung.

%%
% d)
%%

\item Angenommen, das Eingabearray A für den Algorithmus für die binäre
Suche enthält nur die Zahlen 0 und 1, aufsteigend sortiert. Zudem ist
jede der beiden Zahlen mindestens ein Mal vorhanden. Ändern Sie den
Algorithmus für die binäre Suche so ab, dass er den bzw. einen Index k
zurückgibt, für den gilt: Alk] =1 und Ak—1]=0.

%%
% e)
%%

\item Betrachten Sie die folgende rekursive Variante von BinarySearch.

1 int RekBinarySearch(int[] A. int x. int £. int r)

| mi

3 | (rekursive Implementierung)

Der initiale Aufruf der rekursiven Variante lautet:
RekBinarySearch (A, z, 1, A.length)
\end{liAntwort}

\end{document}
