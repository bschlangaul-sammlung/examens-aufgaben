\documentclass{bschlangaul-aufgabe}
\bLadePakete{automaten}
\begin{document}
\bAufgabenMetadaten{
  Titel = {Aufgabe 3},
  Thematik = {Minimierung von Endlichen Automaten},
  RelativerPfad = Staatsexamen/46115/2021/03/Thema-1/Teilaufgabe-1/Aufgabe-3.tex,
  ZitatSchluessel = examen:46115:2021:03,
  BearbeitungsStand = unbekannt,
  Korrektheit = unbekannt,
  Stichwoerter = {Minimierungsalgorithmus},
  ExamenNummer = 46115,
  ExamenJahr = 2021,
  ExamenMonat = 03,
  ExamenThemaNr = 1,
  ExamenTeilaufgabeNr = 1,
  ExamenAufgabeNr = 3,
}

Betrachten Sie den unten gezeigten deterministischen endlichen
Automaten, der Worte über dem Alphabet X = {a,b} verarbeitet. Bestimmen
Sie den dazugehörigen Minimalautomaten, d. h. einen deterministischen
endlichen Automaten, der die gleiche Sprache akzeptiert und eine mini-
male Anzahl an Zuständen benutzt. Erläutern Sie Ihre Berechnung, indem
Sie z. B. eine Minimierungstabelle angeben.\index{Minimierungsalgorithmus}
\footcite{examen:46115:2021:03}

\begin{center}
\begin{tikzpicture}[li automat]
  \node[state,initial] (q0) at (3.29cm,-2.71cm) {$q_0$};
  \node[state] (q1) at (6cm,-2.71cm) {$q_1$};
  \node[state,accepting] (q2) at (8.86cm,-2.71cm) {$q_2$};
  \node[state] (q3) at (3.29cm,-5cm) {$q_3$};
  \node[state] (q4) at (6cm,-5.14cm) {$q_4$};
  \node[state,accepting] (q5) at (8.57cm,-5.29cm) {$q_5$};
  \node[state,accepting] (q6) at (11.57cm,-5.14cm) {$q_6$};
  \node[state,accepting] (q7) at (13.86cm,-5.14cm) {$q_7$};

  \path (q0) edge[auto] node{$a$} (q1);
  \path (q0) edge[auto,bend left] node{$b$} (q3);
  \path (q1) edge[auto] node{$b$} (q2);
  \path (q1) edge[auto,bend left] node{$a$} (q4);
  \path (q2) edge[auto,loop above] node{$a$} (q2);
  \path (q2) edge[auto,bend left] node{$b$} (q5);
  \path (q3) edge[auto,bend left] node{$b$} (q0);
  \path (q3) edge[auto,loop above] node{$a$} (q3);
  \path (q4) edge[auto] node{$b$} (q5);
  \path (q4) edge[auto,bend left] node{$a$} (q1);
  \path (q5) edge[auto,bend left] node{$b$} (q6);
  \path (q5) edge[auto,bend left] node{$a$} (q2);
  \path (q6) edge[auto,bend left] node{$a$} (q5);
  \path (q6) edge[auto,loop above] node{$b$} (q6);
  \path (q7) edge[auto,bend left] node{$a$} (q6);
  \path (q7) edge[auto,loop above] node{$b$} (q7);
\end{tikzpicture}
\end{center}
\bFlaci{Ah5v10or9}

\begin{bAntwort}
\begin{center}
\begin{tikzpicture}[li automat]
  \node[state,initial] (q0) at (4cm,-2.43cm) {$q_0$};
  \node[state,accepting] (q2+q5+q6+q7) at (8.43cm,-2.43cm) {q2+q5+q6+q7};
  \node[state] (q3) at (4cm,-4.29cm) {$q_3$};
  \node[state] (q1+q4) at (6.14cm,-2.43cm) {q1+q4};

  \path (q0) edge[auto] node{$a$} (q1+q4);
  \path (q0) edge[auto,bend left] node{$b$} (q3);
  \path (q2+q5+q6+q7) edge[auto,loop above] node{$a,b$} (q2+q5+q6+q7);
  \path (q3) edge[auto,bend left] node{$b$} (q0);
  \path (q3) edge[auto,loop above] node{$a$} (q3);
  \path (q1+q4) edge[auto] node{$b$} (q2+q5+q6+q7);
  \path (q1+q4) edge[auto,loop above] node{$a$} (q1+q4);
\end{tikzpicture}
\end{center}
\bFlaci{Apkyuoo1g}
\end{bAntwort}

\end{document}
