\documentclass{lehramt-informatik-aufgabe}
\liLadePakete{formale-sprachen,automaten}
\begin{document}
\let\m=\liMenge

\liAufgabenTitel{Alphabet ab}
\section{Aufgabe 1
\index{Reguläre Sprache}
\footcite{46115:2010:03}}

Gegeben ist die folgende Sprache L1 über dem Alphabet \liAlphabet{a,b}:

$L1 = \m{w \in \Sigma^* \, | \, \text{die Anzahl der a in w ist gerade und b kommt in w genau einmal vor}}$.

\begin{enumerate}

%%
% a)
%%

\item Geben Sie einen deterministischen endlichen Automaten an, der die
Sprache L1 akzeptiert.

%%
% b)
%%

\item Geben Sie einen regulären Ausdruck an, der die Sprache $L1$
beschreibt.

\begin{liAntwort}
(aba|((aa)*b(aa)*))
\end{liAntwort}

Die folgende Sprache $L2$ ist eine Erweiterung von $L1$:

$L1 = \m{w \in \Sigma^* \, | \, \text{die Anzahl der } a \text{ in } w \text{ ist gerade und die Anzahl der } b \text{ in } w \text{ ist ungerade}}.$

%%
% c)
%%

\item Geben Sie einen deterministischen endlichen Automaten an, der die
Sprache $L2$ akzeptiert.

\begin{liAntwort}
\liFussnoteUrl{https://www.informatik.uni-hamburg.de/TGI/lehre/vl/SS14/FGI1/Folien/fgi1_v2_handout.pdf}

gu = gerade Anzahl a’s, ungerade Anzahl b’s

ug = ungerade Anzahl a’s, gerade Anzahl b’s

\def\z#1{$Z_{#1}$}

\begin{center}
\begin{tikzpicture}[->,node distance=4cm]
\node[state,initial] (gg) {\z{gg}};
\node[state,right of=gg] (ug) {\z{ug}};
\node[state,below of=ug] (uu) {\z{uu}};
\node[state,below of=gg] (gu) {\z{gu}};

\path (gg) edge[bend left,above] node{0} (ug);
\path (ug) edge[bend left,above] node{0} (gg);

\path (gg) edge[bend left,left] node{1} (gu);
\path (gu) edge[bend left,left] node{1} (gg);

\path (gu) edge[bend left,above] node{0} (uu);
\path (uu) edge[bend left,above] node{0} (gu);

\path (ug) edge[bend left,left] node{1} (uu);
\path (uu) edge[bend left,left] node{1} (ug);

\end{tikzpicture}
\end{center}

\liFussnoteUrl{https://flaci.com/Af0vcjys9}
\end{liAntwort}

%%
% d)
%%

\item Geben Sie eine rechtslineare Grammatik an, die die Sprache L2
erzeugt.

\end{enumerate}
\end{document}
