\documentclass{lehramt-informatik-minimal}
\InformatikPakete{graph}
\begin{document}

\section{Frühjahr 2014 (46115) - Thema 1 Aufgabe 8
\index{minimaler Spannbaum}
\index{Algorithmus von Kruskal}
\footcite[Seite 1-2, Aufgabe 2: Spannbäume]{aud:ab:6}}

Bestimmen Sie einen minimalen Spannbaum für einen ungerichteten Graphen,
der durch die nachfolgende Entfernungsmatrix gegeben ist! Die Matrix ist
symmetrisch und $\infty$ bedeutet, dass es keine Kante gibt. Zeichnen
Sie den Graphen und geben Sie die Spannbaumkanten ein
\footcite[Seite 5 (PDF 4)]{examen:46115:2014:03}!

\[
\begin{blockarray}{ccccccccc}
  & A      & B      & C      & D      & E      & F      & G      & H      \\
\begin{block}{c(cccccccc)}
A & 0      & 8      & -1     & \infty & 8      & \infty & 7      & \infty \\
B & 8      & 0      & \infty & 2      & \infty & \infty & \infty & 9      \\
C & -1     & \infty & 0      & 5      & 8      & 1      & 7      & \infty \\
D & \infty & 2      & 5      & 0      & 6      & 6      & \infty & \infty \\
E & 8      & \infty & 8      & 6      & 0      & 6      & 3      & \infty \\
F & \infty & \infty & 1      & 6      & 6      & 0      & 11     & 4      \\
G & 7      & \infty & 7      & \infty & 3      & 11     & 0      & 5      \\
H & \infty & 9      & \infty & \infty & \infty & 4      & 5      & 0      \\
\end{block}
\end{blockarray}
\]

% http://graphonline.ru/en/?graph=JACsZrMiExxkFxBK

% 0,8,-1,0,8,0,7,0
% 8,0,0,2,0,0,0,9
% -1,0,0,5,8,1,7,0
% 0,2,5,0,6,6,0,0
% 8,0,8,6,0,6,3,0
% 0,0,1,6,6,0,11,4
% 7,0,7,0,3,11,0,5
% 0,9,0,0,0,4,5,0

\begin{minipage}{8cm}
\graph knoten {
  \knoten{A}(2,5)
  \knoten{B}(5,6)
  \knoten{C}(5,1)
  \knoten{D}(7,0)
  \knoten{E}(2,0)
  \knoten{F}(5,3)
  \knoten{G}(0,1)
  \knoten{H}(3,4.5)
} kanten {
  \kante(A-B){8}
  \KANTE(A-C){-1}
  \kante(A-E){8}
  \kante(A-G){7}
  \KANTE(B-D){2}
  \kante(B-H){9}
  \KANTE(C-D){5}
  \kante(C-E){8}
  \KANTE(C-F){1}
  \kante(C-G){7}
  \kante(D-E){6}
  \kante(D-F){6}
  \kante(E-F){6}
  \KANTE(E-G){3}
  \kante(F-G){11}
  \KANTE(F-H){4}
  \KANTE(G-H){5}
}
\end{minipage}
\begin{minipage}{4cm}
\begin{tabular}{lr}
Kante & Gewicht \\
\hline
AC & -1 \\
BD & 2 \\
CF & 1 \\
EG & 3 \\
FH & 4 \\
GH & 5 \\
CD & 5 \\\hline
& \textbf{19} \\
\end{tabular}
\end{minipage}

Nach dem Algorithmus von Kruskal wählt man aus den noch nicht gewählten
Kanten immer die kürzeste, die keinen Kreis mit den bisher gewählten
Kanten bildet.
\end{document}
