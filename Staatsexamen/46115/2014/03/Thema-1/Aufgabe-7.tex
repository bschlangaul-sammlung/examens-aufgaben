\documentclass{lehramt-informatik-aufgabe}
\liLadePakete{baum}
\begin{document}
\liAufgabenTitel{Heap und binärer Suchbaum und AVL Baum}
\section{Aufgabe 7 Heap und binärer Suchbaum und AVL Baum
\index{Halde (Heap)}\index{Binärbaum}\index{AVL-Baum}
\footcite{46115:2014:03}}

Fügen Sie nacheinander die Zahlen 11, 1, 2, 13, 9, 10, 7, 5

\begin{liProjektSprache}{Baum}
baum binär (
  setzte: 11 1 2 13 9 10 7 5;
  drucke;
)
\end{liProjektSprache}

\begin{enumerate}

%%
% (i)
%%

\item in einen leeren binären Suchbaum und zeichnen Sie den Suchbaum
jeweils nach dem Einfügen von „9“ und „5“

\begin{liDiagramm}{Nach Einfügen von „9“}
\begin{tikzpicture}[li binaer baum]
\Tree
[.11
  [.1
    \edge[blank]; \node[blank]{};
    [.2
      \edge[blank]; \node[blank]{};
      [.9 ]
    ]
  ]
  [.13 ]
]
\end{tikzpicture}
\end{liDiagramm}

\begin{liDiagramm}{Nach Einfügen von „5“}
\begin{tikzpicture}[li binaer baum]
\Tree
[.11
  [.1
    \edge[blank]; \node[blank]{};
    [.2
      \edge[blank]; \node[blank]{};
      [.9
        [.7
          [.5 ]
          \edge[blank]; \node[blank]{};
        ]
        [.10 ]
      ]
    ]
  ]
  [.13 ]
]
\end{tikzpicture}
\end{liDiagramm}

%%
% (ii)
%%

\item in einen leeren Min-Heap ein, der bzgl. „$\leq$“ angeordnet ist und
geben Sie den Heap nach „9“ und nach „5“ an

%%
% (ii)
%%

\item in einen leeren AVL-Baum ein! Geben Sie den AVL Baum nach „2“ und
„5“ an und beschreiben Sie die ggf. notwendigen Rotationen beim Einfügen
dieser beiden Elemente!
\end{enumerate}

\end{document}
