\documentclass{lehramt-informatik-aufgabe}
\liLadePakete{graph,syntax}
\begin{document}
\liAufgabenTitel{dfs-number}

\section{Aufgabe 8
\index{Tiefensuche}
\footcite[Thema 1 Aufgabe 8]{examen:46115:2014:09}
}

\graph knoten {
  \knoten{a}(8,3)
  \knoten{b}(2,3)
  \knoten{c}(2,1)
  \knoten{d}(0,1)
  \knoten{e}(7,5)
  \knoten{f}(7,2)
  \knoten{g}(5,0)
  \knoten{h}(0,5)
  \knoten{s}(5,3)
} kanten {
  \kanteO(a-e)
  \kanteO(a-f)
  \kanteO(a-s)
  \kanteO(b-c)
  \kanteO(b-d)
  \kanteO(b-h)
  \kanteO(c-d)
  \kanteO(c-h)
  \kanteO(c-s)
  \kanteO(d-h)
  \kanteO(e-f)
  \kanteO(f-s)
  \kanteO(g-s)
  \kanteO(h-s)
}

\begin{enumerate}

%%
% (a)
%%

\item Führen Sie auf dem folgenden ungerichteten Graphen \texttt{G} eine
Tiefensuche ab dem Knoten \texttt{s} aus (graphische Umsetzung).
Unbesuchte Nachbarn eines Knotens sollen dabei in \emph{alphabetischer
Reihenfolge} abgearbeitet werden. Die Tiefensuche soll auf Basis eines
\emph{Stacks}\index{Stapel (Stack)} umgesetzt werden. Geben Sie die
Reihenfolge der besuchten Knoten, also die \texttt{dfs-number} der
Knoten, und den Inhalt des Stacks in jedem Schritt an.
\footcite[Seite 2, Aufgabe 3: Tiefensuche, Breitensuche]{aud:ab:6}

\begin{minted}{java}
public static void main(String[] args) {
  TiefenSucheStapel ts = new TiefenSucheStapel(20);
  ts.fügeKnotenUndKantenEin("a-e a-f a-s b-c b-d b-h c-d c-h c-s d-h e-f f-s g-s h-s");
  ts.gibMatrixAus();
  ts.starteTiefenSucheStapel("s");
}
\end{minted}

\begin{liAntwort}
In der Musterlösung auf Seite 3 lautet das Ergebnis s, a, e, f, c, b, d, h, g.
Ich glaube jedoch diese Lösung ist richtig:

\textbf{fett:} Knoten, der entnommen wird.

\textit{kursiv:} Knoten, die zum Stapel hinzugefügt werden.

\begin{tabular}{|r|r|l|}
\hline
\textbf{Reihenfolge} & \textbf{Stapel} & \textbf{besucht} \\\hline\hline
1 & \textit{\textbf{s}} & s \\\hline
2 & \textit{a, c, f, g, \textbf{h}} & h \\\hline
3 & a, c, f, g, \textit{b, \textbf{d}} & d \\\hline
4 & a, c, f, g, \textbf{b} & b \\\hline
5 & a, c, f, \textbf{g} & g \\\hline
6 & a, c, \textbf{f} & f \\\hline
7 & a, c, \textbf{e} & e \\\hline
8 & a, \textbf{c} & c \\\hline
9 & \textbf{a} & a \\\hline
\end{tabular}
\end{liAntwort}

%%
% (b)
%%

\item Führen Sie nun eine Breitensuche auf dem gegebenen Graphen aus,
diese soll mit einer Queue umgesetzt werden. Als Startknoten wird wieder
$s$ verwendet. Geben Sie auch hier die Reihenfolge der besuchten Knoten
und den Inhalt der Queue bei jedem Schritt an.

\begin{minted}{java}
public static void main(String[] args) {
  BreitenSucheWarteschlange bs = new BreitenSucheWarteschlange(20);
  bs.fügeKnotenUndKantenEin("a-e a-f a-s b-c b-d b-h c-d c-h c-s d-h e-f f-s g-s h-s");
  bs.gibMatrixAus();
  bs.starteBreitenSuche("s");
}
\end{minted}

\begin{liAntwort}
\textbf{fett:} Knoten, der entnommen wird.

\textit{kursiv:} Knoten, die zur Warteschlange hinzugefügt werden.

\begin{tabular}{|r|l|l|}
\hline
\textbf{Reihenfolge} & \textbf{Warteschlange} & \textbf{besucht} \\\hline\hline
1 & \textit{\textbf{s}} & s\\\hline
2 & \textit{\textbf{a}, c, f, g, h} & a\\\hline
3 & \textbf{c}, f, g, h, \textit{e} & c\\\hline
4 & \textbf{f}, g, h, e, \textit{b, d} & f\\\hline
5 & \textbf{g}, h, e, b, d & g\\\hline
6 & \textbf{h}, e, b, d & h\\\hline
7 & \textbf{e}, b, d & e\\\hline
8 & \textbf{b}, d & b\\\hline
9 & \textbf{d} & d\\\hline
\end{tabular}
\end{liAntwort}

%%
% (c)
%%

\item Geben Sie in Pseudocode den Ablauf von Tiefen- und Breitensuche
an, wenn diese wie beschrieben mit einem Stack bzw. einer Queue
implementiert werden.

\end{enumerate}
\end{document}
