\documentclass{lehramt-informatik-aufgabe}
\liLadePakete{automaten}
\begin{document}
\liAufgabenTitel{Reguläre Sprache}
\section{Aufgabe 1
\index{Reguläre Sprache}
\footcite{46115:2020:03}}

\begin{enumerate}

%%
% (a)
%%

\item Betrachten Sie die formale Sprache $L \subseteq \{ 0, 1 \}^*$
aller Wörter, die $01$ oder $110$ als Teilwort enthalten.

Geben Sie einen regulären Ausdruck für die Sprache $L$ an.
\index{Regulärer Ausdruck}

\begin{liAntwort}
(0|1)*(01|110)(0|1)*
\end{liAntwort}

%%
% (b)
%%

\item Entwerfen Sie einen (vollständigen) deterministischen endlichen
Automaten, der die Sprache $L$ aus Teilaufgabe (a) akzeptiert. (Hinweis:
es werden nicht mehr als 6 Zustände benötigt.)

\begin{liAntwort}
\begin{liAntwort}
\begin{center}
\begin{tikzpicture}[->,node distance=2cm]
\node[state,initial] (0) {$z_0$};
\node[state,below of=0] (1) {$z_1$};
\node[state,right of=0] (2) {$z_2$};
\node[state,right of=2] (3) {$z_3$};
\node[state,below of=3,accepting] (4) {$z_4$};

\path (0) edge[left] node{0} (1);
\path (0) edge[above] node{1} (2);
\path (1) edge[above,loop left] node{1} (1);
\path (1) edge[above] node{1} (4);
\path (2) edge[above] node{0} (1);
\path (2) edge[above] node{1} (3);
\path (3) edge[above,loop right] node{1} (3);
\path (3) edge[right] node{0} (4);
\path (4) edge[above,loop right] node{0,1} (4);
\end{tikzpicture}
\end{center}
\end{liAntwort}

\liFussnoteUrl{https://flaci.com/A54gek0vz}
\end{liAntwort}

%%
% (c)
%%

\item Minimieren Sie den folgenden deterministischen endlichen Automaten:

Machen Sie dabei Ihren Rechenweg deutlich!

%%
% (d)
%%

\item Ist die folgende Aussage richtig oder falsch? Begründen Sie Ihre
Antwort!

„Zu jeder regulären Sprache $L$ über dem Alphabet $\Sigma$ gibt es eine
Sprache $L' \subseteq \Sigma^*$, die $L$ enthält (d.\,h. $L \subseteq
L’$) und nicht regulär ist.“
\end{enumerate}
\end{document}
