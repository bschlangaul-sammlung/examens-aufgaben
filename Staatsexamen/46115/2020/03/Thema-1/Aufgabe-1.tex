\documentclass{lehramt-informatik-aufgabe}
\liLadePakete{automaten}
\begin{document}
\liAufgabenTitel{Reguläre Sprache}
\section{Aufgabe 1
\index{Reguläre Sprache}
\footcite{46115:2020:03}}

\begin{enumerate}

%%
% (a)
%%

\item Betrachten Sie die formale Sprache $L \subseteq \{ 0, 1 \}^*$
aller Wörter, die $01$ oder $110$ als Teilwort enthalten.

Geben Sie einen regulären Ausdruck für die Sprache $L$ an.
\index{Reguläre Ausdrücke}

\begin{liAntwort}
(0|1)*(01|110)(0|1)*
\end{liAntwort}

%%
% (b)
%%

\item Entwerfen Sie einen (vollständigen) deterministischen endlichen
Automaten, der die Sprache $L$ aus Teilaufgabe (a) akzeptiert. (Hinweis:
es werden nicht mehr als 6 Zustände benötigt.)

\begin{liAntwort}
\begin{center}
\begin{tikzpicture}[->,node distance=2cm]
\node[state,initial] (0) {$z_0$};
\node[state,below of=0] (1) {$z_1$};
\node[state,right of=0] (2) {$z_2$};
\node[state,right of=2] (3) {$z_3$};
\node[state,below of=3,accepting] (4) {$z_4$};

\path (0) edge[left] node{0} (1);
\path (0) edge[above] node{1} (2);
\path (1) edge[above,loop left] node{1} (1);
\path (1) edge[above] node{1} (4);
\path (2) edge[above] node{0} (1);
\path (2) edge[above] node{1} (3);
\path (3) edge[above,loop right] node{1} (3);
\path (3) edge[right] node{0} (4);
\path (4) edge[above,loop right] node{0,1} (4);
\end{tikzpicture}
\end{center}

\liFussnoteUrl{https://flaci.com/A54gek0vz}
\end{liAntwort}

%%
% (c)
%%

\item Minimieren Sie den folgenden deterministischen endlichen Automaten:

Machen Sie dabei Ihren Rechenweg deutlich!

\begin{liAntwort}
\begin{center}
\begin{tikzpicture}[->,node distance=1.5cm]
\node[state] (a) {a};
\node[state,below=of a,initial] (b) {b};
\node[state,right=of a,accepting] (c) {c};
\node[state,right=of b,accepting] (d) {d};
\node[state,right=of c] (e) {e};
\node[state,right=of d] (f) {f};
\node[state,right=of e,accepting] (g) {g};
\node[state,right=of f,accepting] (h) {h};

\path (a) edge[above] node{0} (c);
\path (a) edge[right,pos=0.2] node{1} (d);
\path (b) edge[left,pos=0.2] node{0} (c);
\path (b) edge[above] node{1s} (d);
\path (c) edge[right,bend left] node{0} (d);
\path (c) edge[above] node{1} (e);
\path (c) edge[above,pos=0.2] node{1} (f);
\path (d) edge[left,bend left] node{0} (c);
\path (d) edge[below,pos=0.2] node{1} (e);
\path (e) edge[above,pos=0.2] node{0} (h);
\path (f) edge[above,pos=0.2] node{0} (g);
\path (f) edge[above] node{1} (d);
\path (g) edge[right,bend left] node{1} (h);
\path (g) edge[above] node{0} (e);
\path (h) edge[left,bend left] node{1} (g);
\path (h) edge[above] node{0} (f);
\end{tikzpicture}
\end{center}
\liFussnoteUrl{https://flaci.com/A54gek0vz}
\end{liAntwort}

%%
% (d)
%%

\item Ist die folgende Aussage richtig oder falsch? Begründen Sie Ihre
Antwort!

„Zu jeder regulären Sprache $L$ über dem Alphabet $\Sigma$ gibt es eine
Sprache $L' \subseteq \Sigma^*$, die $L$ enthält (d.\,h. $L \subseteq
L’$) und nicht regulär ist.“
\end{enumerate}
\end{document}
