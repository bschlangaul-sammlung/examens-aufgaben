\documentclass{lehramt-informatik-aufgabe}
\liLadePakete{}
\begin{document}
\liAufgabenTitel{Nächstes rot-blaues Paar auf der x-Achse}
\section{Aufgabe 6
\index{Komplexität}
\footcite{46115:2020:03}}

Gegeben seien zwei nichtleere Mengen R und B von roten bzw. blauen
Punkten auf der x-Achse. Gesucht ist der minimale euklidische Abstand
d(r, b) über alle Punktepaare (r,b) mitr € R und be B. Hier ist eine
Beispielinstanz:

Die Eingabe wird in einem Feld A übergeben. Jeder Punkt Al) mit 1<i<n
hat eine r- Koordinate Ali].x und eine Farbe Alil.color € { rot, blau }.
Das Feld A ist nach x-Koordinate sortiert, \dh es gilt A[1].x < Al2].x
< --- < Alnl.x, wobein = |R| + |B|.

\begin{enumerate}

%%
% (a)
%%

\item Geben Sie in Worten einen Algorithmus an, der den gesuchten
Abstand in O(n) Zeit berechnet.

%%
% (b)
%%

\item Begründen Sie kurz die Laufzeit Ihres Algorithmus.

%%
% (c)
%%

\item Begründen Sie die Korrektheit Ihres Algorithmus.

%%
% (d)
%%

\item Wir betrachten nun den Spezialfall, dass alle blauen Punkte links
von allen roten Punkten liegen. Beschreiben Sie in Worten, wie man in
dieser Situation den gesuchten Abstand in o(n) Zeit berechnen kann. (Ihr
Algorithmus darf also insbesondere nicht Laufzeit O(n) haben.)

\end{enumerate}
\end{document}
