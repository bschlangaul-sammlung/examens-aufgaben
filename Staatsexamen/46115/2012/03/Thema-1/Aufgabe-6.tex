\documentclass{lehramt-informatik-aufgabe}
\liLadePakete{graph}
\begin{document}
\liAufgabenTitel{}
\section{Aufgabe 6
\index{Algorithmus von Dijkstra}
\footcite{46115:2012:03}}

Gegeben sei der folgende gerichtete Graph $G = (V, E, d)$ mit den
angegebenen Kantengewichten.

\begin{liGraphenFormat}
A: 0 1
B: 1 2
C: 2 2
D: 3 1
E: 1 0
F: 2 0
A -> B: 1
A -> E: 7
B -> C: 3
C -> D: 6
C -> E: 3
C -> F: 5
E -> F: 1
F -> C: 1
F -> D: 2
\end{liGraphenFormat}

\begin{center}
\begin{tikzpicture}[li graph,x=2cm,y=1.5cm]
\node (A) at (0,1) {A};
\node (B) at (1,2) {B};
\node (C) at (2,2) {C};
\node (D) at (3,1) {D};
\node (E) at (1,0) {E};
\node (F) at (2,0) {F};

\path[->] (A) edge node {} (B);
\path[->] (A) edge node {7} (E);
\path[->] (B) edge node {3} (C);
\path[->] (C) edge node {3} (E);
\path[->] (C) edge[bend left] node {5} (F);
\path[->] (C) edge node {6} (D);
\path[->] (E) edge node {} (F);
\path[->] (F) edge[bend left] node {} (C);
\path[->] (F) edge node {2} (D);
\end{tikzpicture}
\end{center}

\begin{enumerate}

%%
% a)
%%

\item Geben Sie eine formale Beschreibung des abgebildeten Graphen $G$
durch Auflistung von $V$, $E$ und $d$ an.
\index{Adjazenzliste}

\begin{liAntwort}
\begin{tabular}{llll}
A & $\rightarrow$ B & $\rightarrow$ E \\
B & $\rightarrow$ C \\
C & $\rightarrow$ D & $\rightarrow$ E & $\rightarrow$ F \\
D \\
E & $\rightarrow$ F \\
F & $\rightarrow$ C & $\rightarrow$ D \\
\end{tabular}
\end{liAntwort}

%%
% b)
%%

\item Erstellen Sie die Adjazenzmatrix $A$ zum Graphen $G$.
\index{Adjazenzmatrix}

\begin{liAntwort}
\[
\begin{blockarray}{ccccccc}
   & A & B & C & D & E & F \\
\begin{block}{c(cccccc)}
 A & * & 1 & - & - & 7 & - \\
 B & - & * & 3 & - & - & - \\
 C & - & - & * & 6 & 3 & 5 \\
 D & - & - & - & * & - & - \\
 E & - & - & - & - & * & 1 \\
 F & - & - & 1 & 2 & - & * \\
\end{block}
\end{blockarray}
\]

\end{liAntwort}

%%
% c)
%%

\item Berechnen Sie unter Verwendung des Algorithmus nach Dijkstra -
vom Knoten $A$ beginnend - den kürzesten Weg, um alle Knoten zu besuchen.
Die Restknoten werden in einer Halde (engl. Heap) gespeichert. Geben Sie
zu jedem Arbeitsschritt den Inhalt dieser Halde an.
\index{Halde (Heap)}

\end{enumerate}
\end{document}
