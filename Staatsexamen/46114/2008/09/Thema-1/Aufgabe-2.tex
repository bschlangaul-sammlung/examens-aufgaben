\documentclass{lehramt-informatik-aufgabe}
\liLadePakete{graph}
\begin{document}
\liAufgabenTitel{Dijkstra}
\section{Aufgabe 2
\index{Algorithmus von Dijkstra}
\footcite{46114:2008:09}}

Gegeben sei folgender Graph:

\begin{tabbing}
\hspace{1cm} \= \hspace{3cm} \kill
V: \> \{a, b, c, d, e\} \\
E: \> a $\rightarrow$ a, b \\
\> b $\rightarrow$ b, d, e \\
\> c $\rightarrow$ c, d \\
\> d $\rightarrow$ a, e \\
\end{tabbing}

\begin{liEinfachesGraphenFormat}
a: 0 0
b: 1 1
c: 4 1
d: 3 0
e: 2 2
a -- b
b -- d
b -- e
c -- d
d -- e
d -- a
\end{liEinfachesGraphenFormat}

\begin{center}
\begin{tikzpicture}[li graph]
\node (a) at (0,0) {a};
\node (b) at (1,1) {b};
\node (c) at (4,1) {c};
\node (d) at (3,0) {d};
\node (e) at (2,2) {e};

\path[li graph kante] (a) edge (b);
\path[li graph kante] (b) edge (d);
\path[li graph kante] (b) edge (e);
\path[li graph kante] (c) edge (d);
\path[li graph kante] (d) edge (a);
\path[li graph kante] (d) edge (e);
\end{tikzpicture}
\end{center}

\begin{enumerate}

%%
% a)
%%

\item Stellen Sie den Graphen grafisch dar!

%%
% b)
%%

\item Berechnen Sie mit dem Algorithmus von Dijkstra schrittweise die Länge der kürzesten
Pfade ab dem Knoten a! Nehmen Sie dazu an, dass alle Kantengewichte 1 sind. Erstellen
Sie eine Tabelle gemäß folgendem Muster:

ausgewählt |a| b c d e

Ergebnis:

Hinweis: Nur mit Angabe der jeweiligen Zwischenschritte gibt es Punkte. Es reicht also
nicht, nur das Endergebnis hinzuschreiben.

%%
% c)
%%

\item Welchen Aufwand hat der Algorithmus von Dijkstra bei Graphen mit
|V| Knoten und |E| Kanten,

\begin{itemize}
\item wenn die Kantengewichte alle 1 sind? Mit welcher Datenstruktur und
welchem Vorgehen lässt sich der Aufwand in diesem Fall reduzieren (mit
kurzer Begründung)?

\item wenn die Kantengewichte beliebig sind und als Datenstruktur eine
Halde verwendet wird (mit kurzer Begründung)?
\end{itemize}

\end{enumerate}

\end{document}
