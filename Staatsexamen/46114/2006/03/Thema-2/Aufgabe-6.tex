\documentclass{lehramt-informatik-aufgabe}
\liLadePakete{graph}
\begin{document}
\liAufgabenTitel{Adjazenzmatrix und Adjazenzliste}
\section{Aufgabe 6 (Graphrepräsentation)
\index{Graphen}
\footcite{46114:2006:03}}

Repräsentieren Sie den folgenden Graphen sowohl mit einer
Adjazenzmatrix\index{Adjazenzmatrix} als auch mit einer
Adjazenzliste\index{Adjazenzmatrix}.

\begin{liEinfachesGraphenFormat}
A: 1 1
B: 1 -1
C: 2 1
D: 2 -1
E: 0 0
F: 3 0

A -> E
B -> A
E -> B
D -> A
A -> C
C -> D
D -> F
F -> C
\end{liEinfachesGraphenFormat}

\begin{tikzpicture}[x=1.5cm]
\node[li graph knoten] (A) at (1,1) {A};
\node[li graph knoten] (B) at (1,-1) {B};
\node[li graph knoten] (C) at (2,1) {C};
\node[li graph knoten] (D) at (2,-1) {D};
\node[li graph knoten] (E) at (0,0) {E};
\node[li graph knoten] (F) at (3,0) {F};

\path[li graph kante,->] (A) edge (C);
\path[li graph kante,->] (A) edge (E);
\path[li graph kante,->] (B) edge (A);
\path[li graph kante,->] (C) edge (D);
\path[li graph kante,->] (D) edge (A);
\path[li graph kante,->] (D) edge (F);
\path[li graph kante,->] (E) edge (B);
\path[li graph kante,->] (F) edge (C);
\end{tikzpicture}

\begin{liAntwort}
\[
\begin{blockarray}{cccccc}
   & A & B & C & D & E & F \\
\begin{block}{c(ccccc)}
 A & 0 & 0 & 1 & 0 & 1 & 0 \\
 B & 1 & 0 & 0 & 0 & 0 & 0 \\
 C & 0 & 0 & 0 & 1 & 0 & 0 \\
 D & 1 & 0 & 0 & 0 & 0 & 1 \\
 E & 0 & 1 & 0 & 0 & 0 & 0 \\
 F & 0 & 0 & 1 & 0 & 0 & 0 \\
\end{block}
\end{blockarray}
\]

\begin{tabular}{lll}
A & $\rightarrow$ C & $\rightarrow$ E \\
B & $\rightarrow$ A \\
C & $\rightarrow$ D \\
D & $\rightarrow$ A & $\rightarrow$ F \\
E & $\rightarrow$ B \\
F & $\rightarrow$ C \\
\end{tabular}
\end{liAntwort}

\end{document}
