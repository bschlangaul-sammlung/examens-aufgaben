\documentclass{lehramt-informatik-aufgabe}
\liLadePakete{syntax}
\begin{document}

\section{Aufgabe 2: Suchbaum
\index{Binärbaum}
\footcite[Seite 1]{aud:pu:5}}

\footcite[Abgewandelt von Herbst 2003, Thema 2 Aufgabe 8, Seite 8]{examen:66112:2003:09}

\begin{enumerate}

%%
% (a)
%%

\item Implementieren Sie in einer objektorientierten Sprache einen
binären Suchbaum für ganze Zahlen! Dazu gehören Methoden zum Setzen und
Ausgeben der Attribute \java{zahl}, \java{linker_teilbaum} und
\java{rechter_teilbaum}. Design: eine Klasse \java{Knoten} und eine
Klasse \java{BinBaum}. Ein Knoten hat einen linken und einen rechten
Nachfolger. Ein Baum verwaltet die Wurzel. Er hängt neue Knoten an und
löscht Knoten.

%%
% (b)
%%

\item Schreiben Sie eine Methode \java{fuegeEin(...)}, die eine Zahl in
den Baum einfügt!

%%
% (c)
%%

\item Schreiben Sie eine Methode \java{postOrder(...)}, die die Zahlen
in der Reihenfolge postorder ausgibt!

%%
% (d)
%%

\item Ergänzen Sie Ihr Programm um die rekursiv implementierte Methode
\java{summe(...)}, die die Summe der Zahlen des Unterbaums, dessen
Wurzel der Knoten ist, zurückgibt! Falls der Unterbaum leer ist, ist der
Rückgabewert 0!

\begin{minted}{java}
int summe (Knoten x)...
\end{minted}

%%
% (e)
%%

\item Schreiben Sie eine Folge von Anweisungen, die einen Baum mit Namen
BinBaum erzeugt und nacheinander die Zahlen 5 und 7 einfügt! In den
binären Suchbaum werden noch die Zahlen 4, 11, 6 und 2 eingefügt.
Zeichnen Sie den Baum, den Sie danach erhalten haben, und schreiben Sie
die eingefügten Zahlen in der Reihenfolge der Traversierungsmöglichkeit
\texttt{postorder} auf!

%%
% (f)
%%

\item Implementieren Sie eine Operation \java{isSorted(...)}, die für
einen (Teil-)baum feststellt, ob er sortiert ist.
\end{enumerate}

\end{document}
