\documentclass{lehramt-informatik-aufgabe}
\liLadePakete{syntax,mathe}
\begin{document}

\section{Aufgabe 5
\index{Vollständige Induktion}
\footcite[Thema 2 Aufgabe 5]{examen:66112:2003:09}}

5. a) Zeigen Sie mit Hilfe vollständiger Induktion, dass das folgende
Programm bzgl. der Vorbedingung $x > 0$ und der Nachbedingung drei\_hoch
$x =  3^x$ partiell korrekt ist!

\begin{minted}{lisp}
(define (drei_hoch x)
  (cond ((= x 0) 1)
    (else (* 3 (drei_hoch (- x 1))))
  )
)
\end{minted}

\begin{antwort}
% IA: x=1
% drei_hoch 1 = 3*(drei_hoch 0) = 3*1=3 = 3 1
% IV: für alle x < x 0 gilt drei_hoch x = 3 x
% IS: x->x+1

\begin{align*}
\text{drei\_hoch} (x + 1)
& = 3 \cdot (\text{drei\_hoch} (- (x + 1) 1))\\
& = 3 \cdot (\text{drei\_hoch} x)\\
& = 3 \cdot 3^x\\
& = 3^{x+1}
\end{align*}
\end{antwort}
\end{document}
