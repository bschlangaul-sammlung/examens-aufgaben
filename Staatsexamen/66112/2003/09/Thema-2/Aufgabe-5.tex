\documentclass{lehramt-informatik-aufgabe}
\liLadePakete{vollstaendige-induktion}
\begin{document}
\liAufgabenTitel{drei hoch}

\section{Aufgabe 5
\index{Vollständige Induktion}
\footcite[Thema 2 Aufgabe 5]{examen:66112:2003:09}}

5. a) Zeigen Sie mit Hilfe vollständiger Induktion, dass das folgende
Programm bzgl. der Vorbedingung $x > 0$ und der Nachbedingung drei\_hoch
$x =  3^x$ partiell korrekt ist!

\begin{minted}{lisp}
(define (drei_hoch x)
  (cond ((= x 0) 1)
    (else (* 3 (drei_hoch (- x 1))))
  )
)
\end{minted}

\def\drei{\text{drei\_hoch}\,}

\begin{liAntwort}

%%
%
%%

\liInduktionAnfang

$\drei 1 = 3 \cdot (\drei 0) = 3 \cdot 1 = 3$

%%
%
%%

\liInduktionVoraussetzung

für alle $x < x 0$ gilt $\drei x = 3 x$

%%
%
%%

\liInduktionSchritt

x->x+1

\begin{align*}
\drei (x + 1)
& = 3 \cdot \drei (- (x + 1) 1))\\
& = 3 \cdot (\drei x)\\
& = 3 \cdot 3^x\\
& = 3^{x+1}
\end{align*}
\end{liAntwort}
\end{document}
