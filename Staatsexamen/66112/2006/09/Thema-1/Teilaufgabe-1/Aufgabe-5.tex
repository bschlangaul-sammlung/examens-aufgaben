\documentclass{lehramt-informatik-minimal}
\InformatikPakete{}
\begin{document}

\section{Systementwurf\footcite[Thema 1 Aufgabe 5]{examen:66112:2006:09}}

\begin{enumerate}

%%
% a)
%%

\item Erklären Sie den Begriff Vererbung und benennen Sie die damit
verbundenen Vorteile!

%%
% b)
%%

\item Erstellen Sie zu der folgenden Beschreibung eines Systems zur
Buchung von Flügen ein Klassendiagramm, das neben Attributen und
Assoziationen mit Kardinalitäten auch Methoden zur Tarifberechnung
enthält! Setzen Sie dabei das Konzept der Vererbung sinnvoll ein!

\begin{itemize}
\item Die Fluggesellschaft bietet verschiedene Flugrouten an, die durch
den jeweiligen Startflughafen und Zielflughafen charakterisiert werden.

\item Jeder Flug besitzt eine Flugnummer, eine Abflugzeit, eine geplante
Ankunftszeit und ist genau einer Flugroute zugeordnet. Flugrouten sollen
auch gespeichert werden, falls noch keine zugehörigen Flüge existieren.

\item Flugbuchungen beziehen sich auf einzelne Plätze im Flugzeug.
Sowohl in der Economy Class als auch in der Business Class gibt es
Nichtraucher- und Raucherplätze. Zu jeder Buchung wird das Datum
vermerkt.

\item Zu jedem Passagier müssen die Adressinformationen erfasst werden.

\item Die Berechnung des Tarifs soll vom System unterstützt werden.
Jeder Flug besitzt einen Grundpreis. Für Plätze der Business Class wird
ein Aufschlag verrechnet. Auf diesen ermittelten Zwischenpreis sind zwei
Arten von Rabatten möglich:

\begin{itemize}
\item Jugendliche Privatkunden unter 25 Jahren erhalten einen Nachlass
auf den Flugpreis.

\item Geschäftsreisende erhalten Vergünstigungen in Abhängigkeit ihrer
gesammelten Flugmeilen.
\end{itemize}
\end{itemize}

%%
% c)
%%

\item Erstellen Sie ein exemplarisches Objektdiagramm! Es soll
mindestens einen Flug enthalten, in dem sowohl ein privater Kunde als
auch ein Geschäftskunde einen Platz gebucht haben! Wählen Sie geeignete
Attributwerte!

%%
% d)
%%

\item Beschreiben Sie den Vorgang „Tarifberechnung“ wahlweise als
Sequenzdiagramm oder Kommunikationsdiagramm!

\end{enumerate}
\end{document}
