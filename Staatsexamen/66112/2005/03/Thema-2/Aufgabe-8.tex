\documentclass{lehramt-informatik-minimal}
\InformatikPakete{mathe}
\begin{document}

\section{Aufgabe 3: Hashing
\index{Streutabellen (Hashing)}
\footcite[Seite 2]{aud:pu:5}
}

Gegeben seien die folgenden Zahlen: 7, 4, 3, 5, 0, 1\footcite[Frühjahr
2005, Thema 2 A 8]{examen:66112:2005:03}

\begin{enumerate}

%%
% a)
%%

\item Zeichnen Sie eine Hash-Tabelle mit 8 Zellen und tragen Sie diese
Zahlen genau in der oben gegebenen Reihenfolge in Ihre Hash-Tabelle ein.
Verwenden Sie dabei die Streufunktion $f(n) = n^2 \mod 7$ und eine
Kollisionsauflösung durch lineares Sondieren.

\begin{antwort}
{
\footnotesize
$f(7) = 7^2 \mod 7 = 49 \mod 7 = 0$

$f(4) = 4^2 \mod 7 = 16 \mod 7 = 2$

$f(3) = 3^2 \mod 7 = 9 \mod 7 = 2$ lineares Sondieren: $+1 = 3$

$f(5) = 5^2 \mod 7 = 25 \mod 7 = 4$

$f(0) = 0^2 \mod 7 = 0 \mod 7 = 0$ lineares Sondieren: $+1 = 1$

$f(1) = 1^2 \mod 7 = 1 \mod 7 = 1$ lineares Sondieren: $-1 = 0$, $-1 = 7$
}

\begin{tabular}{|c|c|c|c|c|c|c|c|}
\hline
0 & 1 & 2 & 3 & 4 & 5 & 6 & 7 \\\hline
7 & 0 & 4 & 3 & 5 &   &   & 1 \\\hline
\end{tabular}

\end{antwort}

%%
% b)
%%

\item Welcher Belegungsfaktor ist für die Streutabelle und die
Streufunktion aus Teilaufgabe a zu erwarten, wenn sehr viele Zahlen
eingeordnet werden und eine Kollisionsauflösung durch Verkettung
(verzeigerte Listen) verwendet wird? Begründen Sie Ihre Antwort kurz.

\end{enumerate}
\end{document}
