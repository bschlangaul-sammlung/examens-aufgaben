\documentclass{bschlangaul-aufgabe}

\begin{document}
\bAufgabenTitel{}
\section{Aufgabe 3
\index{Berechenbarkeit}
\footcite{66112:2004:09}}

Ist die Funktion f : {0, 1} * → {0, 1} * mit f (w) = def { 1, 0 sonst
entscheidbar?

falls die Anzahl der Einsen in wdurch 3 teilbar ist

(b) Hierfür lässt sich sogar ein deterministischer endlicher Automat einfach angeben:
0
q 0
1
0
1
q 1
1
q 2
0
Analog dazu lässt sich auch eine Turing-Maschine M = (Z, Σ, Γ, δ, q 0 , 2, F ) konstruie-
ren:
δ :
q 0
q 1
q 2
q 3
0
(q 0 , 0, R) (q 1 , 0, R) (q 2 , 0, R) ∅
1
(q 1 , 0, R) (q 2 , 1, R) (q 0 , 1, R) ∅
2 (q 3 , 1, N ) (q 3 , 0, N ) (q 3 , 0, N ) ∅
Z = {q 0 , q 1 , q 2 , q 3 }, Σ = {1, 0}, Γ = {0, 1, 2}, F = {q 3 }
Auf dem Feld, auf das der Schreib-/Lesekopf am Ende zeigt, steht der gewünschte Aus-
gabewert.\footcite[Aufgabe 7b)]{theo:ab:4}
\end{document}
