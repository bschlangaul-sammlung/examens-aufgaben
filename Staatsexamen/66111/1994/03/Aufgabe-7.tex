\documentclass{lehramt-informatik-aufgabe}
\liLadePakete{mathe,normalformen,synthese-algorithmus}
\begin{document}
\let\ah=\liAttributHuelleOhneMathe
\let\m=\liAttributMenge
\let\FA=\liFunktionaleAbhaengigkeiten
\let\fa=\liFunktionaleAbhaengigkeit
\let\schrittE=\liSyntheseUeberErklaerung

\liAufgabenTitel{Studentenbibliothek}

\section{Staatsexamen Frühjahr 1994 - Aufgabe 7
\index{Synthese-Algorithmus}
\footcite[Aufgabe 7 Seite 4]{examen:66111:1994:03}
}

Betrachten Sie das relationale Schema
\footcite[Seite 1, Aufgabe 2]{db:ab:6}

\begin{center}
\liRelation[R]{Signatur, Titel, Fachgebiet, Art, ErschOrt, MatrNr,
StudName, Gebdatum, StudWohnort, StudFachrichtung, AutNr, AutName,
AutWohnort, AutBuchHonorar}
\end{center}

\noindent
und die Menge\bigskip

\FA{Signatur -> Titel, Fachgebiet, Art, ErschOrt;
Signatur -> MatrNr;
MatrNr -> StudName, Gebdatum, StudWohnort, StudFachrichtung;
AutNr -> AutName, AutWohnort;
AutNr, Signatur -> AutBuchHonorar}

\bigskip

\noindent
Geben Sie eine abhängigkeitserhaltende und verlustfreie Zerlegung von R
in 3. Normalform an!
\index{Dritte Normalform}

\begin{liAntwort}
\begin{enumerate}
\item \schrittE{1-1}

\ah{F, \m{Autnr}} = \m{Autnr, AutName, AutWohnort}

\ah{F, \m{Signatur}} =
\m{Signatur, Titel, Fachgebiet, Art, ErschOrt, MatrNr, StudName, Gebdatum, StudWohnort, StudFachrichtung}

\item \schrittE{1-2}

\ah{F - \m{\fa{Signatur -> MatrNr}}, \m{Signatur}} = \m{Signatur, Titel, Fachgebiet, Art, ErschOrt}

Es kann nichts weggelassen werden

\end{enumerate}

\end{liAntwort}
\end{document}
