\documentclass{lehramt-informatik-aufgabe}
\liLadePakete{mathe}
\begin{document}
\liAufgabenTitel{Studentenbibliothek}

\section{Staatsexamen Frühjahr 1994 - Aufgabe 7
\index{Synthese-Algorithmus}
\footcite[Aufgabe 7 Seite 4]{examen:66111:1994:03}
}

Betrachten Sie das relationale Schema
\footcite[Seite 1, Aufgabe 2]{db:ab:6}

$R(Signatur, Titel, Fachgebiet, Art, ErschOrt, MatrNr, StudName, Gebdatum,
StudWohnort, StudFachrichtung, AutNr, AutName, AutWohnort, AutBuchHonorar)$

und die Menge

\begin{multline}
F = \{\\
Signatur \rightarrow Titel, Fachgebiet, Art, ErschOrt, \\
Signatur \rightarrow MatrNr, \\
MatrNr \rightarrow StudName, Gebdatum, StudWohnort, StudFachrichtung,\\
AutNr \rightarrow AutName, AutWohnort, \\
AutNr, Signatur \rightarrow AutBuchHonorar \\
\}
\end{multline}

Geben Sie eine abhängigkeitserhaltende und verlustfreie Zerlegung von R
in 3. Normalform an!
\index{Dritte Normalform}

\begin{liAntwort}
\begin{enumerate}
\item Linksreduktion

$AttrHull(F, \{Autnr\}) = \{ Autnr, AutName, AutWohnort \}$

$AttrHull(F, \{Signatur\}) = \{ Signatur, Titel, Fachgebiet, Art, ErschOrt, MatrNr, StudName, Gebdatum, StudWohnort, StudFachrichtung, \}$

\item Rechtsreduktion

$AttrHull(F - \{Signatur \rightarrow MatrNr\}, \{Signatur\}) = \{ Signatur, Titel, Fachgebiet, Art, ErschOrt \}$ Es kann nichts weggelassen werden

\end{enumerate}

\end{liAntwort}
\end{document}
