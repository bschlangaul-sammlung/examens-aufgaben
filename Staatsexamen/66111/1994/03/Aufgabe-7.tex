\documentclass{lehramt-informatik-minimal}
\InformatikPakete{mathe}
\begin{document}

\section{Staatsexamen Frühjahr 1994 - Aufgabe 7\footcite[Seite 1, Aufgabe 2]{db:ab:6}}

Betrachten Sie das relationale Schema
\footcite[Seite 4, Staatsexamen Frühjahr 1994, Aufgabe 7]{examen:66111:1994:03}

$R(Signatur, Titel, Fachgebiet, Art, ErschOrt, MatrNr, StudName, Gebdatum,
StudWohnort, StudFachrichtung, AutNr, AutName, AutWohnort, AutBuchHonorar)$

und die Menge

\begin{multline}
F = \{\\
Signatur \rightarrow Titel, Fachgebiet, Art, ErschOrt, \\
Signatur \rightarrow MatrNr, \\
MatrNr \rightarrow StudName, Gebdatum, StudWohnort, StudFachrichtung,\\
AutNr \rightarrow AutName, AutWohnort, \\
AutNr, Signatur \rightarrow AutBuchHonorar \\
\}
\end{multline}

Geben Sie eine abhängigkeitserhaltende und verlustfreie Zerlegung von R
in 3. Normalform an!

\begin{antwort}

\begin{enumerate}
\item Linksreduktion

$AttrHull(F, \{Autnr\}) = \{ Autnr, AutName, AutWohnort \}$

$AttrHull(F, \{Signatur\}) = \{ Signatur, Titel, Fachgebiet, Art, ErschOrt, MatrNr, StudName, Gebdatum, StudWohnort, StudFachrichtung, \}$

\item Rechtsreduktion

$AttrHull(F - \{Signatur \rightarrow MatrNr\}, \{Signatur\}) = \{ Signatur, Titel, Fachgebiet, Art, ErschOrt \}$ Es kann nichts weggelassen werden

\end{enumerate}

\end{antwort}
\end{document}
