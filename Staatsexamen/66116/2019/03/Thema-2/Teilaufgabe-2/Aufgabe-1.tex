\documentclass{lehramt-informatik-aufgabe}
\liLadePakete{}
\begin{document}
\liAufgabenTitel{Test-getriebene Entwicklung}
\section{Aufgabe 1
\index{Testen}
\footcite{66116:2019:03}}

Bewerten Sie die folgenden Aussagen und nehmen Sie jeweils Stellung.

\begin{enumerate}
%%
% a)
%%

\item Tests zuerst

In test-getriebener Softwareentwicklung wird der Test vor der zu
testenden Funktion programmiert.

\begin{liAntwort}
Bei der testgetriebenen Entwicklung erstellt der Programmierer
Softwaretests konsequent vor den zu testenden Komponenten.
\liFussnoteUrl{https://de.wikipedia.org/wiki/Testgetriebene_Entwicklung}
\end{liAntwort}

%%
% b)
%%

\item Komponententests

Komponententests sind immer White-Box-Tests und nie Black-Box-Tests.

\begin{liAntwort}
Falsch: Komponententests (anderes Wort für Unit- oder Modul-Tests) können beides sein.
\liFussnoteUrl{https://qastack.com.de/software/362746/what-is-black-box-unit-testing}
\end{liAntwort}

%%
% c)
%%

\item Akzeptanztests

Akzeptanz- resp. Funktionstests sind immer Black-Box-Tests und nie
White-Box-Tests.

\begin{liAntwort}
In systems engineering, it may involve black-box testing performed on a system

Acceptance testing in extreme programming:  Acceptance tests are black-box system tests.

\liFussnoteUrl{https://en.wikipedia.org/wiki/Acceptance_testing}
\end{liAntwort}

%%
% d)
%%

\item Fehler

Ein fehlgeschlagener Test und ein Testausführungsfehler bezeichnen
denselben Sachverhalt.

%%
% e)
%%

\item Test Suiten

Tests können hierarchisch strukturiert werden, so dass mit einem Befehl
das gesamte zu testende System getestet werden kann.

\end{enumerate}
\end{document}
