\documentclass{lehramt-informatik-aufgabe}
\begin{document}
\liAufgabenTitel{Test-getriebene Entwicklung}
\section{Aufgabe 1
\index{Testen}
\footcite{66116:2019:03}}

Bewerten Sie die folgenden Aussagen und nehmen Sie jeweils Stellung.

\begin{enumerate}
%%
% a)
%%

\item Tests zuerst

In test-getriebener Softwareentwicklung wird der Test vor der zu
testenden Funktion programmiert.

\begin{liAntwort}
Bei der testgetriebenen Entwicklung erstellt der/die ProgrammiererIn
Softwaretests konsequent vor den zu testenden Komponenten.
\footcite{wiki:testgetriebene-entwicklung}
\end{liAntwort}

%%
% b)
%%

\item Komponententests

Komponententests sind immer White-Box-Tests und nie Black-Box-Tests.

\begin{liAntwort}
Falsch: Komponententests (anderes Wort für Unit- oder Modul-Tests) können beides sein.
\liFussnoteUrl{https://qastack.com.de/software/362746/what-is-black-box-unit-testing}
\end{liAntwort}

%%
% c)
%%

\item Akzeptanztests

Akzeptanz - resp. Funktionstests sind immer Black-Box-Tests und nie
White-Box-Tests.

\begin{liAntwort}
In systems engineering, it may involve black-box testing performed on a system

Acceptance testing in extreme programming:  Acceptance tests are black-box system tests.

\liFussnoteUrl{https://en.wikipedia.org/wiki/Acceptance_testing}
\end{liAntwort}

%%
% d)
%%

\item Fehler

Ein fehlgeschlagener Test und ein Testausführungsfehler bezeichnen
denselben Sachverhalt.

\begin{liAntwort}
Falsch:

Fehler (Error)

Der Software-Test konnte durchgeführt werden, das ermittelte
Ist-Ergebnis und das vorgegebene Soll-Ergebnis weichen jedoch
voneinander ab. Ein derartiger Fehler liegt z. B. dann vor, wenn ein
Funktionsaufruf einen abweichenden Wert berechnet.

Fehlschlag (Failure)

Während der Testdurchführung wurde ein Systemversagen festgestellt. Ein
Fehlschlag liegt z. B. dann vor, wenn das zu testende System einen
Programmabsturz verursacht und hierdurch erst gar kein Ist-Ergebnis
ermittelt werden konnte.\footcite{hoffmann:software}

Das Misslingen kann als Ursache einen Fehler (Error) oder ein falsches
Ergebnis (Failure) haben, die beide per Exception signalisiert werden.
Der Unterschied zwischen den beiden Begriffen liegt darin, dass Failures
erwartet werden, während Errors eher unerwartet auftreten.
\liFussnoteUrl{https://de.wikipedia.org/wiki/JUnit}

\end{liAntwort}

%%
% e)
%%

\item Test Suiten

Tests können hierarchisch strukturiert werden, so dass mit einem Befehl
das gesamte zu testende System getestet werden kann.

\begin{liAntwort}
Richtig:

test suite (englisch „Testsammlung“, aus französisch suite Folge,
Verkettung) bezeichnet eine organisierte Sammlung von Werkzeugen zum
Testen technischer Apparate und Vorgänge.
\liFussnoteUrl{https://de.wikipedia.org/wiki/Testsuite}
\end{liAntwort}

\end{enumerate}
\end{document}
