\documentclass{lehramt-informatik-aufgabe}
\liLadePakete{}
\begin{document}
\liAufgabenTitel{}
\section{Aufgabe 3
\index{Relationale Algebra}
\footcite{66116:2019:03}}

Gegeben seien folgende Relationen:

\liPseudoUeberschrift{X}

\begin{tabular}{|l|l|l|}
\hline
A & B & C \\ \hline
1 & 2 & 3 \\ \hline
2 & 3 & 4 \\ \hline
4 & 3 & 3 \\ \hline
1 & 2 & 2 \\ \hline
2 & 3 & 2 \\ \hline
3 & 3 & 2 \\ \hline
3 & 1 & 2 \\ \hline
2 & 2 & 1 \\ \hline
1 & 1 & 1 \\ \hline
\end{tabular}

\liPseudoUeberschrift{Y}

\begin{tabular}{|l|l|l|}
\hline
C & D & E \\ \hline
2 & 3 & 1 \\ \hline
1 & 1 & 3 \\ \hline
2 & 1 & 3 \\ \hline
3 & 2 & 3 \\ \hline
2 & 2 & 3 \\ \hline
1 & 3 & 2 \\ \hline
\end{tabular}

Geben Sie die Ergebnisrelationen folgender Ausdrücke der relationalen
Algebra als Tabellen an; machen Sie Ihren Rechenweg kenntlich.

% https://dbis-uibk.github.io/relax/calc/local/uibk/local/0

% X = {
%   A B C
%   1 2 3
%   2 3 4
%   4 3 3
%   1 2 2
%   2 3 2
%   3 3 2
%   3 1 2
%   2 2 1
%   1 1 1
% }

% Y = {
%   C D E
%   2 3 1
%   1 1 3
%   2 1 3
%   3 2 3
%   2 2 3
%   1 3 2
% }

\begin{enumerate}
\item $\sigma_{A=2}(X) \bowtie Y$
% sigma A = 2 (X) natural join Y

\begin{liAntwort}
\begin{tabular}{|l|l|l|l|l|}
\hline
A & B & C & D & E \\ \hline
2 & 3 & 2 & 3 & 1 \\ \hline
2 & 3 & 2 & 1 & 3 \\ \hline
2 & 3 & 2 & 2 & 3 \\ \hline
2 & 2 & 1 & 1 & 3 \\ \hline
2 & 2 & 1 & 3 & 2 \\ \hline
\end{tabular}

\end{liAntwort}

\begin{liAntwort}
\item
%pi A,C (X) ÷ pi C (Y)

\begin{tabular}{|l|}
\hline
A \\ \hline
1 \\ \hline
2 \\ \hline
3 \\ \hline
\end{tabular}
\end{liAntwort}

\end{enumerate}

\end{document}
