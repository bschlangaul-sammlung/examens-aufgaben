\documentclass{lehramt-informatik-aufgabe}
\liLadePakete{rmodell,normalformen}
\begin{document}
\let\FA=\liFunktionaleAbhaengigkeiten

\liAufgabenTitel{}
\section{Aufgabe 3
\index{Relationale Entwurfstheorie}
\footcite{66116:2019:03}}

Gegeben sei folgendes relationales Schema R in erster Normalform:

\begin{center}
\liRelationMenge{R}{A, B, C, D, E, F}
\end{center}

\noindent
Für R gelte folgende Menge FD funktionaler Abhängigkeiten:

\bigskip

\FA{
  A, D, F -> E;
  B, C -> A, E;
  D -> B;
  D, E -> C, B;
  A -> F;
}

\begin{enumerate}

%%
% a)
%%

\item Bestimmen Sie alle Kandidatenschlüssel/Schlüsselkandidaten von R
mit FD. Hinweis: Die Angabe von Attributmengen, die keine
Kandidatenschlüssel sind, führt zu Abzügen.

\begin{liAntwort}
\begin{itemize}
\item D A
\item D C
\item D E
\end{itemize}
\end{liAntwort}

%%
% b)
%%

\item Prüfen Sie, ob R mit FD in 2NF bzw. 3NF ist.

\begin{liAntwort}

\end{liAntwort}

%%
% c)
%%

\item Bestimmen Sie mit folgenden Schritten eine kanonische Überdeckung
FD. von FD:

\begin{enumerate}

%%
% I.
%%

\item Führen Sie eine Linksreduktion von FD durch. Geben Sie die Menge
funktionaler Abhängigkeiten nach der Linksreduktion an (FD,).

\begin{liAntwort}

\end{liAntwort}

%%
% II.
%%

\item Führen Sie eine Rechtsreduktion des Ergebnisses der Linksreduktion
(F'D,) durch. Geben Sie die Menge funktionaler Abhängigkeiten nach der
Rechtsreduktion an (F’D,).

\begin{liAntwort}

\end{liAntwort}

%%
% III.
%%

\item Bestimmen Sie eine kanonische Überdeckung FD. von FD auf Basis des
Ergebnisses der Rechtsreduktion (FD;).

\begin{liAntwort}

\end{liAntwort}
\end{enumerate}

%%
% d)
%%

\item Zerlegen Sie R mit FD. mithilfe des Synthesealgorithmus in 3NF.
Geben Sie zudem alle funktionalen Abhängigkeiten der erzeugten
Relationenschemata an.

\begin{liAntwort}

\end{liAntwort}

%%
% e)
%%

\item Prüfen Sie für alle Relationen der Zerlegung aus d), ob sie
jeweils in BCNF sind.

\begin{liAntwort}

\end{liAntwort}

\end{enumerate}
\end{document}
