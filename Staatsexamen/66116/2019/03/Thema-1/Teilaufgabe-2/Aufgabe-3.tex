\documentclass{lehramt-informatik-aufgabe}
\liLadePakete{}
\begin{document}
\liAufgabenTitel{Roboter in einer Montagehalle}
\section{Aufgabe 3
\index{Testen}
\footcite{66116:2019:03}}

\begin{enumerate}
%%
% a)
%%

\item Erläutern Sie kurz, was man unter der Methode der testgetriebenen
Entwicklung versteht.

\begin{liAntwort}
Bei der testgetriebenen Entwicklung erstellt der/die ProgrammiererIn
Softwaretests konsequent vor den zu testenden Komponenten.
\footcite{wiki:testgetriebene-entwicklung}
\end{liAntwort}

%%
% b)
%%

\item Geben Sie für obige Aufgabenstellung (Abarbeitung der Aufträge
durch den Roboter) einen Testfall für eine typische Situation an (d. h.
das Einsammeln von 4 Objekten an unterschiedlichen Positionen).
Spezifizieren Sie als Input alle für die Abarbeitung eines Auftrages
relevanten Eingabe- und Klassen-Parameter sowie den vollständigen und
korrekten Output.
\end{enumerate}
\end{document}
