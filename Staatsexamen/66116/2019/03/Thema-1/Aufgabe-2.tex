\documentclass{lehramt-informatik-aufgabe}
\liLadePakete{rmodell,syntax}
\begin{document}

\let\a=\liAttribut
\let\f=\liFremd
\let\p=\liPrimaer
\let\r=\liRelationMenge

\liAufgabenTitel{Medikamente}
\section{Aufgabe 2
\index{SQL}
\footcite{66116:2019:03}}

Gegeben sei der folgende Ausschnitt des Schemas für die Verwaltung der
Einnahme von Medikamenten:

\begin{liRmodell}
\r{Person}{
  ID : INTEGER,
  Name : VARCHAR(255),
  Wohnort : VARCHAR(255)
}

\r{Hersteller}{
  ID : INTEGER,
  Name : VARCHAR(255),
  Standort : VARCHAR(255),
  AnzahlMitarbeiter : INTEGER
}

\r{Medikament}{
  ID : INTEGER
  Name : VARCHAR(255),
  Kosten : INTEGER,
  Wirkstoff : VARCHAR(255),
  produziert\_von : INTEGER
}

\r{nimmt}{
  Person : INTEGER,
  Medikament : INTEGER,
  von : DATE,
  bis : DATE
}

\r{hat\_Unverträglichkeit\_gegen}{
  Person : INTEGER,
  Medikament : INTEGER
}

\end{liRmodell}

Die Tabelle \a{Person} beschreibt Personen über eine eindeutige ID,
deren Namen und Wohnort. Die Tabelle \a{Medikament} enthält
Informationen über Medikamente, unter anderem deren Namen, Kosten,
Wirkstoffe und einen Verweis auf den Hersteller dieses Medikaments. Die
Tabelle \a{Hersteller} verwaltet verschiedene Hersteller von
Medikamenten. Die Tabelle \a{hat\_Unverträglichkeit\_gegen} speichert
die IDs von Personen zusammen mit den IDs von Medikamenten, gegen die
diese Person eine Unverträglichkeit hat. Die Tabelle \a{nimmt} hingegen
verwaltet die Einnahme der verschiedenen Medikamente und speichert zudem
in welchem Zeitraum eine Person welches Medikament genommen hat bzw.
nimmt.

Beachten Sie bei der Formulierung der SQL-Anweisungen, dass die
Ergebnisrelationen keine Duplikate enthalten dürfen. Sie dürfen
geeignete Views definieren.
\begin{enumerate}

%%
% a)
%%

\item Schreiben Sie SQL-Anweisungen, die für die bereits existierende
Tabelle \a{nimmt} alle benötigten Fremdschlüsselconstraints anlegt.
Erläutern Sie kurz, warum die Spalten \a{von} und \a{bis} Teil des
Primärschlüssels sind.

\begin{liAntwort}
\begin{minted}{sql}
ALTER TABLE nimmt
ADD CONSTRAINT FK_Person
FOREIGN KEY (Person) REFERENCES Person(ID),
ADD CONSTRAINT FK_Medikament
FOREIGN KEY (Medikament) REFERENCES Medikament(ID);
\end{minted}
\end{liAntwort}

%%
% b)
%%

\item Schreiben Sie eine SQL-Anweisung, welche sowohl den \a{Namen} als
auch die \a{ID} von \a{Personen} und \a{Medikamenten} ausgibt, bei denen
die Person das jeweilige Medikament nimmt.

\begin{liAntwort}
\begin{minted}{sql}
SELECT DISTINCT
  p.ID as Personen_ID,
  p.Name as Peronen_Name,
  m.ID as Medikamenten_ID,
  m.Name as Medikamenten_Name
FROM Person p, Medikament m, nimmt n
WHERE n.Person = p.ID AND n.Medikament = n.Medikament;
\end{minted}
\end{liAntwort}

%%
% c)
%%

\item Schreiben Sie eine SQL-Anweisung, welche die ID und den Namen der
Medikamente mit den niedrigsten Kosten je Hersteller bestimmt.

%%
% d)
%%

\item Schreiben Sie eine SQL-Anweisung, die die Anzahl aller Personen
ermittelt, die ein Medikament genommen haben, gegen welches sie eine
Unverträglichkeit entwickelt haben.

%%
%
%%

\item Schreiben Sie eine SQL-Anweisung, die die ID und den Namen
derjenigen Personen ermittelt, die weder ein Medikament mit dem
Wirkstoff Paracetamol noch ein Medikament mit dem Wirkstoff Ibuprofen
genommen haben.

%%
%
%%

\item Schreiben Sie eine SQL-Anweisung, welche die Herstellernamen und
die Anzahl der bekannten Unverträglichkeiten gegen Medikamente dieses
Herstellers ermittelt. Das Ergebnis soll aufsteigend nach der Anzahl der
bekannten Unverträglichkeiten sortiert werden.

%%
%
%%

\item Formulieren Sie eine Anfrage in relationaler Algebra, welche die
Wohnorte aller Personen bestimmt, welche ein Medikament mit demWirkstoff
Paracetamol nehmen oder genommen haben. Die Lösung kann als Baum- oder
als Term-Schreibweise angegeben werden.

%%
%
%%

\item Formulieren Sie eine Anfrage in relationaler Algebra, welche die
Namen aller Personen bestimmt, die von allen bekannten Herstellern,
deren Standort München ist, Medikamente nehmen oder genommen haben. Die
Lösung kann als Baum- oder als Term-Schreibweise angegeben werden.

\end{enumerate}
\end{document}
