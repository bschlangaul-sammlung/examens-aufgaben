\documentclass{lehramt-informatik-aufgabe}
\liLadePakete{sql}
\begin{document}
\liAufgabenMetadaten{
  Titel = {Aufgabe 2},
  Thematik = {Einnahme von Medikamenten},
  RelativerPfad = Staatsexamen/66116/2019/03/Thema-1/Teilaufgabe-1/Aufgabe-2.tex,
  ZitatSchluessel = examen:66116:2019:03,
  BearbeitungsStand = unbekannt,
  Korrektheit = unbekannt,
  Stichwoerter = {SQL},
  ExamenNummer = 66116,
  ExamenJahr = 2019,
  ExamenMonat = 03,
  ExamenThemaNr = 1,
  ExamenTeilaufgabeNr = 1,
  ExamenAufgabeNr = 2,
}

Gegeben sei der folgende Ausschnitt des Schemas für die Verwaltung der
Einnahme von Medikamenten:\index{SQL}
\footcite{examen:66116:2019:03}

Person : {[ Hersteller : {[
ID : INTEGER, ID : INTEGER,
Name : VARCHAR(255), Name : VARCHAR(255),
Wohnort : VARCHAR(255) Standort : VARCHAR(255),
} AnzahlMitarbeiter : INTEGER
}
Medikament : {[ nimmt : {[
ID : INTEGER, Person : INTEGER,
Name : VARCHAR(255), Medikament : INTEGER,
Kosten : INTEGER, von : DATE,
Wirkstoff : VARCHAR(255), bis : DATE
produziert \_von : INTEGER ]}
}

hat\_Unvertraglichkeit\_gegen : {[
Person : INTEGER,
Medikament : INTEGER

}

Die Tabelle Person beschreibt Personen über eine eindeutige ID, deren
Namen und Wohnort. Die Tabelle Medikament enthalt Informationen über
Medikamente, unter anderem deren Namen, Kosten, Wirkstoffe und einen
Verweis auf den Hersteller dieses Medikaments. Die Tabelle Hersteller
verwaltet verschiedene Hersteller von Medikamenten. Die Tabelle
hat\_Unvertrdglichkeit\_gegen speichert die IDs von Personen zusammen mit
den IDs von Medikamenten, gegen die diese Person eine Unverträglichkeit
hat. Die Tabelle nimmt hingegen verwaltet die Einnahme der verschiedenen
Medikamente und speichert zudem in welchem Zeitraum eine Person welches
Medikament genommen hat bzw. nimmt.

Beachten Sie bei der Formulierung der SQL-Anweisungen, dass die
Ergebnisrelationen keine Duplikate enthalten dürfen. Sie dürfen
geeignete Views definieren.

\begin{enumerate}

%%
% a)
%%

\item Schreiben Sie SQL-Anweisungen, die für die bereits existierende
Tabelle nimmt alle benötigten Fremdschlüsselconstraints anlegt.
Erläutern Sie kurz, warum die Spalten von und bis Teil des
Primärschlüssels sind.

\begin{liAntwort}
\begin{minted}{sql}
ALTER TABLE nimmt
ADD FOREIGN KEY(Person) REFERENCES Person(ID),
ADD FOREIGN KEY(Medikament) REFERENCES Medikament(ID);
\end{minted}
\end{liAntwort}

%%
% b)
%%

\item Schreiben Sie eine SQL-Anweisung, welche sowohl den Namen als auch
die ID von Personen und Medikamenten ausgibt, bei denen die Person das
jeweilige Medikament nimmt.

\begin{liAntwort}
\begin{minted}{sql}
SELECT p.Name, p.ID, m.Name, m.ID
FROM Person p, Medikament m, nimmt n
WHERE p.ID = n.Person AND n.Medikament = m.ID;
\end{minted}
\end{liAntwort}

%%
% c)
%%

\item Schreiben Sie eine SQL-Anweisung, welche die ID und den Namen der
Medikamente mit den niedrigsten Kosten je Hersteller bestimmt.

\begin{liAntwort}
\begin{minted}{sql}
SELECT m.Name, m.ID, MIN(m.Kosten)
FROM Medikament m
GROUP BY produziert_von;
\end{minted}
\end{liAntwort}

%%
% d)
%%

\item Schreiben Sie eine SQL-Anweisung, die die Anzahl aller Personen
ermittelt, die ein Medikament genommen haben, gegen welches sie eine
Unverträglichkeit entwickelt haben.

\begin{liAntwort}
\begin{minted}{sql}
SELECT COUNT(DISTINCT n.Person)
FROM nimmt n, hat_Unverträglichkeit_gegen h
WHERE n.Person = h.Person;
\end{minted}
\end{liAntwort}

%%
% e)
%%

\item Schreiben Sie eine SQL-Anweisung, die die ID und den Namen derjenigen
Personen ermittelt, die weder ein Medikament mit dem Wirkstoff
Paracetamol noch ein Medikament mit dem Wirkstoff Ibuprofen genommen
haben.

\begin{liAntwort}
\begin{minted}{sql}
SELECT p.ID, p.Name
FROM Person p, nimmt n, Medikament m
WHERE p.ID = n.Person AND n.Medikament = m.ID AND m.ID NOT IN(
SELECT DISTINCT m.ID
FROM Medikament m
WHERE m.Wirkstoff = "Paracetamol" OR m.Wirkstoff = "Ibuprofen";
\end{minted}
\end{liAntwort}

%%
% f)
%%

\item Schreiben Sie eine SQL-Anweisung, welche die Herstellernamen und die
Anzahl der bekannten Unverträglichkeiten gegen Medikamente dieses
Herstellers ermittelt. Das Ergebnis soll aufsteigend nach der Anzahl der
bekannten Unverträglichkeiten sortiert werden.

\begin{liAntwort}
\begin{minted}{sql}
CREATE VIEW MMU AS
SELECT DISTINCT m.ID, m.produziert_von
FROM Medikament m, hat_Unverträglichkeit h
WHERE m.ID = h.Medikament
SELECT h.ID, COUNT(m.ID)
FROM Hersteller h, MMU m
WHERE m.produziert_von = h.ID
GROUP BY h.ID
\end{minted}
\end{liAntwort}

%%
% g)
%%

\item Formulieren Sie eine Anfrage in relationaler Algebra, welche die
Wohnorte aller Personen bestimmt, welche ein Medikament mit demWirkstoff
Paracetamol nehmen oder genommen haben. Die Lösung kann als Baum- oder
als Term-Schreibweise angegeben werden.

\begin{liAntwort}
π Wohnort (Person ▷◁ p.ID=n.Person σ m.Wirksto f f = ′ Ibupro f en ′ (nimmt ▷◁ m.ID=n.Medikament Medikament))
\end{liAntwort}

%%
% h)
%%

\item Formulieren Sie eine Anfrage in relationaler Algebra, welche die Namen
aller Personen bestimmt, die von allen bekannten Herstellern, deren
Standort München ist, Medikamente nehmen oder genommen haben. Die Lösung
kann als Baum- oder als Term-Schreibweise angegeben werden.

\begin{liAntwort}
π p.Name (Person ▷◁ p.Name=n.Person (nimmt ▷◁ n.Medikament=m.ID (
Medikament ▷◁ m.produziert v on=h.ID σ h.Standort= ′ Mnchen ′ (Hersteller)))
\end{liAntwort}

\end{enumerate}
\end{document}
