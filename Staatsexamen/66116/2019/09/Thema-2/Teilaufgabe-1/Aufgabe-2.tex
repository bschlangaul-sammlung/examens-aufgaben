\documentclass{lehramt-informatik-aufgabe}
\liLadePakete{syntax}
\begin{document}
\liAufgabenTitel{Assertions}
\section{Aufgabe 2
\index{Formale Verifikation}
\footcite{examen:66116:2019:09}}

Methoden in Programmen funktionieren nicht immer für alle möglichen
Eingaben - klassische Beispiele sind die Quadratwurzel einer negativen
Zahl oder die Division durch Null. Zulässige (bzw. in negierter Form
unzulässige) Eingabewerte sollten dann spezifiziert werden, was mit
Hilfe sog. assertions geschehen kann. Assertions prüfen zur Laufzeit den
Wertebereich der Parameter einer Methode, bevor der Methodenkörper
ausgeführt wird. Wenn ein oder mehrere Argumente zur Laufzeit unzulässig
sind, wird eine Ausnahme geworfen.

\liJavaExamen{66116}{2019}{09}{Assertion}

Betrachten Sie das folgende Java-Programm:

\begin{enumerate}

%%
% a)
%%

\item Beschreiben Sie kurz, was dieses Programm tut.

%%
% b)
%%

\item Implementieren Sie drei nützliche Assertions, die zusammen
verhindern, dass das Programm abstürzt.

%%
% c)
%%

\item Wann und warum kann es sinnvoll sein, keine expliziten Assertions
anzugeben? Erläutern Sie, warum das potentiell gefährlich ist.

%%
% d)
%%

\item Assertions können wie oben sowohl für Vorbedingungen am Anfang
einer Methode als auch für Nachbedingungen am Ende einer Methode
verwendet werden. Solche Spezifikationen bilden dann sog. Kontrakte.
Kontrakte werden insbesondere auch statisch verwendet, d. h. nicht zur
Laufzeit. Skizzieren Sie ein sinnvolles Anwendungsgebiet und beschreiben
Sie kurz den Vorteil der Verwendung von Kontrakten in diesem
Anwendungsgebiet.

\end{enumerate}
\end{document}
