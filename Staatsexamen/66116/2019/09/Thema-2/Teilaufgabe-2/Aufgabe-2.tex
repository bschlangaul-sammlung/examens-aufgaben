\documentclass{lehramt-informatik-aufgabe}
\liLadePakete{}
\begin{document}
\liAufgabenTitel{Mitarbeiterverwaltung}
\section{Aufgabe 2
\index{Entity-Relation-Modell}
\footcite{examen:66116:2019:09}}

In einer Datenbank zur Mitarbeiterverwaltung werden die Mitarbeiter über
ihr Ausscheiden aus dem Betrieb hinaus (z. B. Ruhestand oder
Arbeitsplatzwechsel) gespeichert. Im Folgenden ist ein Ausschnitt aus
dem ER-Diagramm dargestellt. Erweitern Sie das Diagramm um genau eine
Entität, welche die derzeit aktiven Mitarbeiter aus allen Unterklassen
umfasst. Benennen und erläutern Sie das von Ihnen verwendete
Modellierungskonstrukt.

fn nn enntn re een,

|
Mitarbeiter |

a | m

Sekretär Forscher Techniker

\end{document}
