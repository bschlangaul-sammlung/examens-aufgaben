\documentclass{lehramt-informatik-aufgabe}
\liLadePakete{}
\begin{document}
\liAufgabenTitel{Formel-1-Rennen}
\section{Aufgabe 7: (SQL)
\index{SQL}
\footcite{66116:2019:09}}

Aufgabe 7: (SQL) (18 Punkte)
Gegeben sind folgende Relationen aus einem Verwaltungssystem für die jährlichen Formel 1-Rennen:

Strecke (Strecken ID, Streckenname, Land, Lange)

Fahrer(Fahrer ID, Fahrername, Nation, Rennstall)
Rennen(Strecken\_ID, Jahr, Wetter)

FK (Strecken ID) referenziert Strecke(Strecken\_ID)
Rennteilnahme(Fahrer ID, Strecken\_ID, Jahr, Rundenbestzeit, Gesamtzeit,

disqualifiziert)
FK (Fahrer ID) referenziert Fahrer(Fahrer\_ ID)
FK (Strecken\_ ID, Jahr) referenziert Rennen (Strecken \_ID, Jahr)

Der Einfachheit halber wird angenommen, dass Fahrer den Rennstall nicht wechseln können. Das
Attribut „disqualifiziert“ kann die Ausprägungen „ja“ und „nein“ haben. Formulieren Sie folgende
Abfragen in SQL. Vermeiden Sie nach Möglichkeit übermäßige Nutzung von Joins und Views.

a) [3 Punkte] Geben Sie für jeden Fahrer seine ID sowie die Anzahl seiner Disqualifikationen in den
Jahren 2005 bis 2017 aus. Ordnen Sie die Ausgabe absteigend nach der Anzahl der
Disqualifikationen.

b) [4 Punkte] Gesucht sind alle Länder, aus denen noch nie ein Fahrer disqualifiziert wurde.

c) [8 Punkte] Gesucht sind die ersten fünf Plätze des Rennens von 2011 in „Abu Dhabi“
(Streckenname). Die Ausgabe soll nach der Platzierung absteigend erfolgen. Geben Sie Fahrer \_ID,
Fahrername, Nation und Rennstall mit aus.

d) [3 Punkte] Führen Sie eine neue Spalte Gehalt in die Tabelle Fahrer ein. Da sich die Prämien für
die Fahrer nach einem Rennstallwechsel ändern, soll ein Trigger geschrieben werden, mit dem das
Gehalt des betreffenden Fahrers um 10\% angehoben wird.
\end{document}
