\documentclass{lehramt-informatik-aufgabe}
\liLadePakete{rmodell}

\begin{document}
\liAufgabenTitel{Formel-1-Rennen}
\section{Aufgabe 7: (SQL)
\index{SQL}
\footcite{66116:2019:09}}

Gegeben sind folgende Relationen aus einem Verwaltungssystem für die
jährlichen Formel-1-Rennen:

\begin{liRmodell}
Strecke (\liPrimaer{Strecken\_ID}, Streckenname, Land, Lange)

Fahrer(\liPrimaer{Fahrer\_ID}, Fahrername, Nation, Rennstall)

Rennen(\liPrimaer{Strecken\_ID[Strecke], Jahr}, Wetter)

Rennteilnahme(\liPrimaer{Fahrer\_ID[Fahrer], Strecken\_ID[Rennen], Jahr[Rennen]}, Rundenbestzeit, Gesamtzeit,
disqualifiziert)

FK (Strecken\_ID, Jahr) referenziert Rennen (Strecken\_ID, Jahr)
\end{liRmodell}

\noindent
Der Einfachheit halber wird angenommen, dass Fahrer den Rennstall nicht
wechseln können. Das Attribut „disqualifiziert“ kann die Ausprägungen
„ja“ und „nein“ haben. Formulieren Sie folgende Abfragen in SQL.
Vermeiden Sie nach Möglichkeit übermäßige Nutzung von Joins und Views.

\begin{enumerate}
\item Geben Sie für jeden Fahrer seine ID sowie die Anzahl seiner
Disqualifikationen in den Jahren 2005 bis 2017 aus. Ordnen Sie die
Ausgabe absteigend nach der Anzahl der Disqualifikationen.

\item Gesucht sind alle Länder, aus denen noch nie ein Fahrer
disqualifiziert wurde.

\item Gesucht sind die ersten fünf Plätze des Rennens von 2011 in „Abu
Dhabi“ (Streckenname). Die Ausgabe soll nach der Platzierung absteigend
erfolgen. Geben Sie Fahrer\_ID, Fahrername, Nation und Rennstall mit
aus.

\item Führen Sie eine neue Spalte Gehalt in die Tabelle Fahrer ein. Da
sich die Prämien für die Fahrer nach einem Rennstallwechsel ändern, soll
ein Trigger geschrieben werden, mit dem das Gehalt des betreffenden
Fahrers um 10\% angehoben wird.
\end{enumerate}

\end{document}
