\documentclass{lehramt-informatik-aufgabe}
\liLadePakete{}
\begin{document}
\liAufgabenTitel{R (A, B, C, D, E, F)}
\section{Aufgabe 4
\index{Normalformen}
\footcite{examen:66116:2019:09}}

Gegeben sei das Relationenschema R (A, B, C, D, E, F) sowie die Menge
der zugehörigen funktionalen Abhängigkeiten FD:

AB -> C
A -> D
F -> B
DE -> B
B -> A

a) [5 Punkte] Bestimmen Sie sämtliche Schlüsselkandidaten der Relation R
und begründen Sie, warum es keine weiteren Schlüsselkandidaten geben
kann.

b) [8 Punkte] Ist die gegebene Menge an funktionalen Abhängigkeiten
minimal? Fall sie minimal ist begründen Sie diese Eigenschaft
ausführlich, anderenfalls minimieren Sie FD schrittweise. Vergessen Sie
nicht die einzelnen Schritte entsprechend zu begründen.

c) [9 Punkte] Überführen Sie falls nötig das Schema in dritte
Normalform. Ist die dritte Normalform bereits erfüllt, begründen Sie
dies ausführlich.
\end{document}
