\documentclass{bschlangaul-aufgabe}
\bLadePakete{spalten,relationale-algebra}
\begin{document}
\bAufgabenMetadaten{
  Titel = {Aufgabe 3},
  Thematik = {R1 und R1},
  RelativerPfad = Staatsexamen/66116/2019/09/Thema-2/Teilaufgabe-2/Aufgabe-3.tex,
  ZitatSchluessel = examen:66116:2019:09,
  Stichwoerter = {Relationale Algebra},
  ExamenNummer = 66116,
  ExamenJahr = 2019,
  ExamenMonat = 09,
  ExamenThemaNr = 2,
  ExamenTeilaufgabeNr = 2,
  ExamenAufgabeNr = 3,
}

Gegeben seien die folgenden beiden Relationen:
\index{Relationale Algebra}
\footcite{examen:66116:2019:09}

\begin{multicols}{2}
\bPseudoUeberschrift{R1}

\begin{tabular}{lll}
P  & Q & S\\\hline\hline
10 & a & 5\\\hline
15 & b & 8\\\hline
25 & a & 6\\\hline
\end{tabular}

\bPseudoUeberschrift{R2}

\begin{tabular}{lll}
A  & B & C\\\hline\hline
10 & b & 6\\\hline
25 & c & 3\\\hline
10 & b & 5\\\hline
\end{tabular}
\end{multicols}

\noindent
Geben Sie die Ergebnisse der folgenden relationalen Ausdrücke an:

\begin{enumerate}

%%
% a)
%%

\item $R1 \bowtie_{R1.P=R2.A} R2$ (Equi-Join)

\begin{bAntwort}
\begin{tabular}{llllll}
P  & Q & S & A & B  & C \\\hline\hline
10 & a & 5 & 10 & b & 6 \\\hline
10 & a & 5 & 10 & b & 5 \\\hline
25 & a & 6 & 25 & c & 3 \\\hline
\end{tabular}
\end{bAntwort}

%%
% b)
%%

\item $R1 \rightouterjoin_{R1.Q=R2.B} R2$ (Right-Outer-Join)

\begin{bAntwort}
\begin{tabular}{llllll}
P  & Q & S & A  & B & C \\\hline\hline
15 & b & 8 & 10 & b & 6\\\hline
15 & b & 8 & 10 & b & 5\\\hline
   &   &   & 25 & c & 3\\\hline
\end{tabular}
\end{bAntwort}

%%
% c)
%%

\item Es ist bekannt, dass die minimale Menge relationaler Operatoren
Selektion, Projektion, Vereinigung, Differenz und kartesisches Produkt
umfasst. Wie kann die Division zweier Relationen mit diesen Operatoren
ausgedrückt werden? Begründen Sie kurz die einzelnen Bestandteile Ihres
relationalen Ausdrucks.

\begin{bAntwort}
Seien $R$, $S$ Relationen und $\beta$ die zu $R$ sowie $\gamma$ die zu
$S$ dazugehörigen Attributmengen. $R':=\beta \setminus \gamma$.
Die Division ist dann definiert durch:\footcite[Division]{wiki:relationale-algebra}

\begin{displaymath}
R\div S :=
\pi_{R'}(R) -
\pi_{R'}((\pi_{R'}(R) \times S) - R)
\end{displaymath}
\end{bAntwort}

\end{enumerate}
\end{document}
