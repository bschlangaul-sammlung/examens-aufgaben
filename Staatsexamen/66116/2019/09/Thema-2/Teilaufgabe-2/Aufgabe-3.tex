\documentclass{lehramt-informatik-aufgabe}
\liLadePakete{}
\begin{document}
\liAufgabenTitel{}
\section{Aufgabe 3
\index{Relationale Algebra}
\footcite{examen:66116:2019:09}}

: (Relationale Algebra) (6 Punkte)

Gegeben seien die folgenden beiden Relationen:

Ri P
10
15
25

A

10
25
10

Geben Sie die Ergebnisse der folgenden relationalen Ausdrücke an:

a) [1,5 Punkte] Rı m rı.r-R2AR2 (Equi-Join)

b) [1,5 Punkte] Rı»< r1.9=R2.B Rz (Right-Outer-Join)

c) [3 Punkte] Es ist bekannt, dass die minimale Menge relationaler Operatoren Selektion, Projektion,
Vereinigung, Differenz und kartesisches Produkt umfasst. Wie kann die Division zweier
Relationen mit diesen Operatoren ausgedrückt werden? Begründen Sie kurz die einzelnen
Bestandteile Ihres relationalen Ausdrucks.

\end{document}
