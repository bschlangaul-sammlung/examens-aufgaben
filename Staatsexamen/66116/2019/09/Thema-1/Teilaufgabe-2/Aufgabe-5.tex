\documentclass{lehramt-informatik-aufgabe}
\liLadePakete{normalformen}
\begin{document}
\liAufgabenTitel{}

\let\b=\textbf

\section{Aufgabe 5
\index{Normalformen}
\footcite{examen:66116:2019:09}}

\begin{enumerate}

%-----------------------------------------------------------------------
% 1.
%-----------------------------------------------------------------------

\item Gegeben sind folgende funktionale Abhängigkeiten.

X— YZ
Z— WX
Q— XYZ
V— ZW
ZW — YQV

Berechnen Sie die kanonische Überdeckung.

%-----------------------------------------------------------------------
% 2.
%-----------------------------------------------------------------------

\item Gegeben ist folgende Tabelle.

\begin{center}
\begin{tabular}{|l|l|l|l|l|}
\hline
\underline{JedID} & Name             & Rasse        & Lichtschwert & Seite der Macht\\\hline\hline
2     & Yoda             & Unbekannt & Grün         & Gute Seite\\\hline
3     & Anakin Skywalker & Mensch    & Blau, Rot    & Gute Seite, Dunkle Seite\\\hline
4     & Mace Windou      & Mensch    & Lila         & Gute Seite\\\hline
5     & Count Dooku      & Mensch    & Rot          & Dunkle Seite\\\hline
6     & Ahsoka Tano      & Togruta   & Grün         & Gute Seite\\\hline
7     & Yoda             & Mensch    & Rot          & Dunkle Seite\\\hline
\end{tabular}
\end{center}

Geben Sie zuerst die funktionalen Abhängigkeiten in der Tabelle an.

\begin{liAntwort}
\liFunktionaleAbhaengigkeiten{
JedID -> Name, Rasse, Lichtschwert, Seite der Macht;
Lichtschwert -> Seite der Macht;
}

JedID ist ein Surrogat-Schlüssel\liFussnoteUrl{https://de.wikipedia.org/wiki/Surrogatschlüssel}, \dh ein künstlich eingeführter
Primärschlüssel, von dem alle Attribute abhängen.

Ein rotes Lichtschwert zeigt an, dass der Jedi-Ritter zur dunklen Seite
der Macht gehört. Grüne, blaue und lila Lichtschwerter zeigen an, dass
der Jedi-Ritter zu guten Seite gehört. Wir können die Funktionale
Abhängigkeit nicht umdrehen, weil wir nicht von der guten Seite der
Macht auf die Farbe schließen können.
\end{liAntwort}

%-----------------------------------------------------------------------
% 3.
%-----------------------------------------------------------------------

\item \strut

\begin{enumerate}

%%
%  a)
%%

\item Geben Sie die zentrale Eigenschaft der 1. NF an.

\begin{liAntwort}
Eine Relation befindet sich in erster Normalform (1NF), wenn sie
ausschließlich atomare Attributwerte enthält.\footcite[Seite 448]{schneider}
\end{liAntwort}

%%
% b)
%%

\item Nennen Sie alle Stellen, an denen das Schema die 1. NF verletzt.

\begin{liAntwort}
Im Tupel (JedID = 3) haben die Attribute \emph{Lichtschwert} und
\emph{Seite der Macht} mehrwertige Attribute.
\end{liAntwort}

%%
% c)
%%

\item Überführen Sie die Tabelle in die 1. NF.

\begin{liAntwort}
\begin{center}
\begin{tabular}{|l|l|l|l|l|}
\hline
\underline{JedID} & Name             & Rasse        & Lichtschwert & Seite der Macht\\\hline\hline
2     & Yoda             & Unbekannt & Grün         & Gute Seite\\\hline
3     & Anakin Skywalker & Mensch    & Blau         & Gute Seite\\\hline
4     & Mace Windou      & Mensch    & Lila         & Gute Seite\\\hline
5     & Count Dooku      & Mensch    & Rot          & Dunkle Seite\\\hline
6     & Ahsoka Tano      & Togruta   & Grün         & Gute Seite\\\hline
7     & Yoda             & Mensch    & Rot          & Dunkle Seite\\\hline
\b{8}     & \b{Darth Vader}      & \b{Mensch}    & \b{Rot}          & \b{Dunkle Seite}\\\hline
\end{tabular}
\end{center}
\end{liAntwort}

\end{enumerate}

%-----------------------------------------------------------------------
% 4.
%-----------------------------------------------------------------------

\item \strut

\begin{enumerate}

%%
% a)
%%

\item Geben Sie die Definition der 2. NF an.

\begin{liAntwort}
Eine Relation in in zweiter Normalform (2NF), wenn sie in 1NF und jedes
Nichtschlüsselattribut von jedem Schlüsselkandidaten voll funktional
abhängig ist.\footcite[Seite 449]{schneider}
\end{liAntwort}

%%
% b)
%%

\item Arbeiten Sie bitte mit folgender, nicht korrekten Zwischenlösung weiter.
Erläutern Sie, inwiefern dieses Schema die 2. NF verletzt.

\begin{center}
\begin{tabular}{|l|l|l|l|l|}
\hline
JedID & Name      & Rasse     & Lichtschwert & Seite der Macht\\\hline\hline
2     & Yoda      & Unbekannt & Grün         & Gute Seite\\\hline
3     & Skywalker & Mensch    & Blau         & Gute Seite\\\hline
4     & Windou    & Mensch    & Lila         & Gute Seite\\\hline
5     & Dooku     & Mensch    & Rot          & Dunkle Seite\\\hline
6     & Tano      & Togruta   & Grün         & Gute Seite\\\hline
2     & Yoda      & Mensch    & Rot          & Dunkle Seite\\\hline
\end{tabular}
\end{center}

%%
% c)
%%

\item Überführen Sie die Tabelle in die 2. NF.
\end{enumerate}

%-----------------------------------------------------------------------
% 5.
%-----------------------------------------------------------------------

\item 5.

\begin{enumerate}

%%
% a)
%%

\item Geben Sie die Definition der 3. NF an.

\begin{liAntwort}
Eine Relation befindet sind in der dritten Normalform (3NF), wenn keine
transitiven Abhängigkeiten der Nichtschlüsselattribute existieren.
\footcite[Seite 449]{schneider}
\end{liAntwort}

%%
% b)
%%

\item Erläutern Sie, ob und wenn ja, wie das von Ihnen in 3c) neu
erstellte Schema die 3. NF verletzt.

\end{enumerate}
\end{enumerate}
\end{document}
