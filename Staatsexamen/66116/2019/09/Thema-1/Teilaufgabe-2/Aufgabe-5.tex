\documentclass{lehramt-informatik-aufgabe}
\liLadePakete{}
\begin{document}
\liAufgabenTitel{}
\section{Aufgabe 5
\index{Normalformen}
\footcite{examen:66116:2019:09}}

Gegeben sind folgende funktionale Abhängigkeiten.

X— YZ
Z— WX
Q— XYZ
V— ZW
ZW — YQV

\begin{enumerate}

\item 1. [10 Punkte] Berechnen Sie die kanonische Überdeckung.

Gegeben ist folgende Tabelle.
JediD Name Rasse Lichtschwert Seite der Macht
2 Yoda Unbekannt Grün Gute Seite
3 Anakin Skywalker |Mensch Blau, Rot Gute Seite, dunkle Seite
4 Mace Windou Mensch Lila Gute Seite
5 Count Dooku Mensch Rot Dunkle Seite
6 Ahsoka Tano Togruta Grün Gute Seite
7 Yoda Mensch Rot Dunkle Seite
2. [2 Punkte] Geben Sie zuerst die funktionalen Abhängigkeiten in der Tabelle an.
3. a) [1 Punkt] Geben Sie die zentrale Eigenschaft der 1. NF an.
%%
% b)
%%

\item [3 Punkte] Nennen Sie alle Stellen, an denen das Schema die 1. NF verletzt.

%%
% c)
%%

\item [3 Punkte] Überführen Sie die Tabelle in die 1. NF.
4. a) [1 Punkt] Geben Sie die Definition der 2. NF an.

%%
% b)
%%

\item [2 Punkte] Arbeiten Sie bitte mit folgender, nicht korrekten Zwischenlösung weiter.
Erläutern Sie, inwiefern dieses Schema die 2. NF verletzt.

JediD Name Rasse Lichtschwert Seite der Macht
2 Yoda Unbekannt Grün Gute Seite
3 Skywalker Mensch Blau Gute Seite
4 Windou Mensch Lila Gute Seite
5 Dooku Mensch Rot Dunkle Seite
6 Tano Togruta Grün Gute Seite
2 Yoda Mensch Rot Dunkle Seite

%%
% c)
%%

\item [3 Punkte] Überführen Sie die Tabelle in die 2. NF.

5. a) [1 Punkt] Geben Sie die Definition der 3. NF an.

%%
% b)
%%

\item [1 Punkt] Erläutern Sie, ob und wenn ja, wie das von Ihnen in 3c) neu erstellte Schema die

3. NF verletzt.

\end{enumerate}
\end{document}
