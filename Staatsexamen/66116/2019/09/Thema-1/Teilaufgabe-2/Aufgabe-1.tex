\documentclass{lehramt-informatik-aufgabe}
\liLadePakete{}
\begin{document}
\liAufgabenTitel{Wissensfragen}
\section{Aufgabe 1
\index{DB}
\footcite{examen:66116:2019:09}}

Antworten Sie kurz und prägnant.

\begin{enumerate}

%%
% 1.
%%

\item Nennen Sie einen Vorteil und einen Nachteil der
Schichtenarchitektur.

\begin{liAntwort}
\liPseudoUeberschrift{Vorteil}

Physische Datenunabhängigkeit:

Die interne Ebene ist von der konzeptionellen und externen Ebene getrennt.

Physische Änderungen, z. B. des Speichermediums oder des
Datenbankprodukts, wirken sich nicht auf die konzeptionelle oder externe
Ebene aus.

Logische Datenunabhängigkeit:

Die konzeptionelle und die externe Ebene sind getrennt. Dies bedeutet,
dass Änderungen an der Datenbankstruktur (konzeptionelle Ebene) keine
Auswirkungen auf die externe Ebene, also die Masken-Layouts, Listen und
Schnittstellen haben.

\liFussnoteUrl{https://de.wikipedia.org/wiki/ANSI-SPARC-Architektur}

\liPseudoUeberschrift{Nachteil}

Overhead durch zur Trennung der Ebenen benötigten Schnittstellen
\end{liAntwort}

%%
% 2.
%%

\item Wie ermöglicht es ein Datenbankensystem, verschiedene Sichten
darzustellen?

\begin{liAntwort}
Die Sichten greifen auf die zwei darunterliegenden Abstraktionsebenen
eines Datenbanksystems zu, nämlich auf die logische Ebene und die
logische Ebene greift auf die physische Ebene zu.
\footcite[Seite 23]{kemper}
\end{liAntwort}

%%
% 3.
%%

\item Was beschreibt das Konzept der Transitiven Hülle? Erklären Sie
dies kurz und nennen Sie ein Beispiel für (1) die Transitive Hülle eines
Attributes bei funktionalen Abhängigkeiten und (2) die Transitive Hülle
einer SQL-Anfrage.

\begin{liAntwort}
Die transitive Hülle einer Relation R mit zwei Attributen A und B
gleichen Typs ist definitert als

Sie enthält damit alle Tupel (a, b), für die ein Pfad beliebiger Länge k
in R existiert.
\footcite[Seite 135]{kemper}

Berechnung rekursiver Anfragen (z. B. transitive Hülle) über rekursiv
definierte Sichten (Tabellen)\liFussnoteUrl{https://dbs.uni-leipzig.de/file/dbs2-ss16-kap4.pdf}
\end{liAntwort}

%%
% 4.
%%

\item Nennen Sie zwei Indexstrukturen und beschreiben Sie jeweils ihren
Vorteil.

\begin{liAntwort}
In Hauptspeicher-Datenbanksystemen werden oft Hashtabellen verwendet, um
effizierte Punkt-Abfragen (exact Match) zu unterstützen.

Wenn auch Bereichs-Abfragen (range queries) vorkommen, werden zumeister
balancierte Suchbäume - AVL- oder rot/schwarz-Bäume - verwendet.
\footcite[Seite 625]{kemper}

\liFussnoteUrl{https://de.wikipedia.org/wiki/Indexstruktur}
\end{liAntwort}

\end{enumerate}
\end{document}
