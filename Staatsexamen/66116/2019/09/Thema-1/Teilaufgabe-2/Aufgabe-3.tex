\documentclass{lehramt-informatik-aufgabe}
\liLadePakete{syntax,rmodell}
\begin{document}
\liAufgabenTitel{Game of Thrones}
\section{
\index{SQL}
\footcite{66116:2019:09}}

Formulieren Sie folgende Anfragen in SQL gegen die angegebene Datenbank
aus einer imaginären Serie.

\begin{liRmodell}
\liRelation{Figur}{\liPrimaer{Id}, Name, Schwertkunst, Lebendig, Titel}

\liRelation{gehört\_zu}{\liPrimaer{Id, Familie},
FK (Id) references Figur(Id),
FK (Familie) references Familie(Id) }

\liRelation{Familie}{\liPrimaer{Id}, Name, Reichtum, Anführer}

\liRelation{Drache}{\liPrimaer{Name}, Lebendig}

\liRelation{besitzt}{\liPrimaer{Id, Name},
FK (Id) references Figur(Id),
FK (Name) references Drache(Name)}

\liRelation{Festung}{\liPrimaer{Name}, Ort, Ruine}

\liRelation{besetzt}{\liPrimaer{Familie, Festung},
FK (Familie) references Familie(Id),
FK (Festung) references Festung(Name)}

\liRelation{lebt}{\liPrimaer{Id, Name},
FK (Id) references Figur(Id),
FK (Name) references Festung(Name)}
\end{liRmodell}

% Datenbankname: game_of_thrones
\begin{minted}{sql}
CREATE TABLE Figur (
  Id integer PRIMARY KEY,
  Name varchar(20),
  Schwertkunst integer,
  Lebendig boolean,
  Titel varchar(50)
);

INSERT INTO Figur VALUES
  (1, 'Eddard Stark', 5, FALSE, 'Lord von Winterfell'),
  (2, 'Rodd Stark', 4, FALSE, 'Lord von Winterfell'),
  (3, 'Tywin Lennister', 5, FALSE, 'Lord von Casterlystein'),
  (4, 'Cersei Lennister', 2, TRUE, 'Lady von Casterlystein');

CREATE TABLE Familie (
  Id integer PRIMARY KEY,
  Name varchar(20),
  Reichtum numeric(11,2),
  Anführer varchar(20)
);

CREATE TABLE gehört_zu (
  Id integer REFERENCES Figur(id),
  Familie integer REFERENCES Familie(id)
);

CREATE TABLE Drache (
  Name varchar(20) PRIMARY KEY,
  Lebendig boolean
);

CREATE TABLE besitzt (
  Id integer REFERENCES Figur(Id),
  Name varchar(20) REFERENCES Drache(Name)
);

CREATE TABLE Festung (
  Name varchar(20) PRIMARY KEY,
  Ort varchar(20),
  Ruine boolean
);

INSERT INTO Festung VALUES
  ('Roter Bergfried', 'Königsmund', FALSE);

CREATE TABLE besetzt (
  Familie integer REFERENCES Familie(Id),
  Festung varchar(20) REFERENCES Festung(Name)
);

CREATE TABLE lebt (
  Id integer REFERENCES Figur(Id),
  Name varchar(20) REFERENCES Festung(Name)
);

INSERT INTO lebt VALUES (
  4, 'Roter Bergfried'
);
\end{minted}
\index{SQL mit Übungsdatenbank}

\begin{enumerate}
\item Geben Sie für alle Figuren an, wie oft alle vorhandenen Titel
vorkommen.

\begin{antwort}
\begin{minted}{sql}
SELECT
  Name,
  Titel,
  (SELECT COUNT(*) FROM Figur g WHERE g.Titel = f.Titel) AS Anzahl_Titel
FROM Figur f;
\end{minted}
\end{antwort}

\item Welche Figuren (Name ist gesucht) kommen aus „Kings Landing“?

\begin{antwort}
\begin{minted}{sql}
SELECT
  f.Name
FROM
  Figur f,
  Festung b,
  lebt l
WHERE
  b.Name = l.Name AND
  f.Id = l.Id AND
  b.Ort = 'Königsmund';
\end{minted}
\end{antwort}

\item Geben Sie für jede Familie (Name) die Anzahl der zugehörigen
Charaktere und Festungen an.

\item Gesucht sind die besten fünf Schwertkämpfer aus Festungen aus dem
Ort „Westeros“. Es soll der Name, die Schwertkunst und die Platzierung
ausgegeben werden. Die Ausgabe soll nach der Platzierung sortiert
erfolgen.

\item Schreiben Sie eine Anfrage, die alle Figuren löscht, die tot sind.
Das Attribut Lebendig kann dabei die Optionen „ja“ und „nein“ annehmen.

\item Löschen Sie die Spalten „Lebendig“ aus der Datenbank.

\item Erstellen Sie eine weitere Tabelle mit dem Namen Waffen, welche
genau diese auflistet. Eine Waffe ist genau einer Figur zugeordnet, hat
einen eindeutigen Namen und eine Stärke zwischen 0 und 5. Wählen Sie
sinnvolle Typen für die Attribute.
\end{enumerate}

\end{document}
