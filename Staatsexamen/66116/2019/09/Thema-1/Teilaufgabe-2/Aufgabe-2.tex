\documentclass{bschlangaul-aufgabe}
\bLadePakete{er}
\begin{document}
\bAufgabenMetadaten{
  Titel = {Aufgabe 2: (ER-Modellierung)},
  Thematik = {Schule Hogwarts aus Harry Potter},
  RelativerPfad = Staatsexamen/66116/2019/09/Thema-1/Teilaufgabe-2/Aufgabe-2.tex,
  ZitatSchluessel = examen:66116:2019:09,
  ZitatBeschreibung = {Thema 1 Teilaufgabe 2 Aufgabe 2},
  BearbeitungsStand = unbekannt,
  Korrektheit = unbekannt,
  Stichwoerter = {Entity-Relation-Modell},
  ExamenNummer = 66116,
  ExamenJahr = 2019,
  ExamenMonat = 09,
  ExamenThemaNr = 1,
  ExamenTeilaufgabeNr = 2,
  ExamenAufgabeNr = 2,
}

Entwerfen Sie ein ER-Diagramm für eine Schule aus einer imaginären
Film-Reihe. Geben Sie alle Attribute an und unterstreichen Sie
Schlüsselattribute. Für die Angabe der Kardinalitäten von Beziehungen
soll die Min-Max-Notation verwendet werden. Führen Sie wenn nötig einen
Surrogatschlüssel ein.
\index{Entity-Relation-Modell}
\footcite[Thema 1 Teilaufgabe 2 Aufgabe 2]{examen:66116:2019:09}

An der Schule werden Schüler ausgebildet. Sie haben einen Namen, ein
Geschlecht und ein Alter. Da die Schule klein ist, ist der Name
eindeutig. Jeder Schüler ist Teil seines Jahrgangs, bestimmt durch Jahr
und Anzahl an Schüler (Jahrgänge ohne Schüler sind erlaubt), und besucht
mit diesem Kurse. Dabei wird jeder Kurs von min. einem Jahrgang besucht
und jeder Jahrgang hat zwischen 2 und 5 Kurse. Kurse haben einen
Veranstaltungsort und einen Namen.

Außerdem wird jeder Schüler einem von vier Häusern zugeordnet. Diese
Häuser sind Gryffindor, Slytherin, Hufflepuff und Ravenclaw. Jedes Haus
hat eine Anzahl an Mitgliedern.

Um die Organisation an der Schule zu erleichtern, gibt es pro Haus einen
Vertrauensschiiler und pro Jahrgang einen Jahrgangssprecher. AuBerdem
können Schüler Quidditch spielen. Dabei können sie die Rollen Sucher,
Treiber, Jäger und Hüter spielen. Jedes Haus der Schule hat eine
Mannschaft. Diese besteht aus genau einem Sucher, einem Hüter, drei
Jäger und zwei Treiber. Jedes Jahr gibt es an der Schule eine Trophäe zu
gewinnen. Diese ist abhängig von der Mannschaft und weiterhin durch das
Jahr identifiziert.
\end{document}
