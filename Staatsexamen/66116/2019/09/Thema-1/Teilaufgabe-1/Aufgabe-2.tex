\documentclass{lehramt-informatik-aufgabe}

\begin{document}
\liAufgabenMetadaten{
  Titel = {Aufgabe 2},
  Thematik = {Dienstreise},
  RelativerPfad = Staatsexamen/66116/2019/09/Thema-1/Teilaufgabe-1/Aufgabe-2.tex,
  ZitatSchluessel = examen:66116:2019:09,
  BearbeitungsStand = unbekannt,
  Korrektheit = unbekannt,
  Stichwoerter = {UML-Diagramme, Anwendungsfalldiagramm, Aktivitätsdiagramm},
  ExamenNummer = 66116,
  ExamenJahr = 2019,
  ExamenMonat = 09,
  ExamenThemaNr = 1,
  ExamenTeilaufgabeNr = 1,
  ExamenAufgabeNr = 2,
}

Eine Dienstreise an einer Universität wird folgendermaßen abgewickelt:
\index{UML-Diagramme}
\footcite{examen:66116:2019:09}

Ein Mitarbeiter erstellt einen Dienstreiseantrag und legt ihn seinem
Vorgesetzten zur Unterschrift vor. Mit seiner Unterschrift befürwortet
der Vorgesetzte die Dienstreise. Verweigert er die Unterschrift, wird
der Vorgang abgebrochen. Nach der Befürwortung wird der Antrag an die
Reisekostenstelle weitergeleitet, die über die Annahme des Antrags
entscheidet. Im Falle einer Ablehnung wird der Vorgang abgebrochen;
sonst genehmigt die Reisekostenstelle den Antrag. Der Mitarbeiter kann
nun einen Abschlag beantragen. Stimmt der Vorgesetzte zu, so entscheidet
die Reisekostenstelle über den Abschlag und weist ggf. die Kasse der
Universität an, den Abschlag auszuzahlen. Nach Ende der Dienstreise
erstellt der Mitarbeiter einen Antrag auf Erstattung der Reisekosten an
die Reisekostenstelle. Die Reisekostenstelle setzt den Erstattungsbetrag
fest und weist die Kasse an, den Betrag auszuzahlen.

\begin{enumerate}
%%
% a)
%%

\item Erstellen Sie ein Anwendungsfalldiagramm (Use-Case-Diagramm) für
Dienstreisen.\index{Anwendungsfalldiagramm}

%%
% b)
%%

\item Erstellen Sie ein Aktivitätsdiagramm, das den oben beschriebenen
Ablauf modelliert. Ordnen Sie den Aktionen die Akteure gemäß a) zu.
Beschränken Sie sich auf den Kontrollfluss (keine Objektflüsse).
\index{Aktivitätsdiagramm}
\end{enumerate}
\end{document}
