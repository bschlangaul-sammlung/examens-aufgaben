\documentclass{lehramt-informatik-minimal}
\InformatikPakete{syntax}
\begin{document}

\section{Aufgabe zu Abstrakten Klassen und Interfaces (Parkhaus)}

\begin{quellen}
\item \cite{aud:ab:1}
\item \cite[Seite 11, Thema 2 Teilaufgabe 2, Aufgabe 1]{examen:66116:2014:03}
\end{quellen}

In dieser Aufgabe werden Sie Datentypen für die Verwaltung eines
Parkhauses mit Hilfe objektorientierter Methoden definieren. Bearbeiten
Sie die folgenden Teilaufgaben in einer objektorientierten
Programmiersprache Ihrer Wahl (geben Sie diese an)! Solange nicht anders
definiert, sollen Eigenschaften und Methoden \emph{uneingeschränkt
sichtbar} sein.

\renewcommand{\labelenumi}{(\alph{enumi})}
\renewcommand{\labelenumii}{(\roman{enumii})}
\begin{enumerate}

%%
% (a)
%%

\item Erzeugen Sie eine \emph{Klasse} \java{Fahrzeug}, deren Instanzen
folgende Eigenschaften besitzen (wählen Sie geeignete Typen):

\begin{compactitem}
\item Ein amtliches Kennzeichen (Buchstaben- und Zahlenkombination).

\item Die Dimensionen des Fahrzeugs (Länge, Breite, Höhe) in Metern.

\item Das Datum der Erstzulassung. Definieren Sie hierfür entweder einen
eigenen Datentyp oder machen Sie Gebrauch von der Standardbibliothek
Ihrer gewählten Programmiersprache.
\end{compactitem}

Die Eigenschaften sollen für \emph{Unterklassen nicht sichtbar} sein.
Schreiben Sie außerdem einen Konstruktor, der eine Instanz erzeugt und
die Eigenschaften setzt!

\begin{antwort}
\inputcode[firstline=3]{aufgaben/aud/ab_1/parkhaus/Fahrzeug}
\end{antwort}

%%
% (b)
%%

\item Schreiben Sie eine \emph{Klasse} \emph{Parkplatz}, in der
ebenfalls Eigenschaften für die Dimension (Länge, Breite, Höhe) in
Metern vorgesehen sind! Die Eigenschaften sollen ebenfalls im
\emph{Konstruktor} initialisiert werden können. Außerdem soll eine
Objektmethode hinzugefügt werden, die prüft, ob ein gegebenes Fahrzeug
in den Parkplatz passt.

\begin{antwort}
\inputcode[firstline=3]{aufgaben/aud/ab_1/parkhaus/Parkplatz}
\end{antwort}

%%
% (c)
%%

\item Ein \emph{Interface} \java{Parkhaus} soll Objektmethoden für
folgende Anwendungsfälle deklarieren:

\begin{enumerate}

%%
% (i)
%%

\item Alle freien Parkplätze sollen (z.B. als \emph{Array} oder als
Instanz einer in der Standardbibliothek Ihrer verwendeten Sprache
definierten Kollektionsklasse) zurückgegeben werden.

%%
% (ii)
%%

\item Der erste freie Parkplatz, der zu einem gegebenen Fahrzeug passt,
soll zurückgegeben werden.

%%
% (iii)
%%

\item Ein gegebener Parkplatz soll für ein gegebenes Fahrzeug reserviert
werden. Deklarieren Sie das Interface und geben Sie geeignete Signaturen
die \emph{Objektmethoden} an!
\end{enumerate}

\begin{antwort}
\inputcode[firstline=3]{aufgaben/aud/ab_1/parkhaus/Parkhaus}
\end{antwort}

\item Schreiben Sie eine \emph{abstrakte Klasse}, die das Interface
\java{Parkhaus} partiell implementiert! Geben Sie außerdem eine
geeignete \emph{Implementierung} für die Objektmethode (c.ii) unter
Verwendung der existierenden Objektmethoden aus (b) und (c.i) an!

\begin{antwort}
\inputcode[firstline=3]{aufgaben/aud/ab_1/parkhaus/MeinParkhaus}
\end{antwort}

\end{enumerate}
\end{document}
