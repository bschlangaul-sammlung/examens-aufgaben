\documentclass{lehramt-informatik-aufgabe}
\liLadePakete{}
\begin{document}
\liAufgabenTitel{Wareneingänge}
\section{Normalisierung
\index{Normalformen}
\footcite{examen:66116:2012:03}
}

Gegeben sei folgende Datenbank für Wareneingänge eines Warenlagers. Die
Primärschlüssel-Attribute sind unterstrichen.

Artikelname

ZulieferungsNr |ArtikelNr |Datum Menge
1 1 01.01.2009 |Handschuhe [5

1 2 01.01.2009 |Mütze 10

2 3 05.01.2009 |Schal 2

2 1 05.01.2009 |Handschuhe 18

3 4 06.01.2009 |Jacke 2

\begin{enumerate}

%%
% a)
%%

\item Erläutern Sie, inwiefern obiges Schema die 3. Normalform verletzt.

%%
% b)
%%

\item Geben Sie für obige Datenbank alle vollen funktionalen
Abhängigkeiten (einschließlich der transitiven) an.

%%
% c)
%%

\item Überführen Sie das obige Relationenschema in die 3. Normalform.
Erläutern Sie die dazu durchzuführenden Schritte jeweils kurz.

\end{enumerate}

\end{document}
