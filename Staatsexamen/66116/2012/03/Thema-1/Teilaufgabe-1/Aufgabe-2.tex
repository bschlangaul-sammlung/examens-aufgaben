\documentclass{lehramt-informatik-aufgabe}
\liLadePakete{rmodell,normalformen}
\begin{document}
\let\FA=\liFunktionaleAbhaengigkeiten

\liAufgabenTitel{Wareneingänge}
\section{Normalisierung
\index{Normalformen}
\footcite{examen:66116:2012:03}
}

Gegeben sei folgende Datenbank für Wareneingänge eines Warenlagers. Die
Primärschlüssel-Attribute sind unterstrichen.

\begin{center}
\begin{tabular}{|l|l|l|l|l|}
\hline
ZulieferungsNr & ArtikelNr & Datum & Artikelname & Menge\\\hline
1 & 1 & 01.01.2009 & Handschuhe & 5\\\hline
1 & 2 & 01.01.2009 & Mütze & 10\\\hline
2 & 3 & 05.01.2009 & Schal & 2\\\hline
2 & 1 & 05.01.2009 & Handschuhe & 18\\\hline
3 & 4 & 06.01.2009 & Jacke & 2\\\hline
\end{tabular}
\end{center}

\begin{enumerate}

%%
% a)
%%

\item Erläutern Sie, inwiefern obiges Schema die 3. Normalform verletzt.

\begin{liAntwort}

\end{liAntwort}

%%
% b)
%%

\item Geben Sie für obige Datenbank alle vollen funktionalen
Abhängigkeiten (einschließlich der transitiven) an.

\begin{liAntwort}

\begin{liExkurs}[Voll funktionale Abhängigkeit]
Eine vollständig funktionale Abhängigkeit liegt dann vor, wenn dass
Nicht-Schlüsselattribut nicht nur von einem Teil der Attribute eines
zusammengesetzten Schlüsselkandidaten funktional abhängig ist, sondern
von allen Teilen eines Relationstyps. Die vollständig funktionale
Abhängigkeit wird mit der 2. Normalform (2NF) erreicht.
\liFussnoteLink{datenbank-verstehen.de}{https://www.datenbanken-verstehen.de/datenmodellierung/normalisierung/abhaengigkeiten-normalisierung/}
\end{liExkurs}

\begin{liExkurs}[Transitive Abhängigkeit]
Eine transitive Abhängigkeit liegt dann vor, wenn Y von X funktional
abhängig und Z von Y, so ist Z von X funktional abhängig. Diese
Abhängigkeit ist transitiv. Die transitive Abhängigkeit wird mit 3.
Normalform (3NF) erreicht.
\liFussnoteLink{datenbank-verstehen.de}{https://www.datenbanken-verstehen.de/datenmodellierung/normalisierung/abhaengigkeiten-normalisierung/}
\end{liExkurs}

\FA{
  ZulieferungsNr -> Datum;
  ArtikelNr -> Artikelname;
  ZulieferungsNr, ArtikelNr -> Menge;
}
\end{liAntwort}

%%
% c)
%%

\item Überführen Sie das obige Relationenschema in die 3. Normalform.
Erläutern Sie die dazu durchzuführenden Schritte jeweils kurz.

\begin{liAntwort}

\end{liAntwort}

\end{enumerate}

\end{document}
