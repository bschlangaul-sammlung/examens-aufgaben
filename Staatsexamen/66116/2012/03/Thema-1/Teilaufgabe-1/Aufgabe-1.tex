\documentclass{bschlangaul-aufgabe}

\begin{document}
\liAufgabenMetadaten{
  Titel = {Aufgabe 1},
  Thematik = {Handelsunternehmen},
  RelativerPfad = Staatsexamen/66116/2012/03/Thema-1/Teilaufgabe-1/Aufgabe-1.tex,
  ZitatSchluessel = examen:66116:2012:03,
  BearbeitungsStand = unbekannt,
  Korrektheit = unbekannt,
  Stichwoerter = {Entity-Relation-Modell},
  ExamenNummer = 66116,
  ExamenJahr = 2012,
  ExamenMonat = 03,
  ExamenThemaNr = 1,
  ExamenTeilaufgabeNr = 1,
  ExamenAufgabeNr = 1,
}

Ein Handelsunternehmen möchte seine Struktur verbessern und ein
Datenbanksystem zur Verwaltung seiner Filialen, angebotenen Waren und
Kunden erstellen.\index{Entity-Relation-Modell}
\footcite{examen:66116:2012:03}

Die Basis dieses Systems bilden die Filialen des Unternehmens. Jede
Filiale ist eindeutig durch ihre Filialnummer gekennzeichnet und
befindet sich in einer Stadt. Außerdem hat jede Filiale einen
Filialleiter.

Zu jeder Filiale gehört genau ein Lager mit einer eindeutigen
Lagernummer und ebenfalls einem Leiter. Jedes Lager verfügt über eine
bestimmte Menge an verschiedenen Waren. Jede Ware kann in mehreren
Lagern vorrätig sein und ist über eine Nummer, einen Namen und einen
Preis gekennzeichnet.

Ein Kunde kann in einer Filiale des Unternehmens Bestellungen aufgeben.
Der Kunde hat eine Kundennummer, einen Namen und eine Adresse, Eine
Bestellung enthält dabei jeweils einen Warenartikel, dessen gewünschte
Menge und das Datum, an dem die Bestellung abgeholt wird.

\begin{enumerate}

%%
% a)
%%

\item Erstellen Sie ein Entity-Relationship-Diagramm für obige Datenbank.

%%
% b)
%%

\item Setzen Sie das in Teilaufgabe a) erstellte
Entity-Relationship-Diagramm in ein Relationenschema um. Relationships
sollen mit einer möglichst geringen Anzahl von Relationen realisiert
werden. Dabei sind unnötige Redundanzen zu vermeiden. Ein
Relationenschema ist in folgender Form anzugeben: Relation (Attributl,
Attribut2, ...). Schlüsselattribute sind dabei zu unterstreichen. Achten
Sie bei der Wahl des Schlüssels auf Eindeutigkeit und Minimalität.

\end{enumerate}

\end{document}
