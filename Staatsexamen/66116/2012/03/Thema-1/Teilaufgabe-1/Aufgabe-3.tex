\documentclass{lehramt-informatik-aufgabe}
\liLadePakete{}
\begin{document}
\liAufgabenTitel{Gebrauchtwagen}
\section{SQL
\index{SQL}
\footcite{examen:66116:2012:03}
}

Gegeben sei das folgende Relationenschema:

Fahrzeug (MNR:int (3)

t '
ne a i
KMStand: int (

FZGNR: char (12) Baujahr:int (4),
5), P int

7
r Preis (5))

Modell (MNR:int (3), HNR:int (3), Typ:char{20), Neupreis:int (5),
ps:int(3)) ,

Hersteller (HNR:int (3), Name:char(20))

Dabei sind die Schlüsselattribute jeweils unterstrichen und zusätzlich für alle Attribute die Typen
angegeben. Formulieren Sie die folgenden Anfragen bzw. Anweisungen in SQL.

\begin{enumerate}

%%
% a)
%%

\item Geben Sie die Anweisungen in SQL-DDL an, die notwendig sind, um die Relationen „Fahrzeug“,
„Modell“ und „Hersteller“ zu erzeugen. Achten Sie dabei darauf, die Primärschlüssel der
Relationen zu kennzeichnen.

%%
% b)
%%

\item Bestimmen Sie die Typen aller Modelle des Herstellers mit Namen BMW.

%%
% c)
%%

\item Bestimmen Sie den Mindestpreis, bezogen auf das Attribut „Preis“, der Fahrzeuge eines jeden
Herstellers.

%%
% d)
%%

\item Bestimmen Sie die Namen der Hersteller, für die von jedem ihrer Modelle mindestens ein
Fahrzeug in der Datenbank gespeichert ist.

%%
% e)
%%

\item Bestimmen Sie die Namen aller Hersteller, von denen mindestens fünf Fahrzeuge eines beliebigen
Modells in der Datenbank gespeichert sind.
\end{enumerate}

\end{document}
