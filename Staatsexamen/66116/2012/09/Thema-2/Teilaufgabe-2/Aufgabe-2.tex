\documentclass{lehramt-informatik-aufgabe}
\liLadePakete{cpm,mathe,gantt}
\begin{document}
\liAufgabenTitel{Gantt und CPM}

\section{2 Projektmanagement
\index{CPM-Netzplantechnik}
\footcite[Thema 2 Teilaufgabe 2 Aufgabe 2]{examen:66116:2012:09}
}

Abbildung 2 stellt ein CPM-Netzwerk dar. Die Ereignisse sind fortlaufend
nummeriert (Nummer im Inneren der Kreise) und tragen keine Namen.
Gestrichelte Linien stellen Pseudo-Aktivitäten mit einer Dauer von 0
dar.

\begin{center}
\begin{tikzpicture}
\liCpmEreignis{1}{0}{2}
\liCpmEreignis{2}{1}{4}
\liCpmEreignis{3}{1}{0}
\liCpmEreignis{4}{3}{4}
\liCpmEreignis{5}{3}{2}
\liCpmEreignis{6}{3}{0}
\liCpmEreignis{7}{5}{4}
\liCpmEreignis{8}{5}{2}
\liCpmEreignis{9}{5}{0}
\liCpmEreignis{10}{7}{2}

\liCpmVorgang{1}{2}{10}
\liCpmVorgang{1}{3}{22}
\liCpmVorgang{1}{5}{6}
\liCpmVorgang{1}{6}{5}
\liCpmVorgang{2}{4}{8}
\liCpmVorgang{2}{5}{5}
\liCpmVorgang{3}{6}{8}
\liCpmVorgang{4}{5}{1}
\liCpmVorgang{4}{7}{12}
\liCpmVorgang{6}{9}{11}
\liCpmVorgang{7}{10}{6}
\liCpmVorgang{7}{8}{3}
\liCpmVorgang{8}{10}{7}
\liCpmVorgang{9}{10}{9}

\liCpmVorgang[schein]{5}{6}{}
\liCpmVorgang[schein]{5}{8}{}
\end{tikzpicture}
\end{center}

\begin{enumerate}

%%
% 1.
%%

\item Berechnen Sie die früheste Zeit für jedes Ereignis, wobei
angenommen wird, dass das Projekt zum Zeitpunkt 0 startet!

\begin{liAntwort}
\begin{tabular}{|l|r|r|}
\hline
$\text{FZ}_i$ & Nebenrechnung & \\\hline\hline
1 &                                                                           & $0$ \\\hline
2 &                                                                           & $10$ \\\hline
3 &                                                                           & $22$ \\\hline
4 & \f$\t{10}(1-2) + \t{8}(2-4) $                                             & $18$ \\\hline
5 & \f$\max(\v{10}(2) + 5,\v{6}(1),\v{18}(4) + 1) = \max(15,6,19)$            & $19$ \\\hline
6 & \f$\max(\v{5}(1),\v{22}(3) + 8, \v{19}(5) + 0) = \max(5, 30, 19)$         & $30$ \\\hline
7 & \f$\v{18}(4) + 12$                                                        & $30$ \\\hline
8 & \f$\max(\v{30}(7) + 3, \v{19}(5) + 0) = \max(33,19)$                      & $33$ \\\hline
9 & \f$\v{30}(6) + 11$                                                        & $41$ \\\hline
10 & \f$\max(\v{30}(7) + 6, \v{33}(8) + 7, \v{41}(9) + 9) = \max(36, 40, 50)$ & $50$ \\\hline
\end{tabular}
\end{liAntwort}

%%
% 2.
%%

\item Setzen Sie anschließend beim letzten Ereignis die späteste Zeit
gleich der frühesten Zeit und berechnen Sie die spätesten Zeiten!

\begin{liAntwort}
\begin{tabular}{|l|r|r|}
\hline
$\text{SZ}_i$ & Nebenrechnung & \\\hline\hline
1 & & $0$ \\\hline
2 & \f$\min(\v{28}(v) - 8, \v{30}(5) - 5)$               & $20$ \\\hline
3 & \f$\v{30}(6) - 8$                                    & $22$ \\\hline
4 & \f$\min(\v{30}(6) - 0, \v{40}(7) - 12)$              & $28$ \\\hline
5 & \f$\min(\v{30}(5) - 1, \v{43}(8) - 0)$               & $30$ \\\hline
6 & \f$\v{41}(9) - 11$                                   & $30$ \\\hline
7 & \f$\min(\v{50}(10) - 6, \v{43}(8) - 3) \min(44, 40)$ & $40$ \\\hline
8 & \f$\v{50}(10) - 7$                                   & $43$ \\\hline
9 & \f$\v{50}(10) - 9$                                   & $41$ \\\hline
10 & \f{}siehe $\text{FZ}_10$                            & $50$ \\\hline
\end{tabular}
\end{liAntwort}

%%
% 3.
%%

\item Berechnen Sie nun für jedes Ereignis die Pufferzeiten!

\begin{liAntwort}
\begin{tabular}{|l|l|l|l|l|l|l|l|l|l|l|}
\hline
i             & 1 & 2  & 3   & 4  & 5  & 6  & 7  & 8  & 9  & 10 \\\hline\hline
$\text{FZ}_i$ & 0 & 10 & 22  & 18 & 19 & 30 & 30 & 33 & 41 & 50 \\\hline
$\text{SZ}_i$ & 0 & 20 & 22  & 28 & 30 & 30 & 40 & 43 & 41 & 50 \\\hline
GP            & 0 & 10 & 0   & 10 & 11 & 0  & 10 & 10 & 0  & 0 \\\hline
\end{tabular}
\end{liAntwort}

%%
% 4.
%%

\item Bestimmen Sie den kritischen Pfad!

\begin{liAntwort}
$1 \rightarrow 3 \rightarrow 6 \rightarrow 9 \rightarrow 10$

\begin{center}
\begin{tikzpicture}[scale=0.8,transform shape]
\liCpmEreignis{1}{0}{2}
\liCpmEreignis{2}{1}{4}
\liCpmEreignis{3}{1}{0}
\liCpmEreignis{4}{3}{4}
\liCpmEreignis{5}{3}{2}
\liCpmEreignis{6}{3}{0}
\liCpmEreignis{7}{5}{4}
\liCpmEreignis{8}{5}{2}
\liCpmEreignis{9}{5}{0}
\liCpmEreignis{10}{7}{2}

\liCpmVorgang{1}{2}{10}
\liCpmVorgang{1}{3}{22}
\liCpmVorgang{1}{5}{6}
\liCpmVorgang{1}{6}{5}
\liCpmVorgang{3}{6}{8}
\liCpmVorgang{2}{5}{5}
\liCpmVorgang{2}{4}{8}
\liCpmVorgang{4}{7}{12}
\liCpmVorgang{7}{8}{3}
\liCpmVorgang{7}{10}{6}
\liCpmVorgang{9}{10}{9}
\liCpmVorgang{6}{9}{11}
\liCpmVorgang{8}{10}{7}
\liCpmVorgang{4}{5}{1}

\liCpmVorgang[schein]{5}{6}{}
\liCpmVorgang[schein]{5}{8}{}
\end{tikzpicture}
\end{center}
\end{liAntwort}

%%
% 5.
%%

\item Konvertieren Sie das Gantt-Diagramm\index{Gantt-Diagramm} aus
Abbildung 3 in ein CPM-Netzwerk! (10 Punkte)

\begin{center}
\begin{ganttchart}[x unit=0.75cm, y unit chart=0.8cm]{0}{11}
\gantttitlelist{0,...,11}{1} \\
\ganttbar[name=1]{1}{0}{1} \\
\ganttbar[name=2]{2}{2}{4} \\
\ganttbar[name=3]{3}{3}{3} \\
\ganttbar[name=4]{4}{6}{7} \\
\ganttbar[name=5]{5}{7}{11}

\node at (1) {2};
\node at (2) {3};
\node at (3) {1};
\node at (4) {2};
\node at (5) {5};

\ganttlink[link type=f-f]{3}{2}
\ganttlink[link type=f-s]{1}{2}
\ganttlink[link type=f-s]{1}{3}
\ganttlink[link type=f-s]{2}{4}
\ganttlink[link type=s-s]{4}{5}
\end{ganttchart}
\end{center}

\begin{liAntwort}
\begin{center}
\begin{tikzpicture}
\liCpmEreignis{SP}{-6}{0}
\liCpmEreignis{S1}{-5}{1.5}
\liCpmEreignis{E1}{-4.5}{0}
\liCpmEreignis{S2}{-3}{1.5}
\liCpmEreignis{E2}{-1.5}{1.5}
\liCpmEreignis{S3}{-3}{0}
\liCpmEreignis{E3}{-1.75}{0}
\liCpmEreignis{S4}{0}{1.5}
\liCpmEreignis{E4}{1.5}{1.5}
\liCpmEreignis{S5}{0.5}{0}
\liCpmEreignis{E5}{2}{0}
\liCpmEreignis{EP}{3.5}{0}

\liCpmVorgang{S1}{E1}{2}
\liCpmVorgang{S2}{E2}{3}
\liCpmVorgang{S3}{E3}{1}
\liCpmVorgang{S4}{E4}{2}
\liCpmVorgang{S5}{E5}{5}

\liCpmVorgang[schein]{SP}{S1}{}
\liCpmVorgang[schein]{E5}{EP}{}
\liCpmVorgang[schein]{E1}{S2}{}
\liCpmVorgang{E1}{S3}{1}
\liCpmVorgang{E3}{E2}{1}
\liCpmVorgang{E2}{S4}{1}
\liCpmVorgang{S4}{S5}{1}
\liCpmVorgang[schein]{E4}{EP}{}
\end{tikzpicture}
\end{center}
\end{liAntwort}
\end{enumerate}

\end{document}
