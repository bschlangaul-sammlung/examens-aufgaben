\documentclass{lehramt-informatik-aufgabe}
\begin{document}
\liAufgabenTitel{Client-Server-Modell}
\section{Aufgabe 8
\index{Client-Server-Modell}
\footcite{66116:2021:03}}

Das Client-Server-Modell ist ein Architekturmuster. Nennen Sie zwei
Vorteile einer nach diesem Muster gestalteten Architektur.

\begin{liAntwort}
\begin{enumerate}
\item Einfache Integration weiterer Clients

\item Prinzipiell uneingeschränkte Anzahl der Clients
\liFussnoteUrl{https://www.karteikarte.com/card/164928/vorteile-und-nachteile-des-client-server-modells}

\item Es muss  nur ein Server gewartet werden. Dies gilt \zB für
Updates, die einmalig und zentral auf dem Server durchgeführt werden und
danach für alle Clients verfügbar sind.
\liFussnoteUrl{https://www.eoda.de/wissen/blog/client-server-architekturen-performance-und-agilitaet-fuer-data-science/}
\end{enumerate}
\end{liAntwort}

\end{document}
