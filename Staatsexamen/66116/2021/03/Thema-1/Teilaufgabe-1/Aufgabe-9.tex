\documentclass{lehramt-informatik-aufgabe}
\begin{document}
\liAufgabenMetadaten{
  Titel = {Aufgabe 9},
  Thematik = {Client-Server-Technologien},
  RelativerPfad = Staatsexamen/66116/2021/03/Thema-1/Teilaufgabe-1/Aufgabe-9.tex,
  ZitatSchluessel = examen:66116:2021:03,
  BearbeitungsStand = unbekannt,
  Korrektheit = unbekannt,
  Stichwoerter = {Client-Server-Modell},
  ExamenNummer = 66116,
  ExamenJahr = 2021,
  ExamenMonat = 03,
  ExamenThemaNr = 1,
  ExamenTeilaufgabeNr = 1,
  ExamenAufgabeNr = 9,
}

Betrachten Sie die folgende Liste von Technologien:
\index{Client-Server-Modell}
\footcite{examen:66116:2021:03}

\begin{itemize}
\item Nodejs

\item PHP

\item CSS

\item AJAX

\item Python

\item Java
\end{itemize}

Welche dieser Technologien laufen in einem Client-Server-System
üblicherweise auf der Seite des Klienten und welche auf der Seite des
Servers? Nehmen Sie hierzu an, dass der Client ein Browser ist.

Übertragen Sie die im Folgenden gegebene Tabelle in Ihren
Bearbeitungsbogen und ordnen Sie die aufgelisteten Technologien anhand
der Buchstaben in die Tabelle ein. Fortsetzung nächste Seite!

Hinweis: Mehrfachzuordnungen sind möglich.

\begin{liAntwort}
\begin{tabular}{ll}
Client-seitige Technologien & Server-seitige Technologien \\
CSS & Nodejs \\
AJAX & PHP \\
Python & Python \\
Java & Java \\
\end{tabular}
\end{liAntwort}
\end{document}
