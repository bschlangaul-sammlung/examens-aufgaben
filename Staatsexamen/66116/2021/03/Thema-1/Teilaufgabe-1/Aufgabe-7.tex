\documentclass{lehramt-informatik-aufgabe}
\liLadePakete{syntax}
\begin{document}
\let\j=\liJavaCode

\liAufgabenTitel{MyParser}
\section{Aufgabe
\index{Einbringen von Abhängigkeiten (Dependency Injection)}
\footcite{examen:66116:2021:03}}

Lesen Sie die folgenden alternativen Codestücke.

\begin{enumerate}
\item

\begin{minted}{java}
public class MyParser {
  private InputStream input;

  public MyParser(String filePath) {

  }
}
\end{minted}

\item

\begin{minted}{java}
public class MyParser {
  private InputStream input;

  public MyParser(InputStream stream) {
  }
}
\end{minted}

\end{enumerate}
Beide Codestücke zeigen die Initialisierung einer Klasse namens
\j{MyParser}. Das zweite Codestück nutzt jedoch hierfür eine Technik
namens Abhängigkeits-Injektion (Dependency Injection).

\begin{enumerate}

%%
% a)
%%

\item Erklären Sie den Unterschied zwischen beiden Initialisierungen.
Hinweis: Sie können diese Aufgabe auch lösen, falls Sie die Technik
nicht kennen.

\begin{liAntwort}
Die Abhängigkeit von einer Instanz der Klasse InputStream wird erst bei
der Initialisierung des Objekt übergeben.
\end{liAntwort}

%%
% b)
%%

\item Benennen Sie einen Vorteil dieser Technik.

\begin{liAntwort}
Die Kopplung zwischen einer Klasse und ihrer Abhängigkeit wird verringert.
\footcite{wiki:dependency-injection}
\end{liAntwort}

\end{enumerate}
\end{document}
