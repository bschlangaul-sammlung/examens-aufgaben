\documentclass{lehramt-informatik-aufgabe}
\liLadePakete{syntax}
\begin{document}
\liAufgabenTitel{Verstoß gegen die Prinzipien guter objektorientierter Programmierung}
\section{Aufgabe 2
\index{Objektorientierung}
\footcite{66116:2021:03}}

Lesen Sie die folgenden Beispielcodes gründlich. Identifizieren Sie für
jeden Beispielcode den jeweiligen wesentlichen Verstoß gegen die
Prinzipien guter objektorientierter Programmierung. Benennen und
erklären Sie jeweils den Verstoß (Fehler) in einem Satz und erläutern
Sie für jeden Beispielcode, welche Probleme aus dem jeweiligen Fehler
resultieren können, ebenfalls in einem Satz.

\begin{enumerate}
\item \liJavaExamen{66116}{2021}{03}{Rectangle}

\begin{liAntwort}
Es sollte die Methode computeArea in der Klasse Rectangel implementiert
werden.
\end{liAntwort}

\item \liJavaExamen{66116}{2021}{03}{CalculateSpeed}

\begin{liAntwort}
Klassen sollten nach Objekten modelliert werden und nicht nach
Tätigkeiten (berechne Geschwindigkeit) Besser wäre der Name
SpeedCalculator gewesen. Außerdem sind die beiden Attribute kilometer
und miutes in der Main Methode so nicht ansprechbar, weil sie nicht
statisch sind.
\end{liAntwort}

\item \liJavaExamen{66116}{2021}{03}{Stack}

\begin{liAntwort}
Das Feld stck sollte private sein. So wird die interne Implemtation
verborgen.
\end{liAntwort}
\end{enumerate}

\end{document}
