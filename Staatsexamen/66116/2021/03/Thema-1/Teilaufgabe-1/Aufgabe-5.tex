\documentclass{lehramt-informatik-aufgabe}
\liLadePakete{syntax}
\begin{document}
\let\j=\liJavaCode

\liAufgabenTitel{MyList Kompositium}
\section{Aufgabe 5
\index{Einfach-verkettete Liste}
\footcite{66116:2021:03}}

Die folgende Abbildung stellt den Entwurf der Implementierung einer
verketteten Liste dar, welche Integer-Werte als Elemente enthalten kann.

% «interface»
% MyElement

% cakula®Sumf!: Integer
% de

% MyList

% geiSumf: Integer

% MyEndElement MyRegularElement

% next

% value: Infeger

Die Klassse \j{MyList} stellt die Methode \j{getSum()} zur Verfügung,
welche die Summe über alle in einer Liste befindlichen Elemente
berechnet. Ein Ausschnitt der Implementierung sieht folgendermaßen aus:

Gehen Sie im Folgenden davon aus, dass bereits Methoden existieren,
welche Elemente in die Liste einfügen können.

\begin{enumerate}

%%
% a
%%

\item Implementieren Sie in einer objektorientierten Programmiersprache
Ihrer Wahl, z.\,B. Java, die Methode \j{calculateSum()} der Klassen
\j{MyEndElement} und \j{MyRegularElement}, so dass rekursiv die Summe
der Elemente der Liste berechnet wird. Als Abbruchbedingung darf hierbei
nicht das Feld \j{MyRegluarElement.next} auf den Wert \j{null} überprüft
werden.

Hinweis: Gehen Sie davon aus, die Implementierung von \j{MyList}
garantiert, dass \j{MyRegluarElement.next} niemals den Wert \j{null}
annimmt, sondern das letzte hinzugefügte \j{MyRegularElement} auf eine
Instanz der Klasse \j{MyEndElement} verweist. Es gibt immer nur eine
Instanz der Klasse \j{MyEndElement} in einer Liste.

Hinweis: Achten Sie auf die Angabe einer korrekten Methodensignatur.

\begin{liAntwort}
\liJavaExamen{66116}{2021}{03}{my_list/MyElement}
\liJavaExamen{66116}{2021}{03}{my_list/MyEndElement}
\liJavaExamen{66116}{2021}{03}{my_list/MyList}
\end{liAntwort}

%%
% b
%%

\item Nennen Sie den Namen des Entwurfsmusters, auf welchem das oben
gegebene Klassendiagramm basiert, und ordnen Sie dieses in eine der
Kategorien von Entwurfsmustern ein.

Hinweis: Es genügt die Angabe eines Musters, falls Sie mehrere Muster
identifizieren sollten.

\begin{liAntwort}
Kompositium (Strukturmuster)
\end{liAntwort}
\end{enumerate}

\end{document}
