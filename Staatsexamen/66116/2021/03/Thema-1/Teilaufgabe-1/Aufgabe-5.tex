\documentclass{lehramt-informatik-aufgabe}
\liLadePakete{syntax,uml}
\begin{document}
\liAufgabenMetadaten{
  Titel = {Aufgabe 5},
  Thematik = {MyList Kompositium},
  RelativerPfad = Staatsexamen/66116/2021/03/Thema-1/Teilaufgabe-1/Aufgabe-5.tex,
  ZitatSchluessel = examen:66116:2021:03,
  BearbeitungsStand = unbekannt,
  Korrektheit = unbekannt,
  Stichwoerter = {Einfach-verkettete Liste},
  ExamenNummer = 66116,
  ExamenJahr = 2021,
  ExamenMonat = 03,
  ExamenThemaNr = 1,
  ExamenTeilaufgabeNr = 1,
  ExamenAufgabeNr = 5,
}

\let\j=\liJavaCode

Die folgende Abbildung stellt den Entwurf der Implementierung einer
verketteten Liste dar, welche Integer-Werte als Elemente enthalten kann.
\index{Einfach-verkettete Liste}
\footcite{examen:66116:2021:03}

\begin{center}
\begin{tikzpicture}
\umlclass[x=-1,y=0]{MyList}{}{getSum(): Integer}

\umlclass[x=4,y=0,type=interface]{MyElement}{}{calculateSum() : Integer}

\umlclass[x=2,y=-3]{MyEndElement}{}{}
\umlclass[x=6,y=-3]{MyRegularElement}{value : Integer}{}

\umlVHVreal{MyEndElement}{MyElement}
\umlVHVreal{MyRegularElement}{MyElement}

\umlHVHuniassoc[arm1=3cm,arg1=next,pos1=0.6,mult2=1,pos2=2.9]{MyRegularElement}{MyElement}

\umluniassoc[arg1=head,mult2=1,pos1=0.3]{MyList}{MyElement}
\end{tikzpicture}
\end{center}

\noindent
Die Klassse \j{MyList} stellt die Methode \j{getSum()} zur Verfügung,
welche die Summe über alle in einer Liste befindlichen Elemente
berechnet. Ein Ausschnitt der Implementierung sieht folgendermaßen aus:

\begin{minted}{java}
public class MyList {
  private MyElement head;

  public MyList() {
    this.head = new MyEndElement();
  }

  public int getSum() {
    // ..
  }
}

\end{minted}

\noindent
Gehen Sie im Folgenden davon aus, dass bereits Methoden existieren,
welche Elemente in die Liste einfügen können.

\begin{enumerate}

%%
% a
%%

\item Implementieren Sie in einer objektorientierten Programmiersprache
Ihrer Wahl, \zB Java, die Methode \j{calculateSum()} der Klassen
\j{MyEndElement} und \j{MyRegularElement}, so dass rekursiv die Summe
der Elemente der Liste berechnet wird.

Als Abbruchbedingung darf hierbei nicht das Feld
\j{MyRegluarElement.next} auf den Wert \j{null} überprüft werden.

Hinweis: Gehen Sie davon aus, die Implementierung von \j{MyList}
garantiert, dass \j{MyRegluarElement.next} niemals den Wert \j{null}
annimmt, sondern das letzte hinzugefügte \j{MyRegularElement} auf eine
Instanz der Klasse \j{MyEndElement} verweist. Es gibt immer nur eine
Instanz der Klasse \j{MyEndElement} in einer Liste.

Hinweis: Achten Sie auf die Angabe einer korrekten Methodensignatur.

\begin{liAntwort}
\liJavaExamen[firstline=18,lastline=20]{66116}{2021}{03}{my_list/MyElement}
\liJavaExamen[firstline=15,lastline=17]{66116}{2021}{03}{my_list/MyEndElement}
\end{liAntwort}

%%
% b
%%

\item Nennen Sie den Namen des Entwurfsmusters, auf welchem das oben
gegebene Klassendiagramm basiert, und ordnen Sie dieses in eine der
Kategorien von Entwurfsmustern ein.

Hinweis: Es genügt die Angabe eines Musters, falls Sie mehrere Muster
identifizieren sollten.

\begin{liAntwort}
Kompositium (Strukturmuster)
\end{liAntwort}
\end{enumerate}

\begin{liAdditum}
\liJavaExamen{66116}{2021}{03}{my_list/MyElement}
\liJavaExamen{66116}{2021}{03}{my_list/MyEndElement}
\liJavaExamen{66116}{2021}{03}{my_list/MyList}
\end{liAdditum}

\end{document}
