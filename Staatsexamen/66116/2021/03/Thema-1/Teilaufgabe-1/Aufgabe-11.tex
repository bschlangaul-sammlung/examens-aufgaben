\documentclass{lehramt-informatik-aufgabe}
\liLadePakete{}
\begin{document}
\liAufgabenTitel{HTTP}
\section{Aufgabe 11
\index{Client-Server-Modell}
\footcite{66116:2021:03}}

\begin{enumerate}
%%
% a)
%%

\item Was ist das Hypertext Transfer Protocol (HTTP) und wozu dient es?

\begin{liAntwort}
Das Hypertext Transfer Protocol ist ein zustandsloses Protokoll zur
Übertragung von Daten auf der Anwendungsschicht über ein Rechnernetz. Es
wird hauptsächlich eingesetzt, um Webseiten (Hypertext-Dokumente) aus
dem World Wide Web (WWW) in einen Webbrowser zu laden. Es ist jedoch
nicht prinzipiell darauf beschränkt und auch als allgemeines
Dateiübertragungsprotokoll sehr verbreitet.
\liFussnoteUrl{https://de.wikipedia.org/wiki/Hypertext_Transfer_Protocol}
\end{liAntwort}

%%
% b)
%%

\item Betrachten Sie die folgende Zeile Text. Um welche Art von Text
handelt es sich?

\url{https://developer.mozilla.org/en-US/search?q=client+servertoverview}

\begin{liAntwort}
Es handelt sich um eine HTTP-URL (Uniform Resource Locator). Die URL
lokalisiert eine Ressource, beispielsweise eine Webseite, über die zu
verwendende Zugriffsmethode (zum Beispiel das verwendete
Netzwerkprotokoll wie HTTP oder FTP) und den Ort (engl. location) der
Ressource in Computernetzwerken.
\liFussnoteUrl{https://de.wikipedia.org/wiki/Uniform_Resource_Locator}
\end{liAntwort}

%%
% c)
%%

\item Was sind die vier wesentlichen Bestandteile des Texts aus der
vorigen Teilaufgabe?

\begin{liAntwort}
\begin{description}
\item[Schema] \texttt{https://}
\item[Host] \texttt{developer.mozilla.org}
\item[Pfad] \texttt{/en-US/search}
\item[Query] \texttt{?q=client+servertoverview}
\end{description}
\liFussnoteUrl{https://de.wikipedia.org/wiki/Uniform_Resource_Locator}
\end{liAntwort}

\end{enumerate}
\end{document}
