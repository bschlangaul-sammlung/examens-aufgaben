\documentclass{lehramt-informatik-aufgabe}
\liLadePakete{entwurfsmuster}
\begin{document}
\liAufgabenTitel{Wissen Erbauer}
\section{Aufgabe 6
\index{Erbauer (Builder)}
\footcite{examen:66116:2021:03}}
\begin{enumerate}

%%
% a)
%%

\item Erläutern Sie den Zweck (Intent) des Erzeugungsmusters Erbauer in
max. drei Sätzen, ohne dabei auf die technische Umsetzung einzugehen.

\begin{liAntwort}
Die Erzeugung komplexer Objekte wird vereinfacht, indem der
Konstruktionsprozess in eine spezielle Klasse verlagert wird. Er wird so
von der Repräsentation getrennt und kann sehr unterschiedliche
Repräsentationen zurückliefern.
\footcite[Seite 29]{eilebrecht}
\end{liAntwort}

%%
% b)
%%

\item Erklären Sie, wie das Erzeugungsmuster Erbauer umgesetzt werden
kann (Implementierung). Die Angabe von Code ist hierbei NICHT notwendig!

%%
% c)
%%

\item Nennen Sie jeweils einen Vor- und einen Nachteil des
Erzeugungsmusters Erbauer im Vergleich zu einer Implementierung ohne
dieses Muster.

\begin{liAntwort}
\begin{description}
\item[Vorteil]

Die Implementierungen der Konstruktion und der Repräsentationen werden
isoliert. Die Erbauer verstecken ihre interne Repräsentation vor dem
Direktor.

\item[Nachteil]

Es besteht eine enge Kopplung zwischen Produkt, konkretem Erbauer und
den am Konstruktionsprozess beteiligten Klassen.\footcite{wiki:erbauer}
\end{description}
\end{liAntwort}

\end{enumerate}

\begin{liExkurs}[Erbauer (Builder)]
\liEntwurfsErbauerUml

\liEntwurfsErbauerAkteure
\end{liExkurs}
\end{document}
