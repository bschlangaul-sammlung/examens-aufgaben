\documentclass{bschlangaul-aufgabe}

\begin{document}
\bAufgabenMetadaten{
  Titel = {Aufgabe 3},
  Thematik = {Elementtypen UML-Klassendiagramm},
  RelativerPfad = Staatsexamen/66116/2021/03/Thema-1/Teilaufgabe-1/Aufgabe-3.tex,
  ZitatSchluessel = examen:66116:2021:03,
  BearbeitungsStand = unbekannt,
  Korrektheit = unbekannt,
  Stichwoerter = {UML-Diagramme},
  ExamenNummer = 66116,
  ExamenJahr = 2021,
  ExamenMonat = 03,
  ExamenThemaNr = 1,
  ExamenTeilaufgabeNr = 1,
  ExamenAufgabeNr = 3,
}

Wählen Sie bis zu fünf unterschiedliche Elementtypen aus folgendem
Diagramm aus und benennen Sie diese Elemente und ihre syntaktische
(nicht semantische) Bedeutung.
\index{UML-Diagramme}
\footcite{examen:66116:2021:03}

\begin{bAntwort}
\begin{description}
\item[Klasse] ConcreteCreator
\item[Interface] Produkt
\item[Abstrakte Klasse]  Creator
\item[Kommentar] return new ConcreteProdukt()
\end{description}
\end{bAntwort}

\end{document}
