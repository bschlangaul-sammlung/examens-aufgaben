\documentclass{lehramt-informatik-aufgabe}

\begin{document}
\liAufgabenMetadaten{
  Titel = {Aufgabe 12},
  Thematik = {Richtig-Falsch},
  RelativerPfad = Staatsexamen/66116/2021/03/Thema-1/Teilaufgabe-1/Aufgabe-12.tex,
  ZitatSchluessel = examen:66116:2021:03,
  BearbeitungsStand = unbekannt,
  Korrektheit = unbekannt,
  Stichwoerter = {Peer-to-Peer-Architektur},
  ExamenNummer = 66116,
  ExamenJahr = 2021,
  ExamenMonat = 03,
  ExamenThemaNr = 1,
  ExamenTeilaufgabeNr = 1,
  ExamenAufgabeNr = 12,
}

Es gibt Softwaresysteme, welche auf peer-to-peer (P2P) Kommunikation
basieren und eine entsprechende Architektur aufweisen.
\index{Peer-to-Peer-Architektur}
\footcite{examen:66116:2021:03}

\begin{enumerate}

%%
% a)
%%

\item Bewerten Sie die folgenden Aussagen als entweder richtig oder
falsch.

\begin{enumerate}
\item Mithilfe des Befehls “lookup” können Peers sich
gegenseitig identifizieren.

\begin{liAntwort}
richtig
\end{liAntwort}

\item In einem P2P-System, wie auch bei Client-Server, sind alle
Netzwerkteilnehmer gleichberechtigt.

\begin{liAntwort}
falsch. Im Client-Servermodell sind nicht alle Netzwerkteilnehmer
gleichberechtig. Der Server hat mehr Privilegien wie der Client.
\end{liAntwort}

\item Alle P2P-Systeme funktionieren grundsätzlich ohne einen
zentralen Verwaltungs-Peer.

\begin{liAntwort}
falsch. Es gibt zentralisierte P2P-Systeme (Beispiel: Napster), welche
einen zentralen Server zur Verwaltung benötigen, um zu funktionieren.
\liFussnoteUrl{https://de.wikipedia.org/wiki/Peer-to-Peer}
\end{liAntwort}

\item P2P kann auch für eine Rechner-Rechner-Verbindung
stehen.

\begin{liAntwort}
richtig
\end{liAntwort}

\item Es gibt strukturierte und unstrukturierte P2P-Systeme. In
unstrukturierten P2P-Systemen wird zum Auffinden von
Peers eine verteilte Hashtabelle verwendet (DHT).

\begin{liAntwort}
richtig
\end{liAntwort}

\item In einem P2P-System sind theoretisch alle Peers
gleichberechtigt, praktisch gibt es jedoch
leistungsabhängige Gruppierungen.

\begin{liAntwort}
richtig
\end{liAntwort}

\item Ein Peer kann sowohl ein Client wie auch ein Server für
einen anderen Peer sein.

\begin{liAntwort}
richtig
\end{liAntwort}
\end{enumerate}

%%
% b)
%%

\item Wählen Sie zwei falsche Aussagen aus der vorherigen Tabelle aus
und berichtigen Sie diese in jeweils einem Satz.

\begin{liAntwort}
Sie oben.
\end{liAntwort}
\end{enumerate}

\end{document}
