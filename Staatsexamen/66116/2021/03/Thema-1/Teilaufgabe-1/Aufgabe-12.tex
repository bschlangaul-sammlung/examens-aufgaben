\documentclass{lehramt-informatik-aufgabe}
\liLadePakete{}
\begin{document}
\liAufgabenTitel{}
\section{Aufgabe 12
\index{Peer-to-Peer-Architektur}
\footcite{66116:2021:03}}

Es gibt Softwaresysteme, welche auf peer-to-peer (P2P) Kommunikation basieren und eine
entsprechende Architektur aufweisen.

a) Bewerten Sie die folgenden Aussagen als entweder richtig oder falsch. Notieren Sie zu jeder
Nummer Ihre Entscheidung in Ihren Bearbeitungsbogen. (Keine Begründung erforderlich.)

Nr. Richtig? | Falsch? | Aussage

1 Mithilfe des Befehls “lookup” können Peers sich
gegenseitig identifizieren.

2 In einem P2P-System, wie auch bei Client-Server, sind alle
Netzwerkteilnehmer gleichberechtigt.

3 Alle P2P-Systeme funktionieren grundsätzlich ohne einen
zentralen Verwaltungs-Peer.

4 P2P kann auch für eine Rechner-Rechner-Verbindung
stehen.
Es gibt strukturierte und unstrukturierte P2P-Systeme. In

5 unstrukturierten P2P-Systemen wird zum Auffinden von
Peers eine verteilte Hashtabelle verwendet (DHT).
In einem P2P-System sind theoretisch alle Peers

6 gleichberechtigt, praktisch gibt es jedoch
leistungsabhängige Gruppierungen.

7 Ein Peer kann sowohl ein Client wie auch ein Server für
einen anderen Peer sein.

b) Wählen Sie zwei falsche Aussagen aus der vorherigen Tabelle aus und berichtigen Sie diese in
jeweils einem Satz.

\end{document}
