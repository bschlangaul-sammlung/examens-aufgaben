\documentclass{lehramt-informatik-aufgabe}
\liLadePakete{er}
\begin{document}
\let\a=\liErMpAttribute
\let\d=\liErDatenbankName
\let\e=\liErMpEntity
\let\r=\liErMpRelationship

\liAufgabenTitel{Automobilproduktion}
\section{Aufgabe 2
\index{Entity-Relation-Modell}
\footcite{66116:2021:03}}

\def\tmp#1#2{#1 #2}

\ExplSyntaxOn

\def\liLets#1{
  \prop_new:N \l_lets_prop
  \prop_put_from_keyval:Nn \l_lets_prop {#1}
  \prop_map_inline:Nn \l_lets_prop {##1 ##2}
}
\ExplSyntaxOff

\liLets{a = liErMpAttribute,
  d = liErDatenbankName,
  e = liErMpEntity,
  r = liErMpRelationship,
}

Erstellen Sie ein möglichst einfaches ER-Schema, das alle gegebenen
Informationen enthält. Attribute von Entitäten und Beziehungen sind
anzugeben, Schlüsselattribute durch Unterstreichen zu kennzeichnen.
Verwenden Sie für die Angabe der Kardinalitäten von Beziehungen die
Min-Max-Notation. Führen Sie Surrogatschlüssel (künstlich definierte
Schlüssel) nur dann ein, wenn es nötig ist, und modellieren Sie nur die
im Text vorkommenden Elemente.

\bigskip

Zunächst gibt es \e{Autos}, die einen eindeutigen \a{Namen}, einen
\a{Typ} sowie eine Liste an \a{Ausstattungen} besitzen. Autos werden aus
Bauteilen \r{zusammengesetzt}. Diese besitzen eine ID sowie eine
Beschreibung.

\bigskip

Jedes \e{Bauteil} wird von genau einem \e{Zulieferer}
geliefert. Zu jedem Zulieferer werden sein \a{Name} sowie seine
\a{E-Mail-Adresse} gespeichert.

\bigskip

Weiter gibt es \e{Werke}, die einen
eindeutigen \a{Namen} sowie einen \a{Standort} besitzen.

\bigskip

Jedes Werk
\r{besteht} aus \e{Hallen}, welche werksintern eindeutig \a{nummeriert}
sind. Zudem besitzt eine Halle noch eine gewisse \a{Größe} (in $m^2$).

\bigskip

Es gibt genau zwei Typen von Hallen: \e{Produktionshallen} und
\e{Ersatzteillager}.

\bigskip

In jeder Produktionshalle wird mindestens ein Auto
\r{hergestellt}.

\bigskip

Zu den Ersatzteillagern wird \r{festgehalten}, welche Bauteile und wie
viele davon sich dort befinden.

\bigskip

Zu jedem \e{Mitarbeiter} werden eine eindeutige \a{ID}, sein \a{Vor-}
und \a{Nachname}, die \a{Adresse (Straße, PLZ, Ort)}, das \a{Gehalt}
sowie das \a{Geschlecht} gespeichert.

\bigskip

Mitarbeiter werden unter anderem
in \e{Reinigungskräfte}, \e{Werksarbeiter} und \e{Ingenieure}
unterteilt.

\bigskip

Zu den Ingenieuren wird zusätzlich der
\a{Hochschulabschluss} gespeichert. Ingenieure sind genau einem Werk
\r{zugeordnet}, Werksarbeiter einer \r{Halle}.

\bigskip

Eine Reinigungskraft
\r{reinigt} mindestens eine Halle. Jede Halle muss regelmäßig gereinigt
werden.

\bigskip

Weiter sind Ingenieure Projekten \r{zugeteilt}. Zudem wird zu
jedem Projekt genau ein Ingenieur als \a{Projektleiter} festgehalten.
\end{document}
