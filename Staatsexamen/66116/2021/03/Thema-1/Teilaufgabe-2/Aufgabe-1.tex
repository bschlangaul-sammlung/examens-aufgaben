\documentclass{lehramt-informatik-aufgabe}
\liLadePakete{}
\begin{document}
\liAufgabenTitel{Vermischte Fragen}
\section{Aufgabe 1
\index{Datenbank}
\footcite{66116:2021:03}}

Beantworten Sie die folgenden Fragen und begründen oder erläutern Sie Ihre Antwort.
\begin{enumerate}

%%
% a)
%%

\item Erläutern Sie die Begriffe Kardinalität und Partizipität. Welche
Arten von Partizipität gibt es in der ER-Modellierung? Nennen und
erklären Sie diese kurz.

\begin{liAntwort}
\begin{description}
\item[Kardinalitäten]

Für die noch genauere Darstellung der Beziehungen im ER-Modell verwendet
man Kardinalitäten (auch Grad der Beziehungen genannt). Diese geben an
wie viele Entitätsinstanzen mit wie vielen Entitätsinstanzen einer
anderen Entitätsinstanz in Beziehung stehen.
\liFussnoteUrl{https://usehardware.de/datenbanksysteme-iv-entity-relationship-modell-er-modell-datenbankdarstellungen-i/}

\item[Partizipation] Die Partizipation eines Beziehungstyps (in einem
Entity-Relationship-Modell) bestimmt, ob alle Entities eines beteiligten
Entitätstyps an einer bestimmten Beziehung teilnehmen müssen.
\liFussnoteUrl{https://lehrbuch-wirtschaftsinformatik.org/glossar/kapitel03/Partizipation}

\begin{description}
\item[totale Partizipation:]

Wenn eine Beziehung Entität A und Entität B in Beziehung setzt, dann
muss ein Eintrag in Entität A existieren, damit ein Eintrag in Entität B
existiert und umgekehrt. Beide Entitäten müssen also an der Relation
teilnehmen. Eine Entitätsinstanz aus A kann also nicht ohne eine
in-Beziehung-stehende Entitätsinstanz aus B existieren und umgekehrt.

\item[partielle Partizipation:]

Wenn eine Beziehung Entität A mit Entität B in Beziehung setzt, dann
muss kein Eintrag in Entität A existieren, damit ein Eintrag in  Entität
B existieren kann und umgekehrt. Die beiden Entitäten müssen also nicht
an der Relation teilnehmen (enthalten sein).
\liFussnoteUrl{https://usehardware.de/datenbanksysteme-iv-entity-relationship-modell-er-modell-datenbankdarstellungen-i/}
\end{description}
\end{description}
\end{liAntwort}

%%
% b)
%%

\item Mit welchen beiden Befehlen kann eine Transaktion beendet werden?
Nennen Sie diese und erklären Sie den Unterschied.

%%
% c)
%%

\item Erläutern Sie den Unterschied zwischen einer kurzen und einer
langen Sperre.

%%
% d)
%%

\item Stellen Sie außerdem die Kompatibilitätsmatrix zur Umsetzung des
ACID-Prinzips mit den richtigen Werten dar. S stehe dabei für eine Lese-
und X für eine Schreibsperre.

%%
% e)
%%

\item Nennen und erklären Sie kurz die Armstrong-Axiome. Sind diese
vollständig und korrekt?

%%
% f)
%%

\item Was versteht man unter einem (Daten-) Katalog (Data Dictionary)
und was enthält dieser (es genügt eine Auswahl zu nennen)?

%%
% g)
%%

\item Erklären Sie das konservative und das strikte
Zwei-Phasen-Sperrprotokoll.

%%
% h)
%%

\item Erklären Sie die Begriffe „Steal/NoSteal“ und „Force/NoForce“ im
Kontext der Systempufferverwaltung eines DBS.

\end{enumerate}
\end{document}
