\documentclass{lehramt-informatik-aufgabe}
\liLadePakete{mathe}
\begin{document}
\liAufgabenTitel{}
\section{Aufgabe 5
\index{Transaktionen}
\footcite{66116:2021:03}}

Gegeben sind die folgenden transaktionsähnlichen Abläufe. (Zunächst wird
auf das Setzen von Sperren verzichtet.) Hierbei steht R(X) für ein
Lesezugriff auf X und W(X) für einen Schreibzugriff auf X.

\begin{center}
\begin{tabular}{|l|l|}
\hline
T1 & T2 \\\hline
R(A) & R(D) \\
A := A-10 & D := D-20 \\
W(A) & W(D) \\
R(C)   & R(A) \\
R(B) & A := A+20 \\
B := B+10 & W(A) \\
W(B)      &
\\\hline
\end{tabular}
\end{center}

\noindent
Betrachten Sie folgenden Schedule:

\begin{center}
\begin{tabular}{|l|l|}
\hline
T1 & T2 \\\hline

R(A) & \\

& R(D) \\
& D := D-20 \\
& W(D) \\
& R(A) \\
& A := A+20 \\
& W(A) \\

A := A-10 & \\
W(A) & \\
R(C) & \\
R(B) & \\
B := B+10 & \\
W(B) & \\\hline
\end{tabular}
\end{center}

\begin{enumerate}

%%
% a)
%%

\item Geben Sie die Werte von $A$, $B$, $C$ und $D$ nach Ablauf des Schedules
an, wenn mit $A = 100$, $B = 200$, $C = \text{true}$ und $D = 150$ begonnen wird.

\begin{liAntwort}
\begin{description}
\item[A] 90 (A := A - 10 := 100 - 10) T2 schreibt 120 in A, was aber von T1 wiederüberschrieben wird.
\item[B] 210 (B wird nur in T1 gelesen, verändert und geschriebe)
\item[C] true (C wird nur in T1 gelesen)
\item[D] 130 (D wird nur in T2 gelesen, verändert und geschrieben)
\end{description}
\end{liAntwort}

%%
% b)
%%

\item Geben Sie den Dependency-Graphen des Schedules an.

\begin{liAntwort}

\end{liAntwort}

%%
% c)
%%

\item Geben Sie alle auftretenden Konflikte an.

\begin{liAntwort}

\end{liAntwort}

%%
% d)
%%

\item Begründen Sie, ob der Schedule serialisierbar ist.

\begin{liAntwort}

\end{liAntwort}

%%
% e)
%%

\item Beschreiben Sie, wie die beiden Transaktionen mit LOCK Aktionen
erweitert werden können, so dass nur noch serialisierbare Schedules
ausgeführt werden können. Die Angabe eines konkreten Schedules ist nicht
zwingend notwendig.

\begin{liAntwort}

\end{liAntwort}

\end{enumerate}
\end{document}
