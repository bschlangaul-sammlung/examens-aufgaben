\documentclass{lehramt-informatik-aufgabe}
\liLadePakete{syntax}
\begin{document}
\liAufgabenTitel{Fußballweltmeisterschaft}
\section{Aufgabe 6
\index{SQL}
\footcite{66116:2021:03}}

Gegeben ist folgendes Relationenschema zur Verwaltung von Daten aus der Fußballweltmeisterschaft:

NATION (Land, Kapitän, Trainer)
Kapitän ist Fremdschlüssel zu Spieler\_ID in SPIELER.

MATCH (Match\_ID, Ort, Datum, Teaml, Team2, ToreTeaml, ToreTeam2)
Teami ist Fremdschlüssel zu Land in NATION.
Team2 ist Fremdschlüssel zu Land in NATION.

SPIELER (Spieler\_ID, Name, Vorname, Wohnort, Land)
Land ist Fremdschlüssel zu Land in NATION.

PLATZVERWEISE (Platzverweis\_ID,Match\_ID, Spieler\_ID, Spielminute)
Match\_ID ist Fremdschlüsssel zu Match\_ID in MATCH.
Spieler\_ID ist Fremdschlüssel zu Spieler\_ID in SPIELER.

Die Primärschlüssel der Relationen sind wie üblich durch Unterstreichen
gekennzeichnet. Pro Ort und Datum findet jeweils nur ein Spiel statt.

Formulieren Sie folgende Abfragen in SQL. Vermeiden Sie nach Möglichkeit
übermäßige Nutzung von Joins und Views.
\begin{enumerate}

%%
% a)
%%

\item Ermitteln Sie die Anzahl der Platzverweise pro Spieler und geben
Sie jeweils Name und Vorname des Spielers mit aus. Die Ausgabe soll nach
der Anzahl der Platzverweise absteigend erfolgen.

\begin{liAntwort}
\begin{minted}{sql}
SELECT COUNT(*) AS Anzahl, S.Name, S.Vorname
FROM Platzverweise P, Spieler S
WHERE P.Spieler_ID = S.Spieler_ID
GROUP BY P.Spieler_ID
ORDER BY Anzahl DESC;
\end{minted}
\end{liAntwort}

%%
% b)
%%

\item Welches ist die maximale Anzahl an Toren, die eine Mannschaft
insgesamt im Turnier erzielt hat? (Sie dürfen der Einfachheit halber
annehmen, dass jede Mannschaft jeweils mindestens einmal als Teami und
Team2 angetreten ist.)

\begin{liAntwort}
\begin{minted}{sql}
SELECT Team1Tore.Team, SUM(Team1Tore.Summe + Team2Tore.Summe) AS Tore
FROM (SELECT Team1 AS Team, SUM(ToreTeam1) AS Summe
 FROM Match
 GROUP BY Team1) AS Team1Tore,
 (SELECT Team2 AS Team, SUM(ToreTeam2) AS Summe
 FROM Match
 GROUP BY Team2) AS Team2Tore
WHERE Team1Tore.Team = Team2Tore.Team
\end{minted}

\begin{minted}{sql}
SELECT Team1 AS Team, SUM(ToreTeam1) AS Summe
FROM Match
GROUP BY Team1
UNION
SELECT Team2 AS Team, SUM(ToreTeam2) AS Summe
FROM Match
GROUP BY Team2
\end{minted}
\end{liAntwort}

%%
% c)
%%

\item Wie viele Tore sind im Turnier insgesamt gefallen?

\begin{liAntwort}
\begin{minted}{sql}
SELECT SUM(ToreTeam1 + ToreTeam2) AS GesamtanzahlTore
FROM Match
\end{minted}
\end{liAntwort}

%%
% d)
%%

\item Ermitteln Sie die Namen und Länder der fünf Spieler, die nach der
kürzesten Spielzeit einen Platzverweis erhielten. Die Ausgabe soll
nummeriert erfolgen (beginnend bei 1 für die kürzeste Spielzeit).

\begin{liAntwort}
\begin{minted}{sql}
SELECT S.Name, S.Land, COUNT(*) AS Rang
FROM Spieler S, Platzverweise P1, Platzverweise P2
WHERE S.Spieler_ID = P2.Spieler_ID
AND P1.Spielminute <= P2.Spielminute
GROUP BY P2.Spieler_ID
HAVING COUNT(*) < 6
ORDER BY Rang;
\end{minted}
\end{liAntwort}

\end{enumerate}
\end{document}
