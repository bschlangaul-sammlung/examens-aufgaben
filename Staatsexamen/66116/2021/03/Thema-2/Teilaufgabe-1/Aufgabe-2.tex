\documentclass{lehramt-informatik-aufgabe}
\liLadePakete{}
\begin{document}
\liAufgabenTitel{Entwicklungsprozesse}
\section{Aufgabe 2
\index{Prozessmodelle}
\footcite{66116:2021:03}}

\begin{enumerate}

%%
% a)
%%

\item Erklären Sie den Unterschied zwischen \emph{iterativen} und
\emph{inkrementellen} Entwicklungsprozessen. Nennen Sie zudem je ein
Prozessmodell als Beispiel.\index{Iterative Prozessmodelle}
\index{Inkrementelle Prozessmodelle}

\begin{liAntwort}
\begin{description}
\item[Iterativ:]

Ein Entwicklungszyklus wird immer wieder durchlaufen:

Planung $\rightarrow$ Implementierung $\rightarrow$ Testung
$\rightarrow$ Evaluation.

Mit jeder Iteration wird das Produkt verfeinert. Das Wasserfallmodell in
seiner normalen Form würde dies nicht erfüllen, da hier alle Phasen nur
einmal durchlaufen werden.

\textbf{Beispiel:} Agile Programmierung, erweitertes Wasserfallmodell

\item[Inkrementell:]

Das Projekt wird in einzelne Teile zerlegt. An jedem Teilprojekt kann
bestenfalls separat gearbeitet werden. In den einzelnen Teilprojekten
kann dann ebenfalls wieder iterativ gearbeitet werden, die beiden
Methoden schließen sich gegenseitig also nicht aus.

\textbf{Beispiel:} Agile Programmierung, V-Modell XT, Aufteilung in
Teilprojekte, die alle mit Wasserfallmodell bearbeitet werden
\end{description}
\end{liAntwort}

%%
% b)
%%

\item Nennen und erklären Sie kurz die vier Leitsätze der agilen
Softwareentwicklung laut \emph{Agilem Manifest}.\index{Agiles Manifest}

\begin{liAntwort}
\begin{enumerate}
\item Individuen und Interaktionen sind wichtiger als Prozesse und
Werkzeuge

Ein festgelegter Prozess, der nicht sinnvoll von den beteiligten
Personen umgesetzt werden kann, ist nicht sinnvoll und sollte nicht
durchgeführt werden. Es geht darum, die Stärken der Mitarbeiter
einzusetzen und sich nicht sklavisch an Abläufen zu orientieren. Es geht
darum, Freiräume zu schaffen, um das Potential des Einzelnen sich voll
entfalten zu lassen.

\item Funktionierende Software ist wichtiger als umfassende
Dokumentation

Der Kunde ist in erster Linie an einem funktionierenden Produkt
interessiert. Es soll möglichst früh ein Prototyp erforderlich, der noch
weiter an die Bedürfnisse des Kunden angepasst wird. Eine umfassende
Dokumentation kann dann am Ende immer noch ergänzt werden.

\item Zusammenarbeit mit dem Kunden ist wichtiger als
Vertragsverhandlung

Anforderungen und Spezifikationen können sich flexibel ändern, dadurch
kann der Kunde direkt Rückmeldung geben. \dh es ist nicht wichtig im
Vertrag kleinste Details zu regeln, sondern auf die Bedürfnisse des
Kunden einzugehen. Daraus folgt direkt der letzte Punkt:

\item Reagieren auf Veränderung ist wichtiger als das Befolgen eines
Plans

Softwareentwicklung ist ein dynamischer Prozess. Das starre Befolgen
eines Plans widerspricht der Grundidee der agilen Programmierung.
\footcite[Seite 229]{schneider}
\end{enumerate}
\end{liAntwort}

%%
% c)
%%

\item Beschreiben Sie die wesentlichen Aktivitäten in Scrum (agiles
Entwicklungsmodell) inklusive deren zeitlichen Ablaufs. Gehen Sie dabei
auch auf die Artefakte ein, die im Verlauf der Entwicklung erstellt
werden.\index{SCRUM}

\begin{liAntwort}
\begin{enumerate}
\item Initiale Projekterstellung:

Festlegen eines Product Backlog durch den Product Owner, Erstellen von
ersten User Stories. Diese entstehen gegebenenfalls durch den Kontakt
mit den Stakeholdern, falls der Product Owner nicht selbst der Kunde
ist.

\item Sprint planning:

Der nächste Sprint (zwischen 1 und 4 Wochen) wird geplant. Dabei wird
ein Teil des Product Backlog als Sprint Backlog definiert und auf die
einzelnen Entwickler verteilt. Es wird festgelt, was implementiert
werden soll (Product Owner anwesend) und im zweiten Teil wie das
geschehen soll (Entwicklerteam). Während des Sprints findet jeden Tag
ein Daily Scrum statt, ein fünfzehnminütiger Austausch zum aktuellen
Stand. Am Ende des Sprints gibt es ein Sprint Review und das Product
Inkrement wird evaluiert und das Product Backlog gegebenenfalls
angepasst in Absprache mit Product Owner und
Stakeholdern.

\liPseudoUeberschrift{Artefakte:}

\begin{itemize}
\item Product Backlog
\item Sprint Backlog
\item Product Increment\footcite[Seite 230-231]{schneider}
\end{itemize}
\end{enumerate}
\end{liAntwort}

%%
% d)
%%

\item Nennen und erklären Sie die Rollen in einem Scrum-Team.
\index{SCRUM}

\begin{liAntwort}
\begin{description}
\item[Product Owner]

Verantwortlich für das Projekt, für die Reihenfolge der Bearbeitung und
Implementierung, ausgerichtet auf Maximierung und wirtschaftlichen
Erfolg. Er führt und aktualisiert das Product Backlog.

\item[Scrum Master]

Führungsperson, die das Umsetzen des Scrum an sich leitet und begleitet.
Er mischt sich nicht mit konkreten Arbeitsanweisungen ein, sondern
moderiert und versucht Hindernisse zu beseitigen, die einem Gelingen der
Sprintziele im Weg sind.

\item[Entwickler]

Entwicklung und Implementierung der einzelnen Produktfunktionalitäten,
die vom Product Owner festgelegt wurden in eigenverantwortlicher Zeit.
Es müssen die Qualitätsstandards beachten. Decken in der Regel für das
Projekt voraussichtlich wichtige Fachbereiche ab.
\end{description}
\end{liAntwort}
\end{enumerate}
\end{document}
