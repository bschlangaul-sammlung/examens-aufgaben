\documentclass{lehramt-informatik-aufgabe}
\liLadePakete{}
\begin{document}
\liAufgabenTitel{}
\section{
\index{Prozessmodelle}
\footcite{66116:2021:03}}

Aufgabe 2 (Entwicklungsprozesse) (20 Punkte)

a) Erklären Sie den Unterschied zwischen iterativen und inkrementellen Entwicklungsprozessen.
Nennen Sie zudem je ein Prozessmodell als Beispiel.

b) Nennen und erklären Sie kurz die vier Leitsätze der agilen Softwareentwicklung laut Agilem
Manifest.

c) Beschreiben Sie die wesentlichen Aktivitäten in Scrum (agiles Entwicklungsmodell) inklusive
deren zeitlichen Ablaufs. Gehen Sie dabei auch auf die Artefakte ein, die im Verlauf der Ent-
wicklung erstellt werden.

d) Nennen und erklären Sie die Rollen in einem Scrum-Team.
\end{document}
