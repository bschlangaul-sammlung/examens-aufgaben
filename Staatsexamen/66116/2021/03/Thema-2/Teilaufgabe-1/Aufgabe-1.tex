\documentclass{lehramt-informatik-aufgabe}
\liLadePakete{}
\begin{document}
\liAufgabenTitel{}
\section{
\index{CPM-Netzplantechnik}
\footcite{66116:2021:03}}

Aufgabe 1 (Projektmanagement) (15 Punkte)

Gegeben seien folgende Tätigkeiten mit ihren Abhängigkeiten und Dauern:

Task Dauer (inh) Abhängigkeiten

Tl 3 /
T2 6 /
T3 2 Tl
T4 2 T2
T5 5 Tl
T6 3 T4, T5
T7 6 T3
T8 7 T4
T9 4 T6, T8
T10 1 T7, T9

a) Zeichnen Sie ein CPM-Diagramm basierend auf der gegebenen Aufgabenliste. Benutzen Sie ex-
plizite Start- und Endknoten.

b) Als Slack bezeichnet man die Zeit, um die eine Aufgabe bezüglich ihres frühesten Startzeitpunk-
tes verzögert werden kann, ohne dass es Probleme bei der fristgerechten Fertigstellung des Pro-
jektes gibt. Berechnen Sie den Slack für alle Aktivitäten und ergänzen Sie ihn in Ihrem Dia-
gramm.

c) Zeichnen Sie den kritischen Pfad in Ihr Diagramm ein oder geben Sie die Tasks des kritischen
Pfades in der folgenden Form an: Start !  : : ! Ende. Sollte es mehrere kritische Pfade geben,
geben Sie auch diese an. Wie lange ist die Dauer des kritischen Pfades bzw. der kritischen Pfade?

\end{document}
