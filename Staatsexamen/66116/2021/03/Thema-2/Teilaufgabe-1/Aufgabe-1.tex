\documentclass{lehramt-informatik-aufgabe}
\liLadePakete{cpm}
\begin{document}
\liAufgabenTitel{Projektmanagement}
\section{Aufgabe 1
\index{CPM-Netzplantechnik}
\footcite{66116:2021:03}}

Gegeben seien folgende Tätigkeiten mit ihren Abhängigkeiten und Dauern:

\begin{center}
\begin{tabular}{lll}
\hline
Task & Dauer (in h) & Abhängigkeiten \\\hline
T1   & 3            & / \\
T2   & 6            & / \\
T3   & 2            & T1 \\
T4   & 2            & T2 \\
T5   & 5            & Tl \\
T6   & 3            & T4, T5 \\
T7   & 6            & T3 \\
T8   & 7            & T4 \\
T9   & 4            & T6, T8 \\
T10  & 1            & T7, T9 \\\hline
\end{tabular}
\end{center}
\begin{enumerate}

%%
% a)
%%

\item Zeichnen Sie ein CPM-Diagramm basierend auf der gegebenen
Aufgabenliste. Benutzen Sie explizite Start- und Endknoten.

\begin{liAntwort}
\liPseudoUeberschrift{Abkürzungen}

\begin{description}
\item[S] Start
\item[1S] Start von T1
\item[1E] Ende von T1
\item[E] Ende
\end{description}

\begin{center}
\begin{tikzpicture}[x=1.5cm,y=1.5cm,scale=0.9,transform shape]
\liCpmEreignis{S}{0}{1}

\liCpmEreignis{1S}{1}{1}
\liCpmEreignis{1E}{2}{1}

\liCpmEreignis{2S}{0}{2}
\liCpmEreignis{2E}{1}{2}

\liCpmEreignis{3S}{3}{0}
\liCpmEreignis{3E}{4}{0}

\liCpmEreignis{4S}{2}{2}
\liCpmEreignis{4E}{3}{2}

\liCpmEreignis{5S}{3}{1}
\liCpmEreignis{5E}{4}{1}

\liCpmEreignis{6S}{4}{2}
\liCpmEreignis{6E}{5}{2}

\liCpmEreignis{7S}{5}{0}
\liCpmEreignis{7E}{6}{0}

\liCpmEreignis{8S}{3}{3}
\liCpmEreignis{8E}{4}{3}

\liCpmEreignis{9S}{5}{3}
\liCpmEreignis{9E}{6}{3}

\liCpmEreignis{10S}{6}{1}
\liCpmEreignis{10E}{7}{1}

\liCpmEreignis{E}{7}{2}

\liCpmVorgang{1S}{1E}{3}
\liCpmVorgang{2S}{2E}{6}
\liCpmVorgang{3S}{3E}{2}
\liCpmVorgang{4S}{4E}{2}
\liCpmVorgang{5S}{5E}{5}
\liCpmVorgang{6S}{6E}{3}
\liCpmVorgang{7S}{7E}{6}
\liCpmVorgang{8S}{8E}{7}
\liCpmVorgang{9S}{9E}{4}
\liCpmVorgang{10S}{10E}{1}

\liCpmVorgang[schein]{S}{1S}{}
\liCpmVorgang[schein]{S}{2S}{}

\liCpmVorgang[schein]{1E}{3S}{}

\liCpmVorgang[schein]{2E}{4S}{}

\liCpmVorgang[schein]{1E}{5S}{}

\liCpmVorgang[schein]{4E}{6S}{}
\liCpmVorgang[schein]{5E}{6S}{}

\liCpmVorgang[schein]{3E}{7S}{}

\liCpmVorgang[schein]{4E}{8S}{}

\liCpmVorgang[schein]{6E}{9S}{}
\liCpmVorgang[schein]{8E}{9S}{}

\liCpmVorgang[schein]{7E}{10S}{}
\liCpmVorgang[schein]{9E}{10S}{}

\liCpmVorgang[schein]{10E}{E}{}
\end{tikzpicture}
\end{center}
\end{liAntwort}

%%
% b)
%%

\item Als Slack bezeichnet man die Zeit, um die eine Aufgabe bezüglich
ihres frühesten Startzeitpunktes verzögert werden kann, ohne dass es
Probleme bei der fristgerechten Fertigstellung des Projektes gibt.
Berechnen Sie den Slack für alle Aktivitäten und ergänzen Sie ihn in
Ihrem Diagramm.

%%
% c)
%%

\item Zeichnen Sie den kritischen Pfad in Ihr Diagramm ein oder geben
Sie die Tasks des kritischen Pfades in der folgenden Form an: Start !
$\dots$ ! Ende. Sollte es mehrere kritische Pfade geben, geben Sie auch
diese an. Wie lange ist die Dauer des kritischen Pfades bzw. der
kritischen Pfade?

\end{enumerate}
\end{document}
