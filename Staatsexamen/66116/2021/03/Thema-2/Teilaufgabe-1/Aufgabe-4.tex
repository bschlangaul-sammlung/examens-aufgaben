\documentclass{lehramt-informatik-aufgabe}
\liLadePakete{}
\begin{document}
\liAufgabenTitel{}
\section{Aufgabe 4
\index{Entwurfsmuster}
\footcite{66116:2021:03}}

Aufgabe 4 (Entwurfsmuster) (20 Punkte)

a) Nennen und erklären Sie kurz die drei Kategorien der Organisation von klassischen Entwurfsmus-
tern. Geben Sie zu jeder Kategorie ein Beispiel-Pattern an.

b) Erstellen Sie ein Klassendiagramm zu den Komponenten einer Beobachtungsstation, welche aus
einem einzigen Wasserstandsmesser besteht. Dabei sollen bei Änderungen des Wasserstandes der
aktuelle Wasserstand sowohl auf der Konsole als auch in eine Log-Datei geschrieben werden. Der
momentane Wasserstand soll dabei mittels einer Variablen des Datentyps double dargestellt
werden. Die verschiedenen Anzeigearten (Konsolenanzeige, Logger) sollen als verschiedene Klas-
sen modelliert werden und enthalten jeweils nur eine Methode zum Anzeigen bzw. Schreiben des
aktuellen Wasserstandes. Verwenden Sie für die Realisierung dieses Klassendiagramms die pas-
senden Entwurfsmuster.

Hinweise:
e Getter und Setter müssen nicht eingezeichnet werden.

e Fügen Sie auch Methoden ein, welche durch die einzelnen Klassen implementiert werden.

\end{document}
