\documentclass{lehramt-informatik-aufgabe}
\liLadePakete{syntax}
\begin{document}
\liAufgabenTitel{Webshop}
\section{Aufgabe 5
\index{Implementierung in Java}
\footcite{66116:2021:03}}

\begin{enumerate}

%%
% a)
%%

\item Nennen Sie vier Programmierparadigmen.

%%
% b)
%%

\item Erlautern Sie die Begriffe Overloading und Overriding, sowie deren
Unterschiede.

%%
% c)
%%

\item Erlautern Sie, wie sich zentrale und dezentrale Versionsverwaltung
unterscheiden.

%%
% d)
%%

\item Eiıstellen Sie ein Sequenzdiagramm zur Methode main der Klasse
Webshop.

Hinweise:
\begin{itemize}
\item Arithmetische Operationen müssen nicht weiter aufgelöst werden.

\item Listenoperationen müssen nicht explizit dargestellt werden.

\item Auf das Zeichnen einer passiven Lebenslinie muss nicht geachtet werden.

\item Übertragen Sie das untenstehende Diagramm als Ausgangspunkt in Ihren Bearbeitungsbogen.
\end{itemize}

\end{enumerate}

\liJavaExamen{66116}{2021}{03}{webshop/Webshop}
\liJavaExamen{66116}{2021}{03}{webshop/Artikel}
\liJavaExamen{66116}{2021}{03}{webshop/Bestellung}
\end{document}
