\documentclass{lehramt-informatik-aufgabe}
\liLadePakete{}
\begin{document}
\liAufgabenTitel{}
\section{Aufgabe 5
\index{Implementierung in Java}
\footcite{66116:2021:03}}

 (Implementierung) (20 Punkte)
a) Nennen Sie vier Programmierparadigmen.
b) Erlautern Sie die Begriffe Overloading und Overriding, sowie deren Unterschiede.
c)  Erlautern Sie, wie sich zentrale und dezentrale Versionsverwaltung unterscheiden.
d)  Eiıstellen Sie ein Sequenzdiagramm zur Methode main der Klasse Webshop.
Hinweise:
e Arithmetische Operationen müssen nicht weiter aufgelöst werden.
e Listenoperationen müssen nicht explizit dargestellt werden.
e Aufdas Zeichnen einer passiven Lebenslinie muss nicht geachtet werden.
e Übertragen Sie das untenstehende Diagramm als Ausgangspunkt in Ihren Bearbeitungsbogen.
w: Webshop
create
>» b1:Bestellung
public class Webshop {
1
2 public static void main(String[] args) {
3
4 Bestellung bl = new Bestellung();
5
6 // ab hier soll modelliert werden
7 Artikel al = new Artikel ();
8 al.setName ("Taschenrechner");
9 al.setPrice (10);
10
11 bl.addArticle (al);
12
" Bestellung b2 = new Bestellung ();
te Artikel a2 = new Artikel();
17 a2.setName ("Lineal");
a2.setPrice (2.5);
18
5 Artikel a3 = new Artikel();
>1 a3.setName ("Bleistift");
a3.setPrice (0.7);
22
5, b2.addArticle (a3);
ee bl.addArticle(a2);
27 .
28 bl.getSize();
29
30 b2.getPrice();
31 }
32 |}
33

1 | public class Artikel {

2

3 private String name;

4 private double price;

5

6 public void setName(String name) {

7 this.name = name;

a} 3

9
10 public void setPrice (double price) {
il this.price = price;
12 }

13

14 public double getPrice() {

15 return price;

16 }

171}

1] import java.util.List;

2

3 | public class Bestellung {

4

5 private List<Artikel> articles;

6 //Anzahl an Artikeln

7 private int size = 0;

8 //Gesamtpreis der Bestellung

9 private double price = 0;

10

11 public Bestellung() {

12 articles = new ArrayList<>();

13 }

14

15 public void addArticle(Artikel article) {
16 //mass nicht weiter aufgelést werden, siehe Hinweise
17 articles.add(article);

18 size+t+t;

19 //muss nicht weiter aufgelöst werden, siehe Hinweise
20 price = article.getPrice()+ price;
21 }

22
23 public int getSize() {
24 return size;
25 }
26
27 public double getPrice() {

28 return price;

29 }

30 | }

\end{document}
