\documentclass{lehramt-informatik-aufgabe}
\liLadePakete{}
\begin{document}
\liAufgabenTitel{Autoverleih}
\section{Aufgabe 3
\index{Relationenmodell}
\footcite{66116:2021:03}}

Entwerfen Sie zum untenstehenden ER-Diagramm ein Relationenschema (in
dritter Normalform, 3. NF) mit möglichst wenigen Relationen.

Verwenden Sie dabei folgende Notation: Primärschlüssel werden durch
Unterstreichen gekennzeichnet, Fremdschlüssel durch die Nennung der
Relation, auf die sie verweisen, in eckigen Klammern hinter dem
Fremdschlüsselattribut. Attribute zusammengesetzter Fremdschlüssel
werden durch runde Klammern als zusammengehörig markiert. Wenn ein
Attribut zur korrekten Abbildung des ER-Diagramms als UNIQUE oder NOT
NULL ausgezeichnet werden muss, geben Sie dies an.

Beispiel:

Relation1 (Primärschlüssel, Attributl, Attribut2,
Fremdschlüsselattributl[Relationl],
(Fremdschlüssel2 Attributl, Fremdschltissel2 Attribut2) [Relation2]);
Attribut1 UNIQUE
Attribut2 NOT NULL

% https://en.wikipedia.org/wiki/Enhanced_entity%E2%80%93relationship_model
% https://www.tutorialride.com/dbms/enhanced-entity-relationship-model-eer-model.htm

\end{document}
