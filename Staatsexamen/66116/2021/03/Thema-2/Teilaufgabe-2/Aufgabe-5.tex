\documentclass{lehramt-informatik-aufgabe}
\liLadePakete{rmodell,relationale-algebra}
\begin{document}
\liAufgabenTitel{Mitarbeiter einer Abteilung}
\section{Aufgabe 5
\index{Relationale Algebra}
\footcite{examen:66116:2021:03}}

Formulieren Sie basierend auf den in der letzten Aufgabe gegebenen
Relationen die geforderten Anfragen in der Relationenalgebra.
\index{RelaX - relational algebra calculator}

\begin{liRmodell}
Mitarbeiter (\liPrimaer{MitarbeiterID}, Vorname, Nachname, Adresse,
Gehalt, \liFremd{Vorgesetzter [Mitarbeiter]} NOT NULL, \liFremd{AbteilungsID[Abteilung]})

\bigskip

Abteilung (\liPrimaer{AbteilungsID}, Bezeichnung UNIQUE NOT NULL)
\end{liRmodell}

\begin{enumerate}

%%
% a)
%%

% https://dbis-uibk.github.io/relax/calc/local/uibk/local/0

% Mitarbeiter = {
% 	Vorname Nachname AbteilungsID
% 	Sergej Puschkin 1
% 	Eduard Hans NULL
% }

% Abteilung = {
% 	AbteilungsID Bezeichnung
% 	1 Buchhaltung
% }

% π Vorname, Nachname (σ Bezeichnung = 'Buchhaltung' (Mitarbeiter ⨝ Abteilung))

\item Formulieren Sie eine Anfrage, welche die Vornamen und Nachnamen
der Mitarbeiter ausgibt, die in der Buchhaltung arbeiten.

\begin{liAntwort}
\begin{multline*}
\pi_{\text{Vorname, Nachname}}(\\
  \sigma_{\text{Bezeichnung = 'Buchhaltung'}}(\\
    \text{Mitarbeiter}
    \bowtie_{\text{Mitarbeiter.AbteilungsID = Abteilung.AbteilungsID}}
    \text{Abteilung}\\
  )\\
)
\end{multline*}
\end{liAntwort}

%%
% b)
%%

\item Formulieren Sie eine Anfrage, welche die Vornamen und Nachnamen
der Mitarbeiter ausgibt, die in keiner Abteilung arbeiten.

% σ AbteilungsID = NULL (Mitarbeiter)

\begin{liAntwort}
\begin{multline*}
\pi_{\text{Vorname, Nachname}}(\text{Mitarbeiter})
-\\
\pi_{\text{Vorname, Nachname}}(\text{Mitarbeiter} \bowtie \text{Abteilung})
\end{multline*}
\end{liAntwort}
\end{enumerate}
\end{document}
