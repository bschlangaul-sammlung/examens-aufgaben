\documentclass{lehramt-informatik-aufgabe}
\liLadePakete{}
\begin{document}
\liAufgabenTitel{Online-Marktplatze}
\section{Aufgabe 2
\index{Entity-Relation-Modell}
\footcite{66116:2021:03}}

Im Folgenden finden Sie die Beschreibung eines Online-Marktplatzes.
Erstellen Sie zu dieser Beschreibung ein erweitertes ER-Diagramm.
Kennzeichnen Sie die Primärschlüssel durch passendes Un- terstreichen
und geben Sie die Kardinalitäten in Chen-Notation (= Funktionalitäten)
an. Kennzeichnen Sie auch die totale Teilnahme von Entity-Typen an
Beziehungstypen.

Es gibt Produkte. Diese haben eine eindeutige Bezeichnung, einen
Beschreibungstext und eine Bewertung. Außerdem gibt es Personen, die
entweder Kunde, Händler oder beides sind. Jede Person hat einen
Nachnamen, einen oder mehrere Vornamen, ein Geburtsdatum und eine
E-Mail-Adresse, mit der diese eindeutig identifiziert werden kann.

Das System verwaltet außerdem Zahlungsmittel. Jedes Zahlungsmittel ist
entweder eine Kreditkarte oder eine Bankverbindung für Lastschriften.
Für das Lastschriftverfahren wird die international eindeutige IBAN und
der Name des Kontoinhabers erfasst, bei Zahlung mit Kreditkarte der Name
des Karteninhabers, die eindeutige Kartennummer, das Ablaufdatum sowie
der Kartenanbieter. Es gibt Transaktionen. Jede Transaktion bezieht sich
stets auf ein Produkt, einen Kunden, einen Händler und auf ein
Zahlungsmittel, das für die Transaktion verwendet wird. Jede Transaktion
enthält außerdem den Preis, auf den sich Kunde und Händler geeinigt
haben, das Abschlussdatum sowie eine Lieferadresse, an die das Produkt
versandt wird.

\end{document}
