\documentclass{lehramt-informatik-aufgabe}
\liLadePakete{er}
\begin{document}
\let\a=\liErMpAttribute
\let\d=\liErDatenbankName
\let\e=\liErMpEntity
\let\r=\liErMpRelationship

\liAufgabenTitel{Online-Marktplatze}
\section{Aufgabe 2
\index{Entity-Relation-Modell}
\footcite{66116:2021:03}}

Im Folgenden finden Sie die Beschreibung eines Online-Marktplatzes.
Erstellen Sie zu dieser Beschreibung ein erweitertes ER-Diagramm.
Kennzeichnen Sie die Primärschlüssel durch passendes Unterstreichen
und geben Sie die Kardinalitäten in Chen-Notation (= Funktionalitäten)
an. Kennzeichnen Sie auch die totale Teilnahme von Entity-Typen an
Beziehungstypen.

Es gibt \e{Produkte}. Diese haben eine eindeutige \a{Bezeichnung}, einen
\a{Beschreibungstext} und eine \a{Bewertung}. Außerdem gibt es
\e{Personen}, die entweder \e{Kunde}, \e{Händler} oder beides sind. Jede
Person hat einen \a{Nachnamen}, einen oder mehrere \a{Vornamen}, ein
\a{Geburtsdatum} und eine \a{E-Mail-Adresse}, mit der diese eindeutig
identifiziert werden kann.

Das System verwaltet außerdem \e{Zahlungsmittel}. Jedes Zahlungsmittel
ist entweder eine \e{Kreditkarte} oder eine \e{Bankverbindung} für
Lastschriften. Für das Lastschriftverfahren wird die international
eindeutige \a{IBAN} und der Name des \a{Kontoinhabers} erfasst, bei
Zahlung mit Kreditkarte der Name des \a{Karteninhabers}, die eindeutige
\a{Kartennummer}, das \a{Ablaufdatum} sowie der \a{Kartenanbieter}. Es
gibt \e{Transaktionen}. Jede Transaktion bezieht sich stets auf ein
Produkt, einen Kunden, einen Händler und auf ein Zahlungsmittel, das für
die Transaktion verwendet wird. Jede Transaktion enthält außerdem den
\a{Preis}, auf den sich Kunde und Händler geeinigt haben, das
\a{Abschlussdatum} sowie eine \a{Lieferadresse}, an die das Produkt
versandt wird.

\begin{liAntwort}
\begin{center}
\begin{tikzpicture}[er2,scale=0.7,transform shape]

% Person
\node[entity] (Person) {Person};
\node[attribute,right=1cm of Person] {E-Mail-Adresse} edge (Person);
\node[attribute,above left=1cm of Person] {Vornamen} edge (Person);
\node[attribute,above=1cm of Person] {Nachnamen} edge (Person);
\node[attribute,above right=1cm of Person] {Geburtsdatum} edge (Person);

% Kunde
\node[entity,below left=3cm of Person] (Kunde) {Kunde};

% Händler
\node[entity,below right=3cm of Person] (Händler) {Händler};

% Transaktion
\node[relationship,below=4cm of Person] (Transaktion) {Transaktion};
\node[attribute,left=1cm of Transaktion] {Preis} edge (Transaktion);
\node[attribute,below=1cm of Transaktion] {Abschlussdatum} edge (Transaktion);
\node[attribute,right=1cm of Transaktion] {Lieferadresse} edge (Transaktion);

% Zahlungsmittel
\node[entity,below=3cm of Händler] (Zahlungsmittel) {Zahlungsmittel};

% Bankverbindung
\node[entity,below left=of Zahlungsmittel] (Bankverbindung) {Bankverbindung};
\node[attribute,below=1cm of Bankverbindung] {IBAN} edge (Bankverbindung);

% Kreditkarte
\node[entity,below right=of Zahlungsmittel] (Kreditkarte) {Kreditkarte};
\node[attribute,below left=1cm of Kreditkarte] {Kartennummer} edge (Kreditkarte);
\node[attribute,below=1cm of Kreditkarte] {Ablaufdatum} edge (Kreditkarte);
\node[attribute,below right=1cm of Kreditkarte] {Kartenanbieter} edge (Kreditkarte);

% Produkt
\node[entity,below=of Kunde] (Produkt) {Produkt};
\node[attribute,left=1cm of Produkt] {Bezeichnung} edge (Produkt);
\node[attribute,below left=1cm of Produkt] {Beschreibungstext} edge (Produkt);
\node[attribute,below=1cm of Produkt] {Bewertung} edge (Produkt);
\end{tikzpicture}
\end{center}
\end{liAntwort}

% Person(Nachname, Vorname, Geburtsdatum, E-Mail-Adresse)

% Produkt(Bezeichnung, Beschreibungstext, Bewertung)

% Transaktion(Käufer, Händler, Produkt, Zahlungsmittel)

% Kreditkarte()

% Bankverbindung()
\end{document}
