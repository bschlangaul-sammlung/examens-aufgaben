\documentclass{lehramt-informatik-aufgabe}
\liLadePakete{normalformen}
\begin{document}
\let\FA=\liFunktionaleAbhaengigkeiten

\liAufgabenTitel{}
\section{Aufgabe 6
\index{Normalformen}
\footcite{66116:2021:03}}

Gegeben ist die Relation Prüfung (Prüfungsnummer, Fakultät,
Prüfungsname, Dozent, Prüfungstyp, ECTS) mit den beiden
Schlüsselkandidaten (Prüfungsnummer, Fakultät) und (Fakultät,
Prüfungsname, Dozent).

\FA{
  Prüfungsnummer, Fakultät -> Prüfungsname, Dozent, Prüfungstyp, ECTS;
  Fakultät, Prüfungsname, Dozent -> Prüfungsnummer, Prüfungstyp, ECTS;
}

Alle Attributwerte sind atomar. Es gelten nur die durch die
Schlüsselkandidaten implizierten funktionalen Abhängigkeit.

Geben Sie die höchste Normalform an, die die Relation Prüfung erfüllt.
Zeigen Sie, dass alle Bedingungen für diese Normalform erfüllt sind
und dass mindestens eine Bedingung der nächsthöheren Normalform verletzt
ist. Beziehen Sie sich bei der Begründung auf die gegebene Relation und
nennen Sie nicht nur die allgemeinen Definitionen der Normalformen.
\end{document}
