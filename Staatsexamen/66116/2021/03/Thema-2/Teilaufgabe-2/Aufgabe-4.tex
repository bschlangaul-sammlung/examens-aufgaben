\documentclass{lehramt-informatik-aufgabe}
\liLadePakete{rmodell,syntax}
\begin{document}
\liAufgabenTitel{Mitarbeiter einer Abteilung}
\section{Aufgabe 4
\index{SQL}
\footcite{66116:2021:03}}

Gegeben sind folgende Relationen:

\begin{liRmodell}
Mitarbeiter (\liPrimaer{MitarbeiterID}, Vorname, Nachname, Adresse,
Gehalt, Vorgesetzter [Mitarbeiter], AbteilungsID[Abteilung])

Vorgesetzter NOT NULL

Abteilung (\liPrimaer{AbteilungsID}, Bezeichnung)

Bezeichnung UNIQUE NOT NULL
\end{liRmodell}

Verwenden Sie im Folgenden nur Standard-SQL und keine
produktspezifischen Erweiterungen. Sie dürfen bei Bedarf Views anlegen.
Geben Sie einen Datensatz nicht mehrfach aus.

% Datenbankname: Abteilung
\begin{minted}{sql}
CREATE TABLE Abteilung(
  AbteilungsID INTEGER PRIMARY KEY,
  Bezeichnung VARCHAR(30) UNIQUE NOT NULL
);

CREATE TABLE Mitarbeiter(
  MitarbeiterID INTEGER PRIMARY KEY,
  Vorname VARCHAR(30),
  Nachname VARCHAR(30),
  Adresse VARCHAR(60),
  Gehalt DECIMAL(7, 2),
  Vorgesetzter INTEGER NOT NULL REFERENCES Mitarbeiter(MitarbeiterID),
  AbteilungsID INTEGER REFERENCES Abteilung(AbteilungsID)
);

INSERT INTO Abteilung VALUES (1, 'Vertrieb');

INSERT INTO Mitarbeiter VALUES
(1, 'Karl', 'Landsbach', 'Sigmaringstraße 4, 87153 Farnbach', 2467.23, 1, 1),
(2, 'Lisa', 'Grätzner', 'Scheidplatz 6, 18434 Tullach', 5382.2, 1, 1);
\end{minted}
\index{SQL mit Übungsdatenbank}

\begin{enumerate}

%%
% a)
%%

\item Schreiben Sie eine SQL-Anweisung, die die Tabelle Mitarbeiter
anlegt. Gehen Sie davon aus, dass die Tabelle Abteilung bereits
existiert.

\begin{liAntwort}
Siehe oben
\end{liAntwort}

%%
% b)
%%

\item Schreiben Sie eine SQL-Anweisung, die Vor- und Nachnamen der
Mitarbeiter der Abteilung mit der Bezeichnung Vertrieb ausgibt,
absteigend sortiert nach MitarbeiterID.

\begin{liAntwort}
\begin{minted}{sql}
SELECT m.Vorname, m.Nachname
FROM Mitarbeiter m, Abteilung a
WHERE
  a.AbteilungsID = m.AbteilungsID AND
  a.Bezeichnung = 'Vertrieb'
ORDER BY m.MitarbeiterID DESC;
\end{minted}
\end{liAntwort}

%%
% c)
%%

\item Schreiben Sie eine SQL-Anweisung, die Vor- und Nachnamen sowie das
Gehalt von Mitarbeitern ausgibt, die mehr verdienen als ihr Chef.
Sortieren Sie die Ausgabe absteigend nach dem Gehalt.

%%
% d)
%%

\item Schreiben Sie eine SQL-Anweisung, die das Gehalt von allen
Mitarbeitern aus der Abteilung mit der ID 42 um 10\% erhöht.

%%
% e)
%%

\item Schreiben Sie eine SQL-Anweisung, welche den Vornamen, die
Nachnamen und das Gehalt der sieben bestbezahlten Mitarbeiter aus der
Buchhaltung ausgibt. Standardkonforme Sprachkonstrukte, die eine
Beschränkung der Ausgabe bewirken, sind erlaubt.

%%
% f)
%%

\item Schreiben Sie eine SQL-Anweisung, die für jede Abteilung die
Mitarbeiter ermittelt, die am wenigsten verdienen. Dabei sollen Vorname,
Nachname und die Abteilungsbezeichnung der Mitarbeiter ausgegeben
werden.
\end{enumerate}

\end{document}
