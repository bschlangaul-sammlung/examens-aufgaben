\documentclass{lehramt-informatik-aufgabe}
\liLadePakete{}
\begin{document}
\liAufgabenTitel{}
\section{Aufgabe 3
\index{B-Baum}
\footcite{66116:2013:03}}

Gegeben sei der folgende B-Baum

1 2 4 5 6 7 8 9 10 11 12 13 14 15 16 17 18 19 20 21

1 2 3 4 5

\begin{enumerate}

%%
% 1.
%%

\item Was bedeutet $k$ bei einem B-Baum mit Grad $k$? Geben Sie $k$ für
den obigen B-Baum an.

\begin{liAntwort}
Jeder Knoten außer der Wurzel hat mindestens $k$ und höchstens $2k$
Einträge. Die Wurzel hat zwischen einem und $2k$ Einträgen. Die Einträge
werden in allen Knoten sortiert gehalten. Alle Knoten mit $n$ Einträgen,
außer den Blättern, haben $n + 1$ Kinder.
\footcite[Seite 225]{kemper}

Für den gegeben Baum kann die Ordnung $k = 2$ angegeben werden.
\end{liAntwort}

%%
% 2.
%%

\item Was sind die Vorteile von B-Bäumen im Vergleich zu binären Baumen?

\begin{liAntwort}
B-Bäume sind immer höhenbalanciert und  die  Lokalität
von Operationen.\liFussnoteUrl{http://wwwbayer.in.tum.de/lehre/WS2001/HSEM-bayer/BTreesAusarbeitung.pdf}
\end{liAntwort}

%%
% 3.
%%

\item Wozu werden B-Bäume in der Regel verwendet und wieso?

\begin{liAntwort}
B-Bäume werden für Hintergrundspeicherung (z. B. von Datenbanksystemen,
Dateisystem) verwendet. Die Knotengrößen werden auf die
Seitenkapazitäten abgestimmt.\footcite[Seite 223]{kemper}
\end{liAntwort}

%%
% 4.
%%

\item Fügen Sie den Wert 3 in den B-Baum ein, und zeichnen Sie den
vollständigen B-Baum nach dem Einfügen und möglichen darauf folgenden
Operationen.

\begin{liAntwort}

\end{liAntwort}

%%
% 5.
%%

\item Entfernen Sie aus dem ursprünglichen B-Baum den Wert 19. Zeichnen
Sie das vollständige Ergebnis nach dem Löschen und möglichen darauf
folgenden Operationen. Sollte es mehrere richtige Lösungen geben, reicht
es eine Lösung zu zeichnen.

\begin{liAntwort}

\end{liAntwort}

\end{enumerate}

\end{document}
