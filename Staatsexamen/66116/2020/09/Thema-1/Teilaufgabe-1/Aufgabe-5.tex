\documentclass{lehramt-informatik-aufgabe}
\liLadePakete{}
\begin{document}
\liAufgabenTitel{Lebenszyklus}
\section{Aufgabe 5
\index{Projektplanung}
\footcite{66116:2020:09}}

\begin{enumerate}

%%
%
%%

\item Nennen Sie fünf kritische Faktoren, die bei der Auswahl eines
Vorgehensmodells helfen können und ordnen Sie plangetriebene und agile
Prozesse entsprechend ein.

%%
%
%%

\item Nennen und beschreiben Sie kurz die Rollen im Scrum.
\index{SCRUM}

\begin{liAntwort}
\begin{description}
\item[Product Owner]

Der Product Owner ist für die Eigenschaften und den wirtschaftlichen
Erfolg des Produkts verantwortlich.

\item[Entwickler]
Die Entwickler sind für die Lieferung der Produktfunktionalitäten in der
vom Product Owner gewünschten Reihenfolge verantwortlich.

\item[Scrum Master]

Der Scrum Master ist dafür verantwortlich, dass Scrum als Rahmenwerk
gelingt. Dazu arbeitet er mit dem Entwicklungsteam zusammen, gehört aber
selbst nicht dazu.
\footcite{wiki:scrum}
\end{description}
\end{liAntwort}

%%
%
%%

\item Nennen und beschreiben Sie drei Scrum Artefakte. Nennen Sie die
verantwortliche Rolle für jedes Artefakt.

\begin{liAntwort}
\begin{description}
\item[Product Backlog]

Das Product Backlog ist eine geordnete Auflistung der Anforderungen an
das Produkt.

\item[Sprint Backlog]

Das Sprint Backlog ist der aktuelle Plan der für einen Sprint zu
erledigenden Aufgaben.

\item[Product Increment]

Das Inkrement ist die Summe aller Product-Backlog-Einträge, die während
des aktuellen und allen vorangegangenen Sprints fertiggestellt wurden.
\footcite{wiki:scrum}
\end{description}
\end{liAntwort}

%%
%
%%

\item Beschreiben Sie kurz, was ein Sprint ist. Wie lange sollte ein
Sprint maximal dauern?

\begin{liAntwort}
Ein Sprint ist ein Arbeitsabschnitt, in dem ein Inkrement einer
Produktfunktionalität implementiert wird. Ein Sprint umfasst ein
Zeitfenster von ein bis vier Wochen.
\end{liAntwort}

\end{enumerate}
\end{document}
