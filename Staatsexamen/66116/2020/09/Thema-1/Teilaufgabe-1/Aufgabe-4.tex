\documentclass{lehramt-informatik-aufgabe}
\liLadePakete{syntax}
\begin{document}
\liAufgabenTitel{White-Box-Testverfahren}
\section{Aufgabe 4
\index{White-Box-Testing}
\footcite{66116:2020:09}}

Diese Aufgabe behandelt Wortpalindrome, also Wörter, die vorwärts und
rückwärts gelesen jeweils dieselbe Zeichenkette bilden, z. B. Otto oder
Rentner. Leere Wortpalindrome (also Wortpalindro- me der Wortlänge 0)
sind dabei nicht zulässig.

Folgende Java-Methode prüft, ob das übergebene Zeichen-Array ein
Wortpalindrom darstellt:

\begin{minted}{java}
public static boolean istWortpalindrom(char[] wort) {
boolean resultat = false;
if (wort != null) {
int laenge = wort.length;
if (laenge >= 2) {
resultat = true;
for (int i = 0; i < laenge/2; ++i) {
char cl = wortji];
char c2 = wort/laenge-1-ij;
if (Character. toLowerCase (cl) != Character. toLowerCase (c2))
resultat = false;
break;

}

} else {
if (laenge == 1) {
resultat = true;
}

}
return resultat;
}
}
\end{minted}
\begin{enumerate}

\item Geben Sie für die Methode einen Kontrollflussgraphen an, wobei Sie
die Knoten mit den jeweiligen Zeilennummern im Quelltext beschriften.

\item Geben Sie eine minimale Testmenge an, die das Kriterium der
Anweisungsüberdeckung erfüllt.

Hinweis: Eine Testmenge ist minimal, wenn es keine Testmenge mit einer
kleineren Zahl von Testfällen gibt. Die Minimalität muss nicht bewiesen
werden.

\item Geben Sie eine minimale Testmenge an, die das Kriterium der
Boundary-Interior-Pfadüberdeckung erfüllt.

Hinweis: Das Kriterium Boundary-Interior-Pfadüberdeckung beschreibt
einen Spezialfall der Pfadüberdeckung, wobei nur Pfade berücksichtigt
werden, bei denen jede Schleife nicht mehr als zweimal durchlaufen wird.

\item Im Falle des Kriteriums Pfadüberdeckung können minimale Testmengen
sehr groß werden, da die Anzahl der Pfade sehr schnell zunimmt. Wie
viele mögliche Pfade ergeben sich maximal für eine Schleife, die drei
einseitig bedingte Anweisungen hintereinander enthält und bis zu zweimal
durchlaufen wird? Geben Sie Ihren Rechenweg an (das Ergebnis alleine
gibt keine Punkte).

\item Könnte für das hier abgebildete Quelltext-Beispiel auch das
Verfahren der unbegrenzten Pfadüberdeckung (also Abdeckung aller
möglicher Pfade ohne Beschränkung) als Test- Kriterium gewählt werden?
Begründen Sie.

\end{enumerate}
\end{document}
