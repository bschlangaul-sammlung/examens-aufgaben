\documentclass{lehramt-informatik-aufgabe}
\liLadePakete{syntax,kontrollflussgraph}
\begin{document}
\let\c=\liKontrollCode

\liAufgabenTitel{White-Box-Testverfahren}
\section{Aufgabe 4
\index{White-Box-Testing}
\footcite{66116:2020:09}}

Diese Aufgabe behandelt \emph{Wortpalindrome}, also Wörter, die vorwärts
und rückwärts gelesen jeweils dieselbe Zeichenkette bilden, z. B.
\texttt{Otto} oder \texttt{Rentner}. Leere Wortpalindrome (also
Wortpalindrome der Wortlänge 0) sind dabei nicht zulässig.

Folgende \emph{Java-Methode} prüft, ob das übergebene Zeichen-Array ein
Wortpalindrom darstellt:

\liJavaExamen[firstline=6,lastline=28]{66116}{2020}{09}{Palindrom}

\begin{enumerate}

\item Geben Sie für die Methode einen \emph{Kontrollflussgraphen} an,
wobei Sie die Knoten mit den jeweiligen Zeilennummern im Quelltext
beschriften.
\index{Kontrollflussgraph}

\begin{liAntwort}
\begin{liKontrollflussgraph}[xscale=0.7,yscale=-1.2]
\node[knoten] at (0,0) (S) {S};

\node[knoten,pin=\c{boolean resultat = false;}] at (0,1) (2) {2};
\node[knoten,pin=\c{if (wort != null)}] at (0,2) (3) {3};
\node[knoten,pin=\c{int laenge = wort.length;}] at (1,3) (4) {4};
\node[knoten,pin=\c{if (laenge >= 2)}] at (1,4) (5) {5};
\node[knoten,pin=\c{resultat = true; int i = 0;}] at (2,5) (6) {6};
\node[knoten,pin=\c{for (i < laenge / 2;)}] at (2,6) (7) {7};
\node[knoten,pin=\c{char c1; char c2 …}] at (3,7) (8) {8};
\node[knoten,pin=\c{if (Char…) }] at (2,8) (10) {10};
\node[knoten,pin=\c{resultat = false; break;}] at (2,9) (12) {12};
\node[knoten,pin=\c{if (laenge == 1)}] at (1,10) (17) {17};
\node[knoten,pin=\c{resultat = true;}] at (1,11) (18) {18};
\node[knoten,pin=180:\c{return resultat;}] at (-1,9) (22) {22};
\node[knoten] at (-1,10) (E) {E};

\path (S) -- (2);
\path (2) -- (3);
\path[wahr] (3) -- (4) \liBedingung{right}{wort != null};
\path[falsch] (3) -- (22)  \liBedingung{left}{wort == null};
\path (4) -- (5);
\path[wahr] (5) -- (6) \liBedingung{right}{laenge >= 2};
\path[falsch] (5) -- (17) \liBedingung{left,rotate=70}{laenge < 2};
\path (6) -- (7);
\path (7) -- (8);
\path (8) -- (10);
\path[wahr] (10) -- (12) \liBedingung{right}{c != c};
\path[falsch] (10) -- (7) \liBedingung{left,rotate=70,pos=0.8}{c == c};
\path (12) -- (22);
\path (17) -- (18);
\path (18) -- (22);
\path (22) -- (E);
\end{liKontrollflussgraph}
\end{liAntwort}

\item Geben Sie eine \emph{minimale Testmenge} an, die das Kriterium der
Anweisungsüberdeckung erfüllt.

Hinweis: Eine \emph{Testmenge} ist \emph{minimal}, wenn es keine
Testmenge mit einer kleineren Zahl von Testfällen gibt. Die Minimalität
muss \emph{nicht} bewiesen werden.

\item Geben Sie eine \emph{minimale Testmenge} an, die das Kriterium der
\emph{Boundary-Interior-Pfadüberdeckung} erfüllt.

Hinweis: Das Kriterium \emph{Boundary-Interior-Pfadüberdeckung}
beschreibt einen Spezialfall der Pfadüberdeckung, wobei nur Pfade
berücksichtigt werden, bei denen jede Schleife nicht mehr als zweimal
durchlaufen wird.

\item Im Falle des Kriteriums Pfadüberdeckung können minimale Testmengen
sehr groß werden, da die Anzahl der Pfade sehr schnell zunimmt. Wie
viele \emph{mögliche Pfade} ergeben sich maximal für eine Schleife, die
drei einseitig bedingte Anweisungen hintereinander enthält und bis zu
zweimal durchlaufen wird? Geben Sie Ihren Rechenweg an (das Ergebnis
alleine gibt keine Punkte).

\item Könnte für das hier abgebildete Quelltext-Beispiel auch das
Verfahren der \emph{unbegrenzten Pfadüberdeckung} (also Abdeckung aller
möglicher Pfade ohne Beschränkung) als Test-Kriterium gewählt werden?
Begründen Sie.

\end{enumerate}
\end{document}
