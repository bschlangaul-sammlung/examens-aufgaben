\documentclass{lehramt-informatik-aufgabe}
\liLadePakete{}
\begin{document}
\liAufgabenTitel{Verifikation}
\section{Aufgabe 1
\index{Verifikation}
\footcite{66116:2020:09}}

\begin{enumerate}

%%
% a)
%%

\item Definieren Sie die Begriffe \emph{„partielle Korrektheit”} und
\emph{„totale Korrektheit”} und grenzen Sie sie voneinander ab.

\begin{liAntwort}
\begin{description}
\item[partielle Korrektheit]

Ein Programmcode wird bezüglich einer Vorbedingung $P$ und einer
Nachbedingung $Q$ partiell korrekt genannt, wenn bei einer Eingabe, die
die Vorbedingung $P$ erfüllt, jedes Ergebnis die Nachbedingung $Q$
erfüllt. Dabei ist es noch möglich, dass das Programm nicht für jede
Eingabe ein Ergebnis liefert, also nicht für jede Eingabe terminiert.

\item[totale Korrektheit]

Ein Code wird total korrekt genannt, wenn er partiell korrekt ist und
zusätzlich für jede Eingabe, die die Vorbedingung $P$ erfüllt,
terminiert. Aus der Definition folgt sofort, dass total korrekte
Programme auch immer partiell korrekt sind.
\footcite{wiki:korrektheit}
\end{description}
\end{liAntwort}

%%
% b)
%%

\item Geben Sie die Verifikationsregel für die abweisende Schleife
while(B) A an.

%%
% c)
%%

\item Erläutern Sie kurz und prägnant die Schritte zur Verifikation
einer abweisenden Schleife mit Vorbedingung P und Nachbedingung Q.

%%
% d)
%%

\item Wie kann man die Terminierung einer Schleife beweisen?

%%
% e)
%%

\item Geben Sie für das folgende Suchprogramm die nummerierten
Zusicherungen an. Lassen Sie dabei jeweils die invariante Vorbedingung P
des Suchprogramms weg. Schreiben Sie nicht auf dem Aufgabenblatt!

% HWP=zn>0OAm...mıEeltAme,z
% i=-J;
% // (1)
% 3-0;
% // (2)
% while (i == -1&& j < n) // (3)
% {1
% if (a]lj] == m){
% // (5)
% i=j;
% // (6)
% }
% else {
% // (7)
% ;=j+l
% // (8)
% }
% // (9)
% }

% WIQ=Pr(ii=-11VO<k<n:% #Zm)V(i>0Aa;=m)

\end{enumerate}
\end{document}
