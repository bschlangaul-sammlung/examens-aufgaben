\documentclass{lehramt-informatik-aufgabe}
\liLadePakete{gantt,cpm}
\begin{document}
\liAufgabenTitel{Projektplanung}
\section{Aufgabe 2
\index{Projektplanung}
\footcite{66116:2020:09}}

Die Planung eines Softwareprojekts kann z. B. in Form von
Gantt-Diagrammen oder CPM-Netzwerken (kritischer Pfad Methode)
festgehalten werden.

Folgendes Gantt-Diagramm zeigt einen Teil der Projektplanung in einem
klassischen Softwareentwicklungsprozess:

\begin{center}
\begin{ganttchart}[x unit=0.7cm, y unit chart=0.6cm]{0}{6}
\ganttbar[name=D]{Design}{0}{1} \\
\ganttbar[name=R]{Realization}{2}{4} \\
\ganttbar[name=T]{Testing}{3}{5}
\gantttitlelist[name=Zeit]{0,...,6}{1}\\
\ganttlink{D}{R}
\ganttlink[link type=s-s]{R}{T}
\end{ganttchart}
\end{center}
\begin{enumerate}

%%
% (a)
%%

\item Im Diagramm werden 3 Phasen aus dem klassischen
Softwareentwicklungsprozess genannt. Welche Phase sollte dem Design
(Entwurf) immer vorangehen?

\begin{liAntwort}
Die Anforderungsanalyse
\end{liAntwort}

%%
% (b)
%%

\item Wandeln Sie das Gantt-Diagramm in ein CPM-Netzwerk um. Fügen Sie
dazu einen zusätzlichen Start- und Endknoten hinzu. Das Ende des
Projekts ist durch das Ende aller Aktivitäten bedingt.
\index{CPM-Netzplantechnik}

\begin{liAntwort}
\begin{description}
\item[$D_A$] Design Anfang
\item[$R_A$] Realization Anfang
\item[$T_A$] Testing Anfang
\item[$D_E$] Design Ende
\item[$R_E$] Realization Ende
\item[$T_E$] Testing Ende

\end{description}

\begin{center}
\begin{tikzpicture}[x=1.5cm,y=1.5cm]
\liCpmEreignis{A}(1,1)
\liCpmEreignis{DA}(1,2)
\liCpmEreignis{DE}(3,2)
\liCpmEreignis{RA}(2,1)
\liCpmEreignis{RE}(4,1)
\liCpmEreignis{TA}(3,0)
\liCpmEreignis{TE}(5,0)
\liCpmEreignis{E}(5,1)

\liCpmVorgang[schein]{A}{DA}{}
\liCpmVorgang{DA}{DE}{2}
\liCpmVorgang{RA}{RE}{3}
\liCpmVorgang{TA}{TE}{3}
\liCpmVorgang{RA}{TA}{1}
\liCpmVorgang[schein]{DE}{RA}{}
\liCpmVorgang[schein]{TE}{E}{}
\end{tikzpicture}
\end{center}
\end{liAntwort}
%%
% (c)
%%

\item Welche im obigen Gantt-Diagramm nicht enthaltenen Beziehungsarten
zwischen Aktivitäten können in einem Gantt-Diagramm noch auftreten?
Nennen Sie auch deren Bedeutung.
\index{Gantt-Diagramm}

\begin{liAntwort}
Diese Beziehungsarten sind im obigen Gantt-Diagramm vorhanden:

\begin{description}
\item[Normalfolge EA:]
\emph{end-to-start relationship}
%
Anordnungsbeziehung vom Ende eines Vorgangs zum Anfang seines
Nachfolgers.

\item[Anfangsfolge AA:]
\emph{start-to-start relationship}
%
Anordnungsbeziehung vom Anfang eines Vorgangs zum Anfang seines
Nachfolgers.
\end{description}

Diese Beziehungsarten sind im obigen Gantt-Diagramm \emph{nicht}
vorhanden:
\begin{description}

\item[Endefolge EE:]
\emph{finish-to-finish relationship}
%
Anordnungsbeziehung vom Ende eines Vorgangs zum Ende seines Nachfolgers.

\item[Sprungfolge AE:]
\emph{start-to-finish relationship }
%
Anordnungsbeziehung vom Anfang eines Vorgangs zum Ende seines
Nachfolgers
\end{description}
\end{liAntwort}

Gegeben sei nun das folgende CPM-Netzwerk:

%%
% (d)
%%

\item Geben Sie für jedes Ereignis die früheste Zeit an.

%%
% (e)
%%

\item Geben Sie für jedes Ereignis die späteste Zeit an.

%%
% (f)
%%

\item Geben Sie einen kritischen Pfad durch das Netz an! Wie wirkt sich
eine Verzögerung von 5 Zeiteinheiten auf dem kritischen Pfad auf das
Projektende aus?

\end{enumerate}
\end{document}
