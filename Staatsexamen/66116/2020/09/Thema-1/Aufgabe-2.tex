\documentclass{lehramt-informatik-aufgabe}
\liLadePakete{gantt}
\begin{document}
\liAufgabenTitel{Projektplanung}
\section{Aufgabe 2
\index{Projektplanung}
\footcite{66116:2020:09}}

Die Planung eines Softwareprojekts kann z. B. in Form von
Gantt-Diagrammen oder CPM-Netzwerken (kritischer Pfad Methode)
festgehalten werden.

Folgendes Gantt-Diagramm zeigt einen Teil der Projektplanung in einem
klassischen Softwareentwicklungsprozess:

\begin{center}
\begin{ganttchart}[x unit=0.7cm, y unit chart=0.6cm]{0}{6}
\ganttbar[name=D]{Design}{0}{2} \\
\ganttbar[name=R]{Realization}{2}{5} \\
\ganttbar[name=T]{Testing}{3}{6}
\gantttitlelist[name=Zeit]{0,...,6}{1}\\
\ganttlink{D}{R}
\ganttlink[link type=s-s]{R}{T}
\end{ganttchart}
\end{center}
\begin{enumerate}

%%
% (a)
%%

\item Im Diagramm werden 3 Phasen aus dem klassischen
Softwareentwicklungsprozess genannt. Welche Phase sollte dem Design
(Entwurf) immer vorangehen?

\begin{liAntwort}
Die Anforderungsanalyse
\end{liAntwort}

%%
% (b)
%%

\item Wandeln Sie das Gantt-Diagramm in ein CPM-Netzwerk um. Fügen Sie
dazu einen zusätzlichen Start- und Endknoten hinzu. Das Ende des
Projekts ist durch das Ende aller Aktivitäten bedingt.
\index{CPM-Netzplantechnik}
%%
% (c)
%%

\item Welche im obigen Gantt-Diagramm nicht enthaltenen Beziehungsarten
zwischen Aktivitäten können in einem Gantt-Diagramm noch auftreten?
Nennen Sie auch deren Bedeutung.
\index{Gantt-Diagramm}

\begin{liAntwort}
Diese Beziehungsarten sind im obigen Gantt-Diagramm vorhanden:

\begin{description}
\item[Normalfolge EA:]
\emph{end-to-start relationship}
%
Anordnungsbeziehung vom Ende eines Vorgangs zum Anfang seines
Nachfolgers.

\item[Anfangsfolge AA:]
\emph{start-to-start relationship}
%
Anordnungsbeziehung vom Anfang eines Vorgangs zum Anfang seines
Nachfolgers.
\end{description}

Diese Beziehungsarten sind im obigen Gantt-Diagramm \emph{nicht}
vorhanden:
\begin{description}

\item[Endefolge EE:]
\emph{finish-to-finish relationship}
%
Anordnungsbeziehung vom Ende eines Vorgangs zum Ende seines Nachfolgers.

\item[Sprungfolge AE:]
\emph{start-to-finish relationship }
%
Anordnungsbeziehung vom Anfang eines Vorgangs zum Ende seines
Nachfolgers
\end{description}
\end{liAntwort}

Gegeben sei nun das folgende CPM-Netzwerk:

%%
% (d)
%%

\item Geben Sie für jedes Ereignis die früheste Zeit an.

%%
% (e)
%%

\item Geben Sie für jedes Ereignis die späteste Zeit an.

%%
% (f)
%%

\item Geben Sie einen kritischen Pfad durch das Netz an! Wie wirkt sich
eine Verzögerung von 5 Zeiteinheiten auf dem kritischen Pfad auf das
Projektende aus?

\end{enumerate}
\end{document}
