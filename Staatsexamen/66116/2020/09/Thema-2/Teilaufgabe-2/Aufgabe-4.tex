\documentclass{lehramt-informatik-aufgabe}
\liLadePakete{normalformen}
\begin{document}
\let\FA=\liFunktionaleAbhaengigkeiten

\liAufgabenTitel{(Entwurfstheorie)}
\section{Aufgabe 4
\index{Normalformen}
\footcite{66116:2020:09}}

Gegeben ist das folgende Relationenschema R in erster Normalform.

R:{[A,B, C, D, E, F]}

Für R gelte folgende Menge FD funktionaler Abhängigkeiten:

\FA{
  AC -> DE;
  ACE -> B;
  E -> B;
  D -> F;
  AC -> F;
  AD -> F;
}

\begin{enumerate}

%%
% a)
%%

\item R mit FD hat genau einen Kandidatenschlüssel X. Bestimmen Sie
diesen und begründen Sie Ihre Antwort.

\begin{liAntwort}
AC ist der Kandidatenschlüssel. AC kommt in keiner rechten Seite der
Funktionalen Abhängigkeiten vor.
\end{liAntwort}

%%
% b)
%%

\item Berechnen Sie Schritt für Schritt die Hülle $X^+$ von $X := \{ K
\}$.

\begin{liAntwort}
\begin{enumerate}
\item $AC \cup DE$
\item $ACDE \cup B$ (ACE -> B)
\item $ACDEB$ (E -> B)
\item $ACDEB \cup F$ (D -> F)
\item $ACDEBF$ (AC -> F)
\item $ACDEBF$ (AD -> F)
\end{enumerate}
\end{liAntwort}

%%
% c)
%%

\item Nennen Sie alle primen und nicht-primen Attribute.

\begin{liAntwort}
prim: AC

nicht-prim: BDEF
\end{liAntwort}

%%
% d)
%%

\item Geben Sie die höchste Normalform an, in der sich die Relation
befindet. Begründen Sie.

\begin{liAntwort}
2NF

 D --> F hängt transitiv von AC ab: AC -> D, D-> F
\end{liAntwort}

%%
% e)
%%

\item Gegeben ist die folgende Zerlegung von R:

R1 (A,C,D,E)
R2 (B,E)
R3 (D, F)

Weisen Sie nach, dass es sich um eine verlustfreie Zerlegung handelt.

\end{enumerate}
\end{document}
