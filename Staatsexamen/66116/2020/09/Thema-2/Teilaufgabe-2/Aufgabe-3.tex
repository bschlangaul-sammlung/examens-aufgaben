\documentclass{lehramt-informatik-aufgabe}
\liLadePakete{}
\begin{document}
\liAufgabenTitel{Relationale Algebra und Optimierung}
\section{
\index{Relationale Algebra}
\footcite{66116:2020:09}}
\begin{enumerate}

%%
% a)
%%

\item Betrachten Sie die Relation V. Sie enthält eine Spalte Name sowie
ein dazugehörendes Jahr.

\begin{center}
\begin{tabular}{|l|l|}
\hline
Name & Jahr \\\hline\hline
A & 2019 \\\hline
A & 2020 \\\hline
B & 2018 \\\hline
B & 2019 \\\hline
B & 2020 \\\hline
C & 2017 \\\hline
C & 2018 \\\hline
C & 2020 \\\hline
\end{tabular}
\end{center}

\begin{enumerate}

%%
% i.
%%

\item Gesucht ist eine Relation $S$, die das folgende Ergebnis von $V
\div S$ berechnet ($\div$ ist die Division der relationalen Algebra):

$V \div S$

\begin{center}

\begin{tabular}{|l|}
\hline
Name \\\hline\hline
B \\\hline
\end{tabular}
\end{center}

Welche der nachstehenden Ausprägungen für die Relation liefert das
gewünschte Ergebnis? Geben Sie eine Begründung an.

\begin{enumerate}
\item

\begin{tabular}{|l|}
\hline
Jahr\\\hline\hline
2017\\\hline
2018\\\hline
2019\\\hline
2020\\\hline
\end{tabular}

\item

\begin{tabular}{|l|}
\hline
Jahr\\\hline\hline
2018\\\hline
2019\\\hline
2020\\\hline
\end{tabular}

\item

\begin{tabular}{|l|}
\hline
Jahr\\\hline\hline
2017\\\hline
2019\\\hline
2020\\\hline
\end{tabular}

\item weder i., noch ii., noch iii.

\end{enumerate}

%%
%
%%

\begin{liAntwort}
iv) also weder i., noch ii., noch iii.

\begin{enumerate}
\item

\begin{tabular}{|l|}
\hline
Name \\\hline\hline
\end{tabular}

\item

\begin{tabular}{|l|}
\hline
Name \\\hline\hline
C \\\hline
\end{tabular}

\item

\begin{tabular}{|l|}
\hline
Name \\\hline\hline
B \\\hline
C \\\hline
\end{tabular}
\end{enumerate}
\end{liAntwort}

%%
% ji.
%%

\item Formulieren Sie die Divisions-Query aus Teilaufgabe i. in SQL.

\end{enumerate}
%%
% b)
%%

\item Gegeben sind die Tabellen R(A, B) und S(C, D) sowie die folgende View:

ı CREATE VIEW mv (A,C,D) AS

, SELECT DISTINCTA,C,D

»  FROMR,S

« WHEREB=DANDA <> 10;

Auf dieser View wird die folgende Query ausgeführt:

, SELECT DISTINCT A
, FROM mv
;» WHEREC>D:

Konvertieren Sie die Query und die zugrundeliegenden View in einen Ausdruck der re-
lationalen Algebra in Form eines Operatorbaums. Führen Sie anschließend eine relationale
Optimierung durch. Beschreiben und begründen Sie dabei kurz jeden durchgeführten Schritt.

%%
% c)
%%

\item Gegeben sind die Relationen R, S und U sowie deren Kardinalitäten Tr, Ts und Tr:

R (al, a2, a3) Tr = 200
S (al, a2, a3) Ts = 100
U (ul, u2) Iv = 50

Bei der Ausführung des folgenden Query-Plans wurden die Kardinalitäten der Zwischener-
gebnisse mitgezählt und an den Kanten notiert.

Leiten Sie aus den Angaben im Ausführungsplan den Anteil der qualifizierten Tupel aller
Prädikate her und geben Sie diese an.

Tx
s0|
N Ral > Vu

N R.a3 = S.a3 U
 N
OR.al > 100 OS.al < 10

R 5
\end{enumerate}

\end{document}
