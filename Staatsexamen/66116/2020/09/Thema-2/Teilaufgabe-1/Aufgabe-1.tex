\documentclass{lehramt-informatik-aufgabe}
\liLadePakete{}
\begin{document}
\liAufgabenTitel{}
\section{Aufgabe 1
\index{Prozessmodelle}
\footcite{66116:2020:09}}

\begin{enumerate}
%%
% a)
%%

\item Nennen Sie fünf Phasen, die im Wasserfallmodell durchlaufen werden
sowie deren jeweiliges Ziel bzw. Ergebnis(-dokument).

%%
% b)
%%

\item Nennen Sie drei Arten der Softwarewartung und geben Sie jeweils
eine kurze Beschreibung an.

%%
% c)
%%

\item Eine grundlegende Komponente des Extreme Programming ist
„Continuous Integration“. Erklären Sie diesen Begriff und warum man
davon einen Vorteil erwartet.

%%
% d)
%%

\item Nennen Sie zwei Softwaremetriken und geben Sie jeweils einen Vor-
und Nachteil an, der im Projektmanagement von Bedeutung ist.

%%
% e)
%%

\item Nennen und beschreiben Sie kurz drei wichtige Aktivitäten, welche
innerhalb einer Sprint- Iteration von Scrum durchlaufen werden.

%%
% f)
%%

\item Erläutern Sie den Unterschied zwischen dem Product-Backlog und dem
Sprint-Backlog.

%%
% g)
%%

\item Erläutern Sie, warum eine agile Entwicklungsmethode nur für
kleinere Teams (max. 10 Personen) gut geeignet ist, und zeigen Sie einen
Lösungsansatz, wie auch größere Firmen agile Methoden sinnvoll einsetzen
können.

\end{enumerate}
\end{document}
