\documentclass{lehramt-informatik-aufgabe}
\liLadePakete{}
\begin{document}
\liAufgabenTitel{}
\section{
\index{Anwendungsfalldiagramm}
\footcite{66116:2020:09}}

Aufgabe 3 (Use-Case-Diagramm) [21 PUNKTE]

Im Folgenden ist eine Systembeschreibung für einen „Fahrkartenautomat“ angegeben.

Es gibt zwei Arten von Bahnfahrern: Gelegenheitsfahrer und Vielfahrer. Bei der
Benutzung des Kartenautomaten kann sich grundsätzlich jeder die Hilfe anzeigen
lassen und ein Beschwerdeformular ausfüllen.

Gelegenheitsfahrer haben keine Benutzer-Id und können Einzelfahrkarten
auswählen und anschließend kaufen. Damit ein Kaufvorgang erfolgreich abgeschlos-
sen wird, muss ein entsprechender Geldbetrag eingezahlt werden. Dies geschieht
entweder mit Bargeld oder per EC-Karte. Nach dem Kauf eines Tickets kann man
sich dafür optional eine separate Quittung drucken lassen.

Vielfahrer haben eine eindeutige Benutzer-Id mit Passwort, um sich am Automaten
zu authentifizieren. Ein Vielfahrer kann sowohl eine Einzelfahrkarte als auch eine
personalisierte Monatskarte erwerben. Sofern er eine Monatskarte besitzt (Informa-
tion im System hinterlegt), kann er sich kostenfrei eine Ersatzfahrkarte ausstellen
lassen, falls er seine Monatskarte verloren hat. Wenn die Authentifizierung oder ein
Kaufvorgang fehlschlägt, soll eine entsprechende Fehlermeldung erscheinen.

a) Geben Sie die im Text erwähnten Akteure für das beschriebene System an.

b) Identifizieren Sie zwei weitere Stakeholder und nennen Sie dazu je zwei unterschiedliche
Anwendungsfälle des Systems, in die diese involviert sind.

c) Geben Sie mindestens sechs verschiedene Anwendungsfälle für das beschriebene System an.

d) Erstellen Sie aus Ihren vorherigen Antworten ein Use-Case-Diagramm für das beschrie-
bene System, in dem die Akteure und Anwendungsfälle inkl. möglicher Generalisierungen
und Beziehungen eingetragen sind. Achten Sie insbesondere auf mögliche ((include))- und
((extends))- Beziehungen und Bedingungen für Anwendungsfälle.

\end{document}
