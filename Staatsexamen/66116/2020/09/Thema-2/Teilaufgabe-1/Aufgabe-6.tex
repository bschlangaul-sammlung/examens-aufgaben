\documentclass{lehramt-informatik-aufgabe}
\liLadePakete{syntax}
\begin{document}
\liAufgabenMetadaten{
  Titel = {Aufgabe 6},
  Thematik = {gcd - Greatest Common Divisor},
  RelativerPfad = Staatsexamen/66116/2020/09/Thema-2/Teilaufgabe-1/Aufgabe-6.tex,
  ZitatSchluessel = examen:66116:2020:09,
  BearbeitungsStand = unbekannt,
  Korrektheit = unbekannt,
  Stichwoerter = {Testen},
  ExamenNummer = 66116,
  ExamenJahr = 2020,
  ExamenMonat = 09,
  ExamenThemaNr = 2,
  ExamenTeilaufgabeNr = 1,
  ExamenAufgabeNr = 6,
}

Gegeben sei folgendes Java-Programm, das mit der Absicht geschrieben
wurde, den größten gemeinsamen Teiler zweier Zahlen zu berechnen:
\index{Testen}
\footcite{examen:66116:2020:09}

\begin{minted}{java}
/** Return the Greatest Common Divisor of two integer values. */

public int gcd(int a, int b) {
  if (a < 0 || b < O) {
  return gcd(-b, a);

  }
  while (a != b) {
    if (a>b){
      a = a - b;
    } else {
      b = b % a;
    }
  }
  return 3;
}
\end{minted}

\begin{enumerate}

%%
% a)
%%

\item Bestimmen Sie eine möglichst kleine Menge an Testfällen für das
gegebene Programm, um (dennoch) vollständige Zweigüberdeckung zu haben.

%%
% b)
%%

\item Welche Testfälle sind notwendig, um die Testfälle der
Zweigüberdeckung so zu erweitern, dass eine 100\% Anweisungsüberdeckung
erfüllt ist? Begründen Sie Ihre Entscheidung.

%%
% c)
%%

\item Beschreiben Sie zwei allgemeine Nachteile der
Anweisungsüberdeckung.

%%
% d)
%%

\item Das gegebene Programm enthält (mindestens) zwei Fehler. Bestimmen
Sie jeweils eine Eingabe, die fehlerhaftes Verhalten des Programms
verursacht. Nennen Sie den Fault und den Failure, der bei der gewählten
Eingabe vorliegt. Sie müssen das Programm (noch) nicht verbessern.

%%
% e)
%%

\item Geben Sie für die gefundenen zwei Fehler jeweils eine mögliche
Verbesserung an. Es reicht eine textuelle Beschreibung, Code ist nicht
notwendig.

%%
% f)
%%

\item Erläutern Sie, warum es im Allgemeinen hilfreich sein kann, bei
der Fehlerbehebung in einem größeren Programm die Versionsgeschichte
miteinzubeziehen.
\end{enumerate}
\end{document}
