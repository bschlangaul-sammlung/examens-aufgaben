\documentclass{lehramt-informatik-aufgabe}
\liLadePakete{}
\begin{document}
\liAufgabenTitel{Terme über die Rechenarten}
\section{Aufgabe 5
\index{Entwurfsmuster}
\footcite{66116:2020:09}}

Wir betrachten Terme über die Rechenarten op € {+, —, x, /}, die rekursiv definiert sind:

\begin{itemize}
\item Jedes Literal ist ein Term, z.B. „4“.
\item Jedes Symbol ist ein Term, z.B. „x“.
\item Ist t ein Term, so ist „(t)“ ein (geklammerter) Term.
\item Sind tı, t2 Terme, so ist „tı Op ta“ ebenso ein Term.

\end{itemize}
Beispiele für gültige Terme sind also „A +8“, „Ax x“ oder „A+ (8* x)“.
\begin{enumerate}

%%
% a)
%%

\item Welches Design-Pattern eignet sich hier am besten zur Modellierung
dieses Sachverhalts?
%%
% b)
%%

\item Nennen Sie drei wesentliche Vorteile von Design-Pattern im
Allgemeinen.

%%
% c)
%%

\item Modellieren Sie eine Klassenstruktur in UML, die diese rekursive
Struktur von Termen abbildet. Sehen Sie mindestens einzelne Klassen
für die Addition und Multiplikation vor, sowie weitere Klassen für
geklammerte Terme und Literale, welche ganze Zahlen repräsentieren.
Gehen Sie bei der Modellierung der Klassenstruktur davon aus, dass eine
objektorientierte Programmiersprache wie Java zu benutzen ist.

%%
% d)
%%

\item Erstellen Sie ein Objektdiagramm, welches den Term t := 4 + (3*2)
+ (12*y/(8*)) entsprechend Ihres Klassendiagramms repräsentiert.

%%
% e)
%%

\item Überprüfen Sie, ob das Objektdiagramm für den in Teilaufgabe d)
gegebenen Term eindeutig definiert ist. Begründen Sie Ihre Entscheidung.

%%
% f)
%%

\item Die gegebene Klassenstruktur soll mindestens folgende Operationen
unterstützen:

\begin{itemize}
\item das Auswerten von Termen,
\item das Ausgeben in einer leserlichen Form,
\item das Auflisten aller verwendeten Symbole.
Welches Design-Pattern ist hierfür am besten geeignet?

\end{itemize}
%%
% g)
%%

\item Erweitern Sie Ihre Klassenstruktur um die entsprechenden Methoden,
Klassen und Assoziationen, um die in Teilaufgabe f) genannten
zusätzlichen Operationen gemäß dem von Ihnen genannten Design Pattern zu
unterstützen.

\end{enumerate}
\end{document}
