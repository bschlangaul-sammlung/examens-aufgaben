\documentclass{lehramt-informatik-aufgabe}
\liLadePakete{}
\begin{document}
\liAufgabenTitel{}
\section{Aufgabe 4
\index{Transaktionen}
\footcite{66116:2020:03}}

\begin{enumerate}
\item Betrachten Sie den folgenden Schedule S:

\begin{center}
\begin{tabular}{l|l|l}
$T_1$ & $T_2$ & $T_3$ \\\hline
         & $r_2(z)$ & \\
         &          & $w_3(y)$ \\
         & $r_2(x)$ & \\
$w_1(x)$ &          &  \\
         & $w_2(x)$ & \\
         &          & $r_3(z)$ \\
         &          & $c_3$  \\
         & $w_2(z)$ & \\
$w_1(y)$ &          & \\
$c_1$    &          & \\
         & $c_2$    & \\
\end{tabular}
\end{center}

Geben Sie den Ausgabeschedule (einschließlich der Operationen zur
Sperranforderung und -freigabe) im rigorosen Zweiphasen-Sperrprotokoll
für den obigen Eingabeschedule S an.

\begin{liAntwort}

\end{liAntwort}

\item Beschreiben Sie den Unterschied zwischen dem herkömmlichen
Zweiphasen-Sperrprotokoll (2PL) und dem rigorosen
Zweiphasen-Sperrprotokoll. Warum wird in der Praxis häufiger das
rigorose Zweiphasen-Sperrprotokoll verwendet?

\begin{liAntwort}

\end{liAntwort}

\end{enumerate}
\end{document}
