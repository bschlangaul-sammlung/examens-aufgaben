\documentclass{lehramt-informatik-aufgabe}
\liLadePakete{syntax}
\begin{document}
\liAufgabenTitel{Mode-Kollektionen}
\section{Aufgabe 2
\index{SQL}
\footcite{66116:2020:03}}

Gegeben sei der folgende Ausschnitt eines Schemas für die Verwaltung von
Kollektionen:

Promi : {| Kleidungsstueck : {|
Name : VARCHAR(255), ID : INTEGER,
Alter : INTEGER, Hauptbestandteil : VARCHAR(255),
Wohnort : VARCHAR.(255) gehoert\_zu : VARCHAR.(255)

I} 1}

Kollektion : {|
Name : VARCHAR(255),
Jahr : INTEGER,

Saison : VARCHAR(255) hat\_getragen : {[
]} PromiName : VARCHAR.(355),
KleidungsstuecklD : INTEGER,
promotet: {| Datum : DATE
PromiName : VARCHAR(255), 1}

KollektionName : VARCHAR(255)

|}

Die Tabelle Promi beschreibt Promis über ihren eindeutigen Namen, ihr
Alter und ihren Wohnort. Kollektion enthält Informationen über
Kollektionen, nämlich deren eindeutigen Namen, das Jahr und die Saison.
Die Tabelle promotet verwaltet über Referenzen, welcher Promi welche
Kollektion promotet. Kleidungsstück speichert die IDs von
Kleidungsstücken zusammen mit dem Hauptbe- standteil und einer Referenz
auf die zugehörige Kollektion. Die Tabelle hat\_getragen verwaltet über
Referenzen, welcher Promi welches Kleidungsstück an welchem Datum
getragen hat.

Beachten Sie bei der Formulierung der SQL-Anweisungen, dass die
Ergebnisrelationen keine Du- plikate enthalten dürfen. Sie dürfen
geeignete Views definieren.
\begin{enumerate}

%%
% 1.
%%

\item Schreiben Sie SQL-Anweisungen, welche die Tabelle hat\_getragen
inklusive aller benötig- ten Fremdschlüsselconstraints anlegt. Erläutern
Sie kurz, warum die Spalte Datum Teil des Primärschlüssels ist.

%%
% 2.
%%

\item Schreiben Sie eine SQL-Anweisung, welche die Namen der Promis
ausgibt, die eine Sommer- Kollektion promoten (Saison ist ’”Sommer’).

%%
% 3.
%%

\item Schreiben Sie eine SQL-Anweisung, die die Namen aller Promis und
der Kollektionen be- stimmt, welche der Promi zwar promotet, aber daraus
noch kein Kleidungsstück getragen hat.

%%
% 4.
%%

\item Bestimmen Sie für die folgenden SQL-Anweisungen die minimale und
maximale Anzahl an Tupeln im Ergebnis. Beziehen Sie sich dabei auf die
Größe der einzelnen Tabellen.

Verwenden Sie für die Lösung folgende Notation:
— Promi — beschreibt die Größe der Tabelle Promi.

\begin{enumerate}

%%
% a)
%%

\item

\begin{minted}{sql}
SELECT k.Name
FROM Kollektion k, Kleidungsstueck kl
WHERE k.Name = kl.gehoert_zu and k.Jahr = 2018
GROUP BY k.Name
HAVING COUNT(k1.Hauptbestandteil)>10
\end{minted}

%%
% b)
%%

\item

\begin{minted}{sql}
SELECT DISTINCT k.Jahr
FROM Kollektion kK
WHERE k.Name IN (
SELECT pr.KollektionName
FROM Promi p, promotet pr
WHERE p.Alter < 30 AND pr.PromiName = p.Name
\end{minted}
\end{enumerate}

%%
% 5.
%%

\item Beschreiben Sie den Effekt der folgenden SQL-Anfrage in
natürlicher Sprache

\begin{minted}{sql}
SELECT pr.KollektionName
FROM promotet pr, Promi p
WHERE pr.PromiName = p.Name
GROUP BY pr.KollektionNane
HAVING COUNT (*) IN (

SELECT MAX(anzahl)

FROM (
SELECT k.Name, COUNT(*) AS anzahl
FROM Kollektion k, promotet pr
WHERE k.Name = pr.KollektionName
GROUP BY k.Name

)

)
\end{minted}

%%
% 6.
%%

\item Formulieren Sie folgende SQL-Anfrage in relationaler Algebra. Die
Lösung kann in Baum- oder in Term-Schreibweise angegeben werden, wobei
eine Schreibweise genügt.

\begin{minted}{sql}

SELECT p.Wohnort

FROM Promi p, promotet pr, Kollektion k
WHERE p.Name = pr.PromiName

AND k.Name pr.KollektionName

AND k.Jahr 2018
\end{minted}

\begin{enumerate}

%%
% a)
%%

\item Konvertieren Sie zunächst die gegebene SQL-Anfrage in die
zugehörige Anfrage in re- lationaler Algebra nach Standard-Algorithmus.

%%
% b)
%%

\item Führen Sie anschließend eine relationale Optimierung durch.
Beschreiben und be- gründen Sie dabei kurz jeden durchgeführten Schritt.

\end{enumerate}
\end{enumerate}
\end{document}
