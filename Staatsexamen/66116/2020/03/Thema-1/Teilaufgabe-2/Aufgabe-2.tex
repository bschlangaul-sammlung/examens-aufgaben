\documentclass{lehramt-informatik-aufgabe}
\liLadePakete{normalformen}
\begin{document}
\liAufgabenTitel{Sekretäre}
\section{Aufgabe 2
\index{Funktionale Abhängigkeiten}
\footcite{examen:66116:2020:03}}

\liPseudoUeberschrift{Relation „Sekretäre“}
\begin{center}
\begin{tabular}{|l|l|l|l|}
\hline
\textbf{PersNr} & \textbf{Name} & \textbf{Boss} & \textbf{Raum} \\\hline\hline
4000   & Freud   & 2125 & 225 \\\hline
4000   &         &      & 225 \\\hline
4020   & Röntgen & 2163 & 6 \\\hline
4020   & Röntgen &      & 26 \\\hline
4030   & Galileo & 2127 & \\\hline
       & Freud   & 2137 & 80 \\\hline
\end{tabular}
\end{center}

\noindent
Gegeben sei oben stehenden (lückenhafte) Relationenausprägung
\textbf{Sekretäre}
sowie die folgenden funktionalen Abhängigkeiten:

\liFunktionaleAbhaengigkeiten{
  PersNr -> Name;
  PersNr, Boss -> Raum
}

\noindent
Geben Sie für alle leeren Zellen Werte an, so dass keine funktionalen
Abhängigkeiten verletzt werden. (Hinweis: Es gibt mehrere richtige
Antworten.)

\begin{liAntwort}
\begin{center}
\begin{tabular}{|l|l|l|l|}
\hline
\textbf{PersNr} & \textbf{Name} & \textbf{Boss} & \textbf{Raum} \\\hline\hline
4000   & Freud   & 2125 & 225 \\\hline
4000   & \textit{Freud}   & \textit{2143}\footnote{Muss eine andere Boss-ID
sein, sonst gäbe es zwei identische Zeilen.} & 225 \\\hline
4020   & Röntgen & 2163 & 6 \\\hline
4020   & Röntgen & \textit{2163}\footnote{anderer Boss} & 26 \\\hline
4030   & Galileo & 2127 & \textit{27}\footnote{anderer Raum} \\\hline
\textit{4000}    & Freud   & 2137 & 80 \\\hline
\end{tabular}
\end{center}
\end{liAntwort}

\end{document}
