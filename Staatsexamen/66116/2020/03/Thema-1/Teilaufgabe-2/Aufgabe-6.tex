\documentclass{lehramt-informatik-aufgabe}
\liLadePakete{baum}
\begin{document}
\liAufgabenTitel{Physische Datenstrukturen)}
\section{Aufgabe 6
\index{B-Baum}
\footcite{examen:66116:2020:03}}

Fügen Sie die Zahl $4$ in den folgenden B-Baum ein.

\begin{center}
\begin{tikzpicture}[
  li bbaum,
  level 1/.style={level distance=15mm,sibling distance=28mm},
  level 2/.style={level distance=10mm,sibling distance=20mm},
]
\node {10 \nodepart{two} 20 \nodepart{three} . \nodepart{four} .} [->]
  child {
    node {5 \nodepart{two} 6 \nodepart{three} 7 \nodepart{four} 8}
  }
  child {
    node {12 \nodepart{two} 14 \nodepart{three} 16 \nodepart{four} 18}
  }
  child {
    node {25 \nodepart{two} 30 \nodepart{three} . \nodepart{four} .}
  };
\end{tikzpicture}
\end{center}

Zeichnen Sie den vollständigen, resultierenden Baum.

\begin{liAntwort}
\begin{center}
\begin{tikzpicture}[
  li bbaum,
  level 1/.style={level distance=15mm,sibling distance=28mm},
  level 2/.style={level distance=10mm,sibling distance=20mm},
]
\node {6 \nodepart{two} 10 \nodepart{three}  20 \nodepart{four} .} [->]
  child {
    node {4 \nodepart{two} 5 \nodepart{three} . \nodepart{four} .}
  }
  child {
    node {7 \nodepart{two} 8 \nodepart{three} . \nodepart{four} .}
  }
  child {
    node {12 \nodepart{two} 14 \nodepart{three} 16 \nodepart{four} 18}
  }
  child {
    node {25 \nodepart{two} 30 \nodepart{three} . \nodepart{four} .}
  };
\end{tikzpicture}
\end{center}
\end{liAntwort}

\end{document}
