\documentclass{lehramt-informatik-aufgabe}
\liLadePakete{baum}
\begin{document}
\liAufgabenTitel{Physische Datenstrukturen)}
\section{Aufgabe 6
\index{B-Baum}
\footcite{66116:2020:03}}

Fügen Sie die Zahl 4 in den folgenden B-Baum ein.

10 20 5 6 7 8 12 14 16 18 25 30

 5 6 7 8 10  12 14 16 18 20 25 30

Zeichnen Sie den vollständigen, resultierenden Baum.

\begin{tikzpicture}[
  li bbaum,
  level 1/.style={level distance=10mm,sibling distance=32mm},
  level 2/.style={level distance=10mm,sibling distance=20mm},
]
\node {7 \nodepart{two} 12 \nodepart{three} 18} [->]
  child {
    node {5 \nodepart{two} 6}
  }
  child {
    node {8 \nodepart{two} 10}
  }
  child {
    node {14 \nodepart{two} 16}
  }
  child {
    node {20 \nodepart{two} 25 \nodepart{three} 30}
  };
\end{tikzpicture}

\end{document}
