\documentclass{lehramt-informatik-aufgabe}
\liLadePakete{syntax}
\begin{document}
\liAufgabenTitel{Universitätssschema}
\section{Aufgabe 8
\index{SQL}
\footcite{66116:2020:03}}

\begin{minted}{sql}
CREATE TABLE Assistenten (
  PersNr INTEGER PRIMARY KEY,
  Name VARCHAR(20) DEFAULT NULL,
  Fachgebiet VARCHAR(30) DEFAULT NULL,
  Boss INTEGER DEFAULT NULL
);

CREATE TABLE Professoren (
  PersNr INTEGER PRIMARY KEY,
  Name VARCHAR(30) DEFAULT NULL,
  Rang VARCHAR(30) DEFAULT NULL,
  Raum INTEGER DEFAULT NULL
);

CREATE TABLE Studierende (
  MatrNr INTEGER PRIMARY KEYL,
  Name VARCHAR(15) NOT NULL,
  Semester INTEGER NOT NULL
);

CREATE TABLE hören (
  MatrNr INTEGER NOT NULL,
  VorlNr INTEGER NOT NULL
);

CREATE TABLE prüfen (
  MatrNr INTEGER PRIMARY KEY,
  VorlNr INTEGER NOT NULL,
  PersNr INTEGER NOT NULL,
  Note INTEGER NOT NULL
);

CREATE TABLE voraussetzen (
  Vorgaenger INTEGER NOT NULL,
  Nachfolger INTEGER NOT NULL
);

CREATE TABLE Vorlesungen (
  VorlNr INTEGER PRIMARY KEY,
  Titel VARCHAR(30) DEFAULT NULL,
  SWS INTEGER DEFAULT NULL,
  gelesenVon INTEGER DEFAULT NULL
);

INSERT INTO Assistenten (PersNr, Name, Fachgebiet, Boss) VALUES
  (3002, 'Platon', 'Ideenlehre', 2125),
  (3003, 'Aristoteles', 'Syllogistik', 2125),
  (3004, 'Wittgenstein', 'Sprachtheorie', 2126),
  (3005, 'Rhetikus', 'Planetenbewegung', 2127),
  (3006, 'Newton', 'Kaplersche Gesetze', 2127),
  (3007, 'Spinosa', 'Gott und Natur', 2124);

INSERT INTO hören (MatrNr, VorlNr) VALUES
  (25403, 5022),
  (26120, 5001),
  (27550, 4052),
  (27550, 5001),
  (28106, 5041),
  (28106, 5052),
  (28106, 5216),
  (28106, 5259),
  (29120, 5001),
  (29120, 5041),
  (29120, 5049),
  (29555, 5001),
  (29555, 5022);

INSERT INTO Professoren (PersNr, Name, Rang, Raum) VALUES
  (2125, 'Sokrates', 'C4', 226),
  (2126, 'Russel', 'C4', 226),
  (2127, 'Kopernikus', 'C3', 226),
  (2133, 'Popper', 'C3', 226),
  (2134, 'Augustinus', 'C3', 226),
  (2136, 'Curie', 'C4', 226),
  (2137, 'Kant', 'C4', 226);

INSERT INTO prüfen (MatrNr, VorlNr, PersNr, Note) VALUES
  (28106, 5001, 2126, 1),
  (25403, 5041, 2125, 2),
  (27550, 4630, 2137, 2),
  (25403, 4630, 2137, 5);

INSERT INTO Studierende (MatrNr, Name, Semester) VALUES
  (24002, 'Xenokrates', 18),
  (25403, 'Jonas', 12),
  (26120, 'Fichte', 10),
  (26830, 'Aristoxenos', 8),
  (27550, 'Schopenhauer', 6),
  (28106, 'Carnap', 3),
  (29120, 'Theophrastos', 2),
  (2955, 'Feuerbach', 2);

INSERT INTO Vorlesungen (VorlNr, Titel, SWS, gelesenVon) VALUES
  (4052, 'Logik', 4, 2125),
  (4630, 'Die 3 Kritiken', 4, 2137),
  (5001, 'Grundzüge', 4, 2137),
  (5022, 'Glaube und Wissen', 2, 2134),
  (5041, 'Ethik', 4, 2125),
  (5043, 'Erkenntnisstheorie', 3, 2126),
  (5049, 'Mäeutik', 2, 2125),
  (5052, 'Wissenschaftstheorie', 3, 2126),
  (5216, 'Bioethik', 2, 2126),
  (5259, 'Der Wiener Kreis', 2, 2133);
\end{minted}

Gegeben sei das Universitätssschema aus Aufgabe 3 (siehe Anhang - Seite
10). Formulieren Sie folgende Anfragen in SQL-92:

\begin{enumerate}

%%
%
%%

\item Welche Vorlesungen liest der Boss des Assistenten Platon (nur
Vorlesungsnummer und Titel ausgeben)?

%%
%
%%

\item Welche Studierende haben sich schon in mindestens einer direkten
Voraussetzung von ’Wissenschaftstheorie’ prüfen lassen?

%%
%
%%

\item Wie viele Studierende hören ’Ethik’?

%%
%
%%

\item Welche Studierende sind im gleichen Semester? — Geben Sie Paare
von Studierenden aus.

Achten Sie darauf, dass ein/e Studierende/r mit sich selbst kein Paar
bildet. — Achten Sie auch darauf, dass kein Paar doppelt ausgeben wird:
wenn das Paar StudentA, StudentB im Ergebnis enthalten ist, soll nicht
auch noch das Paar StudentB, StudentA ausgegeben werden.

%%
%
%%

\item In welchen Fächern ist die Durchschnittsnote schlechter als 2?
Geben Sie die Vorlesungsnummer und den Titel aus.

%%
%
%%

\item Finden Sie alle Paare von Studierenden (MatrNr duplikatfrei
ausgeben), die mindestens zwei Vorlesungen gemeinsam hören.

\end{enumerate}
\end{document}
