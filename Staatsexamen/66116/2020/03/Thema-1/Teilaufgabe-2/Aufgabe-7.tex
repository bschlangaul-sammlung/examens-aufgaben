\documentclass{lehramt-informatik-aufgabe}
\liLadePakete{normalformen}
\begin{document}
\let\FA=\liFunktionaleAbhaengigkeiten

\liAufgabenTitel{Zehnkampf}
\section{Aufgabe 7
\index{SQL}
\footcite{66116:2020:03}}

Gegeben sei die Relation Zehnkampf, welche die Ergebnisse eines
Zehnkampfwettkampfes verwaltet. Eine beispielhafte Ausprägung ist in
nachfolgender Tabelle gegeben.

Hinweise: Jeder Athlet kann in jeder Disziplin maximal ein Ergebnis
erzielen. Außerdem können Sie davon ausgehen, dass jeder Name eindeutig
ist.

\begin{tabular}{|l|l|l|l|l|}
\hline
Name  & Disziplin  & Leistung & Einheit    & Punkte \\\hline
John  & 100m       & 10.21    & Sekunden   & 845 \\
Peter & Hochsprung & 213      & Zentimeter & 812 \\
Peter & 100m       & 10.10    & Sekunden   & 920 \\
Hans  & 100m       & 10.21    & Sekunden   & 845 \\
Hans  & 400m       & 44.12    & Sekunden   & 910 \\
\end{tabular}
\begin{enumerate}

%%
% a)
%%

\item Bestimmen Sie alle funktionale Abhängigkeiten, die sinnvollerweise
in der Relation Zehnkampf gelten.

\begin{liAntwort}
\FA{
  Disziplin -> Einheit;
  Disziplin, Leistung -> Punkte;
  Name, Disziplin -> Leistung;
}
\end{liAntwort}

%%
% b)
%%

\item Normalisieren Sie die Relation Zehnkampf unter Beachtung der von
Ihnen identifzierten funktionalen Abhängigkeiten. Unterstreichen Sie
alle Schlüssel des resultierenden Schemas.

\begin{liAntwort}

\end{liAntwort}

%%
% c)
%%

\item Bestimmen Sie in SQL den Athleten (oder bei Punktgleichheit, die
Athleten), der in der Summe am meisten Punkte in allen Disziplinen
erzielt hat. Benutzen Sie dazu die noch nicht normalisierte
Ausgangsrelation Zehnkampf.

\begin{liAntwort}

\end{liAntwort}
\end{enumerate}
\end{document}
