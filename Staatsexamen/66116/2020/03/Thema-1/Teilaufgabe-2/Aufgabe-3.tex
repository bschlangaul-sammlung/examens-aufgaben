\documentclass{lehramt-informatik-aufgabe}
\liLadePakete{mathe}
\begin{document}
\liAufgabenTitel{Universitätsschema}
\section{Aufgabe 3
\index{Relationale Algebra}
\footcite{66116:2020:03}}

Gegeben sei ein Universitätsschema.

\begin{enumerate}

%%
% a)
%%

\item Finden Sie alle Studierenden, die keine Vorlesung hören.
Formulieren Sie die Anfrage im Tupelkalkül.
\index{Tupelkalkül}

\begin{liAntwort}
$\{ s \in \text{Studierende} \land h \in \text{hören} | \neg \exists \text{s.MatrNr} = \text{h.MatrNr} \}$
\end{liAntwort}

%%
% b)
%%

\item Geben Sie einen Ausdruck an, der die Relation \neg \texttt{hören}
erzeugt. Diese enthält für jeden Studierenden und jede Vorlesung, die
der Studierende \textbf{nicht} hört, einen Eintrag mit Matrikelnummer
und Vorlesungsnummer. Formulieren Sie die Anfrage in
\textbf{relationaler Algebra}.

\begin{liAntwort}
$\rho_{\neg \text{hören}} \left(
(
  \pi_{\text{MatrNr}}(\text{Studierende})
  \times
  \pi_{\text{VorlNr}}(\text{Vorlesungen})
) - \text{hören}
\right)
$
\end{liAntwort}

%%
% c)
%%

\item Finden Sie alle Studierenden, die \textbf{keine} Vorlesung hören.
Formulieren Sie die Anfrage in \textbf{relationaler Algebra}.

\begin{liAntwort}
$\pi_{\text{MatrNr}}(\text{Studierende}) - \pi_{\text{MatrNr}}(\text{hören})$
\end{liAntwort}
\end{enumerate}
\end{document}
