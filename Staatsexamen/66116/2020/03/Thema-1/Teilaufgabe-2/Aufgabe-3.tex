\documentclass{lehramt-informatik-aufgabe}
\liLadePakete{}
\begin{document}
\liAufgabenTitel{Universitätsschema}
\section{Aufgabe 3
\index{Relationale Algebra}
\footcite{66116:2020:03}}

Gegeben sei ein Universitätsschema (eine beispielhafte Ausprägung hängt
der Klausur an - Seite 10).

\begin{enumerate}

%%
% a)
%%

\item Finden Sie alle Studierenden, die keine Vorlesung hören.
Formulieren Sie die Anfrage im Tupelkalkül.
\index{Tupelkalkül}

%%
% b)
%%

\item Geben Sie einen Ausdruck an, der die Relation -hoeren erzeugt.
Diese enthält für jeden Studierenden und jede Vorlesung, die der
Studierende nicht hört, einen Eintrag mit Matri- kelnummer und
Vorlesungsnummer. Formulieren Sie die Anfrage in relationaler Algebra.

%%
% c)
%%

\item Finden Sie alle Studierenden, die keine Vorlesung hören.
Formulieren Sie die Anfrage in relationaler Algebra.
\end{enumerate}
\end{document}
