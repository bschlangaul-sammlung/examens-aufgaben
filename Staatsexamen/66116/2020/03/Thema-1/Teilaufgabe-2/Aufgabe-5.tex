\documentclass{lehramt-informatik-aufgabe}
\liLadePakete{}
\begin{document}
\liAufgabenTitel{Schlüssel}
\section{Aufgabe 5
\index{Schlüssel}
\footcite{66116:2020:03}}

Gegeben sei die Relation

R: {[A,B,C]}

\begin{enumerate}
\item Schreiben Sie eine SQL-Anfrage, mit der sich zeigen lässt, ob das
Paar A,B ein Superschlüssel der Relation R ist. Beschreiben Sie ggf.
textuell - falls nicht eindeutig ersichtlich - wie das Ergebnis Ihrer
Anfrage interpretiert werden muss, um zu erkennen ob A,B ein
Superschlüssel ist.

\begin{liAntwort}

\end{liAntwort}

\item Erläutern Sie den Unterschied zwischen einem Superschlüssel und
einem Kandidatenschlüssel. Tipp: Was muss gelten, damit A,B ein
Kandidatenschlüssel ist und nicht nur ein Superschlüssel?

\begin{liAntwort}

\end{liAntwort}

\item Sei A,B der Kandidatenschlüssel für die Relation R. Geben Sie eine
minimale Ausprägung der Relation R an, die diese Eigenschaft erfüllt.

\begin{liAntwort}

\end{liAntwort}

\end{enumerate}
\end{document}
