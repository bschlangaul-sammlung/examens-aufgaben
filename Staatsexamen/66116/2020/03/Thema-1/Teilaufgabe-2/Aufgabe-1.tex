\documentclass{lehramt-informatik-aufgabe}
\liLadePakete{er}
\begin{document}
\liAufgabenTitel{Einwohnermeldeamt}
\section{Aufgabe 1
\index{Entity-Relation-Modell}
\footcite{66116:2020:03}}

Gegeben sei folgendes ER-Diagramm:

\begin{enumerate}

%%
% a)
%%

\item Übernehmen Sie das ER-Diagramm auf Ihre Bearbeitung und ergänzen
Sie die Funktionalitätsangaben im Diagramm.

\begin{center}
\begin{tikzpicture}[er2,scale=0.7,transform shape]
% Menschen
\node[entity] (Menschen) {Menschen};
\node[attribute,above left=1cm of Menschen] {\key{Steuernummer}} edge (Menschen);
\node[attribute,above=1cm of Menschen] {Vornamen} edge (Menschen);
\node[attribute,above right=1cm of Menschen] {Name} edge (Menschen);

\node[relationship,left=1cm of Menschen] (verheiratet) {verheiratet\_mit};
\node[relationship,right=1cm of Menschen] (befreundet) {befreundet\_mit};

\draw (verheiratet.north) edge[auto] node{$\bigcirc$} (Menschen);
\draw (verheiratet.south) edge[auto,swap] node{$\bigcirc$} (Menschen);

\draw (befreundet.north) edge[auto,swap] node{$\bigcirc$} (Menschen);
\draw (befreundet.south) edge[auto] node{$\bigcirc$} (Menschen);

\node[relationship,below=1cm of Menschen] (geboren) {geboren}
  edge[auto] node{$\bigcirc$} (Menschen);
\node[attribute,right=1cm of geboren] {Datum}
  edge (geboren);

% Orte
\node[entity,below=3cm of Menschen] (Orte) {Orte}
 edge[auto] node{$\bigcirc$} (geboren);
\node[attribute,below left=1cm of Orte] {Einwohneranzahl} edge (Orte);
\node[attribute,left=1cm of Orte] {\key{Bundesland}} edge (Orte);
\node[attribute,above left=1cm of Orte] {\key{Name}} edge (Orte);
\end{tikzpicture}
\end{center}

\begin{liAntwort}
\begin{center}
\begin{tikzpicture}[er2,scale=0.7,transform shape]
% Menschen
\node[entity] (Menschen) {Menschen};
\node[attribute,above left=1cm of Menschen] {\key{Steuernummer}} edge (Menschen);
\node[attribute,above=1cm of Menschen] {Vornamen} edge (Menschen);
\node[attribute,above right=1cm of Menschen] {Name} edge (Menschen);

\node[relationship,left=1cm of Menschen] (verheiratet) {verheiratet\_mit};
\node[relationship,right=1cm of Menschen] (befreundet) {befreundet\_mit};

\draw (verheiratet.north) edge[auto] node{1} (Menschen);
\draw (verheiratet.south) edge[auto,swap] node{1} (Menschen);

\draw (befreundet.north) edge[auto,swap] node{n} (Menschen);
\draw (befreundet.south) edge[auto] node{m} (Menschen);

\node[relationship,below=1cm of Menschen] (geboren) {geboren}
  edge[auto] node{n} (Menschen);
\node[attribute,right=1cm of geboren] {Datum}
  edge (geboren);

% Orte
\node[entity,below=3cm of Menschen] (Orte) {Orte}
 edge[auto] node{1} (geboren);
\node[attribute,below left=1cm of Orte] {Einwohneranzahl} edge (Orte);
\node[attribute,left=1cm of Orte] {\key{Bundesland}} edge (Orte);
\node[attribute,above left=1cm of Orte] {\key{Name}} edge (Orte);
\end{tikzpicture}
\end{center}
\end{liAntwort}

%%
% b)
%%

\item Übersetzen Sie das ER-Diagramm in ein relationales Schema. -
Datentypen müssen nicht angegeben werden.

\begin{liAntwort}

\end{liAntwort}

%%
% c)
%%

\item Verfeinern Sie das Schema aus Teilaufgabe b) indem Sie die
Relationen zusammenfassen.

\begin{liAntwort}

\end{liAntwort}

%%
% d)
%%

\item Geben Sie sinnvolle SQL Datentypen für Ihr verfeinertes Schema an.

\begin{liAntwort}

\end{liAntwort}

\end{enumerate}
\end{document}
