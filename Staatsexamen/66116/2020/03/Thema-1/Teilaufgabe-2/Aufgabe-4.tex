\documentclass{lehramt-informatik-aufgabe}
\liLadePakete{normalformen,synthese-algorithmus}
\begin{document}

\let\ah=\liAttributHuelle
\let\ahl=\liLinksReduktionInline
\let\ahr=\liRechtsReduktionInline
\let\FA=\liFunktionaleAbhaengigkeiten
\let\fa=\liFunktionaleAbhaengigkeit
\let\m=\liAttributMenge
\let\r=\liRelation
\let\schrittE=\liSyntheseUeberErklaerung
\let\u=\underline

\liAufgabenTitel{Relation A-F}
\section{Aufgabe 4
\index{Synthese-Algorithmus}
\footcite{examen:66116:2020:03}}

Gegeben sei die Relation

\begin{center}
\liRelation{A, B, C, D, E, F}
\end{center}

\noindent
mit den FDs

\bigskip

% https://normalizer.db.in.tum.de/index.py
% A->BCF
% B->ABF
% CD->EF

\FA{
  A -> B, C, F;
  B -> A, B, F;
  C, D -> E, F;
}

\begin{enumerate}

%%
% a)
%%

\item Geben Sie alle Kandidatenschlüssel an.

\begin{liAntwort}
\begin{itemize}
\item \m{A, D}
\item \m{B, D}
\end{itemize}
\end{liAntwort}

%%
% b)
%%

\item Überführen Sie die Relation mittels Synthesealgorithmus in die 3.
NF. Geben Sie alle Relationen in der 3. NF an und \textbf{unterstreichen
Sie in jeder einen Kandidatenschlüssel.} — Falls Sie Zwischenschritte
notieren, machen Sie das Endergebnis \textbf{klar kenntlich.}

\begin{liAntwort}
\begin{enumerate}
\item \schrittE{1}
\begin{enumerate}
\item \schrittE{1-1}

\liPseudoUeberschrift{\fa{C, D -> E, F}}

$\m{E, F} \notin$ \ahl{C, D}{D}{C}\\
$\m{E, F} \notin$ \ahl{C, D}{C}{D}

\FA{
  A -> B, C, F;
  B -> A, B, F;
  C, D -> E, F;
}

\item \schrittE{1-2}

\liPseudoUeberschrift{F}

$F \in$ \ahr{A -> B, C, F}{A -> B, C}{A}{A, B, C, F}

\FA{
  A -> B, C;
  B -> A, B, F;
  C, D -> E, F;
}

$F \notin$ \ahr{B -> A, B, F}{B -> A, B}{B}{A, B, C}\\
$F \notin$ \ahr{C, D -> E, F}{C, D -> E}{C, D}{C, D, E}

\liPseudoUeberschrift{B}

$B \notin$ \ahr{A -> B, C}{A -> C}{A}{A, C}\\
$B \in$ \ahr{B -> A, B, F}{B -> A, F}{B}{A, \textbf{B}, F}

\FA{
  A -> B, C;
  B -> A, F;
  C, D -> E, F;
}

\item \schrittE{1-3}

\liNichtsZuTun

\item \schrittE{1-4}

\liNichtsZuTun

\end{enumerate}
\item \schrittE{2}

\r[R1]{\u{A, B}, C}\\
\r[R2]{\u{A, B}, F}\\
\r[R3]{\u{C, D}, E, F}\\

\item \schrittE{3}

\r[R1]{\u{A, B}, C}\\
\r[R2]{\u{A, B}, F}\\
\r[R3]{\u{C, D}, E, F}\\
\r[R4]{\u{A, D}}\\

\item \schrittE{4}

\liNichtsZuTun
\end{enumerate}
\end{liAntwort}

\end{enumerate}
\end{document}
