\documentclass{lehramt-informatik-aufgabe}
\liLadePakete{normalformen}
\begin{document}
\let\FA=\liFunktionaleAbhaengigkeiten
\liAufgabenTitel{Relation A-F}
\section{Aufgabe 4
\index{Synthesealgorithmus}
\footcite{66116:2020:03}}

Gegeben sei die Relation

R:{[A,B,C,D,E, F]}
mit den FDs

\FA{
  A -> B, C, F;
  B -> A, B, F;
  C, D -> E, F;
}

\begin{enumerate}

%%
% a)
%%

\item Geben Sie alle Kandidatenschlüssel an.

%%
% b)
%%

\item Überführen Sie die Relation mittels Synthesealgorithmus in die 3.
NF. Geben Sie alle Relationen in der 3. NF an und unterstreichen Sie in
jeder einen Kandidatenschlüssel. — Falls Sie Zwischenschritte notieren,
machen Sie das Endergebnis klar kenntlich.

\end{enumerate}
\end{document}
