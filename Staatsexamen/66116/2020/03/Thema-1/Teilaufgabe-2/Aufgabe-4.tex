\documentclass{lehramt-informatik-aufgabe}
\liLadePakete{normalformen,synthese-algorithmus}
\begin{document}
\let\FA=\liFunktionaleAbhaengigkeiten
\let\ah=\liAttributHuelle
\let\FA=\liFunktionaleAbhaengigkeiten
\let\m=\liAttributMenge
\let\r=\liRelation
\let\schrittE=\liSyntheseUeberErklaerung

\liAufgabenTitel{Relation A-F}
\section{Aufgabe 4
\index{Synthese-Algorithmus}
\footcite{66116:2020:03}}

Gegeben sei die Relation

\begin{center}
\liRelation{A, B, C, D, E, F}
\end{center}

\noindent
mit den FDs

\FA{
  A -> B, C, F;
  B -> A, B, F;
  C, D -> E, F;
}

\begin{enumerate}

%%
% a)
%%

\item Geben Sie alle Kandidatenschlüssel an.

\begin{liAntwort}
\begin{itemize}
\item A, D
\item B, D
\end{itemize}
\end{liAntwort}

%%
% b)
%%

\item Überführen Sie die Relation mittels Synthesealgorithmus in die 3.
NF. Geben Sie alle Relationen in der 3. NF an und \textbf{unterstreichen
Sie in jeder einen Kandidatenschlüssel.} — Falls Sie Zwischenschritte
notieren, machen Sie das Endergebnis \textbf{klar kenntlich.}

\begin{liAntwort}
\begin{enumerate}
\item \schrittE{1}
\begin{enumerate}
\item \schrittE{1-1}

\FA{
  A -> B, C, F;
  B -> A, B, F;
  C, D -> E, F;
}

\item \schrittE{1-2}

\FA{
  A -> B, C;
  B -> A, F;
  C, D -> E;
}

\item \schrittE{1-3}

nichts zu tun

\item \schrittE{1-4}

nichts zu tun

\end{enumerate}
\item \schrittE{2}

\liRelation[R1]{A, B, C}
\liRelation[R2]{A, B, F}
\liRelation[R3]{C, D, E, F}

\item \schrittE{3}

\liRelation[R1]{A, B, C}
\liRelation[R2]{A, B, F}
\liRelation[R3]{C, D, E, F}
\liRelation[R4]{A, D}

\item \schrittE{4}

nichts zu tun
\end{enumerate}
\end{liAntwort}

\end{enumerate}
\end{document}
