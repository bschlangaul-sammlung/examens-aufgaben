\documentclass{lehramt-informatik-aufgabe}
\liLadePakete{}
\begin{document}
\liAufgabenTitel{Triathlon}
\section{Aufgabe 4: SQL
\index{SQL}
\footcite{66116:2018:09}}

Gegeben sind folgende Relationen aus einer Datenbank zur Verwaltung von
Triathlon-Wettbewerben.

% Athlet (ID, Vorname, Nachname)

% Ergebnis (Athlet [Athlet], Wettbewerb [Wettbewerb], Schwimmzeit, Radzeit,
% Laufzeit); Schwimmzeit NOT NULL

% Wettbewerb (Name, Jahr)

\begin{liProjektSprache}{RelationenSchema}
Athlet (ID*, Vorname, Nachname)
Ergebnis (Athlet[Athlet]*, Wettbewerb[Wettbewerb]*, Laufzeit)
Wettbewerb (Name*, Schwimmzeit, Radzeit, Jahr)
\end{liProjektSprache}

Verwenden Sie im Folgenden nur Standard-SQL und keine
produktspezifischen Erweiterungen. Sie dürfen bei Bedarf Views anlegen.
Geben Sie einen Datensatz, also eine Entity, nicht mehrfach aus.

\begin{enumerate}

%%
% a)
%%

\item Schreiben Sie eine SQL-Anweisung, die die Tabelle „Ergebnis“
anlegt. Gehen Sie davon aus, dass die Tabellen „Athlet“ und „Wettbewerb
“ bereits existieren. Verwenden Sie sinnvolle Datentypen.

%%
% b)
%%

\item Schreiben Sie eine SQL-Anweisung, die die Radzeit des Teilnehmers
mit der ID 12 beim: Wettbewerb „Zürichsee“ um eins erhöht.

%%
% c)
%%

\item Schreiben Sie eine SQL-Anweisung, die die Namen aller Wettbewerbe
des Jahres 2018 ausgibt, absteigend sortiert nach Name.

%%
% d)
%%

\item Schreiben Sie eine SQL-Anweisung, die die Namen aller Wettbewerbe
ausgibt, in der die durchschnittliche Schwimmzeit größer als 10 ist.

%%
% e)
%%

\item Schreiben Sie eine SQL-Anweisung, die die IDs aller Athleten
ausgibt, die im Jahr 2017 an keinem Wettbewerb teilgenommen haben.

%%
% f)
%%

\item Schreiben Sie eine SQL-Anweisung, die die Nachnamen aller Athleten
ausgibt, die mindestens 10 Wettbewerbe gewonnen haben, das heißt im
jeweiligen Wettbewerb die kürzeste Gesamtzeit erreicht haben. Die
Gesamtzeit ist die Summe aus Schwimmzeit, Radzeit und Laufzeit. Falls
zwei Athleten in einem Wettbewerb die gleiche Gesamtzeit erreichen, sind
beide Sieger.

%%
% g)
%%

\item Schreiben Sie eine SQL-Anweisung, die die Top-Ten der Athleten mit
der schnellsten Schwimmzeit des Wettbewerbs „Paris“ ausgibt. Ausgegeben
werden sollen die Platzierung (1 bis 10) und der Nachname des Athleten,
aufsteigend sortiert nach Platzierung. Gehen Sie davon aus, dass keine
zwei Athleten die gleiche Schwimmzeit haben und verwenden Sie keine
produktspezifischen Anweisungen wie beispielsweise rownum, top oder
limit.

%%
% h)
%%

\item Schreiben Sie einen Trigger, der beim Einfügen neuer Tupel in die
Tabelle „Ergebnis“ die Schwimmzeit auf den Wert O setzt, falls diese
negativ ist.

\end{enumerate}

\end{document}
