\documentclass{lehramt-informatik-aufgabe}
\liLadePakete{gantt,cpm}
\begin{document}
\liAufgabenTitel{Gantt und zwei Softwareentwickler}

\section{Aufgabe 3: Gantt (erweiterte Aufgabenstellung*)
\index{Gantt-Diagramm}
\footcite[Seite 2]{sosy:ab:5}}

Ein Team von zwei Softwareentwicklern soll ein Projekt umsetzen, das in
sechs Arbeitspakete unterteilt ist. Die Dauer der Arbeitspakete und ihre
Abhängigkeiten können Sie aus folgender Tabelle entnehmen:
\footcite[Thema 2 Teilaufgabe 1 Aufgabe 1]{examen:66116:2018:09}

\begin{center}
\begin{tabular}{|l|l|l|}
\hline
Name & Dauer in Wochen & Abhängig von\\\hline\hline
A1 & 2 & - \\\hline
A2 & 5 & - \\\hline
A3 & 2 & A1 \\\hline
A4 & 5 & A1,A2 \\\hline
A5 & 7 & A3 \\\hline
A6 & 4 & A4 \\\hline
\end{tabular}
\end{center}

\begin{enumerate}

%%
% (a)
%%

\item Zeichnen Sie ein Gantt-Diagramm, das eine kürzestmögliche
Projektabwicklung beinhaltet.

\begin{antwort}
\begin{center}
\begin{ganttchart}[x unit=0.7cm, y unit chart=0.6cm]{1}{14}
\gantttitlelist{1,...,14}{1} \\
\ganttbar[name=A1]{A1}{3}{5} \\
\ganttbar[name=A2]{A2}{1}{5} \\
\ganttbar[name=A3]{A3}{6}{7} \\
\ganttbar[name=A4]{A4}{6}{10} \\
\ganttbar[name=A5]{A5}{8}{14} \\
\ganttbar[name=A6]{A6}{11}{14}
\ganttlink{A1}{A3}
\ganttlink{A1}{A4}
\ganttlink[link type=f-s]{A2}{A4}
\ganttlink[link type=f-s]{A3}{A5}
\ganttlink[link type=f-s]{A4}{A6}
\end{ganttchart}
\end{center}
\end{antwort}

%%
% (b)
%%

\item Bestimmen Sie die Länge des kritischen Pfades und geben Sie an,
welche Arbeitspakete an ihm beteiligt sind.

\begin{antwort}
Auf dem kritischen Pfad befinden die Arbeitspakete A2, A4 und A6. Die
Länge des kritischen Pfades ist 14.
\end{antwort}

%%
% (c*)
%%

\item Wandeln Sie das Gantt-Diagramm in ein
CPM-Netzplan\index{CPM-Netzplantechnik} um.

\begin{antwort}
\begin{tikzpicture}[font=\scriptsize,scale=0.95]
\ereignis{SP}(0,1.5)

\ereignis{1A}(1.5,3)
\ereignis{1E}(3,3)
\ereignis{2A}(1.5,0)
\ereignis{2E}(3,0)
\ereignis{3A}(4.5,3)
\ereignis{3E}(6,3)
\ereignis{4A}(4.5,0)
\ereignis{4E}(6,0)
\ereignis{5A}(7.5,3)
\ereignis{5E}(9,3)
\ereignis{6A}(7.5,0)
\ereignis{6E}(9,0)

\ereignis{EP}(10.5,1.5)

\vorgang(1A>1E){2}
\vorgang(2A>2E){5}
\vorgang(3A>3E){2}
\vorgang(4A>4E){5}
\vorgang(5A>5E){7}
\vorgang(6A>6E){4}

\scheinvorgang(1E>3A){}
\scheinvorgang(2E>4A){}
\scheinvorgang(1E>4A){}
\scheinvorgang(3E>5A){}
\scheinvorgang(4E>6A){}

\scheinvorgang(SP>1A){}
\scheinvorgang(SP>2A){}

\scheinvorgang(5E>EP){}
\scheinvorgang(6E>EP){}
\end{tikzpicture}
\end{antwort}

%%
% (d*)
%%

\item Berechnen Sie für jedes Ereignis den \emph{frühesten Termin} und
den \emph{spätesten Termin} sowie die \emph{Gesamtpufferzeiten}.

\begin{antwort}
{\scriptsize
\setlength{\tabcolsep}{5pt}
\begin{tabular}{|l|c|c|c|c|c|c|c|c|c|c|c|c|c|c|}
\hline
Ergebnis &SP&1A&1E&2A&2E&3A&3E&4A&4E&5A&5E&6A&6E&EP\\\hline\hline
$FZ_i$   &0 &0 &2 &0 &5 &2 &4 &5 &10&4 &11&10&14&14\\\hline
$SZ_i$   &0 &3 &5 &0 &5 &5 &7 &5 &10&7 &14&10&14&14\\\hline
$GP$     &0 &3 &3 &0 &0 &3 &3 &0 &0 &3 &3 &0 &0 &0 \\\hline
\end{tabular}
}
\end{antwort}
\end{enumerate}

\end{document}
