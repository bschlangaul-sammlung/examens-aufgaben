
\documentclass{lehramt-informatik-aufgabe}
\liLadePakete{normalformen}
\begin{document}
\let\ah=\liAttributHuelle
\let\m=\liAttributMenge
\let\FA=\liFunktionaleAbhaengigkeiten

\liAufgabenTitel{Synthese-Algorithmus bei Relationenschema A-F}

\section{Aufgabe 6: Normalformen\footcite{examen:66116:2018:03}}

Gegeben sei das Relationenschema R(A,B,C,D,E,F), sowie die Menge der
zugehörigen funktionalen Abhängigkeiten F' \footcite{db:pu:wh}

\FA{
  C -> B;
  B -> A;
  C, E -> D;
  E -> F;
  C, E -> F;
  C -> A;
}

\begin{enumerate}
\item Bestimmen Sie den Schlüsselkandidaten der Relation R und begründen
Sie, warum es keine weiteren Schlüsselkandidaten gibt.

\begin{liAntwort}
 C und E müssen immer Teil des Schlüsselkandidaten
\ah{F, \m{C, E} = \m{C, E, B, A, D, F}}

-> Superschlüssel
-> Schlüsselkandidat, weil minimal denn C und E müssen immer Teil sein.
-> kein anderer SK möglich, weil C und E immer Teil sein müssen.

Sie selbst aber schon minimal sind.
\end{liAntwort}

\item Überführen Sie das Relationenschema R mit Hilfe des
Synthesealgorithmus\index{Synthese-Algorithmus} in die dritte
Normalform\index{Dritte Normalform}. Führen Sie hierfür jeden der vier
Schritte durch und kennzeichnen Sie Stellen, bei denen nichts zu tun
ist.
\end{enumerate}

\end{document}
