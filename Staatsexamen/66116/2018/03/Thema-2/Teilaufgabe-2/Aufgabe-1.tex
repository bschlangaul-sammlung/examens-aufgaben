\documentclass{lehramt-informatik-minimal}
\InformatikPakete{syntax,uml}
\begin{document}

\section{Aufgabe 1: (Objektorientierte Programme und Reverse Engineering)
\footcite[Seite 15-16]{examen:66116:2018:03}}

Gegeben sei das folgende Java-Programm:

\begin{minted}{java}
class M {
  private boolean b;
  private F f;
  private A a;

  public void m() {
    f = new F();
    a = new A(f);
    b = true;
  }
}

class A {
  private R r;
  public A(I i) {
    r = i.createX();
  }
}

interface I {
  public X createX();
}

class F implements I {
  public X createX() {
    return new X(0, 0);
  }
}

abstract class R {
  protected int v;
}

class X extends R {
  private int v, w;
  public X(int v, int w) {
    this.v = v;
    this.w = w;
  }
}
\end{minted}

\begin{enumerate}

%%
% a)
%%

\item Das Subtypprinzip der objektorientierten Programmierung wird in
obigem Programmcode zweimal ausgenutzt. Erläutern Sie wo und wie dies
geschieht.

%%
% b)
%%

\item Zeichnen Sie ein UML-Klassendiagramm, das die statische Struktur
des obigen Programms modelliert. Instanzvariablen mit einem Klassentyp
sollen durch gerichtete Assoziationen mit Rollennamen und Multiplizität
am gerichteten Assoziationsende modelliert werden. Alle aus dem
Programmcode ersichtlichen statischen Informationen (insbesondere
Interfaces, abstrakte Klassen, Zugriffsrechte, benutzerdefinierte
Konstruktoren und Methoden) sollen in dem Klassendiagramm abgebildet
werden.

\begin{tikzpicture}
\umlclass[x=1,y=5]{M}{- boolean b}{+ m(): void}
\umlsimpleclass[x=0,y=2.5]{F}
\umlsimpleclass[x=5,y=5]{A}
\umlclass[x=7,y=2.5,type=abstract]{R}{\# int v}{}
\umlclass[x=3,y=0,type=interface]{I}{}{+ createX(): X}
\umlclass[x=8,y=0]{X}{- int v\\- int w}{}

\umluniassoc[mult=1]{A}{R}
\umluniassoc[mult=1]{M}{A}
\umluniassoc[mult=1]{M}{F}
\umluniassoc[mult=1]{I}{X}
\umldep[mult=1,pos=0.9]{A}{I}
\umlinherit{X}{R}
\umlreal{F}{I}
\end{tikzpicture}

%%
% c)
%%

\item Es wird angenommen, dass ein Objekt der Klasse M existiert, für
das die Methode \java{m()} aufgerufen wird. Geben Sie ein
Instanzendiagramm (Objektdiagramm) an, das alle nach der Ausführung der
Methode \java{m} existierenden Objekte und deren Verbindungen (Links)
zeigt.

%%
% d)
%%

\item Wie in Teil c) werde angenommen, dass ein Objekt der Klasse M
existiert, für das die Methode \java{m()} aufgerufen wird. Diese
Situation wird in Abb. 1 dargestellt. Zeichnen Sie ein Sequenzdiagramm,
das Abb. 1 so ergänzt, dass alle auf den Aufruf der Methode \java{m()}
folgenden Objekterzeugungen und Interaktionen gemäß der im Programmcode
angegebenen Konstruktor- und Methodenrümpfe dargestellt werden.
Aktivierungsphasen von Objekten sind durch längliche Rechtecke deutlich
zu machen.

\begin{tikzpicture}
\begin{umlseqdiag}
\umlobject[class=M]{m}
%,create text={new F()}
\umlcreatecall[class=F]{m}{f}
%,create text={new A(f)
\umlcreatecall[class=A]{m}{a}
\begin{umlcall}[op={createX()},return={:X}]{a}{f}
% ,create text={new X(0,0)
\umlcreatecall[class=X]{f}{x}
\end{umlcall}
\end{umlseqdiag}
\end{tikzpicture}
\end{enumerate}

\end{document}
