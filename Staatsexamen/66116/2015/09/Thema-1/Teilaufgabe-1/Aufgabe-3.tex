\documentclass{lehramt-informatik-aufgabe}
\liLadePakete{}
\begin{document}
\liAufgabenTitel{Vater und Mutter}
\section{3.SQL
\index{SQL}
\footcite{66116:2015:09}}

Gegeben seien folgende Relationen:
Mensch(ID, MutterID, VaterID)
Mann(ID)

Frau(ID)

Das zugehörige ER-Modell für dieses relationale Datenbankschema sieht
folgendermaßen aus:

Disjunkte
Spezialisierung

Bearbeiten Sie folgende Teilaufgaben:

\begin{enumerate}

%%
% a)
%%

\item Finden Sie die Töchter der Frau mit ID 42.

%%
% b)
%%

\item Gibt es Männer, die ihre eigenen Großväter sind? Formulieren Sie eine
geeignete SQL-Anfrage.

%%
% c)
%%

\item Definieren Sie eine View VaterKind (VaterID; KindID), die allen
Vätern (VaterID) ihre Kinder (KinderID) zuordnet. Diese View darf keine
NULL-Werte enthalten.

%%
% d)
%%

\item Verwenden Sie die View aus c), um alle Väter zurückzugeben,
absteigend geordnet nach der Anzahl ihrer Kinder.

%%
% e)
%%

\item Hugo möchte mit folgender Anfrage auf Basis der View aus c) alle
kinderlosen Männer erhalten:

SELECT VaterID

FROM VaterKind

GROUP BY VaterID

HAVING COUNT(KindID) = 0

\begin{enumerate}

%%
% i.
%%

\item Was ist das Ergebnis von Hugos Anfrage und warum?

%%
% ii.
%%

\item Formulieren Sie eine Anfrage, die tatsächlich alle kinderlosen
Männer zurückliefert.

\end{enumerate}

\end{enumerate}

Hinweis: Denken Sie daran, dass SQL auch Mengenoperationen kennt.

\end{document}
