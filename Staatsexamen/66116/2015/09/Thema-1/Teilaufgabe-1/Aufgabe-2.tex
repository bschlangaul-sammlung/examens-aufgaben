\documentclass{lehramt-informatik-aufgabe}

\begin{document}
\liAufgabenTitel{Relation A-H}
\section{2. Normalformen
\index{Normalformen}
\footcite{examen:66116:2015:09}}

Gegeben sei folgendes verallgemeinerte Relationenschema in 1. Normalform:
R:{{[4,B,C,D,E,F,G, H]}

Für R soll die folgende Menge FD von funktionalen Abhängigkeiten gelten:

eF>E
eA—>BD
eAE>D
eA>EF
eAG>H

Bearbeiten Sie mit diesen Informationen folgende Teilaufgaben. Vergessen
Sie dabei nicht Ihr Vorgehen stichpunktartig zu dokumentieren und zu
begründen.

\begin{enumerate}

%%
% a)
%%

\item Bestimmen Sie alle Schlüsselkandidaten von R. Begründen Sie
stichpunktartig, warum es außer den von Ihnen gefundenen
Schlüsselkandidaten keine weiteren geben kann.

%%
% b)
%%

\item Ist R in 2NF, 3NF?

%%
% c)
%%

\item Berechnen Sie eine kanonische Überdeckung von FD. Es genügt, wenn
Sie für jeden der vier Einzelschritte die Menge der funktionalen
Abhängigkeiten als Zwischenergebnis angeben.

%%
% d)
%%

\item Bestimmen Sie eine Zerlegung von R in 3NF. Wenden Sie hierfür den
Synthesealgorithmus an.

\end{enumerate}

\end{document}
