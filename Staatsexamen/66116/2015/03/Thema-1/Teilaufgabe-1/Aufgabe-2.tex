\documentclass{lehramt-informatik-aufgabe}
\liLadePakete{syntax}
\begin{document}
\liAufgabenTitel{Musik-CDs}
\section{DB.2 (Datenbankanfragen und -änderungen in SQL)
\index{SQL}
\footcite{66116:2015:03}}

Formulieren Sie in SQL die folgenden Anfragen, Views bzw.
Datenmanipulations-Statements an Teile der Musik-Datenbank aus
Teilaufgabe DB.1:

Interpret (Interpreten\_ID, Name, Bühnenstart, Geschäftsadresse),
CD (CD\_ID, Name, Interpreten\_ID, Erscheinungsdatum),
Musikstück (CD\_ID, Position, Titel, Länge),
Auszeichnung\_CD (CD\_ID, Typ),

Auszeichnung\_Stück (CD\_ID, Position, Typ).

% Datenbankname: musikcds
\begin{minted}{sql}
CREATE TABLE Interpret (
  Interpreten_ID integer PRIMARY KEY,
  Name varchar(20),
  Bühnenstart date,
  Geschäftsadresse varchar(100)
);

CREATE TABLE CD (
  CD_ID integer PRIMARY KEY,
  Name,
  Interpreten_ID,
  Erscheinungsdatum
);

CREATE TABLE Musikstück (
  CD_ID integer,
  Position,
  Titel,
  Länge
);

CREATE TABLE Auszeichnung_CD (
  CD_ID integer,
  Typ
);

CREATE TABLE Auszeichnung_Stück (
  CD_ID integer,
  Position,
  Typ
);

INSERT INTO Interpret VALUES (1, 'Adele', '2005-01-01', 'London');
\end{minted}

\begin{enumerate}

%%
% a)
%%

\item Welche CDs hat „Adele“ herausgebracht? Geben Sie die Namen der CDs
aus.

\begin{minted}{sql}
SELECT * from Interpret;
\end{minted}

%%
% b)
%%

\item Geben Sie für alle Interpreten - gegeben durch die ID und den
Namen - die Anzahl ihrer veröffentlichten CDs an.

%%
% c)
%%

\item Geben Sie die Länge des längsten Musikstücks auf der CD mit dem
Namen „Thriller“ des Interpreten „Michael Jackson“ an.

%%
% d)
%%

\item Geben Sie die Namen aller Interpreten aus, die eine Auszeichnung
für eine CD oder eines ihrer Musikstücke bekommen haben.

%%
% e)
%%

\item Fügen Sie ein, dass „Adele“ einen „Emmy“ für ihre CD mit dem Namen
„Adele 21“ bekommen hat.

%%
% f)
%%

\item Ändern Sie die Geschäftsadresse von „Genesis“ auf „Hollywood
Boulevard 13, Los Angeles“.

\end{enumerate}

\end{document}
