\documentclass{bschlangaul-aufgabe}
\bLadePakete{normalformen}
\begin{document}
\bAufgabenMetadaten{
  Titel = {Aufgabe 3},
  Thematik = {Relation A-F},
  RelativerPfad = Staatsexamen/66116/2015/03/Thema-1/Teilaufgabe-1/Aufgabe-3.tex,
  ZitatSchluessel = examen:66116:2015:03,
  BearbeitungsStand = unbekannt,
  Korrektheit = unbekannt,
  Stichwoerter = {Normalformen},
  ExamenNummer = 66116,
  ExamenJahr = 2015,
  ExamenMonat = 03,
  ExamenThemaNr = 1,
  ExamenTeilaufgabeNr = 1,
  ExamenAufgabeNr = 3,
}

\let\FA=\bFunktionaleAbhaengigkeiten

Gegeben sei das Relationenschema R=(U, F) mit der Attributmenge
\index{Normalformen}
\footcite{examen:66116:2015:03}

\begin{center}
\bAttributMenge{U}{A, B, C, D, E}
\end{center}

und der folgenden Menge F von funktionalen Abhängigkeiten:

\begin{center}

\FA{
  A -> B;
  A, B, C -> D;
  D -> B, C;
}
\end{center}

\begin{enumerate}

%%
% a)
%%

\item Geben Sie alle Schlüssel für das Relationenschema R (jeweils mit
Begründung) sowie die Nichtschlüsselattribute an.

%%
% b)
%%

\item Ist R in 3NF bzw. in BCNF? Geben Sie jeweils eine Begründung an.

%%
% c)
%%

\item Geben Sie eine Basis G von F an. Zerlegen Sie R mittels des
Synthesealgorithmus in ein 3NF-Datenbankschema. Es genügt, die
resultierenden Attributmengen anzugeben.
\end{enumerate}

\end{document}
