\documentclass{lehramt-informatik-aufgabe}
\liLadePakete{normalformen}
\begin{document}
\let\FA=\liFunktionaleAbhaengigkeiten

\liAufgabenTitel{Relation A-F}
\section{DB.3 (Funktionale Abhängigkeiten und Zerlegungen)
\index{Normalformen}
\footcite{66116:2015:03}}

Gegeben sei das Relationenschema R=(U, F) mit der Attributmenge

\begin{center}
\liAttributMenge{U}{A, B, C, D, E}
\end{center}

und der folgenden Menge F von funktionalen Abhängigkeiten:

\begin{center}

\FA{
  A -> B;
  A, B, C -> D;
  D -> B, C;
}
\end{center}

\begin{enumerate}

%%
% a)
%%

\item Geben Sie alle Schlüssel für das Relationenschema R (jeweils mit
Begründung) sowie die Nichtschlüsselattribute an.

%%
% b)
%%

\item Ist R in 3NF bzw. in BCNF? Geben Sie jeweils eine Begründung an.

%%
% c)
%%

\item Geben Sie eine Basis G von F an. Zerlegen Sie R mittels des
Synthesealgorithmus in ein 3NF-Datenbankschema. Es genügt, die
resultierenden Attributmengen anzugeben.
\end{enumerate}

\end{document}
