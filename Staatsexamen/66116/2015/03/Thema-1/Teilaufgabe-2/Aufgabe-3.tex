\documentclass{lehramt-informatik-aufgabe}
\liLadePakete{syntax,entwurfsmuster}
\begin{document}
\liAufgabenTitel{Kunden und Angestellte einer Firma}

\section{Aufgabe 3
\footcite[Vererbung und Abstrakte Klassen II, Aufgabe 2]{aud:pu:7}}

In dieser Aufgabe implementieren\index{Implementierung in Java} Sie ein
konzeptionelles Datenmodell für eine Firma, die Personendaten von
Angestellten und Kunden verwalten möchte. Gegeben seien dazu folgende
Aussagen:
\index{Vererbung}
\footcite[Thema 1 Teilaufgabe 2 Aufgabe 3 Seite 5]{examen:66116:2015:03}

\begin{itemize}
\item Eine \emph{Person} hat einen \emph{Namen} und ein
\emph{Geschlecht} (männlich oder weiblich).

\item Ein \emph{Angestellter} ist eine \emph{Person}, zu der zusätzlich
das monatliche \emph{Gehalt} gespeichert wird.

\item Ein \emph{Kunde} ist eine \emph{Person}, zu der zusätzlich eine
\emph{Kundennummer} hinterlegt wird.
\end{itemize}

\begin{enumerate}

%%
% (a)
%%

\item Geben Sie in einer objektorientierten Programmiersprache Ihrer
Wahl (geben Sie diese an) eine Implementierung des aus den obigen
Aussagen resultierenden konzeptionellen Datenmodells in Form von
\textbf{Klassen} und \textbf{Interfaces} an. Gehen Sie dabei wie folgt
vor:

\begin{itemize}
\item Schreiben Sie ein Interface\index{Interface} \liJavaCode{Person}
sowie zwei davon erbende Interfaces \liJavaCode{Angestellter} und
\liJavaCode{Kunde}. Die Interfaces sollen jeweils lesende
Zugriffsmethoden (Getter) die entsprechenden Attribute (Name,
Geschlecht, Gehalt, Kundennummer) deklarieren.

\begin{liAntwort}
\liJavaExamen{66116}{2015}{03}{Person}
\liJavaExamen{66116}{2015}{03}{Angestellter}
\liJavaExamen{66116}{2015}{03}{Kunde}
\end{liAntwort}

\item Schreiben Sie eine abstrakte Klasse\index{Abstrakte Klasse}
\liJavaCode{PersonImpl}, die das Interface \liJavaCode{Person}
implementiert. Für jedes Attribut soll ein Objektfeld angelegt werden.
Außerdem soll ein Konstruktor definiert werden, der alle Objektfelder
initialisiert.

\begin{liAntwort}
\liJavaExamen{66116}{2015}{03}{PersonImpl}
\end{liAntwort}

\item Schreiben Sie zwei Klassen \liJavaCode{AngestellterImpl} und
\liJavaCode{KundeImpl}, die von \liJavaCode{PersonImpl} erben und die
jeweils dazugehörigen Interfaces implementieren. Es sollen wiederum
Konstruktoren definiert werden, die alle Objektfelder initialisieren und
dabei auf den Konstruktor der Basisklasse \liJavaCode{PersonImpl} Bezug
nehmen.

\begin{liAntwort}
\liJavaExamen{66116}{2015}{03}{AngestellterImpl}
\liJavaExamen{66116}{2015}{03}{KundeImpl}
\end{liAntwort}
\end{itemize}

%%
% b)
%%

\item Verwenden Sie das Entwurfsmuster \textbf{Adapter}, um zu
ermöglichen, dass vorhandene Angestellte die Rolle eines Kunden
einnehmen können. Der Adapter soll eine zusätzliche Klasse sein, die das
Kunden-Interface implementiert. Wenn möglich, sollen Methodenaufrufe an
den adaptierten Angestellten delegiert werden. Möglicherweise müssen Sie
neue Objektfelder einführen.\index{Adapter}

\begin{liExkurs}[Entwurfsmuster Adapter]
\liEntwurfsAdapter
\end{liExkurs}

%%
% c)
%%

\item Verwenden Sie das Entwurfsmuster \textbf{Simple Factory}, um die
Erzeugung von Angestellten, Kunden, sowie Adapter-Instanzen aus Aufgabe
(b) zu vereinheitlichen. Die entsprechende Erzeugungs-Methode soll neben
einem Typ-Diskriminator (\zB einem Aufzählungstypen oder mehreren
booleschen Werten) alle Parameter übergeben bekommen, die für den
Konstruktor irgendeiner Implementierungsklasse des Interface \j{Person}
notwendig sind.

Hinweis: Um eine Adapter-Instanz zu erzeugen, müssen Sie möglicherweise
zwei Konstruktoren aufrufen.

\begin{liExkurs}[Entwurfsmuster Einfache Fabrik]
\liEntwurfsEinfacheFabrik
\end{liExkurs}

\end{enumerate}
\end{document}
