\documentclass{lehramt-informatik-aufgabe}
\liLadePakete{syntax}
\begin{document}
\liAufgabenTitel{Kunden und Angestellte einer Firma}

\section{Aufgabe 2: Vererbung\index{Vererbung} und Abstrakte Klassen II
\footcite{aud:pu:7}}

In dieser Aufgabe implementieren\index{Implementierung in Java} Sie ein
konzeptionelles Datenmodell für eine Firma, die Personendaten von
Angestellten und Kunden verwalten möchte. Gegeben seien dazu folgende
Aussagen:
\footcite[Thema 1 Teilaufgabe 2 Aufgabe 3 Seite 5]{examen:66116:2015:03}

\begin{itemize}
\item Eine Person hat einen Namen und ein Geschlecht (männlich oder
weiblich).

\item Ein Angestellter ist eine Person, zu der zusätzlich das monatliche
Gehalt gespeichert wird.

\item Ein Kunde ist eine Person, zu der zusätzlich eine Kundennummer
hinterlegt wird.
\end{itemize}

\begin{enumerate}

%%
% (a)
%%

\item Geben Sie in einer objektorientierten Programmiersprache Ihrer
Wahl (geben Sie diese an) eine Implementierung des aus den obigen
Aussagen resultierenden konzeptionellen Datenmodells in Form von Klassen
und Interfaces an. Gehen Sie dabei wie folgt vor:

\begin{itemize}
\item Schreiben Sie ein Interface\index{Interface} \liJavaCode{Person} sowie
zwei davon erbende Interfaces \liJavaCode{Angestellter} und \liJavaCode{Kunde}. Die
Interfaces sollen jeweils lesende Zugriffsmethoden (Getter) die
entsprechenden Attribute (Name, Geschlecht, Gehalt, Kundennummer)
deklarieren.

\begin{liAntwort}
\liJavaExamen{66116}{2015}{03}{Person}
\liJavaExamen{66116}{2015}{03}{Angestellter}
\liJavaExamen{66116}{2015}{03}{Kunde}
\end{liAntwort}

\item Schreiben Sie eine abstrakte Klasse\index{Abstrakte Klasse}
\liJavaCode{PersonImpl}, die das Interface \liJavaCode{Person} implementiert. Für
jedes Attribut soll ein Objektfeld angelegt werden. Außerdem soll ein
Konstruktor definiert werden, der alle Objektfelder initialisiert.

\begin{liAntwort}
\liJavaExamen{66116}{2015}{03}{PersonImpl}
\end{liAntwort}

\item Schreiben Sie zwei Klassen \liJavaCode{AngestellterImpl} und
\liJavaCode{KundeImpl}, die von \liJavaCode{PersonImpl} erben und die jeweils
dazugehörigen Interfaces implementieren. Es sollen wiederum
Konstruktoren definiert werden, die alle Objektfelder initialisieren und
dabei auf den Konstruktor der Basisklasse \liJavaCode{PersonImpl} Bezug
nehmen.

\begin{liAntwort}
\liJavaExamen{66116}{2015}{03}{AngestellterImpl}
\liJavaExamen{66116}{2015}{03}{KundeImpl}
\end{liAntwort}
\end{itemize}
\end{enumerate}
\end{document}
