\documentclass{lehramt-informatik-minimal}
\InformatikPakete{syntax}
\begin{document}

\section{Aufgabe 2: Vererbung und Abstrakte Klassen II
\footcite{aud:pu:7}}

In dieser Aufgabe implementieren Sie ein konzeptionelles Datenmodell für
eine Firma, die Personendaten von Angestellten und Kunden verwalten
möchte. Gegeben seien dazu folgende Aussagen:
\footcite[entnommen aus Staatexamen DB/ST 66116, Frühjahr 2015, Thema 1 Teilaufgabe 2 A3 a, Seite 5]{examen:66116:2015:03}

\begin{itemize}
\item Eine Person hat einen Namen und ein Geschlecht (männlich oder
weiblich).

\item Ein Angestellter ist eine Person, zu der zusätzlich das monatliche
Gehalt gespeichert wird.

\item Ein Kunde ist eine Person, zu der zusätzlich eine Kundennummer
hinterlegt wird.
\end{itemize}

\begin{enumerate}

%%
% (a)
%%

\item Geben Sie in einer objektorientierten Programmiersprache Ihrer
Wahl (geben Sie diese an) eine Implementierung des aus den obigen
Aussagen resultierenden konzeptionellen Datenmodells in Form von Klassen
und Interfaces an. Gehen Sie dabei wie folgt vor:

\begin{itemize}
\item Schreiben Sie ein Interface \java{Person} sowie zwei davon erbende
Interfaces \java{Angestellter} und \java{Kunde}. Die Interfaces sollen
jeweils lesende Zugriffsmethoden (Getter) die entsprechenden Attribute
(Name, Geschlecht, Gehalt, Kundennummer) deklarieren.

\begin{antwort}
\inputcode[firstline=3]{aufgaben/oomup/examen_66116_2015_03/Person}
\inputcode[firstline=3]{aufgaben/oomup/examen_66116_2015_03/Angestellter}
\inputcode[firstline=3]{aufgaben/oomup/examen_66116_2015_03/Kunde}
\end{antwort}

\item Schreiben Sie eine abstrakte Klasse \java{PersonImpl}, die das
Interface \java{Person} implementiert. Für jedes Attribut soll ein
Objektfeld angelegt werden. Außerdem soll ein Konstruktor definiert
werden, der alle Objektfelder initialisiert.

\begin{antwort}
\inputcode[firstline=3]{aufgaben/oomup/examen_66116_2015_03/PersonImpl}
\end{antwort}

\item Schreiben Sie zwei Klassen \java{AngestellterImpl} und
\java{KundeImpl}, die von \java{PersonImpl} erben und die jeweils
dazugehörigen Interfaces implementieren. Es sollen wiederum
Konstruktoren definiert werden, die alle Objektfelder initialisieren und
dabei auf den Konstruktor der Basisklasse \java{PersonImpl} Bezug
nehmen.

\begin{antwort}
\inputcode[firstline=3]{aufgaben/oomup/examen_66116_2015_03/AngestellterImpl}
\inputcode[firstline=3]{aufgaben/oomup/examen_66116_2015_03/KundeImpl}
\end{antwort}
\end{itemize}
\end{enumerate}
\end{document}
