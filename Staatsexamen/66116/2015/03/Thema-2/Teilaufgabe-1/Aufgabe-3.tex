\documentclass{lehramt-informatik-aufgabe}
\liLadePakete{}
\begin{document}
\liAufgabenTitel{Indexstrukturen}
\section{Aufgabe 3
\index{B-Baum}
\footcite{66116:2015:03}}

Als Indexstruktur einer Datenbank sei folgender B-Baum (k = 2) gegeben:

Führen Sie nacheinander die folgenden Operationen aus. Geben Sie die
auftretenden Zwischenergebnisse an. Teilbäume, die sich in einem Schritt
nicht verändern, müssen nicht erneut gezeichnet werden. Sollten
Wahlmöglichkeiten auftreten, so sind größere Schlüsselwerte bzw. weiter
rechts liegende Knoten zu bevorzugen.

\begin{enumerate}

%%
% a)
%%

\item Einfügen des Wertes 56

%%
% b)
%%

\item Löschen des Wertes 37
\end{enumerate}

\end{document}
