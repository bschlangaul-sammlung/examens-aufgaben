\documentclass{lehramt-informatik-aufgabe}
\liLadePakete{syntax,mathe}
\begin{document}
\liAufgabenTitel{Polizei}

\section{Aufgabe 2
\index{Entity-Relation-Modell}
\footcite[Thema 1 Teilaufgabe 1 Aufgabe 2]{examen:66116:2016:03}}

\noindent
Gehen Sie dabei von dem dazugehörigen relationalen Schema aus:
\footcite{db:pu:3}

\bigskip

{
\noindent
\ttfamily
\footnotesize
Polizist: \{[\underline{PersNr}, DSID, Vorname, Nachname, Dienstgrad, Gehalt]\}\\
Dienststelle: \{[\underline{DSID}, Name, Strasse, HausNr, Stadt ]\}\\
Fall: \{[\underline{AkZ}, Titel, Beschreibung, Status ]\}\\
Arbeitet\_An: \{[\underline{PersNr}, AkZ, Von, Bis]\}\\
Vorgesetzte: \{[\underline{PersNr}, PersNr, Vorgesetzter]\}
}

\bigskip

% Datenbankname: polizei
\begin{minted}{sql}
CREATE TABLE Fall (
  AkZ VARCHAR (30) PRIMARY KEY,
  Titel VARCHAR (30),
  Beschreibung VARCHAR (50),
  Status VARCHAR (30)
);

CREATE TABLE Dienststelle (
  DSID INTEGER PRIMARY KEY,
  Name VARCHAR (50),
  Strasse VARCHAR (30),
  HausNr VARCHAR (30),
  Stadt VARCHAR (30)
);

CREATE TABLE Polizist (
  PersNr INTEGER Primary KEY,
  DSID INTEGER REFERENCES Dienststelle(DSID),
  Vorname VARCHAR (30),
  Nachname VARCHAR (30),
  Dienstgrad VARCHAR (30),
  Gehalt INT
);

CREATE TABLE Arbeitet_An (
  PersNr INTEGER References Polizist(PersNr),
  AkZ VARCHAR (30) References Fall(AkZ),
  Von DATE,
  Bis DATE,
  PRIMARY KEY (PersNr, AkZ)
);

CREATE TABLE Vorgesetzte (
  PersNr INTEGER References Polizist(PersNr),
  PersNr_Vorgesetzter INTEGER References Polizist(PersNr),
  PRIMARY KEY (PersNr, PersNr_Vorgesetzter)
);

INSERT INTO Dienststelle VALUES
  (10, 'Dienstelle München (Marienplatz)', NULL, NULL, 'München'),
  (11, 'Dienststelle Nürnberg (Mitte)', NULL, NULL, 'Nürnberg'),
  (12, 'Dieststelle Augsburg Ost', NULL, NULL, 'Augsburg');

INSERT INTO Polizist VALUES
  (1, 10, 'Hans', 'Müller', 'Polizeimeister', 40000),
  (2, 11, 'Josef', 'Fischer', 'Polizeihauptmeister', 45000),
  (3, 10, 'Andreas', 'Schmidt', 'Polizeikommisar', 50000),
  (4, 12, 'Stefan', 'Hoffmann', 'Polizeidirektor', 70000),
  (5, 11, 'Sebastian', 'Wagner', 'Polizeioberkommisar', 60000);

INSERT INTO Fall VALUES
  ('VR30932', 'Mord im Fussballstadion', 'Toter BVB-Fan', 'bearbeitet'),
  ('XZ1508', 'Steuerhinterziehung', 'Durchsuchung eines Hauses', 'bearbeitet');

INSERT INTO Arbeitet_An VALUES
  (1, 'VR30932', '2012-02-15', '2012-04-12'),
  (2, 'XZ1508', '2012-02-13', '2012-02-15');

INSERT INTO Vorgesetzte VALUES
  (1,3),
  (1,4),
  (2,5),
  (2,4);
\end{minted}

\noindent
Gegeben sei folgendes ER-Modell, welches Polizisten, deren Dienststelle
und Fälle, an denen sie arbeiten, speichert:

\begin{enumerate}

%%
% (a)
%%

\item Formulieren Sie eine Anfrage in relationaler
Algebra\index{Relationale Algebra}, welche den \emph{Vornamen} und
\emph{Nachnamen} von Polizisten zurückgibt, deren Dienstgrad
\emph{„Polizeikommissar“} ist und die mehr als 1500 Euro verdienen.

\begin{antwort}
$\pi_{\text{Vorname,Nachname}}(
  \sigma_{
    \text{Dienstgrad} = \mlq \text{Polizeikommissar} \mrq
      \land
    \text{Gehalt} > 1500
  }(\text{Polizist})
)$
\end{antwort}

%%
% (b)
%%

\item Formulieren Sie eine Anfrage in relationaler Algebra, welche die
\emph{Titel} der \emph{Fälle} ausgibt, die von \emph{Polizisten} mit dem
\emph{Nachnamen} \emph{„Mayer“} bearbeitet wurden.

\begin{antwort}
$
\pi_{\text{Titel}}(
  \sigma_{\text{Nachname} = \mlq \text{Mayer} \mrq}(\text{Polizist})
  \bowtie_{\text{PersNr}}
  \text{Arbeitet\_an}
  \bowtie_{\text{AkZ}}
  \text{Fall}
)
$
\end{antwort}

%%
% (c)
%%

\item Formulieren Sie eine SQL-Anfrage\index{SQL}, welche die Anzahl der
Polizisten ausgibt, die in der Stadt \emph{„München“} arbeiten und mit
Nachnamen \emph{„Schmidt“} heißen.

\begin{antwort}
\begin{minted}{sql}
SELECT COUNT(*) AS Anzahl_Polizisten
FROM Polizist p, Dienststelle d
WHERE
  p.DSID = d.DSID AND
  d.Stadt = 'München' AND
  p.Nachname = 'Schmidt';
\end{minted}
\end{antwort}

%%
% (d)
%%

\item Formulieren Sie eine SQL-Anfrage, welche die \emph{Namen} der
\emph{Dienststellen} ausgibt, die am \emph{14.02.2012} an dem Fall mit
dem \emph{XZ1508} beteiligt waren. Ordnen Sie die Ergebnismenge
alphabetisch (aufsteigend) und achten Sie darauf, dass keine Duplikate
enthalten sind.

\begin{antwort}
\begin{minted}{sql}
SELECT DISTINCT d.Name
FROM Dienststelle d, Polizist p, Arbeitet_an a
WHERE
  a.AkZ = 'XZ1508' AND
  p.PersNr = a.PersNr AND
  p.DSID = d.DSID AND
  a.Von >= '2012-02-14' AND
  a.Bis <= '2012-02-14'
ORDER BY d.Name ASC;
\end{minted}
\end{antwort}

%%
% (e)
%%

\item Definieren Sie die View \emph{„Erstrebenswerte Dienstgrade“},
welche Dienstgrade enthalten soll, die in \emph{München} mit
durchschnittlich mehr als \emph{2500} Euro besoldet werden.
\index{VIEW}

\begin{antwort}
\begin{minted}{sql}
CREATE VIEW ErstrebenswerteDienstgrade AS (
  SELECT DISTINCT p.Dienstgrad
  FROM Polizist p, Dienststelle d
  WHERE
    p.DSID = d.DSID AND
    d.Stadt = 'München'
  GROUP BY Dienstgrad
  HAVING (AVG(Gehalt) > 2500)
);
\end{minted}
\end{antwort}

%%
% (f)
%%

\item Formulieren Sie eine SQL-Anfrage, welche \emph{Vorname},
\emph{Nachname} und \emph{Dienstgrad} von \emph{Polizisten} mit
\emph{Vorname}, \emph{Nachname} und \emph{Dienstgrad} ihrer
\emph{Vorgesetzten} als ein Ergebnis-Tupel ausgibt (siehe
Beispiel-Tabelle). Dabei sind nur \emph{Polizisten} zu selektieren, die
an Fällen gearbeitet haben, deren Titel den Ausdruck „Fussball“
beinhalten. An \emph{Vorgesetzte} sind keine Bedingungen gebunden.
Achten Sie darauf, dass Sie nicht nur direkte Vorgesetzte, sondern alle
Vorgesetzte innerhalb der Vorgesetzten-Hierarchie betrachten. Ordnen Sie
ihre Ergebnismenge alphabetisch (absteigend) nach Nachnamen des
Polizisten.\index{WITH}\index{UNION}

Hinweis: Sie dürfen Views verwenden, um Teilergebnisse auszudrücken.

\begin{antwort}

\begin{minted}{sql}
WITH RECURSIVE meins (PersNr, VN, NN, DG, VN_VG, NN_VG, DG_VG) AS (
SELECT p1.PersNr, p1.Vorname as VN, p1.Nachname AS VN, p1.Dienstgrad AS DG_VG
  FROM Polizist p1, Fall f, Arbeitet_An a, Polizist p2, Vorgesetzte v
  WHERE
    p1.PersNr = a.PersNr AND
    a.AkZ = f.Akz AND
    f.Titel LIKE '%Fussball%' AND
    v.PersNr_Vorgesetzter = p2.PersNr
  UNION ALL
  SELECT m.PersNr, m.Vorname, m.Nachname, p.Vorname AS VN,
    p.Nachname AS VN, p.Dienstgrad AS DG_VG
  FROM meins m, Polizist p, Vorgesetzte v
  WHERE m.PersNr = v.PersNr AND p.PersNr_Vorgesetzter = p.PersNr
)
SELECT VN, NN, NN, DG, VN_DG, NN_VG, DG_VG
FROM meins
ORDER BY NN DESC;
\end{minted}
\end{antwort}
\end{enumerate}

\end{document}
