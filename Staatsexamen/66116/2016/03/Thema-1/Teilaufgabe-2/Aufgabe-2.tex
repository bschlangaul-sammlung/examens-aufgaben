\documentclass{lehramt-informatik-minimal}
\InformatikPakete{}
\begin{document}

\section{Aufgabe 2\footcite{sosy:ab:6}}

\includegraphics[width=\linewidth]{\LehramtInformatikRepository/Staatsexamen/66116/2016/03/Thema-1/Teilaufgabe-2/UML-Examen_2016-03.png}

\begin{enumerate}

%%
% 1.
%%

\item Kennzeichnen Sie im folgenden Klassendiagramm die Entwurfsmuster
\emph{„Abstrakte Fabrik“}, \emph{„Iterator“}, \emph{„Adapter“} und
\emph{„Kompositum“}. Geben Sie die jeweils beteiligten Klassen und deren
Zuständigkeit im entsprechenden Muster an.
\footcite[StEx F16, T1, Teilaufgabe 2 A2 (abgeändert)]{examen:66116:2016:03}

\begin{antwort}

\begin{description}

%%
%
%%

\item[Iterator] \strut

\begin{description}
\item[DocumentTraverser (interface)]
Schnittstelle zur Traversierung und zum Zugriff auf Dokumente

\item[Document]
implementiert die Schnittstelle
\end{description}

%%
%
%%

\item[Kompositum] \strut

\begin{description}
\item[AbstractDocument]
abstrakte Basisklasse, die gemeinsames Verhalten der beteiligten
Klassen definiert

\item[Document]
enthält wiederum weitere Documente bzw. DocumentEntities und Images

\item[DocumentEntity, Image]
primitive Unterklassen, besitzen keine Kindobjekte
\end{description}

%%
%
%%

\item[Adapter (Objektadapter)] \strut

\begin{description}
\item[Banking (interface)]
vom Client (hier Application) verwendete Schnittstelle

\item[HBCTBanking]
passt Schnittstelle der unpassenden Klasse an Zielschnittstelle
(Banking) an

\item[DB (interface)]
anzupassende Schnittstelle
\end{description}

%%
%
%%

\item[abstrakte Fabrik] \strut

\begin{description}
\item[JobCreator (interface)]
abstrakte Fabrik

\item[Job (abstrakt) mit Unterklassen RecordFetch und Transfer]
abstraktes Produkt

\item[Job (konkret) mit Unterklassen]
konkretes Produkt
\end{description}
\end{description}
\end{antwort}

%%
% 2.
%%

\item

\begin{enumerate}

%%
% (a)
%%

\item Beschreiben Sie die Funktionsweise der folgenden Entwurfsmuster
und geben Sie ein passendes UML-Diagramm an.

\begin{itemize}
\item Dekorierer
\item Klassenadapter
\item Objektadapter
\end{itemize}

%%
% (b)
%%

\item Erklären Sie mit maximal zwei Sätzen den Unterschied zwischen
Klassenadapter und Objektadapter.
\end{enumerate}

%%
% 3.
%%

\item Implementieren Sie einen Stapel in der Programmiersprache Java.
Nutzen Sie dazu ein Array mit fester Größe. Auf eine Überlaufprüfung
darf verzichtet werden. Implementieren Sie in der Klasse das Iterator
Entwurfsmuster, um auf die Inhalte zuzugreifen, sowie eine Funktion zum
Hinzufügen von Elementen. Als Typ für den Stapel kann zur Vereinfachung
ein Integertyp verwendet werden.
\end{enumerate}
\end{document}
