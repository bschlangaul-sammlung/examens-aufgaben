\documentclass{lehramt-informatik-aufgabe}
\liLadePakete{}
\begin{document}
\liAufgabenTitel{}
\section{Physische Datenorganisation
\index{B-Baum}
\footcite{66116:2016:09}}
\begin{enumerate}

%%
% a)
%%

\item Erläutern Sie die wesentliche Eigenschaft eines
Tupel-Identifikators (TID) in ein bis zwei Sätzen.

%%
% b)
%%

\item Fügen Sie in einen anfangs leeren B-Baum mit k = 1 (maximal 2
Schlüsselwerte pro Knoten) die im Folgenden gegebenen Schlüsselwerte der
Reihe nach ein. Zeichnen Sie den Endzustand des Baums nach jedem
Einfügevorgang. Falls Sie Zwischenschritte zeichnen, kennzeichnen Sie
die sieben Endzustände deutlich.

3, 7, 13, 11, 9, 10, 8

%%
% c)
%%

\item Gegeben ist der folgende B-Baum:

Die folgenden Teilaufgaben sind voneinander unabhängig.
\begin{enumerate}

%%
%
%%

\item Löschen Sie aus dem gegebenen B-Baum den Schlüssel 3 und zeichnen
Sie den Endzustand des Baums nach dem Löschvorgang. Falls Sie
Zwischenschritte zeichnen, kennzeichnen Sie den Endzustand deutlich.

%%
%
%%

\item Löschen Sie aus dem (originalen) gegebenen B-Baum den Schlüssel 17
und zeichnen Sie den Endzustand des Baums nach dem Löschvorgang. Falls
Sie Zwischenschritte zeichnen, kennzeichnen Sie den Endzustand deutlich.

%%
%
%%

\item Löschen Sie aus dem (originalen) gegebenen B-Baum den Schlüssel 43
und zeichnen Sie den Endzustand des Baums nach dem Löschvorgang. Falls
Sie Zwischenschritte zeichnen, kennzeichnen Sie den Endzustand deutlich.
\end{enumerate}
\end{enumerate}
\end{document}
