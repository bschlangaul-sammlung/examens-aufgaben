\documentclass{lehramt-informatik-minimal}
\InformatikPakete{syntax}
\begin{document}

\section{Aufgabe 1: SQL
\index{SQL}
\footcite[Thema 1 Teilaufgabe 1 Aufgabe 4]{examen:66116:2016:09}}

Gegeben sind folgende Relationen aus einer Personalverwaltung:
\footcite{db:pu:3}

\begin{minted}{md}
Mitarbeiter (MitarbeiterID), Vorname, Nachname, Vorgesetzter[Mitarbeiter], AbteilungsID[Abteilung], Telefonnummer, Gehalt)
Abteilung (AbteilungsID, Bezeichnung)
\end{minted}

\begin{enumerate}

%%
% (a)
%%

\item Schreiben Sie eine SQL-Anfrage, die \emph{Vor-} und
\emph{Nachnamen} der \emph{Mitarbeiter} aller \emph{Abteilungen} mit der
Bezeichnung \emph{„Buchhaltung“} ausgibt, absteigend sortiert nach
\emph{Mitarbeiter-ID}.

\begin{antwort}
\begin{minted}{sql}
SELECT Vorname, Nachname
FROM Mitarbeiter m, Abteilung a
WHERE
  m.AbteilungsID = a.AbteilungsID AND
  a.Bezeichung = 'Buchhaltung'
ORDER BY m.MitarbeiterID DESC;
\end{minted}
\end{antwort}

%%
% (b)
%%

\item Schreiben Sie eine SQL-Anfrage, die die Nachnamen aller
Mitarbeiter mit dem Nachnamen ihres jeweiligen direkten Vorgesetzten
ausgibt. Mitarbeiter ohne Vorgesetzten sollen in der Ausgabe ebenfalls
enthalten sein. In diesem Fall soll der Nachname des Vorgesetzten NULL
sein.

\begin{antwort}
\begin{minted}{sql}
SELECT m.Nachname AS Mitarbeiter, v.Nachname AS Vorgesetzter
FROM Mitarbeiter m LEFT OUTER JOIN Mitarbeiter v
ON m.Vorgesetzter = v.MitarbeiterID;
\end{minted}
\end{antwort}

%%
% (c)
%%

\item Schreiben Sie eine SQL-Anfrage, die die 10 Abteilungen ausgibt,
deren Mitarbeiter das höchste Durchschnittsgehalt haben. Ausgegeben
werden sollen der Rang (1 = höchstes Durchschnittsgehalt bis 10 =
niedrigstes Durchschnittsgehalt), die Bezeichnung sowie das
Durchschnittsgehalt der Abteilung. Gehen Sie davon dass es keine zwei
Abteilungen mit gleichem Durchschnittsgehalt gibt. Sie können der
Übersichtlichkeit halber Views oder With-Anweisungen verwenden.
Verwenden Sie jedoch keine datenbanksystemspezifischen Erweiterungen wie
\verb|limit| oder \verb|rownum|.\index{VIEW}\index{WITH}

\begin{antwort}
\begin{minted}{sql}
CREATE VIEW Durchschnittsgehaelter AS
SELECT Abteilung.AbteilungsID, Bezeichnung,
  AVG (Gehalt) AS Durchschnittsgehalt
FROM Mitarbeiter, Abteilung
WHERE Mitarbeiter.AbteilungsID = Abteilung.AbteilungsID
GROUP BY Abteilung.AbteilungsID, Bezeichnung

SELECT a.Bezeichnung, a.Durchschnittsgehalt, COUNT (*) AS Rang
FROM Durchschnittsgehaelter a, Durchschnittsgehaelter b
WHERE a.Durchschnittsgehalt <= b.Durchschnittsgehalt
GROUP BY a.AbteilungsID, a.Bezeichnung, a.Durchschnittsgehalt
HAVING COUNT (*) <= 10
ORDER BY Rang ASC
\end{minted}
\end{antwort}

%%
% (d)
%%

\item Schreiben Sie eine SQL-Anfrage, die das Gehalt aller Mitarbeiter
aus der Abteilung mit der AbteilungsID 42 um 5\% erhöht.

\begin{antwort}
\begin{minted}{sql}
UPDATE Mitarbeiter
SET Gehalt = 1.05 * Gehalt
WHERE AbteilungsID = 42
\end{minted}
\end{antwort}

%%
% (e)
%%

\item Alle \emph{Abteilungen} mit Bezeichnung
\emph{„Qualitätskontrolle“} sollen zusammen mit den Datensätzen ihrer
\emph{Mitarbeiter} gelöscht
werden. \verb|ON DELETE CASCADE| ist für keine der Tabellen gesetzt.
Schreiben Sie die zum Löschen notwendigen SQL-Anfragen.\index{DELETE}

\begin{antwort}
\begin{minted}{sql}
DELETE FROM Mitarbeiter
WHERE AbteilungsID ANY (
  SELECT a.AbteilungsID
  FROM Abteilung a
  WHERE a.Bezeichnung = 'Qualitaetskontrolle'
);

DELETE FROM Abteilung
WHERE Bezeichnung = 'Qualitaetskontrolle';
\end{minted}
\end{antwort}

%%
% (f)
%%

\item Alle Mitarbeiter sollen mit SQL-Anfragen nach den Telefonnummern
anderer Mitarbeiter suchen können. Sie dürfen jedoch das Gehalt der
Mitarbeiter nicht sehen können. Erläutern Sie in zwei bis drei Sätzen
eine Möglichkeit, wie dies in einem Datenbanksystem realisiert werden
kann, ohne die gegebenen Relationen, die Tabellen als abgelegt sind, zu
verändern. Sie brauchen hierzu keinen SQL-Code schreiben.

\begin{antwort}
VIEW Erstellen, die zwar Namen und ID der anderen Mitarbeiter, sowie
ihre Telefonnummern enthält (evtl. auch Abteilungsbezeichnung und ID),
aber eben nicht das Gehalt: Mitarbeiter arbeiten auf eingeschränkter
Sicht

Alternativ mit \verb|GRANT|:

explizit mit \verb|SELECT| die Spalten auswählen,
die man lesen können soll (auf nicht angegebene Spalten ist kein Zugriff
möglich)

\begin{minted}{sql}
GRANT SELECT (Vorname, Nachname, Telefonnummer)
ON Mitarbeiter TO alle anderen Mitarbeiter
\end{minted}
\end{antwort}
\end{enumerate}

\begin{minted}{sql}
-- sudo mysql  < Personalverwaltung.sql
DROP DATABASE IF EXISTS Personalverwaltung;
CREATE DATABASE Personalverwaltung;
USE Personalverwaltung;

CREATE TABLE Mitarbeiter(
MitarbeiterID Integer PRIMARY KEY,
Vorname VARCHAR(30),
Nachname VARCHAR(30),
Vorgesetzter Integer REFERENCES Mitarbeiter(MitarbeiterID),
AbteilungsID Integer REFERENCES Abteilung(AbteilungsID),
Telefonnummer Long,
Gehalt Double
);

CREATE TABLE Abteilung(
AbteilungsID Integer PRIMARY KEY,
Bezeichnung VARCHAR(30)
);

INSERT INTO Abteilung VALUES (1,"Buchhaltung");
INSERT INTO Abteilung VALUES (2,'Geschaeftemacher');
INSERT INTO Abteilung VALUES (3,"Buchhaltung");
INSERT INTO Abteilung VALUES (4, "Qualitaetskontrolle");
INSERT INTO Abteilung VALUES (5, "Qualitaetskontrolle");

INSERT INTO Mitarbeiter VALUES (1,"Hans", 'Meier', 11, 1, 2313432, 12345);
INSERT INTO Mitarbeiter VALUES (2,"Fred", 'Maier', 11, 2, 233413432, 1233);
INSERT INTO Mitarbeiter VALUES (11,"Hans", 'Mueller', NULL,3, 3432, 1230454);
INSERT INTO Mitarbeiter VALUES (3,"Hans", 'Fuchs', 2, 1, 2313344, 2345);
INSERT INTO Mitarbeiter VALUES (4,"Fred", 'Hase', 11, 2, 432, 12334);
INSERT INTO Mitarbeiter VALUES (12,"Gerd", 'Mueller', NULL,3, 345552, 154);
INSERT INTO Mitarbeiter VALUES (5, "Josef", "Huber", 12, 4, 786787, 3443);
INSERT INTO Mitarbeiter VALUES (6, "Juergen", "Schmidt", 12, 5, 97854, 6654);
\end{minted}

\end{document}
