\documentclass{lehramt-informatik-aufgabe}
\liLadePakete{syntax}
\begin{document}

\section{Aufgabe 4: Datenflussorientiertes Testen
\index{Datenflussorientiertes Testen}
\footcite{sosy:ab:7}}

Gegeben Sei folgende Java-Methode sort zum Sortieren eines Feldes ganzer
Zahlen:
\footcite[Thema 1 Teilaufgabe 2 Aufgabe 3]{examen:66116:2016:09}

\inputcode[firstline=4]{aufgaben/sosy/ab_7/Aufgabe4}

\begin{enumerate}

%%
% (a)
%%

\item Konstruieren Sie den
Kontrollflussgraphen\index{Datenfluss-annotierter Kontrollflussgraph}
des obigen Code-Fragments und annotieren Sie an den Knoten und Kanten
die zugehörigen Datenflussinformationen (Definitionen bzw. berechnende
oder prädikative Verwendung von Variablen).

%%
% (b)
%%

\item Nennen Sie die maximale Anzahl linear unabhängiger Programmpfade,
also die zyklomatische Komplexität nach McCabe.\index{Zyklomatische
Komplexität nach Mc-Cabe}

%%
% (c)
%%

\item Geben Sie einen möglichst kleinen Testdatensatz an, der eine
100\%-ige Verzweigungsüberdeckung dieses Moduls erzielt.\index{C1-Test
Zweigüberdeckung (Branch Coverage)}

%%
% (d)
%%

\item Beschreiben Sie kurz, welche Eigenschaften eine Testfallmenge
allgemein haben muss, damit das datenflussorientierte
Überdeckungskriterium „all-uses“\index{all uses} erfüllt.

\end{enumerate}
\end{document}
