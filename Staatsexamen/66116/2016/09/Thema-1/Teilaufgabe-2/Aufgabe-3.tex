\documentclass{lehramt-informatik-minimal}
\InformatikPakete{syntax}
\begin{document}

\section{Aufgabe 4: Datenflussorientiertes Testen\footcite{sosy:ab:7}}

Gegeben Sei folgende Java-Methode sort zum Sortieren eines Feldes ganzer
Zahlen:
\footcite[Herbst 2016 (66116) - Thema 1 Teilaufgabe 2, Aufgabe 3 a-e]{examen:66116:2016:09}

\inputcode[firstline=4]{aufgaben/sosy/ab_7/Aufgabe4}

\begin{enumerate}

%%
% (a)
%%

\item Konstruieren Sie den Kontrollflussgraphen des obigen
Code-Fragments und annotieren Sie an den Knoten und Kanten die
zugehörigen Datenflussinformationen (Definitionen bzw. berechnende
oder prädikative Verwendung von Variablen).

%%
% (b)
%%

\item Nennen Sie die maximale Anzahl linear unabhängiger Programmpfade,
also die zyklomatische Komplexität nach McCabe.

%%
% (c)
%%

\item Geben Sie einen möglichst kleinen Testdatensatz an, der eine
100\%-ige Verzweigungsüberdeckung dieses Moduls erzielt.

%%
% (d)
%%

\item Beschreiben Sie kurz, welche Eigenschaften eine Testfallmenge
allgemein haben muss, damit das datenflussorientierte
Überdeckungskriterium all-uses“ erfüllt.

\end{enumerate}
\end{document}
