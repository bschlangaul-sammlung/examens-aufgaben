\documentclass{lehramt-informatik-minimal}
\InformatikPakete{}
\begin{document}

\section{Aufgabe 2\footcite{sosy:ab:9}}

\begin{description}

%%
%
%%

\item[A] Allgemein\footcite[Thema 2 Teilaufgabe 2 Aufgabe 1]{examen:66116:2016:09}

\begin{description}
\item[Aufgabe 1] Im Software Engineering\index{Software Engineering}
geht es vor allem darum qualitativ hochwertige Software zu entwickeln.

\item[Aufgabe 2] Software Engineering ist gleichbedeutend mit
Programmieren.
\end{description}

%%
%
%%

\item[B] Vorgehensmodelle

\begin{description}
\item[B1] Die Erhebung und Analyse von Anforderungen sind nicht Teil des
Software Engineerings.

\item[B2] Agile Methoden\index{Agile Methoden} eignen sich besonders gut
für die Entwicklung komplexer und sicherer Systeme in verteilten
Entwicklerteams.

\item[B3] Das Spiralmodell\index{Spiralmodell} ist ein Vorläufer
sogenannter Agiler Methoden.
\end{description}

%%
%
%%

\item[C] Anforderungserhebung

\begin{description}
\item[C1] Bei der Anforderungserhebung dürfen in keinem Fall mehrere
Erhebungstechniken (z. B. Workshops, Modellierung) angewendet werden,
weil sonst Widersprüche in Anforderungen zu, Vorschein kommen könnten.

\item[C2] Ein Szenario beinhaltet eine Menge von Anwendungsfällen.

\item[C3] Nicht-funktionale Anforderungen\index{Nicht-funktionale
Anforderungen} sollten, wenn möglich, immer quantitativ spezifiziert
werden.
\end{description}

%%
%
%%

\item[D] Architekturmuster\index{Entwurfsmuster}

\begin{description}
\item[D1] Schichtenarchitekturen\index{Schichtenarchitektur} sind
besonders für Anwendungen geeignet, in denen Performance eine wichtige
Rolle spielt.

\item[D2] Das Black Board Muster\index{Blackboard-Muster} ist besonders
für Anwendungen geeignet, in denen Performance eine wichtige Rolle
spielt.

\item[D3] „Dependency Injection“\index{Einbringen von Abhängigkeiten
(Dependency Injection)} bezeichnet das Konzept, welches Abhängigkeiten
zur Laufzeit reglementiert.
\end{description}

%%
%
%%

\item[E] UML

\begin{description}
\item[E1] Sequenzdiagramme\index{Sequenzdiagramm} beschreiben Teile des
Verhaltens eines Systems.

\item[E2] Zustandsübergangsdiagramme\index{Zustandsdiagramm} beschreiben
das Verhalten eines Systems.

\item[E3] Komponentendiagramme\index{Komponentendiagramm} beschreiben
die Struktur eines Systems.
\end{description}

%%
%
%%

\item[F] Entwurfsmuster

\begin{description}
\item[F1] Das MVC Pattern\index{Modell-Präsentation-Steuerung
(Model-View-Controller)} verursacht eine starke Abhängigkeit zwischen
Datenmodell und Benutzeroberfläche.

\item[F2] Das Singleton Pattern\index{Einzelstück (Singleton)} stellt
sicher, dass es zur Laufzeit von einer bestimmten Klasse höchstens ein
Objekt gibt.

\item[F3] Im Kommando Enwurfsmuster (engl. „Command Pattern“)
\index{Kommando (Command)} werden Befehle in einem sog.
Kommando-Objekt gekapselt, um sie bei Bedarf rückgängig zu machen.
\end{description}

%%
%
%%

\item[G] Testen

\begin{description}
\item[G1] Validation\index{Validation} dient der Überprüfung von
Laufzeitfehlern.

\item[G2] Testen ermöglicht sicherzustellen, dass ein Programm absolut
fehlerfrei ist.

\item[G3] Verifikation\index{Verifikation} dient der Überprüfung, ob ein
System einer
Spezifikation entspricht.
\end{description}

\end{description}
\end{document}
