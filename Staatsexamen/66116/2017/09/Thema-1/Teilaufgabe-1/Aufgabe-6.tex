\documentclass{lehramt-informatik-aufgabe}
\liLadePakete{}
\begin{document}
\liAufgabenTitel{Fluginformationssystem}
\section{6. Relationale Anfragen in SQL
\index{SQL}
\footcite{66116:2017:09}}

Folgende Tabellen veranschaulichen eine Ausprägung eines Fluginformationssystems:

\liPseudoUeberschrift{Flughäfen}

\begin{tabular}{lll}
Code &
Stadt &
Transferzeit (min)\\

LHR &
London &
30\\

LGW &
London &
20\\

JFK &
New York City &
60\\

EWR &
New York City &
35\\

MUC &
München &
30\\

FRA &
Frankfurt &
45\\

\end{tabular}

\liPseudoUeberschrift{Verbindungen}

\begin{tabular}{llllll}
ID &
Von &
Nach &
Linie &
Abflug (MEZ) &
Ankunft (MEZ)\\

410 &
MUC &
FRA &
LH &
2016-02-24 07:00:00 &
2016-02-24 08:10:00\\

411 &
MUC &
FRA &
LH &
2016-02-24 08:00:00 &
2016-02-24 09:10:00\\

412 &
FRA &
JFK &
LH &
2016-02-24 10:50:00 &
2016-02-24 19:50:00\\
\end{tabular}

\liPseudoUeberschrift{Hinweise}

\begin{itemize}
\item Formulieren Sie alle Abfragen in SQL-92 (insbesondere sind
LIMIT, TOP, FETCH FIRST, ROWNUM und dergleichen nicht erlaubt).

\item Alle Datum/Zeit-Angaben erlauben arithmetische Operationen,
beispielsweise wird bei der Operation \texttt{ankunf} +
\texttt{transferzeit} die \texttt{transferzeit} auf den Zeitstempel
\texttt{ankunft} addiert.

\item Es müssen keine Zeitverschiebungen berücksichtigt werden. Alle
Zeitstempel sind in MEZ.

\end{itemize}

\begin{enumerate}

%%
% 1.
%%

\item Ermitteln Sie die Städte, in denen es mehr als einen Flughafen
gibt.

%%
% 2.
%%

\item Ermitteln Sie die Städte,in denen man mit der Linie „LH" an
mind.zwei verschiedenen Flughäfen landen kann.

%%
% 3.
%%

\item Ermitteln Sie die Flugzeit des kürzesten Direktflugs von München
nach London.

%%
% 4.
%%

\item Ermitteln Sie die kürzeste Roundtrip-Zeit(nur Direktflüge)zwischen
den Flughäfen FRA und JFK (Transferzeit am Flughafen JFK beachten).
\end{enumerate}
\end{document}
