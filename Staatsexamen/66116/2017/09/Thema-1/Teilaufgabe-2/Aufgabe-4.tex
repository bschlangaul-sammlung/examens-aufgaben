\documentclass{lehramt-informatik-minimal}
\InformatikPakete{syntax}
\usepackage{tabularx}
\begin{document}

\section{Aufgabe 3\footcite[Seite 3]{sosy:ab:9}}

Die folgende Seite enthält Software-Quellcode, der einen Algorithmus zur
binären Suche\index{Binäre Suche} implementiert. Dieser ist durch
Inspektion zu überprüfen. Im Folgenden sind die Regeln der Inspektion
angegeben.
\footcite[Thema 1 Teilaufgabe 2 Aufgabe 4]{examen:66116:2017:09}

\bigskip

\noindent
\begin{tabularx}{\linewidth}{|l|l|X|}
\hline
RM1 &
(Dokumentation) &
Jede Quellcode-Datei beginnt mit einem Kommentar, der
den Klassennamen, Versionsinformationen, Datum und
Urheberrechtsangaben enthält.
\\\hline

RM2 &
(Dokumentation) &
Jede Methode wird kommentiert. Der Kommentar enthält
eine vollständige Beschreibung der Signatur so wie eine
Design-by-Contract-Spezifikation.
\\\hline

RM3 &
(Dokumentation) &
Deklarationen von Variablen werden kommentiert.
\\\hline

RM4 &
(Dokumentation) &
Jede Kontrollstruktur wird kommentiert.
\\\hline

RM5 &
(Formatierung) &
Zwischen einem Schlüsselwort und einer Klammer steht
ein Leerzeichen.
\\\hline

RM6 &
(Formatierung) &
Zwischen binären Operatoren und den Operanden stehen
Leerzeichen.
\\\hline

RM7 &
(Programmierung) &
Variablen werden in der Anweisung initialisiert, in der sie
auch deklariert werden.
\\\hline

RM8 &
(Bezeichner) &
Klassennamen werden groß geschrieben, Variablennamen klein.
\\\hline
\end{tabularx}

%-----------------------------------------------------------------------
%
%-----------------------------------------------------------------------

\bigskip

\begin{enumerate}

%%
%
%%

\item Überprüfen Sie durch Inspektion, ob die obigen Regeln für den
Quellcode eingehalten wurden. Erstellen Sie eine Liste mit allen
Verletzungen der Regeln. Geben Sie für jede Verletzung einer Regel die
Zeilennummer, Regelnummer und Kommentar an, z.\,B.(07, RM4, while nicht
kommentiert).[...]

%%
%
%%

\item Entspricht die Methode \java{binarySearch} ihrer Spezifikation,
die durch Vor-und Nachbedingungen angeben ist? Geben Sie gegebenenfalls
Korrekturen der Methode an.

%%
%
%%

\item Beschreiben alle Kommentare ab Zeile 24 die Semantik des Codes
korrekt? Geben Sie zu jedem falschen Kommentar einen korrigierten
Kommentar mit Zeilennummer an.

%%
%
%%

\item Geben Sie den Kontrollflussgraphen\index{Kontrollflussgraph} für
die Methode \java{binarySearch} an.

%%
%
%%

\item Geben Sie maximal drei Testfälle für die Methode
\java{binarySearch} an, die insgesamt eine vollständige
Anweisungsüberdeckung\index{Vollständige Anweisungsüberdeckung} leisten.

\inputcode[firstline=3]{aufgaben/sosy/ab_9/BinarySearch}

\end{enumerate}
\end{document}
