\documentclass{bschlangaul-aufgabe}
\liLadePakete{baum}
\begin{document}
\liAufgabenMetadaten{
  Titel = {Aufgabe 2},
  Thematik = {Aufbau eines B-Baums},
  RelativerPfad = Staatsexamen/66116/2017/03/Thema-1/Teilaufgabe-1/Aufgabe-2.tex,
  ZitatSchluessel = examen:66116:2017:03,
  BearbeitungsStand = mit Lösung,
  Korrektheit = korrekt und überprüft,
  Stichwoerter = {B-Baum},
  ExamenNummer = 66116,
  ExamenJahr = 2017,
  ExamenMonat = 03,
  ExamenThemaNr = 1,
  ExamenTeilaufgabeNr = 1,
  ExamenAufgabeNr = 2,
}

Konstruieren\index{B-Baum}
\footcite{examen:66116:2017:03} Sie einen B-Baum, dessen Knoten maximal $4$ Einträge
enthalten können, indem Sie der Reihe nach diese Suchschlüsssel
einfügen:\footcite[Aufgabe 3]{aud:ab:5}

\begin{center}
8, 10, 12, 20, 5, 30, 25, 11
\end{center}

\noindent
Anschließend löschen Sie den Eintrag mit dem Suchschlüssel $8$.

Zeigen Sie jeweils graphisch den entstehenden Baum nach relevanten
Zwischenschritten; insbesondere nach Einfügen der $5$ sowie nach dem
Einfügen der $11$ und nach dem Löschen der $8$.

\begin{liAntwort}

%%
%
%%

\begin{compactitem}
\item Schlüsselwert 8 (einfaches Einfügen)
\item Schlüsselwert 10 (einfaches Einfügen)
\item Schlüsselwert 12 (einfaches Einfügen)
\item Schlüsselwert 20 (einfaches Einfügen)
\end{compactitem}

\begin{center}
\begin{tikzpicture}[
  li bbaum,
  level 1/.style={level distance=10mm,sibling distance=32mm},
]
\node {8 \nodepart{two} 10 \nodepart{three} 12 \nodepart{four} 20};
\end{tikzpicture}
\end{center}

%%
%
%%

\begin{compactitem}
\item Schlüsselwert 5 (Split)
\end{compactitem}

\begin{center}
\begin{tikzpicture}[
  li bbaum,
  level 1/.style={level distance=15mm,sibling distance=20mm},
]
\node {10} [->]
  child {node {5 \nodepart{two} 8}}
  child {node {12 \nodepart{two} 20}}
;
\end{tikzpicture}
\end{center}

%%
%
%%

\begin{compactitem}
\item Schlüsselwert 30 (einfaches Einfügen)
\item Schlüsselwert 25 (einfaches Einfügen)
\item Schlüsselwert 11 (Split)
\end{compactitem}

\begin{center}
\begin{tikzpicture}[
  li bbaum,
  level 1/.style={level distance=15mm,sibling distance=20mm},
]
\node {10 \nodepart{two} 20} [->]
  child {node {5 \nodepart{two} 8}}
  child {node {11 \nodepart{two} 12}}
  child {node {25 \nodepart{two} 30}}
;
\end{tikzpicture}
\end{center}

%%
%
%%

\begin{compactitem}
\item Löschen des Schlüsselwerts 8 (Mischen/Verschmelzen)
\end{compactitem}

\begin{center}
\begin{tikzpicture}[
  li bbaum,
  level 1/.style={level distance=15mm,sibling distance=30mm},
]
\node {20} [->]
  child {node {5 \nodepart{two} 10 \nodepart{three} 11 \nodepart{four} 12}}
  child {node {25 \nodepart{two} 30}}
;
\end{tikzpicture}
\end{center}
\end{liAntwort}

\end{document}
