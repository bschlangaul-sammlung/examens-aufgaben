\documentclass{lehramt-informatik-aufgabe}
\liLadePakete{syntax,mathe,kontrollflussgraph}
\begin{document}
\liAufgabenTitel{Methode „binToInt()“ und Kontrollflussgraph}

\section{Aufgabe 5 (Check-Up)\footcite{sosy:ab:7}}

Gegeben Sei folgende Methode und ihr Kontrollflussgraph:
\footcite[Thema 2 Teilaufgabe 2 Aufgabe 1]{examen:66116:2017:03}

\begin{minipage}{6cm}
\liJavaDatei[firstline=4,lastline=21]{aufgaben/sosy/ab_7/Aufgabe5}
%
\end{minipage}
%
\begin{minipage}{4cm}
\begin{tikzpicture}[kontrollfluss,xscale=1,yscale=-1.2]
\node[knoten] at (0,0) (nS) {nS};
\node[knoten] at (0,1) (n1) {n1};
\node[knoten] at (0,2) (n2) {n2};
\node[knoten] at (0,3) (n3) {n3};
\node[knoten] at (-1.5,3) (n4) {n4};
\node[knoten] at (0,4) (n5) {n5};
\node[knoten] at (-1,5) (n6) {n6};
\node[knoten] at (1,5) (n7) {n7};
\node[knoten] at (0,6) (n8) {n8};
\node[knoten] at (-2,6) (nE) {nE};

\draw[->] (nS) -- (n1);
\draw[->,dashed] (n1) -- (n2) \KanteFalsch{right};
\draw[->] (n2) -- (n3);
\draw[->,dashed] (n3) -- (n4) \KanteFalsch{below};
\draw[->] (n3) -- (n5) \KanteWahr{right};
\draw[->,dashed] (n5) -- (n6) \KanteFalsch{right};
\draw[->] (n5) -- (n7) \KanteWahr{right};
\draw[->] (n6) -- (n8) \KanteWahr{right};
\draw[->] (n7) -- (n8);
\draw[->,dashed] (n6) -- (nE) \KanteFalsch{right};
\draw[->] (n4) -- (nE);
\draw[->] (n1) -- (-2,1) -- (nE) \KanteWahr{left};
\draw[->] (n8) -- (2,6) -- (2,3) -- (n3);
\end{tikzpicture}
\end{minipage}

\begin{enumerate}

%%
% (a)
%%

\item Geben Sie je einen Repräsentanten aller Pfadklassen im
Kontrollflussgraphen an, die zum Erzielen einer vollständigen

\begin{liAntwort}
\begin{tikzpicture}[kontrollfluss,xscale=1,yscale=-1.2]
\node[knoten] at (0,0) (nS) {nS};

\node[knoten,pin={
  \liJavaCode{if (bin.isEmpty())}
}] at (0,1) (n1) {n1};

\node[knoten,pin={
  \liJavaCode{int place; int length}
}] at (0,2) (n2) {n2};

\node[knoten,pin={
  [pin distance=2cm]
  for-Bedingung \liJavaCode{i >= 0}}
] at (0,3) (n3) {n3};

\node[knoten,pin={
  180:\liJavaCode{return value;}
}] at (-1.5,3) (n4) {n4};

\node[knoten,pin={
  \liJavaCode{if (ch = '1')}
}] at (0,4) (n5) {n5};

\node[knoten,pin={
  180:\liJavaCode{else if (ch = '0')}
}] at (-1,5) (n6) {n6};

\node[knoten,pin={
  \liJavaCode{value += place;}
}] at (1,5) (n7) {n7};

\node[knoten,pin={
  270:\liJavaCode{place *= 2; i--;}}
] at (0,6) (n8) {n8};

\node[knoten] at (-2,6) (nE) {nE};

\draw[->] (nS) -- (n1);
\draw[->,dashed] (n1) -- (n2) \KanteFalsch{right};
\draw[->] (n2) -- (n3);
\draw[->,dashed] (n3) -- (n4) \KanteFalsch{below};
\draw[->] (n3) -- (n5) \KanteWahr{right};
\draw[->,dashed] (n5) -- (n6) \KanteFalsch{right};
\draw[->] (n5) -- (n7) \KanteWahr{right};
\draw[->] (n6) -- (n8) \KanteWahr{right};
\draw[->] (n7) -- (n8);
\draw[->,dashed] (n6) -- (nE) \KanteFalsch{right};
\draw[->] (n4) -- (nE);
\draw[->] (n1) -- (-2,1) -- (nE) \KanteWahr{left};
\draw[->] (n8) -- (2,6) -- (2,3) -- (n3);
\end{tikzpicture}
\end{liAntwort}

\def\r#1{\textcolor{red}{#1}}
\def\g#1{\textcolor{green}{#1}}

\begin{enumerate}

%%
% (i)
%%

\item Verzweigungsüberdeckung

\begin{liAntwort}
\begin{description}
\item[p1 (Pfad 1)] S 1 E
\item[p2] S 1 2 3 4 E
\item[p3] S 1 2 3 5 \r{7} 8 3 5 \g{6} 8 3 5 \g{6} E
\end{description}
\end{liAntwort}

%%
% (ii)
%%

\item Schleife-Inneres-Überdeckung

\begin{liAntwort}
\begin{description}
\item[Äußere Pfade (äußere Pfade):]

(ohne Ausführung der Wiederholung)

\begin{description}
\item[p1] S 1 E
\item[p2] S 1 2 3 4 E
\end{description}

\item[Grenzpfade (boundary test)]

(alle Pfade, die die Wiederholung betreten, aber nicht wiederholen;
innerhalb des Schleifenrumpfes alle Pfade!)

\begin{description}
\item[p4] S 1 2 3 5 \g{6} E
\end{description}

\item[Innere Pfade (interior test)]

(alle Pfade mit \emph{einer Wiederholung des Schleifenrumpfes};
innerhalb des Schleifenrumpfes wieder alle Pfade!)

\begin{description}
\item[p5] S 1 2 3 5 \r{7} 8 3 5 \r{7} 8 3 4 E
\item[p6] S 1 2 3 5 \r{7} 8 3 5 \g{6} 8 3 4 E
\item[p7] S 1 2 3 5 \g{6} 8 3 5 \g{6} 8 3 4 E
\item[p8] S 1 2 3 5 \g{6} 8 3 5 \r{7} 8 3 4 E
\item[p9] S 1 2 3 5 \r{7} 8 3 5 \r{7} 8 3 5 \g{6} E
\item[p10 = p3] S 1 2 3 5 \r{7} 8 3 5 \g{6} 8 3 5 \g{6} E
\item[p11] S 1 2 3 5 \g{6} 8 3 5 \g{6} 8 3 5 \g{6} E
\item[p12] S 1 2 3 5 \g{6} 8 3 5 \r{7} 8 3 5 \g{6} E
\end{description}
\end{description}
\end{liAntwort}
\end{enumerate}

mit minimaler Testfallanzahl genügen würden.

%%
% (b)
%%

\item Welche der vorangehend ermittelten Pfade sind mittels Testfälle
tatsächlich überdeckbar („feasible“)? Falls der Pfad ausführbar ist,
geben Sie bitte den Testfall an, andernfalls begründen Sie kurz,
weshalb der Pfad nicht überdeckbar ist.

\begin{liAntwort}
Erweitere Methode, die die Knotennamen ausgibt:

\liJavaDatei[firstline=23]{aufgaben/sosy/ab_7/Aufgabe5}

Alle mit Ausnahme von p2.

\textbf{p2} ist nicht überdeckbar. Passiert ein Wert der Variable
\liJavaCode{bin} die erste if-Verzweigung, dann hat der Wert eine Länge größer
0 und betritt deshalb die Wiederholung mit fester Anzahl.

\begin{tabular}{lll}
p1 &
S 1 E &
\liJavaCode{binToInt("");} \\

p2 &
S 1 2 3 4 E &
not feasible \\

p3 &
S 1 2 3 5 \r{7} 8 3 5 \g{6} 8 3 5 \g{6} E &
\liJavaCode{binToInt("x01");} \\

p4 &
S 1 2 3 5 \g{6} E  &
\liJavaCode{binToInt("x");} \\

p5 &
S 1 2 3 5 \r{7} 8 3 5 \r{7} 8 3 4 E &
\liJavaCode{binToInt("11");} \\

p6 &
S 1 2 3 5 \r{7} 8 3 5 \g{6} 8 3 4 E &
\liJavaCode{binToInt("01");} \\

p7 &
S 1 2 3 5 \g{6} 8 3 5 \g{6} 8 3 4 E &
\liJavaCode{binToInt("00");} \\

p8 &
S 1 2 3 5 \g{6} 8 3 5 \r{7} 8 3 4 E &
\liJavaCode{binToInt("10");} \\

p9 &
S 1 2 3 5 \r{7} 8 3 5 \r{7} 8 3 5 \g{6} E &
\liJavaCode{binToInt("x11");} \\

p10 = p3 &
S 1 2 3 5 \r{7} 8 3 5 \g{6} 8 3 5 \g{6} E &
\liJavaCode{binToInt("x01");} \\

p11 &
S 1 2 3 5 \g{6} 8 3 5 \g{6} 8 3 5 \g{6} E &
\liJavaCode{binToInt("x00");} \\

p12 &
S 1 2 3 5 \g{6} 8 3 5 \r{7} 8 3 5 \g{6} E &
\liJavaCode{binToInt("x10");} \\
\end{tabular}
\end{liAntwort}

%%
% (c)
%%

\item Bestimmen Sie anhand des Kontrollflussgraphen die maximale Anzahl
linear unabhängiger Programmpfade, also die zyklomatische Komplexität
nach Mc-Cabe.\index{Zyklomatische Komplexität nach Mc-Cabe}

\begin{liAntwort}
\begin{description}
\item[Binärverzweigungen] 4
\item[Knoten] 10
\item[Kanten] 13
\end{description}

\noindent
Anhand der Binärverzweigungen:

\begin{align*}
M
&= b + p \\
&= 4 + 1 \\
&= 5
\end{align*}

\noindent
oder durch Anzahl Kanten $e$ und Knoten $n$

\begin{align*}
M
&= e - n + 2p \\
&= 13 - 10 + 2 \cdot 1 \\
&= 5
\end{align*}
\end{liAntwort}

%%
% (d)
%%

\item Kann für dieses Modul eine 100\%-ige Pfadüberdeckung erzielt
werden? Begründen Sie kurz Ihre Antwort.

\begin{liAntwort}
Nein, da \textbf{p2} nicht überdeckbar ist.
\end{liAntwort}

%%
% (e)
%%

\item Geben Sie zu jedem Knoten die jeweilige
Datenfluss-Annotation\index{Datenfluss-annotierter Kontrollflussgraph}
(\texttt{defs} bzw. \texttt{uses}) für jede betroffene Variable in der
zeitlichen Reihenfolge ihres Auftretens zur Laufzeit an.

\begin{liAntwort}
\begin{tikzpicture}[kontrollfluss,xscale=1,yscale=-1.2]
\node[knoten] at (0,0) (nS) {nS};
\node[knoten] at (0,1) (n1) {n1};
\node[knoten] at (0,2) (n2) {n2};
\node[knoten] at (0,3) (n3) {n3};
\node[knoten] at (-1.5,3) (n4) {n4};
\node[knoten] at (0,4) (n5) {n5};
\node[knoten] at (-1,5) (n6) {n6};
\node[knoten] at (1,5) (n7) {n7};
\node[knoten] at (0,6) (n8) {n8};
\node[knoten] at (-2,6) (nE) {nE};

\draw[->] (nS) -- (n1);
\draw[->,dashed] (n1) -- node[name=n1n2]{} (n2);
\draw[->] (n2) -- (n3);
\draw[->,dashed] (n3) -- node[name=n3n4]{} (n4);
\draw[->] (n3) -- node[name=n3n5]{} (n5);
\draw[->,dashed] (n5) -- node[name=n5n6]{} (n6);
\draw[->] (n5) -- node[name=n5n7]{} (n7);
\draw[->] (n6) -- node[name=n6n8]{} (n8);
\draw[->] (n7) -- (n8);
\draw[->,dashed] (n6) -- node[name=n6nE]{} (nE);
\draw[->] (n4) -- (nE);
\draw[->] (n1) -- node[name=n1nE]{} (-2,1) -- (nE);
\draw[->] (n8) -- (1.5,6) -- (1.5,3) -- (n3);

\node[usebox] at (2,0) {def(bin)} edge[dashed] (nS);
\node[usebox] at (2,2) {def(place), def(value)\\def(length), c-use(bin)} edge[dashed] (n2);
\node[usebox] at (2,3) {def(i), c-use(length)} edge[dashed] (n3);
\node[usebox,anchor=east] at (-2.5,3) {c-use(value)} edge[dashed] (n4);
\node[usebox] at (2,4) {def(ch), c-use(bin)} edge[dashed] (n5);
\node[usebox] at (2,6) {def(value), c-use(place)} edge[dashed] (n7);
\node[usebox,anchor=north] at (0,7) {def(place), c-use(place)} edge[dashed] (n8);
\node[usebox,anchor=north] at (-3.5,7) {c-use(value)} edge[dashed] (nE);

\node[usebox] at (-3,0) {p-use(bin)} edge[dashed] (n1n2.center) edge[dashed] (n1nE.center);
\node[usebox,anchor=west] at (-4,1) {p-use(length)} edge[dashed] (n3n4.center) edge[dashed] (n3n5.center);
\node[usebox,anchor=west] at (2,4.8) {p-use(ch)} edge[dashed] (n5n6.center) edge[dashed] (n5n7.center);
\node[usebox] at (-4.5,4.5) {p-use(ch)} edge[dashed] (n6n8.center) edge[dashed] (n6nE.center);

\end{tikzpicture}
\end{liAntwort}

\end{enumerate}
\end{document}
