\documentclass{lehramt-informatik-aufgabe}
\liLadePakete{mathe,syntax,wpkalkuel}
\begin{document}
\liAufgabenTitel{wp-Kalkül mit Invariante bei Methode „mul()“}

\section{Aufgabe 5 (Check-Up)
\index{wp-Kalkül}
}

Sie dürfen im Folgenden davon ausgehen, dass keinerlei Under- oder
Overflows auftreten.
\footcite[Thema 2 Teilaufgabe 2 Aufgabe 4]{examen:66116:2017:03}

Gegeben sei die folgende Methode mit Vorbedingung $P := x \geq 0 \land y
\geq 0$ und Nachbedingung $Q := x \cdot y = z$.

\begin{minted}{java}
int mul (int x , int y) {
  /* P */
  int z = 0, i = 0;
  while (i++ != x)
    z += y;
  /* Q */
  return z;
}
\end{minted}

Betrachten Sie dazu die folgenden drei Prädikate\index{Invariante}:

\begin{itemize}
\item $I_1 := z + i \cdot y = x \cdot y$
\item $I_2 := \text{false}$
\item $I_3 := z + (x - i) \cdot y = x \cdot y$
\end{itemize}

\begin{enumerate}

%%

% (a)
%%

\item Beweisen Sie formal für jedes der drei Prädikate, ob es
unmittelbar vor Betreten der Schleife in \java{mul} gilt oder nicht.

\begin{antwort}
\begin{align*}
\wp{Code vor der Schleife}{I_1}
& \equiv \wp{int z = 0, i = 0;}{z + i \cdot y = x \cdot y} \\
& \equiv \wp{}{0 + 0 \cdot y = x \cdot y} \\
& \equiv 0 = x \cdot y \\
& \equiv \text{falsch} \\
\end{align*}

\begin{align*}
\wp{Code vor der Schleife}{I_2}
& \equiv \wp{int z = 0, i = 0;}{\text{false}} \\
& \equiv \wp{}{\text{false}} \\
& \equiv \text{false} \\
& \equiv \text{falsch} \\
\end{align*}

\begin{align*}
\wp{Code vor der Schleife}{I_3}
& \equiv \wp{int z = 0, i = 0;}{z + (x - i) \cdot y = x \cdot y} \\
& \equiv \wp{}{0 + (x - 0) \cdot y = x \cdot y} \\
& \equiv x \cdot y = x \cdot y \\
& \equiv \text{wahr} \\
\end{align*}
\end{antwort}

%%
% (b)
%%

\item Weisen Sie formal nach, welche der drei Prädikate Invarianten des
Schleifenrumpfs in \java{mul} sind oder welche nicht.

\begin{antwort}
Für den Nachweis muss der Code etwas umformuliert werden:

\begin{minted}{java}
int mul (int x , int y) {
  /* P */
  int z = 0, i = 0;
  while (i != x) {
    i = i + 1;
    z = z + y;
  }
  /* Q */
  return z;
}
\end{minted}

\begin{align*}
\wp{Code Schleife}{I_1 \land i \neq x}
& \equiv \wp{i = i + 1; z = z + y;}{z + i \cdot y = x \cdot y \land i \neq x} \\
& \equiv \wp{i = i + 1;}{z + y + i \cdot y = x \cdot y \land i \neq x} \\
& \equiv \wp{}{z + y + (i + 1) \cdot y = x \cdot y \land i + 1 \neq x} \\
& \equiv z + y + (i + 1) \cdot y = x \cdot y \land i + 1 \neq x \\
& \equiv z + i \cdot y + 2 \cdot y = x \cdot y \land i + 1 \neq x \\
& \equiv \text{falsch} \land i + 1 \neq x \\
& \equiv \text{falsch} \\
\end{align*}

\begin{align*}
\wp{Code Schleife}{I_2 \land i \neq x}
& \equiv \wp{i = i + 1; z = z + y;}{\text{false} \land i \neq x} \\
& \equiv \wp{}{\text{false} \land i \neq x} \\
& \equiv \text{falsch} \land i \neq x \\
& \equiv \text{falsch} \\
\end{align*}

\begin{align*}
\wp{Code Schleife}{I_3 \land i \neq x}
& \equiv \wp{i = i + 1; z = z + y;}{z + (x - i) \cdot y = x \cdot y \land i \neq x} \\
& \equiv \wp{i = i + 1;}{z + y + (x - i) \cdot y = x \cdot y \land i \neq x} \\
& \equiv \wp{}{z + y + (x - i + 1) \cdot y = x \cdot y \land i + 1 \neq x} \\
& \equiv z + y + x \cdot y - i \cdot y + y = x \cdot y \land i + 1 \neq x \\
& \equiv z + 2 \cdot y + x \cdot y - i \cdot y = x \cdot y \land i + 1 \neq x \\
& \equiv \text{wahr} \\
\end{align*}
\end{antwort}

%%
% (c)
%%

\item Beweisen Sie formal, aus welchen der drei Prädikate die
Nachbedingung gefolgert werden darf bzw. nicht gefolgert werden kann.

\begin{antwort}

$I_1 := z + i \cdot y = x \cdot y$
$I_2 := \text{false}$
$I_3 := z + (x - i) \cdot y = x \cdot y$

\begin{align*}
\wp{Code nach Schleife}{I_1 \land i = x}
& \equiv \wp{}{z + i \cdot y = x \cdot y \land i = x} \\
& \equiv z + i \cdot y = x \cdot y \land i = x \\
& \equiv z + x \cdot y = x \cdot y\\
& \neq Q
\end{align*}

\begin{align*}
\wp{Code nach Schleife}{I_2 \land i = x}
& \equiv \wp{}{\text{false} \land i = x} \\
& \equiv \text{false} \land i = x \\
& \equiv \text{falsch} \\
& \neq Q
\end{align*}

\begin{align*}
\wp{Code nach Schleife}{I_3 \land i = x}
& \equiv \wp{}{z + (x - i) \cdot y = x \cdot y \land i = x} \\
& \equiv z + (x - i) \cdot y = x \cdot y \land i = x \\
& \equiv z + (x - x) \cdot y = x \cdot y\\
& \equiv z + 0 \cdot y = x \cdot y\\
& \equiv z + 0 = x \cdot y \\
& \equiv z = x \cdot y \\
& \equiv Q \\
\end{align*}

\end{antwort}

%%
% (d)
%%

\item Skizzieren Sie den Beweis der totalen Korrektheit\index{Totale
Korrektheit} der Methode \java{mul}. Zeigen Sie dazu auch die
Terminierung der Methode.

\begin{antwort}
Aus den Teilaufgaben folgt der Beweis der partiellen Korrektheit mit
Hilfe der Invariante $i_3$. $i$ steigt streng monoton von $0$ an so
lange gilt $i \neq x$. $i = x$ ist die Abbruchbedingung für die bedingte
Wiederholung. Dann terminiert die Methode. Die Methode \java{mul} ist
also total korrekt.
\end{antwort}

\end{enumerate}
\end{document}
