\documentclass{lehramt-informatik-aufgabe}
\liLadePakete{normalformen,synthese-algorithmus}
\begin{document}
\let\ah=\liAttributHuelle
\let\FA=\liFunktionaleAbhaengigkeiten
\let\m=\liAttributMenge
\let\schrittE=\liSyntheseSchrittUeberschriftErklaerung

\liAufgabenTitel{Entwurfstheorie}
\section{Aufgabe 5
\index{Synthese-Algorithmus}
\footcite{66116:2017:03}}

In der folgenden Datenbank sind die Ausleihvorgänge einer Bibliothek gespeichert:

| Ausleihe | LNr | Name | Adresse | BNr | Titel | Kategorie | ExemplarNr |
1 | Müller | Winklerstr. 1 Datenbanksysteme | Informatik 1
1 | Miller | Winklerstr. 1 | Datenbanksysteme | Informatik 2
2 | Huber | Friedrichstr. | 2 Anatomie I Medizin 5
2 Huber | Friedrichstr. 3 Harry Potter Literatur 20
3 Meier | Bismarkstr. 4 OODBS Informatik 1
4 Meier Marktpl. 5 | Pippi Langstrumpf | Literatur 1

Für die Datenbank gilt:

Jeder Leser hat eine eindeutige Lesernummer (LNr), einen Namen und eine
Adresse. Ein Buch hat eine Buchnummer (BNr), einen Titel und eine
Kategorie. Es kann mehrere Exemplare eines Buches geben, welche durch
eine, innerhalb einer Buchnummer eindeutigen, Exemplarnummer
unterschieden werden.

\begin{enumerate}
\item Beschreiben Sie kurz, welche Redundanzen in der Datenbank vorhanden sind
und welche Anomalien auftreten können.

\item Nachfolgend sind alle nicht-trivialen funktionalen Abhängigkeiten,
welche in der obigen Datenbank gelten, angegeben:

\FA{
  LNr -> Name;
  LNr -> Adresse;
  BNr -> Titel;
  BNr -> Kategorie;
  LNr, BNr, ExemplarNr -> Name, Adresse, Titel, Kategorie
}

% http://www.ict.griffith.edu.au/normalization_tools/normalization/ind.php#resources

% LNr,BNr,ExemplarNr,Name,Adresse,Titel,Kategorie

% LNr -> Name
% LNr -> Adresse
% BNr -> Titel
% BNr -> Kategorie
% LNr,BNr,ExemplarNr -> Name,Titel,Adresse,Kategorie

Einziger Schlüsselkandidat ist \m{LNr, BNr, ExemplarNr}. Überführen Sie
das Schema mit Hilfe des Synthesealgorithmus für 3NF in die dritte
Normalform.

\begin{liAntwort}
\begin{enumerate}
\item \schrittE{1}

\begin{enumerate}
\item \schrittE{1-1}

$\ah{F, \m{\m{LNr, BNr, ExemplarNr} - \m{LNr}}} = \m{Titel, Kategorie}$

$\ah{F, \m{\m{LNr, BNr, ExemplarNr} - \m{BNr}}} = \m{Name, Adresse}$

$\ah{F, \m{\m{LNr, BNr, ExemplarNr} - \m{ExemplarNr}}} = \m{Name, Adresse, Titel, Kategorie}$

\FA{
  LNr -> Name;
  LNr -> Adresse;
  BNr -> Titel;
  BNr -> Kategorie;
  LNr, BNr -> Name, Adresse, Titel, Kategorie
}

\item \schrittE{1-2}
\item \schrittE{1-3}
\item \schrittE{1-4}
\end{enumerate}

\item \schrittE{2}
\item \schrittE{3}
\item \schrittE{4}
\end{enumerate}
\end{liAntwort}

\end{enumerate}
\end{document}
