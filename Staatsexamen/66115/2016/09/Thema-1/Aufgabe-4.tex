\documentclass{lehramt-informatik-aufgabe}
\liLadePakete{}
\begin{document}
\liAufgabenTitel{}
\section{Teilaufgabe IV
\index{Teile-und-Herrsche (Divide-and-Conquer)}
\footcite{66115:2016:09}}

Es sei A[0.. n - 1] ein Array von paarweise verschiedenen ganzen Zahlen.

Wir interessieren uns für die Zahl der Inversionen von A; das sind Paare von Indices (i, j), sodass i <j
aber A[i] > A[j]. Die Inversionen im Array [2, 3, 8, 6, 1] sind (0, 4), da A[0] > A[4] und weiter (1, 4),
(2, 3), (2, 4), (3, 4). Es gibt also 5 Inversionen.
\begin{enumerate}

%%
% 1.
%%

\item Wie viel Inversionen hat das Array [3, 7, 1, 4, 5, 9, 2]?

%%
% 2.
%%

\item Welches Array mit den Einträgen {l, ..., 2} hat die meisten Inversionen, welches hat die
wenigsten?

%%
% 3.
%%

\item Entwerfen Sie eine Prozedur
int merge(int[]a, int i, int h, int j);

welche das Teilarray a[i.,j] sortiert und die Zahl der in ihm enthaltenen Inversionen zurückliefert,
wobei die folgenden Vorbedingungen angenommen werden:

- OQ<i<h<j <n, wobei n die Länge von a ist (n = a.length).

- ali.h] und a[h + 1..,j] sind aufsteigend sortiert.

- Die Einträge von ali..j] sind paarweise verschieden.
Ihre Prozedur soll in linearer Zeit, also O(j - i) laufen.
Orientieren Sie sich bei Ihrer Lösung an der Mischoperation des bekannten Mergesort-Verfahrens.

%%
% 4.
%%

\item Entwerfen Sie nun ein Divide-and-Conquer-Verfahren zur Bestimmung der Zahl der Inversionen,
indem Sie angelehnt an das Mergesort-Verfahren einen Algorithmus ZI beschreiben, der ein
gegebenes Array in sortierter Form liefert und gleichzeitig dessen Inversionsanzahl berechnet.

Im Beispiel wäre also

ZI([2, 3, 8, 6, 1) = C1, 2, 3, 6, 8], 5)

Die Laufzeit Ihres Algorithmus auf einem Array der Größe n soll O(n log(n)) sein.

Sie dürfen die Hilfsprozedur merge aus dem vorherigen Aufgabenteil verwenden, auch, wenn Sie
diese nicht gelöst haben.

%%
% 5.
%%

\item Begründen Sie, dass Ihr Algorithmus die Laufzeit O(n log(n)) hat.

%%
% 6.
%%

\item Geben Sie die Lösungen folgender asymptotischer Rekurrenzen (in O-Notation) an:

%%
% (a)
%%

\item T(n)=2 * T(n/2) + Ollog n)
%%
% (b)
%%

\item T(n) =2 * T(n/2) + O(n’)
%%
% (c)
%%

\item T(n) =3 * T(n/2) + O(n)
\end{enumerate}

\end{document}
