\documentclass{lehramt-informatik-aufgabe}
\liLadePakete{formale-sprachen}
\begin{document}
\liAufgabenTitel{Chomsky-Hierarchie}
\section{Aufgabe 2
\index{Formale Sprachen}
\footcite{66115:2016:09}}

Ordnen Sie die folgenden Sprachen über \liAlphabet{a, b} bestmöglich in
die Chomsky-Hierarchie ein und geben Sie jeweils eine kurze Begründung
(1-2 Sätze).\footcite[Aufgabe 8]{theo:ab:5}

\begin{enumerate}

%%
% (a)
%%

\item \liAusdruck[L_1]{a^n b^n}{n \geq 1}

\begin{liAntwort}
Typ-2-Sprache: Die Sprache $L_1$ ist kontextfrei, denn die Sprache
braucht einen Speicher, da sie sich die Anzahl der $a$’s merken muss, um
die gleiche Anzahl an $b$’s produzieren zu können. Dies ist mit einem
Kellerautomaten möglich. Eine Grammtik der Sprache ist
\liGrammatik{variablen={S},produktionen={S -> aSb | EPSILON}}
\end{liAntwort}

%%
% (b)
%%

\item \liAusdruck[L_2]{a^n b^n} {\text{die Turingmaschine mit
Gödelnummer } n \text{ hält auf leerer Eingabe}}

\begin{liAntwort}
Typ-0-Sprache: Die Sprache hat eine Typ-0-Grammatik, da sie
offensichtlich semientscheidbar, aber nicht entscheidbar ist.
\end{liAntwort}

%%
% (c)
%%

\item $L_3 = Σ^* \setminus L_1$

\begin{liAntwort}
Typ-2-Sprache: Die Sprache $L_3$ ist kontextfrei, da ein PDA existiert,
der nicht aktzeptiert, wenn er $L_1$ aktzeptiert. (Ausgänge umgepolt)
\end{liAntwort}

%%
% (d)
%%

\item $L_4 = Σ^* \setminus L_2$

\begin{liAntwort}
Nicht in der Hierachie: Das Komplement einer semi- aber unentscheidbaren
Sprache kann nicht semi-entscheidbar sein, da L sonst entscheidbar wäre.
\end{liAntwort}

%%
% (e)
%%

\item \liAusdruck[L_5]{a^n b^m}
{n + m \text{ ist ein Vielfaches von drei}}

\begin{liAntwort}
Typ-3-Sprache: regulär, 2 Teilautomaten mit je 3 Zuständen (modulo 2
mal)
\end{liAntwort}

%%
% (f)
%%

\item \liAusdruck[L_6]{a^n b^n}{n\text{ Quadratzahl}}

\begin{liAntwort}
nicht regulär, nicht kontextfrei (Pumping-Lemma)
\end{liAntwort}

\end{enumerate}
\end{document}
