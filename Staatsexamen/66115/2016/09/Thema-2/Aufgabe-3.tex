\documentclass{lehramt-informatik-aufgabe}
\liLadePakete{}
\begin{document}
\liAufgabenTitel{}
\section{
\index{Berechenbarkeit}
\footcite{66115:2016:09}}

Sei M 0 , M 1 , . . . eine Gödelisierung aller Registermaschinen (RAMs).
Beantwor- ten Sie folgende Fragen zur Aufzählbarkeit und
Entscheidbarkeit. Beweisen Sie Ihre Antwort.
\footcite[Seite, Aufgabe 6]{theo:ab:4}

\begin{enumerate}

%%
% a)
%%

\item Ist folgende Menge entscheidbar?
A = {x ∈ N|x = 100oderM x hält bei Eingabe x}

\begin{liAntwort}
Ja, x ≥ 100 ist entscheidbar und aufgrund des oder“ ist die 2. Bedingung
nur für”x < 100 relevant. Da x < 100 eine endliche Menge darstellt, kann
eine endliche Liste geführt werden und ein Experte kann für jeden Fall
entscheiden, ob M x hält oder nicht, somit A entscheidbar.
\end{liAntwort}

%%
% b)
%%

\item Ist folgende Menge entscheidbar?
B = {(x, y) ∈ NxN|M x hält bei Eingabe x genau dann, wenn M y bei Eingabe
y hält}

\begin{liAntwort}
Nein, Problem entspricht der parallelen Ausführung des Halteproblems auf
2 Bändern. Das Halteproblem ist unentscheidbar, damit ist auch die
parallele Ausfürhung des Halteproblems und damit B unentscheidbar.
\end{liAntwort}

%%
% c)
%%

\item Ist folgende Menge aufzählbar?
C = {x ∈ N|M x hält bei Eingabe 0 mit dem Ergebnis 1}

\begin{liAntwort}
Ja, aufzählbar, da Menge aller TM aufzählbar und über natürliche Zahlen
definiert (die wiederum aufzählbar sind).
\end{liAntwort}
\end{enumerate}

\end{document}
