\documentclass{lehramt-informatik-aufgabe}
\liLadePakete{mathe,graph}
\begin{document}
\liAufgabenTitel{Karlsruhe nach Kassel}

\section{Dijkstra Algorithmus
\index{Algorithmus von Dijkstra}
\footcite[Thema 2 Aufgabe 6 Seite 9]{examen:66115:2016:03}}

\begin{enumerate}

%%
% a)
%%

\item Berechnen Sie für folgenden Graphen den kürzesten Weg von
Karlsruhe nach Kassel und dokumentieren Sie den Berechnungsweg:

Verwendete Abkürzungen:

\begin{itemize}
\item Augsburg = A
\item Erfurt = EF
\item Frankfurt = F
\item Karlsruhe = KA
\item Kassel = KS
\item Mannheim = MA
\item München = M
\item Nürnberg = N
\item Stuttgart = S
\item Würzburg = WÜ
\end{itemize}

Zahl = Zahl in Kilometern

\begin{liGraphenFormat}
A: 3 1
E: 2 3
F: 4 7
KA: 0 2
KS: 9 6
M: 6 1
MA: 0 6
N: 6 3
S: 7 5
W: 4 5
A -- KA: 250
A -- M: 84
E -- W: 186
F -- KS: 173
F -- MA: 85
F -- W: 217
M -- KS: 502
M -- N: 167
MA -- KA: 80
N -- S: 183
N -- W: 103
\end{liGraphenFormat}

\begin{tikzpicture}
\node[li graph knoten] (A) at (3,1) {A};
\node[li graph knoten] (E) at (2,3) {E};
\node[li graph knoten] (F) at (4,7) {F};
\node[li graph knoten] (KA) at (0,2) {KA};
\node[li graph knoten] (KS) at (9,6) {KS};
\node[li graph knoten] (M) at (6,1) {M};
\node[li graph knoten] (MA) at (0,6) {MA};
\node[li graph knoten] (N) at (6,3) {N};
\node[li graph knoten] (S) at (7,5) {S};
\node[li graph knoten] (W) at (4,5) {W};

\path[li graph kante] (A) edge node {250} (KA);
\path[li graph kante] (A) edge node {84} (M);
\path[li graph kante] (E) edge node {186} (W);
\path[li graph kante] (F) edge node {173} (KS);
\path[li graph kante] (F) edge node {85} (MA);
\path[li graph kante] (F) edge node {217} (W);
\path[li graph kante] (M) edge node {502} (KS);
\path[li graph kante] (M) edge node {167} (N);
\path[li graph kante] (MA) edge node {80} (KA);
\path[li graph kante] (N) edge node {183} (S);
\path[li graph kante] (N) edge node {103} (W);
\end{tikzpicture}

%%
% b)
%%

\item Könnte man den Dijkstra Algorithmus auch benutzen, um das
Travelling-Salesman Problem zu lösen?

\end{enumerate}

\end{document}
