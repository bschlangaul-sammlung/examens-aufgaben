\documentclass{lehramt-informatik-aufgabe}
\liLadePakete{mathe,graph}
\begin{document}
\liAufgabenTitel{Karlsruhe nach Kassel}

\section{Dijkstra Algorithmus
\index{Algorithmus von Dijkstra}
\footcite[Thema 2 Aufgabe 6 Seite 9]{examen:66115:2016:03}}

\begin{enumerate}

%%
% a)
%%

\item Berechnen Sie für folgenden Graphen den kürzesten Weg von
Karlsruhe nach Kassel und dokumentieren Sie den Berechnungsweg:

Verwendete Abkürzungen:

\begin{itemize}
\item Augsburg = A
\item Erfurt = EF
\item Frankfurt = F
\item Karlsruhe = KA
\item Kassel = KS
\item Mannheim = MA
\item München = M
\item Nürnberg = N
\item Stuttgart = S
\item Würzburg = WÜ
\end{itemize}

Zahl = Zahl in Kilometern

\begin{center}
\graph knoten {
  \knoten{A}(3,1)
  \knoten{E}(2,3)
  \knoten{F}(4,7)
  \knoten{KA}(0,2)
  \knoten{KS}(9,6)
  \knoten{M}(6,1)
  \knoten{MA}(0,6)
  \knoten{N}(6,3)
  \knoten{S}(7,5)
  \knoten{W}(4,5)
} kanten {
  \kante(A-KA){250}
  \kante(A-M){84}
  \kante(E-W){186}
  \kante(F-KS){173}
  \kante(F-MA){85}
  \kante(F-W){217}
  \kante(M-KS){502}
  \kante(M-N){167}
  \kante(MA-KA){80}
  \kante(N-S){183}
  \kante(N-W){103}
}
\end{center}

%%
% b)
%%

\item Könnte man den Dijkstra Algorithmus auch benutzen, um das
Travelling-Salesman Problem zu lösen?

\end{enumerate}

\end{document}
