\documentclass{lehramt-informatik-aufgabe}
\liLadePakete{}
\begin{document}
\liAufgabenTitel{Verständnis}
\section{Verständnis Komplexitätstheorie
\index{Komplexitätstheorie}
\footcite{66115:2016:03}}

Beantworten Sie kurz, präzise und mit Begründung folgende Fragen: (Die
Begründungen müssen keine formellen mathematischen Beweise sein)
\footcite[Aufgabe 15: StEx F2016 T2 A3, StEx H2017 T1 A3]{theo:ab:4}

\begin{enumerate}

%%
% a)
%%

\item In der O-Notation insbesondere für die Zeitkomplexität von
Algorithmen lässt man i.A. konstante Faktoren oder kleinere Terme weg.
\ZB schreibt man anstelle O(3n2+5) einfach nur O(n2). Warum macht man
das so?

%%
% b)
%%

\item Was ist die typische Vorgehensweise, wenn man für ein neues
Problem die NP-Vollständigkeit untersuchen will?

%%
% c)
%%

\item Was könnte man tun, um P=NP zu beweisen?
%%
% d)
%%

\item Sind NP-vollständige Problem mit Loop-Programmen lösbar? (Antwort
mit Begründung!)

%%
% e)
%%

\item Wie zeigt man aus der NP-Härte des SAT-Problems die NP-Härte des
3SAT-Problems? (3SAT ist ein SAT-Problem wobei alle Klauseln maximal 3
Literale haben.)

\end{enumerate}
\end{document}
