\documentclass{lehramt-informatik-minimal}
\InformatikPakete{mathe}
\begin{document}

\section{4. Hashing
\index{Streutabellen (Hashing)}
\footcite[Thema 2 Aufgabe 4 Seite 7]{examen:66115:2016:03}}

Betrachte eine Hashtabelle der Größe m = 10.

\begin{enumerate}

%%
% a)
%%

\item Welche der folgenden Hashfunktionen ist für Hashing mit
verketteten Listen am besten geeignet? Begründen Sie Ihre Wahl!

\begin{enumerate}
\item $h_1(x) = (4x + 3) mod m$
\item $h_2(x) = (3x + 3) \mod m$.
\end{enumerate}

%%
% b)
%%

\item Welche der folgenden Hashfunktionen ist für Hashing mit offener
Adressierung am besten geeignet? Begründen Sie Ihre Wahl!

\begin{enumerate}
\item $h_1(x,i) = (7 \cdot x + i \cdot m) \mod m$
\item $h_2(x,i) = (7 \cdot x + i \cdot (m - 1)) \mod m$.
\end{enumerate}
\end{enumerate}

\end{document}
