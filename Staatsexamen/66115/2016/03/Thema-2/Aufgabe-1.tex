\documentclass{lehramt-informatik-aufgabe}
\liLadePakete{formale-sprachen}
\begin{document}
\liAufgabenTitel{Verständnis}
\section{Verständnis formale Sprachen
\index{Formale Sprachen}
\footcite{66115:2016:03}}

Beantworten Sie kurz, präzise und mit Begründung folgende Fragen: (Die
Begründungen müssen keine formellen mathematischen Beweise sein).

\begin{enumerate}

%%
% a)
%%

\item Welche Möglichkeiten gibt es, eine formale Sprache vom Typ 3 zu
definieren?

\begin{liAntwort}
\begin{itemize}
\item reguläre Grammatik
\item endlicher Automat
\item regulärer Ausdruck
\end{itemize}
\end{liAntwort}

%%
% b)
%%

\item Was ist die Komplexität des Wortproblems für Typ-3 Sprachen und
wieso ist das so?\footcite[Aufgabe 5a)]{theo:ab:5}

\begin{liAntwort}
$P$, CYK-Algorithmus löst es ein Polynomialzeit.
\end{liAntwort}

%%
% c)
%%

\item Sind Syntaxbäume zu einer Grammatik immer eindeutig? Falls nicht,
geben Sie ein Gegenbeispiel.

\begin{liAntwort}
Nein. Syntaxbäume zu einer Grammtik sind nicht immer eindeutig.

\liPseudoUeberschrift{Gegenbeispiel}

\liGrammatik{
  variablen={S, A, B},
  alphabet={a}
}

\begin{liProduktionsRegeln}
S -> A A,
S -> B B,
A -> a,
B -> a,
\end{liProduktionsRegeln}

\liAbleitung{S -> A A -> a A -> a a}

\liAbleitung{S -> B B -> a B -> a a}\footcite{wiki:mehrdeutig}
\end{liAntwort}

%%
% d)
%%

\item Wie kann man die Äquivalenz zweier Typ-3 Sprachen nachweisen?

\begin{liAntwort}
Wir können von den auf Äquivalenz zu überprüfenden Sprachen jeweils
einen minimalen endlichen Automaten bilden. Sie diese entstanden zwei
Automaten äquivalent so sind auch die Sprachen äquivalent.
\footcite{wiki:aequivalenzproblem}
\end{liAntwort}

%%
% e)
%%

\item Wie kann man das Wortproblem für das Komplement einer Typ-3
Sprache lösen?

\begin{liAntwort}
Da das Komplement einer regulären Sprache wieder eine reguläre Sprache
ergibt, kann das Wortproblem beim Komplement durch einen deterministisch
endlichen Automaten gelöst werden.
\end{liAntwort}

%%
% f)
%%

\item Weshalb gilt das Pumping-Lemma für Typ 3 Sprachen?

%%
% g)
%%

\item Ist der Nachweis, dass das Typ-3 Pumping-Lemma für eine gegebene
Sprache gilt, ausreichend, um zu zeigen, dass die Sprache vom Typ 3 ist?
Falls nicht, geben Sie ein Gegenbeispiel, mit Begründung.

%%
% h)
%%

\item Geben Sie ein Beispiel, an dem deutlich wird, dass
deterministische und nichtdeterministische Typ-2 Sprachen
unterschiedlich sind.

\begin{liAntwort}
\begin{description}
\item[Deterministisch Kontextfrei]
\liAusdruck{0^n 1^n}{n \geq 0}

\item[Nichtdeterministisch Kontextfrei]
\liAusdruck{\omega \omega^R}{\omega \geq \{0, 1\}^*} (R steht für rückwärts)
\end{description}
\liFussnoteUrl{https://docplayer.org/19566652-Einfuehrung-in-die-theoretische-informatik.html}
\end{liAntwort}

%%
% i)
%%

\item Worin macht sich der Unterschied zwischen Typ 0 und 1 bemerkbar,
wenn man Turingmaschinen benutzt, um das Wortproblem vom Typ 0 oder 1 zu
lösen. Warum ist das so?
\footcite[Aufgabe 5b)]{theo:ab:5}

\begin{liAntwort}
Typ 0: semi-entscheidbar, Typ 1: entscheidbar

Da Typ 1 nur Wörter verlängert, kann daher in Polynomialzeit überprüft
werden, ob das Wort in der Sprache liegt, indem die Regeln angewendet
werden, bis das Wortende erreicht ist.
\end{liAntwort}

\end{enumerate}
\end{document}
