\documentclass{lehramt-informatik-aufgabe}
\liLadePakete{}
\begin{document}
\liAufgabenTitel{Verständnis}
\section{Verständnis formale Sprachen
\index{Formale Sprachen}
\footcite{66115:2016:03}}

Beantworten Sie kurz, präzise und mit Begründung folgende Fragen: (Die
Begründungen müssen keine formellen mathematischen Beweise sein).

\begin{enumerate}

%%
% a)
%%

\item Welche Möglichkeiten gibt es, eine formale Sprache vom Typ 3 zu
definieren?
%%
% b)
%%

\item Was ist die Komplexität des Wortproblems für Typ-3 Sprachen und
wieso ist das so?\footcite[Aufgabe 5a)]{theo:ab:5}

\begin{liAntwort}
$P$, CYK-Algorithmus löst es ein Polynomialzeit.
\end{liAntwort}

%%
% c)
%%

\item Sind Syntaxbäume zu einer Grammatik immer eindeutig? Falls nicht,
geben Sie ein Gegenbeispiel.

%%
% d)
%%

\item Wie kann man die Äquivalenz zweier Typ-3 Sprachen nachweisen?
%%
% e)
%%

\item Wie kann man das Wortproblem für das Komplement einer Typ-3
Sprache lösen?
%%
% f)
%%

\item Weshalb gilt das Pumping-Lemma für Typ 3 Sprachen?

%%
% g)
%%

\item Ist der Nachweis, dass das Typ-3 Pumping-Lemma für eine gegebene
Sprache gilt, ausreichend, um zu zeigen, dass die Sprache vom Typ 3 ist?
Falls nicht, geben Sie ein Gegenbeispiel, mit Begründung.

%%
% h)
%%

\item Geben Sie ein Beispiel, an dem deutlich wird, dass
deterministische und nichtdeterministische Typ-2 Sprachen
unterschiedlich sind.

%%
% i)
%%

\item Worin macht sich der Unterschied zwischen Typ 0 und 1 bemerkbar,
wenn man Turingmaschinen benutzt, um das Wortproblem vom Typ 0 oder 1 zu
lösen. Warum ist das so?
\footcite[Aufgabe 5b)]{theo:ab:5}

\begin{liAntwort}
Typ 0: semi-entscheidbar, Typ 1: entscheidbar

Da Typ 1 nur Wörter verlängert, kann daher in Polynomialzeit überprüft
werden, ob das Wort in der Sprache liegt, indem die Regeln angewendet
werden, bis das Wortende erreicht ist.
\end{liAntwort}

\end{enumerate}
\end{document}
