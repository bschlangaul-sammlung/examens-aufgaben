\documentclass{lehramt-informatik-aufgabe}
\liLadePakete{}
\begin{document}
\liAufgabenTitel{k-COL}
\section{Komplexität
\index{Komplexitätstheorie}
\footcite{66115:2016:03}}

Das Problem k-COL ist wie folgt definiert:

\begin{description}
\item[Gegeben:]

Ein ungerichteter Graph $G = (V, E)$.

\item[Frage:]

Kann man jedem Knoten $v$ in $V$ eine Zahl $z(v) \in \{1, \dots ,k\}$
zuordnen, so dass für alle Kanten $(u_1,u_2) \in E$ gilt: $z(u_1) \neq
z(u_2)?$
\end{description}

Zeigen Sie, dass man 3-COL in polynomieller Zeit auf 4-COL reduzieren
kann. Beschreiben Sie dazu die Reduktion und zeigen Sie anschließend
ihre Korrektheit.

\begin{liAntwort}
Zu Zeigen: 3-COL $\leq_P$ 4-COL

also 4-COL ist mindestens so schwer wie 3-COL Eingabeinstanz von 3-COL
durch eine Funktion in eine Eingabeinstanz von 4-COL umbauen so, dass
jede JA- bzw. NEIN-Instanz von 3-COL eine JA- bzw. NEIN-Instanz von
4-COL ist.
\footcite[Seite 67]{theo:fs:4}

Funktion ergänzt einen beliebigen gegebenen Graphen um einen
weiteren Knoten, der mit allen Knoten des ursprünglichen Graphen
durch eine Kante verbunden ist.

\begin{description}
\item[total] ja

\item[in Polynomialzeit berechenbar] ja

(Begründung: z. B. Adjazenzmatrix $\rightarrow$ neue Spalte)

\item[Korrektheit:] ja

Färbe den „neuen“ Knoten mit einer Farbe. Da er mit allen anderen Knoten
verbunden ist, bleiben für die übrigen Knoten nur drei Farben.
\footcite[Seite 69]{theo:fs:4}
\end{description}
\end{liAntwort}
\end{document}
