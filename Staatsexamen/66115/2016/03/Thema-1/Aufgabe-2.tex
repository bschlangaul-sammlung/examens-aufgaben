\documentclass{lehramt-informatik-aufgabe}
\liLadePakete{formale-sprachen,chomsky-normalform}
\begin{document}
\let\schrittE=\liChomskyUeberErklaerung

\liAufgabenTitel{Alphabet 0,1,2}
\section{Aufgabe 2
\index{Chomsky-Normalform}
\footcite{66115:2016:03}}

Betrachten Sie die folgende Grammatik \liGrammatik{variablen={S, A},
alphabet={0, 1, 2}} mit\footcite[Aufgabe 2e)]{theo:ab:5}

\begin{liProduktionsRegeln}
S -> 0 S 0 | 1 S 1 | 2 A 2 | 0 | 1 | EPSILON,
A -> A 2
\end{liProduktionsRegeln}
\liFlaci{Gf6scqja9}

\noindent
Konstruieren Sie für die Grammatik $G$ schrittweise eine äquivalente
Grammatik in Chomsky-Normalform. Geben Sie für jeden einzelnen Schritt
des Verfahrens das vollständige Zwischenergebnis an und erklären Sie
kurz, was in dem Schritt getan wurde.

\begin{liAntwort}
\begin{enumerate}

% S.S -> S | EPSILON
% S   -> T1 S.1 | T2 S.2 | 0 | 1 | T1 T1 | T2 T2
% T1  -> 0
% T2  -> 1
% S.1 -> S T1
% S.2 -> S T2

\item \schrittE{1}

falls $S \rightarrow \varepsilon \in P$ neuen Startzustand $S'$
einführen

\begin{liProduktionsRegeln}
S -> 0 S 0 | 1 S 1 | 2 A 2 | 0 | 1 | 0 0 | 1 1,
S_1 -> EPSILON | S
A -> A 2
\end{liProduktionsRegeln}

\item \schrittE{2}

\liNichtsZuTun

\item \schrittE{3}

N = Null
E = Eins
Z = Zwei

\begin{liProduktionsRegeln}
S -> N S N | E S E | Z A Z | 0 | 1 | N N | E E,
S_1 -> EPSILON | S,
A -> A Z,
N -> 0,
E -> 1,
Z -> 2,
\end{liProduktionsRegeln}

\item \schrittE{4}

\begin{liProduktionsRegeln}
S -> N S_N | E S_E | Z A | 0 | 1 | N N | E E,
S_N -> S N,
S_E -> S E,
A -> A Z,
N -> 0,
E -> 1,
Z -> 2,
\end{liProduktionsRegeln}

\end{enumerate}
\end{liAntwort}

\end{document}
