\documentclass{lehramt-informatik-aufgabe}
\liLadePakete{syntax}
\begin{document}
\liAufgabenTitel{}
\section{4. Turingmaschinen
\index{Turing-Maschine}
\footcite{66115:2016:03}}

\begin{enumerate}

%%
% a)
%%

\item Geben Sie eine deterministische 2-Band Turingmaschine $M$ an, die
die Funktion

\begin{displaymath}
f_M(a^n) = a^n b^n
\end{displaymath}

berechnet. Die Maschine $M$ nimmt somit immer einen String der Form
$a^n$ (ein String, der aus $n$ $a$’s für beliebiges $n \in N$ besteht)
als Eingabe und produziert anschließend auf Band 2 als Ausgabe den
String $a^n b^n$ (ein String aus $n$ $a$’s gefolgt von $n$ $b$’s).

Beschreiben Sie außerdem die Idee hinter Ihrer Konstruktion.

\begin{liAntwort}
\begin{minted}{md}
name: 66115 2016 03 1 4
init: z0
accept: z2

z0, a,_
z0, a,a, >,>

z0, _,_
z1, _,_, <,-

z1, a,_
z1, a,_, <,-

z1, _,_
z2, _,_, >,-

z2, a,_
z2, a,b, >,>
\end{minted}
\liFussnoteUrl{http://turingmachinesimulator.com/shared/lyptczerhe}
\end{liAntwort}

%%
% b)
%%
\item Geben Sie die Konfigurationsfolge der Turingmaschine aus (a) für
die Eingabe $aa$ an.
\end{enumerate}
\end{document}
