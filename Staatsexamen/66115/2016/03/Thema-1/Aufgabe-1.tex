\documentclass{lehramt-informatik-aufgabe}
\liLadePakete{automaten}
\begin{document}
\def\z#1#2{$q_#1z_#2$}

\liAufgabenTitel{Reguläre Sprachen}
\section{Aufgabe 1
\index{Reguläre Sprache}
\footcite{66115:2016:03}}

\begin{enumerate}

%%
% a)
%%

\item Geben Sie einen möglichst einfachen regulären Ausdruck für die
Sprache L, = aja2--- an n > 3,a; € {a, b} für allei =1,...,n und a,
a, an.

%%
% b)
%%

\item Geben Sie einen möglichst einfachen regulären Ausdruck für die
Sprache Ly = {w € {a,b}* | w enthält genau ein b und ist von ungerader
Länge}

an.

%%
% c)
%%

\item Beschreiben Sie die Sprache des folgenden Automaten $A$,
möglichst einfach und präzise in ihren eigenen Worten.

\begin{center}
\begin{tikzpicture}[li automat]
  \node[state,,initial] (q0) at (2.14cm,-2.14cm) {$q_0$};
  \node[state,] (q1) at (4cm,-2.14cm) {$q_1$};
  \node[state,] (q2) at (6cm,-2.14cm) {$q_2$};
  \node[state,,accepting] (q3) at (7.57cm,-2.14cm) {$q_3$};

  \path (q0) edge[auto,bend left] node{a} (q1);
  \path (q0) edge[auto,loop] node{b} (q0);
  \path (q1) edge[auto] node{b} (q2);
  \path (q1) edge[auto,bend left] node{a} (q0);
  \path (q2) edge[auto] node{a} (q3);
  \path (q2) edge[auto,bend left] node{b} (q0);
  \path (q3) edge[auto,loop] node{a,b} (q3);
\end{tikzpicture}
\end{center}
\liFussnoteUrl{https://flaci.com/Arz003ccg}

%%
% d)
%%

\item Betrachten Sie folgenden Automaten $A_2$:

\begin{center}
\begin{tikzpicture}[li automat]
  \node[state,initial] (z0) at (2.14cm,-2.14cm) {$z_0$};
  \node[state,accepting] (z1) at (4.86cm,-2.14cm) {$z_1$};

  \path (z0) edge[auto,bend left] node{a,b} (z1);
  \path (z1) edge[auto,bend left] node{a,b} (z0);
\end{tikzpicture}
\end{center}
\liFussnoteUrl{https://flaci.com/Ap9qbkumc}

Im Original sind die Zustände mit $q_x$ benannt. Damit wir die
Schnittmenge besser bilden können, wird hier $z_x$ verwendet.

Konstruieren Sie einen endlichen Automaten, der die Schnittmenge der
Sprachen $L(A_1)$ und $L(A_2)$ akzeptiert.

\begin{liAntwort}

$A_1$

\begin{tabular}{l|l|l}
  & a & b \\\hline
0 & 1 & 0 \\
1 & 0 & 2 \\
2 & 3 & 0 \\
3 & 3 & 3 \\
\end{tabular}

$A_2$

\begin{tabular}{l|l|l}
  & a & b \\\hline
0 & 1 & 1 \\
1 & 0 & 0 \\
\end{tabular}

Neuer Endzustand: \z31

\begin{tabular}{l|l|l}
     & a    & b    \\\hline
\z00 & \z11 & \z01 \\
\z10 & \z01 & \z21 \\
\z20 & \z31 & \z01 \\
\z30 & \z31 & \z31 \\
\z01 & \z10 & \z00 \\
\z11 & \z00 & \z20 \\
\z21 & \z30 & \z00 \\
\z31 & \z30 & \z30 \\
\end{tabular}
\end{liAntwort}
\end{enumerate}

\end{document}
