\documentclass{lehramt-informatik-aufgabe}
\liLadePakete{automaten}
\begin{document}
\liAufgabenTitel{Reguläre Sprachen}
\section{Aufgabe 1
\index{Reguläre Sprache}
\footcite{66115:2016:03}}

\begin{enumerate}

%%
% a)
%%

\item Geben Sie einen möglichst einfachen regulären Ausdruck für die
Sprache L, = aja2--- an n > 3,a; € {a, b} für allei =1,...,n und a,
a, an.

%%
% b)
%%

\item Geben Sie einen möglichst einfachen regulären Ausdruck für die
Sprache Ly = {w € {a,b}* | w enthält genau ein b und ist von ungerader
Länge}

an.

%%
% c)
%%

\item Beschreiben Sie die Sprache des folgenden Automaten A,
möglichst einfach und präzise in ihren eigenen Worten.

\begin{liAntwort}
\begin{tikzpicture}[li automat]
  \node[state,,initial] (q0) at (2.14cm,-2.14cm) {$q_0$};
  \node[state,] (q1) at (4cm,-2.14cm) {$q_1$};
  \node[state,] (q2) at (6cm,-2.14cm) {$q_2$};
  \node[state,,accepting] (q3) at (7.57cm,-2.14cm) {$q_3$};

  \path (q0) edge[auto,bend left] node{a} (q1);
  \path (q0) edge[auto,loop] node{b} (q0);
  \path (q1) edge[auto] node{b} (q2);
  \path (q1) edge[auto,bend left] node{a} (q0);
  \path (q2) edge[auto] node{a} (q3);
  \path (q2) edge[auto,bend left] node{b} (q0);
  \path (q3) edge[auto,loop] node{a,b} (q3);
\end{tikzpicture}

\liFussnoteUrl{https://flaci.com/Arz003ccg}
\end{liAntwort}

%%
% d)
%%

\item Betrachten Sie folgenden Automaten Asa:

Konstruieren Sie einen endlichen Automaten, der die Schnittmenge der
Sprachen L(A,) und L(A2) akzeptiert.
\end{enumerate}

\end{document}
