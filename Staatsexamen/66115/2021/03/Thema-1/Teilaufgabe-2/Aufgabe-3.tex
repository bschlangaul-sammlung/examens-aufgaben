\documentclass{lehramt-informatik-aufgabe}
\liLadePakete{}
\begin{document}
\liAufgabenTitel{Schwach zusammenhängend gerichteter Graph}
\section{Aufgabe 3
\index{Breitensuche}
\footcite{66115:2021:03}}

Wir betrachten eine Variante der Breitensuche (BFS), bei der die Knoten
markiert werden, wenn sie das erste Mal besucht werden. Außerdem wird
die Suche einmal bei jedem unmarkierten Knoten gestartet, bis alle
Knoten markiert sind. Wir betrachten gerichtete Graphen. Ein gerichteter
Graph $G$ ist \emph{schwach zusammenhängend}, wenn der ungerichtete
Graph (der sich daraus ergibt, dass man die Kantenrichtungen von $G$
ignoriert) zusammenhängend ist.

\begin{enumerate}
%%
% a)
%%

\item Beschreiben Sie für ein allgemeines $n \in \mathbb{N}$ mit $n \geq
2$ den Aufbau eines schwach zusammenhängenden Graphen $G_n$, mit $n$
Knoten, bei dem die Breitensuche $\Theta(n)$ mal gestartet werden muss,
bis alle Knoten markiert sind.

%%
% b)
%%

\item Welche asymptotische Laufzeit in Abhängigkeit von der Anzahl der
Knoten $(n)$ und von der Anzahl der Kanten $(m)$ hat die Breitensuche
über alle Neustarts zusammen? Beachten Sie, dass die Markierungen nicht
gelöscht werden. Geben Sie die Laufzeit in $\Theta$-Notation an.
Begründen Sie Ihre Antwort.
\end{enumerate}
\end{document}
