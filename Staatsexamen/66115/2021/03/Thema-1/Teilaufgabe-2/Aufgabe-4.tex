\documentclass{lehramt-informatik-aufgabe}
\liLadePakete{graph,spalten}
\begin{document}
\liAufgabenTitel{Kürzeste-Wege-Bäume und minimale Spannbäume}
\section{Aufgabe 4
\index{Graphen}
\footcite{66115:2021:03}}

Die Algorithmen von Dijkstra und Jarník-Prim gehen ähnlich vor. Beide
berechnen, ausgehend von einem Startknoten, einen Baum. Allerdings
berechnet der Algorithmus von Dijkstra\index{Algorithmus von Dijkstra}
einen Kürzesten-Wege-Baum, während der Algorithmus von
Jarník-Prim\index{Algorithmus von Prim} einen minimalen Spannbaum
berechnet.

\begin{enumerate}

%%
% a)
%%

\item Geben Sie einen ungerichteten gewichteten Graphen $G$ mit
höchstens fünf Knoten und einen Startknoten $s$ von $G$ an, so dass
\textbf{Dijkstra}($G$, $s$) und \textbf{Jarník-Prim}($G$, $s$) ausgehend
von $s$ verschiedene Bäume in $G$ liefern. Geben Sie beide Bäume an.

\begin{liGraphenFormat}
A: 0 0
B: 1 1
S: 2 0
C: 1 -1
A -- B: 1
A -- C: 1
A -- S: 2
B -- S: 2
C -- S: 2
\end{liGraphenFormat}

\begin{liAntwort}
\begin{multicols}{2}
\liPseudoUeberschrift{Originalgraph}

\begin{tikzpicture}[li graph]
\node (A) at (0,0) {A};
\node (B) at (1,1) {B};
\node (C) at (1,-1) {C};
\node (S) at (2,0) {S};

\path (A) edge node {1} (B);
\path (A) edge node {1} (C);
\path (A) edge node {2} (S);
\path (B) edge node {2} (S);
\path (C) edge node {2} (S);
\end{tikzpicture}

\liPseudoUeberschrift{Dijkstra-Wegebaum von S aus:}

\begin{tikzpicture}[li graph]
\node (A) at (0,0) {A};
\node (B) at (1,1) {B};
\node (C) at (1,-1) {C};
\node (S) at (2,0) {S};

\path (A) edge node {2} (S);
\path (B) edge node {2} (S);
\path (C) edge node {2} (S);
\end{tikzpicture}

\columnbreak

\liPseudoUeberschrift{Minimaler Spannbaum 1}

\begin{tikzpicture}[li graph]
\node (A) at (0,0) {A};
\node (B) at (1,1) {B};
\node (C) at (1,-1) {C};
\node (S) at (2,0) {S};

\path (A) edge node {1} (B);
\path (A) edge node {1} (C);
\path (A) edge node {2} (S);
\end{tikzpicture}

\liPseudoUeberschrift{Minimaler Spannbaum 2}

\begin{tikzpicture}[li graph]
\node (A) at (0,0) {A};
\node (B) at (1,1) {B};
\node (C) at (1,-1) {C};
\node (S) at (2,0) {S};

\path (A) edge node {1} (B);
\path (A) edge node {1} (C);
\path (B) edge node {2} (S);
\end{tikzpicture}

\liPseudoUeberschrift{Minimaler Spannbaum 3}

\begin{tikzpicture}[li graph]
\node (A) at (0,0) {A};
\node (B) at (1,1) {B};
\node (C) at (1,-1) {C};
\node (S) at (2,0) {S};

\path (A) edge node {1} (B);
\path (A) edge node {1} (C);
\path (C) edge node {2} (S);
\end{tikzpicture}
\end{multicols}
\end{liAntwort}

%%
% b)
%%

\item Geben Sie eine unendlich große Menge von Graphen an, auf denen der
Algorithmus von Jarník-Prim asymptotisch schneller ist als der
Algorithmus von Kruskal, der ebenfalls minimale Spannbäume berechnet.

\emph{Hinweis:} Für einen Graphen mit $n$ Knoten und $m$ Kanten benötigt
Jarník-Prim $\mathcal{O}(m + n \log n)$ Zeit, Kruskal $\mathcal{O}(m
\log m)$ Zeit.

%%
% c)
%%

\item Sei $Z$ die Menge der zusammenhängenden Graphen und $G \in Z$. Sei
$n$ die Anzahl der Knoten von $G$ und $m$ die Anzahl der Kanten von $G$.
Entscheiden Sie mit Begründung, ob $\log m \in \Theta(\log n)$ gilt.

\end{enumerate}
\end{document}
