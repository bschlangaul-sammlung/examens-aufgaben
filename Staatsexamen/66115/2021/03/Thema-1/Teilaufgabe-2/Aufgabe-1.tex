\documentclass{lehramt-informatik-aufgabe}
\liLadePakete{}
\begin{document}
\liAufgabenTitel{Sortieren}
\section{Aufgabe 1
\index{Sortieralgorithmen}
\footcite{66115:2021:03}}

\begin{enumerate}

%%
% a)
%%

\item Geben Sie für folgende Sortierverfahren jeweils zwei Felder A und
B an, so dass das jeweilige Sortierverfahren angewendet auf A seine
Best-Case-Laufzeit und angewendet auf B seine Worst-Case-Laufzeit
erreicht. (Wir messen die Laufzeit durch die Anzahl der Vergleiche
zwischen Elementen der Eingabe.) Dabei soll das Feld A die Zahlen
1,2,...,7 genau einmal enthalten; das Feld B ebenso. Sie bestimmen also
nur die Reihenfolge der Zahlen.

Wenden Sie als Beleg für Ihre Aussagen das jeweilige Sortierverfahren
auf die Felder A und B an und geben Sie nach jedem größeren Schritt des
Algorithmus den Inhalt der Felder an.

Geben Sie außerdem für jedes Verfahren asymptotische Best- und
Worst-Case-Laufzeit für

ein Feld der Länge n an.

Die im Pseudocode verwendete Unterroutine Swap (A,i,j) vertauscht im
Feld A die jeweiligen

Elemente mit den Indizes i und j miteinander.
\begin{enumerate}

%%
% i)
%%

\item Insertionsort

\index{Insertionsort}

%%
% ii)
%%

\item Standardversion von Quicksort (Pseudocode s.u., Feldindizes beginnen bei 1), bei der
das letzte Element eines Teilfeldes als Pivot-Element gewählt wird.

%%
% iii)
%%

\item QuicksortVar: Variante von Quicksort, bei der immer das mittlere Element eines Teil-
feldes als Pivot-Element gewählt wird (Pseudocode s.u., nur eine Zeile neu).

Bei einem Aufruf von PartitionVar auf ein Teilfeld Al£..r] wird also erst mithilfe der
Unterroutine Swap A[|(€+r —1)/2|] mit Afr] vertauscht.

% Quicksort(A, 2 = 1,r = A.length)
% if 2<r then
% m = Partition(A, £,r)
% | Quicksort(A,&,m — 1)
% Quicksort(A,m +1, r)

% int Partition(int[] A, int £, intr)

% pivot = Alr]

% i=f

% for j = £tor—1do

% if A[ 7] < pivot then

% Swap(A, i, 7)
% i=i+l

% Swap(A, i,r)

% return i

% QuicksortVar(A, 2 = 1,r =
% A. length)
% if 2<r then
% m = PartitionVar(A, 2,r)
% QuicksortVar(A, £,m — 1)
% QuicksortVar(A,m + 1,r)

% int PartitionVar(int[] A, int £, intr)
% Swap(A,|(£+r-1)/2]|,r) // neu!
% pivot = Alr]
% t=
% for j =£tor—1do
% if A[ j] < pivot then

% Swap(A, i, j)

% d=i+l
% Swap(A,i,r)
% return 7
\end{enumerate}

%%
% b)
%%

\item Geben Sie die asymptotische Best- und Worst-Case-Laufzeit von Mergesort an.

\end{enumerate}
\end{document}
