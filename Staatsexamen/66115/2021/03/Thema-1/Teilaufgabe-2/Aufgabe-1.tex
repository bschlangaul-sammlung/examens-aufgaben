\documentclass{lehramt-informatik-aufgabe}
\liLadePakete{mathe}
\begin{document}
\liAufgabenTitel{Sortieren}
\section{Aufgabe 1
\index{Sortieralgorithmen}
\footcite{66115:2021:03}}

\begin{enumerate}

%%
% a)
%%

\item Geben Sie für folgende Sortierverfahren jeweils zwei Felder $A$
und $B$ an, so dass das jeweilige Sortierverfahren angewendet auf $A$
seine Best-Case-Laufzeit und angewendet auf $B$ seine
Worst-Case-Laufzeit erreicht. (Wir messen die Laufzeit durch die Anzahl
der Vergleiche zwischen Elementen der Eingabe.) Dabei soll das Feld $A$
die Zahlen $1,2,\dots,7$ genau einmal enthalten; das Feld $B$ ebenso.
Sie bestimmen also nur die Reihenfolge der Zahlen.

Wenden Sie als Beleg für Ihre Aussagen das jeweilige Sortierverfahren
auf die Felder $A$ und $B$ an und geben Sie nach jedem größeren Schritt
des Algorithmus den Inhalt der Felder an.

Geben Sie außerdem für jedes Verfahren asymptotische Best- und
Worst-Case-Laufzeit für ein Feld der Länge $n$ an.

Die im Pseudocode verwendete Unterroutine $\text{Swap} (A,i,j)$
vertauscht im Feld $A$ die jeweiligen Elemente mit den Indizes $i$ und
$j$ miteinander.
\begin{enumerate}

%%
% i)
%%

\item \textbf{Insertionsort}
\index{Insertionsort}

%%
% ii)
%%

\item Standardversion von \textbf{Quicksort} (Pseudocode s.u.,
Feldindizes beginnen bei 1), bei der das letzte Element eines Teilfeldes
als Pivot-Element gewählt wird.
\index{Quicksort}

%%
% iii)
%%

\item \textbf{QuicksortVar}: Variante von Quicksort, bei der immer das
mittlere Element eines Teilfeldes als Pivot-Element gewählt wird
(Pseudocode s.u., nur eine Zeile neu).

Bei einem Aufruf von PartitionVar auf ein Teilfeld $A[l \dots r]$ wird
also erst mithilfe der Unterroutine Swap $A[l + r - 1 / 2]$ mit $A[r]$
vertauscht.

Quicksort(A, l = 1,r = A.length)
if l < r then
  m = Partition(A, l, r)
  Quicksort(A, l, m - 1)
  Quicksort(A, m + 1, r)

int Partition(int[] A, int l, int r)
pivot = A[r]
i = l
for j = l to r - 1 do
  if A[j] < pivot then
    Swap(A, i, 7)
    i = i + l
Swap(A, i,r)
return i

QuicksortVar(A, l = 1,r = A.length)
if l < r then
  m = PartitionVar(A, l, r)
  QuicksortVar(A, l, m - 1)
  QuicksortVar(A, m + 1, r)

int PartitionVar(int[] A, int l, int r)
pivot = A[r]
i = l
for j = l to r - 1 do
  if A[ 7] < pivot then
    Swap(A, i, 7)
    i = i + l
Swap(A, i,r)
return i
\end{enumerate}

%%
% b)
%%

\item Geben Sie die asymptotische Best- und Worst-Case-Laufzeit von
\textbf{Mergesort} an.
\index{Mergesort}

\end{enumerate}
\end{document}
