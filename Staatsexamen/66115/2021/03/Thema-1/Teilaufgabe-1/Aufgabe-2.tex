\documentclass{lehramt-informatik-aufgabe}
\liLadePakete{formale-sprachen,cyk-algorithmus}
\begin{document}
\let\l=\liKurzeTabellenLinie

\liAufgabenTitel{CYK mit Wort „aaaccbbb“}
\section{Aufgabe 2
\index{CYK-Algorithmus}
\footcite{66115:2021:03}}

Sei \liGrammatik{} eine kontextfreie Grammatik mit Variablen $V =
\liMenge{S, A, B, C, D}$, Terminalzeichen \liAlphabet{a,b,c},
Produktionen

\begin{liProduktionsRegeln}
S -> A D | C C | c,
A -> a,
B -> b,
C -> C C | c,
D -> S B | C B,
\end{liProduktionsRegeln}

und Startsymbol $S$. Führen Sie den Algorithmus von Cocke, Younger und
Kasami (CYK-Algorithmus) für $G$ und das Wort $aaaccbbb$ aus. Liegt
$aaaccbbb$ in der durch $G$ erzeugten Sprache? Erläutern Sie Ihr
Vorgehen und den Ablauf des CYK-Algorithmus.

\begin{liAntwort}
\begin{tabular}{|c|c|c|c|c|c|c|c|}
a   & a   & a   & c   & c   & b   & b   & b   \\\hline\hline
-   & -   & -   & S,C & D,D & -   & -   \l7
-   & -   & -   & D,D & -   & -   \l6
-   & -   & S,S & -   & -   \l5
-   & -   & D,D & -   \l4
-   & S,S & -   \l3
-   & D,D  \l2
S,S \l1
\end{tabular}
\end{liAntwort}

\end{document}
