\documentclass{bschlangaul-aufgabe}
\liLadePakete{formale-sprachen,automaten,potenzmengen-konstruktion}
\begin{document}
\liAufgabenMetadaten{
  Titel = {Aufgabe 1},
  Thematik = {Reguläre Sprachen Automaten zuordnen},
  RelativerPfad = Staatsexamen/66115/2021/03/Thema-1/Teilaufgabe-1/Aufgabe-1.tex,
  ZitatSchluessel = examen:66115:2021:03,
  BearbeitungsStand = unbekannt,
  Korrektheit = unbekannt,
  Stichwoerter = {Reguläre Sprache},
  ExamenNummer = 66115,
  ExamenJahr = 2021,
  ExamenMonat = 03,
  ExamenThemaNr = 1,
  ExamenTeilaufgabeNr = 1,
  ExamenAufgabeNr = 1,
}

\def\z#1{
  \liZustandsMengenSammlungNr{#1}{
    {
      {0} {0}
      {1} {1}
      {2} {0,3}
      {3} {2}
      {4} {1,4}
      {5} {2,4}
    }
  }
}

Im Folgenden bezeichnet $a^i = a \dots a$ und $\varepsilon$ steht für
das leere Wort (\dh insbesondere $a^i = \varepsilon$).
\index{Reguläre Sprache}
\footcite{examen:66115:2021:03}

Die Menge $\mathbb{N}_0 = \liMenge{0,1,2,\dots}$ ist die Menge aller
nicht-negativer Ganzzahlen.

Die Sprachen $L_1, \dots , L_{12}$ seien definiert als:

\begin{enumerate}
%%
% a)
%%

\item Ordnen Sie jedem der folgenden nichtdeterministischen endlichen
Automaten $N_j, j = 1,\dots,6$, (die alle über dem Alphabet
\liAlphabet{a} arbeiten) \textbf{jeweils eine} der Sprachen $L_i \in
\liMenge{L_1, \dots , L_{12}}$ zu, sodass $L_i$, genau die von $N_i$,
\textbf{akzeptierte Sprache} ist.

\begin{liAntwort}
\begin{itemize}
\item $N_1 = L_6$ (mindestens ein $a$)

\item $N_2 = L_8$ (ungerade Anzahl an $a$’s: $1, 5, 7, \dots$)

\item $N_3 = L_2$ (gerade Anzahl an $a$’s: $2, 4, 6, \dots$)

\item $N_4 = L_{12}$ (leeres Wort)

\item $N_5 = L_8$ (ungerade Anzahl an $a$’s: $1, 5, 7, \dots$)

\item $N_6 = L_{11}$ (die Sprache akzeptiert nicht)
\end{itemize}
\end{liAntwort}

%%
% b)
%%

\item Zeigen Sie für eine der Sprachen $L_1, \dots, L_{12}$ dass diese
\textbf{nicht regulär} ist.

\begin{liAntwort}
\liAusdruck[L_10]{a^n}{n \in \mathbb{N}_0, n\text{ ist Primzahl}}

ist nicht regulär, da sich sonst jede Primzahl $p$ einer bestimmten
Mindestgröße $j$ als Summe von natürlichen Zahlen $u + v + w$ darstellen
ließe, so dass $v \geq 1$ und für alle $i \geq 0$ auch $u + iv + w = p +
(i − 1)v$ prim ist. Dies ist jedoch für $i = p + 1$ wegen $p + (p + 1 −
1)v = p(1 + v)$ nicht der
Fall.\liFussnoteUrl{https://www.informatik.hu-berlin.de/de/forschung/gebiete/algorithmenII/Lehre/ws13/einftheo/einftheo-skript.pdf}
\end{liAntwort}

%%
% c)
%%

\item Konstruieren Sie für den folgenden nichtdeterministischen
endlichen Automaten (der Worte über dem Alphabet \liAlphabet{a,b}
verarbeitet) einen äquivalenten deterministischen endlichen Automaten
mithilfe der Potenzmengenkonstruktion. Zeichnen Sie dabei nur die vom
Startzustand erreichbaren Zustände. Erläutern Sie Ihr Vorgehen.

\begin{center}
\begin{tikzpicture}[li automat]
  \node[state,initial,accepting] (z0) at (1.71cm,-4.43cm) {$z_0$};
  \node[state] (z1) at (4.86cm,-3.29cm) {$z_1$};
  \node[state,accepting] (z2) at (9.43cm,-4.29cm) {$z_2$};
  \node[state] (z3) at (4.14cm,-5.57cm) {$z_3$};
  \node[state] (z4) at (7.14cm,-5.71cm) {$z_4$};
  \node[state] (z5) at (7.71cm,-2.43cm) {$z_5$};

  \path (z0) edge[auto] node{$a$} (z1);
  \path (z0) edge[auto,bend left] node{$b$} (z3);
  \path (z0) edge[auto,loop above] node{$b$} (z0);
  \path (z1) edge[auto] node{$a$} (z2);
  \path (z1) edge[auto] node{$a,b$} (z5);
  \path (z2) edge[auto,bend left] node{$a$} (z4);
  \path (z2) edge[auto,loop above] node{$a$} (z2);
  \path (z3) edge[auto] node{$a$} (z4);
  \path (z3) edge[auto,bend left] node{$b$} (z0);
  \path (z4) edge[auto,bend left] node{$a$} (z2);
  \path (z5) edge[auto,loop above] node{$b$} (z5);
\end{tikzpicture}
\end{center}
\liFlaci{Af7iooyca}

\begin{liAntwort}
\begin{tabular}{l|l|l}
Zustandsmenge & Eingabe $a$ & Eingabe $b$ \\\hline
\z0 & \z1 & \z2 \\
\z1 & \z3 & $Z_T$ \\
\z2 & \z4 & \z2 \\
\z3 & \z5 & $Z_T$ \\
\z4 & \z3 & $Z_T$ \\
\z5 & \z5 & $Z_T$ \\
\end{tabular}

\begin{liAntwort}
\begin{center}
\begin{tikzpicture}[li automat,scale=0.6,transform shape]
  \node[state,initial,accepting] (Z0) at (3.43cm,-9.14cm) {\z0};
  \node[state] (Z1) at (6.86cm,-11cm) {\z1};
  \node[state,accepting] (Z2) at (3.57cm,-1.29cm) {\z2};
  \node[state,accepting] (Z5) at (9.71cm,-6.86cm) {\z5};
  \node[state] (ZT) at (13.86cm,-6.57cm) {$Z_T$};
  \node[state] (Z4) at (6.86cm,-1.29cm) {\z4};
  \node[state,accepting] (Z3) at (6.86cm,-3.71cm) {\z3};

  \path (Z0) edge[auto] node{$a$} (Z1);
  \path (Z0) edge[auto] node{$b$} (Z2);
  \path (Z1) edge[auto] node{$a$} (Z3);
  \path (Z1) edge[auto] node{$b$} (ZT);
  \path (Z2) edge[auto] node{$a$} (Z4);
  \path (Z2) edge[auto,loop above] node{$b$} (Z2);
  \path (Z5) edge[auto,loop left] node{$a$} (Z5);
  \path (Z5) edge[auto] node{$b$} (ZT);
  \path (ZT) edge[auto,loop above] node{$a,b$} (ZT);
  \path (Z4) edge[auto] node{$a$} (Z3);
  \path (Z4) edge[auto] node{$b$} (ZT);
  \path (Z3) edge[auto] node{$a$} (Z5);
  \path (Z3) edge[auto] node{$b$} (ZT);
\end{tikzpicture}
\end{center}
\liFlaci{Apkyuo4ja}
\end{liAntwort}

\end{liAntwort}

\end{enumerate}
\end{document}
