\documentclass{lehramt-informatik-aufgabe}
\liLadePakete{formale-sprachen}
\begin{document}

\liAufgabenTitel{Reguläre Sprachen Automaten zuordnen}
\section{Aufgabe 1
\index{Reguläre Sprache}
\footcite{66115:2021:03}}

Im Folgenden bezeichnet $a^i = a \dots a$ und $\varepsilon$ steht für
das leere Wort (\dh insbesondere $a^i = \varepsilon$).

Die Menge $\mathbb{N}_0 = \liMenge{0,1,2,\dots}$ ist die Menge aller
nicht-negativer Ganzzahlen.

Die Sprachen $L_1, \dots , L_{12}$ seien definiert als:

\begin{enumerate}
%%
% a)
%%

\item Ordnen Sie jedem der folgenden nichtdeterministischen endlichen
Automaten $N_j, j = 1,\dots,6$, (die alle über dem Alphabet
\liAlphabet{a} arbeiten) \textbf{jeweils eine} der Sprachen $L_i \in
\liMenge{L_1, \dots , L_{12}}$ zu, sodass $L_i$, genau die von $N_i$,
\textbf{akzeptierte Sprache} ist.

\begin{liAntwort}
\begin{itemize}
\item $N_1 = L_6$ (mindestens ein $a$)

\item $N_2 = L_8$ (ungerade Anzahl an $a$’s: $1, 5, 7, \dots$)

\item $N_3 = L_2$ (gerade Anzahl an $a$’s: $2, 4, 6, \dots$)

\item $N_4 = L_{12}$ (leeres Wort)

\item $N_5 = L_8$ (ungerade Anzahl an $a$’s: $1, 5, 7, \dots$)

\item $N_6 = L_11$ (die Sprache akzeptiert nicht)
\end{itemize}
\end{liAntwort}

%%
% b)
%%

\item Zeigen Sie für eine der Sprachen $L_1, \dots, L_{12}$ dass diese
\textbf{nicht regulär} ist.

\begin{liAntwort}
\liAusdruck[L_10]{a^n}{n \in \mathbb{N}_0, n\text{ ist Primzahl}}

ist nicht regulär, da sich sonst jede Primzahl $p$ einer bestimmten
Mindestgröße $j$ als Summe von natürlichen Zahlen $u + v + w$ darstellen
ließe, so dass $v \geq 1$ und für alle $i \geq 0$ auch $u + iv + w = p +
(i − 1)v$ prim ist. Dies ist jedoch für $i = p + 1$ wegen $p + (p + 1 −
1)v = p(1 + v)$ nicht der
Fall.\liFussnoteUrl{https://www.informatik.hu-berlin.de/de/forschung/gebiete/algorithmenII/Lehre/ws13/einftheo/einftheo-skript.pdf}
\end{liAntwort}

%%
% c)
%%

\item Konstruieren Sie für den folgenden nichtdeterministischen
endlichen Automaten (der Worte über dem Alphabet \liAlphabet{a,b}
verarbeitet) einen äquivalenten deterministischen endlichen Automaten
mithilfe der Potenzmengenkonstruktion. Zeichnen Sie dabei nur die vom
Startzustand erreichbaren Zustände. Erläutern Sie Ihr Vorgehen.

\begin{liAntwort}

\end{liAntwort}

\end{enumerate}
\end{document}
