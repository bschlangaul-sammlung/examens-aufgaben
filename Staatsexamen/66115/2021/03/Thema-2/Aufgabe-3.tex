\documentclass{lehramt-informatik-aufgabe}
\liLadePakete{}
\begin{document}
\liAufgabenTitel{}
\section{Aufgabe 3
\index{Breitensuche}
\footcite{66115:2021:03}}

Wir betrachten eine Variante der Breitensuche (BFS), bei der die Knoten
markiert werden, wenn sie das erste Mal besucht werden. Außerdem wird
die Suche einmal bei jedem unmarkierten Kno- ten gestartet, bis alle
Knoten markiert sind. Wir betrachten gerichtete Graphen. Ein gerichteter
Graph G ist schwach zusammenhängend, wenn der ungerichtete Graph (der
sich daraus ergibt, dass man die Kantenrichtungen von G ignoriert)
zusammenhängend ist.

\begin{enumerate}
%%
% a)
%%

\item Beschreiben Sie für ein allgemeines n €e N mit n > 2 den Aufbau
eines schwach zusam- menhängenden Graphen G, mit n Knoten, bei dem die
Breitensuche ®(n) mal gestartet werden muss, bis alle Knoten markiert
sind.

%%
% b)
%%

\item Welche asymptotische Laufzeit in Abhängigkeit von der Anzahl der
Knoten (n) und von _ der Anzahl der Kanten (m) hat die Breitensuche über
alle Neustarts zusammen? Beachten Sie, dass die Markierungen nicht
gelöscht werden. Geben Sie die Laufzeit in 8-Notation an. Begründen Sie
Ihre Antwort.
\end{enumerate}
\end{document}
