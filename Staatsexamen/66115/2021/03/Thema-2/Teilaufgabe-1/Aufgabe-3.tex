\documentclass{bschlangaul-aufgabe}
\liLadePakete{formale-sprachen}
\begin{document}
\liAufgabenMetadaten{
  Titel = {Aufgabe 3},
  Thematik = {Turingmaschine M von w},
  RelativerPfad = Staatsexamen/66115/2021/03/Thema-2/Teilaufgabe-1/Aufgabe-3.tex,
  ZitatSchluessel = examen:66115:2021:03,
  BearbeitungsStand = unbekannt,
  Korrektheit = unbekannt,
  Stichwoerter = {Entscheidbarkeit},
  ExamenNummer = 66115,
  ExamenJahr = 2021,
  ExamenMonat = 03,
  ExamenThemaNr = 2,
  ExamenTeilaufgabeNr = 1,
  ExamenAufgabeNr = 3,
}

Wir betrachten eine Gödelisierung von Turingmaschinen und bezeichnen mit
$M_w$ die Turingmaschine, die gemäß der Kodierung des Binärworts $w$
kodiert wird. Außerdem bezeichnen wir mit $M_w(x)$ die Ausgabe der
Maschine $M_w$ bei Eingabe $x$. Sie dürfen davon ausgehen, dass $x$
immer ein Binärstring ist. Der bekannte Satz von Rice sagt:
\index{Entscheidbarkeit}
\footcite{examen:66115:2021:03}

Sei S eine Menge berechenbarer Funktionen mit $\emptyset \neq S \neq
\mathcal{R}$, wobei $\mathcal{R}$ die Menge aller berechenbaren
Funktionen ist. Dann ist die Sprache \liAusdruck{w}{f_{M_w} \in S}
unentscheidbar.

Hier ist $f_{M_w}$ die von $M_w$ berechnete Funktion.

Zeigen Sie für jede der nachfolgenden Sprachen über dem Alphabet
\liMenge{0,1} entweder, dass sie entscheidbar ist, oder zeigen Sie mit
Hilfe des Satzes von Rice, dass sie unentscheidbar ist. Geben Sie beim
Beweis der Unentscheidbarkeit die Menge $S$ der berechenbaren Funktionen
an, auf die Sie den Satz von Rice anwenden. Wir bezeichnen die Länge der
Eingabe $x$ mit $|x|$.

\begin{enumerate}

%%
% a)
%%

\item \liAusdruck{w}{M_w\text{ akzeptiert die Binarkodierungen der
Primzahlen (und lehnt alles andere ab)}}

%%
% b)
%%

\item \liAusdruck{w}{\text{es gibt eine Hingabe }x\text{, so dass
}M_w(x)\text{ das Symbol }1\text{ enthält}}

%%
% c)
%%

\item \liAusdruck{w}{M_w(x)\text{ hält für jedes }x\text{ mit }|x| <
1000\text{ nach höchstens }100\text{ Schritten an}}

%%
% d)
%%

\item \liAusdruck{w}{M_w\text{ hat für jede Eingabe dieselbe Ausgabe}}

%%
% e)
%%

\item \liAusdruck{w}{\text{die Menge der Eingaben, die von }M_w\text{
akzeptiert werden, ist endlich}}

\end{enumerate}
\end{document}
