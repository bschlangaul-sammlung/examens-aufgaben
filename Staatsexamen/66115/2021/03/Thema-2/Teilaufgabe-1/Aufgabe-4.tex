\documentclass{lehramt-informatik-aufgabe}
\liLadePakete{komplexitaetstheorie}
\begin{document}
\liAufgabenTitel{CLIQUE ALMOSTCLIQUE}
\section{Aufgabe 4
\index{Polynomialzeitreduktion}
\footcite{66115:2021:03}}

 (Komplexität) [30 PUNKTE]

Betrachten Sie die folgenden Probleme:

\liProblemBeschreibung
{Clique}
{Ein ungerichteter Graph $G = (V, E)$, eine Zahl $k \in \mathcal{N}$}
{Gibt es eine Menge $S \subseteq V$ mit $|S| = k$, sodass für alle Knoten $u \neq v \in V$ gilt,
dass $\{ u, v \}$ eine Kante in $E$ ist?}

\liProblemBeschreibung
{Almost Clique}
{Ein ungerichteter Graph G = (V, E), eine Zahl k e N}
{Gibt es eine Menge S C V mit |S| = k, sodass die Anzahl der Kanten zwischen
Knoten in S genau ke) — 1 ist?}

Zeigen Sie, dass das Problem ALMOSTCLIQUE NP-vollständig ist. Nutzen Sie
dafür die NP- Vollständigkeit von CLIQUE.

Hinweis: Die Anzahl der Kanten einer k-Clique sind AD,

\end{document}
