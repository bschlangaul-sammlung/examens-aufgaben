\documentclass{lehramt-informatik-aufgabe}
\liLadePakete{formale-sprachen,pumping-lemma}
\begin{document}
\liAufgabenTitel{w w_1 w w_2}
\section{Aufgabe 2
\index{Kontextfreie Sprache}
\footcite{66115:2021:03}}
\begin{enumerate}

%%
% a)
%%

\item Zeigen Sie, dass die Sprache

\begin{center}
\liAusdruck{w w_1 w w_2}{w, w_1, w_2
\in \{ a,b,c \}^* \text{ und } 2|w| \geq |w_1| + |w_2]}
\end{center}

nicht kontextfrei
ist. \index{Pumping-Lemma (Kontextfreie Sprache)}

\begin{liExkurs}[Pumping-Lemma für Kontextfreie Sprachen]
\liPumpingKontextfrei
\end{liExkurs}

\begin{liAntwort}

\end{liAntwort}

%%
% b)
%%

\item Betrachten Sie die Aussage

\centerline{Seien $L_1, \dots, I_n$ beliebige kontextfreie Sprachen.}

\centerline{Dann ist $\bigcap_{i=1}^n, L_i$ immer eine entscheidbare
Sprache.}

Entscheiden Sie, ob diese Aussage wahr ist oder nicht und begründen Sie
Ihre Antwort.

\begin{liAntwort}
Diese Aussage ist falsch.

Kontextfreie Sprachen sind nicht abgeschlossen unter dem Schnitt, \dh
die Schnittmenge zweier kontextfreier Sprachen kann in einer Sprache
eines anderen Typs in der Chomsky Sprachen-Hierachie resultieren.
Entsteht durch den Schnitt eine Typ-0-Sprache, dann ist diese nicht
entscheidbar.
\end{liAntwort}

%%
% c)
%%

\item Sei $\mathbb{N}_0 = \liMenge{0,1,2,\dots}$ die Menge der nicht
negativen natürlichen Zahlen. Es ist bekannt, dass \liAusdruck{a^n b^n
c^n}{n \in \mathbb{N}} keine kontextfreie Sprache ist. Ist die
Komplementsprache $L_5 = \{a, b, c \}^* \setminus \, L$ kontextfrei?
Begründen Sie Ihre Antwort.

\begin{liAntwort}

\end{liAntwort}

\end{enumerate}
\end{document}
