\documentclass{lehramt-informatik-aufgabe}
\liLadePakete{syntax,formale-sprachen,automaten}
\begin{document}
\liAufgabenTitel{Reguläre Sprachen}
\section{Aufgabe 1
\index{Reguläre Sprache}
\footcite{66115:2021:03}}

\begin{enumerate}

%%
% a)
%%

\item Sei

\liAusdruck[L_1]{w \in \{ a, b, c \}^* }
{w
\text{ enthält genau zweimal den Buchstaben }
a
\text{ und der vorletzte Buchstabe
ist ein }
c}

Geben Sie einen regulären Ausdruck für die Sprache $L_1$ an.

\begin{liAntwort}
\begin{minted}{md}
(
  ((b|c)* a (b|c)* a (b|c)* c (b|c))
  |
  ((b|c)* a (b|c)* c a)
)
\end{minted}
\end{liAntwort}

%%
% b)
%%

\item Konstruieren Sie einen deterministischen endlichen Automaten für
die Sprache $L_2$:

\liAusdruck[L_2]{w \in \{ a, b \}^* }
{w
\text{ enthält genau einmal das Teilwort }
aab}

\begin{liAntwort}
\begin{center}
\begin{tikzpicture}[li automat]
  \node[state,initial] (z0) at (3.14cm,-4cm) {$z_0$};
  \node[state] (z1) at (5.29cm,-4cm) {$z_1$};
  \node[state] (z2) at (8.29cm,-4cm) {$z_2$};
  \node[state,accepting] (z3) at (10.86cm,-4cm) {$z_3$};
  \node[state] (z4) at (3.14cm,-2cm) {$z_4$};
  \node[state] (z5) at (7.29cm,-6.43cm) {$z_5$};
  \node[state] (z6) at (4cm,-6.43cm) {$z_6$};

  \path (z0) edge[auto,bend left] node{$b$} (z4);
  \path (z0) edge[auto] node{$a$} (z1);
  \path (z1) edge[auto] node{$a$} (z2);
  \path (z1) edge[auto] node{$b$} (z6);
  \path (z2) edge[auto] node{$b$} (z3);
  \path (z2) edge[auto] node{$a$} (z5);
  \path (z3) edge[auto,loop above] node{$a,b$} (z3);
  \path (z4) edge[auto,bend left] node{$a$} (z0);
  \path (z4) edge[auto,loop above] node{$b$} (z4);
  \path (z5) edge[auto] node{$a$} (z1);
  \path (z5) edge[auto] node{$b$} (z6);
  \path (z6) edge[auto,loop above] node{$a,b$} (z6);
\end{tikzpicture}
\end{center}
\liFlaci{Ahf2oduri}
\end{liAntwort}

%%
% c)
%%

\item Sei $\mathbb{N} = \liMenge{1,2,3,\dots}$ die Menge der strikt
positiven natürlichen Zahlen. Sei

\liAusdruck[L_3]
%
{\# a^{i_1} \# a^{i_1} \# \cdots a^{i_{n-1}} \# a^{i_n} \#}
%
{n,i_1,\dots,i_n \in \mathbb{N}
\text{ und es existiert }
j \in \mathbb{N}
\text{ mit }
i_j = n + 1}

eine Sprache über Alphabet \liMenge{\#, a}.

So ist \zB $\#a\#aaa\# \in L_3$ (da das Teilwort $a^3 = aaa$ vorkommt)
und $\#a\#a\#a\#a\# \in L_3$ (da das Teilwort $a^5 = aaaaa$ nicht
vorkommt). Beweisen Sie, dass $L_3$ nicht regulär ist.

\end{enumerate}
\end{document}
