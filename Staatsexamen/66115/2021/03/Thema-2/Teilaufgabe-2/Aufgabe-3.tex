\documentclass{lehramt-informatik-aufgabe}
\liLadePakete{}
\begin{document}
\liAufgabenTitel{Binärbäume}
\section{Aufgabe 3
\index{Binärbaum}
\footcite{66115:2021:03}}

\begin{enumerate}
\item Betrachten Sie folgenden Binärbaum T.

Geben Sie die Schliissel der Knoten in der Reihenfolge an, wie sie von
einem Preorder- Durchlauf (= TreeWalk) von T ausgegeben werden.

\item Betrachten Sie folgende Sequenz als Ergebnis eines
Preorder-Durchlaufs eines binären Suchbaumes T. Zeichnen Sie T und
erklären Sie, wie Sie zu Ihrer Schlussfolgerung gelangen.

[8,7,4,2,1,3,5,6,10,9,11]

Hinweis: Welcher Schlüssel ist die Wurzel von 7’? Welche Knoten sind in
seinem linken /rech- ten Teilbaum gespeichert? Welche Schlüssel sind die
Wurzeln der jeweiligen Teilbäume?

\item Anstelle von sortierten Zahlen soll ein Baum nun verwendet werden,
um relative Positions- angaben zu speichern. Jeder Baumknoten enthält
eine Beschriftung und einen Wert (vgl. Abb. 1), der die ganzzahlige
relative Verschiebung in horizontaler Richtung gegenüber sei- nem
Elternknoten angibt. Die zu berechnenden Koordinaten für einen Knoten
ergeben sich

aus seiner Tiefe im Baum als y-Wert und aus der Summe aller
Verschiebungen auf dem Pfad zur Wurzel als x-Wert. Das Ergebnis der
Berechnung ist in Abb. 2 visualisiert. Geben Sie einen Algorithmus mit
linearer Laufzeit in Pseudo-Code oder einer objektorientierten
Programmiersprache Ihrer Wahl an. Der Algorithmus erhält den Zeiger auf
die Wurzel ei- nes Baumes als Eingabe und soll Tupel mit den berechneten
Koordination aller Knoten des Baums in der Form (Beschriftung, x, y)
zurück- oder ausgeben.

\end{enumerate}
\end{document}
