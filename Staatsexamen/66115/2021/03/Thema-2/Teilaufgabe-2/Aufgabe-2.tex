\documentclass{lehramt-informatik-aufgabe}
\liLadePakete{syntax}
\begin{document}
\liAufgabenTitel{DoubleLinkedList}
\section{Aufgabe 2
\index{doppelt-verkettete Liste}
\footcite{66115:2021:03}}

Gegeben sei die folgende Java-Implementierung einer doppelt-verketteten
Liste.

\liJavaExamen{66115}{2021}{03}{DoubleLinkedList}

\begin{enumerate}

%%
% a)
%%

\item Skizzieren Sie den Zustand der Datenstruktur nach Aufruf der
folgenden Befehlssequenz. Um Variablen mit Zeigern auf Objekte
darzustellen, können Sie mit dem Variablennamen beschriftete Pfeile
verwenden.

\begin{minted}{java}
DoubleLinkedList list = new DoubleLinkedList ();
list.append("a");
list.append("b");
list.append("c");
\end{minted}

%%
%
%%

\item Implementieren Sie in der Klasse DoubleLinkedList die Methode
search, die zu einem gegebenen Wert das Item der Liste mit dem
entsprechenden Wert, oder null falls der Wert nicht in der Liste
enthalten ist, zurückgibt.

%%
%
%%

\item Implementieren Sie in der Klasse DoubleLinkedList die Methode
delete, die das erste Vorkommen eines Wertes aus der Liste entfernt. Ist
der Wert nicht in der Liste enthalten, terminiert die Methode
„stillschweigend“, d. h. ohne Änderung der Liste und ohne Fehler-
meldung. Sie dürfen die Methode search aus Teilaufgabe b) verwenden,
auch wenn Sie sie nicht implementiert haben.

%%
%
%%

\item Beschreiben Sie die notwendigen Änderungen an der Datenstruktur
und an den bisherigen Implementierungen, um eine Methode size, die die
Anzahl der enthaltenen Items zurück gibt, mit Laufzeit O(1) zu
realisieren.

\end{enumerate}
\end{document}
