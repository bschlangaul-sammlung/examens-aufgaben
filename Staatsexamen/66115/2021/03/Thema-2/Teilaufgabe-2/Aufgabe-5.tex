\documentclass{lehramt-informatik-aufgabe}
\liLadePakete{}
\begin{document}
\liAufgabenTitel{Hashing}
\section{Aufgabe 5
\index{Streutabellen (Hashing)}
\footcite{66115:2021:03}}
\begin{enumerate}

%%
% a)
%%

\item Nennen Sie zwei wünschenswerte Eigenschaften von Hashfunktionen.

%%
% b)
%%

\item Wie viele Elemente können bei Verkettung und wie viele Elemente
können bei offener Adressierung in einer Hashtabelle mit m Zeilen
gespeichert werden?

%%
% c)
%%

\item Angenommen, in einer Hashtabelle der Größe m sind alle Einträge (mit mindestens einem
Wert) belegt und insgesamt n Werte abgespeichert.

Geben Sie in Abhängigkeit von m und n an, wie viele Elemente bei der Suche nach einem
nicht enthaltenen Wert besucht werden müssen. Sie dürfen annehmen, dass jeder Wert mit
gleicher Wahrscheinlichkeit und unabhängig von anderen Werten auf jeden der m Plätze
abgebildet wird (einfaches gleichmäßiges Hashing).

%%
% d)
%%

\item Betrachten Sie die folgende Hashtabelle mit der Hashfunktion h(x) = x mod 11. Hierbei
steht @ für eine Zelle, in der kein Wert hinterlegt ist.

Führen Sie nun die folgenden Operationen mit offener Adressierung mit linearem Sondieren
aus und geben Sie den Zustand der Datenstruktur nach jedem Schritt an. Werden für eine
Operation mehrere Zellen betrachtet, aber nicht modifiziert, so geben Sie deren Indizes in
der betrachteten Reihenfolge an.

\begin{enumerate}

%%
% i)
%%

\item Insert 7
%%
% ii)
%%

\item Insert 20
%%
% iii)
%%

\item Delete 18
%%
% iv)
%%

\item Search 7

%%
% v)
%%

\item Insert 5
\end{enumerate}

\end{enumerate}
\end{document}
