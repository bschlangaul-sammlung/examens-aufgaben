\documentclass{lehramt-informatik-aufgabe}
\InformatikPakete{syntax}
\begin{document}

\section{Aufgabe 3: Hashing
\index{Streutabellen (Hashing)}
\footcite[Thema 2 Aufgabe 3 Seite 4]{examen:66115:2010:09}
}

Gegegen sei ein Array der Größe 10, z.\,B. \java{int[] hashfeld = new
int [10]}. Die Hashfunktion sei der Wert modulo 10, $h(x) = x \% 10$.
Kollisionen werden mit linearer Verschiebung um 1 (modulo 10) gelöst.

\java{in(x)} bedeutet, dass die Zahl x eingefügt wird,
\java{search(x)}, dass nach x gesucht wird mit den Antworten „ja“ bzw. „nein“ und
\java{out(x)}, dass x gelöscht wird, sofern x gespeichert ist.

Es wird folgende Sequenz von Operationen auf ein anfangs leeres Array
ausgeführt:

\java{in(19)},
\java{in(29)},
\java{in(39)},
\java{in(10)},
\java{out(29)},
\java{out(39)},
\java{search(29)},
\java{in(11)},
\java{in(17)},
\java{out(10)},
\java{in(2)},
\java{in(22)}

Geben Sie den Inhalt von \java{hashfeld} an

nach \java{search(29)}\\
nach \java{out(10)}\\
und nach \java{in(22)}.
\end{document}
