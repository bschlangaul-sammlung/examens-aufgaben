\documentclass{bschlangaul-aufgabe}
\bLadePakete{syntax}
\begin{document}
\bAufgabenTitel{Hashing mit Modulo 10}

\section{Aufgabe 3: Hashing
\index{Streutabellen (Hashing)}
\footcite[Thema 2 Aufgabe 3 Seite 4]{examen:66115:2010:09}
}

Gegegen sei ein Array der Größe 10, \zB \bJavaCode{int[] hashfeld = new
int [10]}. Die Hashfunktion sei der Wert modulo 10, $h(x) = x \% 10$.
Kollisionen werden mit linearer Verschiebung um 1 (modulo 10) gelöst.

\bJavaCode{in(x)} bedeutet, dass die Zahl x eingefügt wird, \bJavaCode{search(x)},
dass nach x gesucht wird mit den Antworten „ja“ bzw. „nein“ und
\bJavaCode{out(x)}, dass x gelöscht wird, sofern x gespeichert ist.

Es wird folgende Sequenz von Operationen auf ein anfangs leeres Array
ausgeführt:

\bJavaCode{in(19)},
\bJavaCode{in(29)},
\bJavaCode{in(39)},
\bJavaCode{in(10)},
\bJavaCode{out(29)},
\bJavaCode{out(39)},
\bJavaCode{search(29)},
\bJavaCode{in(11)},
\bJavaCode{in(17)},
\bJavaCode{out(10)},
\bJavaCode{in(2)},
\bJavaCode{in(22)}

Geben Sie den Inhalt von \bJavaCode{hashfeld} an

nach \bJavaCode{search(29)}\\
nach \bJavaCode{out(10)}\\
und nach \bJavaCode{in(22)}.
\end{document}
