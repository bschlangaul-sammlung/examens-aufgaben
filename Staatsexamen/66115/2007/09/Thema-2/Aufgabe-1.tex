\documentclass{lehramt-informatik-aufgabe}
\liLadePakete{formale-sprachen,mathe,automaten}

\begin{document}
\let\z=\liZustandsmengeOhneMathe
\let\Z=\liZustandsmenge
\let\a=\liAlphabet
\let\f=\liUeberfuehrungsFunktionOhneMathe
\let\F=\liUeberfuehrungsFunktion

\liAufgabenTitel{Reguläre Sprache}
\section{Aufgabe 1
\index{Reguläre Sprache}
\footcite{66115:2007:09}}

Gegeben sei der nichtdeterministische endliche Automat $M$ mit dem
Alphabet \a{a,b}, der Zustandsmenge \Z{z0, z1, z2, z3}, Anfangszustand
$z_0$, Endzustand \Z{z3} und der Überführungsfunktion $\delta$ mit:

\begin{align*}
\f{z0, a} &= \z{z1, z2},\\
\f{z1, b} &= \z{z0, z1},\\
\f{z2, a} &= \z{z2, z3},\\
\f{z0, b} &= \f{z1, a} = \f{z2, b} = \f{z3, a} = \f{z3, b} = \emptyset\\
\end{align*}

\begin{liAntwort}
\begin{center}
\begin{tikzpicture}[->,node distance=4cm]
\node[state,initial]
(0) {$z_0$};

\node[state,right of=0]
(1) {$z_1$};

\node[state,below of=0]
(2) {$z_2$};

\node[state,right of=2,accepting]
(3) {$z_3$};

\node[state,below right= 1.5cm of 0]
(t) {$z_t$};

\path (0) edge[above,bend left] node{a} (1);
\path (0) edge[left] node{a} (2);

\path (1) edge[above,bend left] node{b} (0);
\path (1) edge[above,loop right] node{b} (1);

\path (2) edge[above,loop left] node{a} (2);
\path (2) edge[above] node{a} (3);

\path (0) edge[below left] node{b} (t);
\path (1) edge[below right] node{a} (t);
\path (2) edge[above left] node{b} (t);
\path (3) edge[above right] node{a,b} (t);
\end{tikzpicture}
\liFussnoteUrl{https://flaci.com/Afybo27zc}
\end{center}
\end{liAntwort}

\noindent
$L(M)$ sei die von $M$ akzeptierte Sprache.

\begin{enumerate}

%%
% a)
%%

\item Gelten folgende Aussagen?

\begin{enumerate}

%%
% i)
%%

\item Es gibt Zeichenreihen in $L(M)$, die genauso viele $a$’s enthalten
wie $b$’s.

\begin{liAntwort}
Ja, zum Beispiel das Wort $abbbaa$ oder $abbbbaaa$. Mit der
Überführungsfunktion $\f{z1, b} = \z{ z1 }$ können beliebig viele $b$’s
akzeptiert werden, sodass die Anzahl von $a$’s und $b$’s ausgeglichen
werden kann.
\end{liAntwort}

%%
% ii)
%%

\item Jede Zeichenreihe in $L(M)$, die mindestens vier $b$’s enthält,
enthält auch mindestens vier $a$’s.

\begin{liAntwort}
Nein, z. b. das Wort $abbbbaa$ wird akzeptiert. Ein Wort muss nur
mindestens drei $a$’s enthalten. Mit der Überführungsfunktion $\f{z1, b}
= \z{ z1 }$ können aber beliebig viele $b$’s akzeptiert werden.
\end{liAntwort}

\end{enumerate}

Begründen Sie Ihre Antworten.

%%
% b)
%%

\item Geben Sie eine reguläre (Typ-3-)Grammatik an, die $L(M)$ erzeugt.
\index{Reguläre Grammatik}

%%
% c)
%%

\item Beschreiben Sie $L(M)$ durch einen regulären Ausdruck.
\index{Regulärer Ausdruck}

\begin{liAntwort}
(ab+)*aa+
\end{liAntwort}

%%
% d)
%%

\item Konstruieren Sie aus M mit der Potenzmengen-Konstruktion (und
entsprechender Begründung) einen deterministischen endlichen Automaten,
der $L(M)$ akzeptiert.
\index{Potenzmengenalgorithmus}

\end{enumerate}
\end{document}
