\documentclass{lehramt-informatik-aufgabe}
\liLadePakete{syntax}
\begin{document}
\liAufgabenTitel{Klassen „QueueElement“ und „Queue“}

\section{Aufgabe zu Queues
\index{Warteschlange (Queue)}
\footcite[Seite 1-2, Aufgabe 2: Queue Frühjahr 2007, Thema 1 Aufgabe 7]{aud:pu:4}
}

Implementieren Sie die angegebenen Methoden einer Klasse \liJavaCode{Queue}
für Warteschlangen. Eine Warteschlange soll eine unbeschränkte Anzahl
von Elementen aufnehmen können. Elemente sollen am Ende der
Warteschlange angefügt und am Anfang aus ihr entfernt werden. Sie können
davon ausgehen, dass ein Klasse \liJavaCode{QueueElement} mit der folgenden
Schnittstelle bereits implementiert ist
\footcite[Thema 1 Aufgabe 7 Seite 3]{examen:66115:2007:03}.

\liJavaDatei[firstline=3]{aufgaben/aud/examen_66115_2007_03/QueueElement}

\noindent
Von der Klasse \liJavaCode{Queue} ist folgendes gegeben:

\liJavaDatei[firstline=3,lastline=5]{aufgaben/aud/examen_66115_2007_03/Queue}

\begin{enumerate}
\item Schreiben Sie eine Methode \liJavaCode{void append (Object contents)},
die ein neues Objekt in der Warteschlange einfügt.

\begin{antwort}
\liJavaDatei[firstline=7,lastline=17]{aufgaben/aud/examen_66115_2007_03/Queue}
\end{antwort}

\item Schreiben Sie eine Methode \liJavaCode{Object remove()}, die ein Element
aus der Warteschlange entfernt und dessen Inhalt zurückliefert.
Berücksichtigen Sie, dass die Warteschlange leer sein könnte.

\begin{antwort}
\liJavaDatei[firstline=20,lastline=30]{aufgaben/aud/examen_66115_2007_03/Queue}
\end{antwort}

\item Schreiben Sie eine Methode \liJavaCode{boolean isEmpty()}, die
überprüft, ob die Warteschlange leer ist.

\begin{antwort}
\liJavaDatei[firstline=32,lastline=34]{aufgaben/aud/examen_66115_2007_03/Queue}
\end{antwort}

\end{enumerate}
\end{document}
