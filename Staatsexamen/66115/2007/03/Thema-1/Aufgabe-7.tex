\documentclass{lehramt-informatik-aufgabe}
\liLadePakete{syntax}
\begin{document}
\liAufgabenTitel{Klassen „QueueElement“ und „Queue“}

\section{Aufgabe 7
\index{Warteschlange (Queue)}
\footcite[Seite 1-2, Aufgabe 2]{aud:pu:4}
}

Implementieren Sie die angegebenen Methoden einer Klasse
\liJavaCode{Queue} für Warteschlangen. Eine Warteschlange soll eine
unbeschränkte Anzahl von Elementen aufnehmen können. Elemente sollen am
Ende der Warteschlange angefügt und am Anfang aus ihr entfernt werden.
Sie können davon ausgehen, dass ein Klasse \liJavaCode{QueueElement} mit
der folgenden Schnittstelle bereits implementiert ist \footcite[Thema 1
Aufgabe 7 Seite 3]{examen:66115:2007:03}.
\index{Implementierung in Java}

\liJavaExamen{66115}{2007}{03}{queue/QueueElement}

\noindent
Von der Klasse \liJavaCode{Queue} ist folgendes gegeben:

\liJavaExamen[firstline=3,lastline=5]{66115}{2007}{03}{queue/Queue}

\begin{enumerate}

%%
% a)
%%

\item Schreiben Sie eine Methode \liJavaCode{void append (Object contents)},
die ein neues Objekt in der Warteschlange einfügt.

\begin{liAntwort}
\liJavaExamen[firstline=7,lastline=17]{66115}{2007}{03}{queue/Queue}
\end{liAntwort}

%%
% b)
%%

\item Schreiben Sie eine Methode \liJavaCode{Object remove()}, die ein
Element aus der Warteschlange entfernt und dessen Inhalt zurückliefert.
Berücksichtigen Sie, dass die Warteschlange leer sein könnte.

\begin{liAntwort}
\liJavaExamen[firstline=20,lastline=30]{66115}{2007}{03}{queue/Queue}
\end{liAntwort}

%%
% c)
%%

\item Schreiben Sie eine Methode \liJavaCode{boolean isEmpty()}, die
überprüft, ob die Warteschlange leer ist.

\begin{liAntwort}
\liJavaExamen[firstline=32,lastline=34]{66115}{2007}{03}{queue/Queue}
\end{liAntwort}

\end{enumerate}

\liPseudoUeberschrift{Klasse Queue}

\liJavaExamen{66115}{2007}{03}{queue/Queue}

\liPseudoUeberschrift{Tests}

\liJavaTestDatei{examen/examen_66115/jahr_2007/fruehjahr/queue/QueueTest.java}

\end{document}
