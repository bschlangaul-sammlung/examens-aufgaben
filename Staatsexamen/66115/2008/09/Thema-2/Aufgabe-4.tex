\documentclass{bschlangaul-aufgabe}
\bLadePakete{formale-sprachen}
\begin{document}
\bAufgabenTitel{Automaten und formale Sprachen}
\section{Aufgabe 4
\index{Theoretische Informatik}
\footcite{examen:66115:2008:09}}

Gesucht ist die Menge $L$ aller Dezimalzahlen über \bAlphabet{0,1,2,3}
(mit führenden Nullen), die durch 2 oder (logisches oder) durch 3
teilbar sind.

Beschreiben Sie L durch einen Automaten oder durch eine Grammatik.

In welcher Klasse der Chomsky-Hierarchie liegt L? Geben Sie die
kleinstmögliche Klasse der Chomsky-Hierarchie an.
\end{document}
