\documentclass{lehramt-informatik-aufgabe}
\liLadePakete{syntax,mathe}
\begin{document}
\liAufgabenTitel{Klasse „LeftFactorial“ und Methode „lfBig()“}

\section{Aufgabe 1: „Rekursion und Induktion“
\index{Rekursion}
\index{Vollständige Induktion}
\footcite[Thema 1 Aufgabe 1b Seite 2-3]{examen:66115:2014:03}}

\begin{enumerate}
\item Gegeben sei die Methode \verb|BigInteger lfBig(int n)| zur
Berechnung der eingeschränkten Linksfakultät:\footcite[Seite 25]{aud:fs:1}

\begin{minted}{java}
import java.math.Biginteger;
import static java.math.BigInteger.*;

public class LeftFactorial {
  // returns the left factorial !n
  BigInteger lfBig(int n) {
    if (n <= 0 || n >= Short.MAX_VALUE) {
      return ZERO;
    } else if (n == 1) {
    return ONE;
    } else {
      return sub(mul(n, lfBig(n - 1)), mul(n - 1, lfBig(n - 2)));
    }
  }
}
\end{minted}

Implementieren Sie unter Verwendung des Konzeptes der \emph{dynamischen
Programmierung} die Methode \verb|BigInteger dpBig(int n)|, die jede
$!n$
auch bei mehrfachem Aufrufen mit dem gleichen Parameter höchstens einmal
rekursiv berechnet. Sie dürfen der Klasse \verb|LeftFactorial| genau ein
Attribut beliebigen Datentyps hinzufügen und die in \verb|lfBig(int)|
verwendeten Methoden und Konstanten ebenfalls nutzen.

\item Betrachten Sie nun die Methode \verb|lfLong(int)| zur Berechnung
der vorangehend definierten Linksfakultät ohne obere Schranke. Nehmen
Sie im Folgenden an, dass der Datentyp \verb|long| unbeschränkt ist und
daher kein Überlauf auftritt.

\begin{minted}{java}
long lfLong(int n) {
  if (n <= 0) {
    return 0;
  } else if (n == 1) {
    return 1;
  } else {
    return n * lfLong(n - 1) - (n - 1) * lfLong(n - 2);
  }
}
\end{minted}

Beweisen Sie \emph{formal} mittels \emph{vollständiger Induktion}:

\def\lf#1{\text{lfLong}(#1)}
\def\sk#1{\sum^{#1}_{k=0}k!}

\begin{displaymath}
\forall n \geq 0: \lf{n} \equiv \sk{n-1}
\end{displaymath}

\begin{liAntwort}

IA:

\begin{displaymath}
n=1 \Rightarrow
\lf{1} =
1 =
\sk{n-1} =
0! =
1
\end{displaymath}

\begin{displaymath}
n=2 \Rightarrow
\lf{2} =
2 \cdot \lf{1} - 1 \cdot \lf{0} =
2 =
\sk{1} =
1! + 0! =
1 + 1 =
2
\end{displaymath}

IV:

\begin{displaymath}
\lf{n} = \sk{n-1}
\end{displaymath}

gilt:

IS:

\begin{equation}
\begin{split}
\lf{n+1} & = (n+1) \cdot \lf{n} - n \cdot \lf{n-1} \\
& = (n+1) \cdot \sk{n-1} - n \cdot \sk{n-2} \\
& = (n+1) \cdot \Big((n-1)! + \sk{n-2}\Big) - n \cdot \sk{n-2} \\
& = (n+1)(n-1)! + (n+1) \cdot \sk{n-2} - n \cdot \sk{n-2} \\
& = (n+1)(n-1)! \cdot \sk{n-2} + n \cdot \sk{n-2} - n \cdot \sk{n-2} \\
& = (n+1)(n-1)! + \sk{n-2} \\
& = n \cdot (n-1)! + (n-1)! + \sk{n-2} \\
& = n \cdot (n-1)! + \sk{n-1} \\
& = n! + \sk{n-1} \\
& = \sk{n} \\
& = \sk{(n+1)-1}
\end{split}
\end{equation}
\end{liAntwort}
\end{enumerate}
\end{document}
