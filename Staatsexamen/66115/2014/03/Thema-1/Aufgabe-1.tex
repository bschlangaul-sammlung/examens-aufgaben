\documentclass{lehramt-informatik-aufgabe}
\liLadePakete{vollstaendige-induktion}

\begin{document}
\let\m=\liInduktionMarkierung
\let\e=\liInduktionErklaerung
% lf = lfLong
\def\lf#1{\text{lfLong}(#1)}
% sk = summe k
\def\sk#1{\sum^{#1}_{k=0}k!}

\liAufgabenTitel{Klasse „LeftFactorial“ und Methode „lfBig()“}

\section{Aufgabe 1: „Rekursion und Induktion“
\index{Rekursion}
\index{Vollständige Induktion}
\footcite[Thema 1 Aufgabe 1b Seite 2-3]{examen:66115:2014:03}}

\begin{enumerate}

%%
% a)
%%

\item Gegeben sei die Methode \liJavaCode{BigInteger lfBig(int n)} zur
Berechnung der eingeschränkten Linksfakultät:\footcite[Seite
25]{aud:fs:1}

\begin{equation*}
!n :=
\begin{cases}
n !(n - 1) - (n - 1) !(n - 2) &
\text{falls } 1 < n < 32767 \\

1 &
\text{falls } n = 1 \\

0 &
\text{sonst } \\
\end{cases}
\end{equation*}

\liJavaExamen[firstline=3,lastline=30]{66115}{2014}{03}{LeftFactorial}

Implementieren Sie unter Verwendung des Konzeptes der \emph{dynamischen
Programmierung} die Methode \liJavaCode{BigInteger dp(int n)}, die
jede $!n$ auch bei mehrfachem Aufrufen mit dem gleichen Parameter
höchstens einmal rekursiv berechnet. Sie dürfen der Klasse
\liJavaCode{LeftFactorial} genau ein Attribut beliebigen Datentyps
hinzufügen und die in \liJavaCode{lfBig(int)} verwendeten Methoden und
Konstanten ebenfalls nutzen.\footcite[Aufgabe 5]{aud:pu:1}
\index{Implementierung in Java}

\begin{liAntwort}
Wir führen ein Attribut mit dem Namen \liJavaCode{store} ein und
erzeugen ein Feld vom Typ \liJavaCode{BigInteger} mit der Länge $n + 1$.
Die Länge des Feld $n + 1$ hat den Vorteil, dass nicht ständig $n - 1$
verwendet werden muss, um den gewünschten Wert zu erhalten.

In der untenstehenden Implementation gibt es zwei Methoden mit dem Namen
\liJavaCode{dp}. Die untenstehende Methode ist nur eine Hüllmethode, mit
der nach außen hin die Berechnung gestartet und das
\liJavaCode{store}-Feld neu gesetzt wird. So ist es möglich
\liJavaCode{dp()} mehrmals hintereinander mit verschiedenen Werten
aufzurufen (siehe \liJavaCode{main()}-Methode).

\liJavaExamen[firstline=32,lastline=52]{66115}{2014}{03}{LeftFactorial}
\end{liAntwort}

%%
% b)
%%

\item Betrachten Sie nun die Methode \liJavaCode{lfLong(int)} zur
Berechnung der vorangehend definierten Linksfakultät ohne obere
Schranke. Nehmen Sie im Folgenden an, dass der Datentyp
\liJavaCode{long} unbeschränkt ist und daher kein Überlauf auftritt.

\liJavaExamen[firstline=54,lastline=62]{66115}{2014}{03}{LeftFactorial}

Beweisen Sie \emph{formal} mittels \emph{vollständiger Induktion}:

\begin{displaymath}
\forall n \geq 0: \lf{n} \equiv \sk{n - 1}
\end{displaymath}

\begin{liAntwort}

%%
%
%%

\liInduktionAnfang

\begin{displaymath}
n=1 \Rightarrow
\texttt{\lf{1}} =
1 =
\sk{n - 1} =
0! =
1
\end{displaymath}

\begin{equation*}
\begin{split}
n=2 \Rightarrow & \lf{2}\\
& = (n + 1) \sk{n - 1} - n \sk{n-2} \\
& = \texttt{2 * \lf{1} - 1 * \lf{0}} \\
& = 2 \\
& = \sk{1} \\
& = 1! + 0! \\
& = 1 + 1 \\
& = 2
\end{split}
\end{equation*}

%%
%
%%

\liInduktionVoraussetzung

\begin{displaymath}
\texttt{\lf{n}} = \sk{n - 1}
\end{displaymath}

%%
%
%%

\liInduktionSchritt

\begin{align*}
A(n + 1) \\
& = \texttt{\lf{n + 1}}\\
& = \texttt{(n + 1) * \lf{n} - n * \lf{n - 1}} \\
& = (n + 1) \m{\sk{n - 1}} - n \m{\sk{(n - 1) -1}}
& \e{Formel eingesetzt}\\
%
& = (n + 1) \sk{n - 1} - n \sk{n\m{-2}}
& \e{subtrahiert}\\
%
& = n\sk{n - 1} + \sk{n - 1} - n \sk{n \m{- 2}}
& \e{ausmultipliziert mit $(n + 1)$}\\
%
& = \sk{n - 1} + \m{n\sk{n - 1}} - n \sk{n -2}
& \e{Reihenfolge der Terme ändern}\\
%
& = \sk{n - 1} + n \Bigl(\m{(n - 1)!} + \sk{n - \m{2}}\Bigr) - n \sk{n-2}
& \e{$(n - 1)!$ aus Summenzeichen entfernen}\\
%
& = \sk{n - 1} + \m{n} \Bigl((n - 1)! + \sk{n - 2} \m{- \sk{n-2}}\Bigr)
& \e{Distributivgesetz $ac - bc = (a - b)c$}\\
%
& = \sk{n - 1} + \m{n(n - 1)!}
& \e{$+\Sigma-\Sigma=0$}\\
%
& = \sk{n - 1} + \m{n!}
& \e{Fakultät erhöht}\\
%
& = \sk{\m{n}}
& \e{Element zum Summenzeichen hinzugefügt}\\
%
& = \sk{\m{(n + 1)}-1}
& \e{mit $(n + 1)$ an der Stelle von $n$}\\
\end{align*}
\end{liAntwort}

\begin{liAdditum}
\liPseudoUeberschrift{Komplette Klasse LeftFactorial}

\liJavaExamen{66115}{2014}{03}{LeftFactorial}
\end{liAdditum}

\end{enumerate}
\end{document}
