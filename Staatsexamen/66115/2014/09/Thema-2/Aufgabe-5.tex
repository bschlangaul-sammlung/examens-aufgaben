\documentclass{lehramt-informatik-minimal}
\InformatikPakete{syntax}
\begin{document}

\section{Herbst 2014 (66115) - Thema 2 Aufgabe 5
\index{Stapel (Stack)}
\footcite[Herbst 2014 (66115) - Thema 2 Aufgabe 5, Seite 5]{examen:66115:2014:09}}

Gegeben sei eine Standarddatenstruktur Stapel (Stack) mit den Operationen

\begin{compactitem}
\item \java{void push(Element e)}
\item \java{Element pop()},
\item \java{boolean isEmpty()}.
\end{compactitem}

sowie dem Standardkonstruktor \java{Stapel()}, der einen leeren Stapel
zur Verfügung stellt.\footcite[Seite 2-3, Aufgabe 5]{aud:ab:4}

\begin{enumerate}

%%
% (a)
%%

\item Geben Sie eine Methode \java{Stapel merge(Stapel s, Stapel t)} an,
die einen aufsteigend geordneten Stapel zurückgibt, unter der Bedingung,
dass die beiden übergebenen Stapel aufsteigend sortiert sind, d.\,h.
\java{S.pop()} liefert das größte Element in \java{s} zurück und
\java{T.pop()} liefert das größte Element in \java{t} zurück. Als
Hilfsdatenstruktur dürfen Sie nur Stapel verwenden, keine Felder oder
Listen.

\begin{antwort}
\inputcode[firstline=46,lastline=65]{aufgaben/aud/examen_66115_2014_09/Stapel}
\end{antwort}

%%
% (b)
%%

\item Analysieren Sie die Laufzeit Ihrer Methode.

\end{enumerate}

\noindent
Hinweis: Nehmen Sie an, dass Objekte der Klasse \java{Element}, die auf
dem Stapel liegen mit \java{compareTo()} vergleichen werden können. Zum
Testen haben wir Ihnen eine Klasse \java{StapelTest} zur Verfügung
gestellt, sie können Ihre Methode hier einfügen und testen, ob die
Stapel korrekt sortiert werden. Überlegen Sie auch, was geschieht, wenn
einer der Stapel (oder beide) leer ist!
\end{document}
