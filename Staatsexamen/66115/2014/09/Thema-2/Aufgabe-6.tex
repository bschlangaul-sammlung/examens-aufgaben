\documentclass{lehramt-informatik-aufgabe}
\liLadePakete{mathe,syntax}
\begin{document}
\liAufgabenTitel{Selectionsort}

\section{Aufgabe 6
\index{Selectionsort}
\footcite[Aufgabe 4, Seite 4]{aud:ab:3}}

Gegeben sei ein einfacher Sortieralgorithmus, der ein gegebenes Feld $A$
dadurch sortiert, dass er das \emph{Minimum} $m$ von $A$ \emph{findet},
dann das Minimum von $A$ ohne das Element $m$ usw.
\footcite[Thema 2 Aufgabe 6 Seite 5]{examen:66115:2014:09}

\begin{enumerate}

%%
% (a)
%%

\item Geben Sie den Algorithmus in Java an.
Implementieren\index{Implementierung in Java} Sie den Algorithmus
\emph{in situ}, d.\,h. so, dass er außer dem Eingabefeld nur konstanten
Extraspeicher benötigt. Es steht eine Testklasse zur Verfügung.

\begin{liAntwort}
\liJavaExamen{66115}{2014}{09}{SortierungDurchAuswaehlen}
\end{liAntwort}

%%
% (b)
%%

\item Analysieren Sie die Laufzeit Ihres Algorithmus.\index{Komplexität}

\begin{liAntwort}
Beim ersten Durchlauf des \emph{Selectionsort}-Algorithmus muss $n - 1$
mal das Minimum durch Vergleich ermittel werden, beim zweiten Mal
$n - 2$.
Mit Hilfe der \emph{Gaußschen Summenformel} kann die Komplexität
gerechnet werden:

\begin{displaymath}
(n-1)+(n-2)+\dotsb+3+2+1 =
\frac{(n-1)\cdot n}{2} =
\frac{n^2}{2}-\frac{n}{2} \approx
\frac{n^2}{2} \approx
n^2
\end{displaymath}

Da es bei der Berechnung des Komplexität um die Berechnung der
asymptotischen oberen Grenze geht, können Konstanten und die Addition,
Subtraktion, Multiplikation und Division mit Konstanten z. b.
$\frac{n^2}{2}$ vernachlässigt werden.

Der \emph{Selectionsort}-Algorithmus hat deshalb die Komplexität
$\mathcal{O}(n^2)$, er ist von der Ordnung
$\mathcal{O}(n^2)$.
\end{liAntwort}

%%
% (c)
%%

\item Geben Sie eine Datenstruktur an, mit der Sie Ihren Algorithmus
beschleunigen können.\index{Halde (Heap)}

\begin{liAntwort}
Der \emph{Selectionsort}-Algorithmus kann mit einer Min- (in diesem
Fall) bzw. einer Max-Heap beschleunigt werden. Mit Hilfe dieser
Datenstruktur kann sehr schnell das Minimum gefunden werden. So kann auf
die viele Vergleiche verzichtet werden. Die Komplexität ist dann
$\mathcal{O}(n \log n)$.
\end{liAntwort}
\end{enumerate}

\end{document}
