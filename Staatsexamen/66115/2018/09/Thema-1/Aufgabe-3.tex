\documentclass{lehramt-informatik-aufgabe}
\liLadePakete{formale-sprachen,cyk-algorithmus}
\begin{document}
\liAufgabenMetadaten{
  Titel = Aufgabe 3,
  Thematik = Kontextfreie Sprachen,
  RelativerPfad = Staatsexamen/66115/2018/09/Thema-1/Aufgabe-3.tex,
  ZitatSchluessel = examen:66115:2018:09,
  ExamenNummer = 66115,
  ExamenJahr = 2018,
  ExamenMonat = 09,
  ExamenThemaNr = 1,
  ExamenAufgabeNr = 3,
}

\let\m=\liMenge
\let\l=\liKurzeTabellenLinie

\liAufgabenTitel{Kontextfreie Sprachen}
\section{Aufgabe 3
\index{Kontextfreie Sprache}
\footcite{examen:66115:2018:09}}

\begin{enumerate}

%%
% (a)
%%

\item Entwerfen Sie eine kontextfreie Grammatik für die folgende
kontextfreie Sprache über dem Alphabet \liAlphabet{a, b, c}:

\begin{center}
\liAusdruck
  {w b^{3k} c^{2k+1} v}
  {k \in \mathbb{N}, |w|_c= |u|_a}
\end{center}

(Hierbei bezeichnet $|u|$, die Anzahl des Zeichens $x$ in dem Wort $u$,
und es gilt $0 \in \mathbb{N}$.) Erklären Sie den Zweck der einzelnen
Nichtterminale (Variablen) und der Grammatikregeln Ihrer Grammatik.

%%
% (b)
%%

\item Betrachten Sie die folgende kontextfreie Grammatik

\begin{center}
\liGrammatik{variablen={S, X, Y, Z}, alphabet={z, y}}
\end{center}

mit den Produktionen

\begin{liProduktionsRegeln}
S -> Z X | y,
X -> Z S | S S | x,
Y -> S X | Y Z,
Z -> X X | X S
\end{liProduktionsRegeln}

Benutzen Sie den Algorithmus von Cocke-Younger-Kasami (CYK) um zu
zeigen, dass das Wort $xxxyx$ zu der von G erzeugten Sprache $L(G)$
gehört.
\index{CYK-Algorithmus}

\begin{liAntwort}
\begin{tabular}{|c|c|c|c|c|}
x   & x   & x   & y   & x \\\hline\hline

X   & X   & X   & S   & X \l5
Z   & Z   & Z   & Y \l4
S   & X   & S \l3
Z,X & Z   \l2
X,S,Z \l1
\end{tabular}

\liWortInSprache{xxxyx}
\end{liAntwort}

%%
%  (c)
%%

\item Geben Sie eine Ableitung des Wortes $xxxyx$ mit $G$ an.

\end{enumerate}
\end{document}
