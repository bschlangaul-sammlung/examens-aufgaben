\documentclass{lehramt-informatik-aufgabe}
\liLadePakete{formale-sprachen}
\begin{document}
\let\m=\liMenge

\liAufgabenTitel{}
\section{Aufgabe 3 (Kontextfreie Sprachen)
\index{Kontextfreie Sprache}
\footcite{66115:2018:09}}

\begin{enumerate}

%%
% (a)
%%

\item Entwerfen Sie eine kontextfreie Grammatik für die folgende
kontextfreie Sprache über dem Alphabet \liAlphabet{a, b,c}:

\begin{displaymath}
L=\m{ w b^{3k} c^{2k+1} v \, | \, k \in \mathbb{N}, |w|_c= |u|_a}
\end{displaymath}

(Hierbei bezeichnet $|u|$, die Anzahl des Zeichens $x$ in dem Wort $u$,
und es gilt $0 \in \mathbb{N}$.) Erklären Sie den Zweck der einzelnen
Nichtterminale (Variablen) und der Grammatikregeln Ihrer Grammatik.

%%
% (b)
%%

\item Betrachten Sie die folgende kontextfreie Grammatik

\begin{displaymath}
G = (\m{S, X,Y, Z}, \m{z, y}, P, S)
\end{displaymath}

mit den Produktionen

\begin{liProduktionsRegeln}
S -> ZX | y,
X -> ZS | SS | x,
Y -> SX | YZ,
Z -> XX | XS
\end{liProduktionsRegeln}

Benutzen Sie den Algorithmus von Cocke-Younger-Kasami (CYK) um zu
zeigen, dass das Wort $zzzyx$ zu der von G erzeugten Sprache $L(G)$
gehört.

%%
%  (c)
%%

\item Geben Sie eine Ableitung des Wortes $xxxyx$ mit $G$ an.

\end{enumerate}
\end{document}
