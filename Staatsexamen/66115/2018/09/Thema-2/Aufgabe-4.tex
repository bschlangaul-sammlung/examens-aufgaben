\documentclass{bschlangaul-aufgabe}
\bLadePakete{komplexitaetstheorie}
\begin{document}
\bAufgabenTitel{}
\section{Aufgabe 4
\index{Polynomialzeitreduktion}
\footcite{examen:66116:2018:09}}

Wir betrachten ungerichtete Graphen G = (V, E), wo E eine Teilmenge E.
hat, die wir exklusive Kanten nennen. Eine beschränkte Überdeckung von G
ist eine Teilmenge U von V, so dass

\begin{enumerate}
\item jeder Knoten einen Nachbarknoten in U hat oder selbst in U liegt
(für jeden Knoten u e V U gibt es einen Knoten v € U mit (u,v) € E) und

\item für jede exklusive Kante (u,v) € E. genau einer der Knoten u, v in
U liegt.

\end{enumerate}
Betrachten Sie nun die folgenden Entscheidungsprobleme:

\bProblemBeschreibung
{3SAT}
{Aussagenlogische Formel p in 3KNF}
{Hat o eine erfüllende Belegung?}

\bProblemBeschreibung
{BÜ}
{Graph G = (V, E), exklusive Kantenmenge E. CE
undkeN}
{Hat G eine beschränkte Überdeckung U mit |U| < k?}

Beweisen Sie, dass BÜ NP-vollständig ist. Sie dürfen dabei annehmen,
dass 3SAT NP-vollständig ist.

\end{document}
