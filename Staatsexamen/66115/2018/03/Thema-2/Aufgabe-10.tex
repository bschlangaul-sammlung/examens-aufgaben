\documentclass{lehramt-informatik-aufgabe}
\liLadePakete{}
\begin{document}
\liAufgabenTitel{}
\section{Aufgabe 10:
\index{Algorithmus von Prim}
\footcite{66115:2018:03}}
\begin{enumerate}

%%
% a)
%%

\item Berechnen Sie mithilfe des Algorithmus von Prim ausgehend vom
Knoten a einen minimalen Spannbaum des ungerichteten Graphen G, der
durch folgende Adjazenzmatrix gegeben ist:

ıabedefgh
a-146---5
bi-3-4-7c43-1----
d6-1-9620
e-4-9-55fi---65---
g-7-25--8
h5--0--81

Erstellen Sie dazu eine Tabelle mit zwei Spalten und stellen Sie jeden
einzelnen Schritt des Verfahrens in einer eigenen Zeile dar. Geben Sie
in der ersten Spalte denjenigen Knoten v, der vom Algorithmus als
nächstes in den Ergebnisbaum aufgenommen wird (dieser sog. „schwarze“
Knoten ist damit fertiggestellt), als Tripel (v,p,6) mit v als
Knotenname, p als aktueller Vorgängerknoten und 6 als aktuelle Distanz
von v zu p an. Führen Sie in der zweiten Spalte alle anderen vom
aktuellen Spannbaum direkt erreichbaren Knoten v (sog. „graue Randknoten
“) ebenfalls als Tripel (v,p,6) auf.

Zeichnen Sie anschließend den entstandenen Spannbaum und geben sein
Gewicht an.

%%
% b)
%%

\item Welche Worst-Case-Laufzeitkomplexität hat der Algorithmus von
Prim, wenn die grauen Knoten in einem Heap (= Halde) nach Distanz
verwaltet werden? Sei dabei n die Anzahl an Knoten und m die Anzahl an
Kanten des Graphen. Eine Begründung ist nicht erforderlich.

%%
% c)
%%

\item Zeigen Sie durch ein kleines Beispiel, dass ein minimaler
Spannbaum eines ungerichteten Graphen nicht immer eindeutig ist.

%%
% d)
%%

\item Skizzieren Sie eine Methode, mit der ein maximaler Spannbaum mit
einem beliebigen Algorithmus für minimale Spannbäume berechnet werden
kann. In welcher Laufzeitkomplexität kann ein maximaler Spannbaum
berechnet werden?
\end{enumerate}
\end{document}
