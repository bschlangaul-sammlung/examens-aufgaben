\documentclass{lehramt-informatik-aufgabe}
\liLadePakete{graph}
\begin{document}
\liAufgabenTitel{}
\section{Aufgabe 10:
\index{Algorithmus von Prim}
\footcite{66115:2018:03}}
\begin{enumerate}

%%
% a)
%%

\item Berechnen Sie mithilfe des Algorithmus von Prim ausgehend vom
Knoten $a$ einen minimalen Spannbaum des ungerichteten Graphen $G$, der
durch folgende Adjazenzmatrix gegeben ist:

% Info_2021-06-18-2021-06-18_13.18.21.mp4
% 2h06m

\begin{liGraphenFormat}
a: 1 1
b: 2 2
c: 3 4
d: 6 7
e: 9 0
f: 5 2
g: 9 4
h: 8 4
a -- b: 1
a -- c: 4
a -- d: 6
a -- h: 5
b -- c: 3
b -- e: 4
b -- g: 7
d -- e: 9
d -- f: 6
d -- g: 2
d -- h: 0
e -- f: 5
e -- g: 5
g -- h: 8
h -- h: 1
\end{liGraphenFormat}

\[
\begin{blockarray}{cccccccc}
   & a & b & c & d & e & f & g & h \\
\begin{block}{c(ccccccc)}
 a & 0 & 1 & 4 & 6 & 0 & 0 & 0 & 5 \\
 b & 1 & 0 & 3 & 0 & 4 & 0 & 7 & 0 \\
 c & 4 & 3 & 0 & 0 & 0 & 0 & 0 & 0 \\
 d & 6 & 0 & 0 & 0 & 9 & 6 & 2 & 0 \\
 e & 0 & 4 & 0 & 9 & 0 & 5 & 5 & 0 \\
 f & 0 & 0 & 0 & 6 & 5 & 0 & 0 & 0 \\
 g & 0 & 7 & 0 & 2 & 5 & 0 & 0 & 8 \\
 h & 5 & 0 & 0 & 0 & 0 & 0 & 8 & 0 \\
\end{block}
\end{blockarray}
\]

Erstellen Sie dazu eine Tabelle mit zwei Spalten und stellen Sie jeden
einzelnen Schritt des Verfahrens in einer eigenen Zeile dar. Geben Sie
in der ersten Spalte denjenigen Knoten $v$, der vom Algorithmus als
nächstes in den Ergebnisbaum aufgenommen wird (dieser sog. „schwarze
“Knoten ist damit fertiggestellt), als Tripel $(v, p, \delta)$ mit $v$
als Knotenname, $p$ als aktueller Vorgängerknoten und $\delta$ als
aktuelle Distanz von $v$ zu $p$ an. Führen Sie in der zweiten Spalte
alle anderen vom aktuellen Spannbaum direkt erreichbaren Knoten $v$
(sog. „graue Randknoten“) ebenfalls als Tripel $(v, p, \delta)$ auf.

Zeichnen Sie anschließend den entstandenen Spannbaum und geben sein
Gewicht an.

%%
% b)
%%

\item Welche Worst-Case-Laufzeitkomplexität hat der Algorithmus von
Prim, wenn die grauen Knoten in einem Heap (= Halde) nach Distanz
verwaltet werden? Sei dabei $n$ die Anzahl an Knoten und $m$ die Anzahl
an Kanten des Graphen. Eine Begründung ist nicht erforderlich.

%%
% c)
%%

\item Zeigen Sie durch ein kleines Beispiel, dass ein minimaler
Spannbaum eines ungerichteten Graphen nicht immer eindeutig ist.

%%
% d)
%%

\item Skizzieren Sie eine Methode, mit der ein maximaler Spannbaum mit
einem beliebigen Algorithmus für minimale Spannbäume berechnet werden
kann. In welcher Laufzeitkomplexität kann ein maximaler Spannbaum
berechnet werden?
\end{enumerate}
\end{document}
