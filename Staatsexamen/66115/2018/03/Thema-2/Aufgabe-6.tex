\documentclass{lehramt-informatik-aufgabe}
\liLadePakete{syntax,mathe,master-theorem}
\begin{document}
\liAufgabenTitel{Methode „m()“}

\section{Aufgabe 6 \index{Master-Theorem}
\footcite[Seite 39 (PDF 26)]{aud:fs:2}
\footcite[Thema 2 Aufgabe 6]{examen:66115:2018:03}
}

Der Hauptsatz der Laufzeitfunktionen ist bekanntlich folgendermaßen definiert:

\liMasterExkurs

\begin{enumerate}

\item Betrachten Sie die folgende Methode \liJavaCode{m} in Java, die
initial mit \liJavaCode{m(r, 0, r.length)} für das Array \liJavaCode{r}
aufgerufen wird. Geben Sie dazu eine Rekursionsgleichung $T(n)$ an,
welche die Anzahl an Rechenschritten von \liJavaCode{m} in Abhängigkeit
von der Länge \liJavaCode{n = r.length} berechnet.

\liJavaExamen[firstline=5,lastline=21]{66115}{2018}{03}{MasterTheorem}

\begin{liAntwort}
\liMasterVariablenDeklaration
{3} % a
{3} % b
{\mathcal{O}(1)} % f(n)
\end{liAntwort}

%%
% b)
%%

\item Ordnen Sie die rekursive Funktion T(n) aus (a) einem der drei
Fälle des Mastertheorems zu und geben Sie die resultierende
Zeitkomplexität an. Zeigen Sie dabei, dass die Voraussetzung des Falles
erfüllt ist.

\begin{liAntwort}

\liMasterFallRechnung
% 1. Fall
{$f(n) \in \mathcal{O}\left(n^{\log_{3}3-\varepsilon}\right) =
\mathcal{O}\left(n^{1-\varepsilon}\right) =
\mathcal{O}\left(1\right) \text{ für } \varepsilon = 1
$}
% 2. Fall
{$f(n) \notin \Theta \left(n^{{\log_{3}3}}\right) =
\Theta \left(n^1\right)
$}
% 3. Fall
{$f(n) \notin \Omega \left(n^{\log_{3}3 + \varepsilon}\right) =
\Omega \left(n^{1 + \varepsilon}\right)$}

Also: $T(n)\in \Theta \left(n^{\log_{b}a}\right)$

\end{liAntwort}
\end{enumerate}

\end{document}
