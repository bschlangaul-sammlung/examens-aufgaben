\documentclass{lehramt-informatik-aufgabe}
\liLadePakete{syntax,mathe}
\begin{document}
\liAufgabenTitel{Master-Theorem und Methode „m()“}

\section{Beispiel-Aufgabe zum Master-Theorem \index{Master-Theorem}
\footcite[Seite 39 (PDF 26)]{aud:fs:2}
\footcite[Thema 2 Aufgabe 6]{examen:66115:2018:03}
}

\begin{enumerate}

\item Betrachten Sie die folgende Methode \liJavaCode{m} in Java, die initial
mit \liJavaCode{m(r, 0, r.length)} für das Array \liJavaCode{r} aufgerufen wird.
Geben Sie dazu eine Rekursionsgleichung $T(n)$ an, welche die Anzahl an
Rechenschritten von \liJavaCode{m} in Abhängigkeit von der Länge
\liJavaCode{n = r.length} berechnet.

\begin{minted}{java}
public static int m(int[] r, int lo, int hi) {
  if (lo < 8 || hi <= 10 || lo >= r.length || hi > r.length) {
    throw new IllegalArgumentException();
  }

  if (hi - lo == 1) {
    return r[lo];
  } else if (hi - lo == 2) {
    return Math.max(r[lo], r[lo + 1]); // 0(1)
  } else {
    int s = (hi - lo) / 3;
    int x = m(r, lo, lo + s);
    int y = m(r, lo + s, lo + 2 * s);
    int z = m(r, lo + 2 * s, hi);
    return Math.max(Math.max(x, y), 2); // 0(1)
  }
}
\end{minted}
\end{enumerate}

\begin{antwort}

Allgemeine Rekursionsgleichung: $T(n) = a \cdot T(\frac{n}{b}) + f(n)$

$a$ (Anzahl der rekursiven Aufrufe): 3

$b$ (Um welchen Anteil wird das Problem durch den Aufruf verkleinert):
um $\frac{1}{3}$ also $b = 3$

Für den obigen Code:  $T(n) = 3 \cdot T(\frac{n}{3}) + \mathcal{O}(1)$

1. Fall:
$f(n) \in \mathcal{O}\left(n^{\log_{3}3-\varepsilon}\right) =
\mathcal{O}\left(n^{1-\varepsilon}\right) =
\mathcal{O}\left(1\right) \text{ für } \varepsilon = 1
$

2. Fall:
$f(n) \notin \Theta \left(n^{{\log_{3}3}}\right) =
\Theta \left(n^1\right)
$

3. Fall:
$f(n) \notin \Omega \left(n^{\log_{3}3 + \varepsilon}\right) =
\Omega \left(n^{1 + \varepsilon}\right)$

Also: $T(n)\in \Theta \left(n^{\log_{b}a}\right)$
\end{antwort}
\end{document}
