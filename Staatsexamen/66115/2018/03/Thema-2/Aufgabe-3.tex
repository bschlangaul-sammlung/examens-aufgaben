\documentclass{lehramt-informatik-aufgabe}
\liLadePakete{mathe,automaten}
\begin{document}
\liAufgabenTitel{Exponentieller Blow-Up}
\section{Aufgabe 3
\index{Reguläre Sprache}
\footcite{66115:2018:03}}

Gesucht ist eine reguläre Sprache $C \in \{a, b\}^∗$, deren minimaler
deterministischer endlicher Automat (DEA) mindestens 4 Zustände mehr
besitzt als der minimale nichtdeterministische endliche Automat (NEA).
Gehen Sie wie folgt vor:

\footcite{theo:ab:1}

\begin{enumerate}

%%
% (a)
%%

\item Definieren Sie $C \in \{a, b\}$ und erklären Sie kurz, warum es
bei dieser Sprache NEAs gibt, die deutlich kleiner als der minimale DEA
sind.

\begin{liAntwort}
Sprache mit exponentiellem Blow-Up:

Ein NEA der Sprache

\begin{align*}
L_k &= \{ xay | x, y \in \{a, b\}^* \land |y| = k − 1 \}\\
    &= \{ w \in \{a, b\}^* | \text{ der k-te Buchstabe von hinten ist ein } a\}\\
\end{align*}

kommt mit $k + 1$ Zuständen aus.

Jeder DEA $M$ mit $L(M) = L$ hat dann mindestens $2^k$ Zustände. Wir
wählen $k = 3$. Dann hat der zughörige NEA 4 Zustände und der zugehörige
DEA mindestens $8$. Sei also $L_k = \{ xay | x, y \in \{a, b\}^* \land
|y| = 2 \}$ die gesuchte Sprache.

Der informelle Grund, warum ein DEA für die Sprache $L_k$ groß sein muss, ist
dass er sich immer die letzten n Symbole merken muss.
\liFussnoteUrl{https://www.tcs.ifi.lmu.de/lehre/ss-2013/timi/handouts/handout-02}
\end{liAntwort}

%%
% (b)
%%

\item Geben Sie den minimalen DEA $M$ für $C$ an.
(Zeichnung des DEA genügt; die Minimalität muss nicht begründet werden.)

%%
% (c)
%%

\item Geben Sie einen NEA N für $C$ an, der mindestens 4 Zustände
weniger besitzt als $M$. (Zeichnung des DEA genügt)

\begin{liAntwort}
\begin{center}
\begin{tikzpicture}[->,node distance=2cm]
\node[state,initial] (0) {$z_0$};
\node[state,right of=0] (1) {$z_1$};
\node[state,right of=1] (2) {$z_2$};
\node[state,right of=2,accepting] (3) {$z_3$};

\path (0) edge[above] node{a} (1);
\path (1) edge[above] node{a,b} (2);
\path (2) edge[above] node{a,b} (3);
\path (0) edge[loop,above] node{a,b} (0);
\end{tikzpicture}
\end{center}
\end{liAntwort}
\end{enumerate}
\end{document}
