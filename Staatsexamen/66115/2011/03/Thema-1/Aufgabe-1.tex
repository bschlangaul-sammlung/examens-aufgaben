\documentclass{lehramt-informatik-aufgabe}
\liLadePakete{master-theorem}
\begin{document}
\liAufgabenTitel{}
\section{Aufgabe 1
\index{Master-Theorem}
\footcite{66115:2011:03}}

Bestimmen Sie mit Hilfe des Master-Theorems für die folgenden
Rekursionsgleichungen möglichst scharfe asymptotische untere und obere
Schranken, falls das Master-Theorem anwendbar ist! Geben Sie andernfalls
eine kurze Begründung, warum das Master-Theorem nicht anwendbar ist!

\begin{enumerate}
%%
% a)
%%

\item T(n) = 16T(n/2) + 40n - 6

\begin{liAntwort}
\def\liLet#1#2{
  \texttt{\textbackslash{}let\textbackslash#1=\textbackslash#2}
}
\liLet{T}{liT}
\end{liAntwort}

%%
% b)
%%

\item T(n) = 27T(n/3) + 3n? logn

%%
% c)
%%

\item Tin) = 4T(n/2) +3n? +logn

%%
% d)
%%

\item T(n)=4T(w/16) + 100 logn + Yan +n”
\end{enumerate}

\end{document}
