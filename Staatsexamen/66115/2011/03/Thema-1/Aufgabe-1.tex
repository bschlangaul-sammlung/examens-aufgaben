\documentclass{lehramt-informatik-aufgabe}
\liLadePakete{master-theorem}
\begin{document}
\let\O=\liO
\let\o=\liOmega
\let\T=\liT
\let\t=\liTheta

\liAufgabenTitel{4 Rekursionsgleichungen}
\section{Aufgabe 1
\index{Master-Theorem}
\footcite{66115:2011:03}}

Bestimmen Sie mit Hilfe des Master-Theorems für die folgenden
Rekursionsgleichungen möglichst scharfe asymptotische untere und obere
Schranken, falls das Master-Theorem anwendbar ist! Geben Sie andernfalls
eine kurze Begründung, warum das Master-Theorem nicht anwendbar ist!

\liMasterExkurs

\begin{enumerate}
%%
% a)
%%

% https://www.wolframalpha.com/input/?i=T%5Bn%5D%3D%3D16T%5Bn%2F2%5D%2B40n-6

\item $T(n) = \T{16}{2} + 40n - 6$

\begin{liAntwort}
\liMasterVariablenDeklaration
{16} % a
{2} % b
{40n - 6} % f(n) ohne $mathe$

\liMasterFallRechnung
% 1. Fall
{für $\varepsilon = 14$: \\
$f(n) = 40n - 6 \in \O{n^{\log_2 {16 - 14}}} = \O{n^{\log_2 2}} = \O{n}$}
% 2. Fall
{$f(n) = 40n - 6 \notin \t{n^{\log_2 {16}}} = \t{n^4}$}
% 3. Fall
{$f(n) = 40n - 6 \notin \o{n^{\log_2 {16 + \varepsilon}}}$}

$\Rightarrow T(n) \in \t{n^4}$

\liMasterWolframLink{T[n]=16T[n/2]\%2B40n-6}
\end{liAntwort}

%%
% b)
%%

% T[n]==27T[n/3]+3n^2*log_2(n)

% https://www.wolframalpha.com/input/?i=T%5Bn%5D%3D%3D27T%5Bn%2F3%5D%2B3n%5E2*log_2%28n%29

\item $T(n) = \T{27}{3} + 3n^2 \log n$

\begin{liAntwort}
\liMasterVariablenDeklaration
{27} % a
{3} % b
{3n^2 \log n} % f(n) ohne $mathe$

\liMasterFallRechnung
% 1. Fall
{}
% 2. Fall
{}
% 3. Fall
{}

\liMasterWolframLink{}
\end{liAntwort}

%%
% c)
%%

% https://www.wolframalpha.com/input/?i=T%5Bn%5D%3D%3D4*T%5Bn%2F2%5D%2B3n%5E2%2Blog_2%28n%29

\item $T(n) = \T{4}{2} + 3n^2 + \log n$

\begin{liAntwort}
\liMasterVariablenDeklaration
{4} % a
{2} % b
{3n^2 + \log n} % f(n) ohne $mathe$

\liMasterFallRechnung
% 1. Fall
{}
% 2. Fall
{}
% 3. Fall
{}

\liMasterWolframLink{}
\end{liAntwort}

%%
% d)
%%

% https://www.wolframalpha.com/input/?i=T%5Bn%5D%3D%3D4*T%5Bn%2F2%5D%2B100*log_2%28n%29%E2%88%9A%282N%29+%2B+n%5E%28-2%29

\item $T(n)= \T{4}{2} + 100 \log n + \sqrt{2n} + n^{-2}$

\begin{liAntwort}
\liMasterVariablenDeklaration
{4} % a
{2} % b
{100 \log n + \sqrt{2n} + n^{-2}} % f(n) ohne $mathe$

\liMasterFallRechnung
% 1. Fall
{}
% 2. Fall
{}
% 3. Fall
{}

\liMasterWolframLink{}
\end{liAntwort}

\end{enumerate}

\end{document}
