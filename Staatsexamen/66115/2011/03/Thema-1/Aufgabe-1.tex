\documentclass{lehramt-informatik-aufgabe}
\liLadePakete{master-theorem}
\begin{document}
\let\T=\liT

\liAufgabenTitel{}
\section{Aufgabe 1
\index{Master-Theorem}
\footcite{66115:2011:03}}

Bestimmen Sie mit Hilfe des Master-Theorems für die folgenden
Rekursionsgleichungen möglichst scharfe asymptotische untere und obere
Schranken, falls das Master-Theorem anwendbar ist! Geben Sie andernfalls
eine kurze Begründung, warum das Master-Theorem nicht anwendbar ist!

\begin{enumerate}
%%
% a)
%%

\item $T(n) = \T{16}{2} + 40n - 6$

%%
% b)
%%

\item $T(n) = \T{27}{3} + 3n^2 \log n$

%%
% c)
%%

\item $T(n) = \T{4}{2} + 3n^2 + \log n$

%%
% d)
%%

\item $T(n)= \T{4}{2} + 100 \log n + \sqrt{2n} + n^{-2}$
\end{enumerate}

\end{document}
