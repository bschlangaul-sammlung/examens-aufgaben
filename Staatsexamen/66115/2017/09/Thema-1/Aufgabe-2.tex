\documentclass{lehramt-informatik-aufgabe}
\liLadePakete{formale-sprachen}
\begin{document}
\let\m=\liMenge
\liAufgabenTitel{Kontextfreie Sprachen}
\section{Aufgabe 2
\index{Kontextfreie Sprache}
\footcite{66115:2017:09}}

Betrachten Sie die Sprache $L_1 = L_a \cup L_b$.

\begin{itemize}
\item[] \liAusdruck[L_a]{a^n b c^n}{n \in \mathbb{N}}
\item[] \liAusdruck[L_b]{a b^m c^m}{m \in \mathbb{N}}
\end{itemize}
\begin{enumerate}

%%
% a)
%%

\item Geben Sie für $L_1$ eine kontextfreie Grammatik an.
\begin{liAntwort}
\begin{liProduktionsRegeln}
S -> S_a | S_b,
S_a -> a S_a c | a B_a c,
B_a -> b,
S_b -> a B_b,
B_b -> b B_b c | b c
\end{liProduktionsRegeln}
\end{liAntwort}

%%
% b)
%%

\item Ist Ihre Grammatik aus a) eindeutig? Begründen Sie Ihre Antwort.

\begin{liAntwort}
Nein. Die Sprache ist nicht eindeutig. Für das Wort $abc$ gibt es
zwei Ableitungen, nämlich
\liAbleitung{S -> S_a -> a B_a c -> a b c}
und
\liAbleitung{S -> S_b -> a B_b -> a b c}.
\end{liAntwort}

%%
% c)
%%

\item Betrachten Sie die Sprache \liAusdruck[L_2]{a^ {2^n}}{n \in
\mathbb{N}}. Zeigen Sie, dass $L_2$ nicht kontextfrei ist.
\end{enumerate}

\end{document}
