\documentclass{lehramt-informatik-aufgabe}
\liLadePakete{formale-sprachen}
\begin{document}
\let\m=\liMenge
\liAufgabenTitel{Kontextfreie Sprachen}
\section{Aufgabe 2
\index{Kontextfreie Sprache}
\footcite{examen:66115:2017:09}}

Betrachten Sie die Sprache $L_1 = L_a \cup L_b$.

\begin{itemize}
\item \liAusdruck[L_a]{a^n b c^n}{n \in \mathbb{N}}
\item \liAusdruck[L_b]{a b^m c^m}{m \in \mathbb{N}}
\end{itemize}
\begin{enumerate}

%%
% a)
%%

\item Geben Sie für $L_1$ eine kontextfreie Grammatik an.
\begin{liAntwort}
\begin{liProduktionsRegeln}
S -> S_a | S_b,
S_a -> a S_a c | b,
S_b -> a | a B_b,
B_b -> b B_b c | b c
\end{liProduktionsRegeln}
\end{liAntwort}

%%
% b)
%%

\item Ist Ihre Grammatik aus a) eindeutig? Begründen Sie Ihre Antwort.

\begin{liAntwort}
Nein. Die Sprache ist nicht eindeutig. Für das Wort $abc$ gibt es
zwei Ableitungen, nämlich
\liAbleitung{S -> S_a -> a S_a c -> a b c}
und
\liAbleitung{S -> S_b -> a B_b -> a b c}.
\end{liAntwort}

%%
% c)
%%

% 1h00min

\item Betrachten Sie die Sprache \liAusdruck[L_2]{a^ {2^n}}{n \in
\mathbb{N}}. Zeigen Sie, dass $L_2$ nicht kontextfrei ist.

\begin{liAntwort}
Annahme: $L_2$ ist kontextfrei

$\rightarrow$ Pumping-Lemma gilt für $L_2$

$\rightarrow$ $j \in \mathbb{N}$ als Pumping-Zahl

$\omega \in L_2$: $|\omega| \geq j$

Konsequenz: $\omega = u v w x y$

\begin{itemize}
\item $|vx| \geq 1$
\item $|vwx| \leq j$
\item $u v^i w x^i y \in L_2$ für alle $i \in \mathbb{N}_0$
\end{itemize}

Wir wählen: $\omega = a^{2^i}$: $|\omega| \geq j$

\begin{description}
\item[p] $a \dots a$
\item[r] $a \dots a$
\item[s] $a \dots a$
\item[t] $a \dots a$
\item[q] $a \dots a$
\end{description}

$q + r + s + t + q = 2^j$

$\Rightarrow$ $r + t \geq 1$

$r + s + t \leq j$

%-----------------------------------------------------------------------
%
%-----------------------------------------------------------------------

\liPseudoUeberschrift{1. Fall}

$r + t = 2^{j-1}$

$2^{j-1} + 2^{j-1} = 2 \cdot 2^{j-1} =  2^1 \cdot 2^{j-1} =  2^{1+j-1} = 2^j$

$\omega' = u v^2 w x^2 y$

$p + 2 \cdot r + s + 2 \cdot t + q$

$p + s + q + 2 \cdot (r + t)$

$2^{j-1} + 2 \cdot 2^{j-1} = 3 \cdot 2^{j-1} = 2^{j-1} + 2^i \leq 2^{j+1}$

keine Zweierpotenz

$\Rightarrow$ $\omega \notin L_2$

$\Rightarrow$ Widerspruch zur Annahme

$\Rightarrow$ $L_2$ nicht kontextfrei

%-----------------------------------------------------------------------
%
%-----------------------------------------------------------------------

\liPseudoUeberschrift{2. Fall}

$r + t \neq 2^{j-1}$

$\omega' = u v^0 w x^0 y$

$\Rightarrow$ $p + s + q = 2^j - (r + t)$

$(r + t) \neq 2^{j-i}$

ist keine Zweierpotenz

$\Rightarrow$ $\omega \notin L_2$

$\Rightarrow$ $L_2$ nicht kontextfrei

\end{liAntwort}

\end{enumerate}

\end{document}
