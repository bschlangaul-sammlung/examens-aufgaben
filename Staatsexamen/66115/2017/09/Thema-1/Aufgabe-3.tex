\documentclass{lehramt-informatik-aufgabe}
\liLadePakete{}
\begin{document}
\liAufgabenTitel{}
\section{Aufgabe 3
\index{Komplexitätstheorie}
\footcite{66115:2017:09}}

Betrachten Sie die folgenden Probleme:
\footcite[Aufgabe 15: StEx F2016 T2 A3, StEx H2017 T1 A3 (Check-Up)]{theo:ab:4}

3SAT

Gegeben: Eine aussagenlogische Formel (p in konjunktiver Normalform

(drei Literale pro Klausel).
Frage:

Ist p erfüllbar?

NAE-3SAT

Gegeben: Eine aussagenlogische Formel p in konjunktiver Normalform

(drei Literale pro Klausel).
Frage:

Gibt es eine Belegung, die in jeder Klausel
mindestens ein Literal wahr und

mindestens ein Literal falsch macht?

Wir erlauben, dass NAE-3SAT-Formeln Literale der Form false haben, die immer falsch sind.
So ist

(xi V false V false) A (->a:i V

V xi)

in NAE-3SAT (setze Xi wahr).
\begin{enumerate}

%%
% a)
%%

\item Zeigen Sie, dass sich 3SAT in polynomieller Zeit auf NAE-3SAT
reduzieren lässt.

%%
% b)
%%

\item Was können Sie aus a) folgern, wenn Sie wissen, dass 3SAT
NF-vollständig ist?

%%
% c)
%%

\item Was können Sie aus a) folgern, wenn Sie wissen, dass NAE-3SAT
NF-vollständig ist?

\end{enumerate}
\end{document}
