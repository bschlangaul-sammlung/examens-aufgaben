\documentclass{lehramt-informatik-aufgabe}
\liLadePakete{mathe}
\begin{document}
\liAufgabenTitel{}
\section{Aufgabe 3
\index{Komplexitätstheorie}
\footcite{66115:2017:09}}

% Check-Up
% i o theo ab 4
% i e 66115:2016:03 2 3
% i e 66115:2017:09 1 3

Betrachten Sie die folgenden Probleme:
\footcite[StEx F2016 T2 A3, StEx H2017 T1 A3 (Check-Up), Aufgabe 15]{theo:ab:4}

%%
%
%%

\liPseudoUeberschrift{3SAT}

\begin{description}
\item[Gegeben:]

Eine aussagenlogische Formel $\varpi$ in konjunktiver Normalform (drei
Literale pro Klausel).

\item[Frage:]

Ist $\varpi$ erfüllbar?
\end{description}

%%
%
%%

\liPseudoUeberschrift{NAE-3SAT}

\begin{description}
\item[Gegeben:]

Eine aussagenlogische Formel $\varpi$ in konjunktiver Normalform
(drei Literale pro Klausel).

\item[Frage:]

Gibt es eine Belegung, die in jeder Klausel mindestens ein Literal wahr
und mindestens ein Literal falsch macht?

\end{description}

Wir erlauben, dass \texttt{NAE-3SAT}-Formeln Literale der Form
$\text{false}$ haben, die immer falsch sind. So ist

\begin{displaymath}
(x_1 \lor \text{false} \lor \text{false})
\land
(\neg x_1 \lor x_1 \lor x_1)
\end{displaymath}

in \texttt{NAE-3SAT} (setze $x_1$ wahr).
\begin{enumerate}

%%
% a)
%%

\item Zeigen Sie, dass sich \texttt{3SAT} in polynomieller Zeit auf
\texttt{NAE-3SAT} reduzieren lässt.

%%
% b)
%%

\item Was können Sie aus a) folgern, wenn Sie wissen, dass \texttt{3SAT}
NP-vollständig ist?

%%
% c)
%%

\item Was können Sie aus a) folgern, wenn Sie wissen, dass
\texttt{NAE-3SAT} NF-vollständig ist?

\end{enumerate}
\end{document}
