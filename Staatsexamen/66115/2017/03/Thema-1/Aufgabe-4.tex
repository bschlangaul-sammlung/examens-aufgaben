\documentclass{lehramt-informatik-aufgabe}
\liLadePakete{syntax,mathe}
\begin{document}

\section{Aufgabe 4: Vollständige Induktion
\index{Vollständige Induktion}
\footcite{sosy:ab:8}
}

Sie dürfen im Folgenden davon ausgehen, dass keinerlei Under- oder
Overflows auftreten.
\footcite[Thema 1 Aufgabe 4]{examen:66115:2017:03}

\noindent
Gegeben sei folgende rekursive Methode für $n \geq 0$:

\begin{minted}{java}
long sumOfSquares (long n) {
  if (n == 0)
    return 0;
  else
    return n * n + sumOfSquares(n - 1);
}
\end{minted}

\begin{enumerate}

%%
% (a)
%%

\item Beweisen Sie formal mittels vollständiger Induktion:

\begin{displaymath}
\forall n \in \mathbb{N} : \texttt{sumOfSquares(n)} =
\frac{n(n + 1)(2n + 1)}{6}
\end{displaymath}

\begin{antwort}
Sei $f(n): \frac{n(n + 1)(2n + 1)}{6}$

%%
%
%%

\ueberschrift{Induktionsanfang}

Für $n = 0$ gilt:

$\texttt{sumOfSquares(0)} \overset{\texttt{if}}{=} 0 = f(0)$

%%
%
%%

\ueberschrift{Induktionshypothese}

Für ein festes $n \in \mathbb{N}$ gelte:

$\texttt{sumOfSquares(0)} = f(n)$

%%
%
%%

\ueberschrift{Induktionsschritt}

% https://mathcs.org/analysis/reals/infinity/answers/sm_sq_cb.html

$n \rightarrow n + 1$

$\texttt{sumOfSquares(n+1)} \overset{\texttt{else}}{=}$

$\texttt{(n+1)*(n+1)*sumOfSquares(n)} \overset{\texttt{I.H.}}{=}$

$(n + 1) \cdot (n + 1) + f(n)$

$(n + 1)^2 + \frac{n(n + 1)(2n + 1)}{6}$

$\frac{6(n + 1)^2}{6} + \frac{n(n + 1)(2n + 1)}{6}$

$\frac{6(n + 1)^2 + n(n + 1)(2n + 1)}{6}$

$\frac{(n + 1) \cdot (6(n + 1) + n(2n + 1))}{6}$

$\frac{(n + 1) \cdot (6n + 6 + 2n^2 + n))}{6}$

$\frac{(n + 1) \cdot (2n^2 + 7n + 6))}{6}$
$\frac{(n + 1) \cdot (n + 2) (2n + 3))}{6}$

Neben $2n^2 + 7n + 6 = (n + 2) (2n + 3) = n \cdot 2n + 2 \cdot 2n + 3 \cdot n + 2 \cdot 6$

% $\frac{6(n^2 + 2 \cdot n \cdot 1 + 1^2) + n(n + 1)(2n + 1)}{6}$

% $\frac{6n^2 + 12n + 6 + n(n + 1)(2n + 1)}{6}$

% $\frac{6n^2 + 12n + 6 + (n^2 + n)(2n + 1)}{6}$

% $\frac{6n^2 + 12n + 6 + (n^2 + n)2n + (n^2 + n)1}{6}$

% $\frac{6n^2 + 12n + 6 + 2n^3 + 2n^2 + n^2 + n}{6}$

% $\frac{6n^2 + 13n + 6 + 2n^3 + 2n^2 + n^2 + n}{6}$

% $\frac{9n^2 + 6 + 2n^3 + n}{6}$

\end{antwort}

%%
% (b)
%%

\item Beweisen Sie die Terminierung von \java{sumOfSquares(n)} für alle
$n \geq 0$.

\begin{antwort}
Sei $T (n) = n$. Die Funktion $T (n)$ ist offenbar ganzzahlig. In jedem
Rekursionsschritt wird $n$ um eins verringert, somit ist $T (n)$ streng
monoton fallend. Durch die Abbruchbedingung \java{n==0} ist $T (n)$
insbesondere nach unten beschränkt. Somit ist $T$ eine gültige
Terminierungsfunktion.
\end{antwort}

\end{enumerate}
\end{document}
