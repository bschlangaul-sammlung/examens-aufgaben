\documentclass{lehramt-informatik-aufgabe}
\liLadePakete{}
\begin{document}
\liAufgabenTitel{Bayerischee Autobahnen}
\section{Aufgabe 1 (Graphalgorithmen)
\index{Algorithmus von Dijkstra}
\footcite{66115:2017:03}}

Die folgende Abbildung zeigt die wichtigsten bayerischen Autobahnen
zusammen mit einigen anliegenden Orten und die Entfernungen zwischen
diesen.

Entfernungstabelle

\begin{tabular}{lll}
von & nach & km\\
Würzburg & Nürnberg & 115\\
Nürnberg & Regensburg & 105\\
Regensburg & AK Deggendorf & 70\\
AK Deggendorf & Passau & 50\\
Hof & Nürnberg & 135\\
Nürnberg & Ingolstadt & 90\\
Ingolstadt & AD Holledau & 20\\
AD Holledau & München & 50\\
München & AK Deggendorf & 140\\
Hof & Regensburg & 170\\
Regensburg & AD Holledau & 70\\
\end{tabular}

\begin{enumerate}

%%
% a)
%%

\item Bestimmen Sie mit dem Algorithmus von Dijkstra den kürzesten Weg
von Ingolstadt zu allen anderen Orten. Verwenden Sie zur Lösung eine
Tabelle gemäß folgendem Muster und markieren Sie in jeder Zeile den
jeweils als nächstes zu betrachtenden Ort. Setzen Sie für die noch zu
bearbeitenden Orte eine Prioritätswarteschlange ein, d.h. bei gleicher
Entfernung wird der ältere Knoten gewählt.

Ergebnis:

%%
% b)
%%

\item Die bayerische Landesregierung hat beschlossen, die eben
betrachteten Orte mit einem breitbandigen Glasfaser-Backbone entlang der
Autobahnen zu verbinden. Dabei soll aus Kostengründen so wenig Glasfaser
wie möglich verlegt werden. Identifizieren Sie mit dem Algorithmus von
Kruskal diejenigen Strecken, entlang welcher Glasfaser verlegt werden
muss. Geben Sie die Ortspaare (Autobahnsegmente) in der Reihenfolge an,
in der Sie sie in Ihre Verkabelungsliste aufnehmen.
\index{Algorithmus von Kruskal}

\item Um Touristen den Besuch aller Orte so zu ermöglichen, dass sie
dabei jeden Autobahnabschnitt genau einmal befahren müssen, bedarf es
zumindest eines sogenannten offenen Eulerzugs. Zwischen welchen zwei
Orten würden Sie eine Autobahn bauen, damit das bayerische Autobahnnetz
mindestens einen Euler-Pfad enthält?
\end{enumerate}
\end{document}
