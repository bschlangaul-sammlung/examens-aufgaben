\documentclass{lehramt-informatik-aufgabe}
\liLadePakete{formale-sprachen}
\begin{document}
\liAufgabenTitel{Aussagen}
\section{Aufgabe 5
\index{Formale Sprachen}
\footcite{examen:66115:2017:03}}

Zeigen oder widerlegen Sie die folgenden Aussagen (die jeweiligen
Beweise sind sehr kurz):
\footcite[Aufgabe 6]{theo:ab:5}

\begin{enumerate}

%%
% (a)
%%

\item Alle regulären Sprachen liegen in NP.

\begin{liAntwort}
Stimmt. Alle regulären Sprachen sind in Polynomialzeit entscheidbar (es
existiert ein Automat dazu), sie liegen als in P und folglich auch in
NP.
\end{liAntwort}

%%
% (b)
%%

\item Es gibt Sprachen $A$, $B$ mit $A \subseteq B$, sodass $B$ regulär
und $A$ kontextfrei ist.

\begin{liAntwort}
Stimmt. Es existieren Sprachen mit der Eigenschaft wie gefordert. Wir
wählen: $B = (a|b)^*$ und \liAusdruck[A]{a^n b^n}{n \in \mathbb{N}}. A
ist bekanntermaßen nicht regulär, wie man mit dem Pumping Lemma beweisen
kann, kann aber durch eine Grammatik \liGrammatik{alphabet={a,b},
produktionen={S -> aSb | EPSILON}} erzeugt werden. Für $B$ gibt es einen
deterministischen endlichen Automaten.
\end{liAntwort}

%%
% (c)
%%

\item Es gibt unentscheidbare Sprachen $L$ über den Alphabet $\Sigma$,
so dass sowohl $L$ als auch das Komplement $\overline{L} = \Sigma^*
\setminus L$ rekursiv aufzählbar (= partiell entscheidbar) sind.

\begin{liAntwort}
Stimmt nicht. Ist $L$ und sein Komplement rekursiv aufzählbar, so können
wir $L$ entscheiden, denn wir haben eine Maschine, die auf Eingabe $x$ hält
und akzeptiert, wenn $x \in L$ ist, sowie eine Maschine die hält, wenn
$x$ und akzeptiert, wenn $x \notin L$ ist. Daraus lässt sich eine
Maschine konstruieren, die $L$ entscheidet.
\end{liAntwort}

%%
% (d)
%%

\item Sei $L$ eine beliebige kontextfreie Sprache über dem Alphabet
$\Sigma$. Dann ist das Komplement $\overline{L} = \Sigma^* \setminus L$
entscheidbar.

\begin{liAntwort}
Stimmt. Es gibt einen Entscheider für die Sprache $L$. Dieser
entscheidet für eine Eingabe $x$, ob diese in $L$ ist oder nicht.
Negiert man diese Entscheidung, so ergibt sich ein Entscheider für
$\overline{L}$.
\end{liAntwort}

Schreiben Sie zuerst zur Aussage \emph{„Stimmt“} oder \emph{„Stimmt
nicht“} und dann
Ihre Begründung.

\end{enumerate}
\end{document}
