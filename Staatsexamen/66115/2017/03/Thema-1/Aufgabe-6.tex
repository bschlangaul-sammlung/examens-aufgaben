\documentclass{lehramt-informatik-aufgabe}
\liLadePakete{mathe}
\begin{document}
\liAufgabenTitel{Turing}
\section{Aufgabe 6
\index{Entscheidbarkeit}
\footcite{66115:2017:03}}

Es sei $E$ die Menge aller (geeignet codierten) Turingmaschinen $M$ mit
folgender Eigenschaft: Es gibt eine Eingabe $w$, so dass $M$ gestartet
auf $w$ mindestens 1000 Schritte rechnet und dann irgendwann hält.

Das Halteproblem auf leerer Eingabe $H_0$ ist definiert als die Menge
aller Turingmaschinen, die auf leerer Eingabe gestartet, irgendwann
halten.

\begin{enumerate}

%%
% a)
%%

\item Zeigen Sie, dass $E$ unentscheidbar ist (etwa durch Reduktion vom
Halteproblem $H_0$).

\begin{liAntwort}
Dazu definieren wir die Funktion $f : \Sigma^* \rightarrow \Sigma^*$ wie
folgt:

\begin{equation*}
f(u) =
\begin{cases}
c(M’) &
\text{falls }u = c(M')w\text{ ist für eine Turingmaschine }M\text{ und Eingabe }w\\

0 & \text{sonst}
\end{cases}
\end{equation*}

Dabei sei M’ eine Turingmaschine, die sich wie folgt verhält:
\begin{enumerate}
\item Geht 1000 Schritte nach rechts
\item Schreibt festes Wort w (für M’ ist w demnach fest!)
\item Startet M
\end{enumerate}

\begin{itemize}
\item total

\item berechenbar: Syntaxcheck, 1000 Schritte über 1000 weitere Zustände
realisierbar

\item Korrektheit: $u \in L_{halt} \Leftrightarrow u = c(M)w$ für TM $M$, die auf $w$ hält
$\Leftrightarrow f(u) = c(M')$,
wobei $M'$ 1000 Schritte macht und dann hält $\Leftrightarrow f(u) \in L$
\end{itemize}
\end{liAntwort}
\footcite[Seite 54]{theo:fs:4}

%%
% b)
%%

\item Begründen Sie, dass $E$ partiell entscheidbar ist.

%%
% c)
%%

\item Geben Sie ein Problem an, welches nicht einmal partiell
entscheidbar ist.

\end{enumerate}
\end{document}
