\documentclass{lehramt-informatik-aufgabe}
\liLadePakete{formale-sprachen}
\begin{document}
\liAufgabenTitel{Berechen- und Entscheidbarkeit}
\section{Aufgabe 3
\index{Berechenbarkeit}
\footcite{66115:2017:03}}

\begin{enumerate}

%%
% 1.
%%

\item Primitiv rekursive Funktionen

\begin{enumerate}

%%
% a)
%%

\item Zeigen Sie, dass die folgendermaßen definierte Funktion
if: $\mathbb{N} \times \mathbb{N} \times \mathbb{N}
\mathbb{N}$ primitiv rekursiv ist.

sonst

%%
% b)
%%

\item Wir nehmen eine primitiv rekursive Funktionp: NN an und definieren
g(n) als die Funktion, welche die größte Zahl i < n zurückliefert, für
die p(/) = 0 gilt. Falls kein solches i existiert, soll g(n) = 0 gelten:

a(n) = max ({i <n |p) = 0} U {0})

if (b, x, y) = ( falls b=0
\end{enumerate}

Zeigen Sie, dass g: N > N primitiv rekursiv ist. (Sie dürfen obige
Funktion if als primitiv rekursiv voraussetzen.)

\item Sei \liAlphabet{a, b, c} und $L \subseteq \Sigma^*$ mit
\liAusdruck{a^i b^i c^i}{i \in N}.
\begin{enumerate}

%%
% a)
%%

\item Beschreiben Sie eine Turingmaschine, welche die Sprache Z
entscheidet. Eine textuelle Beschreibung der Konstruktionsidee ist
ausreichend.

%%
% b)
%%

\item Geben Sie Zeit- und Speicherkomplexität (abhängig von der Länge
der Eingabe) Ihrer Turingmaschine an.
\end{enumerate}

\item Sei \& = {0, 1}. Jedes we L* kodiert eine Turingmaschine M,. Die
von M, berechnete Funktion bezeichnen wir mit gy.

\begin{enumerate}

%%
% a)
%%

\item Warum ist {w e £* | 3x: @,(\&) = xx} nicht entscheidbar?

%%
% b)
%%

\item Warum ist {w e Z* | 3x: w= xx} entscheidbar?

\end{enumerate}
\end{enumerate}
\end{document}
