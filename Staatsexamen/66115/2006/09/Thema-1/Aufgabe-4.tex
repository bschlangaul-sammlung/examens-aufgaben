\documentclass{lehramt-informatik-aufgabe}
\liLadePakete{baum}
\begin{document}
\liAufgabenTitel{AVL-Baum mit 6, 13, 4, 8, 11, 9, 10}

\section{Aufgabe Herbst 2006, Thema 1
\index{AVL-Baum}
\footcite[Thema 1 Aufgabe 4 Seite 2]{examen:66115:2006:09}}

\begin{enumerate}
\item Gegeben sei die folgende Folge ganzer Zahlen: 6, 13, 4, 8, 11, 9, 10.

\begin{enumerate}
\item Fügen Sie obige Zahlen der Reihe nach in einen anfangs leeren
AVL-Baum ein und stellen Sie den Baum nach jedem Einfügeschritt dar!

\item Löschen Sie das Wurzelelement des entstandenen AVL-Baums und
stellen Sie die AVL-Eigenschaft wieder her!
\end{enumerate}
\end{enumerate}

\NewDocumentCommand{\tmpEinfuegen}{ m }{
  \bigskip
  \noindent
  \textbf{Einfügen von $#1$:}
}

\NewDocumentCommand{\tmpAvl}{ m m } {
  \bigskip
  \noindent
  \begin{minipage}{6cm}
  \begin{tikzpicture}[li binaer baum]
   #1
  \end{tikzpicture}
  \end{minipage}
  \begin{minipage}{6cm}
  #2
  \end{minipage}
}

%-----------------------------------------------------------------------
%
%-----------------------------------------------------------------------

\tmpEinfuegen{6}

%%
%
%%

\tmpAvl
{\Tree[.6 ]}
{$6$ als Wurzel eingefügt}

%-----------------------------------------------------------------------
%
%-----------------------------------------------------------------------

\tmpEinfuegen{13}

%%
%
%%

\tmpAvl
{\Tree
[.6
  \edge[blank]; \node[blank]{};
  \edge[]; {13}
]
}
{$13 > 6$ deshalb rechts angehängt}

%-----------------------------------------------------------------------
%
%-----------------------------------------------------------------------

\tmpEinfuegen{4}

%%
%
%%

\tmpAvl
{\Tree
[.6
  [.4 ]
  [.13 ]
]}
{$4 < 6$ deshalb links angehängt}

%-----------------------------------------------------------------------
%
%-----------------------------------------------------------------------

\tmpEinfuegen{8}

%%
%
%%

\tmpAvl
{\Tree
[.6
  [.4 ]
  [.13
    \edge[]; {8}
    \edge[blank]; \node[blank]{};
  ]
]}
{$8 > 6$ und $8 < 13$}

%-----------------------------------------------------------------------
%
%-----------------------------------------------------------------------

\tmpEinfuegen{11}

%%
%
%%

\tmpAvl
{\Tree
[.6
  [.4 ]
  [.13
    [.8
      \edge[blank]; \node[blank]{};
      \edge[]; {11}
    ]
    \edge[blank]; \node[blank]{};
  ]
]}
{
  $11 > 6$, $11 < 13$ und $11 > 8$;
  die Balancierung ist jetzt verletzt, mit unterschiedlichen Vorzeichen ($+1$, $-2$)
  deshalb Doppelrotation links-rechts
}

%%
%
%%

\tmpAvl
{\Tree
[.6
  [.4 ]
  [.13
    [.11
      \edge[]; {8}
      \edge[blank]; \node[blank]{};
    ]
    \edge[blank]; \node[blank]{};
  ]
]}
{
  Balancierung wäre nach Linksrotation weiter verletzt, deshalb zusätzliche Rechtsrotation
}

%%
%
%%

\tmpAvl
{\Tree
[.6
  [.4 ]
  [.11
    [.8 ]
    [.13 ]
  ]
]}
{Balancierung wiederhergestellt}

%-----------------------------------------------------------------------
%
%-----------------------------------------------------------------------

\tmpEinfuegen{9}

%%
%
%%

\tmpAvl
{\Tree
[.6
  [.4 ]
  [.11
    [.8
      \edge[blank]; \node[blank]{};
      \edge[]; {9}
    ]
    [.13 ]
  ]
]}
{
  $9 > 8$;
  Balancierung verletzt mit unterschiedlichen Vorzeichen ($-1$, $+2$),
  deshalb Doppelrotation rechts-links
}

%%
%
%%

\tmpAvl
{\Tree
[.6
  [.4 ]
  [.8
    [.9 ]
    [.11
      \edge[blank]; \node[blank]{};
      \edge[]; {13}
    ]
  ]
]}
{
  Balancierung wäre nach Rechtsrotation weiter verletzt,
  deshalb zusätzliche Linksrotation
}

%%
%
%%

\tmpAvl
{\Tree
[.8
  [.6
    \edge[]; {4}
    \edge[blank]; \node[blank]{};
  ]
  [.11
    [.9 ]
    [.13 ]
  ]
]}
{
  Balancierung wiederhergestellt
}

%-----------------------------------------------------------------------
%
%-----------------------------------------------------------------------

\tmpEinfuegen{10}

%%
%
%%

\tmpAvl
{\Tree
[.8
  [.6
    \edge[]; {4}
    \edge[blank]; \node[blank]{};
  ]
  [.11
    [.9
      \edge[blank]; \node[blank]{};
      \edge[]; {10}
    ]
    [.13 ]
  ]
]}
{
  $10 > 8$
}
\end{document}
