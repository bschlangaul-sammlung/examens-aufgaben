\documentclass{lehramt-informatik-aufgabe}
\liLadePakete{baum}
\begin{document}
\liAufgabenTitel{Heap und binärer Suchbaum}
\section{7. Aufgabe: Heap und binärer Suchbaum
\index{Binärbaum}\index{Halde (Heap)}
\footcite{66115:2013:09}}

\begin{enumerate}

\item \strut

\begin{enumerate}

%%
% i)
%%

\item Fügen Sie nacheinander die Zahlen 7, 1, 12, 8, 10, 3, 5 in einen
leeren binären Suchbaum ein und zeichnen Sie den Suchbaum nach „8“ und
nach „3“.

\begin{liDiagramm}{Nach Einfügen von „8“}
\begin{tikzpicture}[li binaer baum]
\Tree
[.7
  [.1 ]
  [.12
    [.8 ]
    \edge[blank]; \node[blank]{};
  ]
]
\end{tikzpicture}
\end{liDiagramm}

\begin{liDiagramm}{Nach Einfügen von „3“}
\begin{tikzpicture}[li binaer baum]
\Tree
[.7
  [.1
    \edge[blank]; \node[blank]{};
    [.3 ]
  ]
  [.12
    [.8
      \edge[blank]; \node[blank]{};
      [.10 ]
    ]
    \edge[blank]; \node[blank]{};
  ]
]
\end{tikzpicture}
\end{liDiagramm}

\begin{liDiagramm}{Nach Einfügen von „5“}
\begin{tikzpicture}[li binaer baum]
\Tree
[.7
  [.1
    \edge[blank]; \node[blank]{};
    [.3
      \edge[blank]; \node[blank]{};
      [.5 ]
    ]
  ]
  [.12
    [.8
      \edge[blank]; \node[blank]{};
      [.10 ]
    ]
    \edge[blank]; \node[blank]{};
  ]
]
\end{tikzpicture}
\end{liDiagramm}

%%
% ii)
%%

\item Löschen Sie die „1“ aus dem in (i) erstellten Suchbaum und
zeichnen Sie den Suchbaum.

\begin{liDiagramm}{Nach Löschen von „1“}
\begin{tikzpicture}[li binaer baum]
\Tree
[.7
  [.3
    \edge[blank]; \node[blank]{};
    [.5 ]
  ]
  [.12
    [.8
      \edge[blank]; \node[blank]{};
      [.10 ]
    ]
    \edge[blank]; \node[blank]{};
  ]
]
\end{tikzpicture}
\end{liDiagramm}

%%
% iii)
%%

\item Fügen Sie 7, 1, 12, 8, 10, 3, 5 in einen leeren MIN-Heap ein,
der bzgl. „$\leq$“ angeordnet ist.
Geben Sie den Heap nach jedem Element an.
\end{enumerate}

%%
% b)
%%

\item Was ist die worst-case Laufzeit in O-Notation
für das Einfügen eines Elements in einen Heap der Größe n?
Begründen Sie ihre Antwort.
\end{enumerate}

\end{document}
