\documentclass{lehramt-informatik-aufgabe}
\liLadePakete{baum}
\begin{document}
\liAufgabenTitel{AVL-Baum 12 5 20 2 9 16 25 3 21}
\section{8. Aufgabe: AVL-Bäume
\index{AVL-Baum}
\footcite{66115:2013:09}}

Gegeben sei der folgende AVL-Baum $T$. Führen Sie auf $T$ folgende
Operationen durch.

\begin{liProjektSprache}{Baum}
baum avl (
  setze: 12 5 20 2 9 16 25 3 21;
  drucke;
)
\end{liProjektSprache}

\begin{center}
\begin{tikzpicture}[li binaer baum]
\Tree
[.\node[label=0]{12};
  [.\node[label=-1]{5};
    [.\node[label=+1]{2};
      \edge[blank]; \node[blank]{};
      [.\node[label=0]{3}; ]
    ]
    [.\node[label=0]{9}; ]
  ]
  [.\node[label=+1]{20};
    [.\node[label=0]{16}; ]
    [.\node[label=-1]{25};
      [.\node[label=0]{21}; ]
      \edge[blank]; \node[blank]{};
    ]
  ]
]
\end{tikzpicture}
\end{center}

\begin{enumerate}

%%
% (a)
%%

\item Fügen Sie den Wert $22$ in $T$ ein. Balancieren Sie falls nötig und
geben Sie den entstandenen Baum (als Zeichnung) an.

\begin{liDiagramm}{Einfügen von „22“}
\begin{tikzpicture}[li binaer baum]
\Tree
[.\node[label=0]{12};
  [.\node[label=-1]{5};
    [.\node[label=+1]{2};
      \edge[blank]; \node[blank]{};
      [.\node[label=0]{3}; ]
    ]
    [.\node[label=0]{9}; ]
  ]
  [.\node[label=+2]{20};
    [.\node[label=0]{16}; ]
    [.\node[label=-2]{25};
      [.\node[label=+1]{21};
        \edge[blank]; \node[blank]{};
        [.\node[label=0]{22}; ]
      ]
      \edge[blank]; \node[blank]{};
    ]
  ]
]
\end{tikzpicture}
\end{liDiagramm}

\begin{liDiagramm}{Linksrotation}
\begin{tikzpicture}[li binaer baum]
\Tree
[.\node[label=0]{12};
  [.\node[label=-1]{5};
    [.\node[label=+1]{2};
      \edge[blank]; \node[blank]{};
      [.\node[label=0]{3}; ]
    ]
    [.\node[label=0]{9}; ]
  ]
  [.\node[label=+2]{20};
    [.\node[label=0]{16}; ]
    [.\node[label=-2]{25};
      [.\node[label=-1]{22};
        [.\node[label=0]{21}; ]
        \edge[blank]; \node[blank]{};
      ]
      \edge[blank]; \node[blank]{};
    ]
  ]
]
\end{tikzpicture}
\end{liDiagramm}

\begin{liDiagramm}{Rechtsrotation}
\begin{tikzpicture}[li binaer baum]
\Tree
[.\node[label=0]{12};
  [.\node[label=-1]{5};
    [.\node[label=+1]{2};
      \edge[blank]; \node[blank]{};
      [.\node[label=0]{3}; ]
    ]
    [.\node[label=0]{9}; ]
  ]
  [.\node[label=+1]{20};
    [.\node[label=0]{16}; ]
    [.\node[label=0]{22};
      [.\node[label=0]{21}; ]
      [.\node[label=0]{25}; ]
    ]
  ]
]
\end{tikzpicture}
\end{liDiagramm}

%%
% (b)
%%

\item Löschen Sie danach die $5$. Balancieren Sie $T$ falls nötig und
geben Sie den entstandenen Baum (als Zeichnung) an.
\end{enumerate}

\begin{liDiagramm}{Löschen von „5“}
\begin{tikzpicture}[li binaer baum]
\Tree
[.\node[label=0]{12};
  [.\node[label=-2]{9};
    [.\node[label=+1]{2};
      \edge[blank]; \node[blank]{};
      [.\node[label=0]{3}; ]
    ]
    \edge[blank]; \node[blank]{};
  ]
  [.\node[label=+1]{20};
    [.\node[label=0]{16}; ]
    [.\node[label=0]{22};
      [.\node[label=0]{21}; ]
      [.\node[label=0]{25}; ]
    ]
  ]
]
\end{tikzpicture}
\end{liDiagramm}

\begin{liDiagramm}{Linksrotation}
\begin{tikzpicture}[li binaer baum]
\Tree
[.\node[label=0]{12};
  [.\node[label=-2]{9};
    [.\node[label=-1]{3};
      [.\node[label=0]{2}; ]
      \edge[blank]; \node[blank]{};
    ]
    \edge[blank]; \node[blank]{};
  ]
  [.\node[label=+1]{20};
    [.\node[label=0]{16}; ]
    [.\node[label=0]{22};
      [.\node[label=0]{21}; ]
      [.\node[label=0]{25}; ]
    ]
  ]
]
\end{tikzpicture}
\end{liDiagramm}

\begin{liDiagramm}{Rechtsrotation}
\begin{tikzpicture}[li binaer baum]
\Tree
[.\node[label=+1]{12};
  [.\node[label=0]{3};
    [.\node[label=0]{2}; ]
    [.\node[label=0]{9}; ]
  ]
  [.\node[label=+1]{20};
    [.\node[label=0]{16}; ]
    [.\node[label=0]{22};
      [.\node[label=0]{21}; ]
      [.\node[label=0]{25}; ]
    ]
  ]
]
\end{tikzpicture}
\end{liDiagramm}

\end{document}
