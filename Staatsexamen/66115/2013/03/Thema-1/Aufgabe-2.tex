\documentclass{lehramt-informatik-aufgabe}
\liLadePakete{formale-sprachen,cyk-algorithmus}
\begin{document}
\let\l=\liKurzeTabellenLinie

\liAufgabenTitel{Kontextfreie Grammatiken}
\section{Aufgabe 2
\index{Kontextfreie Sprache}
\footcite{66115:2013:03}}

Gegeben sei die Grammatik \liGrammatik{variablen={S, A, B, C},
alphabet={a,b}} und

\begin{liProduktionen}
S -> A B,
S -> C S,
A -> B C,
A -> B B,
A -> a,
B -> A C,
B -> b,
C -> A A,
C -> B A
\end{liProduktionen}
\liFlaci{Gr46a6j0a}

$L = L(G)$ ist die von G erzeugte Sprache.

\begin{enumerate}
%%
% a)
%%

\item Zeigen Sie, dass $G$ mehrdeutig ist.

%%
% b)
%%

\item Entscheiden Sie mithilfe des Algorithmus von Cocke, Younger und
Kasami (CYK), ob das Wort $w = babaaa$ zur Sprache L gehört. Begründen
Sie Ihre Entscheidung.
\index{CYK-Algorithmus}

\begin{liAntwort}
\begin{tabular}{|c|c|c|c|c|c|}
b     & a     & b    & a    & a    & a \\\hline\hline

B     & A     & B    & A    & A    & A \l6
C     & S     & C    & C    & C \l5
-     & B     & A    & B \l4
A     & C     & A,C \l3
A,C   & B,C,A \l3
A,C,B \l1
\end{tabular}
\end{liAntwort}

%%
% c)
%%

\item Geben Sie eine Ableitung für $w = babaaa$ an.
\end{enumerate}

\end{document}
