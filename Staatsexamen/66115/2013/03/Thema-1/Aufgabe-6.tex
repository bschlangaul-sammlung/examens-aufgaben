\documentclass{lehramt-informatik-aufgabe}
\liLadePakete{syntax,grafik}
\begin{document}
\liAufgabenTitel{IP und ULR mit Hashes}
\section{Aufgabe 6 Streutabellen (Hash-Tables)
\index{Streutabellen (Hashing)}
\footcite{66115:2013:03}}

Um die URL (zum Beispiel google.de) und die zugehörige IP des Servers
(hier 173.194.69.9) zu verwalten, werden Streutabellen verwendet, die
eine bestimmte Zone von Adressen abbilden. Die Streutabellen werden als
zwei dynamische Arrays (in Java: ArrayLists) realisiert. Kollisionen
innerhalb einer Zone werden ebenfalls in dynamischen Arrays verwaltet.

\usetikzlibrary {shapes.multipart,positioning}

\def\TmpTabelle#1#2#3{
  \node[tabelle,label=+270:#1] (#1) at (#2,#3) {
    Streuwert 0
    \nodepart[text height=3cm]{two}Streuwert 1
    \nodepart{three}…
    \nodepart{four}Streuwert 45
  };
}

\def\TmpListeEintrag#1#2#3#4#5{
  \node[liste,#5=of #1] (#4) {#2\nodepart{two}Index #3};
  \draw (#4) -- (#1);
}

\begin{center}
\begin{tikzpicture}[
  scale=0.7,transform shape,
  li hash/.style={
    rectangle split,
    draw,
    anchor=center
  },
  liste/.style={
    li hash,
    rectangle split parts=2,
    rectangle split horizontal,
  },
  tabelle/.style={
    li hash,
    rectangle split parts=4,
    inner ysep=4mm,
    inner xsep=3mm,
  }
]

\TmpTabelle{url}{2}{0}

\node[rotate=90] at (4.5,0){Zone};

\TmpTabelle{ip}{7}{0}

\TmpListeEintrag{url.one west}{google.de}{0}{google}{left}
\TmpListeEintrag{url.two west}{bayern.de}{0}{bayern}{left}

\TmpListeEintrag{ip.one east}{173.194.69.9}{0}{173}{right}
\TmpListeEintrag{ip.two east}{195.200.71.1}{0}{195}{right}

\node[left=3.5cm of url.four east] (facebook) {
\begin{tabular}{|c|c|}
\hline
facebook.com & Index 0 \\\hline
gmx.net & Index 1 \\\hline
\end{tabular}
};
\draw (facebook) -- (url.four west);

\node[right=1cm of ip.four east] (69) {
\begin{tabular}{|c|c|}
\hline
69.171.229.1 & Index 0 \\\hline
213.165.64.7 & Index 1 \\\hline
\end{tabular}
};
\draw (69) -- (ip.four east);

\end{tikzpicture}
\end{center}

\noindent
Um zu einer URL die IP zu finden, berechnet man zunächst mittels der
Funktion \liJavaCode{hash()} den entsprechenden Streuwert, entnimmt dann den
Index der Tabelle URL und sucht schließlich an entsprechender Stelle in
der Tabelle IP die IP-Adresse.

\begin{enumerate}

%%
% a)
%%

\item Erläutern Sie am vorgestellten Beispiel, wie ein Hash-Verfahren
zum Speichern großer Datenmengen prinzipiell funktioniert und welche
Voraussetzungen und Bedingungen daran geknüpft sind.

%%
% b)
%%

\item Nun implementieren Sie Teile dieser IP- und URL-Verwaltung in
einer objektorientierten Sprache Ihrer Wahl. Verwenden Sie dabei die
folgende Klasse (die Vorgaben sind in der Sprache Java gehalten):

\begin{minted}{java}
class Zone {
private ArrayList<ArrayList<String>> urlList =
  new ArrayList<ArrayList<String>>();
private ArrayList<ArrayList<String>> ipList =
  new ArrayList<ArrayList<String>>();
public int hash(String url) {/* calculates hash-value h, >=0 */}
}
\end{minted}

\begin{enumerate}
%%
% i)
%%

\item Prüfen Sie in einer Methode \liJavaCode{boolean exists(int h)} der
Klasse \liJavaCode{zone}, ob bereits mindestens ein Eintrag für einen
gegebenen Streuwert vorhanden ist. Falls \liJavaCode{h} größer ist als die
derzeitige Größe der Streutabelle, existiert der Eintrag nicht.

\begin{liAntwort}
\liJavaExamen[firstline=56,lastline=76]{66115}{2013}{03}{Zone}
\end{liAntwort}

%%
% ii)
%%

\item Die Methode \liJavaCode{int getIndex (string url, ArrayList<String>
urlList)} soll den Index einer URL in der Kollisionsliste berechnen. Ist
die URL in der Kollisionsliste nicht vorhanden, soll \liJavaCode{-1}
zurückgeliefert werden.

\begin{liAntwort}
\liJavaExamen[firstline=68,lastline=88]{66115}{2013}{03}{Zone}
\end{liAntwort}

%%
% iii)
%%

\item Ergänzen Sie die Klasse Zone um eine Methode \liJavaCode{String lookup
(String url)}, die in der Streutabelle die IP-Adresse zur \liJavaCode{url}
zurückgibt. Wird eine nicht vorhandene Adresse abgerufen, wird eine
Fehlermeldung zurückgegeben.

\begin{liAntwort}
\liJavaExamen[firstline=90,lastline=104]{66115}{2013}{03}{Zone}
\end{liAntwort}

\end{enumerate}
\end{enumerate}

\begin{liAdditum}
\liPseudoUeberschrift{Additum}

\liPseudoUeberschrift{Komplette Java-Datei}

\liJavaExamen{66115}{2013}{03}{Zone}

\liPseudoUeberschrift{Test-Datei}

\liJavaTestDatei[firstline=3]{examen/examen_66115_2013_03/ZoneTest}
\end{liAdditum}
\end{document}
