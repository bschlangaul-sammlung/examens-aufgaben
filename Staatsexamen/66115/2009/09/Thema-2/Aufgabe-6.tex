\documentclass{lehramt-informatik-aufgabe}
\InformatikPakete{}
\begin{document}

\section{Aufgabe 6
\index{Greedy-Algorithmus}
\footcite[Thema 2 Aufgabe 6 Seite 6]{examen:66115:2009:09}}

Die Wäscheleinenaufgabe besteht darin, n Wäschestücke der Breiten bı,
ba,...,b„ auf Wäscheleinen der Breite b aufzuhängen. Idealerweise sollte
die Zahl der benutzten Leinen möglichst klein werden. Formal ist eine
Aufhängung der Wäsche auf ! Leinen also eine Einteilung der Menge
(1,...,n) in I! Klassen Lı,...,L,, sodass für alle j = 1...1 gilt Vier,
b; < b. Eine Lösung der Wäscheleinenaufgabe ist dann eine Zahl ! und
eine Aufhängung der Wäsche auf ! Leinen. Eine Lösung ist umso besser, je
kleiner / ist.

\begin{enumerate}

%%
% a)
%%

\item Beschreiben Sie einen sinnvollen Greedy-Algorithmus für das
Wäscheleinenproblem. (Also nicht einfach für jedes Wäschestück eine neue
Leine)

%%
% b)
%%

\item Geben Sie ein Beispiel einer Wäscheladung (Instanz des
Wäscheleinenproblems), für die Ihr Algorithmus mehr als die minimal
mögliche Zahl von Leinen verbraucht.

%%
% c)
%%

\item Nennen Sie ein Beispiel einer Problemstellung, die mit einem
Greedy-Algorithmus optimal gelöst werden kann.

\end{enumerate}

\end{document}
