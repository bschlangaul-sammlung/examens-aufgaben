\documentclass{lehramt-informatik-aufgabe}
\liLadePakete{syntax}
\begin{document}
\liAufgabenTitel{maximale Teilsumme}
\section{Aufgabe 4
\index{Teile-und-Herrsche (Divide-and-Conquer)}
\footcite{66115:2012:09}}

Gegeben ist ein Array a von ganzen Zahlen der Länge $n$, z.B.:

\begin{center}
\begin{tabular}{l|llllllllll}
i & 0 & 1 & 2 & 3 & 4 & 5 & 6 & 7 & 8 & 9\\\hline

a & 5 & -6 & 4 & 2 & -5 & 7 & -2 & -7 & 3 & 5\\
\end{tabular}
\end{center}

Im Beispiel ist also $n = 10$. Es soll die maximale Teilsumme berechnet
werden, also der Wert des

Ausdrucks
Ok
LIS max =

Im Beispiel ist dieser Wert $8$ und wird für $i = 8$,$j = 10$ erreicht.
Entwerfen Sie ein Divide-And-Conquer Verfahren, welches diese
Aufgabenstellung in Zeit $\mathcal{O}(n \log n)$ löst. Skizzieren Sie
Ihre Lösung hinreichend detailliert.

Tipp: Sie sollten ein geringfügig allgemeineres Problem lösen, welches
neben der maximalen Teilsumme auch noch die beiden „maximalen
Randsummen” berechnet. Die werden dann bei der Endausgabe verworfen.

\liJavaExamen{66115}{2012}{09}{Teilsumme}

\end{document}
