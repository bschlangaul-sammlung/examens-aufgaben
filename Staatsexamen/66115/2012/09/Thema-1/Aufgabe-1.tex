\documentclass{lehramt-informatik-aufgabe}
\liLadePakete{mathe,automaten}
\begin{document}
\liAufgabenTitel{NEA und Minimalisierung}
\section{Aufgabe 1
\index{AUD}
\footcite{theoinf:ab:1}}

Wir fixieren das Alphabet $\Sigma = \{a, b\}$ und definieren $L
\subseteq \Sigma^∗$ durch $L = \{ w \,|\, \text{in w kommt das Teilwort
\texttt{bab} vor} \}$ z.\,B. ist \texttt{babaabb} $\in L$, aber
\texttt{baabaabb} $\notin L$. Der folgende nichtdeterministische
Automat $A$ erkennt $L$:\footcite{examen:66115:2012:09}

\begin{center}
\begin{tikzpicture}[->,node distance=2cm]
\node[state,initial] (0) {$z_0$};
\node[state,right of=0] (1) {$z_1$};
\node[state,right of=1] (2) {$z_2$};
\node[state,right of=2,accepting] (3) {$z_3$};

\path (0) edge[above] node{b} (1);
\path (1) edge[above] node{a} (2);
\path (2) edge[above] node{b} (3);

\path (0) edge[loop,above] node{a,b} (0);

\path (3) edge[loop,above] node{a,b} (3);
\end{tikzpicture}
\end{center}

\begin{enumerate}
\item Wenden Sie die Potenzmengenkonstruktion auf den Automaten an und
geben Sie den resultierenden deterministischen Automaten an. Nicht
erreichbare Zustände sollen nicht dargestellt werden.

\begin{tabular}{l|l|l}

Zustandsmenge & Eingabe a & Eingabe b \\\hline

$\{z_0 \}$ & $\{z_0 \}$ & $\{z_0, z_1 \}$ \\

$\{z_0, z_1 \}$ & $\{z_0, z_2 \}$ & $\{z_0, z_1 \}$ \\

$\{z_0, z_2 \}$ & $\{z_0, \}$ & $\{z_0, z_1, z_3 \}$ \\

$\{z_0, z_1, z_3 \}$ & $\{z_0, z_2, z_3 \}$ & $\{z_0, z_1, z_3 \}$ \\

$\{z_0, z_2, z_3 \}$ & $\{z_0, z_3 \}$ &  $\{z_0, z_1, z_3 \}$ \\

$\{z_0, z_3 \}$ & $\{z_0, z_3 \}$ & $\{z_0, z_1, z_3 \}$\\

\end{tabular}

\item Konstruieren Sie aus dem so erhaltenen deterministischen Automaten
den Minimalautomaten für $L$. Beschreiben Sie dabei die Arbeitsschritte
des verwendeten Algorithmus in nachvollziehbarer Weise.
\end{enumerate}

\end{document}
