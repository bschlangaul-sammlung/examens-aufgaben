\documentclass{lehramt-informatik-aufgabe}
\liLadePakete{baum}
\begin{document}
\liAufgabenTitel{Leere Heap}

\section{Aufgabe 7: Heap
\index{Halde (Heap)}
\footcite[Thema 2 Aufgabe 7]{66115:2012:09}
}

Fügen Sie nacheinander die Zahlen \texttt{3}, \texttt{5}, \texttt{1},
\texttt{2}, \texttt{4} in einen leeren Heap ein. Geben Sie die
Ergebnisse an (Zeichnung).
\footcite[Zeichnen der Heap)]{aud:ab:7}

\begin{minipage}[t][5cm][b]{0.32\linewidth}
\begin{tikzpicture}[li binaer baum]
\Tree
[.\node(3){3};
      [.\node(5){5}; ]
      \edge[blank]; \node[blank]{};
]
\draw[dotted,<->] (3) .. controls +(west:1) .. (5);
\end{tikzpicture}
Erstellen einer Max.-Halde, einfügen von 3 und 5, Versickern notwendig
\end{minipage}
%
\begin{minipage}[t][5cm][b]{0.32\linewidth}
\begin{tikzpicture}[li binaer baum]
\Tree
[.5
  [.\node(3){3};
    [.2 ]
    [.\node(4){4}; ]
  ]
  [.1 ]
]
\draw[dotted,<->] (3) .. controls +(east:0.5) .. (4);
\end{tikzpicture}
Einfügen von 1 und 2 ohne Änderungen, Einfügen von 4, versickern notwendig
\end{minipage}
%
\begin{minipage}[t][5cm][b]{0.32\linewidth}
\begin{tikzpicture}[li binaer baum]
\Tree
[.5
  [.4
    [.2 ]
    [.3 ]
  ]
  [.1 ]
]
\end{tikzpicture}
Fertiger Heap
\end{minipage}
\end{document}
