\documentclass{lehramt-informatik-minimal}
\InformatikPakete{mathe}
\begin{document}

\section{Aufgabe 4: Komplexität\index{Komplexität}\footcite{aud:ab:2}}

Gegeben seien die Funktionen $f : N \rightarrow N$ und $g : N \rightarrow N$ , wobei $f (n)
= (n - 1)^3$ und $g(n) = (2n + 3)(3n + 2)$. Geben Sie an, welche der
folgenden Aussagen gelten. Beweisen Sie Ihre Angaben.
\footcite[Herbst 2012 (66115) - Thema 2, Aufgabe 6: O-Notation]{examen:66115:2012:09}

\begin{enumerate}
\item $f(n) \in \mathcal{O}(g(n))$
\item $g(n) \in \mathcal{O}(f(n))$
\end{enumerate}

\begin{antwort}
Es gilt Aussage (b), da $f(n) \in \mathcal{O}(n^3)$ und $g(n) \in
\mathcal{O}(n^2)$ und der Grenzwert $\lim$ bei größer werdendem $n$
gegen $0$ geht. Damit wächst $f (n)$ stärker als $g(n)$, sodass nur
Aussage (b) gilt und nicht (a). Dafür nutzen wir die formale Definition
des $\mathcal{O}$-Kalküls, indem wir den Grenzwert $\frac{f}{g}$ bzw.
$\frac{g}{f}$ bilden:

{
\footnotesize
\begin{displaymath}
\lim_{n \to \infty}
\frac{f(n)}
{g(n)}
=
\lim_{n \to \infty}
\frac{(n - 1)^3}
{(2n + 3)(3n + 2)}
=
\lim_{n \to \infty}
\frac{3(n - 1)^2}
{(2n + 3) \cdot 3 + 2 \cdot (3n + 2)}
=
\lim_{n \to \infty}
\frac{6(n - 1)}
{12}
=
\infty
\end{displaymath}

\begin{displaymath}
\lim_{n \to \infty}
\frac{g(n)}
{f(n)}
=
\lim_{n \to \infty}
\frac{(2n + 3)(3n + 2)}
{(n - 1)^3}
=
\lim_{n \to \infty}
\frac{(2n + 3) \cdot 3 + 2 \cdot (3n + 2)}
{3(n - 1)^2}
=
\lim_{n \to \infty}
\frac{12}
{6(n - 1)}
=
0
\end{displaymath}
}
\end{antwort}

\end{document}
