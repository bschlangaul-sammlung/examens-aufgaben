\documentclass{lehramt-informatik-aufgabe}
\liLadePakete{formale-sprachen,chomsky-normalform}
\begin{document}
\let\m=\liMenge
\let\schrittE=\liChomskySchrittUeberschriftErklaerung

\liAufgabenTitel{Kontextfreie Sprache}
\section{Aufgabe 4
\index{Chomsky-Normalform}
\footcite{66115:2012:03}}

Gegeben ist die kontextfreie Grammatik $G = (\Sigma, N, S, R)$ mit
\liAlphabet{a,b}, $N = \m{S, A, B}$ und

\begin{liProduktionsRegeln}
S -> A,
S -> B,
A -> a A b,
B -> A A,
B -> b B a,
A -> a
\end{liProduktionsRegeln}
\liFussnoteUrl{https://flaci.com/Gr3rgt2vg}

Geben Sie eine äquivalente Grammatik in Chomsky-Normalform an.

\begin{liAntwort}
\begin{enumerate}
\item \schrittE{1}

\liNichtsZuTun

\item \schrittE{2}

\begin{liProduktionsRegeln}
S -> aAb | a | AA | bBa,
A -> aAb | a,
B -> AA | bBa,
\end{liProduktionsRegeln}

\item \schrittE{3}

\begin{liProduktionsRegeln}
S -> T_a C | T_a | A A | T_b B T_a,
A -> T_a C | T_a,
B -> AA | T_b B T_a,
T_a -> a,
T_b -> b,
\end{liProduktionsRegeln}

\item \schrittE{4}

% Ergebnis von Flaci.com:
% S   -> T1 S.1 | a | A A | T2 S.2
% A   -> T1 S.1 | a
% B   -> A A | T2 S.2
% T_a: T1  -> a
% T_b: T2  -> b
% C:   S.1 -> A T2
% D:   S.2 -> B T1

\begin{liProduktionsRegeln}
S -> T_a C | a | A A | T_b D,
A -> T_a C | a,
B -> A A | T_b D,
T_a -> a,
T_b -> b,
C -> A T_b,
D -> B T_a,
\end{liProduktionsRegeln}

\end{enumerate}
\end{liAntwort}

\end{document}
