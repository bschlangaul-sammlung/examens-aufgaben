\documentclass{bschlangaul-aufgabe}
\liLadePakete{formale-sprachen,automaten}
\begin{document}
\liAufgabenTitel{Kontextfrei aber nicht regulär}
\section{Aufgabe 3
\index{Kontextfreie Sprache}
\footcite{examen:66115:2012:03}}

% Info_2021-04-23-2021-04-23_13.17.40.mp4 3min

Beweisen Sie, dass folgende Sprache kontextfrei, aber nicht regulär ist.

\begin{center}
\liAusdruck[C]{a^n b^m}{ n \geq m \geq 1 }
\end{center}

\begin{liAntwort}
\liPseudoUeberschrift{Nachweis Kontextfrei über Grammatik}

\noindent
\liGrammatik{variablen={S}, alphabet={a, b}}

\begin{liProduktionsRegeln}
S -> a S b | a S | a b
\end{liProduktionsRegeln}

\begin{itemize}
\item Regel 1: $a S b$
\item Regel 2: $a S$
\item Regel 3: $a b$
\end{itemize}

\begin{description}
\item[$ab$:]
$S \xrightarrow{3} ab$

\item[$a^n b$:]
$S \xrightarrow[n-1]{2} a^{n-1} S \xrightarrow{3} a^{n-1}ab$

\item[$a^n b^m$:]
$S \xrightarrow[m-1]{1}
a^{m-1} S b^{m-1} \xrightarrow[n-(m-1)]{2}
a^{n-1}Sb^{m-1}\xrightarrow{3}
a^n b^m$

$\Rightarrow L(G) = C$
\end{description}

\liPseudoUeberschrift{Nachweis Kontextfrei über Kellerautomat}

\begin{center}
\begin{tikzpicture}[li kellerautomat]
  \node[state,initial] (z0) at (2.71cm,-2.57cm) {$z_0$};
  \node[state] (z1) at (5cm,-2.57cm) {$z_1$};
  \node[state,accepting] (z2) at (7.43cm,-2.57cm) {$z_2$};

  \liKellerKante[above,loop above]{z0}{z0}{
    a, KELLERBODEN, A KELLERBODEN;
    a, A, A A;
  }

  \liKellerKante[above]{z0}{z1}{
    b, A, EPSILON;
  }

  \liKellerKante[above]{z1}{z2}{
    EPSILON, KELLERBODEN, EPSILON;
    EPSILON, A, EPSILON;
  }

  \liKellerKante[above,loop above]{z1}{z1}{
    b, A, EPSILON;
  }
\end{tikzpicture}
\end{center}
\liFlaci{Aji151myg}

\liPseudoUeberschrift{Nachweis: C nicht regulär}

C sei regulär

$\Rightarrow$ Pumping-Lemma für C erfüllt

$j$ sei die Pumping-Zahl ($j \in \mathbb{N}$)

$\omega \in C$: $\omega = a^j b^j$

$\omega = u v w$

Dann gilt:

\begin{itemize}
\item $|v| \geq 1$
\item $|uv| \leq j$
\item $u v^i w \in C$ für alle $i \in \mathbb{N}_0$
\end{itemize}

In $uv$ können nur $a$’s vorkommen

$\Rightarrow$
In $v$ muss mindestens ein $a$ vorkommen

$\Rightarrow$
$u v^0 w = a^l (a^{j-l})^0 b^j$ ($(a^{j-l})^0 = \varepsilon$)

$\Rightarrow$
In $\omega'$ sind nur $l$ viele $a$’s, Da $l < j$, $\omega' \notin C$,

$\Rightarrow$
Widerspruch zur Annahme

$\Rightarrow$
C nicht regulär
\end{liAntwort}

\end{document}
