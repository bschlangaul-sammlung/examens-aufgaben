\documentclass{bschlangaul-aufgabe}
\bLadePakete{syntax}
\begin{document}
\bAufgabenTitel{Muffinsorten}

\section{Aufgabe 6 (Algorithmen und Datenstrukturen)
\index{Greedy-Algorithmus}
\footcite[Thema 1 Aufgabe 6 Seite 5]{examen:66115:2019:03}}

Aus dem Känguru-Wettbewerb 2017 — Klassenstufen 3 und 4.

\bigskip

\begin{quote}
Luna hat für den Kuchenbasar Muffins mitgebracht: $10$ Apfelmuffins,
$18$ Nussmuffins, $12$ Schokomuffins und $9$ Blaubeermuffins. Sie nimmt
immer $3$ verschiedene Muffins und legt sie auf einen Teller. Welches
ist die kleinste Zahl von Muffins, die dabei übrig bleiben können?
\end{quote}

\begin{center}
A: 1, B: 3, C: 4, D: 7, E: 8
\end{center}

\begin{enumerate}
\item Geben Sie die richtige Antwort auf die im Känguru-Wettbewerb
gestellte Frage und begründen Sie sie.

\begin{liAntwort}
4
\bFussnoteUrl{https://www.youtube.com/watch?v=ceJW9kAplVY}
\end{liAntwort}

\item Lunas Freundin empfiehlt den jeweils nächsten Teller immer aus den
drei aktuell häufigsten Muffinsorten zusammenzustellen. Leiten Sie aus
dieser Idee einen effizienten GreedyAlgorithmus her, der die
Fragestellung für beliebige Anzahlen von Muffins löst (nach wie vor soll
es nur vier Sorten und je drei pro Teller geben). Skizzieren Sie in
geeigneter Form, wie Ihr Algorithmus die Beispielinstanz von oben
richtig löst.

\begin{liAntwort}
\bJavaExamen[firstline=42,lastline=56]{66115}{2019}{03}{Muffin}
\end{liAntwort}

\item Beschreiben Sie eine mögliche und sinnvolle Verallgemeinerung
Ihrer Lösung auf $n$ Muffinsorten und $k$ Muffins pro Teller für $n > 4$
und $k > 3$.

\begin{liAntwort}
\bJavaExamen[firstline=67,lastline=77]{66115}{2019}{03}{Muffin}
\end{liAntwort}

\item Diskutieren Sie, wie man die Korrektheit des Greedy-Algorithmus
zeigen könnte, also dass er tatsächlich immer eine optimale Lösung
findet. Ein kompletter, rigoroser Beweis ist nicht verlangt.
\end{enumerate}

\begin{liAdditum}

Der komplette Code:

\bJavaExamen{66115}{2019}{03}{Muffin}
\end{liAdditum}
\end{document}
