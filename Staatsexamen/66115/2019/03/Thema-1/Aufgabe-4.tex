\documentclass{lehramt-informatik-aufgabe}
\liLadePakete{automaten,formale-sprachen}
\begin{document}
\let\l=\liTuringLeerzeichen
\liAufgabenTitel{}
\section{Aufgabe 4
\index{Turing-Maschine}
\footcite{66115:2019:03}}

Wir betrachten die Turingmaschine \liTuringMaschine[M]{}. Hierbei ist
die Zustandsmenge Q = {90, da; , 9, Qr} mit Startzustand go und
akzeptierenden Zuständen F = {q;}. Das Eingabealphabet ist © = {a,b,|},
das Bandalphabet ist T = ZU {U} mit Blank-Zeichen D für leeres Feld. Die
Ubergangsfunktion 6: Q xT > Q xI x {L, N, R}, wobei der
Schreib-Lese-Kopf mit L nach links, mit N nicht und mit R nach rechts
bewegt wird, ist durch folgende Tabelle gegeben (bspw. ist ö(go,a) =
(9,I,R)):

% a b | O

% do (da, I, R) (q, O, R) (as, |, N) (qf, O, N)
% da (da, a, R) (da, b, R) (Biss |, R) (az, a, L)
% db (9b a, R) (4, b, R) (9; |, R) (az, b, L)

% qL (91,0, L) (gr, b, L) (qr, |, L) (go, O, R)

\begin{center}
\begin{tikzpicture}[li turingmaschine]
  \node[state,initial] (q0) at (2.71cm,-4.71cm) {$q_0$};
  \node[state] (qa) at (5.86cm,-3.14cm) {$q_a$};
  \node[state] (qb) at (6cm,-6.86cm) {$q_b$};
  \node[state] (qL) at (8.86cm,-4.71cm) {$q_L$};
  \node[state,accepting] (qf) at (2.71cm,-7cm) {$q_f$};

  \liTuringKante[above left]{q0}{qa}{
    a, LEER, R;
  }

  \liTuringKante[left,pos=0.8]{q0}{qb}{
    b, LEER, R;
  }

  \liTuringKante[left]{q0}{qf}{
    |, |, N;
    LEER, LEER, N;
  }

  \liTuringKante[above,loop above]{qa}{qa}{
    a, a, R;
    b, b, R;
    |, |, R;
  }

  \liTuringKante[right,pos=0.1]{qa}{qL}{
    LEER, a, L;
  }

  \liTuringKante[above,loop below]{qb}{qb}{
    a, a, R;
    b, b, R;
    |, |, R;
  }

  \liTuringKante[below right]{qb}{qL}{
    LEER, b, L;
  }

  \liTuringKante[above,loop above]{qL}{qL}{
    a, a, L;
    b, b, L;
    |, |, L;
  }

  \liTuringKante[above]{qL}{q0}{
    LEER, LEER, R;
  }
\end{tikzpicture}
\end{center}
\liFlaci{Aj54q4rd9}

\begin{enumerate}

%%
% (a)
%%

\item Die Notation $(v,q,aw)$ beschreibt eine Konfiguration der
Turingmaschine: der interne Zustand ist $q$, der Schreib-Lesekopf steht
auf einem Feld mit $a \in \Gamma$, rechts vom Schreib-Lesekopf steht $w
\in \Gamma^*$, links vom Schreib-Lesekopf steht $v \in \Gamma^*$.

Vervollständigen Sie die Folge von Konfigurationen, die die
Turingmaschine bei Eingabe $ab|$ bis zum Erreichen des Zustands $q_f$
durchläuft. Sie können auch Ihre eigene Notation zur Darstellung von
Konfigurationen verwenden.

\def\p{ $\vdash$ \\}

\begin{liAntwort}
$(\l, q_0, a b | )$ \p % 0
% a entfernt
$(\l, q_a, \l b | )$ \p % 1
$(\l, q_a, b | )$ \p % 2
$(b, q_a, |)$ \p % 3
% a geschrieben
$(b|, q_L, a)$ \p % 4
$(b, q_L, | a)$ \p % 5
$(\l, q_L, \l b | a)$ \p % 6
$(\l, q_b, \l b | a)$ \p % 7
% b entfernt
$(\l, q_b, \l | a)$ \p % 8
$(\l, q_b, | a)$ \p % 9
$(|, q_b, a)$ \p % 10
% b geschrieben
$(|a, q_L, b)$ \p % 11
$(|, q_L, a b)$ \p % 12
$(\l, q_L, | a b)$ \p % 13
$(\l, q_0, \l | a b)$ \p % 14
$(\l, q_f, | a b)$ % 15
\end{liAntwort}

%%
% (b)
%%

\item Sei $w \in \{ a, b \}^*$ beliebig. Mit welchem Bandinhalt
terminiert die Turingmaschine bei Eingabe von $w|$? Geben Sie auch eine
kurze Begründung an.

\begin{liAntwort}
Die Turingmaschine terminiert bei alle möglichen Wörtern $w \in \{ a, b
\}^*$, auch bei dem leeren Wort vor $|$. Die Maschine verschiebt alle
Vorkommen von $a$’s und $b$’s vor dem Trennzeichen $|$ nach rechts. Ist
das Trennzeichen $|$ schließlich das erste Zeichen von links gesehen,
dann terminiert die Maschine.
\end{liAntwort}
\end{enumerate}
\end{document}
