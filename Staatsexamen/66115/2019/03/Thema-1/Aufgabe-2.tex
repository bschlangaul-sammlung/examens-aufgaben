\documentclass{lehramt-informatik-aufgabe}
\liLadePakete{formale-sprachen,automaten,potenzmengen-konstruktion}
\begin{document}

\def\z#1{
\liZustandsMengenSammlungNr{#1}{
    {
      {0} {0}
      {1} {0,1}
      {2} {0,1,2}
      {3} {0,2}
    }
  }
}
\let\s=\liZustandsnameGross

\liAufgabenTitel{}
\section{Aufgabe 2
\index{Reguläre Sprache}
\footcite{66115:2019:03}}

\begin{enumerate}
%%
% (a)
%%

\item Gegeben sei der nichtdeterministische endliche Automat $A$ über
dem Alphabet \liAlphabet{a, b}
wie folgt:

\begin{center}
\begin{tikzpicture}[li automat]
  \node[state,initial] (z0) at (3.71cm,-2.43cm) {$z_0$};
  \node[state] (z1) at (5.71cm,-2.43cm) {$z_1$};
  \node[state,accepting] (z2) at (8.14cm,-2.43cm) {$z_2$};

  \path (z0) edge[auto] node{$a$} (z1);
  \path (z0) edge[auto,loop above] node{$a,b$} (z0);
  \path (z1) edge[auto] node{$a,b$} (z2);
\end{tikzpicture}
\end{center}
\liFlaci{Arozq4rm2}

Konstruieren Sie einen deterministischen endlichen Automaten, der das
Komplement \liAusdruck[L(A)]{w \in \Sigma^*}{w \notin L(A)} der von $A$
akzeptierten Sprache $L(A)$ akzeptiert.

\begin{liAntwort}
Wir konvertieren zuerst den nichtdeterministischen endlichen Automaten in
einen deterministischen endlichen Automaten mit Hilfe des
Potenzmengenalgorithmus.

\begin{tabular}{l|l|l}
Zustandsmenge & Eingabe $a$ & Eingabe $b$ \\\hline
\z0 & \z1 & \z0 \\
\z1 & \z2 & \z3 \\
\z2 & \z2 & \z3 \\
\z3 & \z1 & \z0 \\
\end{tabular}

\begin{center}
\begin{tikzpicture}[li automat]
  \node[state,initial] (z0) at (2.14cm,-2.14cm) {\s0};
  \node[state,accepting] (z012) at (5.29cm,-5.29cm) {\s1};
  \node[state,accepting] (z02) at (2.14cm,-5.29cm) {\s3};
  \node[state] (z01) at (5.29cm,-2.14cm) {\s2};

  \path (z0) edge[auto] node{$a$} (z01);
  \path (z0) edge[auto,loop above] node{$b$} (z0);
  \path (z012) edge[auto,loop right] node{$a$} (z012);
  \path (z012) edge[auto] node{$b$} (z02);
  \path (z02) edge[auto,bend left] node{$a$} (z01);
  \path (z02) edge[auto] node{$b$} (z0);
  \path (z01) edge[auto] node{$a$} (z012);
  \path (z01) edge[auto,bend left] node{$b$} (z02);
\end{tikzpicture}
\end{center}
\liFlaci{Arxujcbdg}

Wir vertauschen die End- und Nicht-End-Zustände, um das Komplement zu
erhalten:

\begin{center}
\begin{tikzpicture}[li automat]
  \node[state,initial,accepting] (z0) at (2.14cm,-2.14cm) {\s0};
  \node[state] (z012) at (5.29cm,-5.29cm) {\s1};
  \node[state] (z02) at (2.14cm,-5.29cm) {\s3};
  \node[state,accepting] (z01) at (5.29cm,-2.14cm) {\s2};

  \path (z0) edge[auto] node{$a$} (z01);
  \path (z0) edge[auto,loop above] node{$b$} (z0);
  \path (z012) edge[auto,loop right] node{$a$} (z012);
  \path (z012) edge[auto] node{$b$} (z02);
  \path (z02) edge[auto,bend left] node{$a$} (z01);
  \path (z02) edge[auto] node{$b$} (z0);
  \path (z01) edge[auto] node{$a$} (z012);
  \path (z01) edge[auto,bend left] node{$b$} (z02);
\end{tikzpicture}
\end{center}
\liFlaci{A5zqsonq2}

\end{liAntwort}

%%
% (b)
%%

\item Gegeben sei zudem der nichtdeterministische Automat B über dem Alphabet D = {a,b}:

Konstruieren Sie einen endlichen Automaten (möglicherweise mit e-Übergängen), der die

Sprache (L(A)L(B))* C %* akzeptiert (A aus der vorigen Aufgabe). Erläutern Sie auch Ihre
Konstruktionsidee.

\end{enumerate}
\end{document}
