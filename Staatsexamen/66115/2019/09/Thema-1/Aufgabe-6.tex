\documentclass{lehramt-informatik-aufgabe}
\liLadePakete{syntax,mathe}
\begin{document}
\liAufgabenTitel{Sortieren von O-Klassen}
\section{Aufgabe 6 (O-Notation)
\index{Komplexität}
\footcite{66115:2019:09}}

\begin{enumerate}

%%
% (a)
%%

\item Sortieren Sie die unten angegebenen Funktionen der
O-Klassen $O(a(n))$, $O(b(n))$, $O(e(n))$, $O(d(n))$ und $O(e(n))$
bezüglich ihrer Teilmengenbeziehungen. Nutzen Sie ausschließlich die
echte Teilmenge C sowie die Gleichheit = für die Beziehung zwischen den
Mengen. Folgendes Beispiel illustriert diese Schreibweise für einige
Funktionen fı bis fs (diese haben nichts mit den unten angegebenen
Funktionen zu tun):

\begin{displaymath}
O(f_4 (n)) \subset O(f_3(n)) = O(f_5(n)) \subset O(f_1(n)) = O(f_2(n))
\end{displaymath}

Die angegebenen Beziehungen müssen weder bewiesen noch begründet werden.

\begin{itemize}
\item $a(n) = n^2 \cdot \log_2(n) + 42$
\item $b(n) = 2^n + n^4$
\item $c(n) = 2^{2 \cdot n}$
\item $d(n) = 2^{n+3}$
\item $e(n) = \sqrt{n^5}$
\end{itemize}

%%
% (b)
%%

\item Beweisen Sie die folgenden Aussagen formal nach den Definitionen
der O-Notation oder widerlegen Sie sie.

\begin{enumerate}

%%
% (i)
%%

\item Ofn - log,n) C O(n - (log, n)*)

%%
% (ii)
%%

\item 2”*! € O(2")
\end{enumerate}

%%
% (c)
%%

\item Bestimmen Sie eine asymptotische Lösung (in ©-Schreibweise) für
die folgende Rekursionsgleichung:

\begin{enumerate}

%%
% (i)
%%

\item (i) [8 Punkte] T(n) =4-T(8) +n’

%%
% (ii)
%%

\item (ii) [8 Punkte] T(n) =T(2)+3n’+n
\end{enumerate}

\end{enumerate}

\end{document}
