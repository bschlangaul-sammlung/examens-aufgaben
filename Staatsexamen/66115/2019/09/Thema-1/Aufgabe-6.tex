\documentclass{lehramt-informatik-aufgabe}
\liLadePakete{syntax,mathe}
\begin{document}
\liAufgabenTitel{Sortieren von O-Klassen}
\section{Aufgabe 6 (O-Notation)
\index{Komplexität}
\footcite{66115:2019:09}}

\begin{enumerate}

%%
% (a)
%%

\item Sortieren Sie die unten angegebenen Funktionen der
O-Klassen $O(a(n))$, $O(b(n))$, $O(e(n))$, $O(d(n))$ und $O(e(n))$
bezüglich ihrer Teilmengenbeziehungen. Nutzen Sie ausschließlich die
echte Teilmenge C sowie die Gleichheit = für die Beziehung zwischen den
Mengen. Folgendes Beispiel illustriert diese Schreibweise für einige
Funktionen fı bis fs (diese haben nichts mit den unten angegebenen
Funktionen zu tun):
siehe \url{http://www.s-inf.de/Skripte/DaStru.2012-SS-Katoen.(KK).Klausur1MitLoesung.pdf}

\begin{displaymath}
\mathcal{O}(f_4 (n)) \subset \mathcal{O}(f_3(n)) = \mathcal{O}(f_5(n)) \subset \mathcal{O}(f_1(n)) = \mathcal{O}(f_2(n))
\end{displaymath}

Die angegebenen Beziehungen müssen weder bewiesen noch begründet werden.

\begin{itemize}
\item $a(n) = n^2 \cdot \log_2(n) + 42$
\item $b(n) = 2^n + n^4$
\item $c(n) = 2^{2 \cdot n}$
\item $d(n) = 2^{n+3}$
\item $e(n) = \sqrt{n^5}$
\end{itemize}

\begin{antwort}
\begin{displaymath}
\mathcal{O}(a (n)) \subset \mathcal{O}(e(n)) = \mathcal{O}(b(n)) = \mathcal{O}(d(n)) \subset \mathcal{O}(c(n))
\end{displaymath}
\end{antwort}

%%
% (b)
%%

\item Beweisen Sie die folgenden Aussagen formal nach den Definitionen
der O-Notation oder widerlegen Sie sie.

\begin{enumerate}

%%
% (i)
%%

\item $\mathcal{O}(n \cdot \log_2 n) \subseteq \mathcal{O}(n \cdot (\log_2 n)^2)$

%%
% (ii)
%%

\item $2^{(n+1)} \in \mathcal{O}(n \cdot \log_2 n)$
\end{enumerate}

%%
% (c)
%%

\item Bestimmen Sie eine asymptotische Lösung (in $\Theta$-Schreibweise) für
die folgende Rekursionsgleichung:

\begin{enumerate}

%%
% (i)
%%

\item $T(n) = 4 \cdot T(\frac{n}{2}) + n^2$

%%
% (ii)
%%

\item $T(n) =  T(\frac{n}{2}) +\frac{n}{2} n^2 + n$
\end{enumerate}

\end{enumerate}

\end{document}
