\documentclass{bschlangaul-aufgabe}
\bLadePakete{syntax}
\begin{document}
\bAufgabenTitel{}
\section{Aufgabe 7 (Bäume)
\index{Bäume}
\footcite{examen:66115:2019:09}}

Gegeben sei die folgende Realisierung von binären Bäumen (in einer an
Java angelehnten Notation):

\bJavaExamen[firstline=3,lastline=5]{66115}{2019}{09}{Node}

\begin{enumerate}
%%
% (a)
%%

\item Beschreiben Sie in möglichst wenigen Worten, was die folgende
Methode \bJavaCode{foo} auf einem nicht-leeren binären Baum berechnet.

\bJavaExamen[firstline=11,lastline=23]{66115}{2019}{09}{Node}

\begin{liAntwort}
Die Methode \bJavaCode{foo(Node node)} berechnet die Summe aller Knoten
des Unterbaums des als Parameter übergebenen Knotens \bJavaCode{node}.
Der Schüsselwert des Knotens \bJavaCode{node} selbst wird in die
Summenberechnung mit einbezogen.
\end{liAntwort}

%%
% (b)
%%

\item Die Laufzeit der Methode \bJavaCode{foo(tree)} ist linear in $n$,
der Anzahl von Knoten im übergebenen Baum \bJavaCode{tree}. Begründen
Sie kurz, warum  \bJavaCode{foo(tree)} eine lineare Laufzeit hat.

\begin{liAntwort}
Die Methode \bJavaCode{foo(Node node)} wird pro Knoten des Baums
\bJavaCode{tree} genau einmal aufgerufen. Es handelt sich um eine
rekursive Methode, die auf den linken und rechten Kindknoten aufgerufen
wird. Hat ein Knoten keine Kinder mehr, dann wir die Methode nicht
aufgerufen.
\end{liAntwort}

%%
% (c)
%%

\item Betrachten Sie den folgenden Algorithmus für nicht-leere, binäre
Bäume. Beschreiben Sie in möglichst wenigen Worten die Wirkung der
Methode \bJavaCode{magic(tree)}. Welche Rolle spielt dabei die Methode
\bJavaCode{max}?

\bJavaExamen[firstline=25,lastline=52]{66115}{2019}{09}{Node}

\begin{liAntwort}
Methode \bJavaCode{magic(tree)} vertauscht für jeden Knoten des
Unterbaums den Wert mit dem Maximal-Knoten. Die Methode \bJavaCode{max}
liefert dabei den Knoten mit der größten Schlüsselwert im Unterbaum des
übergebenen Knoten (sich selbst eingeschlossen).
\end{liAntwort}

\item Geben Sie in Abhängigkeit von $n$, der Anzahl von Knoten im
übergebenen Baum \bJavaCode{tree}, jeweils eine Rekursionsgleichung für
die asymptotische Best-Case-Laufzeit $(B(n))$ und Worst-Case-Laufzeit
$(W(n))$ des Aufrufs \bJavaCode{magic(tree)} sowie die entsprechende
Komplexitätsklasse ($\Theta$) an. Begründen Sie Ihre Antwort.

Hinweis: Überlegen Sie, ob die Struktur des übergebenen Baumes Einfluss
auf die Laufzeit hat. Die lineare Laufzeit von \bJavaCode{max(t)} in
der Anzahl der Knoten des Baumes \bJavaCode{t} darf vorausgesetzt
werden.
\end{enumerate}

%-----------------------------------------------------------------------
%
%-----------------------------------------------------------------------

\begin{liAdditum}[Kompletter Code]
\bJavaExamen{66115}{2019}{09}{Node}
\end{liAdditum}

\end{document}
