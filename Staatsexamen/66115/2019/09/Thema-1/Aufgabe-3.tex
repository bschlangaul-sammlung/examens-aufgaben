\documentclass{lehramt-informatik-aufgabe}
\liLadePakete{formale-sprachen}
\begin{document}
\liAufgabenTitel{}
\section{Aufgabe 3
\index{Kontextfreie Sprache}
\footcite{66115:2019:09}}

Gegeben sei die kontextfreie Grammatik \liGrammatik{} mit Sprache $L(G)$,
wobei $V = \liMenge{S,T,U} $und \liAlphabet{a,b}. P bestehe aus den folgenden
Produktionen:

\begin{liProduktionsRegeln}
S -> T U U T,
T -> a T | EPSILON,
U -> b U b | a,
\end{liProduktionsRegeln}
\begin{enumerate}

%%
% (a)
%%

\item Geben Sie fünf verschiedene Wörter $w \in \Sigma^*$ mit $w \in
L(G)$ an.

\begin{liAntwort}
\begin{itemize}
\item aa
\item aaaa
\item ababbaba
\item aababbabaa
\item abbabbbbabba
\end{itemize}
\end{liAntwort}

%%
% (b)
%%

\item Geben Sie eine explizite Beschreibung der Sprache $L(G)$ an.

\begin{liAntwort}

\liAusdruck{a^* b^n a b^{2n} a b^n a^*}{n \in \mathbb{N}_0}

\end{liAntwort}

%%
% (c)
%%

\item Bringen Sie $G$ in Chomsky-Normalform und erklären Sie Ihre
Vorgehensweise.

\end{enumerate}
\end{document}
