\documentclass{lehramt-informatik-aufgabe}
\liLadePakete{}
\begin{document}
\liAufgabenTitel{Mastertheorem}
\section{Aufgabe 6
\index{Master-Theorem}
\footcite{66115:2019:09}}

Der Hauptsatz der Laufzeitfunktionen ist bekanntlich folgendermaßen definiert:

Bestimmen und begründen Sie formal mit Hilfe dieses Satzes welche Komplexität folgende Laufzeitfunktionen haben.

\begin{enumerate}
%%
% (a)
%%

\item [12 Punkte] T(n) = 8T(2) + 5n°
%%
% (b)
%%

\item [8 Punkte] T(r) = 97(2) + 5n?

\end{enumerate}
\end{document}
