\documentclass{lehramt-informatik-aufgabe}
\liLadePakete{master-theorem}
\begin{document}
\let\O=\liO
\let\o=\liOmega
\let\T=\liT
\let\t=\liTheta

\liAufgabenTitel{Mastertheorem}
\section{Aufgabe 6
\index{Master-Theorem}
\footcite{66115:2019:09}}

Der Hauptsatz der Laufzeitfunktionen ist bekanntlich folgendermaßen
definiert:

Bestimmen und begründen Sie formal mit Hilfe dieses Satzes welche
Komplexität folgende Laufzeitfunktionen haben.

\begin{enumerate}
%%
% (a)
%%

% https://www.wolframalpha.com/input/?i=T%5Bn%5D%3D%3D8T%5Bn%2F3%5D%2B5n%5E2

\item $T(n) = \T{8}{2} + 5n^2$

%%
% (b)
%%

% https://www.wolframalpha.com/input/?i=T%5Bn%5D%3D%3D9T%5Bn%2F3%5D%2B5n%5E2

\item $T(n) = \T{9}{3} + 5n^2$

\end{enumerate}
\end{document}
