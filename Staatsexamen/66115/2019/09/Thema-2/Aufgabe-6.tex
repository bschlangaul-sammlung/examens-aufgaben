\documentclass{lehramt-informatik-aufgabe}
\liLadePakete{master-theorem}
\begin{document}
\let\O=\liO
\let\o=\liOmega
\let\T=\liT
\let\t=\liTheta

\liAufgabenTitel{Mastertheorem}
\section{Aufgabe 6
\index{Master-Theorem}
\footcite{66115:2019:09}}

Der Hauptsatz der Laufzeitfunktionen ist bekanntlich folgendermaßen
definiert:

\liMasterFaelle

\noindent
Bestimmen und begründen Sie formal mit Hilfe dieses Satzes welche
Komplexität folgende Laufzeitfunktionen haben.

\begin{enumerate}
%%
% (a)
%%

% https://www.wolframalpha.com/input/?i=T%5Bn%5D%3D%3D8T%5Bn%2F3%5D%2B5n%5E2

\item $T(n) = \T{8}{2} + 5n^2$

\begin{liAntwort}
\liMasterVariablenDeklaration
{8} % a
{2} % b
{5n^2} % f(n)

\liMasterFallRechnung
% 1. Fall
{für $\varepsilon = 4$: \\
$f(n) = 5n^2 \in \O{n^{\log_2 {8 - 4}}} = \O{n^{\log_2 4}} = \O{n^2}$}
% 2. Fall
{$f(n) = 5n^2 \notin \t{n^{\log_2 {8}}} = \t{n^3}$}
% 3. Fall
{$f(n) = 5n^2 \notin \O{n^{\log_2 {8 + \varepsilon}}}$}

\liMasterWolframLink{T[n]=8T[n/3]\%2B5n^2}
\end{liAntwort}

%%
% (b)
%%

\item $T(n) = \T{9}{3} + 5n^2$

\begin{liAntwort}
\liMasterVariablenDeklaration
{9} % a
{3} % b
{5n^2} % f(n)

\liMasterFallRechnung
% 1. Fall
{$f(n) = 5n^2 \notin \O{n^{\log_3 {9 - \varepsilon}}}$ für $\varepsilon > 0$}
% 2. Fall
{$f(n) = 5n^2 \in \t{n^{\log_3 {9}}} = \t{n^2}$}
% 3. Fall
{$f(n) = 5n^2 \notin \O{n^{\log_3 {9 + \varepsilon}}}$ für $\varepsilon > 0$}

$\Rightarrow T(n) \in \t{n^2 \cdot \log n}$

\liMasterWolframLink{T[n]=9T[n/3]\%2B5n^2}
\end{liAntwort}

\end{enumerate}
\end{document}
