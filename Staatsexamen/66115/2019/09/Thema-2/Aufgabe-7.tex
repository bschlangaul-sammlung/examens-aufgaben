\documentclass{lehramt-informatik-aufgabe}
\liLadePakete{baum}
\begin{document}
\liAufgabenTitel{Gegener Baum erweitern und Knoten entfernen}
\section{Aufgabe 7 (AVL-Bäume)
\index{AVL-Baum}
\footcite{66115:2019:09}}

Fügen Sie (manuell) nacheinander die Zahlen 20, 31, 2, 17, 7 in
folgenden AVL-Baum ein. Löschen Sie anschließend aus dem entstandenen
Baum nacheinander 14 und 25.

\begin{center}
\begin{tikzpicture}[li binaer baum]
\Tree
[.\node{10};
  [.\node{5}; ]
  [.\node{25};
    [.\node{14}; ]
    \edge[blank]; \node[blank]{};
  ]
]
\end{tikzpicture}
\end{center}

\noindent
Zeichnen Sie jeweils direkt nach jeder einzelnen Operation zum Einfügen
oder Löschen eines Knotens, sowie nach jeder elementaren Rotation den
entstehenden Baum. Insbesondere sind evtl. anfallende Doppelrotationen
in zwei Schritten darzustellen. Geben Sie zudem an jedem Knoten die
Balancewerte an.

\begin{liDiagramm}{Nach Einfügen von „20“}
\begin{tikzpicture}[li binaer baum]
\Tree
[.\node[label=2]{10};
  [.\node[label=0]{5}; ]
  [.\node[label=-2]{25};
    [.\node[label=1]{14};
      \edge[blank]; \node[blank]{};
      [.\node[label=0]{20}; ]
    ]
    \edge[blank]; \node[blank]{};
  ]
]
\end{tikzpicture}
\end{liDiagramm}

\begin{liDiagramm}{Nach der Linksrotation}
\begin{tikzpicture}[li binaer baum]
\Tree
[.\node[label=2]{10};
  [.\node[label=0]{5}; ]
  [.\node[label=-2]{25};
    [.\node[label=-1]{20};
      [.\node[label=0]{14}; ]
      \edge[blank]; \node[blank]{};
    ]
    \edge[blank]; \node[blank]{};
  ]
]
\end{tikzpicture}
\end{liDiagramm}

\begin{liDiagramm}{Nach der Rechtsrotation}
\begin{tikzpicture}[li binaer baum]
\Tree
[.\node[label=1]{10};
  [.\node[label=0]{5}; ]
  [.\node[label=0]{20};
    [.\node[label=0]{14}; ]
    [.\node[label=0]{25}; ]
  ]
]
\end{tikzpicture}
\end{liDiagramm}

\begin{liDiagramm}{Nach Einfügen von „31“}
\begin{tikzpicture}[li binaer baum]
\Tree
[.\node[label=2]{10};
  [.\node[label=0]{5}; ]
  [.\node[label=1]{20};
    [.\node[label=0]{14}; ]
    [.\node[label=1]{25};
      \edge[blank]; \node[blank]{};
      [.\node[label=0]{31}; ]
    ]
  ]
]
\end{tikzpicture}
\end{liDiagramm}

\begin{liDiagramm}{Nach der Linksrotation}
\begin{tikzpicture}[li binaer baum]
\Tree
[.\node[label=0]{20};
  [.\node[label=0]{10};
    [.\node[label=0]{5}; ]
    [.\node[label=0]{14}; ]
  ]
  [.\node[label=1]{25};
    \edge[blank]; \node[blank]{};
    [.\node[label=0]{31}; ]
  ]
]
\end{tikzpicture}
\end{liDiagramm}

\begin{liDiagramm}{Nach Einfügen von „2“}
\begin{tikzpicture}[li binaer baum]
\Tree
[.\node[label=-1]{20};
  [.\node[label=-1]{10};
    [.\node[label=-1]{5};
      [.\node[label=0]{2}; ]
      \edge[blank]; \node[blank]{};
    ]
    [.\node[label=0]{14}; ]
  ]
  [.\node[label=1]{25};
    \edge[blank]; \node[blank]{};
    [.\node[label=0]{31}; ]
  ]
]
\end{tikzpicture}
\end{liDiagramm}

\begin{liDiagramm}{Nach Einfügen von „17“}
\begin{tikzpicture}[li binaer baum]
\Tree
[.\node[label=-1]{20};
  [.\node[label=0]{10};
    [.\node[label=-1]{5};
      [.\node[label=0]{2}; ]
      \edge[blank]; \node[blank]{};
    ]
    [.\node[label=1]{14};
      \edge[blank]; \node[blank]{};
      [.\node[label=0]{17}; ]
    ]
  ]
  [.\node[label=1]{25};
    \edge[blank]; \node[blank]{};
    [.\node[label=0]{31}; ]
  ]
]
\end{tikzpicture}
\end{liDiagramm}

\begin{liDiagramm}{Nach Einfügen von „7“}
\begin{tikzpicture}[li binaer baum]
\Tree
[.\node[label=-1]{20};
  [.\node[label=0]{10};
    [.\node[label=0]{5};
      [.\node[label=0]{2}; ]
      [.\node[label=0]{7}; ]
    ]
    [.\node[label=1]{14};
      \edge[blank]; \node[blank]{};
      [.\node[label=0]{17}; ]
    ]
  ]
  [.\node[label=1]{25};
    \edge[blank]; \node[blank]{};
    [.\node[label=0]{31}; ]
  ]
]
\end{tikzpicture}
\end{liDiagramm}

\liPseudoUeberschrift{Löschen}

\begin{liDiagramm}{Nach Löschen von „14“}
\begin{tikzpicture}[li binaer baum]
\Tree
[.\node[label=-1]{20};
  [.\node[label=-1]{10};
    [.\node[label=0]{5};
      [.\node[label=0]{2}; ]
      [.\node[label=0]{7}; ]
    ]
    [.\node[label=0]{17}; ]
  ]
  [.\node[label=1]{25};
    \edge[blank]; \node[blank]{};
    [.\node[label=0]{31}; ]
  ]
]
\end{tikzpicture}
\end{liDiagramm}

\begin{liDiagramm}{Nach Löschen von „25“}
\begin{tikzpicture}[li binaer baum]
\Tree
[.\node[label=-2]{20};
  [.\node[label=-1]{10};
    [.\node[label=0]{5};
      [.\node[label=0]{2}; ]
      [.\node[label=0]{7}; ]
    ]
    [.\node[label=0]{17}; ]
  ]
  [.\node[label=0]{31}; ]
]
\end{tikzpicture}
\end{liDiagramm}

\begin{liDiagramm}{Nach der Rechtsrotation}
\begin{tikzpicture}[li binaer baum]
\Tree
[.\node[label=0]{10};
  [.\node[label=0]{5};
    [.\node[label=0]{2}; ]
    [.\node[label=0]{7}; ]
  ]
  [.\node[label=0]{20};
    [.\node[label=0]{17}; ]
    [.\node[label=0]{31}; ]
  ]
]
\end{tikzpicture}
\end{liDiagramm}

\end{document}
