\documentclass{lehramt-informatik-aufgabe}
\liLadePakete{mathe,automaten,formale-sprachen}
\begin{document}
\liAufgabenTitel{Multiplikation mit 3}
\section{Aufgabe 1
\index{Turing-Maschine}
\footcite{66115:2019:09}}

Gesucht ist eine Turing-Maschine mit genau einem beidseitig unendlichen
Band, die die Funktion $f : \mathbb{N} \rightarrow \mathbb{N}$ mit $f(x)
= 3x$ berechnet. Zu Beginn der Berechnungsteht die Eingabe binär codiert
auf dem Band, wobei der Kopf auf die linkeste Ziffer (most significant
bit) zeigt. Am Ende der Berechnung soll der Funktionswert binär codiert
auf dem Band stehen, wobei der Kopf auf ein beliebiges Feld zeigen darf.
\begin{enumerate}

%%
% (a)
%%

\item Beschreiben Sie zunächst in Worten die Arbeitsweise Ihrer
Maschine.
%%
% (b)
%%

\item Geben Sie dann das kommentierte Programm der Turing-Maschine an
und erklären Sie die Bedeutung der verwendeten Zustände.

\begin{liAntwort}
\begin{center}
\begin{tikzpicture}[li turingmaschine]
  \node[state,initial] (RECHT) at (2.14cm,-1.86cm) {RECHT};
  \node[state] (Ü0) at (5.43cm,-1.86cm) {Ü0};
  \node[state] (2A+2) at (9.29cm,-7.43cm) {2A+2};
  \node[state] (2A+1) at (5.57cm,-7.43cm) {2A+1};
  \node[state] (2A+0) at (2.29cm,-7.29cm) {2A+0};
  \node[state] (1A+0) at (3.43cm,-4.86cm) {1A+0};
  \node[state] (1A+1) at (7.29cm,-4.86cm) {1A+1};
  \node[state] (2A+3) at (12.43cm,-7.57cm) {2A+3};
  \node[state] (1A+2) at (11cm,-4.71cm) {1A+2};
  \node[state] (Ü1) at (9.29cm,-2cm) {Ü1};
  \node[state,accepting] (FINE) at (4.43cm,-9.14cm) {FINE};

  \liTuringKante[above,loop above]{RECHT}{RECHT}{
    0, 0, R;
    1, 1, R;
  }

  \liTuringKante[above]{RECHT}{Ü0}{
    LEER, LEER, N;
  }

  \liTuringKante[above]{Ü0}{1A+0}{
    0, 0, L;
    LEER, LEER, L;
  }

  \liTuringKante[above]{Ü0}{1A+1}{
    1, 1, L;
  }

  \liTuringKante[above]{2A+2}{Ü1}{
    0, 0, L;
    1, 0, L;
    LEER, 0, L;
  }

  \liTuringKante[above,bend left]{2A+1}{Ü0}{
    0, 1, L;
    1, 1, L;
    LEER, 1, L;
  }

  \liTuringKante[above,bend left]{2A+0}{Ü0}{
    0, 0, L;
    1, 0, L;
  }

  \liTuringKante[above]{2A+0}{FINE}{
    LEER, LEER, N;
  }

  \liTuringKante[above]{1A+0}{2A+0}{
    0, 0, R;
    LEER, LEER, R;
  }

  \liTuringKante[above]{1A+0}{2A+1}{
    1, 1, R;
  }

  \liTuringKante[above]{1A+1}{2A+2}{
    1, 1, R;
  }

  \liTuringKante[above]{1A+1}{2A+1}{
    LEER, LEER, R;
    0, 0, R;
  }

  \liTuringKante[above,bend left]{2A+3}{Ü1}{
    0, 1, L;
    1, 1, L;
    LEER, 1, L;
  }

  \liTuringKante[above]{1A+2}{2A+3}{
    1, 1, R;
  }

  \liTuringKante[above]{1A+2}{2A+2}{
    0, 0, R;
    LEER, LEER, R;
  }

  \liTuringKante[above]{Ü1}{1A+2}{
    1, 1, L;
  }

  \liTuringKante[above]{Ü1}{1A+1}{
    0, 0, L;
    LEER, LEER, L;
  }
\end{tikzpicture}
\end{center}
\liFlaci{Ahjkw50kg}
\end{liAntwort}

\end{enumerate}
\end{document}
