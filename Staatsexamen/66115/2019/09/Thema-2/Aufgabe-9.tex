\documentclass{lehramt-informatik-aufgabe}
\liLadePakete{mathe}
\begin{document}
\liAufgabenTitel{Hashing mit mod 11 und 13}

\section{Aufgabe 9 (Hashing)
\index{Streutabellen (Hashing)}
\footcite[Thema 2 Aufgabe 9]{examen:66115:2019:09}}

Verwenden Sie die Hashfunktion $h(k,i) = (h’(k) + i^2) \mod 11$ mit
$h’(k) = k \mod 13$, um die Werte $12$, $29$ und $17$ in die folgende
Hashtabelle einzufügen. Geben Sie zudem jeweils an auf welche Zellen der
Hashtabelle zugegriffen wird.

\begin{center}
\begin{tabular}{|c|c|c|c|c|c|c|c|c|c|c|}
\hline
0&1&2&3&4&5&6&7&8&9&10\\\hline
&&&16&&5&&&&22&\\\hline
\end{tabular}
\end{center}
\end{document}
