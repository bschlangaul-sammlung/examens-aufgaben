\documentclass{lehramt-informatik-aufgabe}
\liLadePakete{formale-sprachen,automaten}
\begin{document}
\liAufgabenTitel{Automaten mit Zuständen q, r, s, t}
\section{Aufgabe 2
\index{Reguläre Sprache}
\footcite{66115:2020:03}}

\begin{enumerate}
%%
% (a)
%%

\item Es sei $L \subseteq \{ a, b, c \}^*$ die von dem folgenden
nichtdeterministischen Automaten akzeptierte Sprache:

\begin{center}
\begin{tikzpicture}[li automat]
  \node[state,initial] (z0) at (2.14cm,-2.14cm) {$z_0$};
  \node[state] (z1) at (4cm,-2.14cm) {$z_1$};
  \node[state] (z2) at (5.71cm,-2.14cm) {$z_2$};
  \node[state,accepting] (z3) at (7.57cm,-2.14cm) {$z_3$};

  \path (z0) edge[auto] node{$b$} (z1);
  \path (z0) edge[auto,loop above] node{$a,b,c$} (z0);
  \path (z1) edge[auto] node{$c$} (z2);
  \path (z2) edge[auto] node{$a$} (z3);
  \path (z2) edge[auto,loop above] node{$a,b,c$} (z2);
\end{tikzpicture}
\end{center}
\liFlaci{Apmac9bwc}

Beschreiben Sie (in Worten) wie die Wörter aus der Sprache $L$ aussehen.

\begin{liAntwort}
Alle Wörter der Sprache $L$ enthalten die Symbolfolge $bc$ und enden auf
$a$. Am Anfang der Wörter und vor dem letzten $a$ können beliebige
Kombination aus $a,b,c$ vorkommen.
\end{liAntwort}

%%
% (b)
%%

\item Benutzen Sie die Potenzmengenkonstruktion, um einen
deterministischen Automaten zu konstruieren, der zu dem Automaten aus
Teil (a) äquivalent ist. (Berechnen Sie nur erreichbare Zustände.)

%%
% (c)
%%

\item Ist der resultierende deterministische Automat schon minimal?
Begründen Sie Ihre Antwort.

%%
% (d)
%%

\item Minimieren Sie den folgenden deterministischen Automaten:

\end{enumerate}
\end{document}
