\documentclass{lehramt-informatik-aufgabe}
\liLadePakete{formale-sprachen}
\begin{document}
\let\m=\liMenge
\liAufgabenTitel{Kontextfreie Sprachen}
\section{Aufgabe 3
\index{Reguläre Sprache}
\footcite{66115:2020:03}}

\begin{enumerate}
%%
% (a)
%%

\item Entwerfen Sie eine kontextfreie Grammatik für die folgende kontextfreie Spra-
che über dem Alphabet \liAlphabet{a, b, c}:

L = {a"*’wuc” |n € No,2- |w|s = |vla}-

(Hierbei bezeichnet |u|, die Anzahl des Zeichens x in dem Wort u.)

Erklären Sie den Zweck der einzelnen Nichtterminale (Variablen) und der Grammatikregeln
Ihrer Grammatik.

%%
% (b)
%%

\item Betrachten Sie die folgende kontextfreie Grammatik

\begin{displaymath}
G=(\m{A,B,C,D}, \m{a,b,c}, P, A)
\end{displaymath}

mit den Produktionen

\begin{liProduktionsRegeln}
A -> A B | C D | a,
B -> C C | c,
C -> D C | C B | b,
D -> D B | a,
\end{liProduktionsRegeln}
\liFussnoteUrl{https://flaci.com/Gf7556jn2}

Benutzen Sie den Algorithmus von Cocke-Younger-Kasami (CYK), um zu
zeigen, dass das Wort $abcab$ zu der von $G$ erzeugten Sprache $L(G)$
gehört.
\index{CYK-Algorithmus}

\begin{liAntwort}
\begin{tabular}{|c||c|c|c|c|c|}
\hline
    & a    & b    & c    & a    & b \\
\hline
i/j & 1    & 2    & 3    & 4    & 5 \\\hline\hline
1   & A,D  & C    & B    & A,D  & C \\\cline{1-6}
2   & C    & C    & -    & C    \\\cline{1-5}
3   & C,C  & A    & -    \\\cline{1-4}
4   & A,A  & B    \\\cline{1-3}
5   & A,D,B,B \\\cline{1-2}
\end{tabular}
\end{liAntwort}

%%
% (c)
%%

\item Finden Sie nun ein größtmögliches Teilwort von $abcab$, dass von
keinem der vier Nichtterminale von $G$ ableitbar ist.

%%
% (d)
%%

\item Geben Sie eine Ableitung des Wortes $abcab$ mit $G$ an.
\index{Ableitung (Kontextfreie Sprache)}

\begin{liAntwort}
\liAbleitung{
  A ->
  AB ->
  ACC ->
  ACBC ->
  ACBDC ->
  aCBDC ->
  abBDC ->
  abcDC ->
  abcaC ->
  abcab
}
\end{liAntwort}

%%
% (e)
%%

\item Beweisen Sie, dass die folgende formale Sprache über Z = {a,b} nicht kon-
textfrei ist: ,
L={a"b" |neN}.

\end{enumerate}
\end{document}
