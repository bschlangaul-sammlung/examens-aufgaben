\documentclass{lehramt-informatik-aufgabe}
\liLadePakete{mathe,syntax}
\begin{document}
% identisch mit Staatsexamen/46115/2020/03/Thema-2/Aufgabe-6.tex

\liAufgabenTitel{Nächstes rot-blaues Paar auf der x-Achse}
\section{Aufgabe 8
\index{Algorithmische Komplexität (O-Notation)}
\footcite{66115:2020:03}}

Gegeben seien zwei nichtleere Mengen $R$ und $B$ von roten bzw. blauen
Punkten auf der x-Achse. Gesucht ist der minimale euklidische Abstand
$d(r, b)$ über alle Punktepaare $(r,b)$ mit $r \in R$ und $b \in B$.
Hier ist eine Beispielinstanz:

Die Eingabe wird in einem Feld $A$ übergeben. Jeder Punkt $A[i]$ mit $1
\leq i \leq n$ hat eine x-Koordinate $A[i].x$ und eine Farbe $A[i].color
\in \{ \text{rot}, \text{blau} \}$. Das Feld $A$ ist nach x-Koordinate
sortiert, \dh es gilt $A[1].x < A[2].x < \dots < A[n].x$, wobei $n = |R|
+ |B|$.

\begin{enumerate}

%%
% (a)
%%

\item Geben Sie in Worten einen Algorithmus an, der den gesuchten
Abstand in $\mathcal{O}(n)$ Zeit berechnet.

\begin{liAntwort}
\liPseudoUeberschrift{Pseudo-Code}
\begin{itemize}
\item Iteriere über die Indizes des Punkte-Arrays $P$ bis zum vorletzten
Index ($P[n-1]$)

\item Falls $P[n].color = P[n + 1].color$, dann gehe zum nächsten
Durchlauf mit $P[n + 1]$

\item Andernfalls berechne $d = P[n + 1].x - P[n].x$

\item Vergleiche $d$ mit $d_{min}$ (ist anfangs als $d_{min} =
Integer.MAX\_VALUE$ definiert). Falls $d < d_{min}$ definiere $d_{min} =
d$
\end{itemize}

\liPseudoUeberschrift{Java}

\liJavaExamen[firstline=30,lastline=41]{66115}{2020}{03}{RedBluePairCollection}
\end{liAntwort}

%%
% (b)
%%

\item Begründen Sie kurz die Laufzeit Ihres Algorithmus.

\begin{liAntwort}
Da das Array der Länge $n$ nur einmal durchlaufen wird, ist die Laufzeit
$\mathcal{O}(n)$ sichergestellt.
\end{liAntwort}

%%
% (c)
%%

\item Begründen Sie die Korrektheit Ihres Algorithmus.

\begin{liAntwort}
In $d_{min}$ steht am Ende der gesuchte Wert (sofern nicht $d_{min} =
Integer.MAX\_VALUE$ geblieben ist)
\end{liAntwort}

%%
% (d)
%%

\item Wir betrachten nun den Spezialfall, dass alle blauen Punkte links
von allen roten Punkten liegen. Beschreiben Sie in Worten, wie man in
dieser Situation den gesuchten Abstand in $o(n)$ Zeit berechnen kann. (Ihr
Algorithmus darf also insbesondere nicht Laufzeit $\Theta(n)$ haben.)

\begin{liAntwort}
Zuerst muss man den letzten blauen Punkt finden. Das ist mit einer
binären Suche möglich. (Man beginnt mit dem ganzen Array als Suchbereich
und betrachtet den mittleren Punkt. Wenn er blau ist, wiederholt man die
Suche in der zweiten Hälfte des Suchbereichs, sonst in der ersten, bis
man einen blauen Punkt gefolgt von einem roten Punkt gefunden hat.) Der
gesuchte minimale Abstand ist dann der Abstand zwischen dem gefundenen
blauen und dem nachfolgenden roten Punkt. Die Binärsuche hat eine
Worst-case-Laufzeit in $\mathcal{O}(\log n)$ und damit auch in $o(n)$.
\end{liAntwort}
\end{enumerate}

\liJavaExamen{66115}{2020}{03}{RedBluePairCollection}

\end{document}
