\documentclass{lehramt-informatik-aufgabe}
\liLadePakete{}
\begin{document}
\liAufgabenTitel{}
\section{Aufgabe 10
\index{Binärbaum}
\footcite{66115:2020:03}}

Sei $B$ ein binärer Suchbaum. In jedem Knoten $v$ von $B$ wird ein
Schlüssel $v.key \in \mathbb{N}$ gespeichert sowie Zeiger $v.left$,
$v.right$ und $v.parent$ auf sein linkes Kind, auf sein rechtes Kind und
auf seinen Elternknoten. Die Zeiger sind \emph{nil}, wenn der
entsprechende Nachbar nicht existiert. Für zwei Knoten $u$ und $v$ ist
wie üblich der \emph{Abstand} die Anzahl der Kanten auf dem kürzesten
Pfad von $u$ nach $v$.

Für einen Knoten $w$ von $B$ sei $B(w)$ der Teilbaum von $B$ mit Wurzel
$w$. Für zwei Knoten $u$ und $v$ von $B$ ist $w$ ein \emph{gemeinsamer}
Vorfahre, wenn $u$ und $v$ in $B(w)$ liegen. Wir suchen den niedrigsten
gemeinsamen Vorfahren $ngV(u,v)$ von $u$ und $v$, also einen gemeinsamen
Vorfahren $w$, so dass für jeden Vorfahren $w’$ von $u$ und $v$ gilt,
dass $w$ in $B(w’)$ liegt. Wir betrachten verschiedene Szenarien, in
denen Sie jeweils den niedrigsten gemeinsamen Vorfahren von $u$ und $v$
berechnen sollen.

\begin{liExkurs}[Lowest Common Ancestor]
Als Lowest Common Ancestor (LCA) oder „letzter gemeinsamer Vorfahre“
wird in der Informatik und Graphentheorie ein Ermittlungskonzept
bezeichnet, das einen gegebenen gewurzelten Baum von Datenstrukturen
effizient vorverarbeitet, sodass anschließend Anfragen nach dem letzten
gemeinsamen Vorfahren für beliebige Knotenpaare in konstanter Zeit
beantwortet werden können.
\liFussnoteUrl{https://de.wikipedia.org/wiki/Lowest_Common_Ancestor}
\end{liExkurs}

\begin{enumerate}

%%
% (a)
%%

\item Wir bekommen $u$ und $v$ als Zeiger auf die entsprechenden Knoten
in $B$ geliefert. Beschreiben Sie in Worten und in Pseudocode einen
Algorithmus, der den niedrigsten gemeinsamen Vorfahren von $u$ und $v$
berechnet. Analysieren Sie die Laufzeit Ihres Algorithmus.
%%
% (b)
%%

\item Wir bekommen $u$ und $v$ wieder als Zeiger auf die entsprechenden
Knoten in $B$ geliefert. Seien $d_u$, und $d_v$, die Abstände von $u$
bzw. $v$ zum niedrigsten gemeinsamen Vorfahren von $u$ und $v$. Die
Laufzeit Ihres Algorithmus soll $O(max\{d_u,d_v\})$ sein. Dabei kann Ihr
Algorithmus in jedem Knoten $v$ eine Information $v.info$ speichern.
Skizzieren Sie Ihren Algorithmus in Worten.

%%
% (c)
%%

\item Wir bekommen die Schlüssel $u.key$ und $v.key$. Die Laufzeit Ihres
Algorithmus soll proportional zum Abstand der Wurzel von $B$ zum
niedrigsten gemeinsamen Vorfahren von $u$ und $v$ sein. Skizzieren Sie
Ihren Algorithmus in Worten.
\end{enumerate}
\end{document}
