\documentclass{lehramt-informatik-aufgabe}
\liLadePakete{formale-sprachen,automaten}
\begin{document}
\liAufgabenTitel{Kontextfreie Sprachen}
\section{Aufgabe 3
\index{Kontextfreie Sprache}
\footcite{examen:66115:2020:03}}

\begin{enumerate}

%%
% (a)
%%

\item Ist die folgende Sprache \liAusdruck[L_1]{a^{n+2}b^{2n+1}}{n \geq 2}
über dem Alphabet \liAlphabet{a,b} kontextfrei?
\index{Kontextfreie Grammatik}

Falls ja, geben Sie eine kontextfreie Grammatik für $L_1$, an, falls
nein, eine kurze Begründung (ein vollständiger Beweis ist hier nicht
gefordert).

\begin{liAntwort}
$L_1$ ist kontextfrei

\liGrammatik{variablen={S,A,B}, alphabet={a,b}}

\begin{liProduktionsRegeln}
S -> a A b b,
A -> a A b b | a B bb,
B -> a a b,
\end{liProduktionsRegeln}
\liFlaci{Grxk1oczg}

\begin{description}
\item[$n = 2$] 4a 5b: aaaabbbbb
\item[$n = 3$] 5a 7b: aaaaabbbbbbb
\item[$n = 4$] 6a 9b: aaaaaabbbbbbbbb
\end{description}
\end{liAntwort}

%%
% (b)
%%

\item Geben Sie einen Kellerautomaten (PDA) formal an, der die Sprache

\liAusdruck[L_1]
{w_1 w_2 w_3}
{w_1, w_2, w_3 \in \Sigma^* \string\ \{ \lambda \}
\text{ und }
w_1 = w_3^{\text{rev}}
} $\in CFL$
über dem Alphabet \liAlphabet{0,1}
akzeptiert.

Dabei bezeichnet $\lambda$ das leere Wort und $w_3^{\text{rev}}$
bezeichnet das Wort $w_3$ rückwärts gelesen. Bei Akzeptanz einer Eingabe
soll sich der PDA in einem Endzustand befinden und der Keller geleert
sein.

\begin{liAntwort}
\liFlaci{Gpkctmk3g}

\begin{liProduktionsRegeln}
S -> 0 S 0 | 1 S 1 | 0 A 0 | 1 A 1,
A -> 0 A | 1 A | 0 | 1
\end{liProduktionsRegeln}

\begin{center}
\begin{tikzpicture}[li kellerautomat]
  \node[state,initial] (z0) at (2.2cm,-4.71cm) {$z_0$};
  \node[state,accepting] (z2) at (7.5cm,-4.71cm) {$z_2$};
  \node[state] (z1) at (5cm,-4.71cm) {$z_1$};

  \liKellerKante{z0}{z1}{
    epsilon raute Sraute
  }

  \liKellerKante{z1}{z2}{
    epsilon raute epsilon
  }

  \liKellerKante[above,loop]{z1}{z1}{
    0 0 epsilon,
    1 1 epsilon,
    epsilon S 0S0,
    epsilon S 1S1,
    epsilon S 0A0,
    epsilon S 1A1,
    epsilon A 0A,
    epsilon A 1A,
    epsilon A 0,
    epsilon A 1
  }

\end{tikzpicture}
\end{center}

\liFlaci{A5z2zfkdw}
\end{liAntwort}

%%
% (c)
%%

\item Beschreiben Sie in Worten die Arbeitsweise Ihres PDA aus
Aufgabenteil (b).

\end{enumerate}
\end{document}
