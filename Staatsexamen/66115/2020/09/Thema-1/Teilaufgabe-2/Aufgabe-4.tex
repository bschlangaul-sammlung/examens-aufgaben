\documentclass{lehramt-informatik-aufgabe}
\liLadePakete{syntax,mathe,master-theorem}
\begin{document}
\let\j=\liJavaCode
\let\T=\liT

\liAufgabenTitel{O-Notation}
\section{Aufgabe 4
\index{Algorithmische Komplexität (O-Notation)}
\footcite{examen:66115:2020:09}}

\begin{enumerate}

%%
% a)
%%

\item Betrachten Sie das folgende Code-Beispiel (in Java-Notation):

\liJavaDatei[firstline=4,lastline=13]{examen/examen_66115/jahr_2020/herbst/o_notation/Mystery1}

Bestimmen Sie die asymptotische worst-case Laufzeit des Code-Beispiels
in $\mathcal{O}$-Notation bezüglich der Problemgröße $n$. Begründen Sie
Ihre Antwort.

\begin{liAntwort}
Die asymptotische worst-case Laufzeit des Code-Beispiels
in $\mathcal{O}$-Notation ist $\mathcal{O}(n)$.

Die \j{while}-Schleife wird genau $n$ mal ausgeführt. In der Schleife
wird die Variable \j{i} in der Zeile \j{i = i + 1;} inkrementiert. \j{i}
wird mit 0 initialisiert. Die \j{while}-Schleife endet, wenn \j{i} gleich
groß ist als \j{n}.
\end{liAntwort}

%%
% b)
%%

\item Betrachten Sie das folgende Code-Beispiel (in Java-Notation):

\liJavaDatei[firstline=5,lastline=18]{examen/examen_66115/jahr_2020/herbst/o_notation/Mystery2}

Bestimmen Sie für das Code-Beispiel die asymptotische worst-case
Laufzeit in $\mathcal{O}$-Notation
bezüglich der Problemgröße $n$. Begründen Sie Ihre Antwort.

\begin{liAntwort}
\begin{description}
\item[\j{while}:]
$n$-mal

\item[1. \j{for}:]
$n, n-1, \dots, 2, 1$

\item[2. \j{for}:]
$1, 2, \dots, n-1, n$
\end{description}

$n \times n \times n = \mathcal{O}(n^3)$
\end{liAntwort}

%%
% c)
%%

\item Bestimmen Sie eine asymptotische Lösung (in $\Theta$-Schreibweise)
für die folgende Rekursionsgleichung:
\index{Master-Theorem}

\def\fn{\textstyle{\frac{1}{2}}n^2 + n}

\begin{displaymath}
T(n) = \T{}{2} + \fn
\end{displaymath}

\liMasterExkurs

\begin{liAntwort}

\liMasterVariablenDeklaration
{1} % a
{2} % b
{\fn} % f(n) ohne $mathe$

Nebenrechnung: $\log_b a = \log_2 1 = 0$

\liMasterFallRechnung
% 1. Fall
{$\fn \notin \liO{n^{-1}}$}
% 2. Fall
{$\fn \notin \liTheta{1}$}
% 3. Fall
{$\varepsilon = 2$ \\
$\fn \in \liOmega{n^2}$ \\
Für eine Abschätzung suchen wir eine Konstante, damit gilt: \\
$1 \cdot f(\frac{n}{2}) \leq c \cdot f(n)$ \\
$\frac{1}{2} \cdot \frac{1}{4} n^2 + \frac{1}{2}n  \leq c \cdot (\frac{1}{2} \cdot n^2 + n)$ \\
Damit folgt $c = \frac{1}{4}$ \\
und $0 < c < 1$}

$\Rightarrow \Theta(\frac{1}{2}n^2 + n)$

$\Rightarrow \Theta(n^2)$

\liMasterWolframLink{T[n]=T[n/2]\%2B1/2n^2\%2Bn}
\end{liAntwort}
\end{enumerate}

\end{document}
