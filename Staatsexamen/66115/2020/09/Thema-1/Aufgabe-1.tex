\documentclass{lehramt-informatik-aufgabe}
\liLadePakete{}
\begin{document}
\liAufgabenTitel{}
\section{Aufgabe 1
\index{Berechenbarkeit}
\footcite{66115:2020:09}}

% Info_2021-05-07-2021-05-07_13.30.16.mp4
% 1h57min-2h8min

Antworten Sie mit \emph{„Stimmt“} oder \emph{„Stimmt nicht“}. Begründen
Sie Ihr Urteil kurz.
\footcite[Seite 55-56]{theo:fs:4}
\begin{enumerate}

%%
% a)
%%

\item Eine Sprache ist genau dann regulär, wenn sie unendlich viele
Wörter enthält.

\begin{liAntwort}
Stimmt nicht, da endliche Sprachen immer regulär sind.
\end{liAntwort}

%%
% b)
%%

\item Zu jedem nichtdeterministischen endlichen Automaten mit $n$
Zuständen gibt es einen deterministischen endlichen Automaten, der die
gleiche Sprache erkennt und höchstens $n^2$ Zustände hat.

\begin{liAntwort}
Stimmt nicht, da hier maximal exponentielle Zustandszunahme eintreten
kann.

Die Aussäge wäre richtig mit $2^n$ Zustände.
\end{liAntwort}

%%
% c)
%%

\item Das Komplement einer kontextfreien Sprache ist wieder kontextfrei.

\begin{liAntwort}
Stimmt nicht, da kontextfreien Sprache nicht abgeschlossen sind unter
dem Komplement.
\end{liAntwort}

%%
% d)
%%

\item Wenn ein Problem unentscheidbar ist, dann ist es nicht
semientscheidbar.

\begin{liAntwort}
Stimmt nicht, unentscheidbar ist das Gegenteil von entscheidbar. Es kann
auch semi-entscheidbar sein.
\end{liAntwort}

%%
% e)
%%

\item Sei $f$ eine totale Funktion. Dann gibt es ein
\texttt{WHILE}-Programm, das diese berechnet.

\begin{liAntwort}
Stimmt nicht, da $f$ nicht berechenbar sein muss, aber Voraussetzung für
entscheidbar.
\end{liAntwort}

%%
% f)
%%

\item Das Halteproblem für \texttt{LOOP}-Programme ist entscheidbar.

\begin{liAntwort}
Stimmt, LOOP immer haltend. Jeder LOOP-Programm terminiert. Es gibt für
jede Eingabe eine Ausgabe.
\end{liAntwort}

%%
% g)
%%

\item Die Komplexitätsklasse NP enthält genau die Entscheidungsprobleme,
die in nichtpolynomieller Zeit entscheidbar sind.

\begin{liAntwort}
Stimmt, die Aussage entspricht genau der Definition der
Komplexitätsklasse NP.
\end{liAntwort}

%%
% h)
%%

\item Falls $P \neq NP$, dann gibt es keine NP-vollständigen Probleme,
die in $P$ liegen.

\begin{liAntwort}
Stimmt, genau die Definition.
\end{liAntwort}

\end{enumerate}
\end{document}
