\documentclass{lehramt-informatik-aufgabe}
\liLadePakete{formale-sprachen}
\begin{document}
\liAufgabenTitel{}
\section{Aufgabe 3
\index{Kontextfreie Sprache}
\footcite{66115:2020:09}}

Seien \liAlphabet{a,b,c} und \liAusdruck{wc\hat{w}}{w \in \liMenge{a,b}*}.
Dabei ist $\hat{w}$ das zu $w$ gespiegelte Wort.

\begin{enumerate}

%%
% a)
%%

\item Zeigen Sie, dass $L$ nicht regulär ist.
\index{Pumping-Lemma (Reguläre Sprache)}

\begin{liAntwort}
$L$ ist regulär.
Dann gilt für $L$ das Pumping-Lemma.
Sei $j$ die Zahl aus
dem Pumping-Lemma.
Dann muss sich das Wort $a^j b c b a^j \in L$
aufpumpen lassen (da $|a^j b c b a^j| \geq j$).
$a^j b c b a^j = uvw$ ist eine passende Zerlegung laut Lemma.
Da $|uv| < j$, ist
$u = a^x$, $v = a^y$, $w = a^z b c b a^j$, wobei
$y > 0$ und
$x + y + z = j$
Aber dann
$u v^0 w= a^{x+z} b c b a^j \notin L$, da $x + z < j$.
Widerspruch.
\liFussnoteUrl{https://userpages.uni-koblenz.de/~sofronie/gti-ss-2015/slides/endliche-automaten6.pdf}
\end{liAntwort}

%%
% b)
%%

\item Zeigen Sie, dass $L$ kontextfrei ist, indem Sie eine geeignete
Grammatik angeben und anschließend begründen, dass diese die Sprache $L$
erzeugt.

\begin{liAntwort}
\begin{liProduktionsRegeln}
S -> a S a | a C a | b S b | b C b,
C -> c
\end{liProduktionsRegeln}

\liAbleitung{S -> aSa -> abCba -> abcba}

\end{liAntwort}

\end{enumerate}
\end{document}
