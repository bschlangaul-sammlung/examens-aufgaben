\documentclass{bschlangaul-aufgabe}

\begin{document}
\bAufgabenMetadaten{
  Titel = {Aufgabe 1},
  Thematik = {Vermische Fragen},
  RelativerPfad = Staatsexamen/66115/2020/09/Thema-1/Teilaufgabe-1/Aufgabe-1.tex,
  ZitatSchluessel = examen:66115:2020:09,
  BearbeitungsStand = unbekannt,
  Korrektheit = unbekannt,
  Stichwoerter = {Theoretische Informatik, Reguläre Sprache},
  ExamenNummer = 66115,
  ExamenJahr = 2020,
  ExamenMonat = 09,
  ExamenThemaNr = 1,
  ExamenTeilaufgabeNr = 1,
  ExamenAufgabeNr = 1,
}

Antworten Sie mit \emph{„Stimmt“} oder \emph{„Stimmt nicht“}. Begründen
Sie Ihr Urteil kurz.\footcite[Seite 55-56]{theo:fs:4}
\index{Theoretische Informatik}
\footcite{examen:66115:2020:09}

% Info_2021-05-07-2021-05-07_13.30.16.mp4
% 1h57min-2h8min

\begin{enumerate}

%%
% a)
%%

\item Eine Sprache ist genau dann regulär, wenn sie unendlich viele
Wörter enthält.
\index{Reguläre Sprache}

\begin{liAntwort}
Stimmt nicht. Sprachen mit endlicher Mächtigkeit sind immer regulär.
Endliche Spachen sind in obenstehender Aussage ausgeschlossen.
\end{liAntwort}

%%
% b)
%%

\item Zu jedem nichtdeterministischen endlichen Automaten mit $n$
Zuständen gibt es einen deterministischen endlichen Automaten, der die
gleiche Sprache erkennt und höchstens $n^2$ Zustände hat.

\begin{liAntwort}
Stimmt nicht. Müsste $2^n$ heißen.
\end{liAntwort}

%%
% c)
%%

\item Das Komplement einer kontextfreien Sprache ist wieder kontextfrei.

\begin{liAntwort}
Stimmt nicht. Kontextfreie Sprachen sind nicht abgeschlossen unter dem
Komplement. Das Komplement einer kontextfreien Sprache kann regulär,
kontextfrei oder kontextsensitiv sein.
\end{liAntwort}

%%
% d)
%%

\item Wenn ein Problem unentscheidbar ist, dann ist es nicht
semientscheidbar.

\begin{liAntwort}
Stimmt nicht. Semientscheidbarkeit ist eine typische Form der
Unentscheidbarkeit. Unentscheidbarkeit ist das Gegenteil von
Entscheidbarkeit. Unentscheidbar kann entweder völlig unentscheidbar
sein oder semientscheidbar.
\end{liAntwort}

%%
% e)
%%

\item Sei $f$ eine totale Funktion. Dann gibt es ein WHILE-Programm, das
diese berechnet.

\begin{liAntwort}
Stimmt nicht. Wir wissen nicht, ob die totale Funktion $f$ berechenbar
ist. Wenn $f$ berechenbar ist, dann wäre die Aussage richtig.
\end{liAntwort}

%%
% f)
%%

\item Das Halteproblem für \texttt{LOOP}-Programme ist entscheidbar.

\begin{liAntwort}
Stimmt. Alle LOOP-Programme terminieren (halten). Es gibt für
jede Eingabe eine Ausgabe.
\end{liAntwort}

%%
% g)
%%

\item Die Komplexitätsklasse $\mathcal{NP}$ enthält genau die
Entscheidungsprobleme, die in nicht-polynomieller Zeit entscheidbar
sind.

\begin{liAntwort}
Stimmt. Die Aussage entspricht der Definiton der Komplexitätsklasse
$\mathcal{NP}$.
\end{liAntwort}

%%
% h)
%%

\item Falls $P \geq NP$, dann gibt es keine $\mathcal{NP}$-vollständigen
Probleme, die in $P$ liegen.

\begin{liAntwort}
Stimmt. Entspricht der Definition.
\end{liAntwort}
\end{enumerate}
\end{document}
