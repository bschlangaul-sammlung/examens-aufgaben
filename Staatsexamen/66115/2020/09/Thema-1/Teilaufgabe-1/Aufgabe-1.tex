\documentclass{lehramt-informatik-aufgabe}
\liLadePakete{}
\begin{document}
\liAufgabenTitel{Vermische Fragen}
\section{Aufgabe 1
\index{Theoretische Informatik}
\footcite{66115:2020:09}}

Antworten Sie mit „Stimmt“ oder „Stimmt nicht“. Begründen Sie Ihr Urteil
kurz.

\begin{enumerate}

%%
% a)
%%

\item Eine Sprache ist genau dann regulär, wenn sie unendlich viele
Wörter enthält.
\index{Reguläre Sprache}

\begin{liAntwort}
Stimmt nicht. Auch das leere Wort gehört zu den regulären Sprachen. Eine
Sprache ist dann regulär, wenn sie durch eine reguläre Grammatik,
einen endlichen Automaten oder einen regulären Ausdruck dargestellt
werden kann.
\end{liAntwort}

%%
% b)
%%

\item Zu jedem nichtdeterministischen endlichen Automaten mit $n$
Zuständen gibt es einen deterministischen endlichen Automaten, der die
gleiche Sprache erkennt und höchstens $n^2$ Zustände hat.

\begin{liAntwort}
Stimmt: Englische Wikipedia:
Because the DFA states consist of sets of NFA states, an n-state NFA may
be converted to a DFA with at most 2n states.[2] For every n, there
exist n-state NFAs such that every subset of states is reachable from
the initial subset, so that the converted DFA has exactly 2n states,
\end{liAntwort}

%%
% c)
%%

\item Das Komplement einer kontextfreien Sprache ist wieder kontextfrei.

\begin{liAntwort}
Stimmt.

Sei $A$ ein deterministischer endlicher Automat, der $L$
erkennt. Der Automat $A$ erreicht für jedes Wort $\omega$ Element $L$
einen Endzustand und für jedes Wort $\omega$ nicht Element L einen
Nicht-Endzustand. Indem in $A$ alle Endzustände zu Nicht-Endzuständen
gemacht werden und umgekehrt, entsteht ein deterministischer endlicher
Automat $A$, der $L$ erkennt. Also ist $L$ regulär.
\liFussnoteUrl{https://www.inf.hs-flensburg.de/lang/theor/regulaer-abgeschlossen.htm}
\end{liAntwort}

%%
% d)
%%

\item Wenn ein Problem unentscheidbar ist, dann ist es nicht
semientscheidbar.

%%
% e)
%%

\item Sei f eine totale Funktion. Dann gibt es ein WHILE-Programm, das
diese berechnet.
%%
% f)
%%

\item Das Halteproblem für LOOP-Programme ist entscheidbar.

%%
% g)
%%

\item Die Komplexitätsklasse NP enthält genau die Entscheidungsprobleme,
die in nicht-polynomieller Zeit entscheidbar sind.

%%
% h)
%%

\item Falls PZNP, dann gibt es keine NP-vollständigen Probleme, die in P
liegen.
\end{enumerate}
\end{document}
