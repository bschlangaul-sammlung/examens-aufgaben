\documentclass{lehramt-informatik-aufgabe}
\liLadePakete{formale-sprachen}
\begin{document}
\liAufgabenMetadaten{
  Titel = {Aufgabe 4},
  Thematik = {Funktion klein sigma von w},
  RelativerPfad = Staatsexamen/66115/2020/09/Thema-1/Teilaufgabe-1/Aufgabe-4.tex,
  ZitatSchluessel = examen:66115:2020:09,
  Stichwoerter = {Entscheidbarkeit},
  ExamenNummer = 66115,
  ExamenJahr = 2020,
  ExamenMonat = 09,
  ExamenThemaNr = 1,
  ExamenTeilaufgabeNr = 1,
  ExamenAufgabeNr = 4,
}

Geben Sie für jede der folgenden Mengen an, ob sie entscheidbar ist oder
nicht. Dabei ist $\sigma_w$, die Funktion, die von der Turingmaschine
berechnet wird, die durch das Wort $w$ kodiert wird. Beweisen Sie Ihre
Behauptungen.\index{Entscheidbarkeit}
\footcite{examen:66115:2020:09}

\begin{enumerate}
%%
% a)
%%

\item \liAusdruck[L_1]{w \in \Sigma^*}{\sigma_w(0) = 0}

\begin{liAntwort}
Nicht entscheidbar wegen dem Halteproblem.
\end{liAntwort}

%%
% b)
%%

\item \liAusdruck[L_2]{w \in \Sigma^*}{\sigma_w(w) = w}

\begin{liAntwort}
Nicht entscheidbar wegen dem Halteproblem.
\end{liAntwort}

%%
% c)
%%

\item \liAusdruck[L_3]{w \in \Sigma^*}{\sigma_0(0) = w}

\begin{liAntwort}
Entscheidbar wegen $\sigma_0$.
\end{liAntwort}

\end{enumerate}
\end{document}
