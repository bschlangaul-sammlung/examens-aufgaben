\documentclass{lehramt-informatik-aufgabe}
\liLadePakete{komplexitaetstheorie}
\begin{document}
\liAufgabenTitel{}
\section{Aufgabe 5
\index{Theoretische Informatik}
\footcite{examen:66115:2020:09}}

Sei \liProblemName{If} die Menge aller aussagenlogischen Formeln, die ausschließlich mit
den Konstanten 0 und 1, logischen Variablen x, mit © € N und der
Implikation => als Operationszeichen aufgebaut sind, wobei natürlich
Klammern zugelassen sind. Beachten Sie, dass ; => 7; die gleiche
Wahrheitstabelle wie -z; V z; hat.

Wir betrachten das Problem \liProblemName{Isat}. Eine Formel F' e IF ist
genau dann in \liProblemName{Isat} enthalten, wenn sie erfüllbar ist,
das heißt, falls es eine Belegung der Variablen mit Konstanten 0 oder 1
gibt, sodass F' den Wert 1 annimmt.

Zeigen Sie: \liProblemName{Isat} ist NP-vollständig. Sie dürfen
benutzen, dass das \liProblemName{Sat}-Problem NP-vollständig ist.

\begin{liAntwort}

\end{liAntwort}

\end{document}
