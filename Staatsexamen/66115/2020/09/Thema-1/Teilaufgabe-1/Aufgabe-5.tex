\documentclass{lehramt-informatik-aufgabe}
\liLadePakete{}
\begin{document}
\liAufgabenTitel{}
\section{Aufgabe 5
\index{Theoretische Informatik}
\footcite{66115:2020:09}}

Sei IF die Menge aller aussagenlogischen Formeln, die ausschließlich mit den Konstanten 0 und
1, logischen Variablen x, mit © € N und der Implikation => als Operationszeichen aufgebaut
sind, wobei natürlich Klammern zugelassen sind. Beachten Sie, dass ; => 7; die gleiche
Wahrheitstabelle wie -z; V z; hat.

Wir betrachten das Problem ISAT. Eine Formel F' e IF ist genau dann in ISAT enthalten, wenn
sie erfüllbar ist, das heißt, falls es eine Belegung der Variablen mit Konstanten 0 oder 1 gibt,
sodass F' den Wert 1 annimmt.

Zeigen Sie: ISAT ist NP-vollständig. Sie dürfen benutzen, dass das SAT-Problem NP-vollständig
ist.

\end{document}
