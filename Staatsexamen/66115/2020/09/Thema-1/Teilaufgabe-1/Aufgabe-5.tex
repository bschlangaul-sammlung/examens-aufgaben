\documentclass{lehramt-informatik-aufgabe}
\liLadePakete{komplexitaetstheorie}
\begin{document}
\liAufgabenTitel{IF ISAT SAT}
\section{Aufgabe 5
\index{Komplexitätstheorie}
\footcite{examen:66115:2020:09}}

Sei \liProblemName{If} die Menge aller aussagenlogischen Formeln, die
ausschließlich mit den Konstanten $0$ und $1$, logischen Variablen $x_i$
mit $i \in N$ und der Implikation $\Rightarrow$ als Operationszeichen
aufgebaut sind, wobei natürlich Klammern zugelassen sind. Beachten Sie,
dass $x_i \Rightarrow x_j$ die gleiche Wahrheitstabelle wie $\neg x_i
\lor x_j$ hat.

Wir betrachten das Problem \liProblemName{Isat}. Eine Formel $F \in
\liProblemName{If}$ ist genau dann in \liProblemName{Isat} enthalten,
wenn sie erfüllbar ist, das heißt, falls es eine Belegung der Variablen
mit Konstanten $0$ oder $1$ gibt, sodass $F'$ den Wert $1$ annimmt.

Zeigen Sie: \liProblemName{Isat} ist NP-vollständig. Sie dürfen
benutzen, dass das \liProblemName{Sat}-Problem NP-vollständig ist.
\index{Polynomialzeitreduktion}

\begin{liAntwort}

\end{liAntwort}

\end{document}
