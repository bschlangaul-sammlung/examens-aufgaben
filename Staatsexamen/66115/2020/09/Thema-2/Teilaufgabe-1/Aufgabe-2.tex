\documentclass{lehramt-informatik-aufgabe}
\liLadePakete{formale-sprachen}
\begin{document}
\liAufgabenTitel{Kontextfreie Sprachen}
\section{Aufgabe 2
\index{Kontextfreie Sprache}
\footcite{examen:66115:2020:09}}

\begin{enumerate}

%%
% a)
%%

\item Sei \liAusdruck{0^n 1^m 1^p 0^q}{n + m = p + q \text{ und } n, m,
p, q \in \mathbb{N}_0}. Geben Sie eine kontextfreie Grammatik für $L$
an. Sie dürfen dabei $\varepsilon$-Produktionen der Form
\liProduktionen{A -> EPSILON} verwenden.

\begin{liAntwort}
\begin{liProduktionsRegeln}
S -> 0 S 0 | 0 A 0 | 0 B 0 | EPSILON | A | B | C,
A -> 0 A 1 | 0 C 1,
B -> 1 B 0 | 1 C 1,
C -> 1 C 1 | EPSILON,
\end{liProduktionsRegeln}
\end{liAntwort}

%%
% b)
%%

\item Für eine Sprache $L$ sei \liAusdruck[L^r]{x^r}{x \in L} die
Umkehrsprache. Dabei bezeichne $x^r$ das Wort, das aus $r$ entsteht,
indem man die Reihenfolge der Zeichen umkehrt, beispielsweise $(abb)^r =
bba$.

\begin{enumerate}

%%
% (i)
%%

\item Sei $L$ eine kontextfreie Sprache. Zeigen Sie, dass dann auch
$L^r$ kontextfrei ist.

%%
% (ii)
%%

\item Geben Sie eine kontextfreie Sprache $L_1$, an, sodass $L_1 \cap
L^r_1$ kontextfrei ist.

%%
% (iii)
%%

\item Geben Sie eine kontextfreie Sprache $L_2$, an, sodass $L_2 \cap
L^r_2$ nicht kontextfrei ist.

\end{enumerate}
\end{enumerate}
\end{document}
