\documentclass{lehramt-informatik-aufgabe}
\liLadePakete{syntax,mathe,master-theorem}
\begin{document}
\let\j=\liJavaCode
\let\T=\liTheta

\liAufgabenTitel{O-Notation}
\section{Aufgabe 4
\index{Komplexität}
\footcite{66115:2020:09}}

\begin{enumerate}

%%
% a)
%%

\item Betrachten Sie das folgende Code-Beispiel (in Java-Notation):

\liJavaDatei[firstline=4,lastline=13]{examen/examen_66115/jahr_2020/herbst/o_notation/Mystery1}

Bestimmen Sie die asymptotische worst-case Laufzeit des Code-Beispiels
in $\mathcal{O}$-Notation bezüglich der Problemgröße $n$. Begründen Sie
Ihre Antwort.

\begin{liAntwort}
Die asymptotische worst-case Laufzeit des Code-Beispiels
in $\mathcal{O}$-Notation ist $\mathcal{O}(n)$.

Die \j{while}-Schleife wird genau $n$ mal ausgeführt. In der Schleife
wird die Variable \j{i} in der Zeile \j{i = i + 1;} inkrementiert. \j{i}
wird mit 0 initialisiert. Die \j{while}-Schleife endet wenn \j{i} gleich
groß ist wir \j{n}.
\end{liAntwort}

%%
% b)
%%

\item Betrachten Sie das folgende Code-Beispiel (in Java-Notation):

\liJavaDatei[firstline=5,lastline=18]{examen/examen_66115/jahr_2020/herbst/o_notation/Mystery2}

Bestimmen Sie für das Code-Beispiel die asymptotische worst-case
Laufzeit in $\mathcal{O}$-Notation
bezüglich der Problemgröße $n$. Begründen Sie Ihre Antwort.

\begin{liAntwort}
\begin{description}
\item[\j{while}:]
$n$-mal

\item[1. \j{for}:]
$n, n-1, \dots, 2, 1$

\item[2. \j{for}:]
$1, 2, \dots, n-1, n$
\end{description}

$n \times n \times n = \mathcal{O}(n^3)$
\end{liAntwort}

%%
% c)
%%

\item Bestimmen Sie eine asymptotische Lösung (in $\Theta$-Schreibweise)
für die folgende Rekursionsgleichung:
\index{Master-Theorem}

\begin{displaymath}
T(n) = T\left(\frac{n}{2}\right) + \textstyle{\frac{1}{2}}n^2 + n
\end{displaymath}

\end{enumerate}

\end{document}
