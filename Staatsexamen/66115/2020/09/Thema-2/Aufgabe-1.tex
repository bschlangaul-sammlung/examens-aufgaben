\documentclass{lehramt-informatik-aufgabe}
\liLadePakete{syntax}
\begin{document}
\newmintinline[p]{pascal}{}

\liAufgabenTitel{Algorithmenanalyse}
\section{Aufgabe 1
\index{Algorithmen und Datenstrukturen}
\footcite{66115:2020:09}}

Betrachten Sie die folgende Prozedur \p{countup}, die aus zwei
ganzzahligen Eingabewerten \p{n} und \p{m} einen ganzzahligen
Ausgabewert berechnet:

\begin{minted}{pascal}
procedure countup(n, m : integer): integer
var x, y : integer;
begin
  x := n;
  y := 0;
  while (y < m) do
    x := x - 1;
    y := y + 1;
  end while
  return x;
end
\end{minted}
\begin{enumerate}

%%
% a)
%%

\item Führen Sie \p{countup(3,2)} aus. Geben Sie für jeden
Schleifendurchlauf jeweils den Wert der Variablen \p{n}, \p{m}, \p{x}
und \p{y} zu Beginn der while-Schleife und den Rückgabewert der Prozedur
an.

\begin{liAntwort}
\begin{tabular}{llll}
n & m & x & y \\\hline
3 & 2 & 3 & 0 \\
3 & 2 & 2 & 1 \\
3 & 2 & 1 & 2 \\
\end{tabular}

Rückgabewert: 1
\end{liAntwort}

%%
% b)
%%

\item Gibt es Eingabewerte von \p{n} und \p{m}, für die die Prozedur
\p{countup} nicht terminiert? Begründen Sie Ihre Antwort.

\begin{liAntwort}
Nein. Mit jedem Schleifendurchlauf wird der Wert der Variablen \p{y} um
eins hochgezählt, die Werte, die \p{y} annimmt, sind also mathematisch
ausgedrückt streng monoton steigend. \p{y} nähert sich \p{m}
an, bis \p{y} nicht mehr kleiner ist als \p{m} und die Prozedur
terminiert. An diesem Sachverhält ändern auch sehr große Zahlen, die
über die Variable \p{m} der Prozedur übergeben werden, nichts.
\end{liAntwort}

%%
% c)
%%

\item Geben Sie die asymptotische worst-case Laufzeit der Prozedur
\p{countup} in der $\Theta$-Notation in Abhängigkeit von den
Eingabewerten \p{n} und/oder \p{m} an. Begründen Sie Ihre Antwort.

\begin{liAntwort}
Die Laufzeit der Prozedur ist immer $\Theta(m)$. Die Laufzeit hängt nur
von \p{m} ab. Es kann nicht zwischen best-, average and worst-case
unterschieden werden.
\end{liAntwort}
%%
% d)
%%

\item

Betrachten Sie nun die folgende Prozedur \p{countdown}, die aus zwei
ganzzahligen Eingabewerten \p{n} und \p{m} einen ganzzahligen
Ausgabewert berechnet:

\begin{minted}{pascal}
procedure countdown(n, m : integer) : integer
var x, y : integer;
begin
  x := n;
  y := 0;
  while (n > 0) do
    if (y < m) then
      x := x - 1;
      y := y + 1;
    else
      y := 0;
      n := n / 2; /* Ganzzahldivision */
    end if
  end while
  return X;
end
\end{minted}

Führen Sie \p{countdown(3,2)} aus. Geben Sie für jeden
Schleifendurchlauf jeweils den Wert der Variablen \p{n}, \p{m}, \p{x}
und \p{y} zu Beginn der \p{while}-Schleife und den Rückgabewert der
Prozedur an.

\begin{liAntwort}
\begin{tabular}{llll}
n & m & x & y \\\hline
3 & 2 & 3 & 0 \\
3 & 2 & 2 & 1 \\
3 & 2 & 1 & 2 \\
3 & 2 & 0 & 0 \\
\end{tabular}

Rückgabewert: 1
\end{liAntwort}

%%
% e)
%%

\item Gibt es Eingabewerte von \p{n} und \p{m}, für die die Prozedur
\p{countdown} nicht terminiert?

Begründen Sie Ihre Antwort.

\begin{liAntwort}
Nein.

\begin{description}
\item[\p{n <= 0}]

terminiert sofort

\item[\p{m < 0}]

Falsch-Block der Wenn-Dann-Bedingung erniedrigt \p{n} bis \p{0} erreicht
ist. Dann terminiert die Prozedur.

\item[\p{m > 0}]

Wahr-Block der Wenn-Dann-Bedingung erhöht \p{y} streng monoton bis \p{y
>= m}. Falsch-Block der Wenn-Dann-Bedingung erniedrigt \p{n} bis \p{0}
erreicht ist. Dann terminiert die Prozedur.

\end{description}
\end{liAntwort}

%%
% f)
%%

\item Geben Sie die asymptotische Laufzeit der Prozedur \p{countdown} in
der $\Theta$-Notation in Abhängigkeit von den Eingabewerten \p{n}
und/oder \p{m} an unter der Annahme, dass \p{m >= 0} und \p{n > 0}.
Begründen Sie Ihre Antwort.

\begin{liAntwort}
$\Theta(m) + \Theta(\log_2 n)$
\end{liAntwort}

\end{enumerate}
\end{document}
