\documentclass{lehramt-informatik-aufgabe}
\liLadePakete{syntax}
\begin{document}
\newmintinline[p]{pascal}{}

\liAufgabenTitel{Algorithmenanalyse}
\section{Aufgabe 1
\index{Algorithmen und Datenstrukturen}
\footcite{66115:2020:09}}

Betrachten Sie die folgende Prozedur \p{countup}, die aus zwei
ganzzahligen Eingabewerten \p{n} und \p{m} einen ganzzahligen
Ausgabewert berechnet:

\begin{minted}{pascal}
procedure countup(n, m: integer): integer
var x, y: integer;
begin
  x := n;
  y := 0;
  while (y < m) do
    x := x - 1;
    y := y + 1;
  end while
  return x;
end
\end{minted}
\begin{enumerate}

%%
% a)
%%

\item Führen Sie \p{countup(3,2)} aus. Geben Sie für jeden
Schleifendurchlauf jeweils den Wert der Variablen \p{n}, \p{m}, \p{x}
und \p{y} zu Beginn der while-Schleife und den Rückgabewert der Prozedur
an.

\begin{liAntwort}
\begin{tabular}{llll}
n & m & x & y \\\hline
3 & 2 & 3 & 0 \\
3 & 2 & 2 & 1 \\
3 & 2 & 1 & 2 \\
\end{tabular}

Rückgabewert: 1
\end{liAntwort}

%%
% b)
%%

\item Gibt es Eingabewerte von \p{n} und \p{m}, für die die Prozedur
\p{countup} nicht terminiert? Begründen Sie Ihre Antwort.

\begin{liAntwort}
Nein. Mit jedem Schleifendurchlauf wird der Wert der Variablen \p{y} um
eins hochgezählt, die Werte, die \p{y} annimmt, sind also mathematisch
ausgedrückt streng monoton steigend. \p{y} nähert sich \p{m}
an, bis \p{y} nicht mehr kleiner ist als \p{m} und die Prozedur
terminiert. An diesem Sachverhält ändern auch sehr große Zahlen, die
über die Variable \p{m} der Prozedur übergeben werden, nichts.
\end{liAntwort}

%%
% c)
%%

\item Geben Sie die asymptotische worst-case Laufzeit der Prozedur
\p{countup} in der $\Theta$-Notation in Abhängigkeit von den
Eingabewerten \p{n} und/oder \p{m} an. Begründen Sie Ihre Antwort.

\begin{liAntwort}
Die Laufzeit der Prozedur ist immer $\Theta(m)$. Die Laufzeit hängt nur
von \p{m} ab. Es kann nicht zwischen best-, average and worst-case
unterschieden werden.
\end{liAntwort}

Betrachten Sie nun die folgende Prozedur countdown, die aus zwei
ganzzahligen Eingabewerten n und Mm einen ganzzahligen Ausgabewert
berechnet:

\begin{minted}{pascal}
procedure countdown(n, m : integer) : integer
varx, y:integer;
begin
X:=N;
Y=0;
while (n > 0) do
if(y< m) then
X=X-]1;
Y:-y+1;
else
y:=0;
n :=n/2; /» Ganzzahldivision x/
end if
end while
return X;
end
\end{minted}

%%
% d)
%%

\item Führen Sie countdown(3,2) aus. Geben Sie für jeden
Schleifendurchlauf jeweils den Wert der Variablen n, m, x und y zu
Beginn der while-Schleife und den Rückgabewert der Prozedur
an.

%%
% e)
%%

\item Gibt es Eingabewerte von n und m, für die die Prozedur countdown
nicht terminiert?

Begründen Sie Ihre Antwort.

%%
% f)
%%

\item Geben Sie die asymptotische Laufzeit der Prozedur countdown in der
8-Notation in Abhängigkeit von den Eingabewerten n und/oder m an unter
der Annahme, dass m> 0
und n > 0. Begründen Sie Ihre Antwort.

\end{enumerate}
\end{document}
