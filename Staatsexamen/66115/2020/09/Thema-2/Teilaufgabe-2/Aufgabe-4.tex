\documentclass{lehramt-informatik-aufgabe}
\liLadePakete{syntax,mathe}
\begin{document}
\liAufgabenTitel{Gutschein}
\section{Aufgabe 4
\index{Dynamische Programmierung}
\footcite{66115:2020:09}}

Das GUTSCHEIN-Problem ist gegeben durch eine Folge $w_1, \dots, w_n$ von
Warenwerten (wobei $w \in N_0$ für $i = 1, \dots, n$ ) und einem
Gutscheinbetrag $G \in \mathbb{N}_0$.

Da Gutscheine nicht in Bargeld ausgezahlt werden können, ist die Frage,
ob man eine Teilfolge der Waren findet, sodass man genau den Gutschein
ausnutzt. Formal ist dies die Frage, ob es eine Menge von Indizes $I$
mit $I \subseteq \{1, \dots, n \}$ gibt, sodass $\sum_{i \in I} w_i = G$
\begin{enumerate}

%%
% a)
%%

\item Sei $w_1 = 10, w_2 = 30, w_3 = 40, w_4 = 20, w_5 = 15$ eine Folge
von Warenwerten.

\begin{enumerate}

%%
% (ii)
%%

\item Geben Sie einen Gutscheinbetrag $40 < G < 115$ an, sodass die
GUT\-SCHEIN-Instanz eine Lösung hat. Geben Sie auch die lösende Menge $I
\subseteq \{ 1, 2, 3, 4, 5 \}$ von Indizes an.

\begin{liAntwort}
50

$I = \{ 1, 3 \}$
\end{liAntwort}

%%
% (ii)
%%

\item Geben Sie einen Gutscheinbetrag $G$ mit $40 < G < 115$ an, sodass
die GUTSCHEIN-Instanz keine Lösung hat.

\begin{liAntwort}
51
\end{liAntwort}

\end{enumerate}

%%
% b)
%%

\item Sei \emph{table} eine $(n \times (G + 1))$-Tabelle mit Einträgen
$\text{table}[i,k]$, für $1 < i < n$ und $0 \leq k \leq G$, sodass

\begin{liAntwort}
\begin{equation*}
\text{table}[i,k] =
\begin{cases}
\textbf{true} &
\text{falls es } I \subseteq \{1, \dots, n \} \text{ mit }
\sum_{i \in I} w_i = G \text{ gibt}\\

\textbf{false} & \text{sonst}
\end{cases}
\end{equation*}
\end{liAntwort}

Geben Sie einen Algorithmus in Pseudo-Code oder Java an, der die Tabelle
\emph{table} mit \emph{dynamischer Programmierung} in
Worst-Case-Laufzeit $\mathcal{O}(n \times G)$) erzeugt. Begründen Sie
die Korrektheit und die Laufzeit Ihres Algorithmus. Welcher Eintrag in
\emph{table} löst das GUTSCHEIN-Problem?

\begin{liAntwort}
\liJavaExamen{66115}{2020}{09}{Gutschein}

Die äußere for-Schleife läuft n mal und die innere for-Schleife G mal.
Der letzte Eintrag in der Tabelle, also der Wert in der Zelle:\\
\liJavaCode{tabelle[warenWerte.length][gutscheinBetrag]}.

\end{liAntwort}

\liJavaTestDatei{examen/examen_66115/jahr_2020/herbst/GutscheinTest}

\begin{itemize}
\item \liJavaCode{gutscheinDP(3, new int[] { 1, 2, 3 }));}: wahr (w)

\begin{minted}{md}
[
  [w, f, f, f],
  [w, w, f, f],
  [w, w, w, w],
  [w, w, w, w]
]
\end{minted}

\item \liJavaCode{gutscheinDP(7, new int[] { 1, 2, 3 });}: falsch (f)

\begin{minted}{md}
[
  [w, f, f, f, f, f, f, f],
  [w, w, f, f, f, f, f, f],
  [w, w, w, w, f, f, f, f],
  [w, w, w, w, w, w, w, f]
]
\end{minted}

\item \liJavaCode{gutscheinDP(10, new int[] { 10, 30, 40, 20, 15 });}: wahr (w)

\begin{minted}{md}
[
  [w, f, f, f, f, f, f, f, f, f, f],
  [w, f, f, f, f, f, f, f, f, f, w],
  [w, f, f, f, f, f, f, f, f, f, w],
  [w, f, f, f, f, f, f, f, f, f, w],
  [w, f, f, f, f, f, f, f, f, f, w],
  [w, f, f, f, f, f, f, f, f, f, w]
]
\end{minted}
\end{itemize}

\end{enumerate}
\end{document}
