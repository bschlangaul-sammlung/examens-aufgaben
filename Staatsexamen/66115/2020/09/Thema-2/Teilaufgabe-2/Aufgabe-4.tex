\documentclass{lehramt-informatik-aufgabe}
\liLadePakete{}
\begin{document}
\liAufgabenTitel{}
\section{Aufgabe 4
\index{Dynamische Programmierung}
\footcite{66115:2020:09}}

Das GUTSCHEIN-Problem ist gegeben durch eine Folge wı,...,w„ von
Warenwerten (wobei welNo fürö=1,...,n) und einem Gutscheinbetrag G € No.

Da Gutscheine nicht in Bargeld ausgezahlt werden können, ist die Frage,
ob man eine Teilfolge der Waren findet, sodass man genau den Gutschein
ausnutzt. Formal ist dies die Frage, ob es eine Menge von Indizes J mit
IC
\begin{enumerate}

%%
% a)
%%

\item Sei $w_1 = 10, w_2 = 30, w_3 = 40, w_4 = 20, w_5 = 15$ eine Folge
von Warenwerten.

\begin{enumerate}

%%
% (ii)
%%

\item Geben Sie einen Gutscheinbetrag $40 < G < 115$ an, sodass die
GUTSCHEIN-Instanz eine Lösung hat. Geben Sie auch die lösende Menge $I
\subseteq \{ 1, 2, 3, 4, 5 \}$ von Indizes an.

\begin{liAntwort}
50

$I = \{ 1, 3 \}$
\end{liAntwort}

%%
% (ii)
%%

\item Geben Sie einen Gutscheinbetrag $G$ mit $40 < G < 115$ an, sodass die
GUTSCHEIN-Instanz keine Lösung hat.

\begin{liAntwort}
51
\end{liAntwort}

\end{enumerate}

%%
% b)
%%

\item Sei table eine (n x (G + 1))-Tabelle mit Einträgen tablefi,k], für 1 <i<nundO<sk<G,
sodass
true, falses/IC
tableli, k] = u
false, sonst

Geben Sie einen Algorithmus in Pseudo-Code oder Java an, der die Tabelle
table mit dynami- scher Programmierung in Worst-Case-Laufzeit O(n x G))
erzeugt. Begründen Sie die Kor- rektheit und die Laufzeit Ihres
Algorithmus. Welcher Eintrag in table löst das GUTSCHEIN- Problem?

\end{enumerate}
\end{document}
