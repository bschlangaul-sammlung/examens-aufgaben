\documentclass{lehramt-informatik-aufgabe}
\liLadePakete{syntax}
\begin{document}
\liAufgabenTitel{Schnelle Suche von Schlüsseln: odd-ascending-even-descending-Folge}
\section{Aufgabe 5
\index{Binäre Suche}
\footcite{66115:2020:09}}

Eine Folge von Zahlen ist eine
\emph{odd-ascending-even-descending}-Folge, wenn gilt:

Zunächst enthält die Folge alle Schlüssel, die \emph{ungerade} Zahlen
sind, und diese Schlüssel sind aufsteigend sortiert angeordnet. Im
Anschluss daran enthält die Folge alle Schlüssel, die \emph{gerade}
Zahlen sind, und diese Schlüssel sind absteigend sortiert angeordnet.

\begin{enumerate}
\item Geben Sie die Zahlen $10, 3, 11, 20, 8, 4, 9$ als
\emph{odd-ascending-even-descending}-Folge an.

\begin{liAntwort}
$3, 9, 11, 20, 10, 8, 4$
\end{liAntwort}

\item Geben Sie einen Algorithmus (z. B. in Pseudocode oder Java) an,
der für eine \emph{odd-ascending-even-descending}-Folge $F$ gegeben als
Feld und einem Schlüsselwert $S$ prüft, ob $S$ in $F$ vorkommt und
\liJavaCode{true} im Erfolgsfall und ansonsten \liJavaCode{false}
liefert. Dabei soll der Algorithmus im Worst-Case eine echt bessere
Laufzeit als Linearzeit (in der Größe der Arrays) haben. Erläutern Sie
Ihren Algorithmus und begründen Sie die Korrektheit.

\begin{liAntwort}
\liJavaExamen{66115}{2020}{09}{UngeradeGerade}

\liJavaTestDatei{examen/examen_66115/jahr_2020/herbst/UngeradeGeradeTest}
\end{liAntwort}

%%
% c)
%%

\item Erläutern Sie schrittweise den Ablauf Ihres Algorithmus für die
Folge $1, 5, 11, 8, 4, 2$ und
den Suchschlüssel 4.

%%
% d)
%%

\item Analysieren Sie die Laufzeit Ihres Algorithmus für den Worst-Case,
geben Sie diese in $\mathcal{O}$-Notation an und begründen Sie diese.

\end{enumerate}
\end{document}
