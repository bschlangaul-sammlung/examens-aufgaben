\documentclass{lehramt-informatik-aufgabe}
\liLadePakete{master-theorem}
\begin{document}
\liAufgabenTitel{}
\section{Aufgabe 2
\index{Komplexitätstheorie}
\footcite{examen:66115:2020:09}}

 (O-Notation) [20 PUNKTE]

Beweisen Sie die folgenden Aussagen:

\begin{enumerate}

%%
% a)
%%

\item Sei fn)=2-.n?+3-n?+4- (log,n) + 7. Dann gilt f € O(n?).
%%
% b)
%%

\item Sei f(n) = 4”. Dann gilt nicht f € O(2").

\begin{liAntwort}

\end{liAntwort}

%%
% c)
%%

\item Sei fn) = (n+1)! (d. h. die Fakultät von n +1). Dann gilt f € O(n”).

\begin{liAntwort}
z. B. $5 \cdot 4 \cdot 3 \cdot  2 \cdot  1$ $5^5$
\end{liAntwort}
%%
% d)
%%

\item Sei f:N— N definiert durch die folgende Rekursionsgleichung:
\liMasterExkurs

3, fürn=1
/n) =
(n—-1)+f(n-1), fürn>1

Dann gilt f e O(n?).

\end{enumerate}
\end{document}
