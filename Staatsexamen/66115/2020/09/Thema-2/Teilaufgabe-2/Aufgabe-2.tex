\documentclass{lehramt-informatik-aufgabe}
\liLadePakete{mathe}
\begin{document}
\liAufgabenTitel{}
\section{Aufgabe 2
\index{Komplexitätstheorie}
\footcite{examen:66115:2020:09}}

Beweisen Sie die folgenden Aussagen:

\begin{enumerate}

%%
% a)
%%

\item Sei fn)=2-.n?+3-n?+4- (log,n) + 7. Dann gilt f € O(n?).
%%
% b)
%%

\item Sei f(n) = 4”. Dann gilt nicht f € O(2").

\begin{liAntwort}

\end{liAntwort}

%%
% c)
%%

\item Sei fn) = (n+1)! (d. h. die Fakultät von n +1). Dann gilt f € O(n”).

\begin{liAntwort}
z. B. $5 \cdot 4 \cdot 3 \cdot  2 \cdot  1$ $5^5$
\end{liAntwort}
%%
% d)
%%

\item Sei $f \colon \mathbb{N} \leftarrow \mathbb{N}$ definiert durch
die folgende Rekursionsgleichung:

\begin{equation*}
f(n)=
\begin{cases}
3,& \text{für }n = 1\\

(n - 1)^2 + f(n - 1),
& \text{für }n > 1
\end{cases}
\end{equation*}

Dann gilt $f \in \mathcal{O}(n^3)$

\begin{liAntwort}

\begin{align*}
f(n)
&=(n - 1)^2 + f(n - 1) + \dots + f(1)\\
&=(n - 1)^2 + (n - 1)^2 + f(n - 2) + \dots + f(1)\\
&=\underbrace{(n - 1)^2 + \dots + (n - 1)^2 + 3}_{n}\\
&=\underbrace{(n - 1)^2 + \dots + (n - 1)^2}_{n - 1} + 3\\
&= (n - 1)^2 \cdot (n - 1) + 3 \\
&= (n - 1)^3 + 3
\end{align*}
\end{liAntwort}

\end{enumerate}
\end{document}
