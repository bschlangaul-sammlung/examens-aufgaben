\documentclass{lehramt-informatik-aufgabe}
\liLadePakete{syntax,baum}
\begin{document}
\newmintinline[p]{pascal}{}

\liAufgabenTitel{Bäume}
\section{Aufgabe 2
\index{Binärbaum}
\footcite{66115:2020:09}}

Wir betrachten ein Feld A von ganzen Zahlen mit \p{n} Elementen, die über
die Indizes \p{A[O]} bis \p{A[n-1]} angesprochen werden können. In dieses Feld
ist ein binärer Baum nach den folgenden Regeln eingebettet: Für das
Feldelement mit Index $i$ befindet sich

\begin{itemize}
\item der Elternknoten im Feldelement mit Index $\lfloor\frac{i-1}{2}\rfloor$,

\item der linke Kindknoten im Feldelement mit Index $2 \cdot i + 1$, und

\item der rechte Kindknoten im Feldelement mit Index $2 \cdot i + 2$.
\end{itemize}
\begin{enumerate}

%%
% a)
%%

\item Zeichnen Sie den durch das folgende Feld repräsentierten binären
Baum.

\begin{center}
\begin{tabular}{|r|c|c|c|c|c|c|c|c|c|c|c|}
\hline
i        & 0 & 1 & 2 & 3  & 4  & 5  & 6 & 7  & 8  & 9  & 10\\\hline
\p{A[i]} & 2 & 4 & 6 & 14 & 12 & 10 & 8 & 22 & 20 & 18 & 16\\\hline
\end{tabular}
\end{center}

\begin{liAntwort}
\begin{center}
\begin{tikzpicture}[li binaer baum]
\Tree
[.2
  [.4
    [.14
      [.22 ]
      [.20 ]
    ]
    [.12
      [.18 ]
      [.16 ]
    ]
  ]
  [.6
    [.10 ]
    [.8 ]
  ]
]
\end{tikzpicture}
\end{center}
\end{liAntwort}

%%
% b)
%%

\item Der folgende rekursive Algorithmus sei gegeben:

\begin{minted}{pascal}
procedure magic(i, n : integer) : boolean
begin
  if (i > (n - 2) / 2) then
    return true;
  endif
  if (A[i] <= A[2 * i + 1] and A[i] <= A[2 * i + 2] and
      magic(2 * i + 1, n) and magic(2 * i + 2, n)) then
    return true;
  endif
  return false;
end
\end{minted}

\liJavaDatei{examen/examen_66115/jahr_2020/herbst/Baum}

Gegeben sei folgendes Feld:

\begin{center}
\begin{tabular}{|r|c|c|c|c|c|}
\hline
i        & 0 & 1 & 2 & 3  \\\hline
\p{A[i]} & 2 & 4 & 6 & 14 \\\hline
\end{tabular}
\end{center}

Führen Sie \p{magic(0,3)} auf dem Feld aus. Welches Resultat liefert der
Algorithmus zurück?

\begin{liAntwort}
true
\end{liAntwort}

%%
% c)
%%

\item Wie nennt man die Eigenschaft, die der Algorithmus \p{magic} auf
dem Feld A prüft? Wie lässt sich diese Eigenschaft formal beschreiben?

\begin{liAntwort}
Haldeneigenschaft einer Minhalde.

Die Methode funktioniert jedoch nicht fehlerfrei. \liJavaCode{magic(0,
3)} liefert bei dem Feld \liJavaCode{2, 4, 6, 1} \liJavaCode{true}
zurück.

\end{liAntwort}

%%
% d)
%%

\item Welche Ausgaben sind durch den Algorithmus \p{magic} möglich, wenn
das Eingabefeld aufsteigend sortiert ist? Begründen Sie Ihre Antwort.

\begin{liAntwort}
true. Eine sortierte aufsteigende Zahlenfolge entspricht den
Haldeneigenschaften einer Min-Heap.
\end{liAntwort}

%%
% e)
%%

\item Geben Sie zwei dreielementige Zahlenfolgen (bzw. Felder) an, eine
für die \p{magic(0,2)} den Wert \p{true} liefert und eine, für die
\p{magic(0,2)} den Wert \p{false} liefert.

%%
% f)
%%

\item Betrachten Sie folgende Variante almostmagic der oben bereits erwähnten Prozedur magic,
bei der die Anweisungen in Zeilen 3 bis 5 entfernt wurden:

\begin{minted}{pascal}
procedure almostmagic(i, n : integer) : boolean
begin
  // leer
  // leer
  // leer
  if (A[i] <= A[2 * i + 1] and A[i] <= A[2 * i + 2] and
      magic(2 * i + 1, n) and magic(2 * i + 2, n)) then
    return true;
  endif
  return false;
end
\end{minted}

Beschreiben Sie die Umstände, die auftreten können, wenn almostmagic auf
einem Feld der Größe n aufgerufen wird. Welchen Zweck erfüllt die
entfernte bedingte Anweisung?

%%
% g)
%%

\item Fügen Sie jeweils den angegebenen Wert in den jeweils angegebenen
AVL-Baum mit aufsteigender Sortierung ein und zeichnen Sie den
resultierenden Baum vor möglicherweise erforderlichen Rotationen. Führen
Sie danach bei Bedarf die erforderliche(n) Rotation(en) aus und zeichnen
Sie dann den resultierenden Baum. Sollten keine Rotationen erforderlich
sein, so geben Sie dies durch einen Text wie “keine Rotationen nötig”
an.

\begin{enumerate}

%%
% i.)
%%

\item Wert 7 einfügen

%%
% ii.)
%%

\item Wert 11 einfügen

%%
% iii.)
%%

\item Wert 5 einfügen
\end{enumerate}
\end{enumerate}
\end{document}
