\documentclass{lehramt-informatik-aufgabe}
\liLadePakete{}
\begin{document}
\liAufgabenTitel{}
\section{Aufgabe 1
\index{Reguläre Sprache}
\footcite{66115:2015:03}}

Die Sprache $L$ über den Alphabet $\Sigma = \{0, 1\}$ enthält alle
Wörter, bei denen beim Lesen von links nach rechts der Unterschied in
der Zahl der 0en und 1en stets höchstens 3 ist. Also ist $w \in L$ genau
dann, wenn für alle $u$, $v$ mit $w = uv$ gilt $||u|_0 − |u|_1| \leq 3$.
Erinnerung: $|w|_a$ bezeichnet die Zahl der $a$’s im Wort $w$.
\footcite{theo:ab:1}

\begin{enumerate}

%%
% (a)
%%

\item Sei $A = (Q, \Sigma, \delta, q_0 , E)$ ein deterministischer
endlicher Automat für L. Es sei $w_1 = 111$, $w_2 = 11$, $w_3 = 1$,$ w_4
= \epsilon$, $w_5 = 0$, $w_6 = 00$, $w_7 = 000$. Machen Sie sich klar,
dass der Automat jedes dieser Wörter verarbeiten können muss. Folgern
Sie, dass der Automat mindestens sieben Zustände haben muss. Schreiben
Sie Ihr Argumentation schlüssig und vollständig auf.

%%
% (b)
%%

\item Begründen Sie, dass $L$ regulär ist.

%%
% (c)
%%

\item Jemand behauptet, diese Sprache sei nicht regulär und gibt
folgenden „Beweis“ dafür an: Wäre $L$ regulär, so sei $n$ eine
entsprechende Pumping-Zahl. Nun ist $w = (01) n \in L$. Zerlegt man nun
$w = uxv$, wobei $u = 0$, $x = 1$, $v = (01) n−1$ , so ist zum Beispiel
$ux 5 v \notin L$, denn es ist $ux 5 v = 01111101010101$.... Legen Sie
genau dar, an welcher Stelle dieser „Beweis” fehlerhaft ist.
\index{Pumping-Lemma (Reguläre Sprache)}

%%
% (d)
%%

\item In anderen Fällen können nichtdeterministische endliche Automaten
echt kleiner sein als die besten deterministischen Automaten. Ein
Beispiel ist die Sprache $L_2 = \Sigma^∗ 1 \Sigma$ aller Wörter, deren
vorletztes Symbol $1$ ist. Geben Sie einen nicht-deterministischen
Automaten mit nur drei Zuständen an, $L_2$ erkennt.

%%
% (e)
%%

\item Führen Sie auf Ihrem Automaten die Potenzmengenkonstruktion und
anschließend den Minimierungsalgorithmus durch. Wie viele Zustände
muss ein deterministischer Automat für $L_2$ also mindestens haben?
\index{Potenzmengenalgorithmus}
\end{enumerate}
\end{document}
