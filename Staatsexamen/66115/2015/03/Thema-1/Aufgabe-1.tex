\documentclass{lehramt-informatik-aufgabe}
\liLadePakete{formale-sprachen,automaten,minimierung}
\begin{document}
\let\l=\liLeereZelle
\let\f=\liFussnote
\def\z#1{
  \liZustandsMengenSammlungNr{#1}{
    {
      {0} {0}
      {1} {0,1}
      {2} {0,2}
      {3} {0,1,2}
    }
  }
}
\let\s=\liZustandsnameGross
\let\Z=\liZustandsPaar

\liAufgabenTitel{Alphabet 0 1 Anzahl Unterschied höchstes 3}
\section{Aufgabe 1
\index{Reguläre Sprache}
\footcite{66115:2015:03}}

Die Sprache $L$ über den Alphabet \liAlphabet{0, 1} enthält alle Wörter,
bei denen beim Lesen von links nach rechts der Unterschied in der Zahl
der $0$en und $1$en stets höchstens 3 ist. Also ist $w \in L$ genau
dann, wenn für alle $u$, $v$ mit $w = uv$ gilt $||u|_0 - |u|_1| \leq 3$.
Erinnerung: $|w|_a$ bezeichnet die Zahl der $a$’s im Wort $w$.
\footcite[Aufgabe 10: StEx Frühjahr 2015, Thema 1, Aufgabe 1 (Check-Up)]{theo:ab:1}

\begin{enumerate}

%%
% (a)
%%

\item Sei $A = (Q, \Sigma, \delta, q_0 , E)$ ein deterministischer
endlicher Automat für $L$. Es sei $w_1 = 111$, $w_2 = 11$, $w_3 = 1$,$ w_4
= \epsilon$, $w_5 = 0$, $w_6 = 00$, $w_7 = 000$. Machen Sie sich klar,
dass der Automat jedes dieser Wörter verarbeiten können muss. Folgern
Sie, dass der Automat mindestens sieben Zustände haben muss. Schreiben
Sie Ihr Argumentation schlüssig und vollständig auf.

\begin{liAntwort}
Ein deterministischen endlicher Automat hat keinen zusätzlichen Speicher
zur Verfügung, in dem die Anzahl der bisher vorkommenden $0$ und $1$
gespeichert werden könnte. Ein DEA kann die von der Sprache benötigten
Anzahl an $0$ und $1$ nur in Form von Zustanden speichern. Um die Anzahl
von 3 Einsen bzw. 3 Nullen zu speichern, sind also 6 Zustände nötig. Da
die Sprache auch das leere Wort erkennen soll, ist noch ein zusätzlicher
Zustand für dieses leere Wort nötig.
\end{liAntwort}

%%
% (b)
%%

\item Begründen Sie, dass $L$ regulär ist.

\begin{liAntwort}
$L$ ist nicht regulär. Widerspruchsbeweis durch das Pumping Lemma.

\begin{description}
\item[j] = Pumping-Zahl
\item[k] = zur Pumping-Zahl abhängige Zahl
\item[i] = Zahl, die die Werte $\{ 0, 1, 2, 3 \dots \}$ annimmt und mit
der aufgepumpt wird.
\end{description}

$\omega = 0^j 1^k = uvw$ mit $|j - k| \leq 3$ und $j, k \geq 0$

$\omega = 0^j 1^{j+k} = uvw$ mit $|k| \leq 3$

Wir teilen $0^j$ in $uv$ auf.

$u = 0^{j-i}$ $v = 0^i$ $w = 1^{j+k}$

$\rightarrow$ man kann nicht unendlich aufpumpen. $L$ ist nicht regulär.

\end{liAntwort}

%%
% (c)
%%

\item Jemand behauptet, diese Sprache sei nicht regulär und gibt
folgenden „Beweis“ dafür an: Wäre $L$ regulär, so sei $n$ eine
entsprechende Pumping-Zahl. Nun ist $w = (01)^n \in L$. Zerlegt man nun
$w = uxv$, wobei $u = 0$, $x = 1$, $v = (01)^{n-1}$ , so ist zum Beispiel
$ux^5 v \notin L$, denn es ist $ux^5 v = 01111101010101$.... Legen Sie
genau dar, an welcher Stelle dieser „Beweis” fehlerhaft ist.
\index{Pumping-Lemma (Reguläre Sprache)}

\begin{liAntwort}
\begin{enumerate}
\item Das Wort $(01)^n$ ist schlecht gewählt: Das für den Pumping-Lemma
Beweis gewählte Wort $(01)^n$ spiegelt nicht die Eigenschaft der
Sprache wieder, dass sich die Anzahl von Nullen und Einsen um maximal
drei unterscheidet.

\item Außerdem wurde das $(01)^n$ Wort falsch zerlegt. Für die
Pumping-Zahl $n = 3$ gibt es sehr wohl eine Zerlegung, die beim
Aufpumpen regulär ist, also: $\omega = 010101$ ($u = 01$, $x = 01$ und
$v = 01$). $ux^5 v = 01 0101010101 01 \in L$.
\end{enumerate}
\end{liAntwort}

%%
% (d)
%%

\item In anderen Fällen können nichtdeterministische endliche Automaten
echt kleiner sein als die besten deterministischen Automaten. Ein
Beispiel ist die Sprache $L_2 = \Sigma^* 1 \Sigma$ aller Wörter, deren
vorletztes Symbol $1$ ist. Geben Sie einen nicht-deterministischen
Automaten mit nur drei Zuständen an, $L_2$ erkennt.

\begin{liAntwort}
\begin{center}
\begin{tikzpicture}[li automat,node distance=3cm]
\node[state,initial]
(0) {$z_0$};

\node[state,right of=0]
(1) {$z_1$};

\node[state,right of=1,accepting]
(2) {$z_2$};

\path (0) edge[above,loop] node{$0$,$1$} (0);
\path (0) edge[above] node{$1$} (1);
\path (1) edge[above] node{$0$,$1$} (2);
\end{tikzpicture}
%\liFussnoteUrl{https://flaci.com/Apwzjufbg}
\end{center}
\end{liAntwort}

%%
% (e)
%%

\item Führen Sie auf Ihrem Automaten die Potenzmengenkonstruktion und
anschließend den Minimierungsalgorithmus durch. Wie viele Zustände
muss ein deterministischer Automat für $L_2$ also mindestens haben?
\index{Potenzmengenalgorithmus}

\begin{liAntwort}

%%
%
%%

\liPseudoUeberschrift{Potenzmengenkonstruktion}

\begin{tabular}{l|l|l|l}
Name & Zustandsmenge & Eingabe $0$ & Eingabe $1$ \\\hline\hline
\s0 & \z0 & \z0 & \z1 \\
\s1 & \z1 & \z2 & \z3 \\
\s2 & \z2 & \z0 & \z1 \\
\s3 & \z3 & \z2 & \z3 \\
\end{tabular}

\begin{center}
\begin{tikzpicture}[li automat,node distance=3cm]
\node[state,initial]
(0) {\s0};

\node[state,below of=0]
(1) {\s1};

\node[state,right of=0,accepting]
(2) {\s2};

\node[state,right of=1,accepting]
(3) {\s3};

\path (0) edge[above,loop] node{$0$} (0);
\path (0) edge[left] node{$1$} (1);
\path (1) edge[above,bend left] node{$0$} (2);
\path (1) edge[above] node{$1$} (3);
\path (2) edge[above] node{$0$} (0);
\path (2) edge[above,bend left] node{$1$} (1);
\path (3) edge[right] node{$0$} (2);
\path (3) edge[above,loop below] node{$1$} (3);

\end{tikzpicture}
%\liFussnoteUrl{https://flaci.com/Ajfcofpb9}
\end{center}

%%
%
%%

\liPseudoUeberschrift{Minimierungsalgorithmus}

\begin{center}
\begin{tabular}{|c||c|c|c|c|}
\hline
\s0 & \l  & \l  & \l  & \l  \\ \hline
\s1 & \f2 & \l  & \l  & \l  \\ \hline
\s2 & \f1 & \f1 & \l  & \l  \\ \hline
\s3 & \f1 & \f1 & \f2 & \l  \\ \hline\hline
    & \s0 & \s1 & \s2 & \s3 \\ \hline
\end{tabular}
\end{center}

\liFussnoten

\def\liZustandsPaarVariablenName{Z}

\begin{liUebergangsTabelle}{0}{1}
\Z01 & \Z02 \f2 & \Z13 \f2 \\
\Z23 & \Z01 & \Z13 \f2 \\
\end{liUebergangsTabelle}

Wie aus der oben stehenden Tabelle abzulesen ist, gibt es keine
äquivalenten Zustände. Der Automat kann nicht minimiert werden. Er ist
bereits minimal.

\end{liAntwort}
\end{enumerate}
\end{document}
