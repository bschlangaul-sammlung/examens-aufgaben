\documentclass{lehramt-informatik-aufgabe}
\liLadePakete{syntax}
\begin{document}
\liAufgabenTitel{Sortieren mit Stapel}

\section{Aufgabe 4
\index{Stapel (Stack)}
\footcite[Thema 2 Aufgabe 5]{examen:66115:2015:03}}

Gegeben seien die Standardstrukturen Stapel (Stack) und Schlange (Queue)
mit folgenden Standardoperationen:

\begin{center}
\begin{tabular}{l|l}
Stapel & Schlange \\\hline
\liJavaCode{boolean isEmpty()} & \liJavaCode{boolean isEmpty()} \\
\liJavaCode{void push(int e)} & \liJavaCode{enqueue(int e)} \\
\liJavaCode{int pop()} & \liJavaCode{int dequeue()} \\
\liJavaCode{int top()} & \liJavaCode{int head()} \\
\end{tabular}
\end{center}

\noindent
Beim Stapel gibt die Operation \liJavaCode{top()} das gleiche Element wie
\liJavaCode{pop()} zurück, bei der Schlange gibt \liJavaCode{head()} das gleiche
Element wie \liJavaCode{dequeue()} zurück. Im Unterschied zu \liJavaCode{pop()},
beziehungsweise \liJavaCode{dequeue()}, wird das Element bei \liJavaCode{top()} und
\liJavaCode{head()} nicht aus der Datenstruktur entfernt.

\begin{enumerate}

%%
%
%%

\item Geben Sie in Pseudocode einen Algorithmus \liJavaCode{sort(Stapel s)}
an, der als Eingabe einen Stapel \liJavaCode{s} mit \liJavaCode{n} Zahlen erhält und
die Zahlen in \liJavaCode{s} sortiert. (Sie dürfen die Zahlen wahlweise
entweder aufsteigend oder absteigend sortieren.) Verwenden Sie als
Hilfsdatenstruktur ausschließlich eine Schlange \liJavaCode{q}. Sie erhalten
volle Punktzahl, wenn Sie außer \liJavaCode{s} und \liJavaCode{q} keine weiteren
Variablen benutzen. Sie dürfen annehmen, dass alle Zahlen in \liJavaCode{s}
verschieden sind.

\begin{liAntwort}
\begin{minted}{md}
q := neue Schlange
while s not empty:
    q.enqueue(S.pop())
while q not empty:
    while s not empty and s.top() < q.head():
        q.enqueue(s.pop())
    s.push(q.dequeue)
\end{minted}

\liPseudoUeberschrift{Als Java-Code}

\liJavaExamen[firstline=5,lastline=25]{66115}{2015}{03}{Sort}
\end{liAntwort}

%%
%
%%

\item Analysieren Sie die Laufzeit\index{Algorithmische Komplexität (O-Notation)} Ihrer Methode in
Abhängigkeit von $n$.

\begin{liAntwort}
Zeitkomplexität: $\mathcal{O}(n^2)$, da es zwei ineinander
verschachtelte \liJavaCode{while}-Schleifen gibt, die von der Anzahl der
Elemente im Stapel abhängen.
\end{liAntwort}
\end{enumerate}
\end{document}
