\documentclass{lehramt-informatik-aufgabe}
\liLadePakete{wasserfall}
\begin{document}
\liAufgabenTitel{Vorgehensmodelle}

\section{Aufgabe 2: Vorgehensmodelle
\footcite[Thema 2 Teilaufgabe 1 Aufgabe 2]{examen:46116:2014:09}}

Software wird oft in definierten Prozessen entwickelt. Diese nennt man
Vorgehensmodelle.

\bigskip

\noindent
Allgemein

\begin{enumerate}

%%
% a)
%%

\item Was sind die Aufgaben eines Vorgehensmodells im Allgemeinen?

\begin{liAntwort}
\begin{itemize}
\item liefert Erfahrungen und bewährte Methoden
\item beschreibt die am Projekt beteiligten Rollen
\item legt Aufgaben und Aktivitäten fest
\item definiert einheitliche Begriffe
\item gibt Techniken, Werkzeuge, Richtlinien / Standards an
\end{itemize}
\end{liAntwort}

%%
% b)
%%

\item Was sind die wesentlichen Bestandteile eines
Vorgehensmodells\index{Prozessmodelle} und in welcher Beziehung stehen
diese zueinander?

\begin{liAntwort}
Bestandteile: Anforderungsanalyse, Modellierung, Implementierung, Test,
Auslieferung, Wartung

Die einzelnen Phasen bauen immer aufeinander aus. Je nach Vorgehensmodell
können sich Phasen auch wiederholen (Prototyping, Scrum...).
\end{liAntwort}

\end{enumerate}

\noindent
Ein frühes Vorgehensmodell ist das von Dr. Winston Royce 1970
formalisierte Wasserfallmodell.

\begin{enumerate}

%%
% c)
%%

\item Geben Sie eine schematische Darstellung des
Wasserfallmodells\index{Wasserfallmodell} an.

\begin{liAntwort}
\begin{tikzpicture}[wasserfall]
\node {Systemanforderung}; % A-1
\node {Softwareanforderung};
\node {Analyse};
\node {Entwurf};
\node {Implementierung};
\node {Test};
\node {Betrieb};

%
\foreach \i [count=\j] in {2,...,7}
{
  \draw[->, thick] (A-\i) -| (A-\j);
  \draw[->, thick] (A-\j) -| (A-\i);
}
\end{tikzpicture}
\end{liAntwort}

%%
% d)
%%

\item Nennen Sie zwei Probleme des Modells und erläutern Sie diese kurz.

\begin{liAntwort}
Fehler werden ggf. erst am Ende des Entwicklungsprozesses erkannt, da
erst dort das Testen stattfindet. Dadurch kann die Behebung eines
Fehlers sehr aufwändig und somit teuer werden.

Der Kunde / Endanwender wird erst nach der Implementierung wieder
eingebunden. Das bedeutet, dass er nach der Stellung der Anforderungen
keinen Einblick mehr in den Prozess hat und somit auch nicht
gegensteuern kann, falls ihm etwas nicht gefällt oder er etwas nicht
bedacht hat.
\end{liAntwort}

%%
% e)
%%

\item In welchen Situationen lässt sich das Wasserfallmodell gut
einsetzen?

\begin{liAntwort}
Das Wasserfallmodell ist geeignet, wenn es sich um ein von Anfang an
klar definiertes Projekt ohne große Komplexität handelt, bei dem alle
Anforderungen, Aufwand und Kosten schon zu Beginn des Projekts
feststehen bzw. abgeschätzt werden können.
\end{liAntwort}

\end{enumerate}

\noindent
Barry Boehm erweiterte das Wasserfallmodell 1979 zum so genannten
V-Modell.

\begin{enumerate}

%%
% f)
%%

\item Geben Sie eine schematische Darstellung des
V-Modells\index{V-Modell} an.

%%
% g)
%%

\item Nennen Sie zwei Probleme des Modells und erläutern Sie diese kurz.

\begin{liAntwort}
\begin{itemize}
\item Die Nachteile des Wasserfallmodells bestehen weiterhin!

\item Nicht für kleine Projekte geeignet, da aufwändige Tests vorgesehen
sind, die im Kleinen detailliert meist nicht stattfinden (können).
\end{itemize}
\end{liAntwort}

%%
% h)
%%

\item Welchen Vorteil hat das V-Modell gegenüber dem Wasserfallmodell?

\begin{liAntwort}
Für jedes Dokument besteht ein entsprechender Test (Validierung /
Verifikation). Dabei kann die Planung der Tests schon vor der
eigentlichen Durchführung geschehen, so dass Aktivitäten im Projektteam
parallelisiert werden können. So kann zum Beispiel der Tester die
Testfälle für den Akzeptanztest (=Test des Systementwurfs) entwicklen,
auch wenn noch keine Implementierung existiert.
\end{liAntwort}

\end{enumerate}

\noindent
In neuerer Zeit finden immer häufiger iterative und inkrementelle
Vorgehensweisen Anwendung.

\begin{enumerate}

%%
% i)
%%

\item Erklären Sie den Begriff iterative
Softwareentwicklung\index{Iterative Prozessmodelle}.

\begin{liAntwort}
Iterativ heißt, dass der Entwikclungsprozess mehrfach wiederholt wird:
statt den „Wasserfall“ einmal zu durchlaufen, werden
„kleine Wasserfälle“ hintereinander gesetzt.
\end{liAntwort}

%%
% j)
%%

\item Erklären Sie den Begriff inkrementelle Softwareentwicklung und
grenzen Sie ihn von iterativer Softwareentwicklung ab.

\begin{liAntwort}
Bei der inkrementellen Entwicklung wird das System Schritt für Schritt
fertig gestellt. D.\,h., dass ein Prototyp immer etwas mehr kann als der
Prototyp davor. Dies wird durch die iterative Entwicklung unterstützt,
da bei jeder Wiederholung des Entwicklungsprozesses ein neues
Inkrement entsteht, d. h. ein neuer Prototyp, der mehr Funktionalitäten
benutzt als der vorangegangene.
\end{liAntwort}

%%
% k)
%%

\item Nennen Sie jeweils zwei Vor- und Nachteile eines iterativen und
inkrementellen Vorgehens im Vergleich zum Wasserfallmodell.

\begin{liAntwort}
Vorteile:

\begin{itemize}
\item Risiken können früher erkannt werden.
\item volatile Anforderungen können besser berücksichtigt werden.
\item inkrementelle Auslieferung wird erleichtert.
\end{itemize}

Nachteile:

\begin{itemize}
\item komplexeres Projektmanagement
\item schwerer messbar
\item (Mehrarbeit)\footnote{Quelle: https://www.pst.ifi.lmu.de/Lehre/WS0607/pm/vorlesung/PM-02-Prozess.pdf}
\end{itemize}
\end{liAntwort}
\end{enumerate}
\end{document}
