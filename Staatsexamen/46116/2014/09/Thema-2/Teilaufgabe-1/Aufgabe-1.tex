\documentclass{lehramt-informatik-aufgabe}
\InformatikPakete{checkbox}
\begin{document}

\section{Aufgabe 1: Allgemeine SWT, Vorgehensmodelle und Requirements
Engineering
\footcite[Thema 2 Teilaufgabe 1 Aufgabe 1]{examen:46116:2014:09}}

Kreuzen Sie für die folgenden Multiple-Choice-Fragen genau die richtigen
Antworten deutlich an. Es kann mehr als eine Antwort richtig sein.

Jedes korrekt gesetzte oder korrekt nicht gesetzte Kreuz wird mit 1
Punkt gewertet. Jedes falsch gesetzte oder falsch nicht gesetzte Kreuz
wird mit -1 Punkt gewertet. Eine Frage kann entwertet werden, dann wird
sie nicht in der Korrektur berücksichtigt. Einzelne Antworten können
nicht entwertet werden. Entwerten Sie eine Frage wie folgt

Die gesamte Aufgabe wird nicht mit weniger als 0 Punkten gewertet.

\begin{enumerate}

%%
% 1.
%%

\item Welche Aussage ist wahr?

\begin{checkbox}
\item Je früher ein Fehler entdeckt wird, umso teurer ist seine
Korrektur.

\item Je später ein Fehler entdeckt wird, umso teurer ist seine
Korrektur.

\item Der Zeitpunkt der Entdeckung hat keinen Einfluss auf die Kosten.
\end{checkbox}

\begin{antwort}
2 ist richtig: Je später der Fehler entdeckt wird, desto mehr wurde er
schon in das Projekt „eingearbeitet“, daher dauert das Beseitigen des
Fehlers länger und das kostet mehr Geld.
\end{antwort}

%%
% 2.
%%

\item  Mit welcher Methodik können Funktionen spezifiziert werden?

\begin{checkbox}
\item Als Funktionsvereinbarung in einer Programmiersprache
\item Mit den Vor- und Nachbedingungen von Kontrakten
\item Als Zustandsautomaten
\end{checkbox}

\begin{antwort}
2 und 3 ist richtig: Die Spezifikation soll unabhängig von einer
Programmiersprache sein.
\end{antwort}

%%
% 3.
%%

\item Welche Vorgehensmodelle sind für Projekte mit häufigen Änderungen
geeignet?

\begin{checkbox}
\item Extreme Programming (XP)\index{EXtreme Programming}
\item Das V-Modell 97\index{V-Modell}
\item Scrum\index{SRUM}
\end{checkbox}

\begin{antwort}
1 und 3 ist richtig. Das V-Modell ist ein starres Vorgehensmodell, bei
dem alle Anforderungen zu Beginn vorhanden sein müssen.
\end{antwort}

%%
% 4.
%%

\item Welche der folgenden Aussagen ist korrekt?

\begin{checkbox}
\item Mittels Prototyping\index{Prototyping} versucht man die Anzahl an
nötigen Unit-Tests zu reduzieren.

\item Ein Ziel von Prototyping ist die Erhöhung der Qualität während der
Anforderungsanalyse.

\item Mit Prototyping versucht man sehr früh Feedback von Stakeholdern
zu erhalten.
\end{checkbox}

\begin{antwort}
2 und 3 ist richtig: Prototypen müssen auch getestet werden. Es kann
nicht an Tests gespart werden. Durch das häufige Feedback des Kunden /
der Stakeholder können die Anforderungen immer genauer und klarer
erfasst werden.
\end{antwort}

%%
% 5.
%%

\item Welche der folgenden Aussagen ist korrekt?

\begin{checkbox}
\item Bei der Architektur sollten funktionale\index{Funktionale
Anforderungen} und nicht-funktionale
Anforderungen\index{Nicht-funktionale Anforderungen} beachtet werden.

\item Bei der Architektur soliten nur funktionale Anforderungen beachtet
werden.

\item Bei der Architektur sollten nur nicht-funktionale Anforderungen
beachtet werden.

\item Bei der Architektur sollte auf die mögliche Änderungen von
Komponenten geachtet werden.
\end{checkbox}

\begin{antwort}
1 und 4 ist richtig: Mögliche Änderungen werden durch klar definierte
Schnittstellen und wenig Kopplung der Komponenten erleichtert. (Kopplung
handelt von Abhängigkeiten zwischen Modulen. Kohäsion handelt von
Abhängigkeiten zwischen Funktionen innerhalb eines Moduls.)
\end{antwort}
\end{enumerate}
\end{document}
