\documentclass{lehramt-informatik-aufgabe}
\InformatikPakete{mathe,syntax}
\begin{document}

\section{Aufgabe 1: „Formale Verifikation“
\index{Vollständige Induktion}
\footcite[Thema 2 Teilaufgabe 1 Aufgabe 1]{examen:46116:2014:03}}

Gegeben sei folgende Methode zur Berechnung der Anzahl der notwendigen
Züge beim Spiel \emph{„Die Türme von Hanoi“}:
\footcite{sosy:pu:5:1}

\inputcode[firstline=4,lastline=15]{aufgaben/sosy/examen_46116_2014_03/Hanoi}

\begin{enumerate}

%%
% a)
%%

\item Beweisen Sie formal mittels vollständiger Induktion, dass zum
Umlegen von $k$ Scheiben (z.\,B. vom Turm A zum Turm C) insgesamt
$2^k-1$ Schritte notwendig sind, also dass für $k \geq 0$ folgender
Zusammenhang gilt:

\begin{displaymath}
\text{hanoi}(k,\text{'A'},\text{'C'}) = 2^k - 1
\end{displaymath}

\begin{antwort}
Zu zeigen:

\begin{displaymath}
\text{hanoi}(k,\text{'A'},\text{'C'}) = 2^k - 1
\end{displaymath}

%%
%
%%

\ueberschrift{I. A.} $k=0$

\begin{displaymath}
\text{hanoi}(0,\text{'A'},\text{'C'}) = 0
\end{displaymath}

\begin{displaymath}
2^0 - 1 = 1 - 1 = 0
\end{displaymath}

%%
%
%%

\ueberschrift{I. V.}

\begin{displaymath}
\text{hanoi}(k,\text{'A'},\text{'C'}) = 2^k - 1
\end{displaymath}

%%
%
%%

\ueberschrift{I. S.} $k \rightarrow k + 1$

\begin{align*}
\text{hanoi}(k +  1, \text{'A'},\text{'C'})
& = 1 + \text{hanoi}(k, \text{'A'},\text{'B'}) + \text{hanoi}(k, \text{'B'},\text{'C'})\\
& = 1 + 2^k - 1 +  2^k - 1 \\
& = 2 \cdot 2^k - 1 \\
& = 2^{k+1} - 1 \\
\end{align*}

\end{antwort}

%%
% b)
%%

\item Geben Sie eine geeignete
Terminierungsfunktion\index{Terminierungsfunktion} an und begründen Sie
kurz Ihre Wahl!

\begin{antwort}
Betrachte die Argumentenfolge $k, k-1, k-2, \dots, 0$.

$\Rightarrow$ Terminierungsfunktion: $T(k) = k$

Nachweis für ganzzahlige $k \geq 0$:

\begin{itemize}
\item $T(k)$ ist auf der Folge der Argumente streng monoton fallend bei
jedem Rekursionsschritt.

\item Bei der impliziten Annahme $k$ ist ganzzahlig und $k \geq 0$ ist
$T(k)$ nach unten durch $0$ beschränkt.
\end{itemize}
\end{antwort}
\end{enumerate}
\end{document}
