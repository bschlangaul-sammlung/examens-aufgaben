\documentclass{lehramt-informatik-aufgabe}
\liLadePakete{mathe,spalten,rmodell,syntax}
\begin{document}

\section{Aufgabe 2: Relationale Algebra
\index{Relationale Algebra}
\footcite[Thema 2 Teilaufgabe 2 Aufgabe 2]{examen:46116:2014:03}
}

Gegeben sei das folgende relationale Schema mitsamt Beispieldaten für
eine Datenbank von Mitfahrgelegenheiten. Die Primärschlüssel-Attribute
sind jeweils unterstrichen, Fremdschlüssel sind überstrichen.
\footcite{db:pu:wh}

{
\footnotesize
\begin{multicols}{2}
„Kunde":

\begin{tabular}{|l|l|l|l|}
\hline
\liPrimaer{KID} & Name & Vorname & \liFremd{Stadt}\\\hline\hline
K1 & Meier & Stefan & S3\\\hline
K2 & Müller & Peta & S3\\\hline
K3 & Schmidt & Christine & S2\\\hline
K4 & Schulz & Michael & S4\\\hline
\end{tabular}

„Stadt"

\begin{tabular}{|l|l|l|}
\hline
\liPrimaer{SID} & SName & Bundesland\\\hline\hline
S1 & Berlin & Berlin\\\hline
S2 & Nürn & Bayern\\\hline
S3 & Köln & Nordrhein-Wesffalen\\\hline
S4 & Stuttgart & Baden-Württemberg\\\hline
S5 & München & Bayer\\\hline
\end{tabular}
\end{multicols}

\begin{multicols}{2}
„Angebot":

\begin{tabular}{|l|l|l|l|l|}
\hline
\liPrimaer{KID} & \liFremd{Start} & \liFremd{Ziel} & \liPrimaer{Datum} & Plätze\\\hline\hline
K4 & S4 & S5 & 08.07.2011 & 3\\\hline
K4 & S5 & S4 & 10.07.2011 & 3\\\hline
K1 & S1 & S5 & 08.07.2011 & 3\\\hline
K3 & S2 & S3 & 15.07.2011 & 1\\\hline
K4 & S4 & S1 & 15.07.2011 & 3\\\hline
K1 & S5 & S5 & 09.07.2011 & 2\\\hline
\end{tabular}

„Anfrage":

\begin{tabular}{|l|l|l|l|}
\hline
\liPrimaer{KID} & \liFremd{Start} & \liFremd{Ziel} & \liPrimaer{Datum}\\\hline\hline
K2 & S4 & S5 & 08.07.2011\\\hline
K2 & S5 & S4 & 10.07.2011\\\hline
K3 & S2 & S3 & 08.07.2011\\\hline
K3 & S3 & S2 & 10.07.2011\\\hline
K2 & S4 & S5 & 05.07.2011\\\hline
K2 & S5 & S4 & 17.07.2011\\\hline
\end{tabular}
\end{multicols}
}

\renewcommand{\labelenumi}{\arabic{enumi}.}
\begin{enumerate}
\item Formulieren Sie die folgenden Anfragen auf das gegebene Schema in
relationaler Algebra:

\begin{itemize}
\item Finden Sie die Namen aller Städte in Bayern!

\begin{antwort}
$\pi_{\text{SName}}(\sigma_{\text{Bundesland} = \text{Bayern}}(\text{Stadt}))$
\end{antwort}

%%
%
%%

\item Finden Sie die SIDs aller Städte, für die weder als Start noch als
Ziel eine Anfrage vorliegt!

\begin{antwort}
$
\pi_{\text{SID}}(\text{Stadt}) - \pi_{\text{Start}}(\text{Anfage}) - \pi_{\text{Ziel}}(\text{Anfrage})
$
\end{antwort}

%%
%
%%

\item Finden Sie alle IDs von Kunden, welche eine Fahrt in ihrer
Heimatstadt starten.

\begin{antwort}
\begin{multline*}
\pi_{\text{KID}}(\\
  \text{Kunde} \bowtie_{\text{Kunde.KID} = \text{Anfrage.KID} \land \text{Kunde.Stadt} = \text{Anfrage.Stadt}} \text{Anfrage}
)\\
\land\\
\pi_{\text{KID}}(\\
  \text{Kunde} \bowtie_{\text{Kunde.KID} = \text{Angebot.KID} \land \text{Kunde.Stadt} = \text{Angebot.Stadt}} \text{Angebot}
  )
\end{multline*}
\end{antwort}

%%
%
%%

\item Geben Sie das Datum aller angebotenen Fahrten von München nach
Stuttgart aus!

\begin{antwort}
\begin{multline*}
\pi_{\text{Datum}}(\\
  (\text{Angebot} \bowtie_{\text{Start} = \text{SID} \land \text{SName} = \mlq\text{München}\mrq} \text{Stadt})\\
  \bowtie_{\text{Ziel} = \text{SID} \land \text{SName} = \mlq\text{Stuttgart}\mrq}\\
  \text{Stadt}\\
)
\end{multline*}
\end{antwort}

Variante 2:

\begin{antwort}
\begin{multline*}
\pi_{\text{Datum}}(\\
  \sigma_{
    \text{Sname} = \mlq\text{München}\mrq \land
    \text{Zname} = \mlq\text{Stuttgart}\mrq
  }(\\
    \rho_{
      \text{Zname} \leftarrow \text{Sname},
      \text{SID1} \leftarrow \text{SID}
    }(\text{Stadt})\\
    \bowtie_{\text{Ziel} = \text{SID1}}\\
    \text{Angebot}\\
    \bowtie_{\text{Start} = \text{SID}}\\
    \text{Stadt}
  )
)
\end{multline*}
\end{antwort}

%%
%
%%

\end{itemize}

\item Geben Sie das Ergebnis (bezüglich der Beispieldaten) der folgenden
Ausdrücke der relationalen Algebra als Tabellen an:

%%
%
%%

\begin{itemize}
\item $\pi_{\text{KID}} (\text{Angebot}) \bowtie \text{Kunde}$

\begin{antwort}
Zeile mit der Petra Müller fällt weg.

\begin{tabular}{|l|l|l|l|}
\hline
\liPrimaer{KID} & Name & Vorname & \liFremd{Stadt}\\\hline\hline
K1 & Meier & Stefan & S3\\\hline
K3 & Schmidt & Christine & S2\\\hline
K4 & Schulz & Michael & S4\\\hline
\end{tabular}
\end{antwort}

%%
%
%%

\item $
\pi_{(\text{KID},\text{Stadt})} (\text{Kunde})
\bowtie_{\text{Kunde.Stadt} = \text{Angebot.Ziel}}
\pi_{\text{Plaetze}} (\text{Angebot})$

\begin{antwort}

\begin{tabular}{|l|l|l|}
\hline
KID & Stadt & Plätze \\\hline\hline
K1 & S3 & 1 \\\hline
K2 & S3 & 1 \\\hline
K4 & S4 & 1 \\\hline
K4 & S4 & 2 \\\hline
\end{tabular}
\end{antwort}
\end{itemize}
\end{enumerate}

\begin{minted}{SQL}
CREATE TABLE `Anfrage` (
  `KID` varchar(100) NOT NULL,
  `Start` varchar(100) DEFAULT NULL,
  `Ziel` varchar(100) DEFAULT NULL,
  `Datum` date NOT NULL
) ENGINE=InnoDB DEFAULT CHARSET=latin1;

INSERT INTO `Anfrage` (`KID`, `Start`, `Ziel`, `Datum`) VALUES
('K2', 'S4', 'S5', '2011-07-05'),
('K2', 'S4', 'S5', '2011-07-08'),
('K3', 'S2', 'S3', '2011-07-08'),
('K2', 'S5', 'S4', '2011-07-10'),
('K3', 'S3', 'S2', '2011-07-10'),
('K2', 'S5', 'S4', '2011-07-17');

CREATE TABLE `Angebot` (
  `KID` varchar(100) NOT NULL,
  `Start` varchar(100) DEFAULT NULL,
  `Ziel` varchar(100) DEFAULT NULL,
  `Datum` date NOT NULL,
  `Plätze` int(10) DEFAULT NULL
) ENGINE=InnoDB DEFAULT CHARSET=latin1;

INSERT INTO `Angebot` (`KID`, `Start`, `Ziel`, `Datum`, `Plätze`) VALUES
('K1', 'S1', 'S5', '2011-07-08', 3),
('K4', 'S4', 'S5', '2011-07-08', 3),
('K1', 'S5', 'S4', '2011-07-09', 2),
('K4', 'S5', 'S4', '2011-07-10', 3),
('K3', 'S2', 'S3', '2011-07-15', 1),
('K4', 'S4', 'S1', '2011-07-15', 3);

CREATE TABLE `Kunde` (
  `KID` varchar(100) NOT NULL,
  `Name` varchar(100) DEFAULT NULL,
  `Vorname` varchar(100) DEFAULT NULL,
  `Stadt` varchar(100) DEFAULT NULL
) ENGINE=InnoDB DEFAULT CHARSET=latin1;

INSERT INTO `Kunde` (`KID`, `Name`, `Vorname`, `Stadt`) VALUES
('K1', 'Meier', 'Stefan', 'S3'),
('K2', 'Müller', 'Petra', 'S3'),
('K3', 'Schmidt', 'Christine', 'S2'),
('K4', 'Schulz', 'Michael', 'S4');

CREATE TABLE `Stadt` (
  `SID` varchar(100) NOT NULL,
  `SName` varchar(100) NOT NULL,
  `Bundesland` varchar(100) NOT NULL
) ENGINE=InnoDB DEFAULT CHARSET=latin1;

INSERT INTO `Stadt` (`SID`, `SName`, `Bundesland`) VALUES
('S1', 'Berlin', 'Berlin'),
('S2', 'Nürnberg', 'Bayern'),
('S3', 'Köln', 'NRW'),
('S4', 'Stuttgart', 'BW'),
('S5', 'München', 'Bayern');

ALTER TABLE `Anfrage`
  ADD PRIMARY KEY (`Datum`,`KID`),
  ADD KEY `Start` (`Start`),
  ADD KEY `Ziel` (`Ziel`);

ALTER TABLE `Angebot`
  ADD PRIMARY KEY (`Datum`,`KID`),
  ADD KEY `Start` (`Start`),
  ADD KEY `Ziel` (`Ziel`);

ALTER TABLE `Kunde`
  ADD PRIMARY KEY (`KID`),
  ADD KEY `Stadt` (`Stadt`);

ALTER TABLE `Stadt`
  ADD PRIMARY KEY (`SID`);

ALTER TABLE `Anfrage`
  ADD CONSTRAINT `Anfrage_ibfk_1` FOREIGN KEY (`Start`) REFERENCES `Stadt` (`SID`),
  ADD CONSTRAINT `Anfrage_ibfk_2` FOREIGN KEY (`Ziel`) REFERENCES `Stadt` (`SID`);

ALTER TABLE `Angebot`
  ADD CONSTRAINT `Angebot_ibfk_1` FOREIGN KEY (`Start`) REFERENCES `Stadt` (`SID`),
  ADD CONSTRAINT `Angebot_ibfk_2` FOREIGN KEY (`Ziel`) REFERENCES `Stadt` (`SID`);

ALTER TABLE `Kunde`
  ADD CONSTRAINT `Kunde_ibfk_1` FOREIGN KEY (`Stadt`) REFERENCES `Stadt` (`SID`);
\end{minted}

\end{document}
