\documentclass{lehramt-informatik-aufgabe}
\liLadePakete{syntax}
\begin{document}
\liAufgabenTitel{Schuldatenbank}

\section{Aufgabe 3
\index{SQL}
\footcite[Thema 2 Teilaufgabe 2 Aufgabe 3]{46116:2018:09}
}

Gegeben sei das folgende Datenbank-Schema, das für die Speicherung der
Daten einer Schule entworfen wurde, zusammen mit einem Teil seiner
Ausprägung. Die Primärschlüssel-Attribute sind jeweils unterstrichen.
\footcite{db:ab:3}

Die Relation \emph{Schüler} enthält allgemeine Daten zu den
Schülerinnen und Schülern. Schülerinnen und Schüler nehmen an Prüfungen
in verschiedenen Unterrichtsfächern teil und erhalten dadurch Noten.
Diese werden in der Relation \emph{Noten} abgespeichert. Prüfungen haben
ein unterschiedliches Gewicht. Beispielsweise hat ein mündliches
Ausfragen oder eine Extemporale das Gewicht 1, während eine Schulaufgabe
das Gewicht 2 hat.

Schüler:

\begin{tabular}{llll}
SchülerID  & Vorname & Nachname  & Klasse \\
1          & Laura   & Müller    & 4A     \\
2          & Linus   & Schmidt   & 4A     \\
3          & Jonas   & Schneider & 4A     \\
4          & Liam    & Fischer   & 4B     \\
5          & Tim     & Weber     & 4B     \\
6          & Lea     & Becker    & 4B     \\
7          & Emilia  & Klein     & 4C     \\
8          & Julia   & Wolf      & 4C
\end{tabular}

Noten:

\begin{tabular}{lllll}
SchülerID {[}Schüler{]} & Schulfach  & Note & Gewicht & Datum      \\
1                       & Mathematik & 3    & 2       & 23.09.2017 \\
1                       & Mathematik & 1    & 1       & 03.10.2017 \\
1                       & Mathematik & 2    & 2       & 15.10.2017 \\
1                       & Mathematik & 4    & 1       & 11.11.2017
\end{tabular}

% Datenbankname: schuldatenbank
\begin{minted}{sql}
CREATE TABLE Schüler (
  SchülerID INTEGER PRIMARY KEY NOT NULL,
  Vorname VARCHAR(20),
  Nachname VARCHAR(20),
  Klasse VARCHAR(5)
);

CREATE TABLE Noten (
  SchülerID INTEGER NOT NULL,
  Schulfach VARCHAR(20),
  Note INTEGER,
  Gewicht INTEGER,
  Datum DATE,
  PRIMARY KEY (SchülerID, Schulfach, Datum),
  FOREIGN KEY (SchülerID) REFERENCES Schüler(SchülerID)
);

INSERT INTO Schüler VALUES
  (1, 'Laura', 'Müller', '4A'),
  (2, 'Linus', 'Schmidt', '4A'),
  (3, 'Jonas', 'Schneider', '4A'),
  (4, 'Liam', 'Fischer', '4B'),
  (5, 'Tim', 'Weber', '4B'),
  (6, 'Lea', 'Becker', '4B'),
  (7, 'Emilia', 'Klein', '4C'),
  (8, 'Julia', 'Wolf', '4C');

INSERT INTO Noten VALUES
  (1, 'Mathematik', 3, 2, '23.09.2017'),
  (1, 'Mathematik', 1, 1, '03.10.2017'),
  (1, 'Mathematik', 2, 2, '15.10.2017'),
  (1, 'Mathematik', 4, 1, '11.11.2017');
\end{minted}
\index{SQL mit Übungsdatenbank}

\begin{enumerate}

%%
% (a)
%%

\item Geben Sie die SQL-Befehle an, die notwendig sind, um die oben
dargestellten Tabellen in einer SQL-Datenbank anzulegen.
\index{CREATE TABLE}

\begin{antwort}
\begin{minted}{sql}
CREATE TABLE IF NOT EXISTS Schüler (
  SchülerID INTEGER PRIMARY KEY NOT NULL,
  Vorname VARCHAR(20),
  Nachname VARCHAR(20),
  Klasse VARCHAR(5)
);

CREATE TABLE IF NOT EXISTS Noten (
  SchülerID INTEGER NOT NULL,
  Schulfach VARCHAR(20),
  Note INTEGER,
  Gewicht INTEGER,
  Datum DATE,
  PRIMARY KEY (SchülerID, Schulfach, Datum),
  FOREIGN KEY (SchülerID) REFERENCES Schüler(SchülerID)
);
\end{minted}
\end{antwort}

%%
% (b)
%%

\item Entscheiden Sie jeweils, ob folgende Einfügeoperationen vom
gegebenen Datenbanksystem (mit der angegebenen Ausprägungen) erfolgreich
verarbeitet werden können und begründen Sie Ihre Antwort kurz.
\index{INSERT}

\begin{minted}{sql}
INSERT INTO
Schüler (SchülerID, Vorname, Nachname, Klasse)
VALUES (9, 'Johannes', 'Schmied', '4C');
\end{minted}

\begin{antwort}[richtig]
Nein, denn die Spalte, die Primärschlüssel heißt Schüler und nicht
SchülerID. Außerdem existiert bereits ein Schuüler mit der ID 6.
\end{antwort}

\begin{antwort}
nein, SchülerID muss als Primärschlüssel eindeutig sein und ist bereits vergeben
\end{antwort}

\begin{minted}{sql}
INSERT INTO Noten VALUES (6, 'Chemie', 1, 2, '1.4.2020');
\end{minted}

\begin{antwort}[richtig]
Nein, Datum ist zwingend nötig, da es im Primärschlüssel enthalten ist.
Es gibt keine Schüler mit der ID 9. Der müsste vorher angelegt sein,
da die Spalte SchülerID von Noten auf den Fremdschlüssel Schüler aus
der Schülertabelle verweist.
\end{antwort}

\begin{antwort}
nein, Schüler mit der ID 9 existiert noch nicht, sodass diese Noten
nicht eingetragen werden können, da Integritätsbedingungen nicht erfüllt
(SchülerID ist Fremdschlüssel), zudem fehlt das Datum. Da es sich
hier um ein Schlüsselattribut handelt, kann es nicht NULL sein.
\end{antwort}

%%
% (c)
%%

\item Geben Sie die Befehle für die folgenden Aktionen in SQL an.
Beachten Sie dabei, dass die Befehle auch noch bei Änderungen des oben
gegebenen Datenbank- zustandes korrekte Ergebnisse zurückliefern müssen.
\index{ALTER TABLE}

\begin{itemize}
\item Die Schule möchte verhindern, dass in die Datenbank mehrere Kinder
mit dem selben Vornamen in die gleiche Klasse kommen. Dies soll bereits
auf Datenbankebene verhindert werden. Dabei sollen die Primärschlüssel
nicht verändert werden. Geben Sie den Befehl an, der diese Änderung
durchführt.

\begin{antwort}[falsch]
\begin{minted}{sql}
ALTER TABLE Schüler ALTER COLUMN Vorname ADD UNIQUE;
\end{minted}
\end{antwort}

\begin{antwort}
\begin{minted}{sql}
ALTER TABLE Schüler
ADD CONSTRAINT Vorname_eindeutig UNIQUE (Vorname, Klasse);
\end{minted}
\end{antwort}

\item Der Schüler Tim Weber (SchülerID: 5) wechselt die Klasse. Geben
Sie den SQL-Befehl an, der den genannten Schüler in die Klasse “4C“
überführt.
\index{UPDATE}

\begin{antwort}[richtig]
\begin{minted}{sql}
UPDATE Schüler
SET Klasse = '4C'
WHERE Vorname = 'Tim' AND Nachname = 'Weber' AND SchülerID = 5;
\end{minted}
\end{antwort}

\begin{antwort}
\begin{minted}{sql}
UPDATE Schüler
SET Klasse = '4C' WHERE SchülerID = 5;
\end{minted}
\end{antwort}

\item Die Schülerin Laura Müller (SchülerID: 1) zieht um und wechselt
die Schule. Löschen Sie die Schülerin aus der Datenbank. Nennen Sie
einen möglichen Effekt, welcher bei der Verwendung von Primär- und
Fremdschlüsseln auftreten kann.
\index{DELETE}

\begin{antwort}[richtig]
\begin{minted}{sql}
DELETE FROM Noten WHERE SchülerID = 1;
DELETE FROM Schüler WHERE Schüler = 1
\end{minted}

Es könnte passieren, dass man vergisst die Einträge in Noten zu löschen,
denn der Fremdschlüssel SchülerID verweist auf den Primärschlüssel
SchülerID in Schüler.
\end{antwort}

\begin{antwort}
Löschen aller Noten von Laura Müller, falls ON DELETE CASCADE gesetzt.
Oder es müssen erst alle Fremdschlüsselverweise auf diese SchülerID in
Noten gelöscht werden
\end{antwort}

\item Erstellen Sie eine View „DurchschnittsNoten“, die die folgenden
Spalten beinhaltet: Klasse, Schulfach, Durchschnittsnote Hinweis:
Beachten Sie die Gewichte der Noten.
\index{VIEW}
\index{GROUP BY}

\begin{antwort}
\begin{minted}{sql}
CREATE VIEW DurchschnittsNoten AS (
  (SELECT s.Klasse, n.Schulfach, (SUM(n.Note * n.Gewicht) / SUM(n.Gesicht)) AS Durchschnittsnote
  FROM Noten n, Schüler s
  WHERE s.SchülerID = n.SchülerID AND n.Gewicht = 1
  GROUP BY s.Klasse, n.Schulfach)
);
\end{minted}
\end{antwort}

\item Geben Sie den Befehl an, der die komplette Tabelle „Noten“ löscht.
\index{DROP TABLE}

\begin{antwort}
\begin{minted}{sql}
DROP TABLE Noten;
\end{minted}
\end{antwort}

\end{itemize}

%%
% (d)
%%

\item Formulieren Sie die folgenden Anfragen in SQL. Beachten Sie dabei,
dass sie SQL-Befehle auch noch bei Änderungen der Ausprägung die
korrekten Anfrageergebnisse zurückgeben sollen.

\begin{itemize}

%%
%
%%

\item Gesucht ist die durchschnittliche Note, die im Fach Mathematik
vergeben wird.

Hinweis: Das Gewicht ist bei dieser Anfrage nicht relevant

\begin{antwort}
\begin{minted}{sql}
SELECT AVG(Note)
FROM Noten
WHERE Schulfach = 'Mathematik';
\end{minted}
\end{antwort}

Musterlösung nimmt View zu hand.

\begin{antwort}
\begin{minted}{sql}
SELECT AVG(Note) AS durchschnittlicheNote
FROM DurchschnittsNoten
WHERE Schulfach = 'Mathematik';
\end{minted}
\end{antwort}

%%
%
%%

\item Berechnen Sie die Anzahl der Schüler, die im Fach Mathematik am
23.09.2017 eine Schulaufgabe (d.h. Gewicht=2) geschrieben haben.

\begin{antwort}
Ich habe in der WHERE Klauses das Schulfach vergessen. Sonst richtig.
Außerdem habe ich das Datum einfach übernommen.
\begin{minted}{sql}
SELECT COUNT(SchülerID) AS AnzahlSchüler
FROM Noten
WHERE Datum = 2017-09-23 AND Gewichtung = 2 AND Schulfach = 'Mathematik';
\end{minted}
\end{antwort}

%%
%
%%

\item Geben Sie die SchülerID aller Schüler zurück, die im Fach
Mathematik mindestens drei mal die Schulnote 6 geschrieben haben.

\begin{antwort}
\begin{minted}{sql}
SELECT SchülerID
FROM Noten
WHERE Schulfach = 'Mathematik' AND Note = 6
GROUP BY SchülerID
HAVING COUNT(*) >= 3;
\end{minted}
\end{antwort}

%%
%
%%

\item Gesucht ist der Notendurchschnitt bezüglich jedes Fachs der Klasse
„4A“.

\begin{antwort}
\begin{minted}{sql}
SELECT n.Schulfach, AVG (n.Note)
FROM Schüler s, Noten n
WHERE s.SchülerID = n.SchülerID AND s.Klasse = ’4 A ’
GROUP BY n.Schulfach
\end{minted}

Hier wäre Gewicht unberücksichtigt, also möglicherweise besser, auf die oben erstellte
View zurückgreifen:

\begin{minted}{sql}
SELECT Schulfach, durchschnittlicheNote,
FROM DurchschnittsNoten
WHERE Klasse = '4a';
\end{minted}
\end{antwort}
\end{itemize}

%%
% (e)
%%

\item Geben Sie jeweils an, welchen Ergebniswert die folgenden
SQL-Befehle für die gegebene Ausprägung zurückliefern.

\begin{minted}{sql}
SELECT COUNT(DISTINCT Klasse)
FROM
Schüler NATURAL JOIN Noten;
\end{minted}

\begin{antwort}
1 $\rightarrow$ 4A von Laura Müller
\end{antwort}

\begin{minted}{sql}
SELECT COUNT(all Klasse)
FROM
Noten, Schüler;
\end{minted}

\begin{antwort}
32 $\rightarrow$ Kreuzprodukt, zählt alle Einträge in Klasse
\end{antwort}

\begin{minted}{sql}
SELECT COUNT(Note)
FROM
Schüler NATURAL LEFT OUTER JOIN Noten;
\end{minted}

\begin{antwort}
4 $\rightarrow$ null-Werte nicht mitgezählt, 4 Noten von Laura
\end{antwort}

\begin{minted}{sql}
SELECT COUNT(*)
FROM
Schüler NATURAL LEFT OUTER JOIN Noten;
\end{minted}

\begin{antwort}
11 $\rightarrow$ alle Schüler, Laura dabei 4-mal, weil 4 Noten
\end{antwort}
\end{enumerate}
\end{document}
