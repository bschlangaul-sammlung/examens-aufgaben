\documentclass{lehramt-informatik-aufgabe}
\liLadePakete{syntax,uml}
\begin{document}
\liAufgabenTitel{Fußballweltmeisterschaft}

\section{Aufgabe 3 (Objektorientierte Implementierung)
\footcite[Thema 1 Teilaufgabe 1 Aufgabe 3]{examen:46116:2018:03}}

Für die nächste Fußballweltmeisterschaft möchte ein Wettbüro ein
Programm zur Verwaltung von Spielern, Vereinen und
(National-)Mannschaften entwickeln. Dazu wurde bereits das folgende
UML-Klassendiagramm\index{Klassendiagramm}
entworfen.\footcite{oomup:pu:4}

\begin{center}
\begin{tikzpicture}
\umlclass{Mannschaft}{
  -land: String\\
  -anzahlSpieler: Integer
}{
  +Mannschaft(land: String)\\
  +hinzufuegen(s: Spieler)\\
  +torschuetzenkoenig():String
}

\umlclass[right=1.5cm of Mannschaft]{Spieler}{
  -name: String
  -tore: Integer
}{
  +Spieler(name: String, verein: Verein)\\
  +getName(): String\\
  +getTore(): Integer\\
  +schiesseTor()\\
  +wechsleVerein(neuerVerein: Verein)
}

\umlclass[below=1cm of Spieler]{Verein}{
  -name: String
}{
  +Verein(name:String)\\
  +getName():String
}

\umluniassoc[mult2=0..20,arg2=-team,pos2=0.6]{Mannschaft}{Spieler}
\umluniassoc[mult2=0..20,arg2=-team]{Spieler}{Verein}
\end{tikzpicture}
\end{center}

\noindent
Es kann angenommen werden, dass die Klasse \java{Verein} bereits
implementiert ist. In den folgenden
Implementierungsaufgaben\index{Implementierung in Java} können Sie eine
objektorientierte Programmiersprache Ihrer Wahl verwenden. Die
verwendete Sprache ist anzugeben. Zu beachten sind jeweils die im
Klassendiagramm angegebenen Sichtbarkeiten von Attributen, Rollennamen,
Konstruktorn und Operationen.

\begin{enumerate}

%%
% a)
%%

\item Es ist eine Implementierung der Klasse \java{Spieler} anzugeben.
Der Konstruktor soll die Instanzvariablen mit den gegebenen Parametern
initialisieren, wobei die. Anzahl der Tore nach der Objekterzeugung
gleich 0 sein soll. Ansonsten kann die Funktionalität der einzelnen
Operationen aus deren Namen geschlossen werden.

\begin{antwort}
\inputcode[firstline=3]{aufgaben/oomup/pu_4/verein/Spieler}
\end{antwort}

%%
% b)
%%

\item Es ist eine Implementierung der Klasse \java{Mannschaft}
anzugeben. Der Konstruktor soll das Land initialisieren, die Anzahl der
Spieler auf 0 setzen und das Team mit einem noch „leeren“
Array\index{Feld (Array)} der Länge 20 initialisieren. Das Team soll mit
der Methode \java{hinzufuegen} um einen Spieler erweitert werden. Die
Methode \java{torschuetzenkoenig} soll den Namen eines Spielers aus dem
Team zurückgeben, der die meisten Tore für die Mannschaft geschossen
hat. Ist das für mehrere Spieler der Fall, dann kann der Name eines
beliebigen solchen Spielers zurückgegeben werden. Ist noch kein Spieler
im Team, dann soll der String \java{"Kein Spieler vorhanden"}
zurückgegeben werden.

\begin{antwort}
\inputcode[firstline=3]{aufgaben/oomup/pu_4/verein/Mannschaft}
\end{antwort}

%%
% c)
%%

\item Schreiben Sie den Rumpf einer
\java{main}-Methode\index{main-Methode}, so dass nach Ausführung der
Methode eine deutsche Mannschaft existiert mit zwei Spielern Namens
\java{"Hugo Meier"} und \java{"Frank Huber"}. Beide Spieler sollen zum
selben Verein \java{"FC Staatsexamen"} gehören. \java{"Hugo Meier"} soll
nach Aufnahme in die deutsche Mannschaft genau ein Tor geschossen haben,
während \java{"Frank Huber"} noch kein Tor erzielt hat. (Wir
abstrahieren hier von der Realität, in der ein 2-er Team noch gar nicht
spielbereit ist.)

\begin{antwort}
\inputcode[firstline=3]{aufgaben/oomup/pu_4/verein/Verein}
\end{antwort}
\end{enumerate}

\end{document}
