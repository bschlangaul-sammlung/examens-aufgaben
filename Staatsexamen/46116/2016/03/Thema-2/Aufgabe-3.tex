\documentclass{lehramt-informatik-aufgabe}
\liLadePakete{}
\begin{document}
\liAufgabenTitel{Bruchsicherheit von Smartphones}
\section{Aufgabe 3
\index{Komplexität}
\footcite{46116:2016:03}}

Sie sollen mithilfe von Falltests eine neue Serie von Smartphones auf
Bruchsicherheit testen.

Dazu wird eine Leiter mit $n$ Sprossen verwendet; die höchste Sprosse,
von der ein Smartphone heruntergeworfen werden kann ohne zu zerbrechen,
heiße „höchste sichere Sprosse“. Das Ziel ist, die höchste sichere
Sprosse zu ermitteln. Man kann davon ausgehen, dass die höchste sichere
Sprosse nicht von der Art des Wurfs abhängt und dass alle verwendeten
Smartphones sich gleich verhalten. Eine Möglichkeit, die höchste sichere
Sprosse zu ermitteln, besteht darin, ein Gerät erst von Sprosse $1$,
dann von Sprosse $2$, etc. abzuwerfen, bis es schließlich beim Wurf von
Sprosse $k$ beschädigt wird (oder Sie oben angelangt sind). Sprosse $k -
1$ (bzw. $n$) ist dann die höchste sichere Sprosse. Bei diesem Verfahren
wird maximal ein Smartphone zerstört, aber der Zeitaufwand ist
ungünstig.

\begin{enumerate}
%%
% a)
%%

\item Bestimmen Sie die Zahl der Würfe bei diesem Verfahren im
schlechtesten Fall.
%%
% b)
%%

\item Geben Sie nun ein Verfahren zur Ermittlung der höchsten sicheren
Sprosse an, welches nur $\mathcal{O} (\log n)$ Würfe benötigt, dafür
aber möglicherweise mehr Smartphones verbraucht.

%%
% c)
%%

\item Es gibt eine Strategie zur Ermittlung der höchsten sicheren
Sprosse mit $\mathcal{O}(\sqrt{n})$ Würfen, bei dessen Anwendung
höchstens zwei Smartphones kaputtgehen. Finden Sie diese Strategie und
begründen Sie Ihre Funktionsweise und Wurfzahl.

Tipp: der erste Testwurf erfolgt von Sprosse [ vn | .

\end{enumerate}
\end{document}
