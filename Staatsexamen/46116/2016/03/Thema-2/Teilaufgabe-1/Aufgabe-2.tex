\documentclass{lehramt-informatik-aufgabe}
\liLadePakete{petri}
\begin{document}

\liSetzeAufgabenTitel{
  Titel = Aufgabe 2,
  Thematik = Petri-Netz,
  Fussnote = sosy:pu:4,
  ExamenNummer = 46116,
  ExamenJahr = 2016,
  ExamenMonat = 03,
  ExamenThemaNr = 2,
  ExamenTeilaufgabeNr = 1,
  ExamenAufgabeNr = 2,
}

\index{Petri-Netz}

Gegeben sei das folgende Petri-Netz:\footcite[Seite 9]{examen:46116:2016:03}

\begin{center}
\begin{tikzpicture}[li petri,x=2cm,y=2cm]
  \node[place,label=A,label=south east:1,tokens=1] at (1,2) (A) {};
  \node[place,label=south:B] at (0,0) (B) {};
  \node[place,label=south:C,tokens=1] at (2,0) (C) {};

  \node[transition] at (1,0) {$t_1$}
    edge[pre] node[auto]{2} (B)
    edge[post] (C);

  \node[transition] at (0,2) {$t_2$}
    edge[pre] (A)
    edge[post] node[auto]{2} (B);

  \node[transition] at (2,2) {$t_3$}
    edge[pre] (C)
    edge[post] (A)
    edge[inhibitor,bend right=50] (A);

  \node[transition] at (1,1) {$t_4$}
    edge[pre] (A)
    edge[pre] node[auto]{2} (B)
    edge[post] (C);
\end{tikzpicture}
\end{center}

\begin{enumerate}

%%
% a)
%%

\item Erstellen Sie den zum Petri-Netz gehörenden
Erreichbarkeitsgraphen\index{Erreichbarkeitsgraph}. Die Belegungen sind
jeweils in der Form [A, B, C] anzugeben. Beschriften Sie auch jede Kante
mit der zugehörigen Transition. Beachten Sie die auf 1 beschränkte
Kapazität von Stelle A oder alternativ die Inhibitor-Kante von A zu
$t_3$ (beides ist hier semantisch äquivalent).

\def\s#1#2#3{(#1,#2,#3)}
\def\t#1{$\rightarrow \hspace{0.4cm} t_#1 \rightarrow$}

\begin{center}
\begin{tabular}{lll}
\s 1 0 1 & \t2 & \s 0 2 1 \\
\s 0 2 1 & \t1 & \s 0 0 2 \\
\s 0 2 1 & \t3 & \s 1 2 0 \\
\s 1 2 0 & \t4 & \s 0 0 1 \\
\s 0 0 1 & \t3 & \s 1 0 0 \\
\s 1 0 0 & \t2 & \s 0 2 0 \\
\s 0 2 0 & \t1 & \s 0 0 1 \\
\s 0 0 2 & \t3 & \s 1 0 1 \\
\end{tabular}
\end{center}

%%
% b)
%%

\item Wie kann man mit Hilfe des Erreichbarkeitsgraphen feststellen, ob
ein Petri-Netz lebendig ist?

%%
% c)
%%

\item Aufgrund von Transition $t_4$ ist das gegebene Petri-Netz nicht
stark lebendig. Wie müssten die Pfeilgewichte der Transition $t_4$
verändert werden, damit das Petri-Netz mit der gegebenen Startmarkierung
beschränkt bleibt und lebendig wird?

\begin{liAntwort}
$t_4$ nach C mit Gewicht 2 versehen
\end{liAntwort}

\end{enumerate}

\end{document}
