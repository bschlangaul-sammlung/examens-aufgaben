\documentclass{lehramt-informatik-minimal}
\InformatikPakete{petri}
\begin{document}

\section{Aufgabe 2: „Petri-Netze“\footcite{sosy:pu:4}}

Gegeben sei das folgende Petri-Netz:\footcite[Seite 9]{examen:46116:2016:03}

\begin{tikzpicture}[x=2cm,y=2cm]
  \node[place,label=A,label=south east:1,tokens=1] at (1,2) (A) {};
  \node[place,label=south:B] at (0,0) (B) {};
  \node[place,label=south:C,tokens=1] at (2,0) (C) {};

  \node[transition] at (1,0) {$t_1$}
    edge[pre] node[auto]{2} (B)
    edge[post] (C);

  \node[transition] at (0,2) {$t_2$}
    edge[pre] (A)
    edge[post] node[auto]{2} (B);

  \node[transition] at (2,2) {$t_3$}
    edge[pre] (C)
    edge[post] (A);

  \node[transition] at (1,1) {$t_4$}
    edge[pre] (A)
    edge[pre] node[auto]{2} (B)
    edge[post] (C);

\end{tikzpicture}

\begin{enumerate}

%%
% a)
%%

\item Erstellen Sie den zum Petri-Netz gehörenden
Erreichbarkeitsgraphen. Die Belegungen sind jeweils in der Form [A, B,
C] anzugeben. Beschriften Sie auch jede Kante mit der zugehörigen
Transition. Beachten Sie die auf 1 beschränkte Kapazität von Stelle A
oder alternativ die Inhibitor-Kante von A zu t3 (beides ist hier
semantisch äquivalent).

%%
% b)
%%

\item Wie kann man mit Hilfe des Erreichbarkeitsgraphen feststellen, ob
ein Petri-Netz lebendig ist?

%%
% c)
%%

\item Aufgrund von Transition t4 ist das gegebene Petri-Netz nicht stark
lebendig. Wie müssten die Pfeilgewichte der Transition t4 verändert
werden, damit das Petri-Netz mit der gegebenen Startmarkierung
beschränkt bleibt und lebendig wird?

\end{enumerate}

\end{document}
