\documentclass{lehramt-informatik-aufgabe}
\liLadePakete{}
\begin{document}

\section{Aufgabe 1: Transaktionen, Schedules
\index{Transaktionen}
\footcite{db:pu:5}}

\begin{enumerate}

\item Nennen Sie die vier wesentlichen Eigenschaften\index{ACID} einer
Transaktion und erläutern Sie jede Eigenschaft kurz (ein Satz pro
Eigenschaft).
\footcite[Thema 1 Teilaufgabe 1 Aufgabe 5]{examen:46116:2016:03}

\begin{antwort}
ACID
\end{antwort}

\item Gegeben ist die folgende Historie (Schedule) dreier Transaktionen:

$
r_1 (B) \rightarrow
w_1 (C) \rightarrow
r_3 (C) \rightarrow
r_1 (A) \rightarrow
c_1 \rightarrow
r_2 (C) \rightarrow
r_3 (C) \rightarrow
r_2 (C) \rightarrow
w_2 (B) \rightarrow
c_2 \rightarrow
c_3
$

\begin{antwort}
\begin{tabular}{lll}
T1 & T2 & T3 \\
$r_1 (B)$ &           &           \\
$w_1 (C)$ &           &           \\
          &           & $r_3 (C)$ \\
$r_1 (A)$ &           &           \\
$c_1$     &           &           \\
          & $r_2 (C)$ &           \\
          &           & $r_3 (C)$ \\
          & $r_2 (C)$ &           \\
          & $w_2 (B)$ &           \\
          & $c_2$     &           \\
          &           & $c_3$     \\
\end{tabular}

$w_1 (C) < r_3 (C)$ Konfliktoperation

Kante von T1 nach T3

Keine weitere Konfliktoperation

Serialisierbarkeitsgraphen

Kein Zyklus im Graph -> serialisierbar

Falls ein Zyklus dann wäre es nicht serialisierbar
\end{antwort}

\item Zeichnen Sie den
Serialisierbarkeitsgraphen\index{Serialisierbarkeitsgraph} zu dieser
Historie* und begründen Sie, warum die Historie serialisierbar ist oder
nicht.

*In Transaktionssystemen existiert ein Ausführungsplan für die parallele
Ausführung mehrerer Transaktionen. Der Plan wird auch Historie genannt
und gibt an, in welcher Reihenfolge die einzelnen Operationen der
Transaktion ausgeführt werden. Als serialisierbar bezeichnet man eine
Historie, wenn sie zum selben Ergebnis führt wie eine nacheinander
(seriell) ausgeführte Historie über dieselben Transaktionen.

\item Geben Sie an, wodurch die erste und die zweite Phase des
Zwei-Phasen-Sperrprotokolls\index{Zwei-Phasen-Sperrprotokoll} jeweils
charakterisiert sind (ein Satz pro Phase).

\begin{antwort}
Siehe Folie Seite 16 Zwei-Phasen-Sperrprotokoll
\end{antwort}

\end{enumerate}
\end{document}
