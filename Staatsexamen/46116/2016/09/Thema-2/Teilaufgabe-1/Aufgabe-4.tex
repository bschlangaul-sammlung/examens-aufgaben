\documentclass{lehramt-informatik-aufgabe}
\liLadePakete{syntax}
\begin{document}
\liAufgabenTitel{Formale Verifikation mit vollständiger Induktion}
\section{Aufgabe 4:
\index{Formale Verifikation}
\footcite{46116:2016:09}}

Gegeben sei folgende rekursive Methodendeklaration in der Sprache Java.
Es wird als Vorbedingung vorausgesetzt, dass die Methode \liJavaCode{cn}
nur für Werte $n \geq 0$ aufgerufen wird.

\begin{minted}{java}
int cn(int n) {
  if(n == 0)
    return 1;
  else
    return (4 * (n - 1) + 2) * cn(n - 1) / (n + 1);
}
\end{minted}

\noindent
Sie können im Folgenden vereinfachend annehmen, dass es keinen Überlauf
in der Berechnung gibt, \dh dass der Datentyp \liJavaCode{int} für die
Berechnung des Ergebnisses stets ausreicht.

\begin{enumerate}

%%
% a)
%%

\item Beweisen Sie mittels vollständiger Induktion, dass der
Methodenaufruf cn(n) für jedes n > 0 die n-te Catalan-Zahl $C_n$
berechnet, wobei

\begin{displaymath}
C_n =
\frac{(2n)!}
{(n + 1)!-n!}
\end{displaymath}

Beim Induktionsschritt können Sie die beiden folgenden Gleichungen
verwenden:

\begin{enumerate}

\item $(2(n + 1))! = (4n + 2) \cdot (n + 1) \cdot (2n)!$

\item $(a + 2)! \cdot (n+1)! = (n + 2) \cdot (n + 1) \cdot (n + 1)! \cdot n!$
\end{enumerate}

%%
% b)
%%

\item Geben Sie eine geeignete Terminierungsfunktion an und begründen
Sie, warum der Methodenaufruf \liJavaCode{cn(n)} für jedes $n \geq 0$
terminiert.

\end{enumerate}
\end{document}
