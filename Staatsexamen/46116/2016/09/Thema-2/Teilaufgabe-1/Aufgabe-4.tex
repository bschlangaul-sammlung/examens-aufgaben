\documentclass{lehramt-informatik-aufgabe}
\liLadePakete{vollstaendige-induktion}
\begin{document}
\liAufgabenTitel{Formale Verifikation mit vollständiger Induktion}
\section{Aufgabe 4:
\index{Vollständige Induktion}
\footcite{46116:2016:09}}

Gegeben sei folgende rekursive Methodendeklaration in der Sprache Java.
Es wird als Vorbedingung vorausgesetzt, dass die Methode \liJavaCode{cn}
nur für Werte $n \geq 0$ aufgerufen wird.

\begin{minted}{java}
int cn(int n) {
  if (n == 0)
    return 1;
  else
    return (4 * (n - 1) + 2) * cn(n - 1) / (n + 1);
}
\end{minted}

\noindent
Sie können im Folgenden vereinfachend annehmen, dass es keinen Überlauf
in der Berechnung gibt, \dh dass der Datentyp \liJavaCode{int} für die
Berechnung des Ergebnisses stets ausreicht.

\begin{enumerate}

%%
% a)
%%

\item Beweisen Sie mittels vollständiger Induktion, dass der
Methodenaufruf \liJavaCode{cn(n)} für jedes $n \geq 0$ die $n$-te
Catalan-Zahl $C_n$ berechnet, wobei

\begin{displaymath}
C_n =
\frac{(2n)!}
{(n + 1)! \cdot n!}
\end{displaymath}

\begin{liExkurs}[Fakultät]

Für alle natürlichen Zahlen  $n$ ist

\begin{displaymath}
n!=1\cdot 2\cdot 3\dotsm n=\prod _{k=1}^{n}k
\end{displaymath}

als das Produkt der natürlichen Zahlen von 1 bis $n$ definiert. Da das
leere Produkt stets 1 ist, gilt

\begin{displaymath}
0! = 1
\end{displaymath}

Die Fakultät lässt sich auch rekursiv definieren:

\begin{displaymath}
n!={
\begin{cases}1, & n = 0\\
n \cdot (n - 1)!, & n > 0
\end{cases}
}
\end{displaymath}

Fakultäten für negative oder nicht ganze Zahlen sind nicht definiert. Es
gibt aber eine Erweiterung der Fakultät auf solche Argumente
\liFussnoteUrl{https://de.wikipedia.org/wiki/Fakultät_(Mathematik)}
\end{liExkurs}

\begin{liExkurs}[Catalan-Zahl]
Die Catalan-Zahlen bilden eine Folge natürlicher Zahlen, die in vielen
Problemen der Kombinatorik auftritt. Sie sind nach dem belgischen
Mathematiker Eugène Charles Catalan benannt.

Die Folge der Catalan-Zahlen $C_{0},C_{1},C_{2},C_{3},\dotsc$ beginnt
mit $1, 1, 2, 5, 14, 42, 132,\dotsc$
\liFussnoteUrl{https://de.wikipedia.org/wiki/Catalan-Zahl}
\end{liExkurs}

Beim Induktionsschritt können Sie die beiden folgenden Gleichungen
verwenden:

\begin{enumerate}

\item $(2(n + 1))! = (4n + 2) \cdot (n + 1) \cdot (2n)!$

\item $(a + 2)! \cdot (n+1)! = (n + 2) \cdot (n + 1) \cdot (n + 1)! \cdot n!$
\end{enumerate}

\begin{liAntwort}
%%
%
%%

\liInduktionAnfang

\begin{align*}
C_0
& = \text{cn}(0)\\
& = \frac{(2 \cdot 0)!}{(0 + 1)! \cdot 0!}\\
& = \frac{0!}{1! \cdot 0!}\\
& = \frac{1}{1 \cdot 1}\\
& = \frac{1}{1}\\
& = 1\\
\end{align*}

%%
%
%%

\liInduktionVoraussetzung

\begin{align*}
C_n
& = \text{cn}(n)\\
& = \frac{(2n)!}{(n + 1)! \cdot n!}
\end{align*}

%%
%
%%

\liInduktionSchritt

% erklaerung
\def\e#1{\scriptsize#1}
% erklaerung gleich mit text
\def\et#1{\scriptsize\text{#1}}

\def\r#1{\textcolor{blue}{#1}}

\begin{align*}
C_{n+1}
& = \text{cn}(n + 1)\\
& = \frac
  {(4 \cdot (\r{n + 1} - 1) + 2) \cdot \text{cn}(\r{n + 1} - 1)}
  {\r{n + 1} + 1}
  & \et{Java nach Mathe}\\
%
& = \frac
  {(4 \cdot \r{n + 2}) \cdot \text{cn}(\r{n})}
  {\r{n + 2}}
  & \et{Addition, Subtraktion}\\
%
& = \frac
  {(4 \cdot n + 2) \cdot \r{(2n)!}}
  {(n + 2) \cdot \r{(n + 1)! \cdot n!}}
  & \et{für cn(n) Formel einsetzen}\\
%
& = \frac
  {(4 \cdot n + 2) \cdot (2n)! \r{\cdot (n + 1)}}
  {(n + 2) \cdot (n + 1)! \cdot n! \r{\cdot (n + 1)}}
  & \e{\text{oben und unten } \cdot (n + 1)} \\
%
& = \frac
  {(4 \cdot n + 2) \cdot \r{(n + 1) \cdot (2n)!}}
  {(n + 2) \cdot (n + 1)! \cdot \r{(n + 1) \cdot n!}} & \et{umsortieren} \\
%
& = \frac
  {\r{(2(n + 1))!}}
  {\r{(n + 2)! \cdot (n + 1)!}} & \et{Hilfsgleichungen verwenden}\\
%
& = \frac
  {(2(\r{n + 1}))!}
  {((\r{n + 1}) + 1)! \cdot (\r{n + 1})!} & \e{(n + 1) \text{deutlicher machen}}\\
\end{align*}

\end{liAntwort}

%%
% b)
%%

\item Geben Sie eine geeignete Terminierungsfunktion an und begründen
Sie, warum der Methodenaufruf \liJavaCode{cn(n)} für jedes $n \geq 0$
terminiert.

\begin{liAntwort}
$T(n) = n$. Diese Funktion verringert sich bei jedem Rekursionsschritt
um eins. Sie ist monoton fallend für die natürlichen Zahlen und für
$T(0) = 0$ definiert. Damit ist sie eine Terminierungsfunktion für
cn(n).
\end{liAntwort}

\end{enumerate}
\end{document}
