\documentclass{bschlangaul-aufgabe}
\liLadePakete{syntax}
\begin{document}
\liAufgabenTitel{CreditCard, Order}
\section{Aufgabe 1
\index{Klassendiagramm}
\footcite{examen:46116:2013:03}}

Gegeben sei folgendes Klassendiagramm:

\begin{enumerate}

%%
% a)
%%

\item Implementieren Sie die Klassen „Order“, „Payment“, „Check“,
„OrderDetail“ und „Item“ in einer geeigneten objektorientierten Sprache
Ihrer Wahl. Beachten Sie dabei insbesondere Sichtbarkeiten, Klassen- vs.
Instanzzugehörigkeiten und Schnittstellen bzw. abstrakte
Klassen.

\begin{liAntwort}
\liJavaExamen{46116}{2013}{03}{t2_ta1_a1/Check.java}
\liJavaExamen{46116}{2013}{03}{t2_ta1_a1/Item.java}
\liJavaExamen{46116}{2013}{03}{t2_ta1_a1/Order.java}
\liJavaExamen{46116}{2013}{03}{t2_ta1_a1/OrderDetail.java}
\liJavaExamen{46116}{2013}{03}{t2_ta1_a1/Payment.java}
\end{liAntwort}

%%
% b)
%%

\item Erstellen Sie ein Sequenzdiagramm (mit konkreten Methodenaufrufen und
Nummerierung der Nachrichten) für folgendes Szenario:

\begin{enumerate}

%%
% i.
%%

\item „Erika Mustermann“ aus „Rathausstraße 1, 10178 Berlin“ wird als
neue Kundin angelegt.

%%
% ii,
%%

\item Frau Mustermann bestellt heute 1 kg Gurken und 2 kg Kartoffeln.

%%
% ii.
%%

\item Sie bezahlt mit ihrer Visa-Karte, die im August 2014 abläuft und
die Nummer „1234 567891 23456“ hat — die Karte erweist sich bei der
Prüfung als gültig.

%%
% iv.
%%

\item Am Schluss möchte sie noch wissen, wie viel ihre Bestellung kostet
— dabei wird der Anteil der Mehrwertsteuer extra ausgewiesen.

\end{enumerate}

\end{enumerate}

\end{document}
