\documentclass{lehramt-informatik-aufgabe}
\liLadePakete{relationale-algebra}
\begin{document}
\liAufgabenTitel{Harry Potter}
\section{Aufgabe 4
\index{Relationale Algebra}
\footcite{46116:2019:09}}

Gegeben ist das Datenbankenschema aus Aufgabe 3.

\bigskip

\noindent
Übertragen Sie die folgenden Ausdrücke in die relationale Algebra.
Beschreiben Sie diese Ausdrücke umgangssprachlich, bevor Sie die
Ausdrücke umformen.

\begin{enumerate}

%%
% a)
%%

\item $\{ s | s \in \text{Schüler} \land \neg \exists t \in \text{teil\_von} (t.\text{Id} = s.\text{Id}) \}$
\index{Tupelkalkül}

\begin{liAntwort}
$\text{Schüler} - (\text{Schüler} \bowtie \pi_{Id}(\text{teil\_von}))$
\end{liAntwort}

%%
% b)
%%

\item $\{ s |
s \in \text{Schüler} \land
\exists t \in \text{teil\_von} (r.\text{Id} = s.\text{Id}) \land
\exists h \in \text{Haus} (f.\text{Name} = h.\text{Name}) \land
\exists q \in \text{Quidditch} (h.\text{Name} = q.\text{Haus} \land q.\text{Captain} = \text{'Harry Potter'}) \}$

\begin{liAntwort}
\begin{multline*}
\sigma_{\text{Id}, \text{SName}, \text{Patronus}, \text{Haarfarbe}, \text{Aktiv}, \text{Gesamtnote}} \biggl(\\
  \Bigl(
  \rho_{\text{Sname} \leftarrow \text{Name}}(\text{Schüler})
  \bowtie
  \text{teil\_von}
  \Bigr)\\
  \bowtie\\
  \Bigl(
    \text{Haus}
    \bowtie_{\text{Haus.Name} = \text{Quidditch.Haus}}
    \bigl(
      \sigma_{\text{Captain} = \text{'Harry Potter'}} (\text{Quidditch})
    \bigr)
  \Bigr)\\
\biggr)
\end{multline*}
\end{liAntwort}

\end{enumerate}

\end{document}
