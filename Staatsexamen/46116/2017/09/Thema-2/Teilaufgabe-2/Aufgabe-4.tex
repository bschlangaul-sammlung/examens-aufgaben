\documentclass{lehramt-informatik-aufgabe}
\liLadePakete{syntax}
\begin{document}
\liAufgabenTitel{Turmspringen}

\section{Aufgabe 4
\index{SQL}
\footcite[Thema 2 Teilaufgabe 2 Aufgabe 4]{examen:46116:2017:09}}

\noindent
Für die bayerische Meisterschaft im Turmspringen ist folgendes
Datenbankschema angelegt:
\footcite[SQL-Anfragen auf mehreren Tabellen]{db:ab:2}

\begin{minted}{md}
Springer (Startnummer, Nachname, Vorname, Geburtsdatum, Körpergröße)
Sprung (SID, Beschreibung, Schwierigkeit)
springt (SID, Startnummer, Durchgang)
FK (SID) referenziert Sprung (SID)
FK (Startnummer) referenziert Springer (Startnummer)
\end{minted}

\noindent
Das Attribut Schwierigkeit kann die Werte 1 bis 10 annehmen, das
Attribut Durchgang ist positiv und ganzzahlig. Die Körpergröße der
Springer ist in Zentimeter angegeben.

\begin{enumerate}

%%
% (a)
%%

\item Welche Springer sind größer als 1,80 m? Schreiben Sie eine
SQL-Anweisung, welche in der Ausgabe mit dem größten Springer beginnt.

\begin{antwort}
\begin{minted}{sql}
SELECT Vorname, Nachname
FROM Springer
WHERE Koerpergroesse > 180
ORDER BY Koerpergroesse DESC;
\end{minted}
\end{antwort}

%%
% (b)
%%

\item Welche Springer haben im ersten Durchgang einen Sprung mit einer
Schwierigkeit von unter 6 gezeigt? Schreiben Sie eine SQL-Anweisung,
welche Startnummer und Nachname dieser Springer ausgibt.

\begin{antwort}
\begin{minted}{sql}
SELECT Springer.Startnummer, Springer.Nachname
FROM Springer, Sprung, springt
WHERE
  Sprung.SID = springt.SID AND
  Springer.Startnummer = springt.Startnummer AND
  springt.Durchgang = 1
  Sprung.Schwierigkeit < 6;
\end{minted}
\end{antwort}

%%
% (c)
%%

\item Formulieren Sie in Umgangssprache, aber trotzdem möglichst
präzise, wonach mit folgender Abfrage gesucht wird:\index{GROUP BY}

\begin{minted}{sql}
SELECT springt.Startnummer, s.Nachname, s.Vorname,
MAX (springt.Durchgang)
FROM springt, Springer s
WHERE springt.Startnummer = s.Startnummer
GROUP BY springt.Startnummer, s.Nachname, s.Vorname
\end{minted}

\begin{antwort}[richtig]
Die Abfrage gibt die Startnummer, den Nachnamen, den Vornamen und
die Anzahl der Sprünge, d. h. die Anzahl der Durchgänge der
einzelnen Springer an.
\end{antwort}

\begin{antwort}
Die Abfrage liefert die Startnummer, Vorname und Nachname sowie die
maximale Anzahl an Durchgängen jedes Springers.
\end{antwort}

\item Gesucht ist die „durchschnittliche Körpergröße“ all der Springer,
die vor dem 01.01.2000 geboren wurden. Formulieren Sie eine
[entsprechende] SQL-Anweisung, [...] wobei die Spalte mit der
durchschnittlichen Körpergröße genau diesen Namen „durchschnittliche
Körpergröße“ haben soll.

\begin{antwort}
Umlaute und Leerzeichen sind bei Spaltenbeschriftungen nicht erlaubt.

\begin{minted}{sql}
SELECT AVG(Koerpergroesse) AS durchschnittliche_Koerpergroesse
FROM SPRINGER
WHERE Geburtsdatum < DATE('2000-01-01');
\end{minted}
\end{antwort}

\begin{antwort}
\begin{minted}{sql}
SELECT AVG(Koerpergroesse) AS durchschnittliche_Koerpergroesse
FROM SPRINGER
WHERE Geburtsdatum < '01.01.2001';
\end{minted}
\end{antwort}

\end{enumerate}
\end{document}
