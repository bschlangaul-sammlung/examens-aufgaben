\documentclass{lehramt-informatik-aufgabe}
\liLadePakete{}
\begin{document}
\liAufgabenTitel{Tupelidentifikator}

\section{Aufgabe 3: Physische Datenorganisation, TID
\index{Physische Datenorganisation}
\footcite[Thema 1 Teilaufgabe 2 Aufgabe 4]{46116:2017:03}
}

\begin{enumerate}
\item Erläutern Sie in ein bis zwei Sätzen, aus welchen zwei Teilen
sich ein TID (Tupelidentifikator) zusammensetzt.

\begin{liAntwort}
Seitennummer (Seiten bzw. Blöcke sind größere Speichereinheiten auf der
Platte) Relative Indexposition innerhalb der Seite
\end{liAntwort}

\item Erläutern Sie in ein bis zwei Sätzen das Vorgehen, wenn ein
durch einen TID adressierter Satz innerhalb einer Seite verschoben
werden muss.

\begin{liAntwort}
Satzverschiebung innerhalb einer Seite bleibt ohne Auswirkungen auf TID,
\end{liAntwort}

\item Erläutern Sie in ein bis zwei Sätzen das Vorgehen, wenn ein
durch einen TID adressierter Satz erstmalig in eine andere Seite
verschoben werden muss.

\begin{liAntwort}
wird ein Satz auf eine andere Seite migriert, wird eine
„Stellvertreter-TID“ zum Verweis auf den neuen Speicherort verwendet.
Die eigentliche TID-Adresse bleibt stabil
\end{liAntwort}

\item Erläutern Sie in zwei bis drei Sätzen das Vorgehen, wenn ein
durch einen TID adressierter und bereits einmal über Seitengrenzen
hinweg verschobener Satz erneut in eine andere Seite verschoben werden
muss.\footcite{db:pu:3}

\begin{liAntwort}
Es wird eine neue stellvertreter TID aktualisiert.
\end{liAntwort}
\end{enumerate}

\end{document}
