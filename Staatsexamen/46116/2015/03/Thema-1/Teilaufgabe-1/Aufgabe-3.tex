\documentclass{lehramt-informatik-aufgabe}
\liLadePakete{cpm}
\begin{document}
\liAufgabenTitel{Gantt und CPM}

\section{3. Projektmanagement
\index{CPM-Netzplantechnik}
\footcite[Thema 1 Teilaufgabe 1 Aufgabe 3]{examen:46116:2015:03}}

Betrachten Sie folgendes CPM-Netzwerk:\footcite{sosy:pu:4}

\begin{center}
\begin{tikzpicture}
\liCpmEreignis{A}(0,4)
\liCpmEreignis{B}(2,4)
\liCpmEreignis{C}(0,2)
\liCpmEreignis{D}(2,2)
\liCpmEreignis{E}(4,3)
\liCpmEreignis{F}(6,3)
\liCpmEreignis{G}(8,3)
\liCpmEreignis{H}(8,1)
\liCpmEreignis{I}(10,1)
\liCpmEreignis{J}(10,3)

\liCpmVorgang(A>B){4}
\liCpmVorgang(C>D){1}
\liCpmVorgang(E>F){5}
\liCpmVorgang(G>J){8}
\liCpmVorgang(H>I){9}
\liCpmVorgang(J>I){2}

\liCpmScheinvorgang(A>C){}
\liCpmScheinvorgang(B>E){}
\liCpmScheinvorgang(D>E){}
\liCpmScheinvorgang(F>G){}
\liCpmScheinvorgang(G>H){}
\end{tikzpicture}
\end{center}
\begin{enumerate}

%%
% a)
%%

\item Berechnen Sie die früheste Zeit für jedes Ereignis, wobei
angenommen wird, dass das Projekt zum Zeitpunkt 0 startet.

%%
% b)
%%

\item Setzen Sie anschließend beim letzten Ereignis die späteste Zeit
gleich der frühesten Zeit und berechnen Sie die spätesten Zeiten.

%%
% c)
%%

\item Berechnen Sie nun für jedes Ereignis die Pufferzeiten.

%%
% d)
%%

\item Bestimmen Sie den kritischen Pfad.

%%
% e)
%%

\item Was ist ein Gantt-Diagramm\index{Gantt-Diagramm}? Worin
unterscheidet es sich vom CPM-Netzwerk?

\end{enumerate}
\end{document}
