\documentclass{lehramt-informatik-minimal}
\InformatikPakete{cpm}
\begin{document}

\section{3. Projektmanagement\footcite{sosy:pu:4}}

Betrachten Sie folgendes CPM-Netzwerk:\footcite{examen:46116:2015:03}

\begin{center}
\begin{tikzpicture}
\ereignis{A}(0,4)
\ereignis{B}(2,4)
\ereignis{C}(0,2)
\ereignis{D}(2,2)
\ereignis{E}(4,3)
\ereignis{F}(6,3)
\ereignis{G}(8,3)
\ereignis{H}(8,1)
\ereignis{I}(10,1)
\ereignis{J}(10,3)

\vorgang(A>B){4}
\vorgang(C>D){1}
\vorgang(E>F){5}
\vorgang(G>J){8}
\vorgang(H>I){9}
\vorgang(J>I){2}

\scheinvorgang(A>C){}
\scheinvorgang(B>E){}
\scheinvorgang(D>E){}
\scheinvorgang(F>G){}
\scheinvorgang(G>H){}
\end{tikzpicture}
\end{center}
\begin{enumerate}

%%
% a)
%%

\item Berechnen Sie die früheste Zeit für jedes Ereignis, wobei
angenommen wird, dass das Projekt zum Zeitpunkt 0 startet.

%%
% b)
%%

\item Setzen Sie anschließend beim letzten Ereignis die späteste Zeit
gleich der frühesten Zeit und berechnen Sie die spätesten Zeiten.

%%
% c)
%%

\item Berechnen Sie nun für jedes Ereignis die Pufferzeiten.

%%
% d)
%%

\item Bestimmen Sie den kritischen Pfad.

%%
% e)
%%

\item Was ist ein Gantt-Diagramm? Worin unterscheidet es sich vom
CPM-Netzwerk?

\end{enumerate}
\end{document}
