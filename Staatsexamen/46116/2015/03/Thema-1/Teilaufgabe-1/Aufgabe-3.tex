\documentclass{bschlangaul-aufgabe}
\liLadePakete{cpm}
\begin{document}
\let\f=\footnotesize
\let\FZ=\liCpmFruehI
\let\SZ=\liCpmSpaetI
\let\v=\liCpmVon
\let\vz=\liCpmVonZu
\let\z=\liCpmZu

\liAufgabenTitel{Gantt und CPM}

\section{3. Projektmanagement
\index{CPM-Netzplantechnik}
\footcite[Thema 1 Teilaufgabe 1 Aufgabe 3]{examen:46116:2015:03}}

Betrachten Sie folgendes CPM-Netzwerk:\footcite{sosy:pu:4}

\begin{center}
\begin{tikzpicture}
\liCpmEreignis{A}{0}{4}
\liCpmEreignis{B}{2}{4}
\liCpmEreignis{C}{0}{2}
\liCpmEreignis{D}{2}{2}
\liCpmEreignis{E}{4}{3}
\liCpmEreignis{F}{6}{3}
\liCpmEreignis{G}{8}{3}
\liCpmEreignis{H}{8}{1}
\liCpmEreignis{I}{10}{1}
\liCpmEreignis{J}{10}{3}

\liCpmVorgang{A}{B}{4}
\liCpmVorgang{C}{D}{1}
\liCpmVorgang{E}{F}{5}
\liCpmVorgang{G}{J}{8}
\liCpmVorgang{H}{I}{9}
\liCpmVorgang{J}{I}{2}

\liCpmVorgang[schein]{A}{C}{}
\liCpmVorgang[schein]{B}{E}{}
\liCpmVorgang[schein]{D}{E}{}
\liCpmVorgang[schein]{F}{G}{}
\liCpmVorgang[schein]{G}{H}{}
\end{tikzpicture}
\end{center}
\begin{enumerate}

%%
% a)
%%

\item Berechnen Sie die früheste Zeit für jedes Ereignis, wobei
angenommen wird, dass das Projekt zum Zeitpunkt 0 startet.

\begin{liAntwort}
\begin{tabular}{|l|r|r|}
\hline
$i$ & Nebenrechnung & \FZ \\\hline\hline
A & & 0 \\\hline
B & & 4 \\\hline
C & & 0 \\\hline
D & & 1 \\\hline
E & $\max(4, 1)$ & 4 \\\hline
F & & 9 \\\hline
G & & 9 \\\hline
H & & 9 \\\hline
J & & 17 \\\hline
I & $\max(\v{9}(H) + 9, \v{17}(J) + 2)$ & 19 \\\hline
\end{tabular}
\end{liAntwort}

%%
% b)
%%

\item Setzen Sie anschließend beim letzten Ereignis die späteste Zeit
gleich der frühesten Zeit und berechnen Sie die spätesten Zeiten.

\begin{liAntwort}
\begin{tabular}{|l|r|r|}
\hline
$i$ & Nebenrechnung & \SZ \\\hline\hline
A & $\min(3, 0)$  & 0  \\\hline
B &               & 4  \\\hline
C &               & 3  \\\hline
D &               & 4  \\\hline
E &               & 4  \\\hline
F &               & 9  \\\hline
G & $\min(10, 9)$ & 9  \\\hline
H &               & 10 \\\hline
J &               & 17 \\\hline
I &               & 19 \\\hline
\end{tabular}
\end{liAntwort}

%%
% c)
%%

\item Berechnen Sie nun für jedes Ereignis die Pufferzeiten.

\begin{liAntwort}
\begin{tabular}{|l||l|l|l|l|l|l|l|l|l|l|}
\hline
$i$ & A & B & C & D & E & F & G & H  & J  & I  \\\hline\hline
\FZ & 0 & 4 & 0 & 1 & 4 & 9 & 9 & 9  & 17 & 19 \\\hline
\SZ & 0 & 4 & 3 & 4 & 4 & 9 & 9 & 10 & 17 & 19 \\\hline
GP  & 0 & 0 & 3 & 3 & 0 & 0 & 0 & 1  & 0  & 0 \\\hline
\end{tabular}
\end{liAntwort}

%%
% d)
%%

\item Bestimmen Sie den kritischen Pfad.

\begin{liAntwort}
$A \rightarrow
B \rightarrow
E \rightarrow
F \rightarrow
G \rightarrow
J \rightarrow I$
\end{liAntwort}

%%
% e)
%%

\item Was ist ein Gantt-Diagramm\index{Gantt-Diagramm}? Worin
unterscheidet es sich vom CPM-Netzwerk?

\begin{liAntwort}

\end{liAntwort}

\end{enumerate}
\end{document}
