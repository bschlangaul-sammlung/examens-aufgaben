\documentclass{lehramt-informatik-minimal}
\InformatikPakete{er}
\begin{document}

\section{Aufgabe 5: ER-Modell und Relationenmodell
(Check-Up)\footcite{db:ab:2}}

Es sind folgende Informationen zu einer Datenbank für Konsulate
gegeben:\footcite[Staatsexamen Softwaretechnologie/Datenbanksysteme,
Thema Nr. 1, Teilaufgabe II, Aufgabe 3, Frühjahr 2015
Realschule]{examen:46116:2015:03}

\begin{itemize}

\item Jedes Konsulat hat einen Sitz in einer Stadt

\item Zu einem \mpEntity{Konsulat} soll ein eindeutiger
\mpAttribute{Name} \dname{KonsulatName} (z.\,B. Konsulat Bayern), die
\mpAttribute{Adresse} und der \mpAttribute{Vor-} \dname{KVorname} bzw.
\mpAttribute{Nachname} \dname{KNachname} des Konsuls gespeichert werden.

\item Für jede \mpEntity{Stadt} sollen der \mpAttribute{Name}
\dname{StadtName}, die \mpAttribute{Anzahl der Einwohner}
\dname{EinwohnerAnzahl}, sowie das Land in dem es \mpRelationship{liegt},
festgehalten werden. Gehen Sie davon aus, dass eine Stadt nur in
Zusammenhang mit dem zugehörigen Land identifizierbar ist.

\item Für ein \mpEntity{Land} soll der Name in
\mpAttribute{Landessprache}, der \mpAttribute{Name des
Staatspräsidenten} \dname{Staatspräsident} und eine eindeutige
\mpAttribute{ID} \dname{LandesID} gespeichert werden.
\end{itemize}

\begin{enumerate}

%%
% (a)
%%

\item Entwerfen Sie für das obige Szenario ein ER-Diagramm in
Chen-Notation. Bestimmen Sie hierzu:

\begin{itemize}

\item Die Entity-Typen, die Relationship-Typen und jeweils deren
Attribute,

\item Die Primärschlüssel der Entity-Typen, welche Sie anschließend in
das ER-Diagramm eintragen, und

\item Die Funktionalitäten der Relationship-Typen.
\end{itemize}

Hinweis: Achten Sie darauf, alle Totalitäten einzutragen.

\begin{antwort}
\begin{center}
\begin{tikzpicture}[er2]
% Land
\node[entity] (Land) {Land};
\node[attribute,above=1cm of Land] {Staatspräsident} edge (Land);
\node[attribute,above left=0.5cm of Land] {Landessprache} edge (Land);
\node[attribute,left=0.5cm of Land] {\key{LandesID}} edge (Land);

% Stadt
\node[weak entity,right=3cm of Land] (Stadt) {Stadt};
\node[attribute,above right=0.5cm of Stadt] {EinwohnerAnzahl} edge (Stadt);
\node[attribute,right=0.5cm of Stadt] {\discriminator{StadtName}} edge (Stadt);

% liegen
\node[ident relationship,right=0.4cm of Land]{liegen}
  edge (Land) edge[weak] (Stadt);

% Konsulat
\node[entity,below right=1cm of Land] (Konsulat) {Konsulat};
\node[attribute,left=0.5cm of Konsulat] {Adresse} edge (Konsulat);
\node[attribute,below left=0.5cm of Konsulat] {KNachname} edge (Konsulat);
\node[attribute,below=1cm of Konsulat] {KVorname} edge (Konsulat);
\node[attribute,below right=0.5cm of Konsulat] {\key{KonsulatName}} edge (Konsulat);

\end{tikzpicture}
\end{center}
\end{antwort}

%%
% (b)
%%

\item Überführen Sie das ER-Modell aus Aufgabe a) in ein verfeinertes
relationales Modell. Geben Sie hierfür die verallgemeinerten
Relationenschemata an. Achten Sie dabei insbesondere darauf, dass die
Relationenschemata keine redundanten Attribute enthalten.

\begin{antwort}
Konsulat(\underline{KonsulatName}, KVorname, KNachname, Adresse, StadtName, LandesID)

Stadt(\underline{LandesID, StadtName}, EinwohnerAnzahl)

Land(\underline{LandesID}, Landessprache, Staatspraesident)
\end{antwort}
\end{enumerate}

\end{document}
