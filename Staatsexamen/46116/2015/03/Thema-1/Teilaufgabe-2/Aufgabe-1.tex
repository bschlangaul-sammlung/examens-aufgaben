\documentclass{lehramt-informatik-aufgabe}
\InformatikPakete{syntax,mathe}
\begin{document}

\section{1. Relationale Anfragesprachen
\index{Relationale Algebra}
\footcite[Thema 1 Teilaufgabe 2 Aufgabe 1]{examen:46116:2015:09}
}

Gegeben sei folgendes relationales Schema, dessen Attribute nur atomare
Attributwerte besitzen.
\footcite{db:ab:2}

Computer: \{IP, Name, Hersteller, Modell, Standort\}

\begin{enumerate}

%%
% 1.
%%

\item Geben Sie für die folgenden Anfragen einen relationalen Ausdruck
an:

\begin{enumerate}

%%
% (a)
%%

\item Geben Sie die IP-Adresse des Computers mit Namen „Chiemsee“ aus.

\begin{antwort}
$\pi_{\text{IP}}(\sigma_{\text{Name} = \text{Chiemsee}}(\text{Computer}))$
\end{antwort}

%%
% (b)
%%

\item Geben Sie 2er-Tupel von IP-Adressen der Computer am selben
Standort aus.

\begin{antwort}
\begin{multline*}
\pi_{\text{c1.IP},\text{c2.IP}}(
  \sigma_{\text{c1.Standort} = \text{c2.Standort}}(\\
    \rho_{\text{c1}} (\text{Computer})
    \times
    \rho_{\text{c2}} (\text{Computer})
  )
)
\end{multline*}
\end{antwort}
\end{enumerate}

%%
% (2)
%%

\item Formulieren Sie die folgenden Anfragen in SQL\index{SQL}:

\begin{enumerate}

%%
% (a)
%%

\item Geben Sie die IP-Adressen der Rechner am Standort „Büro2“ aus.

\begin{antwort}
\begin{minted}{sql}
SELECT IP FROM Computer WHERE Standort = 'Büro2';
\end{minted}
\end{antwort}

%%
% (b)
%%

\item Geben Sie alle Computer-Namen in aufsteigender Ordnung mit ihren
IP-Adressen aus.

\begin{antwort}
\begin{minted}{sql}
SELECT Name, IP FROM Computer ORDER BY Name ASC;
\end{minted}
\end{antwort}

%%
% (c)
%%

\item Geben Sie für jeden Hersteller die Anzahl der unterschiedlichen
Modelle aus.

\begin{antwort}
\begin{minted}{sql}
SELECT COUNT(DISTINCT Modell), Hersteller
FROM Computer
GROUP BY Hersteller;
\end{minted}
\end{antwort}

%%
% (d)
%%

\item Geben Sie für jeden Hersteller, welcher mindestens 2
unterschiedliche Modelle hat, die Anzahl der unterschiedlichen Modelle
aus.\index{GROUP BY}\index{HAVING}

\begin{antwort}
\begin{minted}{sql}
SELECT Hersteller, COUNT(*) FROM Modelle GROUP BY Hersteller HAVING COUNT(*) > 1;
\end{minted}
\end{antwort}

oder

\begin{antwort}
\begin{minted}{sql}
SELECT COUNT(DISTINCT Modell), Hersteller
FROM Computer
GROUP BY Hersteller
HAVING COUNT(DISTINCT Modell) >= 2;
\end{minted}
\end{antwort}
\end{enumerate}
\end{enumerate}
\end{document}
