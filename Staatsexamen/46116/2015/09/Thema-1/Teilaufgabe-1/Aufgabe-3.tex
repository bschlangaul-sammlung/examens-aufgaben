\documentclass{bschlangaul-aufgabe}
\bLadePakete{gantt}
\begin{document}
\bAufgabenTitel{Gantt und PERT}

\section{3. Projektmanagement
\index{Gantt-Diagramm}
\footcite[Thema 1 Teilaufgabe 1 Aufgabe 3]{examen:46116:2015:09}}

Betrachten Sie die folgende Tabelle zum Projektmanagement:

\begin{center}
\begin{tabular}{|l|l|l|}
\hline
Name & Dauer (Tage) & Abhängig von\\\hline\hline
A1 & 10 & \\\hline
A2 & 5  & A1 \\\hline
A3 & 15 & A1 \\\hline
A4 & 10 & A2, A3 \\\hline
A5 & 15 & A1, A3 \\\hline
A6 & 10 & \\\hline
A7 & 5  & A2, A4 \\\hline
A8 & 10 & A4, A5, A6 \\\hline
\end{tabular}
\end{center}

Tabelle 1: Übersicht Arbeitspakete

\begin{enumerate}

%%
% a)
%%

\item Erstellen Sie ein Gantt-Diagramm, das die in der Tabelle
angegebenen Abhängigkeiten berücksichtigt.

\begin{bAntwort}
\begin{center}
\begin{ganttchart}[
  x unit=0.2cm,
  y unit chart=0.8cm,
  vgrid
]{1}{50}
\ganttbar[name=4]{A4}{26}{35} \\
\ganttbar[name=5]{A5}{26}{40} \\
\ganttbar[name=1]{A1}{1}{10} \\
\ganttbar[name=2]{A2}{11}{15} \\
\ganttbar[name=3]{A3}{11}{25} \\
\ganttbar[name=7]{A7}{36}{40} \\
\ganttbar[name=6]{A6}{1}{10} \\
\ganttbar[name=8]{A8}{41}{50} \\

\gantttitlelist[
  title list options={var=\y, evaluate={} as \x}
]{1,...,50}{1}\\
\gantttitlelist[
  title list options={var=\i, evaluate={int(\i * 5)} as \x}
]{1,...,10}{5}\\

%
\ganttlink[link type=f-s]{1}{2}
\ganttlink[link type=f-s]{1}{3}

\ganttlink[link type=f-s]{2}{4}
\ganttlink[link type=f-s]{3}{4}

\ganttlink[link type=f-s]{1}{5}
\ganttlink[link type=f-s]{3}{5}

\ganttlink[link type=f-s]{2}{7}
\ganttlink[link type=f-s]{4}{7}

\ganttlink[link type=f-s]{4}{8}
\ganttlink[link type=f-s]{5}{8}
\ganttlink[link type=f-s]{6}{8}
\end{ganttchart}
\end{center}
\end{bAntwort}

%%
% b)
%%

\item Wie lange dauert das Projekt mindestens?

%%
% c)
%%

\item Geben Sie den oder die kritischen Pfad(e) an.

%%
% d)
%%

\item Konstruieren Sie ein PERT-Chart zum obigen Problem.

\end{enumerate}
\end{document}
