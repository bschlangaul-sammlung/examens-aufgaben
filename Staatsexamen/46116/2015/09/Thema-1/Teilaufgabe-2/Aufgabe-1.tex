\documentclass{lehramt-informatik-aufgabe}
\liLadePakete{}
\begin{document}
\liAufgabenTitel{Theoriefragen Datenbank}

\section{Aufgabe 1
\footcite[Thema 1 Aufgabe 1]{examen:46116:2015:09}}

Erläutern Sie die folgenden Begriffe in knappen Worten:
\footcite[Aufgabe 12: (Begriffe definieren)]{db:ab:7}

\begin{enumerate}

%%
% (a)
%%

\item Datenunabhängigkeit\index{Datenunabhängigkeit}

\begin{liAntwort}
Änderungen an der physischen Speicher- oder der Zugriffsstruktur
(beispielsweise durch das Anlegen einer Indexstruktur) haben keine
Auswirkungen auf die logische Struktur der Datenbasis, also auf das
Datenbankschema.
\end{liAntwort}

%%
% (b)
%%

\item Superschlüssel\index{Superschlüssel}

\begin{liAntwort}
Ein Superschlüssel ist ein Attribut oder Attributkombination, von der
alle Attribute einer Relation funktional abhängen.
\end{liAntwort}

%%
% (c)
%%

\item Referentielle Integrität\index{Referentielle Integrität}

\begin{liAntwort}
Unter Referentieller Integrität verstehen wir Bedingungen, die zur
Sicherung der Datenintegrität bei Nutzung relationaler Datenbanken
beitragen können. Demnach dürfen Datensätze über ihre
Fremdschlüssel nur auf existierende Datensätze verweisen.

Danach besteht die Referentieller Integrität grundsätzlich aus zwei
Teilen:

\begin{enumerate}

\item Ein neuer Datensatz mit einem Fremdschlüssel kann nur dann in
einer Tabelle eingefügt werden, wenn in der referenzierten Tabelle ein
Datensatz mit entsprechendem Wert im Primärschlüssel oder einem
eindeutigen Alternativschlüssel existiert.

\item Eine Datensatzlöschung oder Änderung des Schlüssels in einem
Primär-Datensatz ist nur möglich, wenn zu diesem Datensatz keine
abhängigen Datensätze in Beziehung stehen.
\footcite{wiki:referentielle-Integritaet}
\end{enumerate}
\end{liAntwort}
\end{enumerate}
\end{document}
