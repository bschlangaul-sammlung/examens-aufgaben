\documentclass{lehramt-informatik-aufgabe}
\liLadePakete{syntax,mathe,wpkalkuel}
\begin{document}
\let\wp=\liWpKalkuel
\let\equivalent=\liWpEquivalent
\let\erklaerung=\liWpErklaerung

\liAufgabenTitel{ASCII}

\section{Aufgabe 3: „Formale Verifikation“
\index{wp-Kalkül}
\footcite{examen:46116:2015:09}}

Sei \text{wp}(A, Q) die schwächste Vorbedingung (weakest precondition)
eines Programmfragments $A$ bei gegebener Nachbedingung $Q$ so, dass $A$
alle Eingaben, die \text{wp}(A,Q) erfüllen, auf gültige Ausgaben
abbildet, die $Q$ erfüllen.

Bestimmen Sie schrittweise und formal (mittels Floyd-Hoare-Kalkül)
jeweils \text{wp}(A, Q) für folgende Code-Fragmente $A$ und
Nachbedingungen $Q$ und vereinfachen Sie dabei den jeweils ermittelten
Ausdruck so weit wie möglich.

Die Variablen $x$, $y$ und $z$ in folgenden Pseudo-Codes seien
ganzzahlig (vom Typ int). Zur Vereinfachung nehmen Sie bitte im
Folgenden an, dass die verwendeten Datentypen unbeschränkt sind und
daher keine Überläufe auftreten können.
\footcite{sosy:pu:5}

\begin{enumerate}
\item Sequenz:
\begin{minted}{java}
x = -2 * (x + 2 * y);
y += 2 * x + y + z;
z -= x - y - z;
\end{minted}

$Q :\equiv x = y + z$

\begin{liAntwort}
Code umformulieren:

\begin{minted}{java}
x = -2 * (x + 2 * y);
y = y + 2 * x + y + z;
z = z - (x - y - z);
\end{minted}

\wp{x=-2*(x+2*y);y=2*y+2*x+z;z=z-(x-y-z);}{x = y + z}

\erklaerung{$z$ eingesetzen}

\equivalent{\wp{x=-2*(x+2*y);y=2*y+2*x+z;}{x = y + (z - (x - y - z))}}

\erklaerung{Innere Klammer auflösen}

\equivalent{\wp{x=-2*(x+2*y);y=2*y+2*x+z;}{x = y + (- x + y - 2z)}}

\erklaerung{Klammer auflösen}

\equivalent{\wp{x=-2*(x+2*y);y=2*y+2*x+z;}{x = - x + 2y + 2z}}

\erklaerung{$-x$ auf beiden Seiten}

\equivalent{\wp{x=-2*(x+2*y);y=2*y+2*x+z;}{0 = -2x + 2y + 2z}}

\erklaerung{$\div 2$ auf beiden Seiten}

\equivalent{\wp{x=-2*(x+2*y);y=2*y+2*x+z;}{0 = -x + y + z}}

\erklaerung{$y$ einsetzen}

\equivalent{\wp{x=-2*(x+2*y);}{0 = -x + (2y + 2x + z) + z}}

\erklaerung{Term vereinfachen}

\equivalent{\wp{x=-2*(x+2*y);}{0 = x + 2y + 2z}}

\erklaerung{$x$ einsetzen}

\equivalent{\wp{}{0 = (-2(x + 2y)) + 2y + 2z}}

\erklaerung{wp weglassen}

\equivalent{0 = (-2(x + 2y)) + 2y + 2z}

\erklaerung{ausmultiplizieren}

\equivalent{0 = (-2x - 4y) + 2y + 2z}

\erklaerung{Klammer auflösen, vereinfachen}

\equivalent{0 = -2x - 2y + 2z}

\erklaerung{$\div 2$ auf beiden Seiten}

\equivalent{0 = -x - y + z}

\erklaerung{$x$ nach links holen mit $+ x$ auf beiden Seiten}

\equivalent{x = -y + z}

\erklaerung{$y$ ganz nach links schreiben}

\equivalent{x = z - y}

$x = -2 \cdot (x + 2 \cdot y)$

\end{liAntwort}

\item Verzweigung:

\begin{minted}{java}
if (x < y) {
  x = y + z;
} else if (y > 0) {
  z = y - 1;
} else {
  x -= y -= z;
}
\end{minted}

$Q :\equiv x > z$

\item Mehrfachauswahl:

\begin{minted}{java}
switch (z) {
  case "x":
    y = "x";
  case "y":
    y = --z;
    break;
  default:
    y = 0x39 + "?";
}
\end{minted}

$Q :\equiv \texttt{'x'} = y$

Hinweis zu den ASCII-Codes

\begin{itemize}
\item $\texttt{'x'} = 120_{(10)}$
\item $\texttt{'y'} = 121_{(10)}$
\item $\texttt{0x39} = 57_{(10)}$
\item $\texttt{'?'} = 63_{(10)}$
\end{itemize}

\end{enumerate}

\begin{liAntwort}
Mehrfachauswahl in Bedingte Anweisungen umschreiben. Dabei beachten,
dass bei fehlendem \liJavaCode{break} die Anweisungen im folgenden Fall
bzw. ggf. in den folgenden Fällen ausgeführt werden:

\begin{minted}{java}
if (z == "x") {
  y = "x";
  y = z - 1;
} else if (z == "y") {
  y = z - 1;
} else {
  y = 0x39 + "?";
}
\end{minted}

\noindent
Da kein \liJavaCode{break} im Fall \liJavaCode{z == "x"}.
\liJavaCode{--z} bedeutet, dass die Variable erst um eins verringert und
dann zugewiesen wird.

\begin{minted}{java}
if (z == 120) {
  y = 120;
  y = 120 - 1;
} else if (z == 121) {
  y = 121 - 1;
} else {
  y = 57 + 63;
}
\end{minted}

\noindent
Vereinfachung / Zusammenfassung:

\begin{minted}{java}
if (z == 120) {
  y = 120;
  y = 119;
} else if (z == 121) {
  y = 120;
} else {
  y = 120;
}
\end{minted}
\end{liAntwort}
\end{document}
