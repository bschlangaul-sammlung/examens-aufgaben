\documentclass{lehramt-informatik-minimal}
\InformatikPakete{syntax,mathe}
\begin{document}

\section{Aufgabe 3: „Formale Verifikation“
\index{wp-Kalkül}
\footcite{examen:46116:2015:09}}

Sei \text{wp}(A, Q) die schwächste Vorbedingung (weakest precondition)
eines Programmfragments $A$ bei gegebener Nachbedingung $Q$ so, dass $A$
alle Eingaben, die \text{wp}(A,Q) erfüllen, auf gültige Ausgaben
abbildet, die $Q$ erfüllen.

Bestimmen Sie schrittweise und formal (mittels Floyd-Hoare-Kalkül)
jeweils \text{wp}(A, Q) für folgende Code-Fragmente $A$ und
Nachbedingungen $Q$ und vereinfachen Sie dabei den jeweils ermittelten
Ausdruck so weit wie möglich.

Die Variablen $x$, $y$ und $z$ in folgenden Pseudo-Codes seien
ganzzahlig (vom Typ int). Zur Vereinfachung nehmen Sie bitte im
Folgenden an, dass die verwendeten Datentypen unbeschränkt sind und
daher keine Überläufe auftreten können.
\footcite{sosy:pu:5}

\begin{enumerate}
% \item Sequenz:
% x = -2 * (x + 2 * vy)?
% yt= 2* x+yt 2
% Z-= X- yr zZ
% Q:=x=yt+z

\item Verzweigung:

% if (x < y) {
% KX =yit Z
% } else if (y > 0) {

% z=y-];
% } else {

% x -= y = 2;
% }
% O:=ex>z

\item Mehrfachauswahl:

\begin{minted}{java}
switch (z) {
  case "x":
    y = "x";
  case "y":
    y = --z;
    break;
  default:
    y = 0x39 + "?";
}
\end{minted}

$Q :\equiv \texttt{"x"} = y$

Hinweis zu den ASCII-Codes

\begin{itemize}
\item $\texttt{"x"} = 120_{(10)}$
\item $\texttt{"y"} = 121_{(10)}$
\item $\texttt{0x39} = 57_{(10)}$
\item $\texttt{"?"} = 63_{(10)}$
\end{itemize}

\end{enumerate}

\begin{antwort}
Mehrfachauswahl in Bedingte Anweisungen umschreiben. Dabei beachten,
dass bei fehlendem \java{break} die Anweisungen im folgenden Fall bzw.
ggf. in den folgenden Fällen ausgeführt werden:

\begin{minted}{java}
if (z == "x") {
  y = "x";
  y = z - 1;
} else if (z == "y") {
  y = z - 1;
} else {
  y = 0x39 + "?";
}
\end{minted}

Da kein \java{break} im Fall \java{z == "x"}. \java{--z} bedeutet, dass
die Variable erst um eins verringert und dann zugewiesen wird.

\begin{minted}{java}
if (z == 120) {
  y = 120;
  y = 120 - 1;
} else if (z == 121) {
  y = 121 - 1;
} else {
  y = 57 + 63;
}
\end{minted}

Vereinfachung / Zusammenfassung:

\begin{minted}{java}
if (z == 120) {
  y = 120;
  y = 119;
} else if (z == 121) {
  y = 120;
} else {
  y = 120;
}
\end{minted}
\end{antwort}
\end{document}
