\documentclass{lehramt-informatik-haupt}
\begin{document}

\chapter{Fallunterscheidung}

%-----------------------------------------------------------------------
%
%-----------------------------------------------------------------------

\cite{oomup:fs:2}

\section{Fallunterscheidung ohne Klammer}

Die Schreibweise einer Fallunterscheidung ohne Klammer ist nur möglich,
wenn nach der Bedingung nur eine Anweisung folgt.

\begin{minted}{java}
int x = 2;
if (x > 1)
  System.out.println("x ist größer als 1");
else
  System.out.println("x ist kleiner gleich 1");
\end{minted}

%-----------------------------------------------------------------------
%
%-----------------------------------------------------------------------

\section{Tenärer Operator}

\texttt{variable = bedingung ? wert1 : wert2}

\bigskip

\noindent
\texttt{bedingung} muss immer ein boolscher Ausdruck sein. Er
entscheidet über die Wertzuweisung. Ist er \liJavaCode{true}, so wird der Wert
nach dem Fragezeichen zugewiesen, ansonsten der Wert nach dem
Doppelpunkt. Der zweite und der dritte Operand können beliebige
Ausdrücke sein, die einen Wert zurückgeben.

\begin{minted}{java}
int x = 2;
String msg = x > 1 ? "x ist größer als 1" : "x ist kleiner gleich 1";
System.out.println(msg);
\end{minted}

%-----------------------------------------------------------------------
%
%-----------------------------------------------------------------------

\section{Mehrfache Fallauswahl}

Bei mehrfachen Fallunterscheidungen wird die Verschachtelung bedingter
Anweisungen unübersichtlich. Hierfür bietet Java die sogenannte
\liJavaCode{switch}-Anweisung.

\bigskip
\noindent
\textbf{Hinweis:} Die switch-Anweisung funktioniert \memph{nur} mit
\memph{ganzen Zahlen} oder \memph{Buchstaben}, nicht aber mir reellen
Zahlen oder Strings.

\bigskip
\noindent
\textbf{Beispiel:} Die Schulnote eines Schülers wird in der Variable
\liJavaCode{note} gespeichert. Geben Sie mit Hilfe der \liJavaCode{switch}-Anweisung
die jeweilige Notenbedeutung als Text aus.
\footcite[Seite 57]{oomup:fs:2}

\begin{minted}{java}
switch (note) {
  case 1:
    System.out.println("sehr gut");
    break;
  case 2:
    System.out.println("gut");
    break;
  case 3:
    System.out.println("befriedigend");
    break;
  case 4:
    System.out.println("ausreichend");
    break;
  case 5:
    System.out.println("mangelhaft");
    break;
  case 6:
    System.out.println("ungenügend");
    break;
}

\end{minted}

\literatur

\end{document}
