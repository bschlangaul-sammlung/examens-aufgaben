\documentclass{lehramt-informatik-haupt}
\liLadePakete{uml}
\begin{document}

%%%%%%%%%%%%%%%%%%%%%%%%%%%%%%%%%%%%%%%%%%%%%%%%%%%%%%%%%%%%%%%%%%%%%%%%
% Theorie-Teil
%%%%%%%%%%%%%%%%%%%%%%%%%%%%%%%%%%%%%%%%%%%%%%%%%%%%%%%%%%%%%%%%%%%%%%%%

\chapter{Zustandsdiagramm (State diagram)}

\begin{tikzpicture}
\begin{umlstate}[name=Amain]{Etat global de l'objet A}
\begin{umlstate}[name=Bgraph]{graphe B}
\umlstateinitial[name=Binit]
\umlbasicstate[y=-4, name=test1]{test1}
\umltrans{Binit}{test1}
\umltrans[recursive=20|60|2.5cm, recursive direction=right to top, arg={op1}, pos=1.5]{test1}{test1}
\umltrans[recursive=160|120|2.5cm, recursive direction=left to top, arg={op2}, pos=1.5]{test1}{test1}
\umltrans[recursive=-160|-120|2.5cm, recursive direction=left to bottom, arg={op3}, pos=1.5]{test1}{test1}
\umltrans[recursive=-20|-60|2.5cm, recursive direction=right to bottom, arg={op4}, pos=1.5]{test1}{test1}
\umlbasicstate[y=-8, name=test2]{test2}
\umltrans[recursive=-160|-120|2.5cm, recursive direction=left to bottom, arg={op5}, pos=1.5]{test2}{test2}
\umltrans{test1}{test2}
\umlstatefinal[x=3, y=-7.75, name=Bfinal]
\umltrans{test2}{Bfinal}
\end{umlstate}
\umlstateinitial[x=6, y=1, name=Ainit]
\umlVHtrans[anchor2=40]{Ainit}{Bgraph}
\umlstatefinal[x=6, y=-3.5, name=Afinal]
\umlHVtrans[anchor1=30]{Bgraph}{Afinal}
\umlbasicstate[x=6, y=-6, name=visu]{Visualisation}
\umlHVtrans{Bfinal}{visu}
\umltrans{visu}{Afinal}
\umltrans[recursive=-20|-60|2.5cm, recursive direction=right to bottom, arg=a, pos=1.5]{visu}{visu}
\end{umlstate}
\end{tikzpicture}

\begin{liQuellen}
\item \cite[Seite 329-400]{rupp}
\item \cite{net:pdf:zustandsdiagramm}
\end{liQuellen}

\noindent
Zustandsautomaten (state diagrams) beschreiben die Systemzustände bei
definierten Ereignissen. Das bedeutet, ein Zustandsautomat bildet die
verschiedenen Zustände ab, die ein Objekt während seiner Lebenszeit
durchläuft.
\footcite[Seite 166]{schatten}

\begin{description}

%%
%
%%

\item[Einfacher Zustand (simple state))] \strut

\begin{description}
\item[Notation]

Ein Rechteck mit abgerundeten Ecken, das durch eine waagrechte Linie
unterteilt sein kann, symbolisiert einen Zustand.

Im zweiten Abschnitt wird angegeben, welches interne Verhalten und
welche internen Transitionen in diesem Zustand ausgeführt werden können.

In der UML sind die folgenden Arten von Verhalten eines Zustandes mit
ihren Auslösern definiert. Die Auslöser gelten als Schlüsselwörter und
dürfen demnach nicht in einem an­ deren Kontext verwendet werden.

\begin{description}
\item[Eintrittsverhalten:] entry / Verhalten
\item[Austrittsverhalten:] exit / Verhalten
\item[Zustandsverhalten:] do / Verhalten
\end{description}

\item[Beschreibung]

Ein einfacher Zustand (simple state) bildet eine Situation ab, in deren
Verlauf eine spezielle Bedingung gilt.
\footcite[Seite 338]{rupp}

\begin{center}
\begin{tikzpicture}
\begin{umlstate}[name=substate, entry=Verhalten, exit=Verhalten, do=Verhalten]{Zustand}
\end{umlstate}
\end{tikzpicture}
\end{center}
\end{description}

%%
%
%%

\item[Transition / Zustandsübergängen] \strut

\begin{description}
\item[Notation] \strut

Eine Transition wird durch eine durchgezogene, gerichtete und
üblicherweise beschriftete Kante abgebildet. Die Beschriftung beinhaltet
die folgenden Elemente (Ereignis(Argumente) [Bedingung] / Aktivität):

\begin{description}
\item[Trigger / Ereignis] der Auslöser für die Transition. Die einzelnen
Trigger werden durch Kommas voneinander getrennt.

\item[Guard / Bedingung] eine Bedingung, die wahr sein muss, damit die
Transition bei Erhalt des Triggers durchlaufen wird. Die Guard wird in
eckigen Klammern notiert.

\item[Verhalten / Aktivität / Aktion] Das Verhalten, das beim
Durchlaufen der Transition ausgeführt wird. Es wird durch den Namen des
gewünschten Verhaltens angegeben.
\end{description}

\item[Beschreibung] Transitionen schaffen einen Übergang von einem
Ausgangs- zu einem Zielzustand.
\footcite[Seite 340-341]{rupp}

\end{description}

\begin{center}
\begin{tikzpicture}
\draw[tikzuml transition style] (0,0) -- (5,0) node[auto,pos=0.5]{Trigger[Guard] / Verhalten};
\end{tikzpicture}
\end{center}

%%
%
%%

\item[Startzustand]
Ein Startzustand wird als ausgefüllter Kreis dargestellt.

\begin{center}
\tikz \umlstateinitial ;
\end{center}

%%
%
%%

\item[Endzustand]
Ein Endzustand wird als ein kleiner ausgefüllter Kreis, umgeben von
einem unausgefüllten Kreis, dargestellt.

\begin{center}
\tikz \umlstatefinal ;
\end{center}
\end{description}

\literatur

\end{document}
