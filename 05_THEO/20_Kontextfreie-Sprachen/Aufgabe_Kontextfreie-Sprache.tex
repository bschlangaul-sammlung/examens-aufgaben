\documentclass{lehramt-informatik-aufgabe}
\liLadePakete{formale-sprachen,automaten}
\begin{document}
\let\m=\liMengeOhneMathe
\let\u=\liKellerUebergang
\let\z=\liZustandsnameTiefgestellt

\liAufgabenTitel{Kontextfreie Sprache}
\section{Kontextfreie Sprache
\index{Kontextfreie Sprache}
\footcite{theo:ab:2}}

Gegeben ist die Grammatik $G = (\m{a, b}, \m{S, A, B}, S, P)$ und den
Produktionen

\bigskip
\noindent
\begin{liProduktionsRegeln}
S -> SAB | epsilon,
BA -> AB,
AA -> aa,
BB -> bb
\end{liProduktionsRegeln}

\begin{enumerate}

%%
% (a)
%%

\item Geben Sie einen (regulären?) Ausdruck an, der die Wörter der
Sprache beschreibt.

% S
%

\begin{liAntwort}
\let\u=\liPseudoUeberschrift

„.“ nur als optische Stütze nach 4 Zeichen eingefügt

\u{4}

\liAbleitung{S -> SAB -> SABAB -> ABAB -> AABB -> aabb}

\u{6}

\liAbleitung{S -> ... -> ABAB.AB -> AABB.AB -> AABA.BB -> AAAB.BB -> nichts}

\u{8}

\liAbleitung{S -> ... -> ABAB.ABAB -> ... -> aabb.aabb }

\liAbleitung{S -> ... -> ABAB.ABAB -> ... -> AABB.AABB -> AABA.BABB -> AABA.ABBB -> AAAB.ABBB -> AAAA.BBBB -> aaaa.bbbb}

\u{12}

\liAbleitung{S -> ... -> ABAB.ABAB.ABAB -> ... -> aabbaabbaabb }

\liAbleitung{S -> ... -> ABAB.ABAB.ABAB -> AABB.ABAB.ABAB -> AABA.BBAB.ABAB -> AAAB.BBAB.ABAB ... -> aaaaaabbbbbb }

(aabb)*
\end{liAntwort}

%%
% (b)
%%

\item Geben Sie eine kontextfreie Grammatik $G'$ an, für die gilt:
$L(G') = L(G)$

\begin{liAntwort}
\begin{liProduktionsRegeln}
S -> SAB | epsilon,
A -> a a,
B -> b b
\end{liProduktionsRegeln}
%\liFussnoteUrl{https://flaci.com/Grn19rt8w}
\end{liAntwort}

%%
% (c)
%%

\item Geben Sie einen Kellerautomaten an, der die Sprache akzeptiert.

\begin{liAntwort}

$K = (\m{\z0, \z1}, \m{a, b}, \m{\#, A, B}, \delta, \z0, \#, \z0)$

\begin{tikzpicture}[li kellerautomat,node distance=3cm]
\node[state,initial,accepting] (0) {\z0};
\node[state,right of=0] (1) {\z1};

\path (0) edge[above,loop] node{
  \u{epsilon raute epsilon}
} (0);

\path (0) edge[above,bend left] node{
  \u{a raute Araute}
} (1);

\path (1) edge[above,loop] node{
  \u{b raute Braute}\\
  \u{a A epsilon}\\
} (1);

\path (1) edge[below,bend left] node{
  \u{b B epsilon}
} (0);

\end{tikzpicture}

% DKA
% https://flaci.com/Ai8krgvor

\end{liAntwort}

\end{enumerate}
\end{document}
