\documentclass{lehramt-informatik-aufgabe}
\liLadePakete{formale-sprachen,automaten}
\begin{document}
\let\m=\liMengeOhneMathe
\let\u=\liKellerUebergang
\let\z=\liZustandsnameTiefgestellt

\liAufgabenTitel{Kontextfreie Sprache}
\section{Kontextfreie Sprache
\index{Kontextfreie Sprache}
\footcite{theo:ab:2}}

Gegeben ist die Grammatik $G = (\m{a, b}, \m{S, A, B}, S, P)$ und den
Produktionen

\bigskip
\noindent
\begin{liProduktionsRegeln}
S -> SAB | epsilon,
BA -> AB,
AA -> aa,
BB -> bb
\end{liProduktionsRegeln}

\begin{enumerate}

%%
% (a)
%%

\item Geben Sie einen Ausdruck an, der die Wörter der Sprache
beschreibt.

\begin{liAntwort}
\let\u=\liPseudoUeberschrift

Einige Testableitungen um die Grammatik in Erfahrung zu bringen:

„.“ nur als optische Stütze nach 4 Zeichen eingefügt.

\u{Mit 4 Buchstaben}

\liAbleitung{S -> SAB -> SABAB -> ABAB -> AABB -> aabb}

\u{Mit 6 Buchstaben}

\liAbleitung{S -> ... -> ABAB.AB -> AABB.AB -> AABA.BB -> AAAB.BB -> nichts}

\u{Mit 8 Buchstaben}

\liAbleitung{S -> ... -> ABAB.ABAB -> ... -> aabb.aabb}

\liAbleitung{S -> ... -> ABAB.ABAB -> ... -> AABB.AABB -> AABA.BABB -> AABA.ABBB -> AAAB.ABBB -> AAAA.BBBB -> aaaa.bbbb}

\u{Mit 12 Buchstaben}

\liAbleitung{S -> ... -> ABAB.ABAB.ABAB -> ... -> aabb.aabb.aabb}

\liAbleitung{S -> ... -> ABAB.ABAB.ABAB -> AAAA.BBBB.AABB  -> aaaa.bbbb.aabb}

\liAbleitung{S -> ... -> ABAB.ABAB.ABAB -> AABB.ABAB.ABAB -> AABA.BBAB.ABAB -> AAAB.BBAB.ABAB ... -> aaaa.aabb.bbbb}
\end{liAntwort}

%%
% (b)
%%

\item Geben Sie eine kontextfreie Grammatik $G'$ an, für die gilt:
$L(G') = L(G)$

\begin{liAntwort}

\begin{liProduktionsRegeln}
S -> aa S bb | S S | epsilon,
\end{liProduktionsRegeln}

%\liFussnoteUrl{https://flaci.com/Grn19rt8w}
\end{liAntwort}

%%
% (c)
%%

\newpage

\item Geben Sie einen Kellerautomaten an, der die Sprache akzeptiert.

\begin{liAntwort}

$K = (\m{\z0, \z1}, \m{a, b}, \m{\#, 1, 2}, \delta, \z0, \#, \z0)$

Bemerkung: 1 steht für 1A und 2 steht für 2A

\begin{tikzpicture}[li kellerautomat,node distance=3cm]
\node[state,initial,accepting] (0) {\z0};
\node[state,right of=0] (1) {\z1};

\path (0) edge[above,loop] node{
  \u{epsilon raute epsilon}
  \u{a 2 12}
  \u{a 1 21}
  \u{a raute 1raute}
} (0);

\path (0) edge[above,bend left] node{
  \u{epsilon 2 2}
} (1);

\path (1) edge[above,loop] node{
  \u{b 1 epsilon}
  \u{b 2 epsilon}
} (1);

\path (1) edge[below,bend left] node{
  \u{b B epsilon}
} (0);

\end{tikzpicture}

% \liFussnoteUrl{https://flaci.com/Ahfqseouz}

\end{liAntwort}

\end{enumerate}
\end{document}
