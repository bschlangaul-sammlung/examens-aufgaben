\documentclass{lehramt-informatik-aufgabe}
\liLadePakete{formale-sprachen,automaten}
\begin{document}
\let\m=\liMengeOhneMathe
\let\z=\liZustandsnameTiefgestellt

\liAufgabenTitel{Kontextfreie Sprache}
\section{Kontextfreie Sprache
\index{Kontextfreie Sprache}
\footcite{theo:ab:2}}

Gegeben ist die Grammatik $G = (\m{a, b}, \m{S, A, B}, S, P)$ und den
Produktionen

\bigskip
\noindent
\begin{liProduktionsRegeln}
S -> S A B | EPSILON,
B A -> A B,
A A -> a a,
B B -> b b
\end{liProduktionsRegeln}

\begin{enumerate}

%%
% (a)
%%

\item Geben Sie einen Ausdruck an, der die Wörter der Sprache
beschreibt.

\begin{liAntwort}

$L = \m{(a^n b^n)^m \, | \, m \in \mathbb{N}_0 \text{ und } n \in \text{ gerade Zahlen}}$

\let\u=\liPseudoUeberschrift

Einige Testableitungen um die Grammatik in Erfahrung zu bringen:

„.“ nur als optische Stütze nach 4 Zeichen eingefügt.

\u{Mit 4 Buchstaben}

\liAbleitung{S -> SAB -> SABAB -> ABAB -> AABB -> aabb}

\u{Mit 6 Buchstaben}

\liAbleitung{S -> ... -> ABAB.AB -> AABB.AB -> AABA.BB -> AAAB.BB -> nichts}

\u{Mit 8 Buchstaben}

\liAbleitung{S -> ... -> ABAB.ABAB -> ... -> aabb.aabb}

\liAbleitung{S -> ... -> ABAB.ABAB -> ... -> AABB.AABB -> AABA.BABB -> AABA.ABBB -> AAAB.ABBB -> AAAA.BBBB -> aaaa.bbbb}

\u{Mit 12 Buchstaben}

\liAbleitung{S -> ... -> ABAB.ABAB.ABAB -> ... -> aabb.aabb.aabb}

\liAbleitung{S -> ... -> ABAB.ABAB.ABAB -> AAAA.BBBB.AABB  -> aaaa.bbbb.aabb}

\liAbleitung{S -> ... -> ABAB.ABAB.ABAB -> AABB.ABAB.ABAB -> AABA.BBAB.ABAB -> AAAB.BBAB.ABAB ... -> aaaa.aabb.bbbb}
\end{liAntwort}

%%
% (b)
%%

\item Geben Sie eine kontextfreie Grammatik $G'$ an, für die gilt:
$L(G') = L(G)$

\begin{liAntwort}

\begin{liProduktionsRegeln}
S -> aa S bb | S S | epsilon,
\end{liProduktionsRegeln}

\liFlaci{Grn19rt8w}
\end{liAntwort}

%%
% (c)
%%

\newpage

\item Geben Sie einen Kellerautomaten an, der die Sprache akzeptiert.
\index{Kellerautomat}

\begin{liAntwort}
\liPseudoUeberschrift{1. Kellerautomat (aus der Grammtik abgeleitet)}

$K = (\m{\z0, \z1, \z2}, \m{a, b}, \m{\#, S, A, B}, \delta, \z0, \#, \z0)$

\begin{center}
\begin{tikzpicture}[li kellerautomat]
  \node[state,initial,accepting] (z0) at (4.29cm,-4.57cm) {$z_0$};

  \liKellerKante[above,loop]{z0}{z0}{
    a, A, EPSILON;
    b, B, EPSILON;
    EPSILON, S, A A S B B;
    EPSILON, S, S S;
    EPSILON, S, EPSILON;
    EPSILON, KELLERBODEN, S KELLERBODEN;
  }
\end{tikzpicture}
\end{center}

\liFlaci{Araj960s2}

\liPseudoUeberschrift{2. Kellerautomat}

$K = (\m{\z0, \z1}, \m{a, b}, \m{\#, 1, 2}, \delta, \z0, \#, \z0)$

Bemerkung zum Kelleralphabet: $1$ steht für $1A$, also ein $a$ befindet
sich im Keller, und $2$ steht für $2A$, also zwei $a$ befinden sich im
Keller.
\begin{center}
\begin{tikzpicture}[li kellerautomat]
  \node[state,initial,accepting] (z0) at (2.43cm,-5.86cm) {$z_0$};
  \node[state] (z1) at (8.29cm,-5.86cm) {$z_1$};

  \liKellerKante[above,bend left]{z0}{z1}{
    EPSILON, 2, 2;
  }

  \liKellerKante[above,loop]{z0}{z0}{
    a, KELLERBODEN, 1 KELLERBODEN;
    a, 1, 2 1;
    a, 2, 1 2;
    EPSILON, KELLERBODEN, EPSILON;
  }

  \liKellerKante[above,loop]{z1}{z1}{
    b, 2, EPSILON;
    b, 1, EPSILON;
  }

  \liKellerKante[above,bend left]{z1}{z0}{
    EPSILON, KELLERBODEN, KELLERBODEN;
  }
\end{tikzpicture}
\end{center}

\liFlaci{Ahfqseouz}

\end{liAntwort}
\end{enumerate}
\end{document}
