\documentclass{lehramt-informatik-aufgabe}
\liLadePakete{formale-sprachen}
\begin{document}
\liAufgabenTitel{Kontextfreie Sprache}
\section{Kontextfreie Sprache
\index{Kontextfreie Sprache}
\footcite{theo:ab:2}}

Gegeben ist die Grammatik $G = (\{a, b\}, \{S, A, B\}, S, P)$ und den
Produktionen

\begin{liProduktionsRegeln}
S -> SAB | epsilon,
BA -> AB,
AA -> aa,
BB -> bb
\end{liProduktionsRegeln}
\begin{enumerate}

%%
% (a)
%%

\item Gib einen Ausdruck an, der die Wörter der Sprache beschreibt.

%%
% (b)
%%

\item Gib eine kontextfreie Grammatik $G_0$ an, für die gilt:
$L(G_0) = L(G)$

%%
% (c)
%%

\item Gib einen Kellerautomaten an, der die Sprache akzeptiert.

\end{enumerate}
\end{document}
