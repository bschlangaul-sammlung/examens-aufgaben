\documentclass{lehramt-informatik-aufgabe}
\liLadePakete{}
\begin{document}
\liAufgabenTitel{Reguläre Sprache in kontextfreier Sprache}
\section{Reguläre Sprache in kontextfreier Sprache
\index{Reguläre Sprache}
\index{Kontextfreie Sprache}
\footcite{theo:ab:2}}

Zeigen Sie, dass sich eine reguläre Sprache ebenfalls als kontextfreie
Sprache auffassen lässt.

\begin{liAntwort}
Im Grunde genommen kann ein DPDA einen deterministischen endlichen
Automaten simulieren. Der PDA behält ein Stack-Symbol $Z_0$ auf seinem
Stack, weil ein PDA einen Stack besitzen muss, ignoriert den Stack aber
eigentlich und arbeitet lediglich mit seinen Zuständen. Formal
ausgedrückt, sei

\begin{displaymath}
A = (Q, \Sigma, \delta, A, q_0, F)
\end{displaymath}

ein DFA. Wir konstruieren einen DPDA

\begin{displaymath}
P = (Q, \Sigma, \{Z_0 \}, \delta_P, q_0, Z_0, F )
\end{displaymath}
,

indem $\delta_P(q, a, Z_0) = \{(p, Z_0)\}$ für alle Zustände $p$ und $q$
aus $Q$ definiert wird, derart dass $\delta_A(q, a) = p$. Wir behaupten,
dass $(q_0, w, Z_0) \rightarrow (p, \epsilon, Z_0)$ genau dann, wenn
$\delta_A (q_0 , w) = p$ . Das heißt, $P$ simuliert $A$ über seinen
Zustand. Beide Richtungen lassen sich durch einfache Induktionsbeweise
über $|w|$ zeigen. Da sowohl $A$ als auch $P$ akzeptieren, indem sie
einen der Zustände aus $F$ annehmen, schließen wir darauf, dass ihre
Sprachen identisch sind.
\end{liAntwort}

\end{document}
