\documentclass{lehramt-informatik-aufgabe}
\liLadePakete{mathe}
\begin{document}
\liAufgabenTitel{Konfigurationsfolge Kellerautomat}
\section{Konfigurationsfolge von Kellerautomaten
\index{}
\footcite{theo:ab:2}}

Gegeben ist der folgende nichtdeterministische Kellerautomat mit

$P = (\{1, 2, 3, 4, \text{Final}\}, \{a, b\}, \{Z, A, B\}, \delta, 1, Z, \{\text{Final}\})$

\begin{enumerate}

\item Gebe für die folgenden Wörter, die in der Sprache enthalten sind,
eine Berechnung (Folge von Konfigurationen) des Kellerautomaten an:

\begin{enumerate}

%%
% (a)
%%

\item $w_1 = \texttt{bab}$

%%
% (b)
%%

\item $w_2 = \texttt{abb}$

%%
% (c)
%%

\item $w_3 = \texttt{abababbbb}$
\end{enumerate}

\item Charakterisiere die Wörter der Sprache in eigenen Worten.

\end{enumerate}
\end{document}
