\documentclass{lehramt-informatik-haupt}
\liLadePakete{formale-sprachen}

\begin{document}

%%%%%%%%%%%%%%%%%%%%%%%%%%%%%%%%%%%%%%%%%%%%%%%%%%%%%%%%%%%%%%%%%%%%%%%%
% Theorie-Teil
%%%%%%%%%%%%%%%%%%%%%%%%%%%%%%%%%%%%%%%%%%%%%%%%%%%%%%%%%%%%%%%%%%%%%%%%

\chapter{Kontextfreie Sprachen}

\section{Grammatik reguläre Sprachen}

Sei $\Sigma$ ein Alphabet. Eine formale Sprache $L$ ist eine Teilmenge
aller Wörter über $\Sigma$:

\begin{displaymath}
L \subseteq \Sigma^*
\end{displaymath}

\bigskip

\noindent
Eine Grammatik ist ein 4-Tupel mit \liGrammatik{} und besteht aus:

\begin{itemize}
\item Einer endlichen Menge $V$ von \memph{Variablen} (Nonterminale)

\item Dem endlichen \memph{Terminalalphabet} $\Sigma$ mit $\Sigma \cap V
= \emptyset$

\item Der endlichen Menge an \memph{Produktionen}

\item Und einer \memph{Startvariablen} $S$ mit $S \in V$
\footcite[Seite 7]{theo:fs:2}
\end{itemize}

%-----------------------------------------------------------------------
%
%-----------------------------------------------------------------------

\section{Grammatik kontextfreie Sprachen $\leftrightarrow$ kontextsensitiv}

Der Kontext ist das, was um die Variable der linken Seite „herum“
steht.

Beispiel für kontextsensitiv:

\begin{liProduktionsRegeln}
1A -> A11,
2A -> 02 | 121
\end{liProduktionsRegeln}

steht vor der Variablen A eine 1, dann
diese Regel...
steht vor dem A ein 2, dann diese Regel.

Es hängt vom Kontext ab, welche Regel zur Anwendung kommt.
(Auch AB -> BA also das Vertauschen ist kontextsensitiv)
\footcite[Seite 9]{theo:fs:2}

Eine kontextfreie Sprache wird durch eine kontextfreie Grammatik
erzeugt, d. \, h eine Grammatik mit Produktionsregeln der Form:

$A \rightarrow X, X \subset (V \cap \Sigma)^*$

linke Seite: ein Nonterminal

rechte Seite: $\epsilon$ oder ein Terminal oder eine
Kombination aus Terminalen mit
Nonterminalen

%-----------------------------------------------------------------------
%
%-----------------------------------------------------------------------

\section{Abschlusseigenschaften\footcite[Seite 77]{theo:fs:2}}

\begin{itemize}
\item Die kontextfreien Sprachen sind abgeschlossen unter

\begin{itemize}
\item Vereinigung
\item Produkt
\item Kleene-Stern
\end{itemize}

\item Die kontextfreien Sprachen sind nicht abgeschlossen unter

\begin{itemize}
\item Schnitt
\item Komplement
\end{itemize}

\item Eine Sprache, die nicht kontextfrei ist, ist insbesondere nicht
regulär!

\end{itemize}

%-----------------------------------------------------------------------
%
%-----------------------------------------------------------------------

\section{Entscheidungsprobleme\footcite[Seite 78]{theo:fs:2}}

\begin{itemize}
\item Für kontextfreie Sprachen ist das Äquivalenzproblem nicht
entscheidbar!

\item Für kontextfreie Sprachen ist entscheidbar:

\begin{itemize}
\item das Wortproblem
\item das Leerheitsproblem
\item das Endlichkeitsproblem
\end{itemize}
\end{itemize}

\literatur

\end{document}
