\documentclass{lehramt-informatik-haupt}
\liLadePakete{formale-sprachen}

\begin{document}

%%%%%%%%%%%%%%%%%%%%%%%%%%%%%%%%%%%%%%%%%%%%%%%%%%%%%%%%%%%%%%%%%%%%%%%%
% Theorie-Teil
%%%%%%%%%%%%%%%%%%%%%%%%%%%%%%%%%%%%%%%%%%%%%%%%%%%%%%%%%%%%%%%%%%%%%%%%

\chapter{Kontextfreie Sprachen}

\section{Grammatik reguläre Sprachen}

Sei $\Sigma$ ein Alphabet. Eine formale Sprache $L$ ist eine Teilmenge
aller Wörter über $\Sigma$:

\begin{displaymath}
L \subseteq \Sigma^*
\end{displaymath}

\bigskip

\noindent
Eine Grammatik ist ein 4-Tupel mit $G = (V, \Sigma, P, S)$ und besteht aus:

\begin{itemize}
\item Einer endlichen Menge $V$ von \memph{Variablen} (Nonterminale)

\item Dem endlichen \memph{Terminalalphabet} $\Sigma$ mit $\Sigma \cap V
= \emptyset$

\item Der endlichen Menge an \memph{Produktionen}

\item Und einer \memph{Startvariablen} $S$ mit $S \in V$
\footcite[Seite 7]{theo:fs:2}
\end{itemize}

%-----------------------------------------------------------------------
%
%-----------------------------------------------------------------------

\section{Grammatik kontextfreie Sprachen $\leftrightarrow$ kontextsensitiv}

Der Kontext ist das, was um die Variable der linken Seite „herum“
steht.

Beispiel für kontextsensitiv:

\begin{liProduktionsRegeln}
1A -> A11,
2A -> 02 | 121
\end{liProduktionsRegeln}

steht vor der Variablen A eine 1, dann
diese Regel...
steht vor dem A ein 2, dann diese Regel.

Es hängt vom Kontext ab, welche Regel zur Anwendung kommt.
(Auch AB -> BA also das Vertauschen ist kontextsensitiv)
\footcite[Seite 9]{theo:fs:2}

Eine kontextfreie Sprache wird durch eine kontextfreie Grammatik
erzeugt, d. \, h eine Grammatik mit Produktionsregeln der Form:

$A \rightarrow X, X \subset (V \cap \Sigma)^*$

linke Seite: ein Nonterminal

rechte Seite: $\epsilon$ oder ein Terminal oder eine
Kombination aus Terminalen mit
Nonterminalen

%-----------------------------------------------------------------------
%
%-----------------------------------------------------------------------

\section{Kellerautomat}

Ein Kellerautomat ist ein um einen „Speicher“ (Keller) erweiterter
endlicher Automat.

Ein nichtdeterministischer Kellerautomat $K$ (PDA = pushdown automaton)
ist ei n 6-Tupel $K = (Z, \Sigma, \Gamma, \delta, z_0 , \#$

\begin{description}
\item[Z] endliche Zustandsmenge
\item[$\Sigma$] Eingabealphabet
\item[$\Gamma$] Kelleralphabet mit \#
\item[$\delta$] Zustandsübergangsfunktion
\item[$z_0$] Startzustand
\item[\#] unterstes Kellersymbol\footcite[Seite 22]{theo:fs:2}
\end{description}

Der Kellerspeicher ist als \memph{Stapel (stack)} organisiert und
erlaubt daher nur einen eingeschränkten Zugriff auf die Elemente. Die
Funktionsweise ist ähnlich derer eines konventionellen Bücherstapels;
hier können wir ein neues Buch entweder oben auf den Stapel packen
(Push-Operation) oder das oberste Buch wegnehmen (Pop-Operation). In
einem Stack dürfen Elemente weder von unten noch aus der Mitte entnommen
werden, d. h., das zuletzt hinzugefügte Element wird immer als erstes
wieder entfernt (\memph{LIFO-Prinzip}, \memph{Last-In-First-Out}).

Kellerautomaten können analog zu endlichen Automaten durch
Zustandsgraphen veranschaulicht werden.

Die \memph{Zustandsübergänge} sind charakterisiert durch das
\memph{gelesene Zeichen}, das \memph{oberste Kellersymbol} und
\memph{die auf den Keller gelegten Zeichen, die das oberste Kellersymbol
ersetzen}.

\subsection{Beispiel}

$L = \{ a^n b^n | n \in \mathbb{Ν}$

Grammatik:
$G = (\{S \}, \{a, b\}, \{ S \rightarrow aSb | ab \}, S)$

Kellerautomat:
$K = (\{z_1, z_2 \}, \{ a, b \}, \{ A, \# \}, \delta, z_1, \#)$
\footcite[Seite 23]{theo:fs:2}
\literatur

\end{document}
