\documentclass{lehramt-informatik-haupt}
\liLadePakete{chomsky-normalform}

\begin{document}
\let\schrittE=\liChomskyUeberErklaerung

\chapter{CYK-Algorithmus}
\index{CYK-Algorithmus}

\begin{liQuellen}
\item \cite[Seite 45-75]{theo:fs:2}
\item \cite[Seite 186-188]{hoffmann}
\item \cite{wiki:cyk}
\end{liQuellen}

\section{Online-Tools}

\begin{itemize}
\item \url{https://www.xarg.org/tools/cyk-algorithm/}
\end{itemize}

Algorithmus nach Cocke, Younger, Kasami, um das Wortproblem für
eine Sprache zu entscheiden, die durch eine CNF-Grammatik
gegeben ist.
• Idee:
Bilde zu einem Wort w alle Teilwörter der Länge 1, 2, 3,... und
bestimme die Ableitbarkeit von Variablen.
Ist das „Teilwort“ der Länge $|w|$ vom Startsymbol ableitbar, so gehört
$w$ zur Sprache.

\literatur

\end{document}
