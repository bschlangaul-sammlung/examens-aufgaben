\documentclass{lehramt-informatik-haupt}
\liLadePakete{syntax}

\begin{document}

\chapter{CYK-Algorithmus}
\index{CYK-Algorithmus}

\begin{liQuellen}
\item \cite[Seite 45-75]{theo:fs:2}
\item \cite[Seite 186-188]{hoffmann}
\item \cite{wiki:cyk}
\end{liQuellen}

\section{Online-Tools}

\begin{itemize}
\item \url{https://www.xarg.org/tools/cyk-algorithm/}
\end{itemize}

\noindent
Der Algorithmus ist nach Cocke, Younger, Kasami benannt. Er dienst dazu
das \memph{Wortproblem} für eine Sprache zu entscheiden, die durch eine
CNF-Grammatik (\memph{Chomsky-Normalform}) gegeben ist.

Bilde zu einem Wort $w$ alle Teilwörter der Länge $1, 2, 3, \dots$ und
bestimme die Ableitbarkeit von Variablen. Ist das „Teilwort“ der Länge
$|w|$ vom Startsymbol ableitbar, so gehört $w$ zur Sprache.

\liJavaDatei{formale_sprachen/CYKAlgorithmus}

\literatur

\end{document}
