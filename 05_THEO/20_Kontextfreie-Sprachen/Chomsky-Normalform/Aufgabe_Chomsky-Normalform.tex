\documentclass{lehramt-informatik-aufgabe}
\liLadePakete{formale-sprachen,chomsky-normalform}
\begin{document}
\let\schrittE=\liChomskyUeberErklaerung

\liAufgabenTitel{Chomsky-Normalform}
\section{Chomsky-Normalform
\index{Chomsky-Normalform}
\footcite{theo:ab:2}}

Überführen Sie jeweils die angegebene kontextfreie Grammatik in
Chomsky-Normalform.

\begin{enumerate}

%-----------------------------------------------------------------------
%
%-----------------------------------------------------------------------

\item $G = (\{a, b, c\}, \{S, A, B, C, X\}, P, S)$ mit $P$:

\begin{liProduktionsRegeln}
S -> XAB | epsilon,
A -> aAB | AB | c,
B -> BB | C | a,
C -> CC | c | epsilon,
X -> A | b
\end{liProduktionsRegeln}

%-----------------------------------------------------------------------
%
%-----------------------------------------------------------------------

\item $G = (\{a, b, c\}, \{S, T\}, P, S)$ mit $P$:

\begin{liProduktionsRegeln}
S -> aSbS | T,
T -> cT | c
\end{liProduktionsRegeln}

\begin{liAntwort}
\begin{enumerate}
\item \schrittE{1}

\liNichtsZuTun

\item \schrittE{2}

\begin{liProduktionsRegeln}
S -> aSbS | cT | c,
T -> cT | c
\end{liProduktionsRegeln}

\item \schrittE{3}

\begin{liProduktionsRegeln}
S -> ASAS | CT | c,
T -> CT | c,
A -> a,
B -> b,
C -> c,
\end{liProduktionsRegeln}

\item \schrittE{4}

\begin{liProduktionsRegeln}
S -> AU | CT | c,
T -> CT | c,
A -> a,
B -> b,
C -> c,
U -> SV
V -> AS
\end{liProduktionsRegeln}

\end{enumerate}
\end{liAntwort}

%-----------------------------------------------------------------------
%
%-----------------------------------------------------------------------

\item $G = (\{a, b, c\}, \{S, A, B\}, P, S)$ mit $P$:

\begin{liProduktionsRegeln}
S -> AB,
A -> aAA | epsilon,
B -> bBB | epsilon
\end{liProduktionsRegeln}
\end{enumerate}

\end{document}
