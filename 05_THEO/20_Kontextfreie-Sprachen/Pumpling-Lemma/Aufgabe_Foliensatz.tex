\documentclass{lehramt-informatik-aufgabe}
\liLadePakete{}
\begin{document}
\liAufgabenTitel{Foliensatz}
\section{Pumping-Lemma
\index{Pumping-Lemma (Kontextfreie Sprache)}
\footcite[Seite 42]{theo:fs:2}}

$L = a^n b^n c^n \in \mathbb{N}$
Ich behaupte, L sei kontextfrei.

\begin{enumerate}
\item Also gibt es eine Pumpzahl. Sie sei j.
(Wähle geschickt ein „langes“ Wort...)
a j b j c j ist ein Wort aus L, das sicher länger als j ist.

\item  Da L kontextfrei ist, muss es nach dem Pumping-Lemma auch für dieses Wort
eine beliebige Zerlegung geben:
a j b j c j = uvwxy mit |vx| ≥ 1 und |vwx| ≤ j

\item  Weil vwx höchstens j lang ist, kann es nie a‘s und c‘s zugleich enthalten (es
stehen j b‘s dazwischen!).

\item Andererseits enthält vx mindestens ein Zeichen.
Das Wort ω = uv 0 wx 0 y = uwy enthält dann nicht mehr gleich viele a‘s, b‘s
und c‘s. (Widerspruch)!

\item Die Behauptung war falsch! => L ist nicht kontextfrei!

\end{enumerate}
\end{document}
