\documentclass{lehramt-informatik-aufgabe}
\liLadePakete{formale-sprachen}
\begin{document}
\liAufgabenTitel{Vorlesungsaufgaben}
\section{Pumping-Lemma
\index{Pumping-Lemma (Kontextfreie Sprache)}
\footcite[Seite 43]{theo:fs:2}}

Zeige, dass die folgenden Sprache nicht kontextfrei sind:

\begin{itemize}
% 2h02
\item $L = \{ a^n b^n c^{2n} | n \in \mathbb{N} \}$

\begin{liAntwort}
Annahme: $L$ ist kontextfrei.

$\forall \omega \in L$: $\omega = / uvwxy$

$j \in \mathbb{N}$: $|\omega| \geq j$

$\omega = a^j b^j c^{2j}$: $|\omega| = 4j > j$

Damit gilt: $|vwx| \leq j$, $|vx| \geq 1$

Zu zeigen: Keine Möglichkeit der Zerlegung, damit $\omega' \in L$

\begin{description}

%%
%
%%

\item[1. Fall]

$vwv$ enthält nur $a$'s

o. E. d. A. (ohne Einschränkung der Allgemeinheit) stecken alle $a$'s in
der Zerlegung $vwx$, d. h. $u$ ist leer

$u$: $\epsilon$
$v$: $a^l$
$w$: $a^{j-(l+m)}$
$x$: $a^m$
$y$: $b\dots b c \dots c$

$v^2 w x^2 y$

$a^{2l} a^{j-(l+m)} a^{2m} b^j c^{2j} =   $

Nebenrechnung: $2l + j - (l + m) + 2m = j + l + m > j$,
da $|vx| \geq 1 \rightarrow l + m \geq 1$

$\Rightarrow$ $\omega' = uv^2wx^2y \notin L$

%%
%
%%

\item[2. Fall]

$vwv$ enthalten $a$'s und $b$'s

o. E. d. A. $|v|_a = |x|_b$

$u$: $a^p$
$v$: $a^l$
$w$: $a^{j-(p+l)} b^{j-(l+r)}$
$x$: $b^l$
$y$: $b^r c^{2j}$

$\Rightarrow$ $u v^0 w x^0 v$

Nebenrechnung:

$a$'s: $p + j - (l + p ) = j - l$

$b$'s: $j - (l + r) = j - l$

ist falsch, da $j-l$ echt kleiner ist, da $|vx| \geq 1 \rightarrow l \geq 1$

$\Rightarrow$ $\omega' \notin L$

%%
%
%%

\item[3. Fall]

$vwx$ enthält nur $b$'s

analog zu Fall 1

%%
%
%%

\item[4. Fall] $vwx$ enthält nur $b$'s und $c$'s

analog zu Fall 2

%%
%
%%

\item[5. Fall] $vwx$ enthält nur  $c$'s

analog zu Fall 1
\end{description}

$\Rightarrow$ Es gibt keine Zerlegung, sodass $\forall i \in \mathbb{N}_0$

$\Rightarrow$ Annahme ist falsch

$\Rightarrow$ $L$ ist nicht kontextfrei
\end{liAntwort}

\item $L = \{ a^n b^{n^2} | n \in N \}$
\end{itemize}

\end{document}
