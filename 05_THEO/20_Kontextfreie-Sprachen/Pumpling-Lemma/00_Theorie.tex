\documentclass{lehramt-informatik-haupt}
\liLadePakete{formale-sprachen}

\begin{document}

%%%%%%%%%%%%%%%%%%%%%%%%%%%%%%%%%%%%%%%%%%%%%%%%%%%%%%%%%%%%%%%%%%%%%%%%
% Theorie-Teil
%%%%%%%%%%%%%%%%%%%%%%%%%%%%%%%%%%%%%%%%%%%%%%%%%%%%%%%%%%%%%%%%%%%%%%%%

% Info_2021-03-26-2021-03-26_09.29.40 1h46

\chapter{Pumping Lemma für die kontextfreie Sprachen}

Sei $L$ eine kontextfreie Sprache. Dann gibt es eine Zahl $j$, so dass
sich alle Wörter $\omega \in L$ mit $|\omega| \geq j$ zerlegen lassen in
$\omega = uvwxy$, so dass die folgenden Eigenschaften erfüllt sind:

\begin{enumerate}
\item $|vx| \geq 1$
\item $|vwx| \leq j$
\item Für alle $i \in \mathbb{N}_0$ gilt $u v^i w x^i y \in L$
\end{enumerate}

Das Pumping-Lemma dient zum Nachweis, dass eine Sprache nicht
kontextfrei ist.
(Widerspruchsbeweis!)

\literatur

\end{document}
