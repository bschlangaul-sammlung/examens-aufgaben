\documentclass{lehramt-informatik-aufgabe}
\liLadePakete{formale-sprachen,syntaxbaum,automaten}
\begin{document}
\liAufgabenTitel{Vorlesungsaufgabe}
\section{Kontextfreie Grammatik
\index{Kontextfreie Sprache}
\footcite{theo:fs:2}}

\begin{enumerate}
\item Erstellen Sie eine Ableitung für die Wörter der Sprache zur
vorgegeben Grammatik\footcite[Seite 10]{theo:fs:2}
\index{Ableitung (Kontextfreie Sprache)}

$V = \{ S, A, B \}$

\liAlphabet{0, 1}

\begin{liProduktionsRegeln}
S -> A1B,
A -> 0A | epsilon,
B -> 0B | 1B | epsilon
\end{liProduktionsRegeln}

$S = S$
\liFussnoteUrl{https://flaci.com/Gi1rgpemg}

\begin{itemize}
\item 00101

\begin{liAntwort}
\liAbleitung{
S ->
A1B ->
0A1B ->
00A1B ->
001B ->
0010B ->
00101B ->
00101
}
\end{liAntwort}

\item 1001

\begin{liAntwort}
\liAbleitung{
S ->
A1B ->
1B ->
10B ->
100B ->
1001B ->
1001
}
\end{liAntwort}
\end{itemize}

\item Erstelle eine kontextfreie Grammatik, die alle Wörter mit gleich
vielen $1$‘s, gefolgt von gleich vielen $0$‘s enthält.
\index{Kontextfreie Grammatik}

\begin{liAntwort}
\begin{liProduktionsRegeln}
S -> 1S0 | epsilon
\end{liProduktionsRegeln}
\liFussnoteUrl{https://flaci.com/Grxmyw2ia}
\end{liAntwort}

\item Erstelle eine kontextfreie Grammatik, die alle regulären Ausdrücke
über den Zeichen $0,1$ darstellt.
(Beispiel: \texttt{01*(1+0)0} für einen möglichen regulären Ausdruck
(Das \texttt{+}-Zeichen ist hier anstelle des Oder-Zeichens (|)))

\begin{liAntwort}
\liAlphabet{1; 0; (; ); +; *}

\begin{liProduktionsRegeln}
S -> epsilon | 0 | 1 | S* | (S) | SS | S+S
\end{liProduktionsRegeln}
\liFussnoteUrl{https://flaci.com/Ghfgrv027}
\end{liAntwort}

\end{enumerate}

\end{document}
