\documentclass{lehramt-informatik-aufgabe}
\liLadePakete{formale-sprachen,automaten}
\begin{document}
\liAufgabenTitel{Balancierte Klammern}
\section{Balancierte Klammern
\index{Kellerautomat}
}

Erstellen Sie einen Kellerautomaten der nur balancierte Klammerausdrücke
mit eckigen Klammern akzeptiert.

\begin{liAntwort}
\begin{center}
\begin{tikzpicture}[li kellerautomat]
  \node[state,initial] (q0) at (2.14cm,-2.14cm) {$q_0$};
  \node[state] (q1) at (7.14cm,-2.14cm) {$q_1$};
  \node[state,accepting] (q2) at (9.71cm,-2.14cm) {$q_2$};
  \node[state,accepting] (q3) at (2.57cm,-4.71cm) {$q_3$};

  \liKellerKante[above,bend left]{q0}{q1}{
    EPSILON, K, K
  }

  \liKellerKante[above,loop]{q0}{q0}{
    [, KELLERBODEN, KKELLERBODEN;
    [, K, KK
  }

  \liKellerKante[left]{q0}{q3}{
    EPSILON, KELLERBODEN, EPSILON
  }

  \liKellerKante[above,bend left]{q1}{q0}{
    EPSILON, KELLERBODEN, KELLERBODEN
  }

  \liKellerKante[above]{q1}{q2}{
    EPSILON, KELLERBODEN, EPSILON
  }

  \liKellerKante[above,loop]{q1}{q1}{
    ], K, EPSILON
  }

\end{tikzpicture}
\end{center}
\liFussnoteUrl[Seite 9]{https://eecs.wsu.edu/~ananth/CptS317/Lectures/PDA.pdf}

\liFlaci{Apwobf482}
\end{liAntwort}

\end{document}
