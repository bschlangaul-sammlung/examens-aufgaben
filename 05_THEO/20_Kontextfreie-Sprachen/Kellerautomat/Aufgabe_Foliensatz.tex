\documentclass{lehramt-informatik-aufgabe}
\liLadePakete{formale-sprachen,automaten,mathe}
\begin{document}
\let\m=\liMengeOhneMathe
\def\z#1{z_#1}
\def\Z#1{$z_#1$}
\let\u=\liKellerUebergang

\liAufgabenTitel{Foliensatz}
\section{Kellerautomat
\index{Kellerautomat}
\footcite[Seite 23]{theo:fs:2}}

Beispiel: $L = \m{a^n b^n | n \in \mathbb{N}}$

\subsection{Grammatik: }

$G = (\m{S}, \m{a, b}, P, S)$

\bigskip

\noindent
\begin{liProduktionsRegeln}
S -> aSb | ab
\end{liProduktionsRegeln}

\subsection{Kellerautomat: }

\begin{center}
$K = (\m{\z1, \z2}, \m{a, b}, \m{A, \#}, \delta, \z1 ,\#)$
\end{center}

\begin{tikzpicture}[li kellerautomat,node distance=3cm]
\node[state,initial] (1) {$z_1$};
\node[state,right of=1] (2) {$z_2$};

\path (1) edge[above,loop] node{
  \u{a raute Araute}\\
  \u{a A AA}\\
} (1);

\path (2) edge[above,loop] node{
  \u{b A epsilon}\\
  \u{epsilon raute epsilon}\\
} (2);

\path (1) edge[above] node{
  \u{b A epsilon}\\
} (2);
\end{tikzpicture}

\begin{tabular}{|l|l|l|l|l|}
Aktueller Zustand &  Eingabe   & Keller & Folgezustand & Keller \\\hline
\Z1               & a          & \#     & \Z1          & A\# \\
\Z1               & a          & A      & \Z1          & AA \\
\Z1               & b          & A      & \Z2          & $\epsilon$ \\
\Z2               & b          & A      & \Z2          & $\epsilon$ \\
\Z2               & $\epsilon$ & \#     & \Z2          & \# \\
\end{tabular}

\end{document}
