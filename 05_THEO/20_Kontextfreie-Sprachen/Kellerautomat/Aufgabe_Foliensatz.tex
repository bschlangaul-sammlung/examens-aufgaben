\documentclass{lehramt-informatik-aufgabe}
\liLadePakete{formale-sprachen,automaten}
\begin{document}
\let\m=\liMengeOhneMathe
\def\z#1{z_#1}
\def\Z#1{$z_#1$}
\let\u=\liKellerUebergang

\liAufgabenTitel{Foliensatz}
\section{Kellerautomat
\index{Kellerautomat}
\footcite[Seite 23]{theo:fs:2}}

Beispiel: $L = \m{a^n b^n | n \in Ν}$

Grammatik: $G = (\m{S}, \m{a, b}, P, S)$

\begin{liProduktionsRegeln}
S -> aSb | ab
\end{liProduktionsRegeln}

Kellerautomat: $K = (\m{\z1, \z2}, \m{a, b}, \m{A, \#}, \delta, \z1 ,\#)$

\begin{tikzpicture}[li kellerautomat,node distance=5cm]
\node[state,initial] (0) {$z_0$};
\node[state,above right of=0] (1) {$z_1$};
\node[state,right of=0] (2) {$z_2$};

\path (0) edge[below,loop below] node{
  \u{a raute Araute}\\
  \u{a A AA}\\
} (0);

\end{tikzpicture}

\end{document}
