\documentclass{lehramt-informatik-aufgabe}
\liLadePakete{formale-sprachen,automaten}
\begin{document}
\let\m=\liMenge
\let\u=\liKellerUebergang
\let\e=\liEpsilon

\liAufgabenTitel{Vorlesungsaufgabe}
\section{Kellerautomaten
\index{Kontextfreie Sprache}
\footcite[Seite 25]{theo:fs:2}}

\begin{enumerate}
\item Gib einen Kellerautomaten an, der die folgende Sprache erkennt:
\index{Kellerautomat}

$L = (a^n c^i b^n | n, i \in N_0)$

\begin{liAntwort}
\begin{center}
\begin{tikzpicture}[li kellerautomat,node distance=5cm]
\node[state,initial] (0) {$z_0$};
\node[state,above right of=0] (1) {$z_1$};
\node[state,right of=0] (2) {$z_2$};

\path (0) edge[below,loop below] node{
  \u{a raute Araute}
  \u{a A AA}
} (0);

\path (0) edge[below] node{
  \u{epsilon raute epsilon}
  \u{b A epsilon}
} (2);

\path (2) edge[below,loop below] node{
  \u{b A epsilon}
  \u{epsilon raute epsilon}
} (2);

\path (1) edge[below] node{
  \u{epsilon raute epsilon}
  \u{b A epsilon}
} (2);

\path (1) edge[above,loop] node{
  \u{c A A}
  \u{c raute raute}
} (1);

\path (0) edge[below] node{
  \u{c raute raute}
  \u{a A AA}
} (1);

\end{tikzpicture}
\end{center}

Tabellenform:

\begin{tabular}{llllll}
z0 & a & \# & z0 & A\#\\
z0 & a & A & z0 & AA\\
z0 & b & A & z2 & A & \e\\
z0 & a \# & z0 & A\#\\
z0 & a & \# & z0 & A\#\\
z0 & a & \# & z0 & A\#\\

z1 & c & \# & z1 & A\#\\
z1 & c & A & z2 & A\e\\
z1 & c & A & z2 & A\e\\
z1 & \e & \# & z2 & \e\\

z2 & b & A & z2 & A\e\\
z2 & \e & \# & z2 & \e\\
\end{tabular}
\end{liAntwort}

\item Gibt eine Grammatik für diese Sprache an.

\begin{liAntwort}
\begin{liProduktionsRegeln}
S -> aSb | epsilon | c | cC,
C -> cC | epsilon
\end{liProduktionsRegeln}

alternativ:

\begin{liProduktionsRegeln}
S -> aSb | epsilon | C,
C -> cC | epsilon
\end{liProduktionsRegeln}
\end{liAntwort}

\item Gib Konfigurationsfolgen an für die Erzeugung des Wortes

\begin{itemize}
\item aacbb

\begin{liAntwort}
\begin{tabular}{llr}
a: & z0, a,\# -> zo A\#                   & A\#\\
c. & z0, c,A -> z1 A                      & A\#\\
c: & z1, c, A -> z1, A                    & A\#\\
b: & z1, b, A -> z2, epsilon              & \#\\
epsilon: & z2, epsilon, \# -> z2, epsilon & -\\
\end{tabular}{llr}
\end{liAntwort}

\item accb
\end{itemize}

\end{enumerate}

\end{document}
