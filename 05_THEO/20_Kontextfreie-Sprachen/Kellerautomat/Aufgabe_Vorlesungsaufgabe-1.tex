\documentclass{lehramt-informatik-aufgabe}
\liLadePakete{formale-sprachen,automaten}
\begin{document}
\let\m=\liMenge
\let\u=\liKellerUebergang
\let\e=\liEpsilon
\let\z=\liZustandsnameTiefgestellt
\def\p{$\vdash$ }

\liAufgabenTitel{Vorlesungsaufgabe}
\section{Kellerautomaten
\index{Kontextfreie Sprache}
\footcite[Seite 25]{theo:fs:2}}

\begin{enumerate}
\item Gib einen Kellerautomaten an, der die folgende Sprache erkennt:
\index{Kellerautomat}

\begin{displaymath}
L = \{ a^n c^i b^n | n, i \in \mathbb{N}_0 \}
\end{displaymath}

\begin{liAntwort}
\begin{center}
\begin{tikzpicture}[li kellerautomat,node distance=5cm]
\node[state,initial] (0) {\z0};
\node[state,above right of=0] (1) {\z1};
\node[state,right of=0,accepting] (2) {\z2};

\path (0) edge[below,loop below] node{
  \u{a raute Araute}
  \u{a A AA}
} (0);

\path (0) edge[below] node{
  \u{epsilon raute epsilon}
  \u{b A epsilon}
} (2);

\path (0) edge[left] node{
  \u{c raute raute}
  \u{c A A}
} (1);

\path (2) edge[below,loop below] node{
  \u{epsilon raute epsilon}
  \u{b A epsilon}
} (2);

\path (1) edge[right] node{
  \u{epsilon raute epsilon}
  \u{b A epsilon}
} (2);

\path (1) edge[above,loop] node{
  \u{c raute raute}
  \u{c A A}
} (1);

\end{tikzpicture}
\liFussnoteUrl{https://flaci.com/Apky9znog}

% akzeptiert
% c
% ab
% aabb
% aacbb

% akzeptiert nicht
% a
% b
% aab

\end{center}

Tabellenform:

\begin{tabular}{|l|l|l|l|l|l|}
\hline
Aktueller Zustand &  Eingabe   & Keller & Folgezustand & Keller \\\hline\hline
\z0 & a  & \# & \z0 & A\# \\
\z0 & a  & A  & \z0 & AA  \\\hline

\z0 & c  & \# & \z1 & \#  \\
\z0 & c  & A  & \z1 & A   \\\hline

\z0 & \e & \# & \z2 & \e  \\
\z0 & b  & A  & \z2 & \e  \\\hline

\z1 & c  & \# & \z1 & \#  \\
\z1 & c  & A  & \z1 & A   \\\hline

\z1 & \e & \# & \z2 & \e  \\
\z1 & b  & A  & \z2 & \e  \\\hline

\z2 & \e & \# & \z2 & \e  \\
\z2 & b  & A  & \z2 & \e  \\\hline
\end{tabular}
\end{liAntwort}

\item Geben Sie eine Grammatik für diese Sprache an.
\index{Kontextfreie Grammatik}

\begin{liAntwort}
\begin{liProduktionsRegeln}
S -> aSb | epsilon | c | cC,
C -> cC | epsilon
\end{liProduktionsRegeln}

alternativ:

\begin{liProduktionsRegeln}
S -> aSb | epsilon | C,
C -> cC | epsilon
\end{liProduktionsRegeln}
\end{liAntwort}

\item Geben Sie Konfigurationsfolgen für die Erzeugung des Wortes an
\index{Konfigurationsfolge}

\begin{itemize}
\item aacbb

\begin{liAntwort}
(\z0, aacbb, \#)   \p
(\z0, acbb,  A\#)  \p
(\z0, cbb,   AA\#) \p
(\z1, bb,    AA\#) \p
(\z2, b,     A\#)  \p
(\z2, \e,    \#)   \p
(\z2, \e,   \e)
\end{liAntwort}

\item accb

\begin{liAntwort}
(\z0, accb, \#)  \p
(\z0, ccb,  A\#) \p
(\z1, cb,   A\#) \p
(\z2, b,    A\#) \p
(\z2, \e,   \#)  \p
(\z2, \e,   \e)
\end{liAntwort}
\end{itemize}
\end{enumerate}
\end{document}
