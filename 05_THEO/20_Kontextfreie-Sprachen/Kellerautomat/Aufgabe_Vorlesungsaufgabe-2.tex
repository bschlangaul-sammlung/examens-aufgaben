\documentclass{lehramt-informatik-aufgabe}
\liLadePakete{formale-sprachen,automaten}
\begin{document}
\let\m=\liMenge

\liAufgabenTitel{Vorlesungsaufgabe}
\section{Kontextfreie Sprache
\index{Kontextfreie Sprache}
\footcite{theo:fs:2}}

\section{Kellerautomaten\footcite[Seite 27]{theo:fs:2}}

Erstellen Sie einen Kellerautomaten zu der Grammatik $G = (\m{P} , \m{0, 1}, P, S)$ mit den folgenden Produktionsregeln
\begin{enumerate}
\item

\begin{liProduktionsRegeln}
S -> epsilon | 0 | 1 | 0P0 | 1P1,
\end{liProduktionsRegeln}

\item

\begin{liProduktionsRegeln}
S -> A1B,
A -> 0A | epsilon,
B -> 0B | 1B | epsilon,
\end{liProduktionsRegeln}
\end{enumerate}

\end{document}
