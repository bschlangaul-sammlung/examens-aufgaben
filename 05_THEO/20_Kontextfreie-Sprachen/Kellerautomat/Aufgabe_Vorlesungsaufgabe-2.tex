\documentclass{lehramt-informatik-aufgabe}
\liLadePakete{formale-sprachen,automaten}
\begin{document}
\let\m=\liMenge
\let\z=\liZustandsnameTiefgestellt
\let\u=\liKellerUebergang

\liAufgabenTitel{Vorlesungsaufgabe}

\section{Kellerautomaten\footcite[Seite 27]{theo:fs:2}}

Erstellen Sie einen Kellerautomaten zu der Grammatik $G = (\m{P} , \m{0,
1}, P, S)$ mit den folgenden Produktionsregeln
\index{Kellerautomat}

\begin{enumerate}

%%
%
%%

\item

\begin{liProduktionsRegeln}
S -> epsilon | 0 | 1 | 0 P 0 | 1 P 1,
\end{liProduktionsRegeln}

\begin{liAntwort}
$K = (\m{\z0, \z1}, \m{0, 1}, \m{\#, N, E}, \delta, z_0, \#)$

N = Null

E = Eins

\begin{center}
\begin{tikzpicture}[li kellerautomat,node distance=5cm]
\node[state,initial] (0) {\z0};
\node[state,right of=0,accepting] (1) {\z1};

\path (0) edge[above,loop] node{\u{
1 N NE,
1 E EE,
1 raute Eraute,
0 E NE,
0 N NN,
0 raute Nraute,
}} (0);

\path (0) edge[above] node{\u{
0 N N,
0 E E,
1 E E,
0 N N,
epsilon E E,
epsilon N N,
epsilon raute epsilon,
}} (1);

\path (1) edge[above,loop] node{\u{
1 E epsilon,
0 N epsilon,
}} (1);
\end{tikzpicture}
\end{center}

\liFussnoteUrl{https://flaci.com/Ahij8jnn7}
\end{liAntwort}

%%
%
%%

\item

\begin{liProduktionsRegeln}
S -> A1B,
A -> 0A | epsilon,
B -> 0B | 1B | epsilon,
\end{liProduktionsRegeln}
\end{enumerate}

\begin{liAntwort}
$K = (\m{\z0, \z1}, \m{0, 1}, \m{\#, E}, \delta, z_0, \#)$

E = Eins ist gesetzt

\begin{center}
\begin{tikzpicture}[li kellerautomat,node distance=5cm]
\node[state,initial] (0) {\z0};
\node[state,right of=0,accepting] (1) {\z1};

\path (0) edge[above,loop] node{\u{
1 raute Eraute,
0 raute raute,
}} (0);

\path (0) edge[above] node{\u{
epsilon E epsilon,
1 E epsilon,
0 E epsilon,
}} (1);

\path (1) edge[above,loop] node{\u{
epsilon raute epsilon,
1 raute raute,
0 raute raute,
}} (1);
\end{tikzpicture}
\end{center}

\liFussnoteUrl{https://flaci.com/Ar3imp8a7}
\end{liAntwort}

\end{document}
