\documentclass{lehramt-informatik-aufgabe}
\liLadePakete{formale-sprachen,automaten}
\begin{document}
\let\m=\liMenge

\liAufgabenTitel{Vorlesungsaufgabe}

\section{Kellerautomaten\footcite[Seite 27]{theo:fs:2}}

Erstellen Sie einen Kellerautomaten zu der Grammatik $G = (\m{P} , \m{0,
1}, P, S)$ mit den folgenden Produktionsregeln
\index{Kellerautomat}

\begin{enumerate}

%%
%
%%

\item

\begin{liProduktionsRegeln}
S -> epsilon | 0 | 1 | 0P0 | 1P1,
\end{liProduktionsRegeln}

\begin{liAntwort}
\liFussnoteUrl{https://flaci.com/Ahij8jnn7}
\end{liAntwort}

%%
%
%%

\item

\begin{liProduktionsRegeln}
S -> A1B,
A -> 0A | epsilon,
B -> 0B | 1B | epsilon,
\end{liProduktionsRegeln}
\end{enumerate}

\begin{liAntwort}
\liFussnoteUrl{https://flaci.com/Ar3imp8a7}
\end{liAntwort}

\end{document}
