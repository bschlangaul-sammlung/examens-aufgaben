\documentclass{lehramt-informatik-aufgabe}
\liLadePakete{formale-sprachen}
\begin{document}
\liAufgabenTitel{Kellerautomat}
\section{Grammatik in Kellerautomat umwandeln
\index{Kellerautomat}
\footcite{theo:ab:2}}

Gebe für die folgenden Grammatiken $G_i$ jeweils einen Kellerautomaten
$P_i$ an, der dieselbe Sprache besitzt wie die Grammatik: $L(G_i) =
L(P_i)$

\begin{enumerate}

%%
% (a)
%%

\item

\begin{liProduktionsRegeln}
S -> 0S1 | P,
P -> 1P 0 | S | epsilon
\end{liProduktionsRegeln}

%%
% (b)
%%

\item

\begin{liProduktionsRegeln}
S -> xTT,
T -> xS | yS | x
\end{liProduktionsRegeln}

%%
% (c)
%%

\item

\begin{liProduktionsRegeln}
S -> aB | bA | ABc | B,
A -> SSa,
B -> cS | bB | b
\end{liProduktionsRegeln}
\end{enumerate}

\end{document}
