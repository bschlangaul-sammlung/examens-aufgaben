\documentclass{lehramt-informatik-aufgabe}
\liLadePakete{formale-sprachen,automaten}
\begin{document}
\liAufgabenTitel{Kellerautomat}
\section{Grammatik in Kellerautomat umwandeln
\index{Kellerautomat}
\footcite[Aufgabe 5: Grammatik in Kellerautomat umwandeln]{theo:ab:2}}

Gebe für die folgenden Grammatiken $G_i$ jeweils einen Kellerautomaten
$P_i$ an, der dieselbe Sprache besitzt wie die Grammatik: $L(G_i) =
L(P_i)$

\begin{enumerate}

%%
% (a)
%%

\item

\begin{liProduktionsRegeln}[P_1]
S -> 0 S 1 | P,
P -> 1 P 0 | S | EPSILON
\end{liProduktionsRegeln}

\begin{liAntwort}
\begin{center}
\begin{tikzpicture}[li kellerautomat]
  \node[state,initial] (q0) at (3.14cm,-4.71cm) {$q_0$};
  \node[state,accepting] (q1) at (5.71cm,-4.71cm) {$q_1$};

  \liKellerKante[above]{q0}{q1}{
    EPSILON, KELLERBODEN, EPSILON
  }

  \liKellerKante[above,loop]{q0}{q0}{
    EPSILON, KELLERBODEN, S KELLERBODEN;
    EPSILON, S, 0S1;
    EPSILON, S, P;
    EPSILON, P, 1P0;
    EPSILON, P, S;
    EPSILON, P, EPSILON;
    0, 0, EPSILON;
    1, 1, EPSILON
  }

\end{tikzpicture}
\end{center}

\liFlaci{Ah5ceyrrz}
\end{liAntwort}

%%
% (b)
%%

\item

\begin{liProduktionsRegeln}[P_2]
S -> x T T,
T -> x S | y S | x
\end{liProduktionsRegeln}

%%
% (c)
%%

\item

\begin{liProduktionsRegeln}[P_3]
S -> a B | b A | A B c | B,
A -> S S a,
B -> c S | b B | b
\end{liProduktionsRegeln}
\end{enumerate}

\end{document}
