\documentclass{lehramt-informatik-aufgabe}
\liLadePakete{formale-sprachen,automaten}
\begin{document}
\liAufgabenTitel{Kellerautomat}
\section{Grammatik in Kellerautomat umwandeln
\index{Kellerautomat}
\footcite[Aufgabe 5: Grammatik in Kellerautomat umwandeln]{theo:ab:2}}

Gebe für die folgenden Grammatiken $G_i$ jeweils einen Kellerautomaten
$P_i$ an, der dieselbe Sprache besitzt wie die Grammatik: $L(G_i) =
L(P_i)$

\begin{enumerate}

%%
% (a)
%%

\item

\begin{liProduktionsRegeln}[P_1]
S -> 0 S 1 | P,
P -> 1 P 0 | S | EPSILON
\end{liProduktionsRegeln}

\begin{liAntwort}
\liKellerAutomat{
  zustaende={z_0, z_1},
  alphabet={0, 1},
  kelleralphabet={\#, S, P, 0, 1},
  ende={z_1},
}

\begin{center}
\begin{tikzpicture}[li kellerautomat]
  \node[state,initial] (z0) at (3.14cm,-4.71cm) {$z_0$};
  \node[state,accepting] (z1) at (5.71cm,-4.71cm) {$z_1$};

  \liKellerKante[above]{z0}{z1}{
    EPSILON, KELLERBODEN, EPSILON;
  }

  \liKellerKante[above,loop]{z0}{z0}{
    EPSILON, KELLERBODEN, S KELLERBODEN;
    EPSILON, S, 0 S 1;
    EPSILON, S, P;
    EPSILON, P, 1 P 0;
    EPSILON, P, S;
    EPSILON, P, EPSILON;
    0, 0, EPSILON;
    1, 1, EPSILON;
  }
\end{tikzpicture}
\end{center}

\liFlaci{Ah5ceyrrz}
\end{liAntwort}

%%
% (b)
%%

\item

\begin{liProduktionsRegeln}[P_2]
S -> x T T,
T -> x S | y S | x
\end{liProduktionsRegeln}

\begin{liAntwort}
\liKellerAutomat{
  zustaende={z_0, z_1},
  alphabet={x, y},
  kelleralphabet={\#, T, S, x, y},
  ende={z_1},
}

\begin{center}
\begin{tikzpicture}[li kellerautomat]
  \node[state,initial] (z0) at (3.86cm,-3.86cm) {$z_0$};
  \node[state,accepting] (z1) at (7.14cm,-3.86cm) {$z_1$};

  \liKellerKante[above]{z0}{z1}{
    EPSILON, KELLERBODEN, EPSILON;
  }

  \liKellerKante[above,loop]{z0}{z0}{
    x, x, EPSILON;
    y, y, EPSILON;
    EPSILON, S, x T T;
    EPSILON, T, x S;
    EPSILON, T, y S;
    EPSILON, T, x;
    EPSILON, KELLERBODEN, S KELLERBODEN;
  }
\end{tikzpicture}
\end{center}

\liFlaci{Aiq4r0162}
\end{liAntwort}

%%
% (c)
%%

\item

\begin{liProduktionsRegeln}[P_3]
S -> a B | b A | A B c | B,
A -> S S a,
B -> c S | b B | b
\end{liProduktionsRegeln}

\begin{liAntwort}
\liKellerAutomat{
  zustaende={z_0, z_1},
  alphabet={a, b},
  kelleralphabet={\#, A, B, a, b},
  ende={z_1},
}

\begin{center}
\begin{tikzpicture}[li kellerautomat]
  \node[state,initial] (z0) at (5.29cm,-5cm) {$z_0$};
  \node[state,accepting] (z1) at (8.14cm,-5cm) {$z_1$};

  \liKellerKante[above]{z0}{z1}{
    EPSILON, KELLERBODEN, EPSILON;
  }

  \liKellerKante[above,loop]{z0}{z0}{
    a, a, EPSILON;
    b, b, EPSILON;
    c, c, EPSILON;
    EPSILON, S, a B;
    EPSILON, S, b A;
    EPSILON, S, A B c;
    EPSILON, S, B;
    EPSILON, A, S S a;
    EPSILON, B, c S;
    EPSILON, B, b B;
    EPSILON, B, b;
    EPSILON, KELLERBODEN, S KELLERBODEN;
  }
\end{tikzpicture}
\end{center}

\liFlaci{Ajh5y0s5r}
\end{liAntwort}
\end{enumerate}

\end{document}
