\documentclass{lehramt-informatik-aufgabe}
\liLadePakete{mathe,automaten,formale-sprachen}
\begin{document}
\let\m=\liMengeOhneMathe
\let\u=\liKellerUebergang

\liAufgabenTitel{Konfigurationsfolge Kellerautomat}
\section{Konfigurationsfolge von Kellerautomaten
\index{Kellerautomat}
\footcite{theo:ab:2}}

Gegeben ist der folgende nichtdeterministische Kellerautomat mit

$P = (\{1, 2, 3, 4, \text{Final}\}, \m{a, b}, \m{Z, A, B}, \delta, 1, Z, \m{\text{Final}})$

\begin{center}
\begin{tikzpicture}[li kellerautomat,node distance=4cm]
\node[state,initial] (1) {1};
\node[state,above right of=1] (2) {2};
\node[state,right of=2] (3) {3};
\node[state,below right of=3] (4) {4};
\node[state,below right of=1] (F) {Final};

\path (1) edge[above left,bend left] node{\u{a Z AAZ}} (2);
\path (2) edge[below right,bend left] node{\u{raute Z Z}} (1);
\end{tikzpicture}
\end{center}

\begin{enumerate}

\item Gebe für die folgenden Wörter, die in der Sprache enthalten sind,
eine Berechnung (Folge von Konfigurationen) des Kellerautomaten an:

\begin{enumerate}

%%
% (a)
%%

\item $w_1 = \texttt{bab}$

%%
% (b)
%%

\item $w_2 = \texttt{abb}$

%%
% (c)
%%

\item $w_3 = \texttt{abababbbb}$
\end{enumerate}

\item Charakterisiere die Wörter der Sprache in eigenen Worten.

\end{enumerate}
\end{document}
