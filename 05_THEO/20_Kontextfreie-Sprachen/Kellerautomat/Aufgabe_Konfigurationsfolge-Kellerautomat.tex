\documentclass{lehramt-informatik-aufgabe}
\liLadePakete{mathe,automaten,formale-sprachen,mathe}
\begin{document}
\let\m=\liMengeOhneMathe
\let\u=\liKellerUebergang
\let\z=\liZustandsnameTiefgestellt

\liAufgabenTitel{Konfigurationsfolge Kellerautomat}
\section{Konfigurationsfolge von Kellerautomaten
\index{Kellerautomat}
\footcite{theo:ab:2}}

Gegeben ist der folgende nichtdeterministische Kellerautomat mit

\begin{displaymath}
P = (
\{1, 2, 3, 4, \text{Final}\},
\m{a, b},
\m{\#, A, B},
\delta,
\z1,
\#,
\m{\z{Final}}
)
\end{displaymath}

\begin{center}
\begin{tikzpicture}[li kellerautomat,node distance=4cm]
\node[state,initial] (1) {\z1};
\node[state,above right of=1] (2) {\z2};
\node[state,right of=2] (3) {\z3};
\node[state,below left of=3] (4) {\z4};
\node[state,below right of=1,accepting] (F) {\z{Final}};

\path (1) edge[bend left=10]
node[above left=-0.3cm]{\u{a raute AAraute}} (2);

\path (1) edge[above,bend left=10]
node{\u{b raute Braute}} (4);

\path (1) edge
node[below left=-0.3cm]{\u{epsilon raute epsilon}} (F);

\path (2) edge[bend left=10]
node[below right=-0.4cm,]{\u{epsilon raute raute}} (1);

\path (2) edge[loop above]
node{\u{b A epsilon}\\\u{a A AAA}} (2);

\path (3) edge[above]
node{\u{epsilon raute Araute}} (2);

\path (3) edge[bend left=10]
node[below right=-0.4cm]{\u{epsilon B epsilon}} (4);

\path (4) edge[below,loop,loop below]
node{\u{b B BB}} (4);

\path (4) edge[bend left=10]
node[above left=-0.4cm]{\u{a B epsilon}} (3);

\path (4) edge[below,bend left=10]
node{\u{epsilon raute raute}} (1);
\end{tikzpicture}
\end{center}

\begin{enumerate}

\item Gebe für die folgenden Wörter, die in der Sprache enthalten sind,
eine Berechnung (Folge von Konfigurationen) des Kellerautomaten an:

\begin{enumerate}

%%
% (a)
%%

\item $w_1 = \texttt{bab}$

\begin{liAntwort}
$(\z1, bab, \#) \vdash
(\z4, ab, B\#) \vdash
(\z3, b, \#) \vdash
(\z2, b, A\#) \vdash
(\z2, \epsilon, \#) \vdash
(\z1, \epsilon, \#) \vdash
(\z{Final}, \epsilon, \epsilon)$
\end{liAntwort}

%%
% (b)
%%

\item $w_2 = \texttt{abb}$

\begin{liAntwort}
$(\z1, abb, \#) \vdash
(\z2, bb, AA\# )\vdash
(\z2, b, A\#) \vdash
(\z2, \epsilon, \#) \vdash
(\z1, \epsilon, \#) \vdash
(\z{Final}, \epsilon, \epsilon)$
\end{liAntwort}

%%
% (c)
%%

\item $w_3 = \texttt{abababbbb}$

\begin{liAntwort}
$(\z1, abababbbb, \#) \vdash
(\z2, bababbbb, AA\#) \vdash
(\z2, ababbbb, A\#) \vdash
(\z2, babbbb, AAA\#) \vdash
(\z2, abbbb, AA\#) \vdash
(\z2, bbbb, AAAA\#) \vdash
(\z2, bbb, AAA\#) \vdash
(\z2, bb, AA\#) \vdash
(\z2, b, A\#) \vdash
(\z2, \epsilon, \#) \vdash
(\z1, \epsilon, \#) \vdash
(\z{Final}, \epsilon, \epsilon)$
\end{liAntwort}

\end{enumerate}

\item Charakterisiere die Wörter der Sprache in eigenen Worten.

\begin{liAntwort}
$L = \{w \, | \, w
\text{ enthält genau doppelt so viele }
b \text{’s wie }
a \text{’s} \}$
\end{liAntwort}

\end{enumerate}
\end{document}
