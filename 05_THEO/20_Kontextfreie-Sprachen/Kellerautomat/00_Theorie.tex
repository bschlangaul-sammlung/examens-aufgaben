\documentclass{lehramt-informatik-haupt}
\liLadePakete{formale-sprachen}

\begin{document}

%%%%%%%%%%%%%%%%%%%%%%%%%%%%%%%%%%%%%%%%%%%%%%%%%%%%%%%%%%%%%%%%%%%%%%%%
% Theorie-Teil
%%%%%%%%%%%%%%%%%%%%%%%%%%%%%%%%%%%%%%%%%%%%%%%%%%%%%%%%%%%%%%%%%%%%%%%%

%-----------------------------------------------------------------------
%
%-----------------------------------------------------------------------

\section{Kellerautomat}

\begin{quellen}
\item \cite[Kapitel 19.1.2.3, Seite 587-88]{schneider}
\item \cite{wiki:kellerautomat}
\item \cite[Seite 224-226]{hoffmannn}
\end{quellen}

Ein Kellerautomat ist ein um einen „Speicher“ (Keller) erweiterter
endlicher Automat.

Ein nichtdeterministischer Kellerautomat $K$ (PDA = pushdown automaton)
ist ein 6-Tupel $K = (Z, \Sigma, \Gamma, \delta, z_0 , \#)$

\begin{description}
\item[Z] endliche Zustandsmenge
\item[$\Sigma$] (Sigma) Eingabealphabet
\item[$\Gamma$] (Gamma) Kelleralphabet mit \#
\item[$\delta$] (delta) Zustandsübergangsfunktion
\item[$z_0$] Startzustand
\item[\#] unterstes Kellersymbol / Anfangssymbol im Keller ($\bot$ (bot) bei Hoffmann, $Z_0 bei$ bei Erk und Blum) \footcite[Seite 22]{theo:fs:2}
\end{description}

Der Kellerspeicher ist als \memph{Stapel (stack)} organisiert und
erlaubt daher nur einen eingeschränkten Zugriff auf die Elemente. Die
Funktionsweise ist ähnlich derer eines konventionellen Bücherstapels;
hier können wir ein neues Buch entweder oben auf den Stapel packen
(Push-Operation) oder das oberste Buch wegnehmen (Pop-Operation). In
einem Stack dürfen Elemente weder von unten noch aus der Mitte entnommen
werden, d. h., das zuletzt hinzugefügte Element wird immer als erstes
wieder entfernt (\memph{LIFO-Prinzip}, \memph{Last-In-First-Out}).

Im Gegensatz zu den bisher betrachteten Akzeptoren besitzen
Kellerautomaten zwei \memph{verschiedene Alphabete} $\Sigma$ und
$\Gamma$. Das \memph{Eingabealphabet $\Sigma$} enthält die Zeichen, aus
denen sich die \memph{Eingabewörter} zusammensetzen, und das
\memph{Kelleralphabet $\Gamma$} alle Symbole, die in den
\memph{Kellerspeicher} geschrieben werden dürfen. Damit beide Alphabete
einfacher unterschieden werden können, halten wir uns an die gängige
Konvention, \memph{Eingabezeichen mit Kleinbuchstaben} und
\memph{Kellerzeichen mit Großbuchstaben} darzustellen.

\footcite[Seite 224]{hoffmann}

Kellerautomaten können analog zu endlichen Automaten durch
Zustandsgraphen veranschaulicht werden.

Die \memph{Zustandsübergänge} sind charakterisiert durch das
\memph{gelesene Zeichen}, das \memph{oberste Kellersymbol} und
\memph{die auf den Keller gelegten Zeichen, die das oberste Kellersymbol
ersetzen}.
\footcite[Seite 23]{theo:fs:2}

\subsection{Zustandübergang}

Hier verwendete Schreibweise:

\begin{displaymath}
(a, \#, A\#)
\end{displaymath}

\begin{displaymath}
(
\text{zu-verarbeitender-Buchstabe},
\text{aktuelles-Kellerzeichen},
\text{Kellerzustand-nach-Bearbeitung}
)
\end{displaymath}

Standardschreibweise:

\begin{displaymath}
(\#, a): A\#)
\end{displaymath}

\begin{displaymath}
(
\text{aktuelles-Kellerzeichen},
\text{zu-verarbeitender-Buchstabe}):
\text{Kellerzustand-nach-Bearbeitung})
\end{displaymath}

\begin{displaymath}
(a, \#, A\#)
\end{displaymath}

\begin{itemize}
\item

\memph{Buchstabe} des Wortes, der \memph{gerade bearbeitet} wird.

\item

das \memph{aktuelle Kellerzeichen} (d.\,h. das oberste
Zeichen auf dem Keller-Stapel)

\item

\memph{Zustand} des Kellers \memph{nach Abarbeitung}. Dabei bedeutet...

\begin{description}
\item[$\epsilon$]

das oberste Zeichen im Keller wurde gelöscht (d.h. pop)

\item[ein Buchstabe]

der Keller hat sich nicht verändert (d.h. nop).

\item[zwei Buchstaben]

es wurde ein Kellerzeichen oben in den Keller reingeschichtet. Der erste
der beiden Buchstaben ist das neue Kellerzeichen, der zweite Buchstabe
ist das bisherige oberste Kellerzeichen.
\end{description}
\end{itemize}
\liFussnoteUrl{http://sibiwiki.de/wiki/index.php?title=Kellerautomat}

\literatur

\end{document}
