\documentclass{lehramt-informatik-aufgabe}
\liLadePakete{automaten,mathe,formale-sprachen}
\begin{document}
\let\m=\liMengeOhneMathe
\let\u=\liKellerUebergang
\def\z#1{$z_#1$}

\liAufgabenTitel{PDA}
\section{PDA
\index{Kellerautomat}
\footcite[Aufgabe 7: PDA aufstellen]{theo:ab:2}}

Erstelle jeweils einen PDA, der die angegebenen Sprachen akzeptiert.

\begin{enumerate}

%%
% (a)
%%

\item \liAusdruck{0^n 1^n}{n \in \mathbb{N}}

\begin{liAntwort}
N = Null

\begin{center}
\begin{tikzpicture}[li kellerautomat]
  \node[state,initial] (z0) at (4.43cm,-4cm) {$z_0$};
  \node[state] (z1) at (5.86cm,-2.29cm) {$z_1$};
  \node[state] (z2) at (8.71cm,-2.29cm) {$z_2$};
  \node[state,accepting] (z3) at (10.29cm,-4.14cm) {$z_3$};

  \liKellerKante[left]{z0}{z1}{
    0, KELLERBODEN, N KELLERBODEN;
  }

  \liKellerKante[above]{z1}{z2}{
    1, N, EPSILON;
  }

  \liKellerKante[above,loop]{z1}{z1}{
    0, N, N N;
  }

  \liKellerKante[right]{z2}{z3}{
    EPSILON, KELLERBODEN, EPSILON;
  }

  \liKellerKante[above,loop]{z2}{z2}{
    1, N, EPSILON;
  }
\end{tikzpicture}
\end{center}

\liFlaci{Aji1r8mf7}
\end{liAntwort}

%%
% (b)
%%

\item Alle Wörter, die gleich viele $a$ wie $b$ enthalten.

\begin{liAntwort}
\begin{center}
\begin{tikzpicture}[li kellerautomat]
  \node[state,initial] (z0) at (2.29cm,-4cm) {$z_0$};
  \node[state] (z1) at (6cm,-1.86cm) {$z_1$};
  \node[state] (z2) at (7.29cm,-5.71cm) {$z_2$};
  \node[state,accepting] (z3) at (1cm,-1.43cm) {$z_3$};

  \liKellerKante[left,bend left]{z0}{z1}{
    a, KELLERBODEN, A KELLERBODEN;
  }

  \liKellerKante[above right,bend left=10]{z0}{z2}{
    b, KELLERBODEN, B KELLERBODEN;
  }

  \liKellerKante[left]{z0}{z3}{
    EPSILON, KELLERBODEN, EPSILON;
  }

  \liKellerKante[right,bend left]{z1}{z0}{
    b, A, EPSILON;
  }

  \liKellerKante[above,loop]{z1}{z1}{
    a, A, A A;
    b, A, EPSILON;
  }

  \liKellerKante[above,loop]{z2}{z2}{
    b, B, B B;
    a, B, EPSILON;
  }

  \liKellerKante[below left,bend left=10]{z2}{z0}{
    a, B, EPSILON;
  }
\end{tikzpicture}
\end{center}

\liFlaci{Arar6fos7}
\end{liAntwort}

%%
% (c)
%%

\item Alle Wörter, bei denen kein Präfix mehr Einsen wie Nullen hat.

\begin{liAntwort}
akzeptiert:

\begin{itemize}
\item 000
\item 00101
\item 01
\end{itemize}

nicht akzeptiert:

\begin{itemize}
\item 011
\item 111
\item 1
\end{itemize}

\begin{center}
\begin{tikzpicture}[li kellerautomat]
  \node[state,initial] (z0) at (2.43cm,-6.43cm) {$z_0$};
  \node[state] (z2) at (7.43cm,-6.43cm) {$z_2$};
  \node[state,accepting] (z1) at (5.57cm,-4.14cm) {$z_1$};

  \liKellerKante[above,bend left]{z0}{z2}{
    0, KELLERBODEN, N KELLERBODEN;
  }

  \liKellerKante[above left]{z0}{z1}{
    0, KELLERBODEN, EPSILON;
    EPSILON, KELLERBODEN, EPSILON;
  }

  \liKellerKante[above,loop below]{z0}{z0}{
    EPSILON, N, EPSILON;
  }

  \liKellerKante[above,loop]{z2}{z2}{
    1, N, EPSILON;
    0, N, N N;
  }

  \liKellerKante[below,bend left]{z2}{z0}{
    0, N, N;
    EPSILON, KELLERBODEN, KELLERBODEN;
    0, KELLERBODEN, EPSILON;
    EPSILON, N, EPSILON;
  }
\end{tikzpicture}
\end{center}

\liFlaci{Af7rfyqqg}
\end{liAntwort}
\end{enumerate}
\end{document}
