\documentclass{lehramt-informatik-aufgabe}
\liLadePakete{automaten,mathe}
\begin{document}
\let\m=\liMengeOhneMathe
\let\u=\liKellerUebergang
\def\z#1{$z_#1$}

\liAufgabenTitel{PDA}
\section{PDA
\index{Kellerautomat}
\footcite{theo:ab:2}}

Erstelle jeweils einen PDA, der die angegebenen Sprachen akzeptiert.

\begin{enumerate}

%%
% (a)
%%

\item $L = \{0^n 1^n | n \in \mathbb{N} \}$

N = Null

\begin{liAntwort}
\begin{center}
\begin{tikzpicture}[li kellerautomat,node distance=1.5cm]
\node[state,initial] (0) {\z0};
\node[state,above right=of 0] (1) {\z1};
\node[state,right=of 1] (2) {\z2};
\node[state,below right=of 2] (3) {\z3};

\path (0) edge[above left] node{\u{0 raute N}} (1);
\path (1) edge[above,loop] node{\u{0 N NN}} (1);
\path (1) edge[above] node{\u{1 N epsilon}} (2);
\path (2) edge[above,loop] node{\u{1 N epsilon}} (2);
\path (2) edge[above right] node{\u{epsilon raute epsilon}} (3);
\end{tikzpicture}
\end{center}
\end{liAntwort}

%%
% (b)
%%

\item Alle Wörter, die gleich viele $a$ wie $b$ enthalten.

%%
% (c)
%%

\item Alle Wörter, bei denen kein Präfix mehr Einsen wie Nullen hat.
\end{enumerate}
\end{document}
