\documentclass{lehramt-informatik-aufgabe}
\liLadePakete{formale-sprachen,syntaxbaum}
\begin{document}
\liAufgabenTitel{Vorlesungsaufgabe}
\section{Kontextfreie Sprache
\index{Kontextfreie Sprache}
\footcite{theo:fs:2}}

\section{Übung}
\begin{enumerate}
\item Erstellen Sie eine Ableitung für die Wörter der Sprache zur
vorgegeben Grammatik\footcite[Seite 10]{theo:fs:2}

$V = \{ S, A, B \}$

\liAlphabet{0, 1}

\begin{liProduktionsRegeln}
S -> A1B,
A -> 0A | epsilon,
B -> 0B | 1B | epsilon
\end{liProduktionsRegeln}

$S = S$

\begin{itemize}
\item 00101

\begin{liAntwort}

\end{liAntwort}

\item 1001

\begin{liAntwort}
S $\rightarrow$
A1B $\rightarrow$
1B $\rightarrow$
10B $\rightarrow$
100B $\rightarrow$
1001B $\rightarrow$
1001
\end{liAntwort}
\end{itemize}

\item Erstelle eine kontextfreie Grammatik, die alle Wörter mit gleich
vielen $1$‘s, gefolgt von gleich vielen $0$‘s enthält.

\begin{liAntwort}
\begin{liProduktionsRegeln}
S -> 1S0 | epsilon
\end{liProduktionsRegeln}
\end{liAntwort}

\item Erstelle eine kontextfreie Grammatik, die alle regulären Ausdrücke
über den Zeichen $0,1$ darstellt.
(Beispiel: $01^* (1 + 0)0$ für einen möglichen regulären Ausdruck
(Das $+$-Zeichen ist hier anstelle des Oderzeichens))

\begin{liAntwort}
\liAlphabet{1; 0; (; ); +; *}

\begin{liProduktionsRegeln}
S -> epsilon | 0 | 1 | S* | (S) | SS | S+S
\end{liProduktionsRegeln}
\end{liAntwort}

\end{enumerate}

%-----------------------------------------------------------------------
%
%-----------------------------------------------------------------------

\section{Übung\footcite[Seite 18]{theo:fs:2}}

\let\m=\liMengeOhneMathe

\begin{enumerate}
\item Erstelle eine Ableitung und einen Parsebaum für die folgende
Grammatik für das Wort

$G = ( \m{P} , \m{ 0, 1 }, P, S)$

\begin{liProduktionsRegeln}
S -> epsilon | 0 | 1 | 0P0 | 1P
\end{liProduktionsRegeln}

\begin{itemize}
\item 0000
\item 01010
\end{itemize}

\item Erstelle eine Ableitung und einen Parsebaum für die nebenstehende
Grammatik für das Wort

$V = \m{ S, A, B }$

\liAlphabet{0, 1}

\begin{liProduktionsRegeln}
S -> A1B,
A -> 0A | epsilon,
B -> 0B | 1B | epsilon
\end{liProduktionsRegeln}

$S = S$

\begin{itemize}
\item 10101

\begin{liAntwort}
\def\p{ $\rightarrow$ }

S \p A1B \p 1B \p 10B \p 101B \p 1010B \p 10101B \p 10101

\begin{tikzpicture}[li parsetree,level distance=0.7cm]
\Tree [.Z
  1 [.A
    0 [.A
      0 [.A
        1 [.A
          0 [.A
            1
          ]
        ]
      ]
    ]
  ]
]
\end{tikzpicture}
\end{liAntwort}

\item 00100
\end{itemize}

\item Sind die Parsebäume eindeutig?

\begin{liAntwort}
Ja, die Parsebäume sind eindeutig.
\end{liAntwort}
\end{enumerate}

%-----------------------------------------------------------------------
%
%-----------------------------------------------------------------------

\section{Übung\footcite[Seite 25]{theo:fs:2}}

\begin{enumerate}
\item Gib einen Kellerautomaten an, der die folgende Sprache erkennt:

$L = (a^n c^i b^n | n, i \in N_0)$

\item Gibt eine Grammatik für diese Sprache an.

\item Gib Konfigurationsfolgen an für die Erzeugung des Wortes

\begin{itemize}
\item aacbb
\item accb
\end{itemize}

\end{enumerate}

%-----------------------------------------------------------------------
%
%-----------------------------------------------------------------------

\section{Übung\footcite[Seite 34]{theo:fs:2}}

\begin{enumerate}
\item Erstelle eine (deterministische) Grammatik für Palindrome, für die
ein DPDA existiert.

$L = \{ w \$ w^R \, | \, w \in (a|b)^* \}$

\item Wandle diese Grammatik in einen DPDA um.
\end{enumerate}

%-----------------------------------------------------------------------
%
%-----------------------------------------------------------------------

\section{Übung\footcite[Seite 37]{theo:fs:2}}

Überführe die folgenden kontextfreien Grammatiken in CNF

\begin{liProduktionsRegeln}
S -> ABC,
A -> aCD,
B -> bCD,
C -> D | epsilon,
D -> C
\end{liProduktionsRegeln}

%-----------------------------------------------------------------------
%
%-----------------------------------------------------------------------

\section{Übung\footcite[Seite 43]{theo:fs:2}}

Zeige, dass die folgenden Sprache nicht kontextfrei sind:

\begin{itemize}
\item $L = \{ a^n b^n c^{2n} | n \in N \}$
\item $L = \{ a^n b^{n^2} | n \in N \}$
\end{itemize}
\end{document}
