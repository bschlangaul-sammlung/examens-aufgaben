\documentclass{lehramt-informatik-aufgabe}
\liLadePakete{formale-sprachen,syntaxbaum,automaten}
\begin{document}
\let\m=\liMenge

\liAufgabenTitel{Vorlesungsaufgabe}
\section{Kontextfreie Sprache
\index{Kontextfreie Sprache}
\footcite{theo:fs:2}}

%-----------------------------------------------------------------------
%
%-----------------------------------------------------------------------

\section{Übung\footcite[Seite 25]{theo:fs:2}}

\begin{enumerate}
\item Gib einen Kellerautomaten an, der die folgende Sprache erkennt:

$L = (a^n c^i b^n | n, i \in N_0)$

\begin{liAntwort}
\begin{center}
\begin{tikzpicture}[->,node distance=5cm]
\node[state,initial] (0) {$z_0$};
\node[state,above right of=0] (1) {$z_1$};
\node[state,right of=0] (2) {$z_2$};

\path (0) edge[below,loop below] node[text width=3cm,align=center]{
  (a, \#, A\#)\\
  (a, A, AA)\\
} (0);

\path (0) edge[below] node[text width=3cm,align=center]{
  ($\epsilon$, \#, $\epsilon$)\\
  (b, A, $\epsilon$)\\
} (2);

\path (2) edge[below,loop below] node[text width=3cm,align=center]{
  (b, A, $\epsilon$)\\
  ($\epsilon$, \#, $\epsilon$)\\
} (2);

\path (1) edge[below] node[text width=3cm,align=center]{
  ($\epsilon$, \#, $\epsilon$)\\
  (b, A, $\epsilon$)\\
} (2);

\path (1) edge[above,loop] node[text width=3cm,align=center]{
  (c, A, A)\\
  (c, \#, \#)\\
} (1);

\path (0) edge[below] node[text width=3cm,align=center]{
  (c, \#, \#)\\
  (a, A, AA)\\
} (1);

\end{tikzpicture}
\end{center}

Tabellenform:

\begin{tabular}{llllll}
z0 & a & \# & z0 & A\#\\
z0 & a & A & z0 & AA\\
z0 & b & A & z2 & A & $\epsilon$\\
z0 & a \# & z0 & A\#\\
z0 & a & \# & z0 & A\#\\
z0 & a & \# & z0 & A\#\\

z1 & c & \# & z1 & A\#\\
z1 & c & A & z2 & A$\epsilon$\\
z1 & c & A & z2 & A$\epsilon$\\
z1 & $\epsilon$ & \# & z2 & $\epsilon$\\

z2 & b & A & z2 & A$\epsilon$\\
z2 & $\epsilon$ & \# & z2 & $\epsilon$\\
\end{tabular}
\end{liAntwort}

\item Gibt eine Grammatik für diese Sprache an.

\begin{liAntwort}
\begin{liProduktionsRegeln}
S -> aSb | epsilon | c | cC,
C -> cC | epsilon
\end{liProduktionsRegeln}

alternativ:

\begin{liProduktionsRegeln}
S -> aSb | epsilon | C,
C -> cC | epsilon
\end{liProduktionsRegeln}
\end{liAntwort}

\item Gib Konfigurationsfolgen an für die Erzeugung des Wortes

\begin{itemize}
\item aacbb

\begin{liAntwort}
\begin{tabular}{llr}
a: & z0, a,\# -> zo A\#                   & A\#\\
c. & z0, c,A -> z1 A                      & A\#\\
c: & z1, c, A -> z1, A                    & A\#\\
b: & z1, b, A -> z2, epsilon              & \#\\
epsilon: & z2, epsilon, \# -> z2, epsilon & -\\
\end{tabular}{llr}
\end{liAntwort}

\item accb
\end{itemize}

\end{enumerate}

%-----------------------------------------------------------------------
%
%-----------------------------------------------------------------------

\section{Kellerautomaten\footcite[Seite 27]{theo:fs:2}}

Erstelle einen Kellerautomaten zu
\begin{enumerate}
\item $G = (\m{P} , \m{0, 1}, P, S)$

\begin{liProduktionsRegeln}
S -> epsilon | 0 | 1 | 0P0 | 1P1,
\end{liProduktionsRegeln}

\item Grammatik mit den Produktionsregeln

\begin{liProduktionsRegeln}
S -> A1B,
A -> 0A | epsilon,
B -> 0B | 1B | epsilon,
\end{liProduktionsRegeln}
\end{enumerate}

%-----------------------------------------------------------------------
%
%-----------------------------------------------------------------------

\section{Übung\footcite[Seite 34]{theo:fs:2}}

\begin{enumerate}
\item Erstelle eine (deterministische) Grammatik für Palindrome, für die
ein DPDA existiert.

$L = \{ w \$ w^R \, | \, w \in (a|b)^* \}$

\item Wandle diese Grammatik in einen DPDA um.
\end{enumerate}

%-----------------------------------------------------------------------
%
%-----------------------------------------------------------------------

\section{Übung\footcite[Seite 37]{theo:fs:2}}

Überführe die folgenden kontextfreien Grammatiken in CNF

\begin{liProduktionsRegeln}
S -> ABC,
A -> aCD,
B -> bCD,
C -> D | epsilon,
D -> C
\end{liProduktionsRegeln}

%-----------------------------------------------------------------------
%
%-----------------------------------------------------------------------

\section{Übung\footcite[Seite 43]{theo:fs:2}}

Zeige, dass die folgenden Sprache nicht kontextfrei sind:

\begin{itemize}
\item $L = \{ a^n b^n c^{2n} | n \in N \}$
\item $L = \{ a^n b^{n^2} | n \in N \}$
\end{itemize}
\end{document}
