\documentclass{lehramt-informatik-aufgabe}
\liLadePakete{}
\begin{document}
\liAufgabenTitel{Vorlesungsaufgabe}
\section{Kontextfreie Sprache
\index{Kontextfreie Sprache}
\footcite{theo:fs:2}}

\section{Übung}
\begin{enumerate}
\item Erstelle eine Ableitung für die Wörter der Sprache zur
vorhergehenden Grammatik\footcite[Seite 10]{theo:fs:2}

\begin{itemize}
\item 00101
\item 1001
\end{itemize}

\item Erstelle eine kontextfreie Grammatik, die alle Wörter mit gleich
vielen 1‘s, gefolgt von gleich vielen 0‘s enthält.

\item Erstelle eine kontextfreie Grammatik, die alle regulären Ausdrücke
über den Zeichen 0,1 darstellt.

Beispiel:

$01^* (1 + 0)0$ für einen möglichen regulären Ausdruck

[Das +-Zeichen ist hier anstelle des Oderzeichens]

\end{enumerate}

%-----------------------------------------------------------------------
%
%-----------------------------------------------------------------------

\section{Übung\footcite[Seite 18]{theo:fs:2}}

\begin{enumerate}
\item Erstelle eine Ableitung und einen Parsebaum für die folgende
Grammatik für das Wort

$G = ( \{P\} , \{0,1\}, \{P \epsilon  0 | 1 | 0P0 | 1P1\}, P)$

\begin{itemize}
\item 0000
\item 01010
\end{itemize}

\item Erstelle eine Ableitung und einen Parsebaum für die nebenstehende
Grammatik für das Wort

$V = {S, A, B}$
$∑= {0,1}$

$P: S \rightarrow A1B$

$A \rightarrow 0A | \epsilon$

$B \rightarrow 0B | 1B | \epsilon$

$S = S$

\begin{itemize}
\item 10101
\item 00100
\end{itemize}

\item Sind die Parsebäume eindeutig?
\end{enumerate}

%-----------------------------------------------------------------------
%
%-----------------------------------------------------------------------

\section{Übung\footcite[Seite 25]{theo:fs:2}}

\begin{enumerate}
\item Gib einen Kellerautomaten an, der die folgende Sprache erkennt:

$L = (a^n c^i b^n | n, i \in N_0 )$

\item Gibt eine Grammatik für diese Sprache an.

\item Gib Konfigurationsfolgen an für die Erzeugung des Wortes

\begin{itemize}
\item aacbb
\item accb
\end{itemize}

\end{enumerate}

%-----------------------------------------------------------------------
%
%-----------------------------------------------------------------------

\section{Übung\footcite[Seite 34]{theo:fs:2}}

\begin{enumerate}
\item Erstelle eine (deterministische) Grammatik für Palindrome, für die
ein DPDA existiert.

$L = \{ w\$w R w \in (a|b)^* \}$

\item Wandle diese Grammatik in einen DPDA um.
\end{enumerate}

%-----------------------------------------------------------------------
%
%-----------------------------------------------------------------------

\section{Übung\footcite[Seite 37]{theo:fs:2}}

Überführe die folgenden kontextfreien Grammatiken in CNF

$S \rightarrow ABC$

$A \rightarrow aCD$

$B \rightarrow bCD$

$C \rightarrow D | \epsilon$

$D \rightarrow C$

%-----------------------------------------------------------------------
%
%-----------------------------------------------------------------------

\section{Übungg\footcite[Seite 43]{theo:fs:2}}

Zeige, dass die folgenden Sprache nicht kontextfrei sind:

\begin{itemize}
\item $L = \{ a^n b^n c^{2n} | n \in N \}$
\item $L = \{ a^n b^{n^2} | n \in N \}$
\end{itemize}
\end{document}
