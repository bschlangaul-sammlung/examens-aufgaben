\documentclass{lehramt-informatik-aufgabe}
\liLadePakete{syntax,formale-sprachen,syntaxbaum}

\begin{document}
\liAufgabenTitel{Klammerausdrücke}
\section{Klammerausdrücke
\index{Kontextfreie Sprache}
\footcite{theo:ab:2}}

In Programmierumgebungen kommen Abfolgen von Klammern, runde, eckige und
geschweifte, vor. Diese müssen in der richtigen Abfolge auf- bzw.
geschlossen werden. Eine korrekte Abfolge von Klammern wäre zum
Beispiel:

\begin{minted}{md}
{[](){(){[]([])}}}{}
\end{minted}

\begin{minted}{md}
{
  []
  ()
  {
    ()
    {
      []
      (
        []
      )
    }
  }
}
{}
\end{minted}

\begin{enumerate}

%%
% (a)
%%

\item Entwerfen Sie eine Grammatik, die die korrekte Abfolge solcher
Klammerfolgen beschreibt.

\begin{liAntwort}
Die Variablennamen:

\begin{description}
\item[S] \textbf{S}tart

\item[G] \textbf{g}eschweifte Klammer (\texttt{\{\}})

\item[E] \textbf{e}ckige Klammern (\texttt{[]})

\item[R] \textbf{r}unde Klammern (\texttt{()})

\item[V] \textbf{V}erkettung
\end{description}

\begin{minted}{md}
S -> G | E | R | V | EPSILON
G -> { S } | { }
E -> [ S ] | [ ]
R -> ( S ) | ( )
V -> S S
\end{minted}

Die Produktionsregeln könnte man auch als nur eine Regel zusammenfassen.

\begin{minted}{md}
S -> { S } | {} | ( S ) | () | [ S ] | [] | S S | EPSILON
\end{minted}

Damit eine „abwechslungsreichere“ Ableitung entsteht, wir hier die auf
mehrere Regeln aufgeteile Grammtik präferiert.

\end{liAntwort}
%%
% (b)
%%

\newpage

\item Geben Sie eine Ableitung für den oben angegebenen Klammerausdruck
an.

\begin{liAntwort}
In übersichtlicher Formatierung

\begin{minted}{md}
01 {
02   []
03   ()
04   {
05     ()
06     {
07       []
08       (
09         []
10       )
11     }
12   }
13 }
14 {}
\end{minted}

% \Uchar91 \Uchar93

Vor den Knoten G E R V müsste eigentlich jedesmal ein S stehen. Da der
Baum sonst zu unübersichtlich wird, wurde auf die S-Knoten verzichtet.

\begin{center}
\begin{tikzpicture}[li parsetree,level distance=1.2cm]
\Tree
[.S
  [.V
    [.G % 01
      \{
      [.V
        [.R ( ) ]  % 02
        [.R ( ) ] % 03
        [.G
          \{ % 04
          [.V
            [.R ( ) ] % 05
            [.G
              \{ % 06
                [.V
                  [.E \Uchar91 \Uchar93 ] % 07
                  [.R
                    ( % 08
                      [.E \Uchar91 \Uchar93 ] % 09
                    ) % 08
                  ]
                ]
              \} % 11
            ]
          ]
          \} % 12
        ]
      ]
      \}
    ]
    [.G \{ \} ] % 14
  ]
]
\end{tikzpicture}
\end{center}
\end{liAntwort}

\end{enumerate}
\end{document}
