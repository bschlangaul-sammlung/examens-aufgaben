\documentclass{lehramt-informatik-aufgabe}
\liLadePakete{syntax}

\begin{document}
\liAufgabenTitel{Klammerausdrücke}
\section{Klammerausdrücke
\index{Kontextfreie Sprache}
\footcite{theo:ab:2}}

In Programmierumgebungen kommen Abfolgen von Klammern, runde, eckige und
geschweifte, vor. Diese müssen in der richtigen Abfolge auf- bzw.
geschlossen werden. Eine korrekte Abfolge von Klammern wäre zum
Beispiel:

\begin{minted}{md}
{[](){(){[]([])}}}{}
\end{minted}

\begin{enumerate}

%%
% (a)
%%

\item Entwerfe eine Grammatik, die die korrekte Abfolge solcher
Klammerfolgen beschreibt.

%%
% (b)
%%

\item Gib eine Ableitung für den oben angegebenen Klammerausdruck an.

\end{enumerate}
\end{document}
