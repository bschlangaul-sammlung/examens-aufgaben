\documentclass{lehramt-informatik-haupt}
\liLadePakete{mathe,syntaxbaum,automaten,formale-sprachen}
\usepackage{tikz}
\usetikzlibrary{shapes.geometric,calc}

\begin{document}

%%%%%%%%%%%%%%%%%%%%%%%%%%%%%%%%%%%%%%%%%%%%%%%%%%%%%%%%%%%%%%%%%%%%%%%%
% Theorie-Teil
%%%%%%%%%%%%%%%%%%%%%%%%%%%%%%%%%%%%%%%%%%%%%%%%%%%%%%%%%%%%%%%%%%%%%%%%

\section{Minimierungsalgorithmus\footcite[Seite 47-57]{vossen}}

\footcite[Seite 51-62]{theo:fs:1}

Es wird eine \memph{Zustandspaartabelle} mit allen möglichen
Zustandspaaren erstellt. Dabei spielt es \memph{keine Rolle}, ob die
Zustände im Automaten \memph{verbunden sind oder nicht}.

Die Diagonale der Tabelle wird komplett gestrichen, da Zustände nicht
mit sich selbst überprüft werden müssen. Außerdem werden die
\memph{Paare nur in einer Richtung} betrachtet. Die \memph{obere Hälfte}
der Tabelle über der Diagonalen \memph{kann gestrichen werden}.

Es werden die Zustandspaar markiert, in denen ein Zustand \memph{ein
Endzustand und der andere kein Endzustand} ist.

Als letztes werden mit Hilfe einer \memph{Übergangstabelle} die noch
nicht markierten Zustandspaare auf alle möglichen Übergangsmöglichkeiten
überprüft. Entsteht hierbei ein \memph{bereits gestrichenes Paar} so
wird das aktuell überprüfte Paar \memph{ebenfalls gestrichen}.

Wir testen jetzt für jedes Zustandspaar die Folgezustände bei der
Eingabe der Zeichen des Alphabets.

Wir suchen uns nun die Zustandspaare, die bei einer Eingabe ein
Zustandspaar ergeben, das bereits gestrichen ist.

\liFussnoteUrl{https://studyflix.de/informatik/dea-minimieren-1212}

\literatur

\end{document}
