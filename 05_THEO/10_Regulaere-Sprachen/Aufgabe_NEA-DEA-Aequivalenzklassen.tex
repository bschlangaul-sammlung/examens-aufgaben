\documentclass{lehramt-informatik-aufgabe}
\liLadePakete{mathe,automaten}
\begin{document}
\liAufgabenTitel{NEA-DEA-Aequivalenzklassen}
\section{NEA-DEA-Äquivalenzklassen
\index{Reguläre Sprache}
\footcite{theo:ab:1}}

Gegeben ist der deterministische endliche Automat
$A = (Q, \{ 0, 1 \}, \delta, q_0 , F)$,
wobei
$Q = \{ A, B, C, D, E \}$, $q_0 = A$, $F = \{ E \}$ und

\begin{center}
\begin{tabular}{l||l|l}
δ & 0 & 1 \\\hline\hline
A & B & C \\\hline
B & E & C \\\hline
C & D & C \\\hline
D & E & A \\\hline
E & E & E \\\hline
\end{tabular}
\end{center}

\begin{enumerate}
\item Minimieren Sie den Automaten mit dem bekannten
Minimierungsalgorithmus. Dokumentieren Sie die Schritte geeignet.
\index{Minimierungsalgorithmus}

\begin{liAntwort}
\begin{center}
\begin{tikzpicture}[->,node distance=2cm]
\node[state,initial] (A) {A};
\node[state,above right of=A] (B) {B};
\node[state,below right of=A] (C) {C};
\node[state,right of=C] (D) {D};
\node[state,above of=D,accepting] (E) {E};

\path (A) edge[above] node{0} (B);
\path (A) edge[above] node{1} (C);
\path (B) edge[above] node{0} (E);
\path (B) edge[right] node{1} (C);
\path (C) edge[above] node{0} (D);
\path (C) edge[above,loop below] node{1} (C);
\path (E) edge[above] node{0} (E);
\path (D) edge[above] node{1} (A);
\path (E) edge[above,loop right] node{0,1} (E);
\end{tikzpicture}
\end{center}

\end{liAntwort}

\item Geben Sie einen regulären Ausdruck für die erkannte Sprache an.
\index{Reguläre Ausdrücke}

\begin{liAntwort}
$r = (0|1)^*00(0|1)^*$
\end{liAntwort}

\item Geben Sie die Äquivalenzklassen der Myhill-Nerode-Äquivalenz der
Sprache durch reguläre Ausdrücke an.
\index{Äquivalenzklassen}

\begin{liAntwort}
Die Äquivalenzklassen lauten: $[A, C], [B, D], [E]$

\begin{align*}
r_A &= (1^*(01)^*)^*\\
r_B &= (1^*(01)^*)^*0\\
r_C &= r\\
\end{align*}
\end{liAntwort}
\end{enumerate}

\end{document}
