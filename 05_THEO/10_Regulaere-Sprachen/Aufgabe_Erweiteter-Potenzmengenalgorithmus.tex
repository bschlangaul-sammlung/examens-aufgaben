\documentclass{lehramt-informatik-aufgabe}
\liLadePakete{automaten,formale-sprachen}

% Info_2021-02-26-2021-02-26_13.01.25.mp4 3h03
\begin{document}
\liAufgabenTitel{Erweiteter Potenzmengenalgorithmus}
\section{Erweiterter Potenzmengenalgorithmus $\epsilon$-NEA zum DEA
\index{Erweiteter Potenzmengenalgorithmus}
\footcite[Seite 47-49]{theo:fs:1}}

Welche Sprache akzeptiert dieser Automat? Beschreibe in Worten
und stelle einen regulären Ausdruck sowie eine Grammatik hierfür
auf.

\begin{center}
\begin{tikzpicture}[->,node distance=2cm]
\node[state,initial] (0) {$z_0$};
\node[state,right of=0] (1) {$z_2$};
\node[state,right of=1,accepting] (2) {$z_1$};

\path (0) edge[above] node{$\epsilon$} (1);
\path (1) edge[above] node{$\epsilon$} (2);
\path (0) edge[above,loop] node{a} (0);
\path (1) edge[above,loop] node{b} (1);
\path (2) edge[above,loop] node{c} (3);
\end{tikzpicture}
\end{center}

\end{document}
