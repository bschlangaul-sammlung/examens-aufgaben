\documentclass{lehramt-informatik-aufgabe}
\liLadePakete{automaten,formale-sprachen}

% Info_2021-02-26-2021-02-26_13.01.25.mp4 3h02m30
\begin{document}
\liAufgabenTitel{Erweiteter Potenzmengenalgorithmus}
\section{Erweiterter Potenzmengenalgorithmus $\epsilon$-NEA zum DEA
\index{Erweiteter Potenzmengenalgorithmus}
\footcite[Seite 47-49]{theo:fs:1}}

\begin{center}
\begin{tikzpicture}[->,node distance=2cm]
\node[state,initial] (0) {$z_0$};
\node[state,right of=0] (1) {$z_2$};
\node[state,right of=1,accepting] (2) {$z_1$};

\path (0) edge[above] node{$\epsilon$} (1);
\path (1) edge[above] node{$\epsilon$} (2);
\path (0) edge[above,loop] node{a} (0);
\path (1) edge[above,loop] node{b} (1);
\path (2) edge[above,loop] node{c} (3);
\end{tikzpicture}
\end{center}

\begin{enumerate}
\item Welche Sprache akzeptiert dieser Automat? Beschreiben Sie in
Worten und stellen Sie einen regulären Ausdruck sowie eine Grammatik
hierfür auf.

\begin{liAntwort}
\begin{description}
\item[in Worten]

Das Alphabet besteht aus $a$, $b$, $c$. Am Anfang stehen $0$ oder
beliebig viele $a$’s, dann kommen $0$ oder beliebig viele $b$’s und dann
$0$ oder beliebig viele $c$’s.

\item[Regulärer Ausdruck]

$a^*b^*c^*$

\item[Grammatik]

\begin{liProduktionsRegeln}
S -> aS | bA | cB | epsilon,
A -> bA | cB | epsilon,
B -> cB | epsilon
\end{liProduktionsRegeln}
\end{description}
\end{liAntwort}
\end{enumerate}

\end{document}
