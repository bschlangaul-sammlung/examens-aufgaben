\documentclass{lehramt-informatik-aufgabe}
\liLadePakete{automaten}
\begin{document}
\liAufgabenTitel{Studiflix-Minimierung}
\section{Minimierungsalgorithmus
\index{Minimierungsalgorithmus}
}

\begin{liAntwort}
\begin{center}
\begin{tikzpicture}[->,node distance=1.5cm]
\node[state,initial] (0) {$z_0$};
\node[state,above=of 0] (1) {$z_1$};
\node[state,right=of 0,accepting] (2) {$z_2$};
\node[state,right=of 1,accepting] (3) {$z_3$};
\node[state,right=of 2,accepting] (4) {$z_4$};
\node[state,right=of 3] (5) {$z_5$};

\path (0) edge[above] node{1} (2);
\path (0) edge[left,bend left] node{0} (1);
\path (1) edge[above] node{1} (3);
\path (1) edge[right,bend left] node{0} (0);
\path (2) edge[above] node{0} (4);
\path (2) edge[above,pos=0.2] node{1} (5);
\path (3) edge[above,pos=0.2] node{0} (4);
\path (3) edge[above] node{1} (5);
\path (4) edge[right] node{1} (5);
\path (5) edge[right,loop right] node{0,1} (5);
\path (4) edge[right,loop right] node{0} (4);

\end{tikzpicture}
\end{center}
\end{liAntwort}
% l = ller
\def\l{$\emptyset$}
\def\z#1{$z_#1$}

\def\a{$*^1$}
\def\b{$*^2$}

\renewcommand{\arraystretch}{1.4}
\begin{tabular}{c||c|c|c|c|c|c|}
\hline
\z0  & \l & \l & \l & \l & \l & \l \\ \hline
\z1  &  & \l & \l & \l & \l & \l \\ \hline
\z2  & \a &  & \l & \l & \l & \l \\ \hline
\z3  &  &  &  & \l & \l & \l \\ \hline
\z4  &  &  &  &  & \l & \l \\ \hline
\z5  &  &  &  &  &  & \l \\ \hline\hline
 & \z0 & \z1 & \z2 & \z3 & \z4 & \z5 \\ \hline
\end{tabular}

\a Paar aus End-/ Nicht-Endzustand kannt nicht äquivalent sein.

\b Test, ob man mit Eingabe zu bereits markiertem Paar kommt.

\liFussnoteUrl{https://studyflix.de/informatik/dea-minimieren-1212}

\end{document}
