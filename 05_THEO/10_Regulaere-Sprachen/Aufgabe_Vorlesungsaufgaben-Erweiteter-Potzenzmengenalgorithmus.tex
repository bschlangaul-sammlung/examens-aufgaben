\documentclass{lehramt-informatik-aufgabe}
\liLadePakete{automaten,formale-sprachen}
\begin{document}
\liAufgabenTitel{Erweiteter Potzenzmengenalgorithmus}
\section{Erweiterter Potenzmengenalgorithmus $\epsilon$-NEA zum NEA
\index{}
\footcite[Seite 50]{theo:fs:1}}

Gegeben ist der folgende $\epsilon$-NEA:

\begin{center}
\begin{tikzpicture}[->,node distance=2cm]
\node[state,initial]              (0) {$z_0$};
\node[state,below right of=0]     (1) {$z_1$};
\node[state,right of=0]           (2) {$z_2$};
\node[state,below of=0]           (3) {$z_3$};
\node[state,right of=2,accepting] (4) {$z_4$};

\path (0) edge[above] node{$\epsilon$} (2);
\path (0) edge[above] node{a} (1);
\path (0) edge[left] node{b} (3);
\path (1) edge[right] node{$\epsilon$} (2);
\path (2) edge[above,loop] node{c} (2);
\path (2) edge[above] node{$\epsilon$} (4);
\path (3) edge[above] node{$\epsilon$} (1);
\end{tikzpicture}
\end{center}

Wandle den gegebenen Automaten in eine $\epsilon$-freien DEA um.

\begin{liAntwort}
\let\p=\liPotenzmenge

\begin{tabular}{llll}
Zustandsmenge & Eingabe a & Eingabe b & Eingabe c\\
\p{z_0, z_2, z_4} &
\p{z_1, z_2, z_4} &
\p{z_1, z_2, z_3, z_4} &
\p{z_2, z_4} \\

\p{z_1, z_2, z_4} &
\p{} &
\p{} &
\p{z_2, z_4} \\

\p{z_1, z_2, z_3, z_4} &
\p{} &
\p{} &
\p{z_2, z_4} \\

\p{z_2, z_4} &
\p{} &
\p{} &
\p{z_2, z_4} \\

\p{} &
\p{} &
\p{} &
\p{} \\
\end{tabular}
\end{liAntwort}

\end{document}
