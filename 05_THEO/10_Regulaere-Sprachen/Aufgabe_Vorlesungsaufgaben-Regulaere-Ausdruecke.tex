\documentclass{lehramt-informatik-aufgabe}
\liLadePakete{syntax,mathe}
\begin{document}
\liAufgabenTitel{Vorlesungsaufgaben}
\section{Übungen zu regulären Ausdrücken
\index{Reguläre Ausdrücke}\footcite[Seite 21]{theo:fs:1}}

\begin{enumerate}

%%
%
%%

\item Gegeben ist eine Sprache $L \subset \Sigma^*$ mit $\Sigma =
\{a,b\}$. Zu der Sprache $L$ gehören alle Wörter, die die Zeichenfolge
\texttt{abba} beinhalten.

Gib einen regulären Ausdruck für diese Sprache an.

\begin{liAntwort}
(a|b)*abba(a|b)*
\end{liAntwort}

%%
%
%%

\item Gebe möglichst einfache reguläre Ausdrücke für die folgenden
Sprachen $L_x \subset \Sigma^*$ mit $\Sigma = \{a,b\}$ und $x \in \{1, 2, 3\}$.

\begin{description}
\item[$L_1$] = $\{ x | x \text{ beinhaltet eine gerade Anzahl von } a \}$

\begin{liAntwort}
\texttt{.*(aa)+.*} oder \texttt{(a|b)*(aa)(a|b)*}
\end{liAntwort}

\item[$L_2$] = $\{ x | x \text{ beinhaltet eine ungerade Anzahl von } b \}$

\begin{liAntwort}
\texttt{.*b(bb)*.*} oder \texttt{(a|b)*b(bb)*(a|b)*}
\end{liAntwort}

\item[$L_3$] = $\{ x | x \text{ beinhaltet an seinen geradzahligen Positionen ausschließlich } a \}$

\begin{liAntwort}
\texttt{((a|b)a)*}
\end{liAntwort}
\end{description}

%%
%
%%

\item Gib einen regulären Ausdruck der eine syntaktisch gültige
E-Mail-Adresse erkennt. (mindestens 1 Zeichen (Groß-/Kleinbuchstabe oder
Zahl) vor dem \texttt{@}; mindestens 1 Zeichen (Groß-/Kleinbuchstabe
oder Zahl) nach dem \texttt{@}; alle E-Mail-Adressen sollen auf
\texttt{.de} oder \texttt{.com} enden.

\begin{liAntwort}
\texttt{[a-zA-Z0-9]+@[a-zA-Z0-9]+\.(de|com)}
\end{liAntwort}
\end{enumerate}

\begin{liAntwort}
\liJavaDatei[firstline=3]{aufgaben/theo_inf/regulaere_ausdruecke/TestRegularExpressions}
\end{liAntwort}

\end{document}
