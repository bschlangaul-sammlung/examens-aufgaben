\documentclass{lehramt-informatik-aufgabe}
\liLadePakete{mathe,automaten}
\begin{document}
\liAufgabenTitel{Vorlesungsaufgaben}
\section{Deterministisch endlicher Automat
\index{Deterministisch endlicher Automat (DEA)}}

\section{Übungen zu Automaten\footcite[Seite 28]{theo:fs:1}}

Stelle einen Automaten zu den folgenden Sprachen auf:

\begin{enumerate}

%%
%
%%

\item $L_1 = \{ x | x \text{ beinhaltet eine gerade Anzahl von } a \}$

\begin{liAntwort}
\begin{center}
\begin{tikzpicture}[->,node distance=2cm]
\node[state,initial] (0) {$z_0$};
\node[state,right of=0] (1) {$z_1$};
\node[state,right of=1,accepting] (2) {$z_2$};

\path (0) edge[above] node{a} (1);
\path (1) edge[above,bend left] node{a} (2);
\path (2) edge[above,bend left] node{a} (1);
\end{tikzpicture}
\end{center}
\end{liAntwort}

%%
%
%%

\item $L_2 = \{ x | x \text{ beinhaltet eine ungerade Anzahl von } b \}$

\begin{liAntwort}
\begin{center}
\begin{tikzpicture}[->,node distance=2cm]
\node[state,initial] (0) {$z_0$};
\node[state,right of=0,accepting] (1) {$z_1$};
\node[state,right of=1] (2) {$z_2$};

\path (0) edge[above] node{b} (1);
\path (1) edge[above,bend left] node{b} (2);
\path (2) edge[above,bend left] node{b} (1);
\end{tikzpicture}
\end{center}
\end{liAntwort}

%%
%
%%

\item Gib einen DEA der eine syntaktisch gültige E-Mail-Adresse erkennt.
(mindestens 1 Zeichen (Groß-/Kleinbuchstabe oder Zahl) vor dem
\texttt{@}; mindestens 1 Zeichen (Groß-/Kleinbuchstabe oder Zahl) nach
dem \texttt{@}; alle E-Mail-Adressen sollen auf \texttt{.de} oder
\texttt{.com} enden.

\end{enumerate}
\end{document}
