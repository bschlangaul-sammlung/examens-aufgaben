\documentclass{lehramt-informatik-aufgabe}
\liLadePakete{mathe,automaten,formale-sprachen}

\begin{document}
\liAufgabenTitel{Vorlesungsaufgaben}
\section{Vorlesungsaufgaben
\index{Potenzmengenalgorithmus}}

\section{Übung Potenzmengenalgorithmus\footcite[Seite 35-45]{theo:fs:1}}

Gegeben ist der folgende NEA:

\begin{center}
\begin{tikzpicture}[->,node distance=2cm]
\node[state,initial] (0) {$z_0$};
\node[state,right of=0] (1) {$z_1$};
\node[state,right of=1,accepting] (2) {$z_2$};

\path (0) edge[above] node{b} (1);
\path (1) edge[above] node{a,b} (2);
\path (0) edge[loop,above] node{a,b} (0);
\path (2) edge[loop,above] node{a,b} (2);
\end{tikzpicture}
\end{center}

\begin{enumerate}

%%
%
%%

\item Welche Sprache akzeptiert dieser Automat? Beschreibe in Worten und
stelle einen regulären Ausdruck hierfür auf.
\footcite[Seite 35]{theo:fs:1}

\begin{liAntwort}
$(a|b)^*b(a|b)(a|b)^*$
\end{liAntwort}

%%
%
%%

\item Überführe den gegebenen NEA mit dem Potenzmengenalgorithmus in
einen DEA.

\begin{liAntwort}
\let\p=\liPotenzmenge

\begin{tabular}{l|l|l|l}
Name & Zustandsmenge & Eingabe a & Eingabe b \\\hline
$z'_0$ & \p{z_0} & \p{z_0} & \p{z_0, z_1} \\
$z'_1$ & \p{z_0, z_1} & \p{z_0, z_2} & \p{z_0, z_1, z_2} \\
$z'_2$ & \p{z_0, z_2} & \p{z_0, z_2} & \p{z_0, z_1, z_2} \\
$z'_3$ & \p{z_0, z_1, z_2} & \p{z_0, z_2} & \p{z_0, z_1, z_2} \\
\end{tabular}

\begin{center}
\begin{tikzpicture}[->,node distance=2cm]
\node[state,initial] (0) {$z'_0$};
\node[state,right of=0] (1) {$z'_1$};
\node[state,accepting, above right of=1] (2) {$z'_2$};
\node[state,accepting, below right of=2] (3) {$z'_3$};

\path (0) edge[above] node{b} (1);
\path (1) edge[above] node{a} (2);
\path (1) edge[above] node{b} (3);
\path (2) edge[right, bend left] node{b} (3);
\path (3) edge[above] node{a} (2);

\path (0) edge[above,loop] node{a} (0);
\path (2) edge[above,loop right] node{a} (2);
\path (3) edge[above,loop right] node{b} (3);
\end{tikzpicture}
\end{center}
\end{liAntwort}

\end{enumerate}

\end{document}
