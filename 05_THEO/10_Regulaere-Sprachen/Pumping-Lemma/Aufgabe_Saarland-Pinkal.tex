\documentclass{lehramt-informatik-aufgabe}
\liLadePakete{formale-sprachen}
\begin{document}

\liAufgabenTitel{Saarland Pinkal}
\section{Pumping-Lemma
\index{Pumping-Lemma (Reguläre Sprache)}
}

\liFussnoteUrl{http://www.coli.uni-saarland.de/courses/I2CL-10/material/Uebungsblaetter/Musterloesung4.4.pdf}

\begin{displaymath}
L_1 = \liMenge{ w c w^R | w \in \liMenge{a, b}^* }
\end{displaymath}

Erläuterung: $w^R$ ist die Spiegelung von $w$, d.\,h. es enthält die
Zeichen von $w$ in umgekehrter Reihenfolge. Worte von $L_1$ sind also
z.\,B. $c$, $abcba$, $bbbaabacabaabbb$

\begin{liAntwort}
$L_1$ ist kontexfrei.

\liPseudoUeberschrift{Beweis, dass $L_1$ nicht regulär ist, durch das
Pumping Lemma: }

\noindent
Wir nehmen an $L_1$ wäre regulär. Dann gibt es einen endlichen
Automaten, der $L_1$ erkennt. Die Anzahl der Zustände dieses Automaten
sei $k$. Wir wählen jetzt das Wort $x = a^k c a^k$. $x$ liegt in $L_1$,
und ist offensichtlich länger als $k$. Dieses Wort muss irgendwo eine
Schleife, also einen aufpumpbaren Teil enthalten, d.\,h. man kann es so
in $uvw$ zerlegen, dass für jede natürliche Zahl $i$ auch $uv^iw$ zu
$L_1$ gehört. Wo könnte dieser aufpumpbare Teil liegen?

\begin{description}
\item[Fall 1:]

Der aufpumbare Teil $v$ liegt komplett im Bereich des ersten
$a^k$-Blocks. Dann würde aber $uv^2w = a^{k + |v|} c a^k$ mehr $a$’s im
ersten Teil als im zweiten Teil enthalten und läge nicht mehr in $L_1$.

\item[Fall 2:]

$v$ enthält das $c$. Dann würde aber $u v^2 w$ zwei $c$’s enthalten und
läge damit nicht mehr in $L_1$.

\item[Fall 3:]

Der aufpumpbare Teil liegt komplett im Bereich des zweiten $a^k$-Blocks.
Dann liegt analog zu Fall 1 $u v^2 w$ nicht mehr in $L_1$. Unser Wort
lässt sich also nicht so zerlegen, dass man den Mittelteil aufpumpen
kann, also ist die Annahme, dass $L_1$ regulär ist, falsch.
\end{description}
\noindent
Beweis, dass $L_1$ kontextfrei ist, durch Angabe einer kontexfreien
Grammatik:

\noindent
\begin{liProduktionsRegeln}
S -> aSa,
S -> bSb,
S -> c
\end{liProduktionsRegeln}
\end{liAntwort}

\end{document}
