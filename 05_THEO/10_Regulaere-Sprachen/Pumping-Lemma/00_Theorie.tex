\documentclass{lehramt-informatik-haupt}
\liLadePakete{mathe}

\begin{document}

\section{Pumping Lemma}

https://studyflix.de/informatik/pumping-lemma-1445

Pumping-Lemma für reguläre Sprachen:

Sei $L$ eine reguläre Sprache. Dann gibt es eine Zahl $j$, sodass für
alle Wörter $ω ∈ L$ mit $|\omega| \geq j$ jeweils eine Zerlegung $\omega
= uvw$ existiert, sodass die folgenden Eigenschaften erfüllt sind:

\begin{enumerate}
\item $|v| \geq 1$
\item $|uv| \leq j$
\item Für alle $i = 0, 1, 2, \dots$ gilt $uv^iw \in L$
\end{enumerate}

Das Pumping-Lemma wird verwendet, um zu zeigen, dass eine
Sprache nicht regulär ist (Widerspruchsbeweis).

%%
%
%%

\subsection{Beispiel}

$L = \{a^n b^n | n \in \mathbb{N})$

Ich behaupte, $L$ sei regulär.

\begin{enumerate}
\item Also gibt es eine Pumpzahl. Sie sei j.
\item (Wähle geschickt ein „langes“ Wort...)
$a^j b^j$ ist ein Wort aus $L$, das sicher langer als $j$ ist.
\item Da $L$ regulär ist, muss es nach dem Pumping-Lemma auch für dieses
Wort eine Zerlegung geben:
\end{enumerate}

$a^j b^j = uvw$ mit $|v| \geq 1$ und $|uv| \leq j$

Weil $uv$ höchstens $j$ lang ist, kann es im Fall von $a^j b^j$ nur aus
$a$‘s bestehen. Da $v$ mindestens ein Zeichen enthält, ist das
mindestens ein $a$. Pumpen führt nun zu mehr $a$‘s als $b$‘s und also zu
einem Wort, das nicht in der Sprache ist. (Widerspruch!)\footcite{wiki:pumping}

$\Rightarrow$ Die Behauptung war falsch!
$\Rightarrow$ $L$ ist nicht regulär!
\footcite[Seite 63-64]{theo:fs:1}

\literatur

\end{document}
