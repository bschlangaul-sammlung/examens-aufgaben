\documentclass{lehramt-informatik-haupt}
\liLadePakete{mathe,syntaxbaum,automaten,formale-sprachen}

\begin{document}

%%%%%%%%%%%%%%%%%%%%%%%%%%%%%%%%%%%%%%%%%%%%%%%%%%%%%%%%%%%%%%%%%%%%%%%%
% Theorie-Teil
%%%%%%%%%%%%%%%%%%%%%%%%%%%%%%%%%%%%%%%%%%%%%%%%%%%%%%%%%%%%%%%%%%%%%%%%

%-----------------------------------------------------------------------
%
%-----------------------------------------------------------------------

\section{Potenzmengenalgorithmus NEA $\rightarrow$ DEA\footcite[Seite 35-47]{theo:fs:1}}

\begin{itemize}
\item Starte im Anfangszustand (in der Menge der Anfangszustände).

\item Gib für jedes Zeichen die Menge der erreichbaren Zustände an.

\item Wiederhole diesen Schritt für jede neu erreichte Menge an
Zuständen.

\item Die Zustandsmengen sind die Zustände des DEA.

\item Mengen, die „alte“ Endzustände enthalten, sind Endzustände des
neuen DEA.\footcite{wiki:potenzmengenkonstruktion}
\end{itemize}

%-----------------------------------------------------------------------
%
%-----------------------------------------------------------------------

\section{Erweiterter Potenzmengenalgorithmus $\epsilon$-NEA zum
DEA\footcite[Seite 48-49]{theo:fs:1}}

Wählen Sie als neuen Anfangszustand den Alten und alle Zustände, die vom
Alten aus mit \memph{$\epsilon$ erreichbar} sind.

Führe Sie den Potenzmengenalgorithmus mit dem neuen Startzustand aus.
Fügen Sie in jedem Schritt auch die \memph{Zustände hinzu}, die über
einen oder mehrere zusätzliche $\epsilon$-Übergänge erreichbar sind.

\section{Komplement}

Um zu zeigen, dass für jede reguläre Sprache L auch das Komplement L =
$\Sigma^* \string\ L$ regulär ist, gehen wir in zwei Schritten vor: Im
ersten Schritt konstruieren wir einen endlichen Automaten A mit L (A) =
L. Dass ein solcher Automat für jede reguläre Sprache existieren muss,
haben wir im vorherigen Abschnitt herausgearbeitet. Im zweiten Schritt
konstruieren wir aus A den Komplementärautomaten A, der die Sprache L
(A) erzeugt. Dass der Komplementärautomat die Sprache L (A) erzeugt, ist
leicht einzusehen. Da wir die Menge der Endzustände invertiert haben,
wird ein Wort $\omega$ genau dann von A akzeptiert, wenn es von A
zurückgewiesen wird. Mit L wird damit immer auch L von einem Automaten
akzeptiert, so dass die Menge der regulären Sprachen in Bezug auf das
Komplement abgeschlossen ist.\footcite[Seite 218-219]{hoffmann}

\literatur

\end{document}
