\documentclass{lehramt-informatik-aufgabe}
\liLadePakete{mathe,automaten,formale-sprachen}

\begin{document}
\liAufgabenTitel{Vorlesungsaufgaben}
\section{Vorlesungsaufgaben
\index{Potenzmengenalgorithmus}}

\section{Übung Potenzmengenalgorithmus\footcite[Seite 46]{theo:fs:1}}

Gegeben ist der folgende NEA:

\begin{center}
\begin{tikzpicture}[->,node distance=2cm]
\node[state,initial] (0) {$z_0$};
\node[state,right of=0] (2) {$z_2$};
\node[state,above of=2] (1) {$z_1$};
\node[state,right of=2,accepting] (3) {$z_3$};

\path (0) edge[above] node{a} (1);
\path (0) edge[above] node{a} (2);
\path (2) edge[right] node{b} (1);
\path (2) edge[above] node{b} (3);
\path (3) edge[loop,above] node{a,b} (3);
\end{tikzpicture}
\end{center}

\noindent
Überführenn Sie den gegebenen NEA mit dem Potenzmengenalgorithmus in
einen DEA.

\begin{liAntwort}
\let\p=\liPotenzmenge

\begin{tabular}{lll}
Zustandsmenge & Eingabe a & Eingabe b \\
\p{z_0} &
\p{z_1, z_2} &
\p{} \\

\p{z_1, z_2} &
\p{}&
\p{z_1, z_3} \\

\p{z_1, z_3} &
\p{z_3} &
\p{z_3} \\

\p{z_3} &
\p{z_3} &
\p{z_3} \\

\p{} &
\p{} &
\p{} \\
\end{tabular}
\end{liAntwort}

\end{document}
