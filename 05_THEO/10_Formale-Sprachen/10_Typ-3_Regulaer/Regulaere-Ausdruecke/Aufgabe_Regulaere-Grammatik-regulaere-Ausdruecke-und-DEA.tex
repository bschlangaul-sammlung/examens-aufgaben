\documentclass{lehramt-informatik-aufgabe}
\liLadePakete{mathe,formale-sprachen,automaten}
\begin{document}
\liAufgabenTitel{Reguläre Grammatik, reguläre Ausdrücke und DEA}
\section{Reguläre Grammatik, reguläre Ausdrücke und DEA
\index{Reguläre Sprache}
\footcite[Aufgabe 2]{theo:ab:1}}

Gegeben sind die folgenden Sprachen über dem Alphabet \liAlphabet{a, b}:

\begin{itemize}
\item \liAusdruck[L_0]{w}{w \text{ enthält mindestens ein \emph{bb}}}
\item \liAusdruck[L_1]{w}{w \text{ endet auf höchstens ein \emph{b}}}
\item \liAusdruck[L_2]{w}{w \text{ fängt mit \emph{aa} an oder hört mit \emph{bb} auf}}
\end{itemize}
\begin{enumerate}

%%
% (a)
%%

\item Geben Sie zu allen Sprachen eine reguläre Grammatik an.
\index{Reguläre Grammatik}

\begin{liAntwort}

\begin{description}
%%
%
%%

\item[$G_0$]

$= (V, \Sigma, P, S)$ mit
$V = \{S, A, B\}$,
\liAlphabet{a, b},
$S = S$ und mit

\begin{liProduktionsRegeln}
S -> a S | b A,
A -> a S | b B | b,
B -> a B | a | b B | b
\end{liProduktionsRegeln}
\liFlaci{Gjp92ri0w}

%%
%
%%

\item[$G_1$]

$= (V, \Sigma, P, S)$ mit
$V = \{S, A, B\}$,
\liAlphabet{a, b},
$S = S$ und mit

Das Wort „b“ ist nicht in der Grammatik?

\begin{liProduktionsRegeln}
S -> a S | b S | a A,
A -> b
\end{liProduktionsRegeln}
\liFlaci{Gfdn0xhwg}

%%
%
%%

\item[$G_2$]

$= (V, \Sigma, P, S)$ mit
$V = \{S, A, B, C, D, E\}$,
\liAlphabet{a, b}, $S = S$ und mit

Stimmt nicht. Wörter können auch mit b beginnen.

\begin{liProduktionsRegeln}
S -> a A | b C,
A -> a B | a | b C,
B -> a B | a | b B | b,
C -> a D | b E | b,
D -> b C | a D,
E -> b E | b | a D
\end{liProduktionsRegeln}
\liFlaci{Gib1z1cwi}

\end{description}
\end{liAntwort}

%%
% (b)
%%

\item Geben Sie zu den folgenden Wörtern eine Ableitung bzw. einen
Syntaxbaum anhand der erstellten Grammatiken aus der Teilaufgabe a) an:
\index{Ableitung (Reguläre Sprache)}

\begin{itemize}
\item zum Wort \emph{abba} aus der Sprache $L_0$.
\item zum Wort \emph{baab} aus der Sprache $L_1$.
\item zum Wort \emph{aabb} aus der Sprache $L_2$.
\end{itemize}

\begin{liAntwort}
\begin{description}
\item[$\text{Ab}_0$] = \liAbleitung{S -> aS -> abA -> abbB -> aabb}
\item[$\text{Ab}_1$] = \liAbleitung{S -> bA -> baS -> baaS -> baab}
\item[$\text{Ab}_2$] = \liAbleitung{S -> aA -> aaB -> aabB -> aabb}
\end{description}

\end{liAntwort}
%%
% (c)
%%

\item Geben Sie zu allen Sprachen einen regulären Ausdruck an.
\index{Reguläre Ausdrücke}

\begin{liAntwort}
\begin{description}
\item[$\text{Reg}_0$] $=$ \texttt{(a|b)*bb(a|b)*}

\item[$\text{Reg}_1$] $=$ \texttt{(b*a+)*b}

\item[$\text{Reg}_2$] $=$ \texttt{(aa(a|b)*)|((a|b)*bb)}
\end{description}
\end{liAntwort}

%%
% (d)
%%

\item Geben Sie zu allen Sprachen einen Automaten an, der die Sprache
akzeptiert.
\index{Deterministisch endlicher Automat (DEA)}

\begin{description}

\item[Automat zu $L_0$]:

\begin{liAntwort}
\begin{center}
\begin{tikzpicture}[li automat]
  \node[state,initial] (z0) at (2.29cm,-2.14cm) {$z_0$};
  \node[state] (z1) at (4.71cm,-2.14cm) {$z_1$};
  \node[state,accepting] (z2) at (7cm,-2.14cm) {$z_2$};

  \path (z0) edge[auto,bend left] node{$b$} (z1);
  \path (z0) edge[auto,loop above] node{$a$} (z0);
  \path (z1) edge[auto] node{$b$} (z2);
  \path (z1) edge[auto,bend left] node{$a$} (z0);
  \path (z2) edge[auto,loop above] node{$a,b$} (z2);
\end{tikzpicture}
\end{center}
\liFlaci{Af75ihbc7}
\end{liAntwort}

\item[Automat zu $L_1$]:

\begin{liAntwort}
\begin{center}
% Automatisch generiert mit Flaci
\begin{tikzpicture}[li automat]
  \node[state,initial] (z0) at (1.57cm,-2.71cm) {$z_0$};
  \node[state] (z1) at (5cm,-2.86cm) {$z_1$};
  \node[state,accepting] (z2) at (4.43cm,-0.86cm) {$z_2$};

  \path (z0) edge[auto,loop above] node{$a$} (z0);
  \path (z0) edge[auto,bend left] node{$b$} (z1);
  \path (z1) edge[auto,bend left] node{$a$} (z0);
  \path (z1) edge[auto] node{$b$} (z2);
  \path (z2) edge[auto,loop above] node{$a,b$} (z2);
\end{tikzpicture}
\end{center}
\liFlaci{Aiq4rcc1c}
\end{liAntwort}

\end{description}

\end{enumerate}
\end{document}
