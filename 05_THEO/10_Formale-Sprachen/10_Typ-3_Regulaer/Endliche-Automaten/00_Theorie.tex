\documentclass{lehramt-informatik-haupt}
\liLadePakete{mathe,syntaxbaum,automaten,formale-sprachen}

\begin{document}

%%%%%%%%%%%%%%%%%%%%%%%%%%%%%%%%%%%%%%%%%%%%%%%%%%%%%%%%%%%%%%%%%%%%%%%%
% Theorie-Teil
%%%%%%%%%%%%%%%%%%%%%%%%%%%%%%%%%%%%%%%%%%%%%%%%%%%%%%%%%%%%%%%%%%%%%%%%

\chapter{Endliche Automaten}

\section{Deterministische endliche Automaten}

Ein deterministischer endlicher Automat (\memph{DEA}; englisch
deterministic finite state machine oder deterministic finite automaton,
\memph{DFA}) ist ein endlicher Automat, der unter \memph{Eingabe eines
Zeichens} seines Eingabealphabetes (den möglichen Eingaben) von einem
Zustand, in dem er sich befindet, in einen \memph{eindeutig bestimmten
Folgezustand} wechselt.
\footcite{wiki:dfa}

Einen endlichen Automaten können wir uns als eine „Black Box“
vorstellen, in die wir etwas eingeben können und die sich aufgrund einer
Folge von Eingaben in einem entsprechenden Zustand befindet. In
Abhängigkeit von ihrem jeweiligen Zustand und einer erfolgten Eingabe
geht sie in einen Folgezustand über. Befindet sich die Black Box nach
der Abarbeitung der Eingabefolge in einem ausgezeichneten Zustand, einem
so genannten Endzustand, dann handelt es sich um eine von ihr
akzeptierte Folge, falls sie sich nicht in einem Endzustand befindet,
hat sie die Eingabe nicht akzeptiert.
\footcite[Seite 11]{vossen}

Automaten sind \memph{deterministisch}, wenn es in jedem Zustand für
jedes Eingabesymbol \memph{höchstens einen Folgezustand} gibt.
\footcite[Seite 28]{vossen}

Ein DEA/DFA ist ein 5-Tupel ($Z, \Sigma, \delta, E, z_0$) mit

\begin{description}
\item[$Z$:] Menge der Zustände (endlich)
\item[$\Sigma$:] Eingabealphabet mit (endlich)
\item[$\delta$:] $Z \times \Sigma \rightarrow Z$ Zustandsübergangsfunktion
\item[$E$:] Menge der Endzustände
\item[$z_0$:] Startzustand\footcite[Seite 26]{theo:fs:1}
\end{description}

%-----------------------------------------------------------------------
%
%-----------------------------------------------------------------------

\section{Nichtdeterministische endliche Automaten}

Ein nichtdeterministischer endlicher Automat (NEA; englisch
nondeterministic finite automaton, NFA) ist ein endlicher Automat, bei
dem es für den \memph{Zustandsübergang mehrere gleichwertige
Möglichkeiten} gibt. Im Unterschied zum deterministischen endlichen
Automaten sind die Möglichkeiten nicht eindeutig, dem Automaten ist also
nicht vorgegeben, welchen Übergang er zu wählen hat.\footcite{wiki:nfa}

\begin{center}
\begin{tikzpicture}[->]
\node[state,initial] (p) {p};
\node[state,accepting,right=of p] (q) {q};

\path (p) edge[above] node{1} (q);

\path (p) edge[loop,above] node{0,1} (q);
\end{tikzpicture}
\end{center}

Nichtdeterministische endliche Automaten sind Automaten, in denen es zu
einem Zustand und einem Eingabesymbol mehrere Folgezustände geben kann.
\footcite[Seite 28]{vossen}

Ein NEA/NFA ist ein 5-Tupel ($Z, \Sigma, \sigma, E, z_0$) mit

\begin{description}
\item[$Z$:] Menge der Zustände (endlich)
\item[$\Sigma$:] Eingabealphabet mit (endlich)
\item[$\delta$:] $Z \times \Sigma \rightarrow 2^Z$ Zustandsübergangsfunktion
\item[$E$:] Menge der Endzustände
\item[$z_0$:] Startzustand\footcite[Seite 30]{theo:fs:1}
\end{description}

Potenzmenge\footcite{wiki:potenzmenge}

\let\p=\liPotenzmengeOhneMathe

Beispiel: $Z = \p{ z_1, z_2 }$

$2^Z = \{\p{}, \p{z_1 }, \p{z_2 }, \p{z_1 , z_2 }\}$

%-----------------------------------------------------------------------
%
%-----------------------------------------------------------------------

\section{Schnitt zweier Automaten}

\url{https://vowi.fsinf.at/images/9/9c/TU_Wien-Formale_Modellierung_VU_%28Salzer%29-L%C3%B6sungen_alte_Tests_-_TU_Wien-Formale_Modellierung_VU_%28Salzer%29_-_%C3%9Cbung_2_-_Musterl%C3%B6sung_%282017WS%29.pdf}

Wir konstruieren einen Automaten A, der die beiden Automaten $A_1$ und
$A_2$ parallel ausführt. Als Zustände für A verwenden wir Paare ($q_1$,
$q_2$), wobei $q_1 \in Q_1$ ein Zustand des ersten Automaten und $q_2$
∈ $Q_2$ ein Zustand des zweiten Automaten ist. Der neue Automat befindet
sich bei Eingabe eines Wortes $w$ im Zustand ($q_1$ , $q_2$), wenn sich
der erste Automat bei diesem Wort im Zustand $q_1$ und der zweite im
Zustand $q_2$ befinden würde. Der Startzustand ($i_1$ , $i_2$)
entspricht der Situation, in der sich die beiden ursprünglichen
Automaten im Startzustand befinden. Ein Übergang mit dem Symbol $s$ von
($q_1$, $q_2$) nach ($q_1$ 0 , $q_2$ 0 ) existiert genau dann, wenn
wenn man mit diesem Symbol in $A_1$
von $q_1$ nach $q_1$ 0 = δ 1 ($q_1$ , s) und in $A_2$ von $q_2$ nach
$q_2$ 0 = δ 2 ($q_2$ , s) gelangt.

Ein Wort wird von beiden Automaten
akzeptiert (und liegt daher im Durchschnitt der Sprachen), wenn der neue
Automat einen Zustand ($q_1$ , $q_2$ ) erreicht, bei dem beide Zustände
Endzustände im jeweiligen Automaten sind, wenn also $q_1 \in F_1$ und
$q_2 \in F_2$ gilt. Ein Automat für den Durchschnitt der Sprachen
L($A_1$ ) und L($A_2$ ) lässt sich somit durch

\literatur

\end{document}
