\documentclass{lehramt-informatik-haupt}
\liLadePakete{mathe}

\begin{document}

%%%%%%%%%%%%%%%%%%%%%%%%%%%%%%%%%%%%%%%%%%%%%%%%%%%%%%%%%%%%%%%%%%%%%%%%
% Theorie-Teil
%%%%%%%%%%%%%%%%%%%%%%%%%%%%%%%%%%%%%%%%%%%%%%%%%%%%%%%%%%%%%%%%%%%%%%%%

\chapter{Entscheidbarkeit}

\begin{liQuellen}
\item \cite[Seite 309-312]{hoffmann}
\item \cite{wiki:entscheidbar}
\item \cite[Kapitel 19.2.3.1 Entscheidbarkeit und Semientscheidbarkeit, Seite 596-597]{schneider}
\end{liQuellen}

\begin{description}
\item[Entscheidbarkeit]

Eine Sprache $L \subseteq \Sigma^*$ heißt entscheidbar, wenn die
charakteristische Funktion $\chi_{L} : \Sigma^* \rightarrow \{0,1\}$ mit

\begin{equation*}
\chi _{L}(\omega)=
\begin{cases}
1,& \text{falls } \omega \in L,\\
0,& \text{falls } \omega \notin L
\end{cases}
\end{equation*}

berechenbar ist.

\item[Semi-Entscheidbarkeit]

Eine Sprache $L \subseteq \Sigma^*$ heißt semi-entscheidbar, wenn die
partielle charakteristische Funktion $\chi_{L}' : \Sigma^* \rightarrow
\{ 1 \}$ mit

\begin{equation*}
\chi _{L}'(\omega)=
\begin{cases}
1,& \text{falls } \omega \in L,\\
\bot,& \text{falls } \omega \notin L
\end{cases}
\end{equation*}

berechenbar ist.
\end{description}

\begin{liExkurs}[Charakteristische Funktion]
Die Indikatorfunktion (auch charakteristische Funktion genannt) ist eine
Funktion in der Mathematik, die sich dadurch auszeichnet, dass sie
nur einen oder zwei Funktionswerte annimmt. Sie ermöglicht es,
komplizierte Mengen mathematisch präzise zu fassen und auf ihnen
Funktionen\footcite{wiki:charakteristische-funktion}
\end{liExkurs}

\section{Wichtige Definitionen}

Jede nichtdeterministische Turing-Maschine kann durch eine
deterministische Turing-Maschine simuliert werden.

Zu jeder Typ-0-Sprache $L$ existiert eine Turing-Maschine, die $L$
akzeptiert.

Zu jeder Typ-1-Sprache $L$ existiert eine nichtdeterministische, linear
beschränkte Turing-Maschine, die $L$ akzeptiert.
\footcite[Seite 35]{theo:fs:4}

\section{Sprache und Abzählbarkeit}

Eine Sprache $L$ heißt

\begin{description}
\item[rekursiv aufzählbar,] falls

\begin{itemize}
\item eine Turing-Maschine $T$ existiert, die $L$ akzeptiert oder

\item $L = \emptyset$ oder

\item eine surjektive und berechenbare Abbildung $f: \mathbb{N}
\leftarrow L$ existiert.
\end{itemize}

\item[abzählbar,] falls eine bijektive Abbildung $f: \mathbb{N}
\leftarrow L$ existiert.

\item[rekursiv oder entscheidbar,] falls eine
Turing-Maschine $T$ existiert, die $L$ akzeptiert und zusätzlich für
jede Eingabe terminiert.
\end{description}
\footcite[Seite 37]{theo:fs:4}

Eine Sprache $L$ ist genau dann entscheidbar, wenn sowohl $L$ als auch
$\overline{L}$ semi-entscheidbar sind.

Eine Sprache ist genau dann aufzählbar, wenn sie semi-entscheidbar ist.

Eine Sprache ist genau dann entscheidbar, wenn $L$ und $\overline{L}$
aufzählbar sind.
\footcite[Seite 39]{theo:fs:4}

\begin{liExkurs}[Abzählbarkeit / Abzählbare Menge]
In der Mengenlehre wird eine Menge $A$ als abzählbar unendlich
bezeichnet, wenn sie die gleiche Mächtigkeit hat wie die Menge der
natürlichen Zahlen $\mathbb{N}$. Dies bedeutet, dass es eine Bijektion
zwischen $A$ und der Menge der natürlichen Zahlen gibt, die Elemente der
Menge $A$ also „durchnummeriert“ werden können.
\liFussnoteUrl{https://de.wikipedia.org/wiki/Abzählbare_Menge}
\end{liExkurs}

\end{document}
