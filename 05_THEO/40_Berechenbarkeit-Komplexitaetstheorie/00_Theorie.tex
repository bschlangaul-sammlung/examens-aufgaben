\documentclass{lehramt-informatik-haupt}

\begin{document}

%%%%%%%%%%%%%%%%%%%%%%%%%%%%%%%%%%%%%%%%%%%%%%%%%%%%%%%%%%%%%%%%%%%%%%%%
% Theorie-Teil
%%%%%%%%%%%%%%%%%%%%%%%%%%%%%%%%%%%%%%%%%%%%%%%%%%%%%%%%%%%%%%%%%%%%%%%%

\chapter{Berechenbarkeit und Komplexitätstheorie}

%%
%
%%

\section{Komplexitätstheorie}

\memph{Komplexität} bezeichnet die „Kompliziertheit“ von Problemen,
Algorithmen und Daten.
%
Die \memph{Komplexitätstheorie} befasst sich dabei mit dem
Ressourcenverbrauch von Algorithmen.
%
Die betrachteten Ressourcen sind fast immer die Anzahl der
benötigten Rechenschritte (\memph{Zeitkomplexität}) oder der
Speicherbedarf (\memph{Platzkomplexität}).
%
Algorithmen und Probleme werden in der Komplexitätstheorie gemäß ihrer
so bestimmten Komplexität in so genannte \memph{Komplexitätsklassen}
eingeteilt. Diese sind ein wichtiges Werkzeug, um bestimmen zu können,
welche Probleme \emph{„gleich schwierig“} sind.
\footcite[Seite 58]{theo:fs:4}

\section{Komplexitätsklassen}

\begin{description}
\item[P]
enthält alle Probleme, für die eine \memph{deterministische}
Turingmaschine in polynomieller Zeit eine Lösung berechnen kann.
\footcite[Seite 59]{theo:fs:4}

\item[NP]
enthält alle Probleme, für die eine \memph{nicht-deterministische} (= NP)
Turingmaschine in polynomieller Zeit eine Lösung berechnen kann.
\footcite[Seite 60]{theo:fs:4}

\begin{description}
\item[NP-schwer] (engl. NP-hard) ist eine Eigenschaft eines
algorithmischen Problems. Ein NP-schweres Problem ist dabei mindestens
genauso „schwer“ wie alle Probleme in NP. Das bedeutet, dass ein
Algorithmus, der ein NP-schweres Problem löst, mithilfe einer Reduktion
benutzt werden kann, um alle Probleme in NP zu lösen.\footcite{wiki:np-schwer}

\item[NP-vollständig] sind die Probleme, die in NP liegen
und NP-schwer sind.\footcite{wiki:np-vollstaendig}
\end{description}
\end{description}

%-----------------------------------------------------------------------
%
%-----------------------------------------------------------------------

\subsection{Satz von Cook}

Stephen A. Cook zeigte, dass eine Teilmenge der Klasse NP existiert, auf
die sich alle Probleme aus NP reduzieren lassen. Diese Teilmenge ist
damit repräsentativ für die Schwierigkeit von NP und wird als
NP-vollständig (englisch: NP complete) bezeichnet. Der nach ihm benannte
Satz von Cook sagt aus, dass das Erfüllbarkeitsproblem der Aussagenlogik
(SAT, v. engl. satisfiability) NP-vollständig ist. Es ist allerdings
nicht das einzige schwerste Problem, denn Richard M. Karp zeigte, dass
es in NP Probleme gibt, auf die SAT reduziert werden kann, die also
genauso schwer sind wie SAT.
\footcite[Seite 70]{theo:fs:4}

\subsection{SAT und k-SAT}

SAT und k-SAT mit $k \geq 3$, $k \in \mathbb{N}$ (Satz von Cook)

Das Problem fragt, ob eine aussagenlogische Formel erfüllbar ist.
Anwendungen: Komplexitätstheorie, Verifikation und Entwurf von logischen
Schaltungen Das Erfüllbarkeitsproblem der Aussagenlogik ist in
exponentieller Zeit in Abhängigkeit der Anzahl der Variablen
entscheidbar Wahrheitstabelle. Diese Wahrheitstabelle kann nicht in
polynomieller Zeit aufgestellt werden.
\footcite[Seite 71]{theo:fs:4}

\section{Karps 21 NP-vollständige Probleme}

Nachdem Stephen Cook 1971 den Nachweis erbrachte, dass SAT
NP-vollständig ist, griff Richard Karp 1972 diese Idee auf und zeigte
die NP-Vollständigkeit für 21 weitere Probleme. Diese werden als
klassische NP-vollständige Probleme bezeichnet.\footcite{wiki:karps-21}

Seit 1972 wurden mehr als 10000 Probleme als NP-vollständig
charakterisiert.
\footcite[Seite 80]{theo:fs:4}

%%
%
%%

\subsection{Problem des Handlungsreisenden}

(TSP: engl. traveling salesman problem, Optimierungsproblem der
Kombinatorik): Die Aufgabe besteht darin, eine Reihenfolge für den
Besuch mehrerer Orte so zu wählen, dass keine Station außer der ersten
mehr als einmal besucht wird, die gesamte Reisestrecke des
Handlungsreisenden möglichst kurz und die

erste Station gleich der letzten Station ist. Eingabe: Ungerichteter
Graph mit Längen als Beschriftungen für Kanten

Ausgabe: Kürzeste Tour, auf der alle Knoten

genau einmal vorkommen\footcite[Seite 73]{theo:fs:4}

%%
%
%%

\subsection{Hamiltonkreisproblem}

Existiert ein geschlossener Pfad in einem
Graphen, der jeden Knoten nur genau einmal enthält?

Wichtiges Problem der Graphentheorie
Ist eine Verallgemeinerung des TSP, bei welchem nur nach dem kürzesten
Hamiltonkreis in einem Graphen gefragt wird.

%%
%
%%

\subsection{Teilsummenproblem}

(auch Untermengensummenproblem, SSP engl. subset sum problem)

Gegeben sei eine Menge von ganzen Zahlen $M = \{z_1 , z_2 , \dots, z_n
\}$. Gesucht ist eine Untermenge, deren Elementsumme maximal, aber nicht
größer als eine gegebene obere Schranke c ist (oft ist auch gefragt, die
Schranke c exakt zu erreichen). Es ist ein spezielles Rucksackproblem.
\footcite[Seite 74]{theo:fs:4}

%%
%
%%

\subsection{Rucksackproblem}

(auch englisch knapsack problem) ist ein Optimierungsproblem der
Kombinatorik.

Aus einer Menge von Objekten, die jeweils ein Gewicht und einen Nutzwert
haben, soll eine Teilmenge ausgewählt werden, deren Gesamtgewicht eine
vorgegebene Gewichtsschranke nicht überschreitet. Unter dieser Bedingung
soll der Nutzwert der ausgewählten Objekte maximiert werden.
\footcite[Seite 75]{theo:fs:4}

%%
%
%%

\subsection{Cliquenproblem}

(CLIQUE) fragt nach der Existenz einer Clique der Mindestgröße n in
einem gegebenen Graphen.

Eine Clique ist eine Teilmenge von Knoten in einem ungerichteten
Graphen, bei der jedes Knotenpaar durch eine Kante verbunden ist
(informell: jeder mit jedem innerhalb des Teilgraphen)

Beispiel: In dem nebenstehenden Graphen wird mit der Brute-Force-Methode
nach einer 4er-Clique gesucht.
\footcite[Seite 76]{theo:fs:4}

%%
%
%%

\subsection{Stabilitätsproblem}

(Independent Set oder Co-Clique) fragt nach der Existenz einer Teilmenge
von Knoten in einem gegebenen Graphen der Mindestgröße n , bei denen die
Knoten zueinander nicht adjazent (=keine direkte Verbindung) sind.

Das Stabilitätsproblem ist das „Gegenproblem“ zum Cliquenproblem

Beispiel: Die neun blauen Knoten bilden im nebenstehenden Graphen eine
(maximale) stabile Menge.\footcite[Seite 77]{theo:fs:4}

%%
%
%%

\subsection{COL und k-COL}

COL und k-COL mit $k \geq 3$, $k \in \mathbb{N}$ (graph coloring –
Färbung eines Graphen) fragt, ob ein gegebener Graph mit k Farben so
färbbar ist, dass zwei benachbarte Knoten niemals die gleiche Farben
haben.

\literatur

\end{document}
