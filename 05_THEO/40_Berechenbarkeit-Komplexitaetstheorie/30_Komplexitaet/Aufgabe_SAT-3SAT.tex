\documentclass{lehramt-informatik-aufgabe}
\liLadePakete{}
\begin{document}
\liAufgabenTitel{SAT-3SAT}
\section{SAT-3SAT
\index{Polynomialzeitreduktion}
\footcite[Aufgabe 14, Seite 18]{theo:ab:4}}

\begin{enumerate}

%%
% (a)
%%

\item Wie zeigt man die aus der NP-Schwere des 3SAT-Problems die NP-Schwere des
SAT-Problems?

\begin{liAntwort}
Reduktion 3SAT ≤ p SAT : Jede 3SAT-Problem ist auch ein SAT-Problem,
weil 3SAT ⊂ SAT → Damit braucht es keine Funktion (bzw.
Identitäts-/Einheitsfunktion). Die Funktion ist korrekt, total und in
Polynomialzeit anwendbar. → SAT-Problem ist ebenfalls NP- schwer.
\end{liAntwort}

%%
% (b)
%%

\item Wie zeigt man die aus der NP-Schwere des SAT-Problems die NP-Schwere des
3SAT-Problems?

\begin{liAntwort}
Reduktion SAT ≤ p 3SAT : Man muss eine Funktion finden, die eine
allgemeine Aussagenlogik in eine Aussagenlogik mit 3 Literalen in
konjunktiver Normalform umformt.

Durch die boolsche Algebra lässt sich jede logische Aussagenlogik in
eine konjunkti- ve Normalform bringen. Dies ist eine Konjunktion von
Disjunktionstermen. Wir formen einen Disjunktionsterm mithilfe einer
Funktion in ein 3SAT-Problem um. Diese Funktion kann auf jeden
Disjunktionsterm angewendet werden und damit wird das gesamte
SAT-Problem auf 3SAT reduzieren.

Die Funktion formt Formel aus SAT mithilfe von Hilfsvariablen h 1 , . .
. , h n−2 derart um (a 1 ∨ . . . ∨ a n ) → (a 1 ∨ a 2 ∨ h 1 ) ∧ (¬h 1 ∨
a 3 ∨ h 2 ) ∧ (¬h 2 ∨ a 4 ∨ h 3 ) ∧ . . . ∧ (¬h n−2 ∨ a n )

Diese Funktion ist total, denn jede in SAT enthaltene Aussagenlogik kann
so umgewandelt werden.

Korrektheit: Die Hilfsvariablen sind wahr, solange bis ein Literal a x
selber true ist. Ab diesem Zeitpunkt sind dann die Hilfsvariablen dann
falsch. JA-Instanzen: Der erste und alle mittleren Disjunktionstermen
sind wahr, weil aufgrund der Nicht-Negierung und Negierung immer ein
wahres Literal in den Disjunktionster- men. Somit ist dann auch der
Disjunktionsterm wahr. Da es eine JA-Instanz ist, existiert ein a x
welches wahr ist. Somit sind ab diesem Zeitpunkt die Hilfvariablen
falsch. Der letzte Disjunktionsterm wird dadurch sicher wahr, weil ¬h
n−2 somit wahr ist. NEIN-Instanz: Alle a x sind falsch. Auch hier sind
wieder der erste und alle mittleren Dis- junktionsterme wahr (gleiche
Begründung wie oben). Der letzte Disjunktionsterm ist al- lerdings
falsch, weil die Hilfvariablen durchgehend wahr bleiben und alle a x
falsch sind. Durch die Konjunktion der Disjunktionsterme ist dann auch
die Gesamtaussage falsch. Polynomialzeit: Der Algorithmus, der Formeln
aus SAT nach 3SAT umformt liegt in O(n) und somit in Polynomialzeit.
\end{liAntwort}
\end{enumerate}

\end{document}
