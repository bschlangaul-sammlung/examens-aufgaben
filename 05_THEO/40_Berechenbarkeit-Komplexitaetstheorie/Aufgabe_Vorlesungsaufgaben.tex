\documentclass{lehramt-informatik-aufgabe}
\liLadePakete{syntax}
\begin{document}
\liAufgabenTitel{Vorlesungsaufgaben}
\section{Vorlesungsaufgaben
\index{Berechenbarkeit}
\footcite{theo:fs:4}}

%-----------------------------------------------------------------------
%
%-----------------------------------------------------------------------

\section{LOOP-Implementierung\footcite[Seite 11]{theo:fs:4}}

\begin{enumerate}
\item Geben Sie eine LOOP-Implementierung für

\begin{enumerate}

%%
%
%%

\item $add(x_i, x_j)$

\begin{liAntwort}
\begin{minted}{md}
x_0 := x_i;
loop x_j do
  x0 := succ(x_0);
end
\end{minted}
\end{liAntwort}

%%
%
%%

\item $mult(x_i, x_j)$

\begin{liAntwort}
\begin{minted}{md}
x_0 := x_i;
loop x_j do
  x0 := add(x_0, x_i);
end
\end{minted}
\end{liAntwort}

%%
%
%%

\item $power(x_i, x_j)$

\begin{liAntwort}
\begin{minted}{md}
x_0 := succ(0);
loop x_j do
  x0 := mult(x_0, x_i);
end
\end{minted}
\end{liAntwort}

%%
%
%%

\item $hyper(x_i, x_j)$

\begin{liAntwort}
\begin{minted}{md}
x_0 := succ(0);
loop x_j do
  x0 := power(x_i, x_0);
end
\end{minted}
\end{liAntwort}

\item $2^{x_i}$
\end{enumerate}

an.

\item Beweisen Sie, dass der größte gemeinsame Teiler zweier natürlicher
Zahlen LOOP-berechenbar ist.

\end{enumerate}

%-----------------------------------------------------------------------
%
%-----------------------------------------------------------------------

\section{WHILE-Programm\footcite[Seite 16]{theo:fs:4}}

Gebe ein WHILE-Programm an, dass

\begin{itemize}
\item $2^{x_i}$
\item $ggt(x_i, x_j)$
\end{itemize}

berechnet.

%-----------------------------------------------------------------------
%
%-----------------------------------------------------------------------

\section{Turing-berechenbar\footcite[Seite 29]{theo:fs:4}}

\begin{enumerate}
\item Zeige, dass es nur abzählbar viele Turingmaschinen gibt.

\item Turing-berechenbar

\begin{enumerate}
\item Definiere eine berechenbare Funktion $f: N \rightarrow N$ mit
entscheidbarem

\item Definitionsbereich und unentscheidbarem Wertebereich. Untersuche
folgende Aussagen

\begin{enumerate}

\item Jede berechenbare Funktion $h: N \rightarrow N$ mit endlichem
Wertebereich besitzt einen entscheidbaren Definitionsbereich.

\item Jede berechenbare Funktion $g: N \rightarrow N$ mit endlichem
Definitionsbereich besitzt einen entscheidbaren Wertebereich.
\end{enumerate}
\end{enumerate}
\end{enumerate}
\end{document}
