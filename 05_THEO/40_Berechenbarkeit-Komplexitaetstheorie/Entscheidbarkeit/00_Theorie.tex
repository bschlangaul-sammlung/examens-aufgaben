\documentclass{lehramt-informatik-haupt}

\begin{document}

%%%%%%%%%%%%%%%%%%%%%%%%%%%%%%%%%%%%%%%%%%%%%%%%%%%%%%%%%%%%%%%%%%%%%%%%
% Theorie-Teil
%%%%%%%%%%%%%%%%%%%%%%%%%%%%%%%%%%%%%%%%%%%%%%%%%%%%%%%%%%%%%%%%%%%%%%%%

\chapter{Entscheidbarkeit}

\section{Wichtige Definitionen}

Jede nichtdeterministische Turing-Maschine kann durch eine
deterministische Turing-Maschine simuliert werden.

Zu jeder Typ-0-Sprache $L$ existiert eine Turing-Maschine, die $L$
akzeptiert.

Zu jeder Typ-1-Sprache $L$ existiert eine nichtdeterministische, linear
beschränkte Turing-Maschine, die $L$ akzeptiert.
\footcite[Seite 35]{theo:fs:4}

Eine Sprache $L$ heißt

\memph{rekursiv aufzählbar}, falls eine Turing-Maschine $T$ existiert,
die $L$ akzeptiert.

oder

falls $L = \emptyset$ oder eine surjektive und berechenbare Abbildung
$f: \mathbb{N} \leftarrow L$ existiert.

\memph{abzählbar}, falls eine bijektive Abbildung $f: \mathbb{N}
\leftarrow L$ existiert.

\memph{rekursiv} oder \memph{entscheidbar}, falls eine Turing-Maschine
$T$ existiert, die $L$ akzeptiert und zusätzlich für jede Eingabe
terminiert.
\footcite[Seite 37]{theo:fs:4}

Eine Sprache $L$ ist genau dann entscheidbar, wenn sowohl $L$ als auch $\overline{L}$
semi-entscheidbar sind.

Eine Sprache ist genau dann aufzählbar, wenn sie semi-entscheidbar ist.

Eine Sprache ist genau dann entscheidbar, wenn $L$ und $\overline{L}$
aufzählbar sind.
\footcite[Seite 39]{theo:fs:4}

\end{document}
