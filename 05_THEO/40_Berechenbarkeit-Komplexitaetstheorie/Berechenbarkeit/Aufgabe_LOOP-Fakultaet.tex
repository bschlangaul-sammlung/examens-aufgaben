\documentclass{lehramt-informatik-aufgabe}
\liLadePakete{}
\begin{document}
\liAufgabenTitel{LOOP-Fakultät}
\section{
\index{}
\footcite[Aufgabe 1]{theo:ab:4}}

\begin{enumerate}

%%
%
%%

\item Geben Sie ein LOOP-Programm an, das die Funktion $f(n) = n!$
berechnet.

\begin{liAntwort}
\begin{minted}{md}
x2 := 1;
LOOP x1 DO
x3 := x3 + 1;
x2 := x2 * x3 ; END
x3 := 0;
RETURN x2;
\end{minted}
\end{liAntwort}

%%
% (b)
%%

\item

Beweisen Sie:

Ist $f : N \rightarrow N$ LOOP-berechenbar, so ist auch
$g : N \rightarrow N$ mit $g(n) = f (i)$
LOOP-berechenbar.

\begin{liAntwort}
Bei einem LOOP-Programm der Form LOOP xi DO P END wird das Programm P so
oft ausgeführt, wie der Wert der Variablen xi zu Beginn angibt. Beweis:

\begin{minted}{md}
x 0 := 0; i := 0;
LOOP n DO
i := i + 1;
y := f(i);
x 0 := x 0 + y;
END
RETURN x 0
\end{minted}

ist LOOP-berechenbar, da $f (n)$ LOOP-berechenbar ist.
\end{liAntwort}

\end{enumerate}
\end{document}
