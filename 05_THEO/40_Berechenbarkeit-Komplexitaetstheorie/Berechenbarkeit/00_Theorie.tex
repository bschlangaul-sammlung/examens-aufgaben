\documentclass{lehramt-informatik-haupt}

\begin{document}

%%%%%%%%%%%%%%%%%%%%%%%%%%%%%%%%%%%%%%%%%%%%%%%%%%%%%%%%%%%%%%%%%%%%%%%%
% Theorie-Teil
%%%%%%%%%%%%%%%%%%%%%%%%%%%%%%%%%%%%%%%%%%%%%%%%%%%%%%%%%%%%%%%%%%%%%%%%

\chapter{Berechbarkeitstheorie}

\begin{liQuellen}
\item \cite[Seite 253-340]{hoffmann}
\end{liQuellen}

\subsection{Berechenbarkeitsmodelle}

%%
%
%%

% Info_2021-05-07-2021-05-07_09.32.08.mp4 1h

\subsubsection{Loop-berechenbar}

\begin{liQuellen}
\item \cite[Seite 7-11]{theo:fs:4}
\item \cite[Seite 254-260]{hoffmann}
\item \cite{wiki:loop}
\end{liQuellen}

\paragraph{LOOP-Sprache (einfache Programme)}

\begin{itemize}
\item endlicher Aufbau
\item Variablen aus den natürlichen Zahlen
\end{itemize}

\begin{description}
\item[Programmelemente:] \strut

\begin{description}
\item[Konstante:]
$0$ (andere Kostanten müssen aus den anderen Element )

\item[Variablen:]
$x_0, x_1, x_2, \dots, x_n$

\item[Operator:]

$succ$ (Successor), $pred$ (Sredecessor), $:=$ (Wertzuweisen), $;$ (Anweisungtrenner)

\item[Schleife:]

$loop \dots do \dots end$ (nach dem loop Schlüsselwort steht genau eine
Variable, die dekrementiert wird. Ist 0 erreicht, stop die Schleife)
\end{description}

\item[Speichervektor v:] \strut

\begin{itemize}
\item endlich, aber beliebig lang
\item Speicherung der Variablen des Programms
\item $(x_0, x_1, x_2, \dots, x_n)$
\item Erste Stelle des Vektors ist für das Speichern des Ergebnisses vorgesehen.
\end{itemize}

\item[Übergangsfunktion $\delta(v, P)$:] \strut

\begin{description}
\item[Eingabe:] Speichervektor $v$ und Programm $P$
\item[Ausgabe:] Speichervektor $v'$
\end{description}

nach Programmausführung von $P$

\item[Makro] \strut

Es gibt die Möglichkeit sogenannte Makros zu definieren.
\end{description}
%%
%
%%

\subsubsection{While-Programme}

\begin{liQuellen}
\item \cite[Seite 7-12]{theo:fs:4}
\item \cite[Seite 260-264]{hoffmann}
\item \cite{wiki:while}
\end{liQuellen}

\literatur

\end{document}
