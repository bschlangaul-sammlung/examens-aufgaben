\documentclass{lehramt-informatik-aufgabe}
\liLadePakete{mathe}
\begin{document}
\liAufgabenTitel{WHILE-berechenbar}
\section{WHILE-berechenbar
\index{WHILE-berechenbar}
\footcite[Aufgabe 2]{theo:ab:4}}

Bestimme jeweils, ob die angegebene Funktion WHILE-berechenbar ist:

\begin{enumerate}

%%
% (a)
%%

\item $x \rightarrow 2^x$

\begin{liAntwort}
erg = 1;
WHILE x 6 = 0 DO
erg = erg * 2;
x = x - 1;
END;
return erg;
\end{liAntwort}

%%
% (b)
%%

\item $\text{ggT}(n, m)$,
also der größte gemeinsame Teiler. Sie dürfen die (ganzzahligen) Operationen
+, −, ∗ und / verwenden, wobei das Minus, wie üblich, eingeschränkt ist.

\begin{liAntwort}
Es bietet sich an, zunächst die modulo Operation x i := x j % x k durch folgendes WHILE-
Programm zu definieren:
x n+1 := x j /x k ;
x n+2 := x n+1 ∗ x k ;
x i := x j − x n+2 ;
Wobei x n+1 und x n+2 im Rest des Programmes nicht verwendet werden sollen. Mit der
Modulo Operation kann man nun z.B. einfach den euklidischen Algorithmus verwenden
(Eingabe seien x 1 und x 2 , Ausgabe ist x 1 :
WHILE x 2 6 = 0 DO
x 3 := x1 % x 2 ;
x 1 := x 2 + 0;
x 2 := x 3 + 0;
END
\end{liAntwort}

%%
% (c)
%%

\item if x i 6 = 0 then P 1 else P 2 fi mit der üblichen Semantik.
Als Nachweis kann jeweils ein WHILE-Programm angegeben werden.

\begin{liAntwort}
Sei x n die höchste in P 1 bzw. P 2 vorkommende Variable (o. E. i ≤ n).
x n+1 := x i + 0;
x n+2 := 1;
WHILE x n+1 6 = 0 DO
x n+1 := 0;
x n+2 := 0;
P 1 ;
END
WHILE x n+2 6 = 0 DO
x n+2 := 0;
P 2 ;
\end{liAntwort}

\end{enumerate}
\end{document}
