\documentclass{lehramt-informatik-haupt}
\liLadePakete{formale-sprachen}

\begin{document}

%%%%%%%%%%%%%%%%%%%%%%%%%%%%%%%%%%%%%%%%%%%%%%%%%%%%%%%%%%%%%%%%%%%%%%%%
% Theorie-Teil
%%%%%%%%%%%%%%%%%%%%%%%%%%%%%%%%%%%%%%%%%%%%%%%%%%%%%%%%%%%%%%%%%%%%%%%%

\chapter{Kontextsensitive und unbeschränkte Sprachen}

\begin{liQuellen}
\item \cite[Seite 191-192]{hoffmann}
\end{liQuellen}

%-----------------------------------------------------------------------
%
%-----------------------------------------------------------------------

\section{Grammatik kontextsensitive Sprachen}

Sei $\Sigma$ ein Alphabet. Eine formale Sprache $L$ ist eine Teilmenge
aller Wörter über $\Sigma$:

\begin{displaymath}
L \subseteq \Sigma^*
\end{displaymath}

\bigskip

\noindent
Eine Grammatik ist ein 4-Tupel mit \liGrammatik{} und besteht aus:

\begin{itemize}
\item Einer endlichen Menge $V$ von \memph{Variablen} (Nonterminale)

\item Dem endlichen \memph{Terminalalphabet} $\Sigma$ mit $\Sigma \cap V
= \emptyset$

\item Der endlichen Menge an \memph{Produktionen}

\item Und einer \memph{Startvariablen} $S$ mit $S \in V$
\end{itemize}

Eine kontextsensitive Sprache wird durch eine kontextsensitive
Grammatik erzeugt, d.\,h. eine Grammatik mit Produktionsregeln der
Form:

$S \rightarrow \epsilon$ oder
$aA \rightarrow ac$ oder
$Ab \rightarrow ab$ oder
$AB \rightarrow BC$ oder
$aBc \rightarrow abc$

Mit $A,B,C \in V$; $a,b,c \in \Sigma$

\memph{linke Seite: Nonterminale und Terminale}

\memph{rechte Seite: $\epsilon$, Terminale, Nonterminale}

Die Produktionsregeln dürfen hierbei die linke Seite allerdings nicht
verkürzen (Ausnahme $S \rightarrow \epsilon$).\footcite{theo:fs:3}

%-----------------------------------------------------------------------
%
%-----------------------------------------------------------------------

\section{Abschlusseigenschaften}

Die kontextsensitiven Sprachen sind abgeschlossen unter:

\begin{itemize}
\item Vereinigung
\item Schnitt
\item Komplement
\item Produkt
\item Kleene-Stern
\end{itemize}

Für kontextsensitive Sprachen ist entscheidbar:

\begin{itemize}
\item Wortproblem
\end{itemize}

In der theoretischen Informatik ist eine rekursiv aufzählbare Sprache
(auch bekannt als semientscheidbare oder erkennbare Sprache) $L$ dadurch
definiert, dass es eine Turingmaschine gibt, die alle Wörter aus $L$
akzeptiert, aber keine Wörter, die nicht in $L$ liegen. Im Unterschied
zu rekursiven Sprachen (entscheidbare Sprachen) muss bei den rekursiv
aufzählbaren Sprachen die Turingmaschine nicht halten, wenn ein Wort
nicht in $L$ liegt. Das heißt, unter Umständen muss man auf die Lösung
unendlich lange warten. Alle rekursiven Sprachen sind deshalb auch
rekursiv aufzählbar.

Rekursiv aufzählbare Sprachen bilden die oberste Stufe der
Chomsky-Hierarchie und heißen deshalb auch Typ-0-Sprachen; die
entsprechenden Grammatiken sind die Typ-0-Grammatiken. Sie können somit
auch als all die Sprachen definiert werden, deren Wörter sich durch eine
beliebige formale Grammatik ableiten lassen.
\footcite{wiki:rekursiv-aufzaehlbare-sprache}

\literatur

\end{document}
