\documentclass{lehramt-informatik-aufgabe}
\liLadePakete{formale-sprachen,mathe}
\begin{document}
\liAufgabenTitel{Vorlesungsaufgaben}
\section{
\index{Kontextsensitive Sprache}
\footcite{theo:fs:3}}

\section{Übung kontextsensitive Grammatiken\footcite[Seite 8]{theo:fs:3}}

Gib die Sprache an, die die folgende Grammatik erzeugt:

$G = (V, \Sigma, P, S)$ mit
$V = \{S, B, C\}$, \liAlphabet{a, b, c},
$S=S$ und

\begin{liProduktionsRegeln}
S -> aSBC | aBC,
CB -> BC,
aB -> ab,
bB -> bb,
bC -> bc,
cC -> cc,
\end{liProduktionsRegeln}

%-----------------------------------------------------------------------
%
%-----------------------------------------------------------------------

\section{Übung zu Turingmaschinenn\footcite[Seite 24]{theo:fs:3}}

\begin{enumerate}
\item Gib eine Turingmaschine an, die die Eingabe über dem Alphabet
$\Sigma = \{ a, b \}$ umkehrt.

Beispiele:

\begin{itemize}
\item abb -> bba
\item aaaaba -> abaaaa
\item aaa -> aaa
\end{itemize}

Tipp:

\begin{itemize}
\item Füge ein extra Zeichen ein, welches das Eingabewort von deinem
umgedrehten Wort trennt.

\item Das Ergebniswort muss nicht an derselben Stelle wie das
Eingabewort stehen.
\end{itemize}

\item Gib anschließend eine Konfigurationsfolge deiner TM für ab an.

\end{enumerate}

%-----------------------------------------------------------------------
%
%-----------------------------------------------------------------------

\section{Übung zu Mehrbandturingmaschinen\footcite[Seite 28]{theo:fs:3}}

Gib eine 2-Bandturingmaschine an, die die Eingabe über dem
Alphabet \liAlphabet{a, b} umkehrt.

Beispiele:

\begin{itemize}
\item abb -> bba
\item aaaaba -> abaaaa
\item aaa -> aaa
\end{itemize}

Tipp: Das Ergebniswort muss nicht auf dem 1. Band stehen.

\end{document}
