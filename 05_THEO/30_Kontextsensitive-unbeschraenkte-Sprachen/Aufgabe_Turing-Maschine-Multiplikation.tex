\documentclass{lehramt-informatik-aufgabe}
\liLadePakete{}
\begin{document}
\liAufgabenTitel{Turing-Maschine Multiplikation}
\section{Turing-Maschine zur Multiplikation mit 3
\index{Turing-Maschine}
\footcite{theo:ab:3}}

Gesucht ist eine Turing-Maschine, die die Funktion $f : \mathbf{N}
\rightarrow \mathbf{N}$ mit $f(x) = 3x$ berechnet. Zu Beginn der
Berechnung steht die Eingabe binär codiert auf dem Band, wobei der Kopf
auf die linkeste Ziffer (most significant bit) zeigt. Am Ende der
Berechnung soll der Funktionswert binär codiert auf dem Band stehen,
wobei der Kopf auf ein beliebiges Feld zeigen darf.

\begin{enumerate}

%%
% (a)
%%

\item Überlege, was bei der Multiplikation mit 3 im Binären tatsächlich
passiert. Leite hieraus die Arbeitsweise des Algorithmus für die
Turingmaschine ab und beschreibe diese. Tipp: Die schriftliche
Multiplikation mit Binärzahlen funktioniert genauso wie die schriftliche
Multiplikation mit Dezimalzahlen!

%%
% (b)
%%

\item Erstelle dazu eine TM mit 3 Bändern. Die ersten beiden Bänder
sollen für die Berechnung herangezogen werden. Auf dem dritten Band soll
das Ergebnis stehen. (analog zur schriftlichen Multiplikation)

%%
% (c)
%%

\item Erstelle dazu eine TM mit 2 Bändern. Auf dem ersten Band steht die
Eingabe und auf dem zweiten Band soll das Ergebnis stehen.
\end{enumerate}
\end{document}
