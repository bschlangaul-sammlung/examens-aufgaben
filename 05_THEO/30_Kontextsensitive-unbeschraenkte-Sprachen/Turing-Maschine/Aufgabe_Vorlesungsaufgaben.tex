\documentclass{lehramt-informatik-aufgabe}
\liLadePakete{formale-sprachen,mathe,automaten}
\begin{document}
\liAufgabenTitel{Vorlesungsaufgaben}
\section{
\index{Kontextsensitive Sprache}
\footcite{theo:fs:3}}

% Info_2021-04-23-2021-04-23_09.35.52 2h25min

\section{Übung zu Turingmaschinen\footcite[Seite 22]{theo:fs:3}}

Gegeben ist eine Binärzahl auf dem Band einer Turingmaschine.

\begin{enumerate}
\item Definiere vollständig eine TM, die das Komplement der Binärzahl
(0110 -> 1001) berechnet. Die Überführungsfunktion kann als Tabelle oder
als Graph angegeben werden.

\begin{liAntwort}
\liFlaci{Ap9qtjgg7}
\end{liAntwort}

\item Erweitere deine Maschine aus Aufgabe a) so, dass der
Schreib-/Lesekopf auf dem ersten Zeichen der Eingabe terminiert.

\begin{liAntwort}
\liFlaci{A5o7tug5r}
\end{liAntwort}
\end{enumerate}
\end{document}
