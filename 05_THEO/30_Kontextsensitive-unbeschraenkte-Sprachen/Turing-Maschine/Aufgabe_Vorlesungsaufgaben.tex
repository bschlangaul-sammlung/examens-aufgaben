\documentclass{lehramt-informatik-aufgabe}
\liLadePakete{formale-sprachen,mathe,automaten}
\begin{document}
\liAufgabenTitel{Vorlesungsaufgaben}
\section{
\index{Kontextsensitive Sprache}
\footcite{theo:fs:3}}

% Info_2021-04-23-2021-04-23_09.35.52 2h25min

\section{Übung zu Turingmaschinen\footcite[Seite 22]{theo:fs:3}}

Gegeben ist eine Binärzahl auf dem Band einer Turingmaschine.

\begin{enumerate}
\item Definiere vollständig eine TM, die das Komplement der Binärzahl
(0110 -> 1001) berechnet. Die Überführungsfunktion kann als Tabelle oder
als Graph angegeben werden.

\begin{liAntwort}
\liFlaci{Ap9qtjgg7}
\end{liAntwort}

\item Erweitere deine Maschine aus Aufgabe a) so, dass der
Schreib-/Lesekopf auf dem ersten Zeichen der Eingabe terminiert.

\begin{liAntwort}
\liFlaci{A5o7tug5r}
\end{liAntwort}

\end{enumerate}

%-----------------------------------------------------------------------
%
%-----------------------------------------------------------------------

\section{Übung zu Turingmaschinenn\footcite[Seite 27]{theo:fs:3}}

% Info_2021-04-23-2021-04-23_13.17.40 3h10

\begin{enumerate}
\item Gib eine Turingmaschine an, die die Eingabe über dem Alphabet
$\Sigma = \{ a, b \}$ umkehrt.

Beispiele:

\begin{itemize}
\item abb -> bba
\item aaaaba -> abaaaa
\item aaa -> aaa
\end{itemize}

Tipp:

\begin{itemize}
\item Füge ein extra Zeichen ein, welches das Eingabewort von deinem
umgedrehten Wort trennt.

\item Das Ergebniswort muss nicht an derselben Stelle wie das
Eingabewort stehen.
\end{itemize}

\item Gib anschließend eine Konfigurationsfolge deiner TM für $ab$ an.

\end{enumerate}

\begin{liAntwort}
\begin{center}
\begin{tikzpicture}[li turingmaschine]
  \node[state,initial] (z0) at (1.86cm,-5.29cm) {$z_0$};
  \node[state] (z1) at (3.86cm,-5.29cm) {$z_1$};
  \node[state] (z2) at (5.71cm,-5.29cm) {$z_2$};
  \node[state] (z3) at (8.29cm,-4.43cm) {$z_3$};
  \node[state] (z4) at (8cm,-7.43cm) {$z_4$};
  \node[state] (z5) at (12.57cm,-5.57cm) {$z_5$};
  \node[state] (z6) at (7.57cm,-10.43cm) {$z_6$};
  \node[state] (z7) at (11.43cm,-10.43cm) {$z_7$};
  \node[state,accepting] (z8) at (13.86cm,-10.43cm) {$z_8$};

  \liTuringKante[above]{z0}{z1}{
    a, a, L;
    b, b, L;
  }

  \liTuringKante[above,loop above]{z0}{z0}{
    LEER, LEER, N;
  }

  \liTuringKante[above]{z1}{z2}{
    LEER, |, R;
  }

  \liTuringKante[above]{z2}{z3}{
    a, |, L;
  }

  \liTuringKante[above]{z2}{z4}{
    b, |, L;
  }

  \liTuringKante[above,loop above]{z3}{z3}{
    b, b, L;
    a, a, L;
    |, |, L;
  }

  \liTuringKante[above]{z3}{z5}{
    LEER, a, R;
  }

  \liTuringKante[above]{z4}{z5}{
    LEER, b, R;
  }

  \liTuringKante[above,loop above]{z4}{z4}{
    |, |, L;
    a, a, L;
    b, b, L;
  }

  \liTuringKante[above]{z5}{z6}{
    |, |, R;
  }

  \liTuringKante[above,loop above]{z5}{z5}{
    a, a, R;
    b, b, R;
  }

  \liTuringKante[above,loop above]{z6}{z6}{
    |, |, R;
  }

  \liTuringKante[above,bend left]{z6}{z2}{
    a, a, N;
    b, b, N;
  }

  \liTuringKante[above]{z6}{z7}{
    LEER, LEER, L;
  }

  \liTuringKante[above,loop above]{z7}{z7}{
    |, LEER, L;
  }

  \liTuringKante[above]{z7}{z8}{
    b, b, N;
    a, a, N;
  }
\end{tikzpicture}
\end{center}
\liFlaci{Af75rdjbc}
\end{liAntwort}
\end{document}
