\documentclass{lehramt-informatik-aufgabe}
\liLadePakete{formale-sprachen,mathe}
\begin{document}
\liAufgabenTitel{Vorlesungsaufgaben}
\section{
\index{Kontextsensitive Sprache}
\footcite{theo:fs:3}}

% Info_2021-04-23-2021-04-23_09.35.52 2h25min

\section{Übung zu Turingmaschinen\footcite[Seite 22]{theo:fs:3}}

Gegeben ist eine Binärzahl auf dem Band einer Turingmaschine.

\begin{enumerate}
\item Definiere vollständig eine TM, die das Komplement der Binärzahl
(0110 -> 1001) berechnet. Die Überführungsfunktion kann als Tabelle oder
als Graph angegeben werden.

\begin{liAntwort}
\liFlaci{Ap9qtjgg7}
\end{liAntwort}

\item Erweitere deine Maschine aus Aufgabe a) so, dass der
Schreib-/Lesekopf auf dem ersten Zeichen der Eingabe terminiert.

\begin{liAntwort}
\liFlaci{A5o7tug5r}
\end{liAntwort}

\end{enumerate}

%-----------------------------------------------------------------------
%
%-----------------------------------------------------------------------

\section{Übung zu Turingmaschinenn\footcite[Seite 27]{theo:fs:3}}

% Info_2021-04-23-2021-04-23_13.17.40 3h10

\begin{enumerate}
\item Gib eine Turingmaschine an, die die Eingabe über dem Alphabet
$\Sigma = \{ a, b \}$ umkehrt.

Beispiele:

\begin{itemize}
\item abb -> bba
\item aaaaba -> abaaaa
\item aaa -> aaa
\end{itemize}

Tipp:

\begin{itemize}
\item Füge ein extra Zeichen ein, welches das Eingabewort von deinem
umgedrehten Wort trennt.

\item Das Ergebniswort muss nicht an derselben Stelle wie das
Eingabewort stehen.
\end{itemize}

\item Gib anschließend eine Konfigurationsfolge deiner TM für $ab$ an.

\end{enumerate}

\begin{liAntwort}
\liFlaci{Af75rdjbc}
\end{liAntwort}

\end{document}
