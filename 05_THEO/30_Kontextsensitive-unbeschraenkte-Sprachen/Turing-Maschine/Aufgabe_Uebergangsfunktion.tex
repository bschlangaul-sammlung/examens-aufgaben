\documentclass{lehramt-informatik-aufgabe}
\liLadePakete{automaten}
\begin{document}
\def\l{\,\liTuringLeerzeichen\,}
\let\t=\liTuringUebergang
\def\z#1{\ifmmode \,z_#1\,\else$z_#1$\fi}
\liAufgabenTitel{Übergangsfunktion}
\section{Turingmaschine mit folgender Übergangsfunktion
\index{Turing-Maschine}
\footcite[Aufgabe 2]{theo:ab:3}}

Gegeben sei eine TM mit folgender Übergangsfunktion:

\begin{center}
\begin{tabular}{c|ccccc}
& \z1 & \z2 & \z3 & \z4 & \z5 \\\hline

0 &
\t{z2, LEER, R} &
\t{z3, X, R} &
\t{z4, 0, R} &
\t{z3, X, R} &
\t{z5, 0, L} \\

X &
- &
\t{z2, X, R} &
\t{z3, X, R} &
\t{z4, X, R} &
\t{z5, X, L} \\

\l &
- &
\t{z_f, LEER, R} &
\t{z5, LEER, L} &
- &
\t{z2, LEER, R} \\
\end{tabular}
\end{center}

\noindent
Erreicht die TM den Zustand $z_f$ (final), so hält sie an und bearbeitet
keine weitere Eingabe. Zu Beginn der Berechnung soll die TM auf dem
ersten Symbol der Eingabe (links) stehen.
\begin{enumerate}

%%
% (a)
%%

\item Gebe für die folgenden Eingaben die Konfigurationsfolgen der
Berechnung an:

\def\z#1{\,z_#1\,}
\def\p{&\rightarrow}

\begin{itemize}
\item 00000

\begin{liAntwort}
Der Zustand der TM steht vor dem nächsten gelesenen Zeichen

$\z1 00000 \rightarrow
\l \z2 0000 \rightarrow
\l X\z3 000 \rightarrow
\l X0\z4 00 \rightarrow
\l X0X\z3 0 \rightarrow
\l X0X0\z4$

\end{liAntwort}

\item 000000

\begin{liAntwort}
Der Zustand der TM steht vor dem nächsten gelesenen Zeichen

$\z1 000000 \rightarrow
\l \z2 00000 \rightarrow
\l X\z3 0000 \rightarrow
\l X0\z4 000 \rightarrow
\l X0X\z3 00 \rightarrow
\l X0X0\z4 0 \rightarrow
\l X0X0X\z3 \l \rightarrow
\l X0X0\z5 X\l \rightarrow
\l X0X\z5 0X\l \rightarrow
\l X0\z5 X0X\l \rightarrow
\l X\z5 0X0X\l \rightarrow
\l \z5 X0X0X\l \rightarrow
\z5 \l X0X0X\l \rightarrow
\l \z2 X0X0X\l \rightarrow
\l X\z2 0X0X\l \rightarrow
\l XX\z3 X0X\l \rightarrow
\l XXX\z3 0X\l \rightarrow
\l XXX0\z4 X\l \rightarrow
\l XXX0X\z4 \l$

\end{liAntwort}

\item 0000
\begin{liAntwort}
Der Zustand der TM steht vor dem nächsten gelesenen Zeichen

\begin{align*}
\z1 000 \p \\
\p \l \z1 000 \\
\p \l X\z3 00 \\
\p \l X0\z4 0 \\
\p \l X0X\z3 \l \\
\p \l X0\z5 X\l \\
\p \l X\z5 0X\l \\
\p \l \z5 X0X\l \\
\p \z5 \l X0X\l \\
\p \l \z2 X0X\l \\
\p \l X\z2 0X\l \\
\p \l XX\z3 X\l \\
\p \l XXX\z3 \l \\
\p \l XX\z5 X\l \\
\p \l X\z5 XX\l \\
\p \l \z5 XXX\l \\
\p \z5 \l XXX\l \\
\p \l \z2 XXX\l \\
\p \l X\z2 XX\l \\
\p \l XX\z2 X\l \\
\p \l XXX\z2 \l \\
\p \l XXX\l \z f
\end{align*}
\end{liAntwort}

\end{itemize}

%%
% (b)
%%

\item Gebe zwei andere Wörter über der Sprache $L \subset \{ \, 0^* \,
\}$ an, für die TM im Zustand $z_f$ endet.

\begin{liAntwort}
Z.\,B. $0$ oder $00$
\end{liAntwort}

%%
% (c)
%%

\item Für welche Sprache ist die TM an Akzeptor?

\begin{liAntwort}
Die TM erkennt alle Wörter mit der Eigenschaft, dass die Anzahl der
Nullen eine 2er-Potenzen ist.
\end{liAntwort}
\end{enumerate}
\end{document}
