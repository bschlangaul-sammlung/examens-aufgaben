\documentclass{lehramt-informatik-aufgabe}
\liLadePakete{automaten}
\begin{document}
\let\t=\liTuringUebergang
\liAufgabenTitel{Übergangsfunktion}
\section{Turingmaschine mit folgender Übergangsfunktion
\index{Turing-Maschine}
\footcite[Aufgabe 2]{theo:ab:3}}

Gegeben sei eine TM mit folgender Übergangsfunktion:
0
X
LEER
z 1
\t{z2, LEER, R}
-
-
z 2
\t{z3, X, R}
\t{z2, X, R}
%\t{z\sb{\text{final}}, LEER, R}
z 3
\t{z4, 0, R}
\t{z3, X, R}
\t{z5, LEER, L}
z 4
\t{z3, X, R}
\t{z4, X, R}
-
z 5
\t{z5, 0, L}
\t{z5, X, L}
\t{z2, LEER, R}
Erreicht die TM den Zustand z f inal , so hält sie an und bearbeitet keine weitere Einga-
be. Zu Beginn der Berechnung soll die TM auf dem ersten Symbol der Eingabe (links)
stehen. Das Raute-Symbol steht für das leere Zeichen auf dem Band.
\begin{enumerate}
\item (a) Gebe für die folgenden Eingaben die Konfigurationsfolgen der Berechnung an:
• 00000
• 000000
• 0000
\item (b) Gebe zwei andere Wörter über der Sprache L ⊂ {0 ∗ } an, für die TM im Zustand
z f inal endet.
\item (c) Für welche Sprache ist die TM an Akzeptor?
\end{enumerate}
\end{document}
