\documentclass{lehramt-informatik-aufgabe}
\liLadePakete{formale-sprachen,automaten}
\begin{document}
\liAufgabenTitel{Binärzahl dekrementieren}
\section{Binärzahl dekrementieren
\index{Turing-Maschine}
\footcite[Aufgabe 3]{theo:ab:3}}

Sei \liAlphabet{0, 1} und $\Gamma = \{ 0, 1, \liTuringLeerzeichen \}$.
Konstruiere eine Turingmaschine $M$, die eine in Binärform gegebene,
natürliche Zahl $(\neq 0)$ um $1$ dekrementiert (und wieder in Binärform
ausgibt). Der Schreib-/Lesekopf steht zu Beginn der Berechnung auf dem
ersten Leerzeichen links von der Eingabe und soll auch am Ende wieder
dort stehen. Beachte, dass führende Nullen in der Eingabe/Ausgabe nicht
vorkommen dürfen.

\begin{liAntwort}
Die Maschine geht zunächst ans rechte Ende des Wortes, dann invertiert
sie alle 0 Bits, bis sie auf eine 1 trifft. Diese wird durch 0 ersetzt.
Damit ist der Dekrementierungsvorgang beendet. Nun sucht Sie das linke
Ende des Wortes und löscht eventuell entstandene führende Nullen. Trifft
Sie dabei auf das Leerzeichen, so war die Ausgabe die Zahl 0 und diese
wird wieder aufs Band geschrieben. Insgesamt ergibt sich M = ({q 0 , q 1
, q 2 , q 3 , q 4 , q 5 }, Σ, Γ, δ, q 0 , 2, {q 5 }) mit unten
angegebener Übergangsfunktion:

\begin{center}
\begin{tikzpicture}[li turingmaschine]
  \node[state,initial] (z0) at (2.86cm,-2.29cm) {$z_0$};
  \node[state] (z1) at (6.43cm,-2.29cm) {$z_1$};
  \node[state] (z2) at (9.57cm,-2.29cm) {$z_2$};
  \node[state] (z3) at (2.57cm,-4.86cm) {$z_3$};
  \node[state] (z4) at (6.14cm,-6.29cm) {$z_4$};
  \node[state,accepting] (z5) at (9.57cm,-6.29cm) {$z_5$};

  \liTuringKante[above]{z0}{z1}{
    1, 1, R;
    0, 0, R;
    LEER, LEER, R;
  }

  \liTuringKante[above]{z1}{z2}{
    LEER, LEER, L;
  }

  \liTuringKante[above,loop above]{z1}{z1}{
    1, 1, R;
    0, 0, R;
  }

  \liTuringKante[above,loop above]{z2}{z2}{
    0, 1, L;
  }

  \liTuringKante[above]{z2}{z3}{
    1, 0, L;
  }

  \liTuringKante[above,loop above]{z3}{z3}{
    1, 1, L;
    0, 0, R;
  }

  \liTuringKante[above]{z3}{z4}{
    LEER, LEER, R;
  }

  \liTuringKante[above,loop above]{z4}{z4}{
    0, LEER, R;
  }

  \liTuringKante[above]{z4}{z5}{
    1, 1, L;
    LEER, 0, R;
  }
\end{tikzpicture}
\end{center}
\liFlaci{Ahifz611c}
\end{liAntwort}

\end{document}
