\documentclass{lehramt-informatik-haupt}
\liLadePakete{formale-sprachen,automaten}

\begin{document}

%%%%%%%%%%%%%%%%%%%%%%%%%%%%%%%%%%%%%%%%%%%%%%%%%%%%%%%%%%%%%%%%%%%%%%%%
% Theorie-Teil
%%%%%%%%%%%%%%%%%%%%%%%%%%%%%%%%%%%%%%%%%%%%%%%%%%%%%%%%%%%%%%%%%%%%%%%%

\chapter{Linear beschränkte Turingmaschine}

\begin{liQuellen}
\item \cite[Seite 312-317]{hoffmann}
\item \cite{wiki:turingmaschine}
\end{liQuellen}

Eine deterministische Turingmaschine (TM, DTM) ist ein 7-Tupel
\liTuringMaschine{} ist gegeben durch:

\begin{description}
\item[$Z$]
Eine endliche Menge $Z$ von Zuständen

\item[$\Sigma$]
Ein Eingabealphabet $\Sigma$

\item[$\Gamma$]
Ein Bandalphabet $\Gamma$ mit $\Sigma \subseteq \Gamma$

\item[$\delta$]
Eine partielle Überführungsfunktion
δ : Z ⨯ Γ → Z ⨯ Γ ⨯ {L,R,N} (L = Links, R = Rechts, N = Neutral)

\item[$z_0$]
Einem Startzustand $z_0 \in Z$

\item[\liTuringLeerzeichen]
Ein Leerzeichen (für das leere Feld) □ ∈ Γ \ Σ

\item[$E$]
Eine endliche Menge E ⊆ Z von
akzeptierenden Zuständen
\end{description}

Die nichtdeterministischen linear beschränkten Turingmaschinen
(LBA = Linear Bounded Automaton) sind Automatenklassen für die
kontextsensitiven Grammatiken/Sprachen.

Ein LBA ist eine Turingmaschine, dessen Band linear beschränkt ist:

\begin{itemize}
\item[Definition 1]

Sie verlässt den Bereich des Bandes auf dem die Eingabe steht nicht.

\item[Definition 2]

Sie kann um einen konstanten Faktor $c$ größeres Band simulieren, sodass
die Turingmaschine höchstens $c \cdot n$ Felder benutzt, wobei $n$ die
Länge des Eingabewortes ist.
\end{itemize}

Eine offene Fragestellung ist, ob jede Sprache, die von einer NLBA
akzeptiert wird, auch durch eine DLBA akzeptiert wird. Sprich:
Akzeptieren DLBA und NLBA die gleiche Sprachklasse
\footcite[Seite 10]{theo:fs:3}

Eine TM terminiert oder hält für ein Wort $w$ genau dann, wenn für
eine gegebene Konfiguration keine Regel definiert ist.

Eine TM akzeptiert ein Wort $w$ genau dann, wenn sie in einem
Endzustand terminiert.

Eine TM mit $k$ Bändern und je einem Schreib-/Lesekopf ist
äquivalent zu einer Einbandturingmaschine. Dabei wird von allen $k$
Bändern zeitgleich je ein Symbol gelesen und jeder der $k$ Köpfe
kann sich unabhängig von den anderen bewegen.
\footcite[Seite 26]{theo:fs:3}

\literatur

\end{document}
