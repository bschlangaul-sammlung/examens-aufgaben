\documentclass{lehramt-informatik-aufgabe}
\liLadePakete{syntax}
\begin{document}
\liAufgabenTitel{Fibonacci-Zahlen}
\section{Fibonacci-Zahlen}

Fibonacci-Zahlen (Check-Up)
Die Fibonacci-Folge ist die unendliche Folge natürlicher Zahlen, die mit zweimal der
Zahl 1 beginnt. Im Anschluss ergibt jeweils die Summe zweier aufeinanderfolgender
Zahlen die unmittelbar danach folgende Zahl:
f ib n = f ib n−1 + f ib n−2
Dabei bezeichnet n die n-te Zahl dieser Reihe. Die darin enthaltenen Zahlen heißen
Fibonacci-Zahlen. Benannt ist die Folge nach Leonardo Fibonacci, der damit im Jahr
1202 das Wachstum einer Kaninchenpopulation beschrieb. Die Folge war aber schon
in der Antike sowohl den Griechen als auch den Indern bekannt. Gleichmaßen lassen
sich Quadratgrößen damit beschreiben:

Die Lösung der Berechnung soll zum Schluss in R5 liegen.
\end{document}
