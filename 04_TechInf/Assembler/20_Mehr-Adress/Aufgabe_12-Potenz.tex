\documentclass{lehramt-informatik-aufgabe}
\liLadePakete{syntax,mathe}
\begin{document}
\liAufgabenTitel{Potenzberechnung}

\section{Potenzberechnung}

Erstelle ein rekursives Assemblerprogramm, das seine beiden Parameter
über zwei Variablen $a$ und $n$ aus dem Speicher übernimmt und den Wert
$\text{power}(a, n)$ berechnet. Das Ergebnis soll in $R0$ liegen. Dabei
soll die Rekursion gelten:

\begin{displaymath}
\text{power}(a, n) = a \cdot \text{power}(a, n − 1)
\end{displaymath}

\noindent
Die Lösung der Berechnung soll zum Schluss in $R5$ liegen.

\begin{liAntwort}
\liAssemblerDatei{Aufgabe_12-Potenz.mi}
\end{liAntwort}
\end{document}
