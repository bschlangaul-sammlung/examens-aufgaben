\documentclass{lehramt-informatik-aufgabe}
\liLadePakete{syntax}
\begin{document}
\liAufgabenTitel{Euklidscher Algorithmus}
\section{Euklidscher Algorithmus}

Nach Euklid lässt sich der größte gemeinsamer Teiler zweier Zahlen a und b bestim-
men mit:
Wenn CD aber AB nicht misst, und man nimmt bei AB, CD abwechselnd immer das
”
kleinere vom größeren weg, dann muss (schließlich) eine Zahl übrig bleiben, die die
vorangehende misst.“
Erstelle ein Assemblerprogramm, das seine beiden Parameter über zwei Variablen a
und b aus dem Speicher übernimmt und den ggt(a, b) berechnet. Das Ergebnis soll in
R0 liegen.
\end{document}
