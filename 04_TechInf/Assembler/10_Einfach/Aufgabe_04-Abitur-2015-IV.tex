\documentclass{lehramt-informatik-aufgabe}
\liLadePakete{syntax,spalten,struktogramm}
\begin{document}
\liAufgabenTitel{Papyrus Rhind}

%%
%
%%

\section{Abitur 2015 IV}

Auf dem ägyptischen \emph{Papyrus Rhind}, der etwa auf das Jahr 1550 v.
Chr. datiert wird, ist eine Möglichkeit zur Multiplikation zweier
natürlicher Zahlen $z_1$ und $z_2$ beschrieben. Als Struktogramm
lässt sich dieser Algorithmus folgendermaßen darstellen:

\begin{center}
\begin{struktogramm}(90,50)
\assign{lies $z_1$}
\assign{lies $z_2$}
\assign{$erg = 0$}

\while{wiederhole solange $z_1 > 0$}
\ifthenelse{4}{4}
{$z_1$ gerade}{wahr}{falsch}
\change
\assign{$erg = erg + z_2$}
\ifend
\assign{$z_1 = z_1 / 2$ (ganzzahlige Division durch 2)}
\assign{$z_2 = z_2 \cdot 2$ }
\whileend
\end{struktogramm}
\end{center}

\noindent
Das berechnete Produkt steht nach Abarbeitung des Algorithmus in der
Variablen $erg$.

\begin{enumerate}

%%
% a)
%%

\item Berechnen Sie mithilfe der beschriebenen ägyptischen
Multiplikation schrittweise das Produkt aus $z_1 = 13$ und $z_2 = 5$

\begin{liAntwort}
\begin{tabular}{lll}
$z_1$ & $z_2$ & erg \\
13 & 5  & 0   \\
   &    & 5   \\
6  & 10 &     \\
3  & 20 &     \\
   &    & 25  \\
1  & 40 &     \\
   &    & 65  \\
0  & 80 &
\end{tabular}

% \begin{tabular}{lll}
% $z_1$ & $z_2$ & erg \\
% 13 & 5  & 0   \\
% 13 & 5  & 5   \\
% 6  & 5  & 5   \\
% 6  & 10 & 5   \\
% 3  & 10 & 5   \\
% 3  & 20 & 5   \\
% 3  & 20 & 25  \\
% 1  & 20 & 25  \\
% 1  & 40 & 25  \\
% 1  & 40 & 65  \\
% 0  & 40 & 65  \\
% 0  & 80 & 65
% \end{tabular}
\end{liAntwort}

%%
% b)
%%

\item Nennen Sie die wesentliche Idee des Speichermodells eines
Rechners, der nach dem von-Neumann-Prinzip aufgebaut ist. Geben Sie
einen Vor- und einen Nachteil dieses Speichermodells an.

\begin{liAntwort}
Wesentliche Idee des Speichermodells eines Rechners, der nach dem
von-Neumann-Prinzip gebaut ist: Programme und Daten sind im selben
Speicher, wobei der Hauptspeicher aus Zellen gleicher Größe besteht.

Vorteil:

Streng sequentieller Ablauf von Befehlen ist ein Vorteil, weil zu jedem
Zeitpunkt klar ist, welcher Schritt durchgeführt wird.

Nachteil:

Der von-Neumann-Flaschenhals, weil alle Daten über denselben Bus
weitergeleitet werden müssen und der Ablauf deshalb eine gewisse Zeit
benötigt.
\end{liAntwort}

%%
% c)
%%

\item Bestätigen Sie anhand zweier Beispiele, dass mithilfe des
folgenden Programmausschnitts entschieden werden kann, ob
Speicherzelle 101 eine gerade oder ungerade Zahl enthält.

\begin{tabular}{lll}
         & Beispiel für gerade    Zahl & Beispiel für ungerade    Zahl \\
LOAD 101 & 6=110                       & 5=101                         \\
SHRI 1   & 3=011                       & 2=010                         \\
SHLI 1   & 6=110                       & 4=100                         \\
SUB 101  & 0                           & not zero    für ungerade
\end{tabular}

\begin{tabular}{lll}
         & Beispiel für gerade    Zahl & Beispiel für ungerade    Zahl \\
LOAD 101 & 14=1110                     & 17=10001                      \\
SHRI 1   & 7=0111                      & 8=01000                       \\
SHLI 1   & 14=1110                     & 16=10000                      \\
SUB 101  & 0                           & not zero    für ungerade
\end{tabular}

\begin{tabular}{lll}
         & Beispiel für gerade    Zahl & Beispiel für ungerade    Zahl \\
LOAD 101 & 44=101100                   & 25=11001                      \\
SHRI 1   & 22=010110                   & 12=01100                      \\
SHLI 1   & 44=101100                   & 24=11000                      \\
SUB 101  & 0                           & not zero    für ungerade
\end{tabular}

\item Schreiben Sie ein Programm für die angegebene Registermaschine,
das den Algorithmus des \emph{Papyrus Rhind} umsetzt. Gehen Sie dabei
davon aus, dass die beiden positiven ganzzahligen Faktoren $z_1$ und
$z_2$ bereits in den Speicherzellen $101$ und $102$ stehen und dass alle
weiteren nicht vom Programm belegten Speicherzellen mit dem Wert $0$
vorbelegt sind.

\liAssemblerDatei{Aufgabe_04-Abitur-2015-IV.mia}

\liJavaDatei[firstline=3]{aufgaben/tech_info/assembler/AegyptischeMultiplikation}

\end{enumerate}

\end{document}
