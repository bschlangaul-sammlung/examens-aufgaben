\documentclass{lehramt-informatik-aufgabe}
\liLadePakete{syntax,spalten}
\begin{document}
\liAufgabenTitel{Minimaschine}

%%
%
%%

\section{Abitur 2013 III (Aufgabe 2, Check-Up)}

%%
%
%%

2a)

\begin{tabular}{llll}
              &                                              & \multicolumn{2}{l}{Speicherzellen}                                                                  \\
Befehl        & Akkumulator                                  & 101                                               & 102                                             \\
              &                                              & 5                                                 & 18                                              \\
LOAD 102      & 18                                           & 5                                                 & 18                                              \\
DIV 101       & 3                                            & 5                                                 & 18                                              \\
MULT 101      & 15                                           & 5                                                 & 18                                              \\
SUB 102       & -3                                           & 5                                                 & 18                                              \\
JMPZ acht     & \multicolumn{3}{l}{Befehl wird ignoriert, da 			das Ergebnis der letzten Operation nicht 0 war. (Bedeutet: 			zero-flag ist nicht gesetzt worden)} \\
LOADI 0       & 0                                            & 5                                                 & 18                                              \\
JMP neun      & 0                                            & 5                                                 & 18                                              \\
acht: LOADI 1 &                                              &                                                   &                                                 \\
neun: END     &                                              &                                                   &
\end{tabular}

Die Werte in den Speicherzellen haben sich nie geändert, weil Zwischenergebnisse mit der Hilfe von STORE nicht gespeichert worden sind.

%%
%
%%

2b) Siehe Extradatei.

Abitur 2015 IV

%%
%
%%

a) Ägyptische Multiplikation:

\begin{tabular}{lll}
z1 & z2 & erg \\
13 & 5  & 0   \\
   &    & 5   \\
6  & 10 &     \\
3  & 20 &     \\
   &    & 25  \\
1  & 40 &     \\
   &    & 65  \\
0  & 80 &
\end{tabular}

\begin{tabular}{lll}
z1 & z2 & erg \\
13 & 5  & 0   \\
13 & 5  & 5   \\
6  & 5  & 5   \\
6  & 10 & 5   \\
3  & 10 & 5   \\
3  & 20 & 5   \\
3  & 20 & 25  \\
1  & 20 & 25  \\
1  & 40 & 25  \\
1  & 40 & 65  \\
0  & 40 & 65  \\
0  & 80 & 65
\end{tabular}

%%
%
%%

b)
Wesentliche Idee des Speichermodells eines Rechners, der nach dem von-Neumann-Prinzip gebaut ist:
Programme und Daten sind im selben Speicher, wobei der Hauptspeicher aus Zellen gleicher Größe besteht.

Vorteil:
Streng sequentieller Ablauf von Befehlen ist ein Vorteil, weil zu jedem Zeitpunkt klar ist, welcher Schritt durchgeführt wird.
Nachteil: Der von-Neumann-Flaschenhals, weil alle Daten über denselben Bus weitergeleitet werden müssen und der Ablauf deshalb eine gewisse Zeit benötigt.

%%
%
%%

c) Durchführung zu Übungszwecken an mehreren Beispielen:

\begin{tabular}{lll}
         & Beispiel für gerade 			Zahl & Beispiel für ungerade 			Zahl \\
LOAD 101 & 6=110                       & 5=101                         \\
SHRI 1   & 3=011                       & 2=010                         \\
SHLI 1   & 6=110                       & 4=100                         \\
SUB 101  & 0                           & not zero 			für ungerade
\end{tabular}

\begin{tabular}{lll}
         & Beispiel für gerade 			Zahl & Beispiel für ungerade 			Zahl \\
LOAD 101 & 14=1110                     & 17=10001                      \\
SHRI 1   & 7=0111                      & 8=01000                       \\
SHLI 1   & 14=1110                     & 16=10000                      \\
SUB 101  & 0                           & not zero 			für ungerade
\end{tabular}

\begin{tabular}{lll}
         & Beispiel für gerade 			Zahl & Beispiel für ungerade 			Zahl \\
LOAD 101 & 44=101100                   & 25=11001                      \\
SHRI 1   & 22=010110                   & 12=01100                      \\
SHLI 1   & 44=101100                   & 24=11000                      \\
SUB 101  & 0                           & not zero 			für ungerade
\end{tabular}

Anmerkungen für mich
Bedeutung der Abkürzungen:
SHRI bedeutet „Shift nach rechts“; SHLI bedeutet „Shift nach links“;
Vorgehen:
1. Gerade und ungerade Beispielzahl überlegen (> LOAD 101) und von 1 beginnend solange das letzte Ergebnis *2 nehmen bis es noch in die Beispielzahl passt (im weiteren Verlauf der Erklärung x genannt); dann „=“ und von links nach rechts ausgehend hinschreiben, wie oft man das x braucht, dann x/2, etc. (nur 1er und 0er zulässig!)
2. Zahlen hinter dem „=“ nach rechts verschieben. Dabei fällt die letzte Zahl weg und links wird die 0 ergänzt. Vor dem „=“ die Zahl mit Hilfe der Zahlenfolge bestimmen. (> SHRI 1)
3. Zahlen hinter dem „=“ nach links verschieben. Dabei fällt die erste Zahl weg und rechts wird die 0 ergänzt. Vor dem „=“ die Zahl mit Hilfe der Zahlenfolge bestimmen. (> SHLI 1)
4. Vom bisherigen Ergebnis die Beispielzahl abziehen. (> SUB 101)

\end{document}
