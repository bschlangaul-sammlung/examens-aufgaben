\documentclass{lehramt-informatik-aufgabe}
\liLadePakete{syntax,spalten}
\usepackage{struktex}
\begin{document}
\liAufgabenTitel{Abitur 2013 IV}

\section{Abitur 2013 IV}

In einer Apotheke werden Aminosäureprodukte in Pulverform verkauft, die
in vollständig gefüllten zylinderförmigen Dosen abgepackt sind. Aufgrund
der Regalhöhe haben alle ausgestellten Dosen eine Höhe von 12cm. Der
Radius der Dosengrundfläche richtet sich nach der jeweiligen
Verkaufsmenge des Pulvers und wird druch folgenden Algorithmus
näherungsweise berechnet:

\begin{struktogramm}(50,40)
\assign{$y = V$}
\assign{$z = 1$}
 \while{wiederholge solange $y > z$}
    \assign{$y = (y + z) / 2$}
    \assign{$z = V / y$}
 \whileend
\assign{Rückgabe $y / 19$}
\end{struktogramm}

Der dabei verwendete Wert von $19$ für den Divisor ergibt sich aus der
vorgegebenen Dosenhöhe in mm und der Kreiszahl $\pi$.

Schreiben Sie ein Assemblerprogramm zur Berechnung des Dosenradius (in
mm) gemäß dem angegebenen Algorithmus, wobei das Volumen V in $mm^3$
eingegeben wird.

Ergänzen Sie dabei die begonnen Implementierung. Das Ergebnis soll am
Ende in Zelle 106 stehen.

\begin{minted}{asm}
Start:
      LOADI 400000 # Beispielwert für V
      STORE 101 # V in Zelle 101
      LOADI 2
      STORE 104 #Konstante 2 in Zelle 104
      LOADI 19
      STORE 105 # Konstante 19 in Zelle 105
      LOADI 1
      STORE 102 # z in Zelle 102
      LOAD 101
      STORE 103 # y in Zelle 103

wdh:  LOAD 103  # y wird geladen
      CMP 102  # y wird mit z verglichen
      JMPZ ENDE # Wenn y>z false, dann Sprung zum Ende.
      LOAD 103 # y wird geladen
      ADD 102  # z wird addiert
      DIV 104  # Division durch Konstante 2
      STORE 103 # Speichern in Zelle 103 (= neues y)
      LOAD 101 # V wird geladen
      DIV 103  # V wird durch y geteilt
      STORE 102 # Ergebnis wird in z gespeichert
      JMP wdh  # Ende und Sprung zum Beginn

ENDE:  LOAD 103  # Laden des Wertes in Zelle 103
  DIV 105  # Teilen durch Konstante 19
  STORE 106 # Rueckgabe in 106
  HOLD
\end{minted}

\end{document}
