\documentclass{lehramt-informatik-aufgabe}
\liLadePakete{syntax,spalten}

\def\liAssemblerCode#1{\mintinline{asm}|#1|}

\begin{document}
\liAufgabenTitel{Abitur 2013 III}

\section{Abitur 2013 III (Aufgabe 2, Check-Up)
\index{Ein-Adress-Befehl-Assembler}}

\begin{enumerate}

%%
% 2a)
%%

\item Vollziehen Sie das nachfolgende Assembler-Programm schrittweise
nach, indem Sie angeben, welche Werte nach jedem Befehl in den
Speicherzellen 101, 102 und im Akkumulator stehen, wenn zu Beginn 101
mit 5 und 102 mit 18 vorbelegt ist.

\begin{minted}{asm}
      LOAD 102
      DIV 101
      MULT 101
      SUB 102
      JMPZ acht
      LOADI 0
      JUMP neun
acht: LOADI 1
neun: END
\end{minted}

\begin{tabular}{|l|l|l|l|}
              &                                              & \multicolumn{2}{l}{Speicherzellen}                                                                  \\
Befehl        & Akkumulator                                  & 101                                               & 102                                             \\\hline
              &                                              & 5                                                 & 18                                              \\
\liAssemblerCode{LOAD 102}      & 18 & 5 & 18 \\
\liAssemblerCode{DIV 101}       & 3  & 5 & 18 \\
\liAssemblerCode{MULT 101 }     & 15 & 5 & 18 \\
\liAssemblerCode{SUB 102}       & -3 & 5 & 18 \\
\liAssemblerCode{JMPZ acht}     & -3 & 5 & 18 \\
\liAssemblerCode{LOADI 0}       & 0  & 5 & 18 \\
\liAssemblerCode{JMP neun}      & 0  & 5 & 18 \\
\liAssemblerCode{acht: LOADI 1} & 0  & 5 & 18 \\
\liAssemblerCode{neun: END}     & 0  & 5 & 18 \\
\end{tabular}

% \multicolumn{3}{l}{Befehl wird ignoriert, da  das Ergebnis der letzten Operation nicht 0 war. (Bedeutet: zero-flag ist nicht gesetzt worden)}

Die Werte in den Speicherzellen haben sich nie geändert, weil
Zwischenergebnisse mit der Hilfe von STORE nicht gespeichert worden
sind.

%%
% 2b)
%%

\item Übersetzen Sie das nachfolgende Struktogramm zur Berechnung der
Fakultät von n in ein Assemblerprogramm. Verwenden Sie die Variable erg
die Speicherzelle 201 und für die Variable n die Speicherzelle 202.

\end{enumerate}

\end{document}
