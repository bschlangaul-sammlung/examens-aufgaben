\documentclass{lehramt-informatik-aufgabe}
\liLadePakete{syntax,spalten}
\begin{document}
\liAufgabenTitel{Primzahl}

\section{Abitur 2017 IV (Check-Up)}

In dem folgenden Struktogramm wird ein Algorithmus dargestellt, der
erkennt, ob eine natürliche Zahl $k$ eine Primzahl ist. In diesem Fall
wir in die Speicherzelle $erg$ die Zahl $1$ abgelegt, sonst $0$.

\begin{enumerate}

%%
% (a)
%%

\item Stellen Sie die Veränderung der Variablenwerte bei Ablauf dieses
Algorithmus jeweils für die Startwerte $k = 5$ und $k = 15$ durch zwei
Speicherbelegungstabellen wie nachfolgend gezeigt dar.

Im Folgenden soll ein Programm für diese Maschine erstellt werden, das
den dargestellten Algorithmus umsetzt. Der Wert von $k$ soll in
Speicherzelle $101$, der von $a$ in $102$, der von $t$ in $103$ und der
von $erg$ in $104$ gespeichert werden.

%%
% (b)
%%

\item Betrachten Sie die folgende kurze Sequenz; xx steht dabei für ein
geeignetes Sprungziel.

\begin{minted}{asm}
LOAD 101
MOD 103
JMPP xx
LOADI 1
STORE 102
\end{minted}

Geben Sie an, welcher Teil des Algorithmus damit umgesetzt wird.

%%
% (c)
%%

\item Setzen Sie unter Verwendung der Sequenz aus Teilaufgabe 2b den
gesamten Algorithmus in eine Programm für die gegebene Registermaschine
um.
\end{enumerate}

\end{document}
