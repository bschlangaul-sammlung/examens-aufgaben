\documentclass{lehramt-informatik-aufgabe}
\liLadePakete{syntax,spalten,struktogramm}
\begin{document}
\liAufgabenTitel{Collatz}

\section{Abitur 2019 IV}

Das Collatz-Problem ist ein immer noch ungelöstes Problem der
Mathematik. Dabei geht es um Zahlenfolgen, die nach folgendem
Algorithmus gebildet werden, wobei der Eingabewert $n$ eine natürliche
Zahl größer $0$ ist:

\begin{center}
\begin{struktogramm}(80,25)[collatzfolge(n)]
\while{wiederhole solange $n$ ungleich $1$ ist.}
\ifthenelse{4}{4}
{$n$ ist gerade}{wahr}{falsch}
\assign[7]{$n = \frac{n}{2}$}
\change
\assign[7]{$n = 3 \cdot n + 1$}
\ifend
\whileend
\end{struktogramm}
\end{center}

\noindent
Obwohl der Algorithmus sehr einfach ist, ist bis heute ungeklärt, ob er
tatsächlich bei jedem beliebigen Startwert von $n$ nach endlich vielen
Durchläufen der Wiederholung terminiert.

\begin{enumerate}

%%
% (a)
%%

\item Geben Sie die Zahlenfolge an, die man mit dem Startwert $7$
erhält, wenn $n$ nach jedem Durchlauf der Wiederholung ausgegeben wird.

\begin{liAntwort}
22
11
34
17
52
26
13
40
20
10
5
16
8
4
2
1
\end{liAntwort}

%%
% (b)
%%

\item Beschreiben Sie, wie man mithilfe der ganzzahligen Division ohne
Rest prüfen kann, ob eine Zahl $a$ durch eine andere Zahl $b$ teilbar
ist.

%%
% (c)
%%

\item Gegeben ist eine Registermaschine mit folgendem Befehlssatz:

Geben Sie ein Programm für die Registermaschine an, das den gegebenen
Algorithmus collatzfolge(n) umsetzt, wobei zusätzlich die Anzahl der
Durchläufe der Wiederholung bestimmt werden soll. Der Startwert für n
steht am Anfang bereits in Speicherzelle 100.

\end{enumerate}

\end{document}
