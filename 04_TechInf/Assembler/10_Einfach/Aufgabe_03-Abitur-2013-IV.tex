\documentclass{lehramt-informatik-aufgabe}
\liLadePakete{syntax,spalten,struktogramm}

\begin{document}
\liAufgabenTitel{Pulverdosen}

\section{Abitur 2013 IV}

In einer Apotheke werden Aminosäureprodukte in Pulverform verkauft, die
in vollständig gefüllten zylinderförmigen Dosen abgepackt sind. Aufgrund
der Regalhöhe haben alle ausgestellten Dosen eine Höhe von 12cm. Der
Radius der Dosengrundfläche richtet sich nach der jeweiligen
Verkaufsmenge des Pulvers und wird druch folgenden Algorithmus
näherungsweise berechnet:

\begin{struktogramm}(50,40)
\assign{$y = V$}
\assign{$z = 1$}
  \while{wiederholge solange $y > z$}
    \assign{$y = (y + z) / 2$}
    \assign{$z = V / y$}
  \whileend
\assign{Rückgabe $y / 19$}
\end{struktogramm}

Der dabei verwendete Wert von $19$ für den Divisor ergibt sich aus der
vorgegebenen Dosenhöhe in mm und der Kreiszahl $\pi$.

Schreiben Sie ein Assemblerprogramm zur Berechnung des Dosenradius (in
mm) gemäß dem angegebenen Algorithmus, wobei das Volumen V in $mm^3$
eingegeben wird.

Ergänzen Sie dabei die begonnen Implementierung. Das Ergebnis soll am
Ende in Zelle 106 stehen.

\liAssemblerDatei{Aufgabe_03-Abitur-2013-IV.mia}

\end{document}
