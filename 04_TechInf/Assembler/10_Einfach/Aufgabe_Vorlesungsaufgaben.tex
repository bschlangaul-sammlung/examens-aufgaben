\documentclass{lehramt-informatik-aufgabe}
\liLadePakete{syntax,spalten}
\begin{document}
\liAufgabenTitel{Vorlesungsaufgaben}
\section{Minimaschine\index{Ein-Adress-Befehl-Assembler}}

\section{Aufgabe 1: Vorlesungsaufgaben (Check-Up)}

Geben Sie die Lösungen zu den Aufgaben aus der Assembler-Vorlesung
ab. Bearbeiten Sie erst danach die folgenden Aufgaben auf diesem
Übungsblatt.

\begin{enumerate}

%%
% (a)
%%

\item Folie 28/2: Berechnung der Potenz $a^n$.

\begin{multicols}{2}
\liPseudoUeberschrift{Assembler}

\begin{minted}{asm}
          LOAD n
          CMPI 0
          JMPP anfang
          JMP ende

anfang:   LOAD n
          JMPP rechnung
          JMP ende

rechnung: LOAD a
          MUL ergebnis
          STORE ergebnis
          LOAD n
          SUBI 1
          STORE n
          JMP anfang

ende:     HOLD
a:        WORD 2
n:        WORD 3
ergebnis: WORD 0
\end{minted}

\liSpaltenUmbruch
\liPseudoUeberschrift{Minisprache}

\begin{minted}{componentpascal}
PROGRAM potenz;
VAR a, n, ergebnis;
BEGIN
  a := 2;
  n := 3;
  ergebnis := 1;
  WHILE n <> 0 DO
    ergebnis := ergebnis * a;
    n := n - 1;
  END
END potenz.
\end{minted}
\end{multicols}

%%
% (b)
%%

\newpage

\item Folie 37/3: Größten gemeinsamen Teiler zweier Zahlen

\begin{multicols}{2}
\liPseudoUeberschrift{Assembler}

\begin{minted}{asm}
anfang: LOAD b
        JMPP innen
        JMP ende

innen:  LOAD a
        SUB b
        JMPN tausch
        STORE a
        JMP innen

tausch: LOAD b
        STORE ggt
        LOAD a
        STORE b
        LOAD ggt
        STORE a
        JMP anfang

ende:   HOLD

a:      WORD 30
b:      WORD 10
ggt:    WORD 0
\end{minted}

\liSpaltenUmbruch
\liPseudoUeberschrift{Minisprache}

\begin{minted}{componentpascal}
PROGRAM ggt;
VAR a, b, ggt;

BEGIN
  a := 30;
  b := 10;
  WHILE b > 0 DO
    IF a - b < 0 THEN
      ggt := b;
      b := a;
      a := ggt;
    ELSE
      a := a - b;
    END
  END
END ggt.
\end{minted}
\end{multicols}
\end{enumerate}

\end{document}
