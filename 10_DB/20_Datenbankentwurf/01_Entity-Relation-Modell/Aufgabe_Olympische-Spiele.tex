\documentclass{lehramt-informatik-aufgabe}
\liLadePakete{}
\begin{document}
\liAufgabenTitel{Olympische Spiele}

\section{Aufgabe 3: Olympische Spiele
\index{Entity-Relation-Modell}
\footcite{db:ab:4}}

Geben Sie ein Entity-Relationship-Diagramm (mit Schlüssel und
Funktionalität) für folgendes Problem an:

Eine europäische Fachzeitschrift für Leichtathletik möchte in einer
relationalen Datenbank die folgenden Informationen zu den Olympischen
Sommerspielen speichern: Für jeden Sportler, der bei den Olympischen
Spielen teilgenommen hat, sollen der Name, die Nationalität, die Größe
und das Gewicht bekannt sein. Zusätzlich soll abgefragt werden können,
ob der betreffenden Sportler aus Europa stammt. Zu den Olympischen
Spielen sollen das Jahr der Austragungsort und die Teilnehmerzahl
gespeichert werden. Außerdem soll bekannt sein, welcher Sportler bei
welchen Spielen welchen Rekord erzielt hat. Zu jedem Rekord sollen die
Disziplin und die Art des Rekords (Weltrekord, Europarekord,...)
gespeichert werden. Darüber soll ohne Umweg abrufbar sein, ob es sich um
einen Rekord in der Leichtathletik handelt. Es sei vorausgesetzt, dass
jeder Sportler eindeutig durch seinen Namen identifizierbar ist und dass
Olympische Spiele in verschiedenen Jahren in derselben Stadt stattfinden
können. Ein Rekord soll eindeutig durch die Disziplin und die Art des
Rekords gekennzeichnet sein.

\end{document}
