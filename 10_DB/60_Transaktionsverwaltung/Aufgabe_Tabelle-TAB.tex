\documentclass{lehramt-informatik-aufgabe}
\liLadePakete{spalten}
\begin{document}
\liAufgabenTitel{Tabelle TAB}

\section{Aufgabe „Tabelle TAB“
\index{Transaktionsverwaltung}
\footcite{db:ab:6}}

Die Transaktionen eines Transaktionsprogramms besteht aus SQL-Befehlen.
Die Transaktionen T1 und T2 arbeiten auf der Tabelle TAB.

\begin{multicols}{2}
\liPseudoUeberschrift{Transaktion T1}

\begin{tabular}{|l|}
\hline
BOT                   \\\hline
SELECT FROM TAB       \\\hline
NEUF:-F+5             \\\hline
UPDATE TAB SET F=NEUF \\\hline
COMMIT WORK           \\\hline
\end{tabular}

\liPseudoUeberschrift{Transaktion T2}

\begin{tabular}{|l|}
\hline
BOT                   \\\hline
SELECT FROM TAB       \\\hline
NEUF:-F*2             \\\hline
UPDATE TAB SET F=NEUF \\\hline
COMMIT WORK           \\\hline
\end{tabular}
\end{multicols}

\noindent
Die quasiparallele Abarbeitung erfolgt in folgenden Schritten:

\begin{center}
\begin{tabular}{|l|l|l|}
\hline
Schritt & T1                    & T2                    \\\hline
1       &                       & BOT                   \\\hline
2       & BOT                   &                       \\\hline
3       & SELCT F FROM TAB      &                       \\\hline
4       &                       & SELECT F FROM TAB     \\\hline
5       &                       & NEUF := F*2           \\\hline
6       & NEUF := F+5           &                       \\\hline
7       & UPDATE TAB SET F=NEUF &                       \\\hline
8       & COMMIT WORK           &                       \\\hline
9       &                       & UPDATE TAB SET F=NEUF \\\hline
10      &                       & COMMIT WORK           \\\hline
\end{tabular}
\end{center}

\begin{enumerate}

%%
%
%%

\item Ist die (quasiparallele) Bearbeitung der Transaktionen korrekt?
Begründung!\index{Lost-Update}

\begin{liAntwort}
Nein, es liegt ein Lost-Update-Fehlerfall vor. In Schritt 3 bzw. 4
lesen T1 bzw. T2 denselben Wert aus der Tabelle TAB. Der von T1 in
Schritt 7 in die Tabelle zurückgeschriebene Wert wird in Schritt 9 von
T2 überschrieben.
\end{liAntwort}

%%
%
%%

\item Konstruieren Sie unter Verwendung von T1 und T2 einen
Dirty-Read-Fehlerfall.
\index{Dirty-Read}
\end{enumerate}
\end{document}
