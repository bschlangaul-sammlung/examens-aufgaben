\documentclass{lehramt-informatik-haupt}

\begin{document}

\chapter{Physische Datenorganisation}

\section{Definition „Physische Datenunabhängigkeit“:}

Änderungen an der \memph{physischen} Speicher- oder der Zugriffsstruktur
(beispielsweise durch das Anlegen einer Indexstruktur) haben keine
Auswirkungen auf die \memph{logische} Struktur der Datenbasis, also auf
das Datenbankschema.
\footcite[Seite 20]{db:fs:3}
\footcite[Kapitel 1.3 Datenunabhängigkeit Seite 24]{kemper}

%-----------------------------------------------------------------------
%
%-----------------------------------------------------------------------

\section{ANSI-SPARC-Architektur (auch Drei-Schema-Architektur)
\footcite[Seite 443, 13.1.3 Architektur eines Datenbanksystems]{schneider}
\footcite[Seite 23, 1.2 Datenabstraktion]{kemper}
\footcite[Seite 20]{db:fs:3}
}

Trennung des Datenbankschemas in 3 verschiedene Beschreibungsebenen:

\begin{description}
\item[Externe Ebene / (Benutzer)sichten (Views)]

Die externe Ebene stellt den Benutzern und Anwendungen individuelle
Benutzersichten bereit, wie beispielsweise \memph{Formulare}, \memph{Masken-Layouts},
\memph{Schnittstellen}.

\item[Konzeptionelle / logische Ebene]

Die konzeptionelle Ebene beschriebt, \memph{welche Daten} in der
Datenbank \memph{gespeichert} sind, sowie \memph{deren Beziehungen.}
Ziel des Datenbankdesigns ist eine vollständige und redundanzfreie
\memph{Darstellung aller zu speichernden Informationen}. Hier findet die
Normalisierung des relationalen Datenbankschemas statt.

\item[Interne / physische Ebene]

Die interne Ebene stellt die physische Sicht der Datenbank im Computer
dar. Sie beschreibt, \memph{wie und wo die Daten in der Datenbank
gespeichert werden}. Oberstes Designziel ist ein effizienter Zugriff auf
die gespeicherten Informationen, der meist durch bewusst in Kauf
genommene Redundanz (\zB Index) erreicht wird.
\end{description}

\subsection{Vorteile des Modells:}

\begin{description}
\item[physischen Datenunabhängigkeit]

Da die interne von der konzeptionellen und externen Ebene getrennt ist,
wirken sich physische Änderungen, \zB des Speichermediums oder des
Datenbankprodukts, nicht auf die konzeptionelle oder externe Ebene aus.

\item[logischen Datenunabhängigkeit]

Da die konzeptionelle und die externe Ebene getrennt sind, haben
Änderungen an der Datenbankstruktur (konzeptionelle Ebene) keine
Auswirkungen auf die externe Ebene, also die Masken-Layouts, Listen und
Schnittstellen.
\end{description}

Allgemein: höhere Robustheit gegenüber Änderungen

%-----------------------------------------------------------------------
%
%-----------------------------------------------------------------------

\section{Physische Datenorganisation
\footcite[Seite 21]{db:fs:3}}

\begin{itemize}
\item Daten werden in Form von \emph{Sätzen} auf der Festplatte
abgelegt, um auf Sätze zugreifen zu können, verfügt jeder Satz über eine
\emph{eindeutige, unveränderliche Satzadresse}

\item TID = Tupel Identifier: dient zur Adressierung von Sätzen in einem
Segment und besteht aus zwei Komponenten

\begin{itemize}
\item Seitennummer (Seiten bzw. Blöcke sind größere Speichereinheiten
auf der Platte)

\item Relative Indexposition innerhalb der Seite
\end{itemize}

\item Satzverschiebung innerhalb einer Seite bleibt ohne Auswirkungen
auf TID, wird ein Satz auf eine andere Seite migriert, wird eine
„Stellvertreter-TID“ zum Verweis auf den neuen Speicherort verwendet.
Die eigentliche TID-Adresse bleibt stabil
\footcite[Seite 219]{kemper}
\end{itemize}

\literatur
\end{document}
