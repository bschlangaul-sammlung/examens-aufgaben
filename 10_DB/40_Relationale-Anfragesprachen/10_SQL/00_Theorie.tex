\documentclass{lehramt-informatik-haupt}
\liLadePakete{syntax}

\begin{document}
%%%%%%%%%%%%%%%%%%%%%%%%%%%%%%%%%%%%%%%%%%%%%%%%%%%%%%%%%%%%%%%%%%%%%%%%
% Theorie-Teil
%%%%%%%%%%%%%%%%%%%%%%%%%%%%%%%%%%%%%%%%%%%%%%%%%%%%%%%%%%%%%%%%%%%%%%%%

\chapter{SQL-Grundlagen}

%-----------------------------------------------------------------------
%
%-----------------------------------------------------------------------

\section{Datentypen
\footcite[Seite 132]{winter}}

\begin{description}
\item[CHAR(n)] Zeichenreihe (String) einer festen Länge von n Zeichen

\item[VARCHAR(n)] Zeichenreihe variabler Länge (String) mit bis zu n
Zeichen

\item[INTEGER, SMALLINT] Ganze Zahl (mit oder ohne Vorzeichen)

\item[NUMERIC(p,q), DEC(p,q)] Dezimalzahl mit p Ziffern insgesamt und q
Ziffern nach dem Dezimalpunkt

\item[FLOAT(n), FLOAT, REAL] Gleitkommazahl mit Stelligkeit n oder
systemabhängiger Genauigkeit

\item[DATE, TIME, TIMESTAMP] Datum und/oder Uhrzeit

\item[BOOLEAN] Boolean constants can be represented in SQL queries by
the SQL key words TRUE, FALSE, and NULL
\end{description}

\section{CREATE TABLE}

\begin{minted}{sql}
CREATE TABLE Tabellennamen
(
   Spaltenname1 Datentyp1 [Spaltenbedingung]
   Spaltenname2 Datentyp2 [Spaltenbedingung]
   [...]
);
\end{minted}

%-----------------------------------------------------------------------
%
%-----------------------------------------------------------------------

\section{Mengenoperationen\footcite[Seite 157]{winter}}

\noindent
\begin{tabular}{ll}
\textbf{Mengenoperation} &
\textbf{SQL - Schlüsselwort} \\

Vereinigung von Mengen &
UNION \\

Schnitt von Mengen &
INTERSECT \\

Differenz von Mengen &
EXCEPT \\
\end{tabular}

%-----------------------------------------------------------------------
%
%-----------------------------------------------------------------------

\section{Referentielle Integrität\footcite[Seite 14]{db:fs:3}}

Unter Referentieller Integrität verstehen wir Bedingungen, die zur
Sicherung der Datenintegrität bei Nutzung relationaler Datenbanken
beitragen können. Demnach dürfen Datensätze über ihre
Fremdschlüssel nur auf existierende Datensätze verweisen.

Danach besteht die Referentieller Integrität grundsätzlich aus zwei
Teilen:

\begin{enumerate}

\item Ein neuer Datensatz mit einem Fremdschlüssel kann nur dann in
einer Tabelle eingefügt werden, wenn in der referenzierten Tabelle ein
Datensatz mit entsprechendem Wert im Primärschlüssel oder einem
eindeutigen Alternativschlüssel existiert.

\item Eine Datensatzlöschung oder Änderung des Schlüssels in einem
Primär-Datensatz ist nur möglich, wenn zu diesem Datensatz keine
abhängigen Datensätze in Beziehung stehen.\footcite{wiki:referentielle-Integritaet}
\end{enumerate}

\section{Integritätsbedingungen:
\footcite[Seite 7]{db:fs:2}}

\begin{description}

\item[NOT NULL]

Alle eingetragenen Tupel müssen an dieser Stelle einen definierten Wert
haben.

\item[UNIQUE]

Kennzeichnung von Schlüsselkandidaten. Gibt an, dass dieses Attribut
eindeutig sein muss.

\item[PRIMARY KEY]

Primärschlüssel-Kennzeichnung, impliziert automatisch \verb|NOT NULL|,
jede Tabelle benötigt einen Primärschlüssel!

\begin{minted}{sql}
CREATE TABLE CUSTOMERS (
   ID INTEGER NOT NULL,
   NAME VARCHAR(20) NOT NULL,
   PRIMARY KEY (ID, NAME)
);
\end{minted}

\item[FOREIGN KEY]

Kennzeichnung von Fremdschlüsseln, null-Werte möglich, wenn nicht
explizit \verb|NOT NULL| gesetzt (\verb|UNIQUE FOREIGN KEY| um
1:1-Beziehung zu modellieren)

\item[CHECK (Attributname IN / (BETWEEN) AND)]

Überprüfung der angegebenen Werte auf eine bestimmte Eigenschaft.

\item[DEFAULT Wert]

setzt das Attribut auf einen bestimmten Standardwert, wenn kein Wert für
das Attribut angegeben wird

\item[ON DELETE / ON UPDATE SET NULL / SET DEFAULT / CASCADE]

für Fremdschlüssel, setzt im Fall des Updates bzw. Löschens des
Primärschlüssels in der anderen Tabelle diesen Attributwert auf
\verb|NULL| / einen Defaultwert oder ändert / löscht ihn kaskadierend
\end{description}

%-----------------------------------------------------------------------
%
%-----------------------------------------------------------------------

\section{Aggregatsfunktionen\footcite[Seite 10]{db:fs:2}}

\begin{description}
\item[AVG()] berechnet den arithmetischen Mittelwert einer Menge von
Werten in einem bestimmten Feld einer Abfrage

\item[COUNT()] berechnet die Anzahl der von einer Abfrage
zurückgegebenen Datensätze.

\item[MAX()/ MIN()] - gibt den größten / kleinsten Wert aus einer Reihe
von Werten zurück, die in einem bestimmten Feld einer Abfrage enthalten
sind.

\item[SUM()] - berechnet die Summe einer Menge von Werten in einem
bestimmten Feld einer Abfrage.
\end{description}

\subsection{Besonderheiten bei Aggregatoperationen}

\begin{compactitem}
\item SQL erzeugt pro Gruppe ein Ergebnistupel

\item Deshalb müssen alle in der SELECT-Klausel aufgeführten Attribute -
außer den aggregierten – auch in der GROUP BY-Klausel aufgeführt werden

\item Nur so kann SQL sicherstellen, dass sich das Attribut nicht
innerhalb der Gruppe ändert
\end{compactitem}

%-----------------------------------------------------------------------
%
%-----------------------------------------------------------------------

\section{Geschachtelte Anfragen in SQL}

(Foliensatz Seite 26)

SQL erlaubt das Schachteln von Anfragen

%%
%
%%

\subsection{Unterscheidung nach Anzahl der Rückgabetupel}

Unterscheidung von Anfragen, die nur ein Tupel zurückgeben und Anfragen,
die beliebig viele Tupel ergeben

Wenn nur ein Tupel mit nur einem Attribut zurückgeliefert wird, dann
kann die Anfrage dort eingesetzt werden, wo ein skalarer Wert gefordert
wird $\rightarrow$ vor allem in WHERE und SELECT-Klauseln

%%
%
%%

\subsection{Korrelierte vs. Nicht-korrelierte Anfragen}

\begin{description}
\item[Nicht-korrelierte Anfragen:]

verwenden lediglich „eigene“ Attribute, müssen nur einmal ausgewertet
werden, zudem kann ihr Ergebnis materialisiert werden $\rightarrow$ wenn
möglich ist dies die bessere Lösung!

\begin{minted}{sql}
SELECT *
FROM prüfen
WHERE Note < (SELECT AVG (Note) FROM prüfen);
\end{minted}

\item[Korrelierte Anfragen:]

greifen auf Attribute der umschließenden Anfrage zu, werden daher für
jedes Tupel der umschließenden Anfrage neu berechnet
\end{description}

\begin{minted}{sql}
SELECT PersNr, Name, (
  SELECT SUM(SWS) AS Lehrbelastung
  FROM Vorlesungen
  WHERE gelesenVon = PersNr
)
FROM Professoren;
\end{minted}

%%
%
%%

\subsection{Verwertung der Ergebnismenge einer Unteranfrage}

\begin{minted}{sql}
SELECT tmp.MatrNr, tmp.Name, tmp.VorlAnzahl
FROM (
  SELECT s.MatrNr, s.Name, COUNT(*) AS VorlAnzahl
  FROM Studenten s, hören h
  WHERE s.MatrNr = h.MatrNr
  GROUP BY s.MatrNr, s.Name
) tmp
WHERE tmp.VorlAnzahl > 2;
\end{minted}

\noindent
Geht allerdings schöner: \verb|WITH| $\rightarrow$ erlaubt
Modularisierung, besonders nützlich, wenn ich das Ergebnis meiner
Unterabfrage in einer Reihe weiterer SQL-Abfragen ebenfalls benötige

\begin{minted}{sql}
WITH tmp AS (
  SELECT s.MatrNr, s.Name, COUNT(*) AS VorlAnzahl
  FROM Studenten s, hören h
  WHERE s.MatrNr = h.MatrNr
  GROUP BY s.MatrNr, s.Name
)
SELECT tmp.MatrNr, tmp.Name, tmp.VorlAnzahl
FROM tmp
WHERE tmp.VorlAnzahl > 2;
\end{minted}

\section{Trigger}

\begin{minted}{sql}
CREATE [OR REPLACE] TRIGGER <trigger_name>
{BEFORE|AFTER|INSTEAD OF} {INSERT|DELETE|UPDATE [OF <feld_name>]}
  ON <table_name>
  [REFERENCING [NEW AS <new_row_name>] [OLD AS <old_row_name>]]
    [FOR EACH ROW [WHEN (<trigger_condition>)]]
    <trigger_body>
\end{minted}
\footcite{wiki:datenbanktrigger}

%-----------------------------------------------------------------------
%
%-----------------------------------------------------------------------

\section{Top-N-Query\footcite[Seite 8]{db:fs:3}}

Ermittle die TOP(10) der Spitzenverdiener für jede Abteilung und gib
den Rang und den Namen aus. (Alle Angestellten haben
unterschiedliches Gehalt)

\begin{minted}{sql}
SELECT P.AbtNr, COUNT(*) AS Rang, P.Nachname
FROM Angestellte P, Angestellte A
WHERE P.AbtNr = A.AbtNr AND P.Gehalt <= A.Gehalt
GROUP BY P.PersNr, P.Nachname, P.AbtNr
HAVING COUNT(*) <= 10;
\end{minted}

%-----------------------------------------------------------------------
%
%-----------------------------------------------------------------------

\section{Rekursive SQL-Abfragen}

\begin{liQuellen}
\item \cite[Seite 136]{kemper}
\item \cite[Seite 27-31]{db:fs:3}
\item \cite{db:html:jaxenter:rekursives-sql}
\end{liQuellen}

\begin{minted}{sql}
WITH RECURSIVE t(v) AS (
  SELECT 1     -- Seed Row
  UNION ALL
  SELECT v + 1 -- Recursion
  FROM t
)
SELECT v
FROM t
LIMIT 5
\end{minted}
\footcite{db:html:jaxenter:rekursives-sql}

Was ist Rekursion? Grundsätzliche Idee: Bestimmte Vorgänge werden auf
ein Produkt, das sie bereits hervorgebracht haben, von neuem angewandt.
Dabei entstehen unendliche Abläufe, die mit Hilfe bestimmter Bedingungen
abgebrochen werden.

Wann ist das erforderlich?

Typisches Beispiel: In einem Unternehmen sollen Hierarchien erfasst
werden. Wir möchten für einen Chef alle Untergebenen ermitteln, aber
eben nicht nur auf der Hierarchieebene direkt unter ihm, sondern ALLE.
Dabei scheitert eine klassische Anweisung!

Basisaufbau:

\begin{minted}{sql}
WITH tabelle (spaltenliste) AS
( UrsprungsSELECT
UNION ALL
RekursionsSELECT )
SELECT spaltenliste FROM tabelle WHERE ...
\end{minted}

Hilfreiche Fragen zur Erstellung:

Welche Spalten sollen in meiner Ergebnismenge auftauchen und/oder werden
für die Rekursionsbedingung benötigt?

Wie lautet der SELECT für den Satz, von dem die Rekursion ausgehen soll?

Wie lautet die Rekursionsbedingung?

%-----------------------------------------------------------------------
%
%-----------------------------------------------------------------------

\section{Ansichten (VIEWS)}

Aus dem SQL92 Standard:

\begin{verbatim}
<view definition> ::=
    CREATE VIEW <table name> [ <left paren> <view column list>
    <right paren> ]
      AS <query expression>
      [ WITH [ <levels clause> ] CHECK OPTION ]

<levels clause> ::=
    CASCADED | LOCAL

<view column list> ::= <column name list>
\end{verbatim}
\footcite[Seite 296]{sql92}

Beispiele\footcite[Seite 140]{kemper}

\begin{minted}{sql}
CREATE VIEW StudProf (Sname, Semester, Titel, Pname) AS
  SELECT s.Name, s.Semester, v.Titel, p.Name
  FROM Studenten s, hören h, Vorlesungen v, Professoren p
    WHERE s.Matr.Nr = h.MatrNr AND
    h.VorlNr=v.VorlNr AND
    v.gelesenVon = p.PersNr;
\end{minted}
\footcite[Seite 11]{db:fs:3}
\literatur

\end{document}
