\documentclass{bschlangaul-aufgabe}
\bLadePakete{syntax}
\begin{document}
\bAufgabenTitel{Vorlesungsaufgaben}
\section{Vorlesungsaufgaben\index{Mehr-Adress-Befehl-Assembler}}

Geben Sie die Lösungen zu den Aufgaben aus der Assembler-Vorlesung ab.
Bearbeiten Sie erst danach die folgenden Aufgaben auf diesem
Übungsblatt.
\begin{enumerate}

%%
% (a)
%%

\item Folie 37/3,4

\begin{enumerate}

%%
% 3.
%%

\item Bestimmung der Summe der ersten $n$ Zahlen (iterativ).

\begin{bAntwort}
\bAssemblerDatei{Aufgabe_07-Vorlesungsaufgaben-Summe-iterativ.mi}

\bJavaDatei[firstline=3]{aufgaben/tech_info/assembler/mehr_adress/SummeIterativ}
\end{bAntwort}

%%
% 4.
%%

\item Bestimmung der $n$-ten Fibonaccizahl (iterativ).
\begin{bAntwort}
\bAssemblerDatei{Aufgabe_07-Vorlesungsaufgaben-Fibonacci-iterativ.mi}

\bJavaDatei[firstline=3]{aufgaben/tech_info/assembler/mehr_adress/FibonacciIterativ}
\end{bAntwort}
\end{enumerate}

%%
% (b)
%%

\item Folie 57/1,2

\begin{enumerate}

%%
% 1.
%%

\item zur Multiplikation zweier Zahlen unter Verwendung eines
Unterprogramms

\begin{bAntwort}
\bAssemblerDatei{Aufgabe_07-Vorlesungsaufgaben-Multiplikation-Unterprogramm.mi}

\bJavaDatei[firstline=3]{aufgaben/tech_info/assembler/mehr_adress/MultiplikationUnterprogramm}
\end{bAntwort}

%%
% 2.
%%

\item Summe der ersten $n$ Zahlen (rekursiv)

\begin{bAntwort}
\bAssemblerDatei{Aufgabe_07-Vorlesungsaufgaben-Summe-rekursiv.mi}

\bJavaDatei[firstline=3]{aufgaben/tech_info/assembler/mehr_adress/SummeRekursiv}
\end{bAntwort}
\end{enumerate}
\end{enumerate}
\end{document}
