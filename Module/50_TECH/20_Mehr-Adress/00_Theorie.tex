\documentclass{bschlangaul-theorie}
\bLadePakete{syntax}
\begin{document}

%%%%%%%%%%%%%%%%%%%%%%%%%%%%%%%%%%%%%%%%%%%%%%%%%%%%%%%%%%%%%%%%%%%%%%%%
% Theorie-Teil
%%%%%%%%%%%%%%%%%%%%%%%%%%%%%%%%%%%%%%%%%%%%%%%%%%%%%%%%%%%%%%%%%%%%%%%%

\chapter{Assembler: Mehr-Adress-Befehle}

% Info_2020-11-20-2020-11-20_09.49.27.mp4 1h16min

% 1h41min Installation

\section{Datentypen\footcite[Seite 6]{borghoff}}

\begin{tabular}{ll}
B & Byte, Zeichen; 8 Bit breit\\
H & Halbwort; 16 Bit breit\\
W & Wort; 32 Bit breit\\
F & Gleitpunktzahl (floating point); 32 Bit breit\\
D & Double-float; 64 Bit breit\\
\end{tabular}

\section{Register}

Jeder Rechnerkern besitzt 16 frei programmierbare Register, die von 0
bis 15 numeriert sind. Die Wortbreite eines Registers beträgt 32 Bit.
Jedoch kann durch geeignete Kennung auch auf Teile eines Registers
zugegriffen werden (bei \textbf{B} auf das rechte Byte, bei \textbf{H}
auf das rechte Halbwort und bei \textbf{W} auf das gesamte Register). Es
gibt zwei ausgezeichnete Register: Der Befehlszähler \textbf{PC}
(Program Counter) enthält die Adresse des nächsten auszuführenden
Befehls bzw. des noch zu interpretierenden Teils des aktuellen Befehls.
Das Register R15 ist gleichzeitig der Befehlszähler PC. Der Kellerpegel
SP (Stack Pointer) enthält die Adresse des aktuellen Kellers. Das
Register R14 ist gleichzeitig der Kellerpegel SP. Es gibt noch eine
ganze Reihe von Sonderregistern, die im folgenden noch erklärt werden.
Alle Adressen sind vorzeichenlose ganze Zahlen im Wortformat (32 Bit
Adressen).
\footcite[Seite 8, 9]{borghoff}

Direkter (immediater) Operand: I

MOVE B I 2,R1
\footcite[Seite 24]{borghoff}

SEG: <Segmentname> SEG END\footcite[Seite 22]{borghoff}

Transportbefehle

MOVE B|H|W|F|D $a_1$, $a_2$\footcite[Seite 26]{borghoff}

H'10000' = Hexadecimal literal

HALT Besonderheit: Der HALT -Befehl führt bei der 4-RK-MI zum Halt aller
vier Rechnerkerne.\footcite[Seite 32]{borghoff}

Datendefinition

DD\footcite[Seite 36]{borghoff}

PUSHR push registers\footcite[Seite 26]{borghoff}

POPR pop registers\footcite[Seite 26]{borghoff}

\subsection{Arithmetische Verknüpfungen}

ADD
SUB
MULT
DIV \footcite[Seite 27]{borghoff}

\begin{minted}{asm}
SEG
JUMP start
a:              DD H 1
b:              DD H 2
erg:            DD H -1

start:          ADD H a, b, erg
                MOVE H erg, R0
END
\end{minted}

\begin{minted}{asm}
division:
SEG
                -- dreiteilige Befehle
                -- 16 / 4 wird in Register 0 gespeichert
                DIV W I 4, I 16, R0 -- gibt 4
                DIV W I 4, I 17, R1 -- gibt auch 4
                DIV W b, a, erg
                MOVE W erg, R2

                -- zweiteiliger Befehle
                MOVE W I 16, R3
                DIV W I 4, R3

                HALT
                a: DD W 16 -- bei DD braucht man kein I
                b: DD W 4
                erg: DD W -1
END
\end{minted}

Unterprogrammsprung
CALL\footcite[Seite 28]{borghoff}

Unbedingter Sprungbefehl
JUMP\footcite[Seite 28]{borghoff}

\begin{minted}{asm}
spruchbefehle:
SEG
                CMP W a, b
                -- a > b wahr JLT falsch
                -- a < b wahr JGT falsch
                JGT falsch
wahr:
                MOVE W I 42, R0
                JUMP ende
falsch:         MOVE W I 23, R0
ende:           HALT

a:              DD W 1
b:              DD W 2
END
\end{minted}

Rückkehr aus Unterprogramm
RET\footcite[Seite 29]{borghoff}

\begin{minted}{asm}
schiebe_befehle:
SEG
                -- nach links schieben
                SH I 1, I 1, R0 -- 1 -> 10
                -- nach rechts schieben
                SH I -1, I 5, R1 -- 101 -> 10
END
\end{minted}

\literatur

\end{document}
