\documentclass{lehramt-informatik-haupt}

\begin{document}

%%%%%%%%%%%%%%%%%%%%%%%%%%%%%%%%%%%%%%%%%%%%%%%%%%%%%%%%%%%%%%%%%%%%%%%%
% Theorie-Teil
%%%%%%%%%%%%%%%%%%%%%%%%%%%%%%%%%%%%%%%%%%%%%%%%%%%%%%%%%%%%%%%%%%%%%%%%

\chapter{Didaktik}

\section{Allgemeinbildung nach Bussmann/Heymann}

„Ich gehe davon aus, daß die allgemeinbildenden Schulen in unserer
Gesellschaft vornehmlich folgende [..] Aufgaben zu erfüllen
haben:\footcite[Heymann.1997, Seite 4]{ddi:fs:1}

\begin{enumerate}
\item Lebensvorbereitung;
\item Stiftung kultureller Kohärenz;
\item Weltorientierung;
\item Anleitung zum kritischen Vernunftgebrauch;
\item Entfaltung von Verantwortungsbereitschaft;
\item Einübung in Verständigung und Kooperation;
\item Stärkung des Schüler-Ichs.“
\end{enumerate}

%%
%
%%

\subsection{1. Lebensvorbereitung}

„ .. so muß im Hinblick auf jedes Schulfach gefragt werden:

Inwieweit fallen in seinen Bereich Kenntnisse, Fähigkeiten und
Fertigkeiten, die im pragmatischen Sinne zu Bewältigung alltäglicher
Lebenssituationen beitragen und die ohne Schule nicht oder nicht in
hinreichendem Maße gelernt würden?“\footcite[Heymann.1997, Seite 5]{ddi:fs:1}

%%
%
%%

\subsection{2. Stiftung kultureller Kohärenz}

„Im Blick auf den Beitrag zur Allgemeinbildung ist unter dem soeben
erläuterten Aspekt für jedes Schulfach zu fragen:

Welche für unseren Kulturkreis, unser kulturelles und gesellschaftliches
Selbstverständnis zentralen kulturellen Errungenschaften werden in dem
betreffenden Fach tradiert?

Welche Bezüge zwischen der für das betreffende Fach (bzw. die
korrespondierenden Wissenschaften) charakteristischen Fachkultur und der
Gesamtkultur (bzw. anderen Subkulturen) gibt es?“\footcite[Seite 6]{ddi:fs:1}

%%
%
%%

\subsection{3. Weltorientierung}

„Deshalb ist mit Blick auf jedes Schulfach zu fragen:

Was von dem unüberschaubaren „Gebirge“ dessen, was man innerhalb dieses
Faches prinzipiell wissen könnte, ist so fundamental, so erhellend, so
beispielhaft, daß es dem Einzelnen helfen kann, eine Gesamtorientierung
zu finden, ein eigenes tragfähiges Weltbild aufzubauen?

Wie sind die Wissensinhalte, die die Lernenden sich in diesem Fach
aneignen sollten, untereinander und mit den Inhalten anderer Fächer
vernetzt?

Wo bieten sich Möglichkeiten, die Grenzen des Fachs zu überschreiten und
„Schlüsselprobleme“ zu thematisieren?

Wird das Exemplarische des Unterrichtsstoffs und seine Vernetzung mit
anderen Elementen des Weltwissens im herkömmlichen Fachunterricht
hinreichend deutlich? [..]“\footcite[Seite 7]{ddi:fs:1}

%%
%
%%

\subsection{4. Anleitung zum kritischen Vernunftsgebrauch}

„Jedes Schulfach hat sich somit den Fragen zu stellen:

\begin{itemize}
\item Bieten die üblichen Fachinhalte hinreichend Gelegenheit zum
kritischen Vernunftgebrauch?

\item Bietet die übliche Praxis des Fachunterrichts ein hinreichend
anregendes geistiges Klima, in dem sich das kritische Denken der
Schülerinnen und Schüler kultivieren läßt?

\item Welche Merkmale des traditionellen Fachunterrichts stehen der
Anleitung zum kritischen Vernunftgebrauch möglicherweise im Wege, und
wie könnte dem begegnet werden?“\footcite[Seite 8]{ddi:fs:1}
\end{itemize}

%%
%
%%

\subsection{5. Entfaltung von Verantwortungsbereitschaft}

An jedes Schulfach ist die Frage zu stellen:

\begin{itemize}
\item Bietet der betreffende Fachunterricht, die Art, wie mit den
Sachthemen und miteinander umgegangen wird, einen Rahmen für die
Entfaltung von Verantwortungsbereitschaft?

\item Gibt es Eigentümlichkeiten des Faches bzw. der fachspezifischen
Sozialisation, die diesem Ziel tendenziell im Wege stehen?\footcite[Seite 9]{ddi:fs:1}
\end{itemize}

%%
%
%%

\subsection{6. Einübung in Verständigung und Kooperation}

„Für die Schulfächer ergeben sich aus dem Ernstnehmen dieses Aspekts von
Allgemeinbildung Fragen, die sich vornehmlich auf die Methodik und die
„Unterrichtskultur“ richten:

\begin{itemize}
\item Bietet der übliche Fachunterricht hinreichend Gelegenheiten zur
Einübung in Verständigung und Kooperation?

\item Gibt es fachspezifische Besonderheiten, die die Einübung in
Verständigung und Kooperation eher behindern?“\footcite[Seite 10]{ddi:fs:1}
\end{itemize}

%%
%
%%

\subsection{7. Stärkung des Schüler-Ichs}

„Die Schulfächer haben sich also den Fragen zu stellen:

\begin{itemize}
\item Geben die herkömmlichen Inhalte und Unterrichtsmethoden genügend
Raum für die Förderung des Einzelnen im beschriebenen Sinne?

\item Welche Elemente der üblichen Fachkultur beeinträchtigen die
angestrebte Stärkung des Schüler-Ichs möglicherweise?“\footcite[Seite 11]{ddi:fs:1}
\end{itemize}

%-----------------------------------------------------------------------
%
%-----------------------------------------------------------------------

\section{Fundamentale Ideen}

Das Konzept der fundamentalen Ideen findet sich erstmals bei Bruner.
Nach Schwill (1993) findet sich keine konkrete Definition, jedoch einige
charakterisierende Aussagen. So heißt es,…

\begin{itemize}
\item … dass die Grundlagen eines jeden Faches jedem Menschen in jedem
Alter in irgendeiner Form beigebracht werden können.

\item … dass die basalen Ideen […] ebenso einfach wie durchschlagend
sind.

\item … dass ein Begriff eine ebenso umfassende wie durchgreifende
Anwendbarkeit besitzt.\footcite[Bruner.1960, Seite 12]{ddi:fs:1}
\end{itemize}

\noindent
Eine fundamentale Idee (bzgl. einer Wissenschaft) ist ein Denk-,
Handlungs-, Beschreibungs- oder Erklärungsschema, das

\begin{enumerate}

%%
% (1)
%%

\item in \emph{verschiedenen Bereichen} (der Wissenschaft) vielfältig
anwendbar oder erkennbar ist (Horizontalkriterium),

%%
% (2)
%%

\item auf \emph{jedem intellektuellen Niveau} aufgezeigt und vermittelt
werden kann (Vertikalkriterium),

%%
% (3)
%%

\item in der \emph{historischen Entwicklung} (der Wissenschaft) deutlich
wahrnehmbar ist und längerfristig relevant bleibt (Zeitkriterium),

%%
% (4)
%%

\item einen \emph{Bezug zu Sprache und Denken des Alltags und der
Lebenswelt} besitzt (Sinnkriterium).

%%
% (5)
%%

\item Das zur \emph{Annäherung an eine gewisse idealisierte
Zielvorstellung} dient, die jedoch faktisch möglicherweise unerreichbar
ist (Zielkriterium)\footcite[Seite 13]{ddi:fs:1}
\end{enumerate}

%-----------------------------------------------------------------------
%
%-----------------------------------------------------------------------

\section{Fachdidaktische Ansätze}\footcite[Forneck.1990, Seite 12]{ddi:fs:1}

\begin{enumerate}
\item Rechnerorientierung (~1968)
\item Algorithmenorientierung (~1972)
\item Anwendungsorientierung (~1976)
\item Benutzerorientierung (~1985)
\item Informationsorientierung
\end{enumerate}

%-----------------------------------------------------------------------
%
%-----------------------------------------------------------------------

\section{Informationszentrierter Ansatz}

%%
%
%%

\subsection{Darstellung von Information}
\begin{itemize}
\item Daten als Repräsentationen, auf denen Verarbeitungsprozesse
operieren: Datentypen und - strukturen, Speicher und Variablen als
Container für Daten

\item Repräsentation von Verarbeitungsvorschriften: Programmiersprachen,
Syntax Algorithmen, Programme und

\item Modellierungstechniken zur Repräsentation von Information(en) über
Systeme: zeitliche Abläufe, Dekomposition in Subsysteme, Kommunikation
mit der Außenwelt und zwischen den Subsystemen.\footcite[Seite 26,
Hubwieser.2000]{ddi:fs:1}
\end{itemize}

%%
%
%%

\subsection{Verarbeitung und Transport von Repräsentationen}

\begin{itemize}
\item Modelle von IS (als Beschreibung von Verarbeitungsautomaten),

\item Einsatz- und Anwendungsmöglichkeiten von IS, Grenzen
(Berechenbarkeit) und Kosten des Einsatzes (Komplexität, Effizienz),

\item zeitliche und räumliche Struktur von Informatiksystemen:
Aufteilung in Komponenten, Kooperation und Kommunikation der
Komponenten, zeitliche (u.U. nebenläufige) Abläufe,

\item Interaktionen von Informatiksystemen mit ihrer Umgebung in
zeitlichem, räumlichem, menschlichem und gesellschaftlichem Kontext:
Geschichte, Entwicklung, Betrieb, Bedienung, Ergonomie, Auswirkungen auf
die Arbeits- und Berufswelt.\footcite[Seite 27,
Hubwieser.2000]{ddi:fs:1}
\end{itemize}

%-----------------------------------------------------------------------
%
%-----------------------------------------------------------------------

\section{Kompetenz}

Nach Weinert (2001) versteht man Kompetenzen als

\begin{quote}
„die bei Individuen verfügbaren oder durch sie erlernbaren kognitiven
Fähigkeiten und Fertigkeiten, um bestimmte Probleme zu lösen, sowie die
damit verbundenen motivationalen, volitionalen und sozialen
Bereitschaften und Fähigkeiten, um die Problemlösungen in variablen
Situationen erfolgreich und verantwortungsvoll nutzen zu können.“\footcite[Forneck.1990, Seite 29]{ddi:fs:1}
\end{quote}

%-----------------------------------------------------------------------
%
%-----------------------------------------------------------------------

\section{Wissen}

Wissen lässt sich […] als Denkinhalt verstehen und Denken als das
Aktualisieren von Wissen. Allgemeiner gesagt: Gewissermaßen ist Wissen
der Inhalt und Denken die Form eines kognitiven
Prozesses.\footcite[Gruber.1999, 30]{ddi:fs:1}

%-----------------------------------------------------------------------
%
%-----------------------------------------------------------------------

\section{Lernziele}

Lernziele haben daher (zumindest) folgende Komponenten:
\begin{enumerate}
\item (Beobachtbare) Verhaltenskomponente
\item Wissenskomponente\footcite[31]{ddi:fs:1}
\end{enumerate}

%%
%
%%

\subsection{Funktionen}

\begin{enumerate}
\item Lernziele helfen, Themen unter einem bestimmten
\emph{Gesichtspunkt} oder in Hinsicht auf eine bestimmte
\emph{Verwendungssituation} zu behandeln.

\item Lernziele können heimliche \emph{Bevorzugungen} oder
\emph{Weglassungen} deutlich machen.

\item Lernziele helfen \emph{verhindern}, dass die Prüfungen, Tests,
Erfolgskontrollen, usw. etwas ganz anderes erfassen, als Lehrer und
Schüler mit dem Unterricht beabsichtigt und tatsächlich realisiert
haben.

\item Lernziele bieten einen Anhaltspunkt für die \emph{Verständigung}
zwischen Lehrern, Schülern, Eltern, Inspektoren, Politikern und anderen
Interessierten bei der Diskussion über die Lerninhalte der Schule.
\footcite[Seite 33]{ddi:fs:1}
\end{enumerate}

%%
%
%%

\subsection{Empirische Ergebnisse}

\begin{itemize}
\item Lehrer, die mit schriftlichen Lernzielen arbeiten, sind besser in
der Lage, lernpsychologische Prinzipien anzuwenden.

\item Die Schüler dieser Lehrer erreichen die höheren Schulleistungen.

\item Formulierte und mitgeteilte Lernziele schaffen Transparenz und
damit eine entspanntere Lernatmosphäre\footcite[Seite 33]{ddi:fs:1}
\end{itemize}

%%
%
%%

\subsection{Zielebenenmodell nach Eigenmann/Strittmacher}

\begin{enumerate}
\item Die \emph{Leitidee} beantwortet die Frage: wozu und warum
überhaupt dieser Unterricht, diese Vorlesung, diese Stunde, dieses
Thema?

\item Das \emph{Dispositionsziel} beantwortet die Frage: was sollen die
Schüler nach dem Unterricht grundsätzlich können? Ausgedrückt als
Disposition oder Verhalten.

\item Das \emph{operationalisierte Lernziel} beantwortet die Frage:
welches konkrete, beobachtbare Tun werden die Schüler nach dem
Unterricht beherrschen und zeigen können?\footcite[Seite 34]{ddi:fs:1}
\end{enumerate}

%%
%
%%

\subsection{Komponenten von operationalisieren Lernzielen}

\begin{enumerate}
\item Benennung des Endverhaltens, das direkt beobachtbar ist

\item Eindeutige Bezeichnung des Gegenstandes, auf den sich das Lernziel
bezieht

\item Beschreibung der Voraussetzungen). notwendigen Bedingungen (z.B.
erlaubte Hilfsmittel,

\item Angabe des Beurteilungsmaßstabes für das als ausreichend geltende
Verhalten.\footcite[Seite 35]{ddi:fs:1}
\end{enumerate}

%%
%
%%

\subsection{Einteilung von Lernzielen}

\begin{enumerate}
\item \emph{Global} objectives: “Complex, multifaced learning outcomes
that require substantial time and instruction to accomplish”;

\item \emph{Educational} objectives: derived from global objectives by
breaking“them down into a more focused, delimited form”;

\item \emph{Instructional} objectives, with the purpose “to focus
teaching and testing on narrow, day-to-day slices of learning in fairly
specific content areas”.
\end{enumerate}

%%
%
%%

\subsection{Lernzieltaxonomien}

\subsubsection{Taxonomie von Bloom [1956]}

\begin{enumerate}
\item Wissen
\item Verständnis
\item Anwendung
\item Analyse
\item Synthese
\item Beurteilung\footcite[Seite 37]{ddi:fs:1}
\end{enumerate}

\subsubsection{Bloom‘s revised Taxonomie}

[Anderson/Krathwohl.2001] übersetzt in [Schobel.2004]
\footcite[Seite 38-39]{ddi:fs:1}

\begin{enumerate}
\item Die kognitive Prozess-Dimension

\begin{enumerate}
\item \textbf{E}rinnern
\item \textbf{V}erstehen
\item \textbf{A}nwenden
\item \textbf{A}nalysieren
\item \textbf{B}ewerten
\item (\textbf{Er})schaffen
\end{enumerate}

Merkwort: Eva, aber

\item Die Wissens-Dimension

\begin{enumerate}
\item Faktenwissen
\item Begriffliches Wissen
\item Verfahrensorientiertes Wissen
\item Metakognitives Wissen
\end{enumerate}
\end{enumerate}

\section{SOLO Taxonomy}

Structure of the Observed Learning Outcome

Klassifiziert Lernergebnisse unter Beachtung ihrer Komplexität. Die
Lernergebnisse werden in vier Kategorien eingeteilt, die jeweils die
Basis für die nächste ist. Jede Kategorie kann einer gewissen kognitiven
Entwicklungsstufe zugeordnet werden.

\begin{description}
\item[1. Unistructural:]

Nur einzelne voneinander unabhängige Aspekte werden erfasst.

\item[2. Multistructural:]

Einige voneinander unabhängige Aspekte werden erfasst.

\item[3. Relational:]

Verschiedene Aspekte werden zu einem Ganzen vereinigt.

\item[4. Extended Abstract:]

Das Ganze kann generalisiert und in einem neuen Kontext angewandt
werden.\footcite[Biggs.1982, Seite 42]{ddi:fs:1}
\end{description}

%-----------------------------------------------------------------------
%
%-----------------------------------------------------------------------

\section{Informatische Bildung}

\begin{enumerate}

\item Medieneinsatz: Keine Thematisierung von Informatik-Systemen

\begin{itemize}
\item Beschaffung von Informationen aus dem Internet,
\item Simulation des waagrechten Wurfs im Physikunterricht,
\item Vokabellernen mit einem Vokabeltrainer
\end{itemize}

\item Bedienerschulung: Thematisierung konkreter Informatik-Systemen

\begin{itemize}
\item Umgang mit MS-Word 2019 oder Excel 2019
\item Bedienung von Firefox 38
\item Arbeiten mit MS-Windows 10
\end{itemize}

\item Informatikunterricht: Thematisierung langlebiger, übertragbarer
Grundlagen von abstrahierten Informatik-Systemen

\begin{itemize}
\item Entwurf eines Algorithmus zu einer bestimmten Aufgabenstellung
\item Klassenmodell einer bestimmten Software
\item Blockschaltbild einer Rechenanlage\footcite[Hubwieser.2000, Seite 44]{ddi:fs:1}
\end{itemize}
\end{enumerate}
\literatur

\end{document}
