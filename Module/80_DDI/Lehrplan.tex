
\documentclass{bschlangaul-haupt}

\begin{document}

%-----------------------------------------------------------------------
%
%-----------------------------------------------------------------------

\chapter{6. Jahrgangsstufe: NT6 2 Schwerpunkt Informatik (ca. 28 Std.)}

%%
%
%%

\section{NT6 2.1 Informationsdarstellung mit Grafik-, Text- und Multimediadokumenten (ca. 18Std.)}

\subsection{Kompetenzerwartungen}

Die Schülerinnen und Schüler ...

\begin{itemize}

\item interpretieren, vergleichen und bewerten verschiedene
Darstellungen von Informationen.

\item analysieren Grafik-, Text- und Multimediadokumente aus
objektorientierter Sicht. Damit entwickelt sich das Verständnis, dass
die Dokumente Informationseinheiten (\zB Absätze bei Textdokumenten,
Folien bei Multimediadokumenten) enthalten, die durch spezifische
Eigenschaften charakterisiert sind und die miteinander in Beziehung
stehen. So erhalten die Schülerinnen und Schüler einerseits Einblick in
ein für die Informatik zentrales Modellierungskonzept; andererseits wird
ihr Verständnis vertieft, dass diese Dokumentarten jeweils eine typische
Grundstruktur haben, was die Nutzung entsprechender Software unabhängig
von einem speziellen Produkt erleichtert.

\item beschreiben Objekte (Informationseinheiten) durch ihre
Eigenschaften sowie Modifikationen an diesen Objekten (insbesondere
Attributwertänderungen mithilfe von Methoden). Hierbei verwenden sie
eine einfache, einheitliche und intuitiv verständliche
Beschreibungssprache in Form der Punktnotation (einer typischen
Notationsform in der Informatik).

\item abstrahieren Informationseinheiten gleicher Art zu Klassen,
interpretieren diese als Bauplan für Objekte und erweitern damit ihre
Fertigkeiten im objektorientierten Modellieren.

\item stellen Struktur und Beziehungen der Informationseinheiten durch
Objekt- und Klassendiagramme sachgerecht dar. Damit steht ihnen eine
übersichtliche grafische Darstellungsmöglichkeit zur Verfügung, die auch
in vielen anderen Bereichen der Informatik verwendet wird.

\item nutzen situationsgerecht Grafik-, Text- und Präsentationsprogramme
zur Gestaltung einfacher Dokumente, um Informationen aussagekräftig
darzustellen.

\item beurteilen anhand von Praxisbeispielen (\zB Foto, Grundrissplan),
ob sich für die Darstellung einer Information abhängig vom Einsatzzweck
eine Vektor- oder Pixelgrafik besser eignet.
\end{itemize}

\subsection{Inhalte zu den Kompetenzen:}

\begin{itemize}
\item Darstellungsformen von Information, \zB Text, Bild, Diagramm, Ton,
Pantomime, Zeichnung, Skizze

\item allgemeine Aspekte der Analyse bzw. Modellierung: Objekt,
Attribut, Methode, Beziehung, Klasse

\item Analyse bzw. Modellierung von Vektorgrafikdokumenten, u.a.
mithilfe der Klassen GRAFIKDOKUMENT, TEXTFELD, RECHTECK oder QUADRAT,
ELLIPSE oder KREIS, LINIE und der Enthält-Beziehung

\item Analyse bzw. Modellierung von Textdokumenten, u.a. mithilfe der
Klassen TEXTDOKUMENT, ABSATZ und ZEICHEN und der Enthält-Beziehung

\item Analyse bzw. Modellierung von Multimediadokumenten, u.a. mithilfe
der Klassen MULTIMEDIADOKUMENT, FOLIE und \zB BILD oder AUDIO und der
Enthält-Beziehung; ggf. Animieren von Objekten mithilfe entsprechender
Methoden

\item Objekt- und Klassenkarte, Objekt- und Klassendiagramm,
Punktnotation

\item Vektor- und Pixelgrafik als unterschiedliche Grafikarten

\item grundlegende Funktionen von Standardsoftware (\zB Speichern,
Öffnen, Schließen, Kopieren, Einfügen)

\item Verbesserung der Informationsdarstellung durch geeignetes Ändern
von Attributwerten (\zB Textformatierung)

\item spezielle Aspekte der Softwarebeherrschung bei
Vektorgrafiksoftware (\zB Gruppieren, Anordnung der Ebenen)

\item spezielle Aspekte der Softwarebeherrschung bei
Textverarbeitungssoftware (\zB Einfügen von Bildern, Erstellen von
Tabellen)

\item spezielle Aspekte der Softwarebeherrschung bei
Präsentationsprogrammen: Bearbeitungs- und Vorführmodus, ggf. Animation

\item Fachbegriffe: Attribut, Methode, Klasse, Beziehung, Punktnotation,
Objektdiagramm, Klassendiagramm
\end{itemize}

%%
%
%%

\section{NT6 2.2 Projekt: Erstellen einer Multimediapräsentation (ca. 5Std.)}

\subsection{Kompetenzerwartungen}

Die Schülerinnen und Schüler ...

\begin{itemize}
\item erstellen innerhalb eines vorgegebenen Zeitrahmens eine
ansprechende Multimediapräsentation zu einem vorgegebenen Thema und
berücksichtigen dabei sinnvolle Kriterien für die Qualität einer
Präsentationsgestaltung.

\item beachten bei der Zusammenstellung der Inhalte für die
Multimediapräsentation grundlegende Vorgaben des Urheberrechts.
\end{itemize}

\subsection{Inhalte zu den Kompetenzen:}

\begin{itemize}
\item Kriterien für die Qualität einer Multimediapräsentation, \zB
Textanteil je Folie, Schriftgröße, Farbwahl, zielgerichtete Auswahl von
Animationen

\item Urheberrecht im Kontext der Erstellung von schulischen Arbeiten,
Quellenangabe
\end{itemize}

%%
%
%%

\section{NT6 2.3 Hierarchische Informationsstrukturen – Dateisystem (ca. 5Std.)}

\subsection{Kompetenzerwartungen}

Die Schülerinnen und Schüler ...

\begin{itemize}
\item ordnen Informationen aus geeigneten einfachen Beispielen ihrer
Erfahrungswelt hierarchisch (\zB Stammbaum).

\item stellen baumartige Informationsstrukturen mithilfe entsprechender
Baumdiagramme dar.

\item analysieren die in einem Dateisystem abgelegte Anordnung von
Dateien und Ordnern, erkennen die zugrunde liegende hierarchische
Struktur und stellen diese in Objektdiagrammen und abstrahiert als
Klassendiagramm dar.

\item nutzen einen Dateimanager, um Dateien und Ordner in einer
hierarchischen Struktur geeignet zu organisieren. Damit können sie
Dateien sicher abspeichern und wiederfinden.

\item geben die Lage von Dateien, die auf einem Rechner gespeichert
sind, durch Pfade an. Damit beherrschen sie eine grundlegende Technik
für den Zugriff auf Dateien, wie sie beispielsweise in Netzwerken
eingesetzt wird.
\end{itemize}

\subsection{Inhalte zu den Kompetenzen:}

\begin{itemize}
\item Modellierung der hierarchischen Struktur im Dateisystem, u.a. mit
den Klassen DATEI und ORDNER und der rekursiven Enthält-Beziehung der
Klasse ORDNER

\item Baum als Möglichkeit der Darstellung bestimmter hierarchischer
Strukturen: Wurzel, Knoten, Kante, Blatt, Pfad

\item Fachbegriffe: Ordner, Datei, Pfad, Baum, Wurzel, Knoten, Kante,
Blatt
\end{itemize}

%-----------------------------------------------------------------------
%
%-----------------------------------------------------------------------

\chapter{7. Jahrgangsstufe: 2 Schwerpunkt Informatik (ca. 28Std.) }

%%
%
%%

\section{NT7 2.1 Vernetzte Informationsstrukturen (ca. 8Std.) }

\subsection{Kompetenzerwartungen}

Die Schülerinnen und Schüler ...

\begin{itemize}
\item analysieren Strukturen vernetzter Informationen aus ihrem
Lebensumfeld (\zB Lexikoneinträge, Webseiten der Schule) und modellieren
entsprechende Hypertextstrukturen mithilfe objektorientierter Konzepte.

\item nutzen effektive Strategien zur Informationsbeschaffung im
Internet und bewerten ihre Suchergebnisse kritisch hinsichtlich
Wahrheitsgehalt und Qualität, indem sie \zB die Art der
Informationsquellen (etwa Werbeseiten, Blogeinträge, wissenschaftliche
Publikationen) berücksichtigen.

\item nutzen die grundlegenden Funktionen eines geeigneten Werkzeugs
(WYSIWYG-Editor) zur Erstellung einer Hypertextstruktur; hierbei
berücksichtigen sie grundlegende Aspekte des Urheberrechts und wenden
einfache Zitiernormen (Quellenangaben) an.

\item beschreiben den prinzipiellen Mechanismus für die Übermittlung
elektronischer Dokumente unter Verwendung ihres Wissens über Struktur
und Funktionsweise des Internets.
\end{itemize}

\subsection{Inhalte zu den Kompetenzen:}

\begin{itemize}
\item Analyse und Modellierung von Hypertextstrukturen, u.a. mithilfe
der Klasse VERWEIS; Zieladresse als Attribut eines Verweises

\item Darstellung der Objektstruktur einer Hypertextstruktur: Graph,
bestehend aus Knoten und Kanten

\item Struktur und Funktionsweise des Internets: Client, Server,
Vermittlungsrechner (Router); Dienste (u.a. World Wide Web); einfache
Beispiele für die Adressierung im Internet (\zB IP-Adresse, URL)

\item Informationsquellen im Internet, \zB Suchmaschinen, Enzyklopädien

\item Fachbegriffe: Hypertext, Verweis, Graph, Client, Server
\end{itemize}

%%
%
%%

\section{NT7 2.2 Chancen und Risiken digitaler Kommunikation (ca. 5Std.) }

\subsection{Kompetenzerwartungen}

Die Schülerinnen und Schüler ...

\begin{itemize}
\item bewerten verschiedene digitale Kommunikationsmöglichkeiten
hinsichtlich ihrer Chancen, Risiken und Auswirkungen auf das Individuum
und die Gesellschaft, um sie im privaten oder schulischen Kontext sicher
sowie aus medienpädagogischer Sicht sinnvoll einsetzen zu können.

\item beachten grundlegende Regeln zum Schutz von Persönlichkeitsrechten
im Rahmen der digitalen Kommunikation. Dabei wird ihnen das
Spannungsfeld zwischen dem Recht des Einzelnen und den Interessen
anderer bewusst.
\end{itemize}

\subsection{Inhalte zu den Kompetenzen:}

\begin{itemize}
\item digitale Kommunikationsmöglichkeiten, \zB E-Mail, Messenger,
soziale Netzwerke

\item Schutz eigener Endgeräte durch Software (\zB Firewall und
Virenschutz) und durch Verwendung sicherer Passwörter

\item Persönlichkeitsrecht und Datenschutz im Rahmen der digitalen
Kommunikation, u.a. personenbezogene Daten, Recht am eigenen Bild

\item Gefahrenpotenziale (\zB Viren, Trojaner, gefälschte virtuelle
Identität, Abofallen, Fake News), Verhaltensregeln im Kontext der
digitalen Kommunikation (auch unter Berücksichtigung ethischer Aspekte),
Cybermobbing als Beispiel eines Missbrauchs digitaler
Kommunikationsmittel
\end{itemize}

%%
%
%%

\section{NT7 2.3 Beschreibung von Abläufen durch Algorithmen (ca. 11Std.) }

\subsection{Kompetenzerwartungen}

Die Schülerinnen und Schüler ...

\begin{itemize}
\item analysieren und strukturieren geeignete Problemstellungen u.a. aus
ihrer Erfahrungswelt (\zB Bedienung eines Geräts), entwickeln
Algorithmen zu deren Lösung und beschreiben diese unter effizienter
Verwendung von Kontrollstrukturen.

\item setzen unter sinnvoller Nutzung algorithmischer Bausteine einfache
Algorithmen mithilfe geeigneter Programmierwerkzeuge um.
\end{itemize}

\subsection{Inhalte zu den Kompetenzen:}

\begin{itemize}
\item Algorithmus: Definition des Begriffs, Strukturelemente (Anweisung,
Sequenz, ein- und zweiseitig bedingte Anweisung, Wiederholung mit fester
Anzahl, Wiederholung mit Bedingung)

\item Fachbegriffe: Algorithmus, Anweisung, Sequenz, ein- und zweiseitig
bedingte Anweisung, Wiederholung mit fester Anzahl, Wiederholung mit
Bedingung
\end{itemize}

%%
%
%%

\section{NT7 2.4 Projekt (ca. 4Std.) }

\subsection{Kompetenzerwartungen}

Die Schülerinnen und Schüler ...

\begin{itemize}
\item entwickeln innerhalb eines festen Zeitrahmens im Team eine
Hypertextstruktur zu einem vorgegebenen Thema oder eine Lösung zu einer
geeigneten Aufgabenstellung aus der Algorithmik (\zB Physical
Computing).

\item nutzen ihre Informatikkenntnisse zur Sicherung von Zwischenständen
und ggf. zur Dokumentation ihres Arbeitsprozesses.
\end{itemize}

\subsection{Inhalte zu den Kompetenzen:}

\begin{itemize}

\item Grundlagen der Projektarbeit, insbesondere Zeitmanagement und
Arbeitsteilung, ggf. Dokumentation
\end{itemize}

%-----------------------------------------------------------------------
%
%-----------------------------------------------------------------------

\chapter{9. Jahrgangsstufe}

%%
%
%%

\section{Inf9 Lernbereich 1: Funktionen und Datenflüsse, Tabellenkalkulationsprogramm (ca. 14Std.) }

\subsection{Kompetenzerwartungen}

Die Schülerinnen und Schüler ...

\begin{itemize}
\item abstrahieren Daten verarbeitende Prozesse mit mehreren Eingaben
und einer Ausgabe zu Funktionen.

\item modellieren die durch Funktionen ausgelösten Datenflüsse mithilfe
von Datenflussdiagrammen.

\item entwickeln neue Funktionen durch Verkettung gegebener Funktionen.
Sie wenden damit ein grundlegendes Konzept der funktionalen Modellierung
an.

\item setzen zur automatisierten Datenverarbeitung Datenflussdiagramme
und Funktionen in Formeln eines Tabellenkalkulationssystems um und
überprüfen durch geeignete Eingaben Modell und Umsetzung.

\item lösen praxisnahe Aufgabenstellungen, beispielsweise aus dem
kaufmännischen Bereich oder der Mathematik, sachgerecht durch Anwendung
der funktionalen Sichtweise, realisieren ihre Lösung mit einem
Tabellenkalkulationsprogramm und bewerten deren Qualität. Dabei nutzen
sie grundlegende Möglichkeiten eines Tabellenkalkulationsprogramms, u.a.
sinnvolle Nutzung von Adressierung und passende Gestaltung.
\end{itemize}

\subsection{Inhalte zu den Kompetenzen:}

\begin{itemize}
\item Tabellenkalkulationsprogramm: Tabellenblatt, Zelle, Formel,
Funktion (auch vordefinierte Funktion), Zellbezug (relative und absolute
Adressierung)

\item Datenflussdiagramm: Repräsentation einer Funktion, Datenfluss,
Ein- und Ausgabe, Verteiler

\item Funktion: Interpretation als Daten verarbeitender Prozess,
vordefinierte Funktionen (u.a. bedingte Funktion), Verkettung von
Funktionen, Parameter

\item Fachbegriffe: Formel, Zellbezug (relativ, absolut), Funktion,
Datenflussdiagramm, Verteiler
\end{itemize}

%%
%
%%

\section{Inf9 Lernbereich 2: Grundlagen der Datenmodellierung und relationaler Datenbanksysteme (ca. 10Std.) }

\subsection{Kompetenzerwartungen}

Die Schülerinnen und Schüler ...

\begin{itemize}
\item analysieren Datenbestände einfacher Beispiele aus der Praxis (\zB
Mitgliederverzeichnis) und modellieren diese objektorientiert.

\item überführen objektorientierte Datenmodelle in entsprechende
Tabellenschemata und setzen diese in einem Datenbanksystem um.

\item konzipieren geeignete SQL-Abfragen, um zielgerichtet Informationen
aus einer Datenbanktabelle zu gewinnen.
\end{itemize}

\subsection{Inhalte zu den Kompetenzen:}

\begin{itemize}
\item objektorientiertes Datenmodell: Objekt, Klasse, Attribut

\item relationales Modell: Tabellenschema, Primärschlüssel, Datentyp

\item relationales Datenbanksystem: Datenbank,
Datenbankmanagementsystem, Tabelle, Datensatz

\item Abfrage: Interpretation als Funktion, Ergebnistabelle als Ergebnis einer Abfrage

\item Abfragesprache am Beispiel von SQL: select, from, where; Verknüpfung von Bedingungen

\item Fachbegriffe: (relationales) Datenbanksystem, (relationale)
Datenbank, Datenbankmanagementsystem, Tabellenschema, Datentyp,
Datensatz, Abfrage, Primärschlüssel, Ergebnistabelle
\end{itemize}

%%
%
%%

\section{Inf9 Lernbereich 3: Grundlagen der objektorientierten Modellierung und Programmierung (ca. 26Std.) }

\subsection{Kompetenzerwartungen}

Die Schülerinnen und Schüler ...

\begin{itemize}
\item analysieren Objekte aus ihrer Erfahrungswelt (\zB Fahrzeuge,
Personen) hinsichtlich ihrer Eigenschaften (Attribute) und Fähigkeiten
(Methoden) und abstrahieren sie zu Klassen. Sie stellen Objekte und
Klassen als Grundlage einer möglichen Implementierung grafisch dar.

\item deklarieren eine Klasse sowie die zugehörigen Attribute und
Methoden in einer objektorientierten Programmiersprache.

\item verwenden bei der Implementierung Wertzuweisungen, um
Attributwerte zu ändern, und interpretieren diese als Zustandsänderung
des zugehörigen Objekts.

\item formulieren unter Verwendung der Kontrollstrukturen Algorithmen zu
geeigneten Problemstellungen, u.a. durch grafische Darstellungen.

\item implementieren Methoden auf der Grundlage gegebener Algorithmen
objektorientiert, wobei sie sich des Unterschiedes zwischen
Methodendefinition und Methodenaufruf bewusst sind. Dabei nutzen sie
ggf. auch Methoden anderer Klassen.

\item analysieren, interpretieren und modifizieren Algorithmen, wodurch
sie die Fähigkeit erlangen, fremde Programme flexibel einzusetzen und
kritisch zu bewerten.

\item modellieren durch Klassendiagramme einfache
Generalisierungshierarchien zu geeigneten Strukturen aus ihrer
Erfahrungswelt.

\item implementieren mithilfe einer objektorientierten Sprache einfache
Generalisierungshierarchien; dabei nutzen sie das Konzept der Vererbung
sowie die Möglichkeit, Methoden zu überschreiben.
\end{itemize}

\subsection{Inhalte zu den Kompetenzen:}

\begin{itemize}
\item objektorientierte Konzepte, u.a. Objekt, Klasse, Attribut,
Attributwert, Methode

\item Variablenkonzept; Arten von Variablen: Parameter, lokale Variable
und Attribute; Übergabewert

\item Wertzuweisung zur Änderung von Variablenwerten

\item Methoden: Methodenkopf, Methodenrumpf, Methodendefinition,
Methodenaufruf, Übergabewert, Rückgabewert; Konstruktor als spezielle
Methode; Standardmethoden zum Geben und Setzen von Attributwerten

\item Algorithmus: Strukturelemente, grafische Darstellung, Pseudocode

\item Datentypen: ganze Zahlen, Gleitkommazahlen, Wahrheitswerte,
Zeichen, Zeichenketten

\item Generalisierung und Spezialisierung: Ober- und Unterklasse,
Vererbung von Attributen und Methoden an Unterklassen, Überschreiben von
Methoden

\item Fachbegriffe: Parameter, Übergabewert, Rückgabewert, lokale
Variable, Wertzuweisung, Konstruktor, Methodenkopf, Methodenrumpf,
Vererbung, Generalisierung, Spezialisierung, Oberklasse, Unterklasse
\end{itemize}

%%
%
%%

\section{Inf9 Lernbereich 4: Datenschutz (ca. 6Std.) }

\subsection{Kompetenzerwartungen}

Die Schülerinnen und Schüler ...

\begin{itemize}
\item bewerten Regelungen zum Datenschutz im Spannungsfeld zwischen den
Persönlichkeitsrechten des Einzelnen und wirtschaftlichen sowie
öffentlichen Interessen, beispielsweise bei der Fahndung nach
Straftätern.

\item nutzen das Internet verantwortungsvoll unter Berücksichtigung
ihrer Kenntnisse über Möglichkeiten und Risiken dieses Mediums und
reflektieren dabei, wodurch der Schutz persönlicher Daten erhöht und die
Gefahr des Missbrauchs minimiert werden kann. Insbesondere wägen sie
kriteriengeleitet ihren Umgang mit datenbankgestützten Diensten und
Portalen ab. In diesem Kontext reflektieren sie den Einfluss digitaler
Medien (\zB Messenger-Dienste, soziale Netzwerke) auf sich und die
Gesellschaft (Medienwirkung).

\item bewerten Chancen und Risiken der automatisierten Analyse von
großen Datenbeständen (Data-Mining), auch im Hinblick auf
gesellschaftliche Auswirkungen.
\end{itemize}

\subsection{Inhalte zu den Kompetenzen:}

\begin{itemize}
\item Datenschutzgesetze: Zweck, Grundsätze (\zB Verbotsprinzip mit
Erlaubnisvorbehalt), Rechte von Betroffenen

\item Datenschutz: Schutz personenbezogener Daten (insbesondere im
Kontext der Mehrbenutzerproblematik bei Datenbanken), Datenmissbrauch,
\zB Identitätsdiebstahl

\item Fachbegriffe: Datenschutz, Data-Mining
\end{itemize}

%-----------------------------------------------------------------------
%
%-----------------------------------------------------------------------

\chapter{10. Jahrgangsstufe}

%%
%
%%

\section{Inf10 Lernbereich 1: Datenmodellierung und relationale Datenbanksysteme (ca. 17Std.) }

Die Schülerinnen und Schüler ...

\begin{itemize}
\item analysieren und strukturieren mithilfe objektorientierter Konzepte
Datenbestände geeigneter Beispiele aus der Praxis (\zB
Bibliotheksverwaltung) und stellen das daraus entwickelte Datenmodell
als Klassendiagramm mit mehreren Klassen dar.

\item überführen objektorientierte Datenmodelle in entsprechende
relationale Modelle und setzen diese in einem Datenbanksystem um.

\item konzipieren geeignete SQL-Abfragen, um zielgerichtet Informationen
aus einer relationalen Datenbank zu gewinnen.

\item erkennen Redundanzen und Anomalien in einer relationalen Datenbank
und beurteilen die dadurch entstehende Problematik im Hinblick auf die
Konsistenz des Datenbestands.

\item bewerten Chancen und Risiken der automatisierten Analyse
verknüpfter Datenbestände, auch im Hinblick auf gesellschaftliche
Auswirkungen.
\end{itemize}

\subsection{Inhalte zu den Kompetenzen:}

\begin{itemize}
\item objektorientiertes Datenmodell: Objekt, Klasse, Attribut,
Beziehung, Kardinalität

\item relationales Modell: Tabellenschema, Datenbankschema, Primär- und
Fremdschlüssel, Datentyp

\item relationales Datenbanksystem

\item Redundanz und Konsistenz von Datenbeständen, Anomalien

\item Abfragesprache am Beispiel von SQL: Verknüpfung von Bedingungen;
Abfrage über verknüpfte Tabellen

\item Fachbegriffe: Datenbankschema, Primär- und Fremdschlüssel,
Kardinalität, Redundanz, Konsistenz, Anomalie
\end{itemize}

%%
%
%%

\section{Inf10 Lernbereich 2: Objektorientierte Modellierung und Programmierung (ca. 27Std.) }

\subsection{Kompetenzerwartungen}

Die Schülerinnen und Schüler ...

\begin{itemize}
\item wenden eine eindimensional indizierte Datenstruktur bei einfachen
Problemstellungen zur Speicherung und Verwaltung gleichartiger Daten an.

\item analysieren und modellieren Objektbeziehungen und das
Kommunikationsverhalten innerhalb eines Systems (beispielsweise eines
Fahrradverleihs).

\item implementieren die im Klassendiagramm festgelegten Beziehungen
sachgerecht durch Referenzen, um während der Laufzeit des Programms die
Kommunikation zwischen den entsprechenden Objekten durch den Aufruf
geeigneter Methoden zu ermöglichen. Dabei wird ihnen die Grundidee der
Datenkapselung bewusst.

\item modellieren und implementieren im Rahmen eines einfachen Beispiels
aus der Praxis (etwa der Mitgliederverwaltung eines Vereins) eine
Struktur zur Verwaltung gleichartiger Objekte. Hierbei entwickeln sie
ausgewählte Methoden, \zB zum Einfügen, Entfernen und Bearbeiten von
Elementen.

\item nutzen in Generalisierungshierarchien das Konzept des
Polymorphismus zur Implementierung verschiedener Verhaltensweisen, \zB
die spartenabhängige Berechnung der Mitgliedsbeiträge eines
Sportvereins.
\end{itemize}

\subsection{Inhalte zu den Kompetenzen:}

\begin{itemize}
\item eindimensional indizierte Datenstruktur (Array/Feld): Index,
Element, Länge

\item Interpretation von Klassen als Datentypen

\item Umsetzung von Klassenbeziehungen unterschiedlicher Kardinalitäten
mithilfe von Referenzen

\item Kommunikation zwischen Objekten durch Methodenaufrufe,
Datenkapselung durch kontrollierten Zugriff auf die Attribute

\item Polymorphismus und Überschreiben von Methoden

\item Fachbegriffe: Referenz, Kardinalität, Array/Feld, Index,
Datenkapselung, Polymorphismus
\end{itemize}

%%
%
%%

\section{Inf10 Lernbereich 3: Projekt (ca. 12Std.) }

\subsection{Kompetenzerwartungen}

Die Schülerinnen und Schüler ...

\begin{itemize}
\item erstellen im Team einen Projektplan, um eine Datenbank bzw. ein
objektorientiertes Programm zu einem Szenario aus ihrer Erfahrungswelt,
wie \zB einem Buchungssystem (ggf. mit Benutzerschnittstelle) oder einem
einfachen Spiel, zu entwickeln.

\item analysieren das gegebene Szenario und modellieren relevante
Ausschnitte durch ein Klassendiagramm und ggf. weitere Diagramme. Dabei
arbeiten sie ggf. mit bereits existierenden Modellen bzw.
Modellausschnitten.

\item implementieren zum gegebenen Szenario arbeitsteilig und auf
Grundlage der vorhandenen Modellierung die Datenbank einschließlich
geeigneter Abfragen bzw. das objektorientierte Programm.

\item testen die entwickelte Datenbank bzw. das entwickelte Programm
hinsichtlich der Anforderungen des gegebenen Szenarios.

\item dokumentieren ihre Projektarbeit geeignet.
\end{itemize}

\subsection{Inhalte zu den Kompetenzen:}

\begin{itemize}

\item Datenbank- bzw. Softwareentwicklungsprojekt: Planung,
Modellierung, Implementierung, Test, Dokumentation
\end{itemize}

%-----------------------------------------------------------------------
%
%-----------------------------------------------------------------------

\chapter{Jahrgangsstufe 11}

Information und Kommunikation sind zentrale Begriffe der Informatik. In
den Jahrgangsstufen 9 und 10 haben sich die Schüler insbesondere mit der
Analyse sowie mit der für eine maschinelle Verarbeitung geeigneten
Darstellung von Information beschäftigt und dabei verschiedene Techniken
der Modellierung kennengelernt. Darauf aufbauend lernen sie in
Jahrgangsstufe 11 neue Konzepte anzuwenden, die es ihnen erlauben,
größere Systeme effizienter zu modellieren. Sie greifen nun auf
rekursive Datenstrukturenzurück und erkennen deren Nutzen als häufig
verwendbare Modellierungsmuster. Bei der Erstellung größerer
Softwareprodukteeignen sichdie Schüler neben effizienten
Designstrategienhilfreiche Vorgehensweisen für die Arbeit in einem Team
an, die auch außerhalb informatischer Aufgabenstellungen gewinnbringend
einsetzbar sind.

Um die Möglichkeiten der Kommunikation zwischen Mensch und Maschine
besser beurteilen zu können, betrachten die Schüler in Jahrgangsstufe 12
das dabei verwendete Hilfsmittel Sprache. DieErkenntnis, dass für die
Kommunikation mit einer Maschine exakte Vereinbarungen unentbehrlich
sind, führt sie zum Begriff der formalen Sprache. Bei deren praktischer
Anwendung wird den Jugendlichen bewusst, dass der Kommunikation
Mensch-Maschine durch den nötigen Formalismus große Beschränkungen
auferlegt sind.

Beim weltweiten Austausch von Information spielt die Kommunikation
zwischen vernetzten Rechnern eine entscheidende Rolle. Die Schüler
lernen, dass es hierzu fester Regeln für das Format der auszutauschenden
Daten sowie für den Ablauf des Kommunikationsvorgangs bedarf. Der
gemeinsame Zugriff auf Ressourcen führt sie zum Problem der
Kommunikation und Synchronisation parallel ablaufender Prozesse, bei
dessen Lösung die Jugendlichen erneut den Anwendungsbereich ihrer
Modellierungstechniken erweitern.

Grundlegende Kenntnisse über den Aufbau eines Rechners und seiner
prinzipiellen Funktionsweise helfen den Schülern, den
Kommunikationsvorgang mit einer Maschine besser zu verstehen. Die
prinzipielle Automatisierbarkeit des Übersetzungsvorgangs von einer
höheren Programmiersprache in eine Maschinensprache wird ihnen bei der
Umsetzung einfacher Algorithmenmit einer maschinennahen Sprache
deutlich.

Ein wichtiges Maß für die Realisierbarkeit von Algorithmen ist die
Effizienz hinsichtlich des Zeitbedarfs. Bei der Untersuchung des
Laufzeitverhaltens ausgewählter Algorithmen erkennen die Jugendlichen
praktische Grenzen der Berechenbarkeit. Daneben gewinnen sie auch
Einblicke in theoretische Grenzen der Berechenbarkeit, sodass sie die
Einsatzmöglichkeitenautomatischer Informationsverarbeitung realistischer
einschätzen können.

Für interessierte Jugendliche bietet sich die Möglichkeit, Informatik
auch als Seminar zu wählen.

%%
%
%%

\section{Inf 11.1 Rekursive Datenstrukturen}

In Jahrgangsstufe 10 haben die Schüler im Rahmen der objektorientierten
Modellierung schwerpunktmäßig am Erstellen geeigneter Methoden von
Objekten sowie am Erkennen und Umsetzen von Objektbeziehungen
gearbeitet. Verschiedenartige Aufgabenstellungen aus der Praxis
verdeutlichen den Schülern nun, dass immer wieder Strukturen von Daten
auftreten, die mit den bisher bekannten Datentypen nicht effizient
beschrieben werden können. Die Schüler beschäftigen sich daher eingehend
mit typischen rekursiven Datentypen, lernen deren Vorteile und breites
Spektrum von Einsatzmöglichkeiten kennen und wenden sie an; dabeinutzen
sie erstmals rekursive Methoden.

%%
%
%%

\section{Inf 11.1.1 Listen  (ca.  29 Std.)}

Die Schüler untersuchen die grundlegenden Eigenschaften der
Datenstruktur Schlange, deren grundsätzlichen Aufbau sie bereits aus
ihrem Alltag, z. B. von Warteschlangen, kennen. Eine erste
Implementierung mit einem Feld zeigt schnell die Grenzen dieser
statischen Lösung auf und führt die Jugendlichen zu einer dynamischen
Datenstruktur wie der einfach verketteten Liste. Sie erarbeiten deren
prinzipielle Funktionsweise sowie deren rekursiven Aufbau und wenden
hierbei das Prinzip der Referenz auf Objekte an. Die Jugendlichen
erkennen, dass die rekursive Struktur der Liste für viele ihrer Methoden
einen rekursiven Algorithmus nahelegt. Sie verstehen, dass eine
universelle Verwendbarkeit der Klasse Liste nur möglich ist, wenn auf
eine klare Trennung von Struktur und Daten geachtet wird. An einfachen
Beispielen aus der Praxis und deren Implementierung vertiefen die
Schüler ihr Wissen und erfahren die flexible Verwendbarkeit dieses
Datentyps.

\begin{itemize}
\item Methoden der Datenstruktur Schlange: Anfügen am Ende, Entfernen am
Anfang

\item allgemeines Prinzip und rekursive Struktur einer einfach
verketteten Liste; graphische Veranschaulichung der Methoden zum
Einfügen (auch an beliebiger Stelle), Suchen und Löschen

\item rekursive Abläufe: rekursiver Methodenaufruf, Abbruchbedingung,
Aufrufsequenz

\item Implementierung einer einfach verketteten Liste als Klasse mittels
Referenzen unter Verwendung eines geeigneten Softwaremusters
(Composite); Realisierung der Methoden zum Einfügen, Suchen und Löschen

\item Einsatz der allgemeinen Datenstruktur Liste bei der Bearbeitung
eines Beispiels aus der Praxis: Verwaltung von Elementen verschiedener
Datentypen mittels Vererbung

\item Stapel und Schlange als spezielle Formen der allgemeinen
Datenstruktur Liste
\end{itemize}

%%
%
%%

\section{Inf 11.1.2 Bäume als spezielle Graphen  (ca.  29 Std.)}

In Erweiterung ihrer Grundkenntnisse über das hierarchische
Ordnungsprinzip lernen die Schüler den Baum als effiziente dynamische
Datenstruktur kennen. Sie stellen fest, dass damit viele Strukturen aus
anderen Gebieten abgebildet werden können. Am Beispiel der Suche in
umfangreichen Datenbeständen wird den Jugendlichen deutlich, dass sich
auch hier sehr oft Baumstrukturen einsetzen lassen, um die Effizienz der
Informationsverarbeitung zu steigern. Bei der sprachlichen bzw.
graphischen Darstellung und insbesondere bei der Implementierung dieser
Datenstruktur vertiefen sie ihr Verständnis für das Prinzip der
Rekur­sion. Im Rahmen praktischer Fragestellungen, z. B. zur Planung von
Verkehrsrouten, wenden die Schüler auch die Datenstruktur Graph als
Erweiterung der Struktur Baum an.

\begin{itemize}
\item allgemeines Prinzip und Struktur eines Baums (insbesondere Wurzel,
Knoten, Kanten, Blatt) und des Spezialfalls geordneter Binärbaum

\item Veranschaulichung und Implementierung der Methoden zum Einfügen
und Suchen von Elementen in einem geordneten Binärbaum unter Verwendung
der Rekursion

\item Verfahren zur Auflistung aller Elemente eines geordneten
Binärbaums: Preorder-, Inorder- und Postorder-Durchlauf; Realisierung
der Methode zum Ausgeben mithilfe eines dieser Verfahren

\item Implementierung der Klasse geordneter Binärbaum mit einer
geeigneten Programmiersprache; Verwenden und Testen der Methoden an
einem Anwendungsbeispiel (z. B. erweiterbares Wörterbuch als Suchbaum)

\item die Datenstruktur Graph als Verallgemeinerung des Baums;
Eigenschaften (gerichtet/ungerichtet, bewertet/unbewertet);
Adjazenzmatrix

\item Algorithmus zum Graphendurchlauf (z. B. Tiefensuche) bei einer
Aufgabenstellung aus der Praxis
\end{itemize}

%%
%
%%

\section{Inf 11.2 Softwaretechnik  (ca.  26 Std.)}

Das Arbeiten in Projekten ist die typische Vorgehensweise bei der
Entwicklung großer Systeme. Mit den bisher erworbenen Kenntnissen und
Fertigkeiten sind die Schüler nun in der Lage, größere Softwaresysteme
(z. B. Geschäftsabläufe einer Bank, Autovermietung) eigenständig zu
gestalten. Hierbei bauen sie auch ihre Fähigkeit zur Planung und
Durchführung von Projekten aus. Die Jugendlichen übernehmen verstärkt
persönliche Verantwortung und erfahren die Notwendigkeit, eigene
Ansichten und Ideen vor anderen darstellen und vertreten zu können. Sie
setzen alle bisher erlernten Beschreibungstechniken der Informatik ein
und machen sich damit deren Zusammenwirken in einem größeren Kontext
bewusst. Die Schüler erkennen, dass sie bei einigen Teilproblemen auf
bereits vorgefertigte Standardlösungen in Form von Softwaremustern
zurückgreifen können.

%%
%
%%

\section{Inf 11.2.1 Planung und Durchführung kooperativer Arbeitsabläufe}

Die Schüler systematisieren und vertiefen am konkreten Beispiel ihre
Kenntnisse über die verschiedenen Schritte bei der Planung und
Durchführung eines Softwareprojekts. Zur Koordinierung paralleler
Arbeitsgruppen nutzen sie das Semaphorprinzip.

\begin{itemize}
\item Projektplanung: Zielsetzung, Arbeitsteilung, Arbeitsgruppen und
deren organisatorische Schnittstellen, Ablauf mit Zwischenergebnissen
(Meilensteinen)

\item Phasen der Softwareentwicklung: Analyse mit Erstellung des
Pflichtenhefts, Entwurf, Implementierung, Test, Bewertung und Abnahme

\item Problematik der Koordination nebenläufiger Arbeitsabläufe beim
Zugriff auf gemeinsame Ressourcen: kritischer Abschnitt,
Semaphorprinzip, Verklemmung und Lösungsmöglichkeiten

\item Reflexion des Projektverlaufs, Aufwandsabschätzung
\end{itemize}

%%
%
%%

\section{Inf 11.2.2 Praktische Softwareentwicklung}

Bisher haben die Schüler verschiedene Modellierungstechniken der
Informatik einzeln angewandt. Nun erkennen sie, dass eine angemessene
Beschreibung größerer Systeme nur durch die kombinierte Verwendung aller
bisher erlernten Modellierungstechniken möglich ist. Bei der
Implementierung ihrer Modelle setzen sie bekannte Datenstrukturen
situationsgerecht ein und achten bei der Gestaltung der
Bedieneroberfläche insbesondere auf Benutzungsfreundlichkeit.

\begin{itemize}
\item Zusammenspiel der verschiedenen Beschreibungstechniken beim
Systementwurf: Datenmodellierung – Ablaufmodellierung – funktionale
Modellierung – Objektmodellierung

\item Implementierung des Systementwurfs unter Nutzung rekursiver
Datenstrukturen; Anwendung des Softwaremusters „Model-View-Controller“

\item Test der Komponenten und des Gesamtsystems, Überprüfung der
Vollständigkeit und Korrektheit des Systementwurfs

\item Dokumentation des Softwareprodukts
\end{itemize}

%%
%
%%

\chapter{Jahrgangsstufe 12}

%%
%
%%

\section{Inf 12.1 Formale Sprachen  (ca.  16 Std.)}

Bisher kennen die Schüler Sprachen vor allem als Mittel zur
Kommunikation zwischen Menschen. Ihnen ist bekannt, dass eindeutiges
gegenseitiges Verstehen nur dann gewährleistet ist, wenn sich die
Kommunikationspartner auf eine gemeinsame Sprache einigen und die zu
deren Gebrauch festgelegten Regeln einhalten. Im Rückblick auf das
bisherige Arbeiten mit dem Computer wird ihnen deutlich, dass die
Verständigung zwischen Mensch und Maschine ebenfalls einen
Kommunikationsprozess darstellt, der ähnlichen Regeln unterliegt. Daher
betrachten sie zunächst den strukturellen Aufbau einer ihnen bereits
bekannten natürlichen Sprache sowie den Aufbau einer künstlichen
Sprache. Die Jugendlichen lernen dabei, die Begriffe Syntax und Semantik
einer Sprache zu unterscheiden.

Anhand einfacher Beispiele wie Autokennzeichen, E-Mail-Adressen oder
Gleitkommazahlen lernen die Schüler den Begriff der formalen Sprache als
Menge von Zeichenketten kennen, die nach bestimmten Regeln aufgebaut
sind. Zur Beschreibung dieser Regeln verwenden sie Textnotationen oder
Syntaxdiagramme und können damit analog zu den natürlichen Sprachen
Grammatiken für formale Sprachen definieren. Die Zweckmäßigkeit der
streng formalen Beschreibung zeigt sich den Jugendlichen bei der
automatischen Überprüfung der syntaktischen Korrektheit von
Zeichenketten mithilfe von endlichen Automaten.

Den Schülern wird bewusst, dass nur Vorgänge, die sich mit Mitteln einer
formalen Sprache ausdrücken lassen, von einem Computer bearbeitet werden
können. Somit stoßen sie über die Theorie der formalen Sprachen auf eine
prinzipielle Grenze des Computereinsatzes.

\begin{itemize}
\item einfache Beispiele für formale Sprachen über einem Alphabet;
Zeichen, Zeichenvorrat (Alphabet), Zeichenkette

\item Unterscheidung zwischen Syntax und Semantik, Vergleich zwischen
natürlichen und formalen Sprachen

\item syntaktischer Aufbau einer formalen Sprache: Grammatik (Terminal,
Nichtterminal, Produktion, Startsymbol)

\item Notation formaler Sprachen: Syntaxdiagramm, einfache Textnotation
(z. B. Backus-Naur-Form)

\item erkennender, endlicher Automat als geeignetes Werkzeug zur
Syntaxprüfung für reguläre Sprachen; Implementierung eines erkennenden
Automaten
\end{itemize}

%%
%
%%

\section{Inf 12.2 Kommunikation und Synchronisation von Prozessen  (ca.  20 Std.)}

Aus vielen Bereichen der Computernutzung wie beispielsweise dem
Homebanking oder einem Buchungssystem für Reisen ist den Schülern die
Notwendigkeit zur Zusammenarbeit mehrerer Computer bekannt. Sie sehen
ein, dass zur korrekten Verständigung von Computern spezielle formale
Regeln (Protokolle) existieren müssen. Anhand der Kommunikation in
Rechnernetzen erfahren die Jugendlichen, dass es sinnvoll ist,
Kommunikationsvorgänge in verschiedene, aufeinander aufbauende Schichten
aufzuteilen.

Die Schüler erkennen z. B. bei der Analyse der Aufgaben eines
Mailservers, dass Rechner anfallende Aufträge oft parallel bearbeiten
und dennoch einen geordneten Ablauf gewährleisten müssen. Sie erarbeiten
Lösungsansätze für die Synchronisation derartiger Vorgänge an Beispielen
aus dem täglichen Leben wie der Regelung des Gegenverkehrs durch eine
einspurige Engstelle. Beim Übertragen dieser Ansätze auf die Prozesse im
Computer wird ihnen bewusst, dass es bei jedem dieser Verfahren
bestimmte Sequenzen gibt, die nicht von mehreren Prozessen gleichzeitig
ausgeführt werden dürfen.

\begin{itemize}
\item Kommunikation zwischen Prozessen, Protokolle zur Beschreibung
dieser Kommunikation; Schichtenmodell

\item Topologie von Rechnernetzen (Bus, Stern, Ring); Internet als
Kombination von Rechnernetzen

\item Modellierung einfacher, nebenläufiger Prozesse, z. B. mithilfe
eines Sequenzdiagramms; Möglichkeit der Verklemmung

\item kritischer Abschnitt; Monitorkonzept zur Lösung des
Synchronisationsproblems

\item Implementierung eines Beispiels für nebenläufige Prozesse

\end{itemize}

%%
%
%%

\section{Inf 12.3 Funktionsweise eines Rechners  (ca.  17 Std.)}

Am Modell der Registermaschine lernen die Schüler den grundsätzlichen
Aufbau eines Computersystems und die Analogie zwischen den bisher von
ihnen verwendeten Ablaufmodellen und Maschinenprogrammen kennen. So wird
ihnen auch bewusst, dass Möglichkeiten und Grenzen theoretischer
algorithmischer Berechnungsverfahren für die reale maschinelle
Verarbeitung von Information ebenfalls gelten.

Beispiele zeigen den Schülern, wie einfache Algorithmen auf systemnaher
Ebene durch Maschinenbefehle realisiert werden können. Dabei beschränken
sich Anzahl und Komplexität der benutzten Maschinenbefehle auf das für
die Umsetzung der gewählten Beispiele Wesentliche. Für das Verstehen des
Programmablaufs ist insbesondere die in Jahrgangsstufe 10 erlernte
Zustandsmodellierung eine große Hilfe. Zur Überprüfung ihrer
Überlegungen setzen die Schüler eine Simulationssoftware für die
Zentraleinheit ein, die die Vorgänge beim Programmablauf
veranschaulicht.

\begin{itemize}
\item Aufbau eines Computersystems: Prozessor (Rechenwerk, Steuerwerk),
Arbeitsspeicher, Ein- und Ausgabeeinheiten, Hintergrundspeicher;
Datenbus, Adressbus und Steuerbus

\item Registermaschine als Modell eines Daten verarbeitenden Systems
(Datenregister, Befehlsregister, Befehlszähler, Statusregister);
Arbeitsspeicher für Programme und Daten (von-Neumann-Architektur),
Adressierung der Speicherzellen

\item ausgewählte Transport-, Rechen- und Steuerbefehle einer
modellhaften Registermaschine; grundsätzlicher Befehlszyklus

\item Zustandsübergänge der Registermaschine als Wirkung von Befehlen

\item Umsetzung von Wiederholungen und bedingten Anweisungen auf
Maschinenebene
\end{itemize}

%%
%
%%

\section{Inf 12.4 Grenzen der Berechenbarkeit  (ca.  10 Std.)}

Nachdem die Jugendlichen bereits erkannt haben, dass sich nicht
formalisierbare Aufgabenstellungen der Bearbeitung durch eine
Rechenanlage entziehen, bauen sie nun ihre Fähigkeit zur Beurteilung der
Einsatzmöglichkeiten maschineller Informationsverarbeitung weiter aus.
Sie vertiefen dabei ihr Bewusstsein für grundlegende Einschränkungen,
denen jede auch noch so leistungsfähige Rechenanlage unterworfen ist.

Einen Einblick in die praktischen Grenzen der Berechenbarkeit gewinnen
die Schüler mithilfe von Aufwandsbetrachtungen an Aufgabenstellungen wie
der Wegesuche, die algorithmisch zwar vollständig lösbar sind, bei denen
die Ausführung des jeweiligen Algorithmus aber nicht mit vertretbarem
Zeitaufwand realisierbar ist. Den Jugendlichen wird deutlich, dass die
Sicherheit moderner Verschlüsselungsverfahren auf den praktischen
Grenzen der Berechenbarkeit beruht.

Anhand des Halteproblems lernen die Schüler schließlich auch
prinzipielle Grenzen der Berechenbarkeit kennen. Sie sehen ein, dass es
mathematisch exakt definierbare Probleme gibt, die algorithmisch und
damit auch technisch unlösbar sind.

\begin{itemize}
\item experimentelle Abschätzung des Laufzeitaufwands typischer
Algorithmen und die damit verbundenen Grenzen der praktischen
Anwendbarkeit

\item hoher Laufzeitaufwand als Schutz vor Entschlüsselung durch
systematisches Ausprobieren aller Möglichkeiten (Brute-Force-Verfahren)

\item prinzipielle Grenzen der Berechenbarkeit anhand von
Plausibilitätsbetrachtungen zum Halteproblem

\end{itemize}
\end{document}
