\documentclass{lehramt-informatik-haupt}
\usepackage{tabularx}

\begin{document}

%%%%%%%%%%%%%%%%%%%%%%%%%%%%%%%%%%%%%%%%%%%%%%%%%%%%%%%%%%%%%%%%%%%%%%%%
% Theorie-Teil
%%%%%%%%%%%%%%%%%%%%%%%%%%%%%%%%%%%%%%%%%%%%%%%%%%%%%%%%%%%%%%%%%%%%%%%%

\chapter{Anforderungsermittlung}

%-----------------------------------------------------------------------
%
%-----------------------------------------------------------------------

\section{Anforderungen und Spezifikationen\footcite[Seite 13]{sosy:fs:1}}

\subsection{Anforderungen (Requirements):}

\begin{itemize}
\item welches Problem soll gelöst werden?
\item welche Leistung soll das geplante Projekt erbringen?
\item Berücksichtigung möglichst aller beteiligten Personen (Stakeholder)
\end{itemize}

\subsection{Anforderungsanalyse (Anforderungsspezifikation):}

Kompromiss aller beteiligten Stakeholder im Hinblick auf das zu
erstellende Produkt\footcite[Seite 17-20]{schatten}

\subsection{Arten von Anforderungen:}\footcite[Seite 14]{sosy:fs:1}

\begin{description}
\item[Funktionale Anforderungen]
Systemverhalten; Funktionen des zu erstellenden Produkts

\item[Nichtfunktionale Anforderungen]
Qualitätsmerkmale wie \zB Leistungsfähigkeit

\item[Designbedingungen]
Festlegung technischer Rahmenbedingungen

\item[Prozessbedingungen]
Rahmenbedingungen für die Vorgehensweise bei der Entwicklung eines
Software-Produkts\footcite[Seite 20-22]{text}
\end{description}

%-----------------------------------------------------------------------
%
%-----------------------------------------------------------------------

\subsection{Ermitteln der Anforderungen (requirements elicitation)}\footcite[Seite
15]{sosy:fs:1}

\begin{itemize}
\item (“Herauslocken“), Problem verstehen
\item Festlegen der Systemgrenzen
\item Definieren der Schnittstellen zwischen Software-System und Umgebung
\end{itemize}

\subsection{Techniken}

\begin{itemize}
\item Brainstorming
\item Fragebogen
\item Interview
\item Simulationsmodelle
\item Anforderungsreview
\item Workshop
\end{itemize}

\literatur

\end{document}
