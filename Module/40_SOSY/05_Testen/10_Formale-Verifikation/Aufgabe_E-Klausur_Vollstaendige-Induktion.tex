\documentclass{lehramt-informatik-aufgabe}
\liLadePakete{vollstaendige-induktion}
\begin{document}
% m = markierung
\def\m#1{\textcolor{blue}{#1}}

\section{Geometrische Summenformel}

Gegeben sei folgende Methode:\footcite{sosy:e-klausur}
\index{Vollständige Induktion}

\liJavaDatei[firstline=3,lastline=11]{aufgaben/sosy/totale_korrektheit/GeoSum}

\noindent
Weisen Sie mittels vollständiger Induktion nach, dass

\begin{displaymath}
\text{geoSum}(n,q) = 1 - q^{n+1}
\end{displaymath}

\noindent
Dabei können Sie davon ausgehen, dass $q > 0$, $ n \in \mathbb{N}_0$

% Manuelle Rückmeldung:
% Beim Induktionsanfang fehlt ein f(0) oder Ähnliches (je nachdem, wie

% Also ist

% du die Funktion nennen möchtest) (-1 BE) Du musst ja zeigen, dass für
% n=0 der Programmcode und die Formel auf das gleiche Ergebnis kommen.
% Beim Induktionsschritt fehlt der "Start" mit dem Programmcode, der ist
% nur in der Behauptung vorhanden. Du musst da näher am Programmcode
% bleiben als bei mathematischen Beispielen (vgl. Aufgabe mit Türme von
% Hanoi) (- 1 BE) Sonst sehr schön!

\begin{liAntwort}
\liInduktionAnfang

\begin{displaymath}
f(0): \text{geoSum}(0, q) = 1 - q^{0+1} = 1 - q^1 = 1 - q
\end{displaymath}

\liInduktionVoraussetzung

\begin{displaymath}
f(n): \text{geoSum}(n, q) = 1 - q^{n+1}
\end{displaymath}

\liInduktionSchritt

\begin{align*}
f(n+1): \text{geoSum}(n + 1, q)
& = (1 - q)^{(n + 1) + 1} + \text{geoSum}(n, q) \\
& = (1 - q)^{n + 1 + 1} + (1 - q)^{n + 1} \\
& = 1 - q^{n + 1} + q^{n + 1} \cdot (1 - q) \\
& = 1 - q^{n + 1} + q^{n + 1} - q^{n + 2} \\
& = 1 - q^{(n + 1) + 1}
\end{align*}
\end{liAntwort}

\end{document}
