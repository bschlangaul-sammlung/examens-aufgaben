\documentclass{bschlangaul-theorie}
\liLadePakete{syntax,mathe,wpkalkuel}
\usepackage{array}
\usepackage{multicol}
\let\wp=\liWpKalkuel
\let\equivalent=\liWpEquivalent
\let\erklaerung=\liWpErklaerung

\begin{document}

%%%%%%%%%%%%%%%%%%%%%%%%%%%%%%%%%%%%%%%%%%%%%%%%%%%%%%%%%%%%%%%%%%%%%%%%
% Theorie-Teil
%%%%%%%%%%%%%%%%%%%%%%%%%%%%%%%%%%%%%%%%%%%%%%%%%%%%%%%%%%%%%%%%%%%%%%%%

\chapter{Formale Verifikation}

%-----------------------------------------------------------------------
%
%-----------------------------------------------------------------------

\section{Programmverifikation und Korrektheit\footcite[Seite 14]{sosy:fs:5}}

Verifikation dient dazu, die Korrektheit eines Programms
\memph{mathematisch} zu beweisen.\footcite[Kapitel 6.1.3.1 Seite 178]{schneider}

Ein Algorithmus ist dabei

\begin{itemize}

\item \memph{partiell korrekt}, wenn er bei gültiger Eingabe
(Vorbedingung) \memph{immer ein gültiges Ergebnis} (Nachbedingung)
liefert. Dabei ist es noch möglich, dass das Programm nicht für jede
Eingabe ein Ergebnis liefert, also nicht für jede Eingabe terminiert.

\item \memph{total korrekt}, wenn er partiell korrekt ist und
\memph{terminiert}.\footcite{wiki:korrektheit}
\end{itemize}

%-----------------------------------------------------------------------
%
%-----------------------------------------------------------------------

\section{\text{wp}-Kalkül zur Verfikation\footcite[Seite 15]{sosy:fs:5}}

Der \text{wp}-Kalkül ist ein Kalkül in der Informatik zur Verifikation
eines imperativen Programmcodes. Die Abkürzung \text{wp} steht für
\memph{weakest precondition} (auf deutsch „schwächste Vorbedingung“).
Bei der Verifikation geht es nicht darum, die Funktion mit einer
bestimmten Menge an Eingabedaten auf korrekte Ergebnisse zu testen,
sondern darum, eine \memph{allgemeingültige Aussage über das korrekte
Ablaufen} des Programms zu erhalten.

Die Überprüfung der Korrektheit geschieht durch \memph{Rückwärtsanalyse}
des Programmcodes. Ausgehend von der Nachbedingung wird geprüft, ob
diese durch die Vorbedingung und den Programmcode garantiert wird.

Alternativ kann auch der \memph{Hoare-Kalkül} benutzt werden, bei dem im
Gegensatz zum \text{wp}-Kalkül eine \memph{Vorwärtsanalyse} stattfindet.

Der \text{wp}-Kalkül hilft gewisse Zusicherungen im Programm zu machen.
Eine Zusicherung ist eine prädikatenlogische Aussage über den Inhalt der
Variablen an der bestimmten Stelle. Eine Zusicherung vor einem
Programmtext heißt \memph{Vorbedingung P}, eine Zusicherung danach
\memph{Nachbedingung Q}.\footcite{wiki:wp-kalkuel}

%-----------------------------------------------------------------------
%
%-----------------------------------------------------------------------

\section{Verzweigung}

\liWpErklaerungVerzweigung

\section{Transformationen}

\begin{enumerate}
\item $\wp{}{Q} = Q$ \\
Nichts passiert, die Vorbedingung bleibt gleich

\item $\wp{Fehler}{Q} = \text{falsch}$ \\
Fehler dürfen nicht auftreten

\item $\wp{A}{Q} \land \wp{A}{R} = \wp{A}{Q \land R}$ \\
Distributivität der Konjunktion

\item $\wp{A}{Q} \lor \wp{A}{R} = \wp{A}{Q \lor R}$ \\
Distributivität der Disjunktion\footcite{wiki:wp-kalkuel}
\end{enumerate}

%-----------------------------------------------------------------------
%
%-----------------------------------------------------------------------

\section{Schleifen}

Die Behandlung von Schleifen ist etwas schwieriger als die von anderen
Konstrukten, da die Variablen innerhalb jedes einzelnen
Schleifendurchgangs verändert werden. Daher ist es nicht einfach möglich
eine starre Ersetzung vorzunehmen. Anstattdessen verwendet man eine Art
Vollständige Induktion um die Funktion der Schleife nachzuweisen.

Um die schwächste Vorbedingung eines Ausdrucks der Form
„\liJavaCode{while(b) { A }}“ zu finden, verwendet man eine
\memph{Schleifeninvariante}. Sie ist ein Prädikat für das

\begin{displaymath}
\{ I \wedge b \} A \{ I \}
\end{displaymath}

\noindent
gilt. Die Schleifeninvariante gilt also sowohl vor, während und nach der
Schleife. Das Schema einer Schleife sieht dann wie folgt aus:

\begin{minted}{java}
// { I } - Invariante gilt vor der Schleife
while (b) {
  // { I && b} - Invariante gilt vor dem Schleifenkörper
  A;
  // { I } - Invariante gilt nach dem Schleifenkörper
}
// { I ∧ (¬b) }
\end{minted}

\noindent
Nun gilt es nur noch folgende Schritte zu beweisen:

\begin{enumerate}
\item Die Invariante gilt vor Schleifeneintritt

\item $\{ I \wedge b \} A \{ I \}$, dass also $I$ wirklich eine
Invariante ist

\item $(I \wedge \neg b) \Rightarrow Q$, dass also bei der Terminierung
auch die Nachbedingung aus der Invariante folgt. Dass die Schleife
terminiert (mittels Schleifenvariante/Terminierungsfunktion)
\end{enumerate}

\subsection{Beispiel}

Dazu ein Beispiel, das die Fakultät einer Variable $x$ ausrechnet und in
der Variable $fak$ ausgibt

\begin{minted}{java}
int berechneFakultaet(int x) {
  int i = 1;
  int fak = 1;
  // I: fak = i!
  while (i < x) {
    // I: fak = i! && i < x
    i = i + 1;
    fak = fak * i;
    // I: fak = i!
  }
  // I: fak = i! && i >= x
  return x;
}
\end{minted}

\noindent
Die Schleifeninvariante ist hier $(\text{fak} = i!)$. Der Ausdruck $(x -
i)$ fällt streng monoton während der Schleifenausführung gegen $0$ und
ist die Abbruchbedingung.

\subsubsection{Am Beispiel von $x = 3$}

\begin{multicols}{2}
\liPseudoUeberschrift{1. Durchgang}

\begin{minted}{java}
int i = 1;
int fak = 1;
// I: fak = i! -> 1 = 1
while (i < x) {
  // I: fak = i! && i < x -> 1 = 1 && 1 < 3
  i = i + 1;
  // i = 2
  fak = fak * i;
  // I: fak = i! -> 2 = 2
}
\end{minted}

\liPseudoUeberschrift{2. Durchgang}

\begin{minted}{java}
while (i < x) {
  // I: fak = i! && i < x -> 2 = 2 && 2 < 3
  i = i + 1;
  // i = 3
  fak = fak * i;
  // I: fak = i! -> 6 = 3! -> 6 = 3
}
// I: fak = i! && i >= x -> 6 = 6 && 3 >= 3
\end{minted}

\end{multicols}

%-----------------------------------------------------------------------
%
%-----------------------------------------------------------------------

\section{Terminierungsfunktion bei Wiederholungen}

Zum Beweis der Terminierung einer Schleife muss eine
Terminierungsfunktion $T$ angeben werden:

\begin{displaymath}
T \colon V \rightarrow \mathbb{N}
\end{displaymath}

\noindent
$V$ ist eine Teilmenge der Ausdrücke über die Variablenwerte der Schleife

Die Terminierungsfunktion muss folgende Eigenschaften besitzen:

\begin{itemize}
\item Ihre Werte sind natürliche Zahlen (einschließlich 0).

\item Jede Ausführung des Schleifenrumpfs verringert ihren Wert (streng
monoton fallend).

\item Die Schleifenbedingung ist false, wenn $T = 0$.
\end{itemize}

$T$ ist die obere Schranke für die noch ausstehende Anzahl von
Schleifendurchläufen.
\footcite[Seite 33]{sosy:fs:5}

%%
%
%%

\subsection{Terminierungsfunktion bei Wiederholungen (Beispiel)}

Beispiel Quadratzahl:

\begin{minted}{java}
// Q: 0 <= a
y = 0;
x = 0;
while (y != a) {
  x = x + 2 * y + 1;
  y = y + 1;
}
// R: x = a^2
\end{minted}

\noindent
Wähle die Terminierungsfunktion:

\begin{displaymath}
T(y) = a - y
\end{displaymath}

\noindent
$T(y)$ wird in jedem Durchgang verringert, da $y$ erhöht wird und $a$
konstant bleibt. Wenn $T(y) = 0$, folgt $y = a$, \dh die
Schleifenbedingung $y \neq a$ ist falsch. Der Schleifenrumpf enthält
keine Schleifen, Rekursionen, Gotos …. Die Wiederholung terminiert.
\footcite[Seite 34]{sosy:fs:5}

\section{Terminierungsfunktion bei Rekursion}

Beim Beweis der Terminierung von Rekursionen kann genauso wie bei
Wiederholungen vorgegangen werden. Auch hier wird eine
Terminierungsfunktion $T$ konstruiert, die mit zunehmender
Rekursionstiefe kleiner wird.

Es muss gelten:
\begin{itemize}
\item Die Werte von $T$ sind natürliche Zahlen (inkl. $0$).

\item Bei jedem Aufruf der Methode (rechte Seite der
Rekursionsgleichung) wird der Wert von $T$ echt kleiner.

\item Abbruch bei (spätestens) $T = 0$ erzwungen.
\footcite[Seite 36]{sosy:fs:5}
\end{itemize}

%%
%
%%

\subsection{Terminierungsfunktion bei Rekursion (Beispiel)}

Beispiel Fibonacci-Zahlen:

\begin{minted}{java}
int fig(int n) {
 if (n < 3) return 1;
 else return (fib(n - 1) + fib(n - 2));
}
\end{minted}
Wähle die Terminierungsfunktion

\begin{displaymath}
T(n) = n:
\end{displaymath}

$T(n)$ wird in jedem Rekursionsschritt verringert. Für $n < 3$ wird in
den Basisfall gesprungen und die Rekursion beendet. Die Rekursion
terminiert.

\literatur

\end{document}
