\documentclass{bschlangaul-haupt}
\usepackage{tabularx}

\begin{document}

%%%%%%%%%%%%%%%%%%%%%%%%%%%%%%%%%%%%%%%%%%%%%%%%%%%%%%%%%%%%%%%%%%%%%%%%
% Theorie-Teil
%%%%%%%%%%%%%%%%%%%%%%%%%%%%%%%%%%%%%%%%%%%%%%%%%%%%%%%%%%%%%%%%%%%%%%%%

\chapter{Entwurfsmuster / Design pattern}

\begin{liQuellen}
\item \cite{wiki:entwurfsmuster}
\item \cite[Seite 39-43]{sosy:fs:3}
\end{liQuellen}

\section{Verwendete Überschriften}

\begin{enumerate}
\item Zweck
\item Szenario
\item UML-Diagramm
\item Akteure
\item Allgemeines Code-Beispiel
\end{enumerate}

Wiederkehrende, geprüfte, bewährte Lösungsschablonen für typische Probleme\footcite[Seite 39]{sosy:fs:3}

Wiederverwendung

Kein Garant für gutes Design – können dieses aber gut unterstützen

\section{Vorteile}

\begin{description}
\item[Wiederkehrende Problemstellungen]

In der Praxis der Software-Entwicklung hat sich gezeigt, dass bestimmte
Problemstellungen immer wieder, wenn auch in leicht veränderter Form,
vorkommen.\footcite[Seite 231]{schatten} In den letzten Jahrzehnten der
Software-Entwicklung wurden daher solche immer wiederkehrenden Probleme
analysiert, aus Fehlern gelernt und abstrakte Lösungsmuster entwickelt
und beschrieben. Diese Muster (Pattern) kann man dann für eigene
Probleme verwenden bzw. an diese anpassen. Der Programmierer einer
Anwendung muss deshalb „das Rad nicht neu erfinden“.\footcite[Seite
232]{schatten}

\item[Verbesserung der Kommunikation]

Design-Patterns werden heute als Grundwissen jedes Software-Entwicklers
angesehen und dienen daher in der Software-Entwicklung auch als
„Sprachmittel“; d. h., die Kommunikation im Team über architektonische
Konzepte wird einfacher.\footcite[Seite 232]{schatten}

\item[Bezug zur dort vorhandenen Diskussion des Problemkontextes]

Wenn der Einsatz von Entwurfsmustern dokumentiert wird, ergibt sich ein
weiterer Nutzen dadurch, dass durch die Beschreibung des Musters ein
Bezug zur dort vorhandenen Diskussion des Problemkontextes und der Vor-
und Nachteile der Lösung hergestellt wird.
\footcite{wiki:entwurfsmuster}
\end{description}

\begin{itemize}
\item Erzeugungsmuster (Creational Patterns)

\begin{itemize}
\item Einzelstück (Singleton)
\item Abstrakte Fabrik (Abstract Factory)
\end{itemize}

\item Strukturmuster (Structural Patterns)

\begin{itemize}
\item Adapter
\item Dekorierer (Decorator)
\item Kompositum (Composite)
\item Stellvertreter (Proxy)
\end{itemize}

\item Verhaltensmuster (Behavioral Patterns)

\begin{itemize}
\item Beobachter (Observer)
\item Wiederholer (Iterator)
\item Schablone (Template)
\item Zustand (State)
\end{itemize}

\item Sonstige

\begin{itemize}
\item Modell-Präsentation-Steuerung (Model-View-Controller)
\end{itemize}
\end{itemize}

\footcite[Seite 39]{sosy:fs:3}

\literatur

\end{document}
