\documentclass{bschlangaul-theorie}
\bLadePakete{syntax,entwurfsmuster}

\begin{document}

%%%%%%%%%%%%%%%%%%%%%%%%%%%%%%%%%%%%%%%%%%%%%%%%%%%%%%%%%%%%%%%%%%%%%%%%
% Theorie-Teil
%%%%%%%%%%%%%%%%%%%%%%%%%%%%%%%%%%%%%%%%%%%%%%%%%%%%%%%%%%%%%%%%%%%%%%%%

\chapter{Beobachter (Observer)}

\begin{bQuellen}
\item \cite{wiki:beobachter}
\item \url{https://www.philipphauer.de/study/se/design-pattern/observer.php}
\item \cite[Seite 249-257]{gof}
\item \cite[Kapitel 8.5.1, Seite 269-274]{schatten}
\item \cite[Kapitel 4.7, Seite 70-75]{eilebrecht}
\item \cite[Seite 269]{siebler}
\end{bQuellen}

%-----------------------------------------------------------------------
%
%-----------------------------------------------------------------------

\section{Zweck}

Das Observer-Muster ermöglicht einem oder mehreren Objekten, automatisch
auf die \bEmph{Zustandsänderung} eines bestimmten Objekts \bEmph{zu
reagieren}, um den eigenen Zustand anzupassen.
\footcite[Seite 70]{eilebrecht}

%-----------------------------------------------------------------------
%
%-----------------------------------------------------------------------

\section{Szenario}

Zusätzlich zur historischen Darstellung von Verkaufszahlen soll Ihre
Software um eine Prognose-Ansicht erweitert werden. Diese soll zum einen
jede Viertelstunde die aktuellen Zahlen einbeziehen und zum anderen eine
durch den Benutzer gewählte Vorhersagestrategie. Im Gegensatz zur
Anzeige des historischen Verlaufs kann sich folglich die Prognose
kurzfristig ändern. Genauer gesagt, sobald neue Verkaufszahlen vorliegen
oder der Anwender die Strategie ändert. Sie möchten erreichen, dass sich
die Anzeige der Prognose automatisch anpasst, sobald sich im Hintergrund
Änderungen ergeben.
\footcite[Seite 70]{eilebrecht}

\bEntwurfsBeobachter

\literatur

\end{document}
