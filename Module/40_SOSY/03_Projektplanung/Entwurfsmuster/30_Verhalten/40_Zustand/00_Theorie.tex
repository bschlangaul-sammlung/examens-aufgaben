\documentclass{bschlangaul-theorie}
\bLadePakete{syntax,uml,entwurfsmuster}

\begin{document}

%%%%%%%%%%%%%%%%%%%%%%%%%%%%%%%%%%%%%%%%%%%%%%%%%%%%%%%%%%%%%%%%%%%%%%%%
% Theorie-Teil
%%%%%%%%%%%%%%%%%%%%%%%%%%%%%%%%%%%%%%%%%%%%%%%%%%%%%%%%%%%%%%%%%%%%%%%%

\chapter{Zustand, Objekte für Zustände / (State, Objects for states)}

\begin{bQuellen}
\item \cite{wiki:zustand}
\item \url{https://www.philipphauer.de/study/se/design-pattern/state.php}
\item \cite[PDF Seite 258-265]{gof}
% \item \cite{schatten}
% \item \cite{eilebrecht}
\item \cite[Seite 69-81]{siebler}
\end{bQuellen}

\section{Zweck}

Das Zustandsmuster wird zur Kapselung unterschiedlicher,
zustandsabhängiger Verhaltensweisen eines Objektes eingesetzt.
\footcite{wiki:zustand}

\bEntwurfsZustand

\literatur

\end{document}
