\documentclass{bschlangaul-haupt}
\liLadePakete{syntax,uml}

\begin{document}

%%%%%%%%%%%%%%%%%%%%%%%%%%%%%%%%%%%%%%%%%%%%%%%%%%%%%%%%%%%%%%%%%%%%%%%%
% Theorie-Teil
%%%%%%%%%%%%%%%%%%%%%%%%%%%%%%%%%%%%%%%%%%%%%%%%%%%%%%%%%%%%%%%%%%%%%%%%

\chapter{Wiederholer (Iterator)}

\begin{liQuellen}
\item \cite{wiki:iterator}
%\item \url{}
\item \cite[PDF Seite 219-231]{gof}
% \item \cite{schatten}
\item \cite[Seite 56-59]{eilebrecht}
\item \cite[Seite 106-116]{siebler}
\end{liQuellen}

%-----------------------------------------------------------------------
%
%-----------------------------------------------------------------------

\section{Zweck}

Das Iterator-Muster erlaubt den \memph{sequenziellen Zugriff} eines
Clients auf die \memph{Elemente einer Aggregation}, \memph{ohne deren
internen Aufbau zu kennen}.
\footcite[Seite 56]{eilebrecht}

%-----------------------------------------------------------------------
%
%-----------------------------------------------------------------------

\section{Szenario}

Innerhalb Ihrer Software wird eine Kundenliste verwaltet. Verschiedene
Anwendungsfälle erfordern die sequenzielle Abarbeitung dieser Liste.
Dabei existieren unterschiedliche Vorgaben bezüglich der Reihenfolge. Es
ist Ihre Aufgabe festzulegen, an welcher Stelle die verschiedenen
Traversierungsvarianten zu implementieren sind.
\footcite[Seite 56]{eilebrecht}

%-----------------------------------------------------------------------
%
%-----------------------------------------------------------------------

\section{UML-Diagramm}

\begin{tikzpicture}
\umlclass[x=0,y=3,type=interface]{Aggregat}{}{
  + erzeugeIterator()\\
  + suche()
}

\umlsimpleclass[x=0,y=0]{KonkretesAggregat}

\umlclass[x=8,y=3,type=interface]{Iterator}{
  + weiter()\\
  + zurueck()\\
  + hatNächstes()\\
  + gibAktuellesElement
}{}

\umlsimpleclass[x=8,y=0]{KonkreterIterator}

\umlsimpleclass[x=4,y=5]{Klient}

\umlsimpleclass[x=0,y=-1.5]{Element}

\umlinherit{KonkretesAggregat}{Aggregat}
\umlinherit{KonkreterIterator}{Iterator}

\umlVHuniassoc{KonkreterIterator}{Element}
\umluniassoc[arg=*]{KonkretesAggregat}{Element}
\umluniassoc{KonkretesAggregat}{KonkreterIterator}
\umluniassoc{KonkreterIterator}{KonkretesAggregat}

\umluniassoc{Klient}{Aggregat}
\umluniassoc{Klient}{Iterator}

\end{tikzpicture}
\footcite{wiki:iterator}

%-----------------------------------------------------------------------
%
%-----------------------------------------------------------------------

\section{Akteure}

\begin{description}
\item[Iterator] definiert die Schnittstelle zum Zugriff auf die Elemente
und zum Traversieren des Aggregates.

\item[Konkreter Iterator] implementiert diese Schnittstelle und
speichert die Position im Aggregat.

\item[Aggregat] definiert die Schnittstelle zum Erzeugen eines
Iterators. Oft enthält die Schnittstelle auch Methoden zum Erzeugen von
Iteratoren die auf spezielle Elemente zeigen, wie \zB erstes, letztes
oder ein Element mit bestimmten Eigenschaften.

\item[Konkretes Aggregat] implementiert diese Schnittstelle.
\footcite{wiki:iterator}
\end{description}

\literatur

\end{document}
