\documentclass{lehramt-informatik-haupt}
\liLadePakete{syntax,entwurfsmuster}
\begin{document}

%%%%%%%%%%%%%%%%%%%%%%%%%%%%%%%%%%%%%%%%%%%%%%%%%%%%%%%%%%%%%%%%%%%%%%%%
% Theorie-Teil
%%%%%%%%%%%%%%%%%%%%%%%%%%%%%%%%%%%%%%%%%%%%%%%%%%%%%%%%%%%%%%%%%%%%%%%%

\chapter{Einzelstück (Singleton)}

\begin{liQuellen}
\item \cite{wiki:singleton}
\item \url{https://www.philipphauer.de/study/se/design-pattern/singleton.php}
\item \cite[Seite 109-115]{gof}
\item \cite[Kapitel 8.3.1, Seite 247-249]{schatten}
\item \cite[Kapitel 3.4 Seite 38-43]{eilebrecht}
\item \cite[Kapitel 1, Seite 1-17]{siebler}
\end{liQuellen}

\section{Zweck}

Stellt sicher, dass nur \memph{genau eine Instanz einer Klasse} erzeugt
wird.
\footcite[Seite 38]{eilebrecht}

%-----------------------------------------------------------------------
%
%-----------------------------------------------------------------------

\section{Klassendiagramm}

\liEntwurfsEinzelstueckUml

%-----------------------------------------------------------------------
%
%-----------------------------------------------------------------------

\section{Teilnehmer}

\liEntwurfsEinzelstueckAkteure

%-----------------------------------------------------------------------
%
%-----------------------------------------------------------------------

\section{Allgemeines Code-Beispiel}

\liEntwurfsEinzelstueckCode

\literatur

\end{document}
