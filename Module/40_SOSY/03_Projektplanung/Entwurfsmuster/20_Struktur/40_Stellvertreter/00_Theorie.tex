\documentclass{bschlangaul-haupt}
\liLadePakete{entwurfsmuster}

\begin{document}

%%%%%%%%%%%%%%%%%%%%%%%%%%%%%%%%%%%%%%%%%%%%%%%%%%%%%%%%%%%%%%%%%%%%%%%%
% Theorie-Teil
%%%%%%%%%%%%%%%%%%%%%%%%%%%%%%%%%%%%%%%%%%%%%%%%%%%%%%%%%%%%%%%%%%%%%%%%

\chapter{Stellvertreter (Proxy)}

\begin{liQuellen}
\item \cite{wiki:stellvertreter}
% \item \url{https://www.philipphauer.de/study/se/design-pattern/}
\item \cite[Seite 176-185]{gof}
\item \cite[Kapitel 8.4.4, Seite 256-262]{schatten}
\item \cite[Kapitel 5.5, Seite 89-92]{eilebrecht}
\item \cite[Kapitel 21, Seite 241-268]{siebler}
\end{liQuellen}

%-----------------------------------------------------------------------
%
%-----------------------------------------------------------------------

\section{Zweck}

Ein Proxy stellt einen \memph{Platzhalter} für eine andere Komponente
(Objekt) dar und \memph{kontrolliert den Zugang zum echten Objekt}.
\footcite[Seite 89]{eilebrecht}

\liEntwurfsStellvertreter

\literatur

\end{document}
