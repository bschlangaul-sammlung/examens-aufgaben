\documentclass{bschlangaul-theorie}
\bLadePakete{syntax,entwurfsmuster}

\begin{document}

%%%%%%%%%%%%%%%%%%%%%%%%%%%%%%%%%%%%%%%%%%%%%%%%%%%%%%%%%%%%%%%%%%%%%%%%
% Theorie-Teil
%%%%%%%%%%%%%%%%%%%%%%%%%%%%%%%%%%%%%%%%%%%%%%%%%%%%%%%%%%%%%%%%%%%%%%%%

\chapter{Adapter}

\begin{liQuellen}
\item \cite{wiki:adapter}
% \item \url{https://www.philipphauer.de/study/se/design-pattern/}
\item \cite[Seite 120-129]{gof}
\item \cite[Kapitel 8., Seite 255]{schatten}
\item \cite[Kapitel 5.1, Seite 77-79]{eilebrecht}
\item \cite[Kapitel 20, Seite 243]{siebler}
\end{liQuellen}

\section{Zweck}

Ein Adapter passt die Schnittstelle einer Klasse an eine andere von
ihren Klienten erwartete Schnittstelle an. Das Adaptermuster lässt
\memph{Klassen zusammenarbeiten}, die andernfalls dazu nicht in der Lage
wären.
\footcite[Seite 77]{eilebrecht}

%-----------------------------------------------------------------------
%
%-----------------------------------------------------------------------

\section{Klassendiagramm}

\bEntwurfsAdapterUml

%-----------------------------------------------------------------------
%
%-----------------------------------------------------------------------

\section{Akteure}

\bEntwurfsAdapterAkteure

%-----------------------------------------------------------------------
%
%-----------------------------------------------------------------------

\section{Allgemeines Code-Beispiel}

\bEntwurfsAdapterCode

\literatur

\end{document}
