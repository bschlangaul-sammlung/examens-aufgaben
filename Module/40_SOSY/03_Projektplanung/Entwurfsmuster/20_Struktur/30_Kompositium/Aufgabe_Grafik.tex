\documentclass{bschlangaul-aufgabe}

\begin{document}
\bAufgabenTitel{Grafik}

\section{Aufgabe 3
\index{Kompositum (Composite)}
\footcite{sosy:ab:6}}

Stellen Sie sich vor, Sie brauchen ein Grafiksystem. In diesem System
wollen Sie aus Linien, Rechtecken, und Text größere Abbildungen
darstellen. Diese Abbildungen sollen auch wieder andere Abbildungen
rekursiv enthalten können. Sie brauchen also primitive Objekte: die
Linie, das Rechteck und den Text. Zusätzlich brauchen Sie Behälter, die
weitere Behälter und primitive Objekte enthalten können.

Erklären Sie, mit welchem Entwurfsmuster man diese Struktur abbilden
kann und zeichnen Sie das dazugehörige Klassendiagramm.
\end{document}
