\documentclass{bschlangaul-theorie}
\liLadePakete{syntax,entwurfsmuster}
\begin{document}

%%%%%%%%%%%%%%%%%%%%%%%%%%%%%%%%%%%%%%%%%%%%%%%%%%%%%%%%%%%%%%%%%%%%%%%%
% Theorie-Teil
%%%%%%%%%%%%%%%%%%%%%%%%%%%%%%%%%%%%%%%%%%%%%%%%%%%%%%%%%%%%%%%%%%%%%%%%

\chapter{Dekorierer (Decorator)}

\begin{liQuellen}
\item \cite{wiki:dekorierer}
\item \cite[Seite 149-156]{gof}
\item \url{https://www.philipphauer.de/study/se/design-pattern/decorator.php}
\item \cite[Kapitel 8.5.2, Seite 274-278]{schatten}
\item \cite[Seite 274]{schatten}
\item \cite[Kapitel 5.3, Seite 83-86]{eilebrecht}
\item \cite[Kapitel 22, Seite 269]{siebler}
\end{liQuellen}

%-----------------------------------------------------------------------
%
%-----------------------------------------------------------------------

\section{Zweck}

Ein Decorator fügt einer Komponente dynamisch neue Funktionalität
hinzu, ohne die Komponente selbst zu ändern.
\footcite[Seite 83]{eilebrecht}

\liEntwurfsDekoriererUml

\section{Szenario}

Stellen Sie sich vor, Sie betreiben eine Espressobar. Dort bieten Sie
typischerweise neben reinem Espresso auch diverse Kaffeevarianten an:
echten Kaffee oder ohne Koffein, mit heißer (Latte Macciato) oder
geschäumter Milch (Cappuccino), mit Sahne, mit einer Kugel Eis, mit
Likör oder doppelt (Doppio). Wenn Sie diese Kombinationen alle explizit
modellieren, erhalten Sie eine unübersichtlich große Zahl von Klassen.
Erweiterungen des Angebotes (etwa mit Ahorn- oder Nussaroma) lassen die
Anzahl Klassen drastisch weiter wachsen.
\footcite[Seite 83]{eilebrecht}

%-----------------------------------------------------------------------
%
%-----------------------------------------------------------------------

\section{Allgemeines Code-Beispiel}

\liEntwurfsDekoriererCode

\literatur

\end{document}
