\documentclass{bschlangaul-theorie}
\bLadePakete{syntax,mathe,petri,spalten}

\usepackage{pifont}

\begin{document}

\def\TmpBeschriftung#1{%
  {%
    \footnotesize%
    \itshape#1%
  }%
}

%%%%%%%%%%%%%%%%%%%%%%%%%%%%%%%%%%%%%%%%%%%%%%%%%%%%%%%%%%%%%%%%%%%%%%%%
% Theorie-Teil
%%%%%%%%%%%%%%%%%%%%%%%%%%%%%%%%%%%%%%%%%%%%%%%%%%%%%%%%%%%%%%%%%%%%%%%%

\chapter{Petri-Netze}

\begin{bQuellen}
\item \cite[Seite 238-242]{hoffmann}
\end{bQuellen}

\section{Simulatoren}

\begin{itemize}
\item \url{http://petri.hp102.ru/pnet.html} (online)
\item \url{https://apo.adrian-jagusch.de/} (online)
\item \url{https://www.oris-tool.org/} (Funktioniert mit Java 11, bei neueren kann kein Text eingegeben werden)
\item \url{https://sourceforge.net/projects/pipe2/files/PIPEv4/PIPEv4.3.0/PIPEv4.3.0.zip/download}

\url{https://github.com/sarahtattersall/PIPE}

(Läuft nicht mit modernen Java Versionen. Funktioniniert mit Java 1.8
/usr/lib/jvm/java-8-openjdk-amd64/jre/bin/java eingeschränkt: Linien
können nicht richtig gezogen werden.)
\end{itemize}

\noindent
Der Name Petri-Netz geht auf den deutschen Mathematiker \bEmph{Carl Adam
Petri} zurück, der diesen Automatentypus Anfang der Sechziger- jahre
ausführlich untersuchte.\footcite[Seite 238]{hoffmann}

Petri-Netze stellen einen Formalismus zur Beschreibung
\bEmph{nebenläufiger Ereignisse} dar.
%
Sie können somit u.\,a. \bEmph{zur Planung eines Softwareprojektes}
genutzt werden.\footcite{sosy:fs:3}

Petri-Netze unterscheiden zwischen \bEmph{Bedingungen} und
\bEmph{Ereignissen}. Erstere werden durch \bEmph{Stellen}, letztere
durch \bEmph{Transitionen} beschrieben.\footcite[Seite 238]{hoffmann}

\begin{center}
\begin{tabular}{l|l}
Bedingungen & Stellen \\
Ereignisse & Trasitionen \\
\end{tabular}\footcite[Seite 238]{hoffmann}
\end{center}

\begin{description}

%%
%
%%

\item[Stellen/Platz (place)] Eine Stelle wird mit einem Kreis
%
(\tikz
\node[place,minimum size=7pt]{};)
%
dargestellt.

\begin{description}
\item[Kapazität]
Jedem Platz wird als „Schranke“ eine natürliche Zahl zugeordnet. Wenn
beim Eintritt einer Transition mehr als in der Kapazitäts-Schranke
angegeben Marken auf dem Platz entstehen würden, ist die Transition
nicht aktiviert.
\footcite{wiki:petri-netz}

\bigskip
\centerline{\fbox{\tikz \node[place,label=1,label=south:\TmpBeschriftung{Kapazität 1}]{};
\tikz \node[place,label=3,label=south:\TmpBeschriftung{Kapazität 3}]{};
\tikz \node[place,label=5,label=south:\TmpBeschriftung{Kapazität 5}]{};}}

\item[Belegung] Belegung nennt man die Anzahl der Tokens auf einem
Platz. Standardmäßig sind Plätze nicht belegt, \dh sie haben keine
Tokens und damit die Belegung 0. Anstatt den kleinen Kreisen kann auch
eine Zahl in die Plätzen eintragen werden. Die Zahl steht dann für die
Anzahl der Tokens.

\bigskip
\centerline{\fbox{
  \tikz \node[place,label=south:\TmpBeschriftung{Belegung 0}]{};
  \tikz \node[place,tokens=1,label=south:\TmpBeschriftung{Belegung 1}]{};
  \tikz \node[place,tokens=3,label=south:\TmpBeschriftung{Belegung 3}]{};
  \tikz \node[place,tokens=5,label=south:\TmpBeschriftung{Belegung 5}]{};
  \tikz \node[place,label=south:\TmpBeschriftung{Belegung 7}]{7};
}}
\end{description}

%%
%
%%

\item[Transition/Schaltstelle (transition)]
Eine Transition wird mit einem Viereck
%
(\tikz[scale=0.5,transform shape] \node[transition]{};)
%
dargestellt. Anstelle des Rechtecks sind auch Striche
%
(\tikz
\node[rectangle,fill=black,minimum width=0.5mm,minimum height=2mm,inner
sep=0pt]{};)
%
gebräuchlich. Eine Transition kann nur schalten, wenn sich
im Eingang ein Token gefindet.

Eine Transition kann feuern (schalten), wenn jede Eingangsstelle mit
mindestens so vielen Token belegt ist, wie das Gewicht des zugehörigen
Pfeils angibt und wenn in jeder Ausgangsstelle die Zahl der vorhandenen
Marken plus dem Gewicht des zugehörigen Pfeils die Kapazität nicht
übersteigt.
\footcite[Seite 9]{sosy:fs:3}

%%
%
%%

\item[Kante/Pfeil (arc)]

Die Knoten sind durch gerichtete Kanten
%
(\tikz \draw[->] (0,0) -- (0.5,0);)
%
verbunden, wobei eine Kante genau eine Stelle mit einer Transition oder
umgekehrt verbindet. Die Kanten dürfen jeweils nur von Stellen zu
Transitionen oder von Transitionen zu Stellen führen.
\footnote{\url{https://kroening-online.de/Method/Petrinetz/m_petri.php}}

\textbf{erlaubt:}

\begin{center}
\begin{tikzpicture}[li petri]
\node[place](p) {};
\node[transition] at (2,0) {} edge[pre] (p);
\node at (1,0.25) {\ding{51}};
\end{tikzpicture}
%
\hspace{1cm}
%
\begin{tikzpicture}[li petri]
\node[place] at (2,0) (p) {};
\node[transition] at (0,0) {} edge[post] (p);
\node at (1,0.25) {\ding{51}};
\end{tikzpicture}
\end{center}

\textbf{nicht erlaubt:}

\begin{center}
\raisebox{5pt}{\begin{tikzpicture}[li petri]
\node[transition] at (0,0) (t1) {};
\node[transition] at (2,0) (t2) {};
\draw[->] (t1) -- (t2);
\node at (1,0) {\ding{55}};
\end{tikzpicture}}
%
\hspace{1cm}
%
\begin{tikzpicture}[li petri]
\node[place] at (0,0) (p1) {};
\node[place] at (2,0) (p2) {};
\draw[->] (p1) -- (p2);
\node at (1,0) {\ding{55}};
\end{tikzpicture}
\end{center}

\begin{description}
\item[Gewichtung]
Durch eine \emph{Gewichtung} der Kanten kann die \emph{Anzahl der
Marken} angegeben werden, die entnommen bzw. erzeugt werden.
\footnote{\url{http://ddi.cs.uni-potsdam.de/HyFISCH/Produzieren/SeminarDidaktik/Nebenlaeufigkeit/marken.htm}}
Die Gewichtung wird durch Ganzezahlen in der Mitte des Pfeils notiert.

\begin{center}
\def\TmpGewichtung#1{\tikz \draw[->] (0,0) -- (1.5,0) node[pos=0.5,auto]{\scriptsize#1};}
\TmpGewichtung{}
\TmpGewichtung{3}
\TmpGewichtung{5}
\end{center}
\end{description}

\item[Marke/Markierung (token)] Die Kennzeichnung der Belegung einer
Stelle wird durch einen kleinen ausgefüllten Kreis gezeichnet (\tikz
\node[token]{};).

\item[„Inhibitor“-Kante (inhibitor arc)]
Anstelle der Pfeilspitze wird bei der Inhibitor-Kante an Kreis gezeichnet
(\tikz \draw[-{Circle[open,length=2mm,fill=white]}] (0,0) -- (1,0);).
Einen Platz auf Abwesenheit von
Marken testen: Ein Platz $p$ ist mit einer Transition $t$ durch eine
„Inhibitor“-Kante verbunden. $t$ ist nicht aktiviert, wenn $p$
eine oder mehrere Marken trägt.
\footcite{wiki:petri-netz}

\def\TmpInhibitor#1{
  \begin{tikzpicture}[li petri,x=1.5cm,y=0.8cm]
    \node[place,tokens=1,label=$p_1$] at (0,2) (A) {};
    \node[place,tokens=#1,label=$p_2$] at (0,0) (B) {};
    \node[place,label=$p_3$] at (2,1) (C) {};
    \node[transition] at (1,1) (t) {$t$}
      edge[pre] (A)
      edge[post] (C)
      edge[inhibitor] (B)
    ;

  \end{tikzpicture}
}

\begin{multicols}{2}
$t$ ist nicht aktiviert:\\
\TmpInhibitor{1}

$t$ ist aktiviert:\\
\TmpInhibitor{0}
\end{multicols}

\end{description}

%-----------------------------------------------------------------------
%
%-----------------------------------------------------------------------

\section{Darstellungsmatrix}

Ein Petri-Netz kann mit einer Matrix dargestellt werden.
Vorgängerstellen werden mit negativem Vorzeichen in der Spalte der
Transition eingetragen. Die Nachfolger mit positivem. Ein  Vektor gibt
die Anfangsmarkierung an.
\footnote{\url{https://www.hs-augsburg.de/informatik/projekte/mebib/emiel/entw_inf/lernprogramme/petrinetze/NetzAnalyse1HTML.html}}

%-----------------------------------------------------------------------
%
%-----------------------------------------------------------------------

\section{Begriffe\footcite[Seite 10]{sosy:fs:3}}

\begin{description}
\item[Verklemmung]
Ein Petri-Netz heißt \emph{verklemmungsfrei}, wenn keine Verklemmung
erreichbar ist. Eine Verklemmung (= Deadlock) ist ein Zustand, in dem
keine einzige Transition schalten kann.

\begin{multicols}{2}
\begin{tikzpicture}[li petri]
\node[place] at (0,0) (p1)  {};
\node[place,tokens=1] at (2,0) (p2) {};
\node[transition] at (1,0) {} edge[pre] (p1) edge[post] (p2);
\end{tikzpicture}

\begin{tikzpicture}[li petri]
\node[place] at (0,0) (p1)  {};
\node[transition] at (1,0) {} edge[pre] (p1);
\end{tikzpicture}
\footnote{\url{https://www.is.inf.uni-due.de/courses/mod_ws14/folien/petri-print.pdf}}
\end{multicols}

\item[Lebendigkeit (Transition)]
Eine Transition heißt \emph{lebendig}, wenn es für jeden möglichen
Zustand endlich viele Schaltvorgänge gibt, so dass diese Transition
schalten kann.

\item[Lebendigkeit (Petri-Netz)]
Ein Petri-Netz heißt \emph{lebendig}, wenn alle Transitionen lebendig
sind. Insbesondere ist jedes lebendige Petri-Netz verklemmungsfrei.

\begin{tikzpicture}[li petri]
\node[place,tokens=1] at (0,0) (p1)  {};
\node[place] at (2,0) (p2) {};
\node[transition] at (1,1) {} edge[pre] (p2) edge[post] (p1);
\node[transition] at (1,-1) {} edge[pre] (p1) edge[post] (p2);
\end{tikzpicture}
\footnote{\url{https://www.is.inf.uni-due.de/courses/mod_ws14/folien/petri-print.pdf}}

\begin{description}
\item[Starke Lebendigkeit]
Starke Lebendigkeit bedeutet, dass es \emph{überall} im System immer
wieder zu Schaltungen kommen kann.

\item[Schwache Lebendigkeit]
Bei schwacher Lebendigkeit kann es immer wieder \emph{irgendwo} zu
Schaltungen kommen.
\footnote{\url{http://wwwlgis.informatik.uni-kl.de/cms/fileadmin/courses/ss2007/Informationssysteme/Kapitel.09.Petrinetze.full.pdf}}

\begin{tikzpicture}[li petri]
\node[place,tokens=1] at (0,1) (p1)  {};

\node[place] at (2,1) (p2) {};

\node[transition] at (0,0) {} edge[pre,bend left=70] (p1) edge[post,bend right=70] (p1);

\node[transition] at (1,1) {} edge[pre] (p1) edge[post] (p2);

\node[transition] at (2,0) {} edge[pre,bend left=70] (p2) edge[post,bend right=70] (p2);
\end{tikzpicture}
\footnote{\url{http://www.ti.inf.uni-due.de/fileadmin/public/teaching/mod/slides/ws201112/petri-2x2.pdf}}

\end{description}

\item[Beschränktheit]
Ein Petri-Netz heißt \emph{beschränkt}, wenn es eine Konstante $1$ gibt,
so dass jede Stelle zu jedem möglichen Zustand höchstens $M$ Marken
enthält.

\item[Umkehrbarkeit]
Ein Petri-Netz heißt \emph{umkehrbar}, wenn die Anfangsmarkierung von
jeder erreichbaren Markierung erreichbar ist. Das Petri-Netz ist nicht
umkehrbar, wenn es verklemmt ist.
\end{description}

\section{Beispiele für Schaltungen}

Die Zustandsänderung erfolgt durch das Schalten der Transition, aber nur
wenn \emph{alle} Stellen des Vorbereichs  markiert sind. Dann wird in
\emph{jeder} Stelle des Nachbereichs eine Marke erzeugt.
\footnote{\url{http://ddi.cs.uni-potsdam.de/HyFISCH/Produzieren/SeminarDidaktik/Nebenlaeufigkeit/marken.htm}}

\def\TmpSchaltungEins#1#2#3#4#5{
  \begin{tikzpicture}[li petri]
  \node[place,tokens=#1] at (0,0.5) (p1) {};
  \node[place,tokens=#2] at (0,-0.5) (p2) {};

  \node[place,tokens=#3] at (4,1) (p3) {};
  \node[place,tokens=#4] at (4,0) (p4) {};
  \node[place,tokens=#5] at (4,-1) (p5) {};

  \node[transition] at (2,0) {}
    edge[pre] (p1)
    edge[post] (p3)
    edge[pre] (p2)
    edge[post] (p4)
    edge[post] (p5)
  ;
  \end{tikzpicture}
}

\begin{multicols}{2}
vorher:

\TmpSchaltungEins{1}{2}{1}{0}{0}

nachher:

\TmpSchaltungEins{0}{1}{2}{1}{1}
\end{multicols}

%%
%
%%

\noindent
Eine Transition kann hier nur Schalten, wenn die Anzahl der Marken
mindestens der Gewichtung im Vorbereich entspricht bzw. die
Gewichtung im Nachbereich nicht überschritten wird.
\footnote{\url{http://ddi.cs.uni-potsdam.de/HyFISCH/Produzieren/SeminarDidaktik/Nebenlaeufigkeit/marken.htm}}

\def\TmpSchaltungZwei#1#2#3#4{
  \medskip
  \begin{tikzpicture}[li petri]
    \node[place,tokens=#1] at (0,2) (p1) {};
    \node[place,tokens=#2] at (0,0) (p2) {};

    \node[place,tokens=#3] at (4,2) (p3) {};
    \node[place,tokens=#4] at (4,0) (p4) {};

    \node[transition] at (2,1) {}
      edge[pre] (p1)
      edge[pre] node[auto] {2} (p2)

      edge[post] node[auto] {3} (p3)
      edge[post] (p4)
    ;
  \end{tikzpicture}
}

\begin{multicols}{2}
vorher:

\TmpSchaltungZwei{2}{2}{0}{0}

\columnbreak

nachher:

\TmpSchaltungZwei{1}{0}{3}{1}
\end{multicols}

%%
%
%%

\def\TmpSchaltungDrei#1#2#3#4{
  \medskip
  \begin{tikzpicture}[li petri]
    \node[place,tokens=#1,label=$s_1$] at (0,2) (p1) {};
    \node[place,tokens=#2,label=$s_2$] at (0,0) (p2) {};

    \node[place,tokens=#3,label=$s_3$] at (4,2) (p3) {};
    \node[place,tokens=#4,label=$s_4$] at (4,0) (p4) {};

    \node[transition] at (2,1) {t}
      edge[pre] node[auto] {2} (p1)
      edge[pre] node[auto] {1} (p2)

      edge[post] node[auto] {1} (p3)
      edge[post] node[auto] {3} (p4)
    ;
  \end{tikzpicture}
}

\begin{multicols}{2}
vorher:\footcite[Seite 9 (von Wikipedia)]{sosy:fs:3}

\TmpSchaltungDrei{3}{1}{1}{0}

\columnbreak

nachher:

\TmpSchaltungDrei{1}{0}{2}{3}
\end{multicols}

%-----------------------------------------------------------------------
%
%-----------------------------------------------------------------------

\section{Erreichbarkeitsgraph
\footcite[Seite 11]{sosy:fs:3}}

Ein Erreichbarkeitsgraph (auch Markierungsgraph genannt) ist ein
gerichteter Graph, der aus einem Petri-Netz und einer Anfangsmarkierung
gewonnen werden kann. Er wird dadurch erzeugt, dass, mit der
Anfangsmarkierung beginnend, die Menge der in der Markierung aktivierten
Transitionen ermittelt und jeweils die Folgemarkierung berechnet wird.
Die Markierungen werden durch Knoten im Erreichbarkeitsgraphen
dargestellt und der Übergang einer Markierung zu ihrer Folgemarkierung
wird als Kante im Graphen vermerkt. Für jede Folgemarkierung wird dieser
Vorgang wiederholt.\footcite{wiki:erreichbarkeitsgraph}

\def\TmpErreichbar(#1,#2,#3,#4)#5{
\begin{tikzpicture}[li petri,scale=#5,transform shape]
    \node[place,label=$s_1$,tokens=#1] at (0,1.5) (s1) {};
    \node[place,label=$s_2$,tokens=#2] at (3,1.5) (s2) {};
    \node[place,label=$s_3$,tokens=#3] at (0,0)   (s3) {};
    \node[place,label=$s_4$,tokens=#4] at (3,0)   (s4) {};

    \node[transition] at (1.5,1.5) {$t_1$}
      edge[pre] (s1)
      edge[post] (s2);
    \node[transition] at (1.5,0) {$t_2$}
      edge[pre] (s3)
      edge[post] (s4);
    \node[transition] at (4.5,0.75) {$t_3$}
      edge[pre] (s2)
      edge[pre] (s4)
      edge[post,bend right=70] (s1)
      edge[post,bend left=70] (s3);
  \end{tikzpicture}
}

\bigskip
\bPseudoUeberschrift{Beispiel} Gegeben sie folgendes Petri-Netz:

\begin{center}
\TmpErreichbar(2,0,1,0){1}
\end{center}

\bigskip
\bPseudoUeberschrift{Ergebnis} Daraus ergibt sich folgender Erreichbarkeitsgraph:

\begin{center}
\begin{tikzpicture}[li petri,x=2cm,y=2cm]
\def\pfeil#1#2#3#4{\draw[->] (#1) -- (#2) node[pos=0.5,auto,sloped,#4]{#3};}
\node at (2,3) (2010) {(2,0,1,0)};
\node at (1,2) (1110) {(1,1,1,0)};
\node at (3,2) (2001) {(2,0,0,1)};
\node at (0,1) (0210) {(0,2,1,0)};
\node at (2,1) (1101) {(1,1,0,1)};
\node at (1,0) (0201) {(0,2,0,1)};

\pfeil{2010}{1110}{$t_1$}{}
\pfeil{2010}{1101}{$t_1$, $t_2$}{}
\pfeil{2010}{2001}{$t_2$}{swap}
\pfeil{1110}{0210}{$t_1$}{}
\pfeil{1110}{0201}{$t_1$, $t_2$}{}
\pfeil{1110}{1101}{$t_2$}{}
\pfeil{0210}{0201}{$t_2$}{}
\pfeil{1101}{0201}{$t_1$}{}
\pfeil{2001}{1101}{$t_1$}{swap}

\draw[->,rounded corners=2cm] (1101) -- (4,2) -- (2010) node[pos=0.5,sloped,auto,swap]{$t_3$};
\draw[->,rounded corners=2cm] (0201) -- (-1,1) -- (1110) node[pos=0.5,sloped,auto]{$t_3$};
\end{tikzpicture}
\end{center}

\begin{multicols}{5}
\noindent
(1,1,1,0):
\TmpErreichbar(1,1,1,0){0.4}

\noindent
(2,0,0,1):
\TmpErreichbar(2,0,0,1){0.4}

\noindent
(1,1,0,1):
\TmpErreichbar(1,1,0,1){0.4}

\noindent
(0,2,1,0):
\TmpErreichbar(0,2,1,0){0.4}

\noindent
(0,2,0,1):
\TmpErreichbar(0,2,0,1){0.4}
\end{multicols}

\literatur

\end{document}
