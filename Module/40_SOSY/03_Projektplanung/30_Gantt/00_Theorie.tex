\documentclass{bschlangaul-theorie}
\bLadePakete{gantt,cpm}
\begin{document}

%%%%%%%%%%%%%%%%%%%%%%%%%%%%%%%%%%%%%%%%%%%%%%%%%%%%%%%%%%%%%%%%%%%%%%%%
% Theorie-Teil
%%%%%%%%%%%%%%%%%%%%%%%%%%%%%%%%%%%%%%%%%%%%%%%%%%%%%%%%%%%%%%%%%%%%%%%%

\chapter{Gantt-Diagramm}

\begin{liQuellen}
\item \cite[Seite 194]{schatten}
\item \cite{wiki:gantt-diagramm}
\item \cite[Seite 370-371]{kleuker}
\end{liQuellen}

\noindent
Ein Gantt-Diagramm ist ein Instrument des Projektmanagements, das die
zeitliche Abfolge von Aktivitäten graphisch in Form von \memph{Balken}
auf einer \memph{Zeitachse} darstellt.

\begin{center}
\begin{ganttchart}[
  x unit=0.7cm,
  y unit chart=0.6cm
]{1}{14}
\gantttitlelist{1,...,14}{1} \\
\ganttbar[name=A1]{A1}{3}{5} \\
\ganttbar[name=A2]{A2}{1}{5} \\
\ganttbar[name=A3]{A3}{6}{7} \\
\ganttbar[name=A4]{A4}{6}{10} \\
\ganttbar[name=A5]{A5}{8}{14} \\
\ganttbar[name=A6]{A6}{11}{14}
\ganttlink{A1}{A3}
\ganttlink{A1}{A4}
\ganttlink[link type=f-s]{A2}{A4}
\ganttlink[link type=f-s]{A3}{A5}
\ganttlink[link type=f-s]{A4}{A6}
\end{ganttchart}
\end{center}

\section{Anordnungsbeziehungen (AOB) im Gantt-Diagramm\footcite[Seite
34]{sosy:fs:3}}

\begin{description}

%%
%
%%

\item[Normalfolge EA:]
\emph{end-to-start relationship}
%
Anordnungsbeziehung vom \textbf{E}nde eines Vorgangs zum \textbf{A}nfang
seines
Nachfolgers.

\begin{center}
\begin{ganttchart}{1}{9}
\ganttbar{}{2}{4} \\
\ganttbar{}{6}{8}
\ganttlink[link type=f-s]{elem0}{elem1}
\end{ganttchart}
\end{center}

Vorgang 1 muss abgeschlossen sein, bevor Vorgang 2 beginnen kann
(sequentielle Vorgänge). Diese Anordnungsbeziehung kommt sehr häufig
vor.\footcite[Seite 31]{sosy:fs:3}

%%
%
%%

\item[Anfangsfolge AA:]
\emph{start-to-start relationship}
%
Anordnungsbeziehung vom \textbf{A}nfang eines Vorgangs zum
\textbf{A}nfang seines Nachfolgers.

\begin{center}
\begin{ganttchart}{1}{9}
\ganttbar{}{2}{4} \\
\ganttbar{}{6}{8}
\ganttlink[link type=s-s]{elem0}{elem1}
\end{ganttchart}
\end{center}

Der Beginn von Vorgang 1 ist Voraussetzung für den Beginn von Vorgang 2
(paralleler Start der Vorgänge)\footcite[Seite 31]{sosy:fs:3}

%%
%
%%

\item[Endefolge EE:]
\emph{finish-to-finish relationship}
%
Anordnungsbeziehung vom \textbf{E}nde eines Vorgangs zum \textbf{E}nde
seines Nachfolgers.

\begin{center}
\begin{ganttchart}{1}{9}
\ganttbar{}{2}{4} \\
\ganttbar{}{6}{8}
\ganttlink[link type=f-f]{elem0}{elem1}
\end{ganttchart}
\end{center}

Vorgang 1 muss abgeschlossen sein, bevor Vorgang 2 beendet werden kann
(paralleles Ende der Vorgänge).
\footcite[Seite 32]{sosy:fs:3}

%%
%
%%

\item[Sprungfolge AE:]
\emph{start-to-finish relationship}
%
Anordnungsbeziehung vom \textbf{A}nfang eines Vorgangs zum \textbf{E}nde
seines Nachfolgers

\begin{center}
\begin{ganttchart}{1}{9}
\ganttbar{}{6}{8} \\
\ganttbar{}{2}{4}
\ganttlink[link type=s-f]{elem0}{elem1}
\end{ganttchart}
\end{center}

Vorgang 2 kann erst dann abgeschlossen werden, wenn Vorgang 1 begonnen
wurde. Diese Anordnungsbeziehung kommt sehr selten vor.
\footcite[Seite 32]{sosy:fs:3}

\end{description}

%-----------------------------------------------------------------------
%
%-----------------------------------------------------------------------

\section{Von Gantt nach CPM}

\begin{enumerate}

\item Für das \memph{ganze Projekt} einen Start- und Endknoten
einzeichnen (falls gefordert bzw. nötig).

\item Für \memph{jede Aktivität} (= Balken im Gantt-Diagramm) einen
Start- und einen Endknoten einzeichnen.

\item Zwischen Start- und Endknoten einen \memph{Vorgang} einzeichnen.
Seine Dauer entspricht der Länge des Balkens (falls Pufferzeiten in
Klammern angegeben sind, werden sie von der Balkenlänge abgezogen).

\item Für jede \memph{Anordnungsbeziehung}, die im Gantt-Diagramm
vorkommt ist, einen \memph{Vorgang} einzeichnen. Ist der
\memph{zeitliche Versatz} zwischen dem Balken der Aktivität und dem der
Nachfolge-Aktivität

\begin{quote}
\begin{description}
\item[positiv,] so entspricht die Richtung des Pfeils der im Gantt
Diagramm ist.

\item[negativ,] so wird die Richtung des Pfeils im CPM Netz umgekehrt.

\item[null,] so wird einen Scheinvorgang eingezeichnet.
\end{description}
\end{quote}

Der Betrag des zeitlichen Versatzes der Balken ist die Dauer des
Vorgangs (also immer positiv).\footcite[Seite 35]{sosy:fs:3}
\end{enumerate}

%-----------------------------------------------------------------------
%
%-----------------------------------------------------------------------

\section{Von CPM nach Gantt}

\begin{enumerate}
\item Zeichne für jeden Vorgang einen Balken ein. Beginne beim FAZ und
ende beim SEZ des Vorgangs. Optional: Trage die Bezeichnung des Vorgangs
ein.

\item Optional (falls gefordert): Trage Dauer und Puffer ein.

\item Pfeile im CPM die zwischen Anfangs- und Endpunkt verschiedener
Vorgänge verlaufen werden zu Anordnungsbeziehungen (AOBs):

\begin{itemize}
\item identifiziere im CPM denjenigen Vorgang, der der Nachfolgevorgang
ist (\zB Entwurf ist Nachfolger der Anforderungsanalyse).

\item identifiziere damit die Art der AOB (EA, AA, EE, AE)

\item Zeichne sie im Gantt Diagramm ein

\item Optional (falls gefordert): beschrifte sie mit der identifizierten
Art (EA, AA, EE, AE) und ggf. der

\begin{itemize}
\item Verzögerung = positiver Wert: Pfeil zeigt im CPM zum
Nachfolgevorgang

\item Überlappung = negativer Wert: Pfeil geht im CPM vom
Nachfolgevorgang aus
\end{itemize}
\end{itemize}
\end{enumerate}

\footcite[Seite 36]{sosy:fs:3}

\literatur

\end{document}
