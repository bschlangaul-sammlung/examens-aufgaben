\documentclass{lehramt-informatik-aufgabe}
\liLadePakete{cpm}
\begin{document}
\liAufgabenMetadaten{
  Titel = {Vorlesungsaufgabe},
  Thematik = {Ereignisse 1-8},
  RelativerPfad = Module/40_SOSY/03_Projektplanung/20_CPM-Netzplantechnik/Aufgabe_Vorlesungsaufgabe.tex,
  ZitatSchluessel = sosy:fs:3,
  ZitatBeschreibung = {Seite 16 - 21},
  BearbeitungsStand = unbekannt,
  Korrektheit = unbekannt,
}

\let\SZ=\liCpmSpaetI
\let\FZ=\liCpmFruehI
\let\f=\footnotesize
\let\vz=\liCpmVonZu

Die untenstehende Abbildung stellt ein CPM-Netzwerk dar. Die Ereignisse
sind fortlaufend nummeriert (Nummer im Inneren der Kreise) und tragen
keine Namen.

\begin{center}
\begin{tikzpicture}
\liCpmEreignis{1}{0}{2}
\liCpmEreignis{2}{1}{3.5}
\liCpmEreignis{3}{3}{2}
\liCpmEreignis{4}{2}{0.5}
\liCpmEreignis{5}{4}{3.5}
\liCpmEreignis{6}{5}{2}
\liCpmEreignis{7}{5}{0.5}
\liCpmEreignis{8}{7.5}{2}

\liCpmVorgang{1}{2}{5}
\liCpmVorgang{1}{3}{18}
\liCpmVorgang{1}{4}{7}
\liCpmVorgang{4}{7}{15}
\liCpmVorgang{7}{8}{6}
\liCpmVorgang{3}{6}{8}
\liCpmVorgang{3}{7}{1}
\liCpmVorgang{2}{5}{14}
\liCpmVorgang{5}{8}{11}
\liCpmVorgang{6}{8}{4}
\end{tikzpicture}
\end{center}

\begin{enumerate}

%%
% a)
%%

\item Berechnen Sie die früheste Zeit für jedes Ereignis, wobei
angenommen wird, dass das Projekt zum Zeitpunkt 0 startet!

\begin{liAntwort}
\liCpmFruehErklaerung

\begin{tabular}{|l|l|r|}
\hline
i & Nebenrechnung            & \FZ \\\hline
1 &                          & 0   \\
2 &                          & 5   \\
3 &                          & 18  \\
4 &                          & 7   \\
5 &                          & 19  \\
6 &                          & 26  \\
7 & $\max(19_3, 22_4)$       & 22  \\
8 & $\max(30_5, 30_6, 28_7)$ & 30  \\\hline
\end{tabular}
\end{liAntwort}

%%
%
%%

\item Setzen Sie anschließend beim letzten Ereignis die späteste Zeit
gleich der frühesten Zeit und berechnen Sie die spätesten Zeiten!

\begin{liAntwort}
\liCpmSpaetErklaerung

\begin{tabular}{|l|l|r|}
\hline
$i$ & Nebenrechnung         & \SZ \\\hline
8 & siehe \FZ[8]            & 30  \\
7 &                         & 24  \\
6 &                         & 26  \\
5 &                         & 19  \\
4 &                         & 9   \\
3 & $\min(18_6, 23_7)$      & 18  \\
2 &                         & 5   \\
1 & $\min(0_2, 0_3, 2_4)$   & 0   \\\hline
\end{tabular}
\end{liAntwort}

%%
%
%%

\item Berechnen Sie nun für jedes Ereignis die Pufferzeiten!

\begin{liAntwort}
GP: gesamter Pufferzeit ($\text{GP} = \text{SZ} - \text{FZ}$)
\end{liAntwort}

%%
%
%%

\item Bestimmen Sie den kritischen Pfad!

\begin{liAntwort}
Kritische Pfade: Pfad(e) mit minimaler Pufferzeit, meist $0$

\begin{tabular}{|l|l|l|l|l|l|l|l|l|}
\hline
$i$           & 1 & 2 & 3  & 4 & 5  & 6  & 7  & 8  \\\hline\hline
\FZ & 0 & 5 & 18 & 7 & 19 & 26 & 22 & 30 \\\hline
\SZ & 0 & 5 & 18 & 9 & 19 & 26 & 24 & 30 \\\hline
GP            & 0 & 0 & 0  & 2 & 0  & 0  & 2  & 0  \\\hline
\end{tabular}

\bigskip

\bigskip

\end{liAntwort}
\end{enumerate}
\end{document}
