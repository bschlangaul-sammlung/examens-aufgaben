\documentclass{bschlangaul-theorie}
\bLadePakete{syntax,cyk-algorithmus,grafik}
\begin{document}
\def\t#1#2{$t_{#1,#2}$}
\let\l=\bKurzeTabellenLinie

\chapter{CYK-Algorithmus}
\index{CYK-Algorithmus}

\begin{bQuellen}
\item \cite[Seite 45-75]{theo:fs:2}
\item \cite[Seite 186-188]{hoffmann}
\item \cite{wiki:cyk}
\end{bQuellen}

\section{Online-Tools}

\begin{itemize}
\item \url{https://www.xarg.org/tools/cyk-algorithm/}
\end{itemize}

\noindent
Der Algorithmus ist nach Cocke, Younger, Kasami benannt. Er dient dazu
das \bEmph{Wortproblem} für eine Sprache zu entscheiden, die durch eine
CNF-Grammatik (\bEmph{Chomsky-Normalform}) gegeben ist.

Bilde zu einem Wort $w$ alle Teilwörter der Länge $1, 2, 3, \dots$ und
bestimme die Ableitbarkeit von Variablen. Ist das „Teilwort“ der Länge
$|w|$ vom Startsymbol ableitbar, so gehört $w$ zur Sprache.

\begin{center}
\begin{tabular}{|c||c|c|c|c|c|}
\hline
    & a    & b    & a    & b    & a \\
\hline
i/j & 1    & 2    & 3    & 4    & 5 \\\hline\hline
1   & \t11 & \t12 & \t13 & \t14 & \t15 \l6
2   & \t21 & \t22 & \t23 & \t24 \l5
3   & \t31 & \t32 & \t33 \l4
4   & \t41 & \t42 \l3
5   & \t51 \l2
\end{tabular}
\end{center}

\section{Funktionsweise des Algorithmus}

Wir leiten die erste Tabellenzeile aus den Produktionsregeln ab.
Beginnend mit der zweiten Tabellenzeile wählen wir paarweise zwei
Tabellenzellen aus und zwar mit folgendem Muster:

\bPseudoUeberschrift{Zum Beispiel: \t21}

\begin{enumerate}
\item \t11 und \t12
\end{enumerate}

\bPseudoUeberschrift{Zum Beispiel: \t51}

\begin{enumerate}
\item \t11 und \t42
\item \t21 und \t33
\item \t31 und \t24
\item \t41 und \t15
\end{enumerate}

\def\n#1#2{\node at (#2, #1) {\t#1#2};}

\begin{center}
\begin{tikzpicture}[y=-1cm]

\draw[->] (0,1) -- (0,4);

\n11
\n21
\n31
\n41

\draw[->] (3,4) -- (6,1);

\n42
\n33
\n24
\n15
\end{tikzpicture}
\end{center}

Sind die richtigen Tabellenzellen ausgewählt, dann kombinieren wir alle
Nonterminale von der ersten ausgewählten Zelle mit der zweiten.
Dabei muss die Reihenfolge eingehalten werden.

\begin{center}
Zum Beispiel: \t11 (A,B) \t12 (C,D): A,C A,D B,C B,D
\end{center}

Steht in der letzten Zelle die Start-Variable, dann ist das Wort
ableitbar.

\bJavaDatei{formale_sprachen/CYKAlgorithmus}

\literatur

\end{document}
