\documentclass{bschlangaul-aufgabe}
\liLadePakete{formale-sprachen,syntaxbaum,automaten}
\begin{document}
\liAufgabenTitel{Vorlesungsaufgabe}
\section{Kontextfreie Grammatik
\index{Kontextfreie Sprache}
\footcite{theo:fs:2}}

\begin{enumerate}
\item Erstellen Sie eine Ableitung für die Wörter der Sprache zur
vorgegeben Grammatik\footcite[Seite 10]{theo:fs:2}
\index{Ableitung (Kontextfreie Sprache)}

\liGrammatik{alphabet={0, 1}, variablen={S, A, B }}

\begin{liProduktionsRegeln}
S -> A 1 B,
A -> 0 A | EPSILON,
B -> 0 B | 1 B | EPSILON,
\end{liProduktionsRegeln}

\liFlaci{Gi1rgpemg}

\begin{itemize}
\item 00101

\begin{liAntwort}
\liAbleitung{
S ->
A1B ->
0A1B ->
00A1B ->
001B ->
0010B ->
00101B ->
00101
}
\end{liAntwort}

\item 1001

\begin{liAntwort}
\liAbleitung{
S ->
A1B ->
1B ->
10B ->
100B ->
1001B ->
1001
}
\end{liAntwort}
\end{itemize}

\item Erstellen Sie eine kontextfreie Grammatik, die alle Wörter mit
gleich vielen $1$‘s, gefolgt von gleich vielen $0$‘s enthält.
\index{Kontextfreie Grammatik}

\begin{liAntwort}
\begin{liProduktionsRegeln}
S -> 1S0 | epsilon
\end{liProduktionsRegeln}
\liFlaci{Grxmyw2ia}
\end{liAntwort}

\item Erstellen Sie eine kontextfreie Grammatik, die alle regulären
Ausdrücke über den Zeichen $0,1$ darstellt. (Beispiel:
\texttt{01*(1+0)0} für einen möglichen regulären Ausdruck (Das
\texttt{+}-Zeichen ist hier anstelle des Oder-Zeichens (|)))
\footcite[Aufgabe 2a)]{theo:ab:5}

\begin{liAntwort}
\liGrammatik{alphabet={1; 0; (; ); +; *}, variablen={S}}

\begin{liProduktionsRegeln}
S -> EPSILON | 0 | 1 | S * | ( S ) | S S | S + S
\end{liProduktionsRegeln}
\liFlaci{Ghfgrv027}
\end{liAntwort}

\end{enumerate}

\end{document}
