\documentclass{lehramt-informatik-aufgabe}
\liLadePakete{formale-sprachen,pumping-lemma}
\begin{document}
\liAufgabenTitel{„w w“ und „a^k b^l c^m“}
\section{Pumping-Lemma
\index{Pumping-Lemma (Kontextfreie Sprache)}
\footcite[Aufgabe 11: Pumping-Lemma]{theo:ab:2}}

Zeigen Sie jeweils, dass die angegebene Sprache nicht kontextfrei ist:

\begin{liExkurs}[Pumping-Lemma für Kontextfreie Sprachen]
\liPumpingKontextfrei
\end{liExkurs}

\begin{enumerate}

%%
% (a)
%%

\item \liAusdruck[L_1]{ww}{ w \in \{ a, b \}^*}

\begin{liAntwort}
Sei $L_1$ kontextfrei. Dann existiert nach dem Pumping-Lemma eine Zahl
$j$, so dass für jedes Wort $\omega \in L_1$ mit $|\omega| \geq j$ eine
Zerlegung $\omega = uvwxy$ existiert, für die gilt: $|vx| > 0$, $|vwx|
\leq n$ und für jedes $i \in \mathbb{N}$ ist $u v^i x y^i z \in L_1$.

Wähle $\omega = a^n b^n a^n b^n$. Dann gibt es für jede Zerlegung
$\omega = uvxyz$ mit den obigen Bedingungen zwei Möglichkeiten:

\begin{itemize}
\item $vwx$ besteht aus $a^j b^k$ mit $j + k > 0$.

\item $vwx$ besteht aus $b^j a^k$ mit $j + k > 0$.
\end{itemize}
Dann ist in beiden Fällen $u v^0 x y^0 z \notin L_1$.
\end{liAntwort}

%%
% (b)
%%

\item \liAusdruck[L_2]{a^k b^l c^m}{k > l > m; k, l, m \in N}

\begin{liAntwort}
Sei $L_2$ kontextfrei. Dann existiert nach dem Pumping-Lemma eine Zahl
$j$, so dass für jedes Wort $\omega \in L_2$ mit $|\omega| \geq j$ eine
Zerlegung $\omega = uvwxy$ existiert, für die gilt: $|vx| > 0$, $|vwx|
\leq j$ und für jedes $i \in \mathbb{N}$ ist $u v^i w x^i y \in L_2$.

Wähle $\omega = a^n b^{n-1} c^{n-2}$.
Dann gibt es für jede Zerlegung $\omega = uvwxy$ mit den obigen
Bedingungen zwei Möglichkeiten:

\begin{itemize}
\item $vwx$ enthält kein $a$.
Dann ist $u v^2 w x^2 y \notin L_2$.

\item $vwx$ enthält mindestens ein $a$.
Dann ist $u v^0 w x^0 y \notin L_2$.
\end{itemize}
\end{liAntwort}
\end{enumerate}

\end{document}
