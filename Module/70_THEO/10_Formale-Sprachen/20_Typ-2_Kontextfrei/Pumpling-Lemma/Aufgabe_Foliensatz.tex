\documentclass{bschlangaul-aufgabe}
\liLadePakete{formale-sprachen,pumping-lemma}
\begin{document}
\liAufgabenTitel{a^n b^n c^n}
\section{Pumping-Lemma
\index{Pumping-Lemma (Kontextfreie Sprache)}
\footcite[Seite 42]{theo:fs:2}}

Gegeben sei die Sprachen

\begin{center}
\liAusdruck{a^n b^n c^n}{n \in \mathbb{N}}
\end{center}

\noindent
Weisen Sie nach, dass $L$ nicht kontextfrei ist.

\begin{liExkurs}[Pumping-Lemma für Kontextfreie Sprachen]
\liPumpingKontextfrei
\end{liExkurs}

\begin{liAntwort}
Also gibt es eine Pumpzahl. Sie sei $j$. (Wähle geschickt ein „langes“
Wort...) $a^j b^j c^j$ ist ein Wort aus $L$, das sicher länger als $j$
ist.

Da $L$ kontextfrei ist, muss es nach dem Pumping-Lemma auch für
dieses Wort eine beliebige Zerlegung geben:

$a^j b^j c^j = uvwxy$ mit $|vx| \geq 1$ und $|vwx| \leq j$

Weil $vwx$ höchstens $j$ lang ist, kann es nie $a$‘s und $c$‘s
zugleich enthalten (es stehen $j$ $b$‘s dazwischen!).

Andererseits enthält $vx$ mindestens ein Zeichen. Das Wort $\omega
= uv^0 wx^0 y = uwy$ enthält dann nicht mehr gleich viele $a$‘s, $b$‘s
und $c$‘s. (Widerspruch)!

Die Behauptung ist falsch.

\noindent
$\Rightarrow$ $L$ ist nicht kontextfrei!

\end{liAntwort}
\end{document}
