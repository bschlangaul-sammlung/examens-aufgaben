\documentclass{bschlangaul-aufgabe}
\bLadePakete{formale-sprachen,automaten}
\begin{document}
\bAufgabenTitel{Balancierte Klammern}
\section{Balancierte Klammern
\index{Kellerautomat}
}

Erstellen Sie einen Kellerautomaten der nur balancierte Klammerausdrücke
mit eckigen Klammern akzeptiert.
\bFussnoteUrl[Seite 9]{https://eecs.wsu.edu/~ananth/CptS317/Lectures/PDA.pdf}

\begin{bAntwort}
\bKellerAutomat{
  zustaende={z_0, z_1, z_2, z_3},
  alphabet={[, ]},
  kelleralphabet={\#, K},
  ende={z_2, z_3},
}

\begin{center}
\begin{tikzpicture}[li kellerautomat]
  \node[state,initial] (z0) at (2.14cm,-2.14cm) {$z_0$};
  \node[state] (z1) at (7.14cm,-2.14cm) {$z_1$};
  \node[state,accepting] (z2) at (9.71cm,-2.14cm) {$z_2$};
  \node[state,accepting] (z3) at (2.57cm,-4.71cm) {$z_3$};

  \bKellerKante[above,bend left]{z0}{z1}{
    EPSILON, K, K;
  }

  \bKellerKante[above,loop]{z0}{z0}{
    [, KELLERBODEN, K KELLERBODEN;
    [, K, K K;
  }

  \bKellerKante[left]{z0}{z3}{
    EPSILON, KELLERBODEN, EPSILON;
  }

  \bKellerKante[above,bend left]{z1}{z0}{
    EPSILON, KELLERBODEN, KELLERBODEN;
  }

  \bKellerKante[above]{z1}{z2}{
    EPSILON, KELLERBODEN, EPSILON;
  }

  \bKellerKante[above,loop]{z1}{z1}{
    ], K, EPSILON;
  }
\end{tikzpicture}
\end{center}

\bFlaci{Apwobf482}
\end{bAntwort}

\end{document}
