\documentclass{bschlangaul-aufgabe}
\liLadePakete{formale-sprachen,automaten,mathe}
\begin{document}
\let\m=\liMengeOhneMathe
\def\z#1{z_#1}
\def\Z#1{$z_#1$}

\liAufgabenTitel{a^n b^n}
\section{Kellerautomat
\index{Kellerautomat}
\footcite[Seite 23]{theo:fs:2}}

Erstellen Sie einen Kellerautomaten, der folgende Sprache

\begin{center}
\liAusdruck{a^n b^n}{n \in \mathbb{N}}
\end{center}

\noindent
mit folgender Grammatik

\begin{center}
\liGrammatik{
  variablen=S,
  alphabet={a,b},
  produktionen={S -> aSb | ab}
}
\end{center}

\noindent
erkennt.

\begin{liAntwort}
\begin{center}
\liKellerAutomat{
  zustaende={\z1, \z2},
  alphabet={a, b},
  kelleralphabet={\#, A},
  start=\z1,
  ende={\z2},
}
\end{center}

\begin{center}
\begin{tikzpicture}[li kellerautomat]
  \node[state,initial] (z1) at (2cm,-2.14cm) {$z_1$};
  \node[state,accepting] (z2) at (5.57cm,-2.14cm) {$z_2$};

  \liKellerKante[above]{z1}{z2}{
    b, A, EPSILON;
    EPSILON, KELLERBODEN, EPSILON
  }

  \liKellerKante[above,loop]{z1}{z1}{
    a, A, AA;
    a, KELLERBODEN, AKELLERBODEN
  }

  \liKellerKante[above,loop]{z2}{z2}{
    b, A, EPSILON;
    EPSILON, KELLERBODEN, EPSILON
  }

\end{tikzpicture}
\end{center}

\liFlaci{Ah5v17t52}

\bigskip

\begin{tabular}{|l|l|l|l|l|}
Aktueller Zustand &  Eingabe   & Keller & Folgezustand & Keller \\\hline
\Z1               & a          & \#     & \Z1          & A\# \\
\Z1               & a          & A      & \Z1          & AA \\
\Z1               & b          & A      & \Z2          & $\varepsilon$ \\
\Z2               & b          & A      & \Z2          & $\varepsilon$ \\
\Z2               & $\varepsilon$ & \#     & \Z2          & \# \\
\end{tabular}

\end{liAntwort}
\end{document}
