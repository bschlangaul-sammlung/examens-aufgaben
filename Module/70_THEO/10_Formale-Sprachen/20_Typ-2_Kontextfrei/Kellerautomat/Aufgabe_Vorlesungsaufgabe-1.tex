\documentclass{lehramt-informatik-aufgabe}
\liLadePakete{formale-sprachen,automaten}
\begin{document}
\liAufgabenMetadaten{
  Titel = {Vorlesungsaufgabe Kellerautomaten},
  Thematik = {a hoch n c hoch i b hoch n},
  RelativerPfad = Module/70_THEO/10_Formale-Sprachen/20_Typ-2_Kontextfrei/Kellerautomat/Aufgabe_Vorlesungsaufgabe-1.tex,
  ZitatSchluessel = theo:fs:2,
  ZitatBeschreibung = {Seite 25},
  BearbeitungsStand = mit Lösung,
  Korrektheit = korrekt,
  Stichwoerter = {Kontextfreie Sprache, Kellerautomat, Kontextfreie Grammatik, Konfigurationsfolge},
}

\let\m=\liMenge
\let\e=\liEpsilon
\let\z=\liZustandsnameTiefgestellt
\def\p{$\vdash$ }

\begin{enumerate}
\item Geben Sie einen Kellerautomaten an, der die folgende Sprache erkennt:
\index{Kontextfreie Sprache}
\footcite[Seite 25]{theo:fs:2}
\index{Kellerautomat}

\liAusdruck{a^n c^i b^n}{n, i \in \mathbb{N}_0}

\begin{liAntwort}
\liKellerAutomat{zustaende={z_0, z_1, z_2}, alphabet={a, b, c}, kelleralphabet={\#, A}, ende=z_2}

\begin{center}
\begin{tikzpicture}[li kellerautomat]
  \node[state,initial] (z0) at (3cm,-6.57cm) {$z_0$};
  \node[state] (z1) at (5.29cm,-2.86cm) {$z_1$};
  \node[state,accepting] (z2) at (8cm,-6.57cm) {$z_2$};

  \liKellerKante[left]{z0}{z1}{
    c, A, A;
    c, KELLERBODEN, KELLERBODEN;
  }

  \liKellerKante[above]{z0}{z2}{
    EPSILON, KELLERBODEN, EPSILON;
    b, A, EPSILON;
  }

  \liKellerKante[below,loop below]{z0}{z0}{
    a, KELLERBODEN, A KELLERBODEN;
    a, A, A A;
  }

  \liKellerKante[right]{z1}{z2}{
    b, A, EPSILON;
    EPSILON, KELLERBODEN, EPSILON;
  }

  \liKellerKante[above,loop]{z1}{z1}{
    c, A, A;
    c, KELLERBODEN, KELLERBODEN;
  }

  \liKellerKante[below,loop below]{z2}{z2}{
    EPSILON, KELLERBODEN, EPSILON;
    b, A, EPSILON;
  }
\end{tikzpicture}
\end{center}

\liFlaci{Apky9znog}

% akzeptiert
% c
% ab
% aabb
% aacbb

% akzeptiert nicht
% a
% b
% aab

Tabellenform:

\begin{tabular}{|l|l|l|l|l|l|}
\hline
Aktueller Zustand &  Eingabe   & Keller & Folgezustand & Keller \\\hline\hline
\z0 & a  & \# & \z0 & A\# \\
\z0 & a  & A  & \z0 & AA  \\\hline

\z0 & c  & \# & \z1 & \#  \\
\z0 & c  & A  & \z1 & A   \\\hline

\z0 & \e & \# & \z2 & \e  \\
\z0 & b  & A  & \z2 & \e  \\\hline

\z1 & c  & \# & \z1 & \#  \\
\z1 & c  & A  & \z1 & A   \\\hline

\z1 & \e & \# & \z2 & \e  \\
\z1 & b  & A  & \z2 & \e  \\\hline

\z2 & \e & \# & \z2 & \e  \\
\z2 & b  & A  & \z2 & \e  \\\hline
\end{tabular}
\end{liAntwort}

\item Geben Sie eine Grammatik für diese Sprache an.
\index{Kontextfreie Grammatik}

\begin{liAntwort}
\begin{liProduktionsRegeln}
S -> a S b | EPSILON | c | c C,
C -> c C | EPSILON
\end{liProduktionsRegeln}

alternativ:

\begin{liProduktionsRegeln}
S -> a S b | EPSILON | C,
C -> c C | EPSILON
\end{liProduktionsRegeln}
\end{liAntwort}

\item Geben Sie Konfigurationsfolgen für die Erzeugung des Wortes an
\index{Konfigurationsfolge}

\begin{itemize}
\item aacbb

\begin{liAntwort}
(\z0, aacbb, \#)   \p
(\z0, acbb,  A\#)  \p
(\z0, cbb,   AA\#) \p
(\z1, bb,    AA\#) \p
(\z2, b,     A\#)  \p
(\z2, \e,    \#)   \p
(\z2, \e,   \e)
\end{liAntwort}

\item accb

\begin{liAntwort}
(\z0, accb, \#)  \p
(\z0, ccb,  A\#) \p
(\z1, cb,   A\#) \p
(\z2, b,    A\#) \p
(\z2, \e,   \#)  \p
(\z2, \e,   \e)
\end{liAntwort}
\end{itemize}
\end{enumerate}
\end{document}
