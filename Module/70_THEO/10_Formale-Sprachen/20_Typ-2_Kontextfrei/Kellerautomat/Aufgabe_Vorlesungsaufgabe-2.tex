\documentclass{lehramt-informatik-aufgabe}
\liLadePakete{formale-sprachen,automaten}
\begin{document}
\let\m=\liMenge
\let\z=\liZustandsnameTiefgestellt

\liAufgabenTitel{Vorlesungsaufgabe}

\section{Kellerautomaten\footcite[Seite 27]{theo:fs:2}}

Erstellen Sie einen Kellerautomaten zu der Grammatik
\liGrammatik{variablen={P}, alphabet={0, 1}} mit den folgenden
Produktionsregeln
\index{Kellerautomat}

\begin{enumerate}

%%
%
%%

\item

\begin{liProduktionsRegeln}
S -> EPSILON | 0 | 1 | 0 P 0 | 1 P 1,
\end{liProduktionsRegeln}

\begin{liAntwort}
\liKellerAutomat{
  zustaende={\z0, \z1},
  alphabet={0, 1},
  kelleralphabet={\#, N, E},
  ende={z_1},
}

N = Null

E = Eins

\begin{center}
\begin{tikzpicture}[li kellerautomat]
  \node[state,initial] (z0) at (3.57cm,-3.57cm) {$z_0$};
  \node[state,accepting] (z1) at (6.86cm,-3.57cm) {$z_1$};

  \liKellerKante[above]{z0}{z1}{
    EPSILON, KELLERBODEN, EPSILON;
    EPSILON, N, N;
    EPSILON, E, E;
    1, E, E;
    0, E, E;
    0, N, N;
  }

  \liKellerKante[above,loop]{z0}{z0}{
    0, KELLERBODEN, N KELLERBODEN;
    0, N, N N;
    0, E, N E;
    1, KELLERBODEN, E KELLERBODEN;
    1, E, E E;
    1, N, N E;
  }

  \liKellerKante[above,loop]{z1}{z1}{
    0, N, EPSILON;
    1, E, EPSILON;
  }
\end{tikzpicture}
\end{center}

\liFlaci{Ahij8jnn7}
\end{liAntwort}

%%
%
%%

\item

\begin{liProduktionsRegeln}
S -> A 1 B,
A -> 0 A | EPSILON,
B -> 0 B | 1 B | EPSILON,
\end{liProduktionsRegeln}
\end{enumerate}

\begin{liAntwort}
\liKellerAutomat{
  zustaende={\z0, \z1},
  alphabet={0, 1},
  kelleralphabet={\#, E},
  ende={z_1},
}

\noindent
E = Eins ist gesetzt

\begin{center}
\begin{tikzpicture}[li kellerautomat]
  \node[state,initial] (z0) at (4cm,-8cm) {$z_0$};
  \node[state,accepting] (z1) at (7.86cm,-8cm) {$z_1$};

  \liKellerKante[above,loop]{z0}{z0}{
    0, KELLERBODEN, KELLERBODEN;
    1, KELLERBODEN, E KELLERBODEN;
  }

  \liKellerKante[above]{z0}{z1}{
    0, E, EPSILON;
    1, E, EPSILON;
    EPSILON, E, EPSILON;
  }

  \liKellerKante[above,loop]{z1}{z1}{
    0, KELLERBODEN, KELLERBODEN;
    1, KELLERBODEN, KELLERBODEN;
    EPSILON, KELLERBODEN, EPSILON;
  }
\end{tikzpicture}
\end{center}

\liFlaci{Ar3imp8a7}
\end{liAntwort}

\end{document}
