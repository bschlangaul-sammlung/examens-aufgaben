\documentclass{bschlangaul-aufgabe}
\bLadePakete{formale-sprachen,chomsky-normalform}
\begin{document}
\let\schrittE=\bChomskyUeberErklaerung

\bAufgabenTitel{Drei Grammatiken (SABCX, ST, SAB)}
\section{Chomsky-Normalform
\index{Chomsky-Normalform}
\footcite{theo:ab:2}}

Überführen Sie jeweils die angegebene kontextfreie Grammatik in
Chomsky-Normalform.

\begin{enumerate}

%-----------------------------------------------------------------------
%
%-----------------------------------------------------------------------

\item \bGrammatik{variablen={S, A, B, C, X}, alphabet={a, b, c}} mit $P$:

\begin{liProduktionsRegeln}
S -> X A B | EPSILON,
A -> a A B | A B | c,
B -> B B | C | a,
C -> C C | c | EPSILON,
X -> A | b
\end{liProduktionsRegeln}

%-----------------------------------------------------------------------
%
%-----------------------------------------------------------------------

\item \bGrammatik{variablen={S, T}, alphabet={a, b, c}} mit $P$:

\begin{liProduktionsRegeln}
S -> a S b S | T,
T -> c T | c
\end{liProduktionsRegeln}

\begin{liAntwort}
\begin{enumerate}
\item \schrittE{1}

\bNichtsZuTun

\item \schrittE{2}

\begin{liProduktionsRegeln}
S -> a S b S | c T | c,
T -> c T | c
\end{liProduktionsRegeln}

\item \schrittE{3}

\begin{liProduktionsRegeln}
S -> A S A S | C T | c,
T -> C T | c,
A -> a,
B -> b,
C -> c,
\end{liProduktionsRegeln}

\item \schrittE{4}

\begin{liProduktionsRegeln}
S -> A U | C T | c,
T -> C T | c,
A -> a,
B -> b,
C -> c,
U -> S V
V -> A S
\end{liProduktionsRegeln}

\end{enumerate}
\end{liAntwort}

%-----------------------------------------------------------------------
%
%-----------------------------------------------------------------------

\item \bGrammatik{variablen={S, A, B}, alphabet={a, b, c}} mit $P$:

\begin{liProduktionsRegeln}
S -> A B,
A -> a A A | EPSILON,
B -> b B B | EPSILON
\end{liProduktionsRegeln}
\end{enumerate}

\end{document}
