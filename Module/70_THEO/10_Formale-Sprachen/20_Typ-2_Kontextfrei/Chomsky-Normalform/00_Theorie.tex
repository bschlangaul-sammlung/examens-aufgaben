\documentclass{bschlangaul-theorie}
\bLadePakete{chomsky-normalform}

\begin{document}
\let\schrittE=\bChomskyUeberErklaerung

\chapter{Chomsky-Normalform}

\begin{bQuellen}
\item \cite[Seite 35-37]{theo:fs:2}
\item \cite[Seite 179-181]{hoffmann}
\item \cite{wiki:chomsky-normalform}
\item \cite[Kapitel 19.1.3.2 Seite 590]{schneider}
\item \cite[Seite 192-194]{vossen}
\end{bQuellen}

\section{Online-Tools}

\begin{itemize}
\item \url{https://flaci.com/kfgedit} (Transformieren / Chomsky-Normalform)
\item \url{https://cyberzhg.github.io/toolbox/cfg2cnf}
\end{itemize}

\noindent
Kontextfreie Regeln in Chomsky-Normalform haben die Gestalt

\begin{center}
$A\rightarrow BC$ oder $A \rightarrow a$
\end{center}

\noindent
mit $A, B, C \in \mathbb{N}$ und $a \in \Sigma$. Ihre \bEmph{rechten}
Seiten bestehen also entweder aus \bEmph{genau zwei
Nichtterminalsymbolen} oder aus \bEmph{genau einem
Terminalsymbol}.\footcite[Seite 192]{vossen}
%
Jede kontextfreie Grammatik $G$ mit $\varepsilon \notin L(G)$ kann in
die Chomsky-Normalform gebracht werden.
%
Die Syntaxbäume von Grammatiken in Chomsky-Normalform haben
die Form von Binärbäumen.\footcite[Seite 35]{theo:fs:2}

\begin{enumerate}
\item \schrittE{1}

\item \schrittE{2}

\item \schrittE{3}

\item \schrittE{4}
\end{enumerate}

\literatur

\end{document}
