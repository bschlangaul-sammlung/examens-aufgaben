\documentclass{bschlangaul-aufgabe}
\bLadePakete{syntax,formale-sprachen,syntaxbaum}

\begin{document}
% Zurückbekommen 22.03.21, 08:31
% Korrigiert 23.03.21

\bAufgabenTitel{Klammerausdrücke}
\section{Klammerausdrücke
\index{Kontextfreie Sprache}
\footcite{theo:ab:2}}

In Programmierumgebungen kommen Abfolgen von Klammern, runde, eckige und
geschweifte, vor. Diese müssen in der richtigen Abfolge auf- bzw.
geschlossen werden. Eine korrekte Abfolge von Klammern wäre zum
Beispiel:

\begin{minted}{md}
{[](){(){[]([])}}}{}
\end{minted}

\begin{enumerate}

%%
% (a)
%%

\item Entwerfen Sie eine Grammatik, die die korrekte Abfolge solcher
Klammerfolgen beschreibt.

\begin{liAntwort}
\begin{minted}{md}
S -> { S } | ( S ) | [ S ] | S S | EPSILON
\end{minted}
\bFlaci{Ghjeb39xr}
\end{liAntwort}

%%
% (b)
%%

\item Geben Sie eine Ableitung für den oben angegebenen Klammerausdruck
an.

\begin{liAntwort}
Der oben angegebene Klammerausdruck mit Einrückungen

\begin{minted}{md}
01 {
02   []
03   ()
04   {
05     ()
06     {
07       []
08       (
09         []
10       )
11     }
12   }
13 }
14 {}
\end{minted}

Die Ableitung:

\begin{minted}{md}
S ->
SS ->
{S}S ->                    siehe Zeile 01 13
{SS}S ->
{[S]S}S ->                 siehe Zeile 02
{[]S}S ->
{[]SS}S ->
{[](S)S}S ->               siehe Zeile 03
{[](){S}}S ->              siehe Zeile 04 12
{[](){SS}}S ->
{[](){(S)S}}S ->
{[](){()S}}S ->            siehe Zeile 05
{[](){(){S}}}S ->          siehe Zeile 06 11
{[](){(){SS}}}S ->
{[](){(){[S]S}}}S ->       siehe Zeile 07
{[](){(){[]S}}}S ->
{[](){(){[](S)}}}S ->      siehe Zeile 08 10
{[](){(){[]([S])}}}S ->    siehe Zeile 09
{[](){(){[]([])}}}S ->
{[](){(){[]([])}}}{S} ->   siehe Zeile 14
{[](){(){[]([])}}}{}
\end{minted}
\end{liAntwort}

\newpage

\item Zeichnen Sie eine Ableitungsbaum für den oben angegebenen
Klammerausdruck.

\begin{liAntwort}
\begin{center}
\begin{tikzpicture}[li parsetree,level distance=1.2cm,scale=0.7]
\Tree
[.S
  [.S % 01
    \{
    [.S
      [.S
        [.S \Uchar91 [.S $\varepsilon$ ] \Uchar93 ]  % 02
        [.S ( [.S $\varepsilon$ ] ) ] % 03
      ]
      [.S
        \{ % 04
        [.S
          [.S ( [.S $\varepsilon$ ] ) ] % 05
          [.S
            \{ % 06
              [.S \Uchar91 [.S $\varepsilon$ ] \Uchar93 ] % 07
              [.S
                ( % 08
                  [.S \Uchar91 [.S $\varepsilon$ ] \Uchar93 ] % 09
                ) % 08
              ]
            \} % 11
          ]
        ]
        \} % 12
      ]
    ]
    \}
  ]
  [.S \{ [.S $\varepsilon$ ] \} ] % 14
]
\end{tikzpicture}
\end{center}
\end{liAntwort}

\end{enumerate}
\end{document}
