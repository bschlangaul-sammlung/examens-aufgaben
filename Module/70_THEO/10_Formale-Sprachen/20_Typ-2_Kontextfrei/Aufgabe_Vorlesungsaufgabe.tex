\documentclass{bschlangaul-aufgabe}
\bLadePakete{formale-sprachen,syntaxbaum,automaten}
\begin{document}
\bAufgabenMetadaten{
  Titel = {Kontextfreie Sprache},
  Thematik = {Vorlesungsaufgabe},
  Referenz = THEO.Formale-Sprachen.Typ-2_Kontextfrei.Vorlesungsaufgabe,
  RelativerPfad = Module/70_THEO/10_Formale-Sprachen/20_Typ-2_Kontextfrei/Aufgabe_Vorlesungsaufgabe.tex,
  ZitatSchluessel = theo:fs:2,
  BearbeitungsStand = unbekannt,
  Korrektheit = unbekannt,
  Ueberprueft = {unbekannt},
  Stichwoerter = {Kontextfreie Sprache},
}

\let\m=\bMenge

\section{Kontextfreie Sprache
\index{Kontextfreie Sprache}
\footcite{theo:fs:2}}

\section{Übung\footcite[Seite 34]{theo:fs:2}}

\begin{enumerate}
\item Erstelle eine (deterministische) Grammatik für Palindrome, für die
ein DPDA existiert.

\bAusdruck[L]{w \$ w^R}{w \in (a|b)^*}

\item Wandle diese Grammatik in einen DPDA um.
\end{enumerate}

%-----------------------------------------------------------------------
%
%-----------------------------------------------------------------------

\section{Übung\footcite[Seite 37]{theo:fs:2}}

Überführe die folgenden kontextfreien Grammatiken in CNF

\begin{bProduktionsRegeln}
S -> ABC,
A -> aCD,
B -> bCD,
C -> D | epsilon,
D -> C
\end{bProduktionsRegeln}

\end{document}
