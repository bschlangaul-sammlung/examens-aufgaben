\documentclass{bschlangaul-aufgabe}
\bLadePakete{mathe,automaten,formale-sprachen}
\begin{document}
\bAufgabenTitel{Vorlesungsaufgaben}
\section{Nichtdeterministischer endlicher Automat
\index{Nichtdeterministisch endlicher Automat (NEA)}
\footcite[Seite 34]{theo:fs:1}}

\begin{enumerate}

%%
%
%%

\item Stellen Sie einen nichtdeterministischen endlichen Automaten auf,
der alle durch 2 teilbaren Binärzahlen (letztes Wort ist 0) akzeptiert.

\begin{liAntwort}
\begin{center}
\begin{tikzpicture}[->,node distance=2cm]
\node[state,initial] (0) {$z_0$};
\node[state,right of=0,accepting] (1) {$z_1$};

\path (0) edge[above,loop] node{0,1} (0);
\path (0) edge[above] node{0} (1);
\end{tikzpicture}
\end{center}
\end{liAntwort}

%%
%
%%

\item Stellen Sie einen NEA auf, der alle Wörter über einem Alphabet
\bAlphabet{a, b} akzeptiert, die als vorletztes Zeichen ein $b$
besitzen.

\begin{liAntwort}
\begin{center}
\begin{tikzpicture}[->,node distance=2cm]
\node[state,initial] (0) {$z_0$};
\node[state,right of=0] (1) {$z_1$};
\node[state,right of=1,accepting] (2) {$z_2$};

\path (0) edge[above] node{b} (1);
\path (0) edge[above,loop] node{a,b} (0);
\path (1) edge[above] node{a,b} (2);
\end{tikzpicture}
\end{center}
\end{liAntwort}
\end{enumerate}
\end{document}
