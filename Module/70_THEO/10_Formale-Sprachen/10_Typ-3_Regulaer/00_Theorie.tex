\documentclass{lehramt-informatik-haupt}
\liLadePakete{mathe,syntaxbaum,automaten,formale-sprachen}
\usepackage{tikz}
\usetikzlibrary{shapes.geometric,calc}

\begin{document}

%%%%%%%%%%%%%%%%%%%%%%%%%%%%%%%%%%%%%%%%%%%%%%%%%%%%%%%%%%%%%%%%%%%%%%%%
% Theorie-Teil
%%%%%%%%%%%%%%%%%%%%%%%%%%%%%%%%%%%%%%%%%%%%%%%%%%%%%%%%%%%%%%%%%%%%%%%%

\chapter{Reguläre Sprachen}

%-----------------------------------------------------------------------
%
%-----------------------------------------------------------------------

\section{Grundbegriffe}

\begin{description}
\item[Alphabete]

Die Symbole, aus denen Eingabewörter bestehen können, fassen wir in
einer Menge, dem Eingabealphabet, zusammen. Dabei wollen wir nur
endliche Eingabealphabete zulassen. Ihre Elemente heißen Symbole oder
Buchstaben. Zur Bezeichnung von Eingabealphabeten wird zumeist das
Symbol $\Sigma$ (Sigma) benutzt.
\footcite[Seite 15]{vossen}

\item[Wörter]
Die endlich langen Zeichenfolgen, die über einem Alphabet $\Sigma$
gebildet werden können, heißen Wörter über $\Sigma$. Wörter entstehen,
indem Symbole oder bereits erzeugte Wörter aneinandergereiht
(miteinander verkettet, konkateniert) werden.

$\Sigma^*$, die Menge aller Wörter, die über dem Alphabet $\Sigma$
gebildet werden kann, ist wie folgt definiert:

\begin{enumerate}
\item Jeder Buchstabe $a \in \Sigma$ ist auch ein Wort über $\Sigma$,
\dh $a \in \Sigma^*$.

\item Werden bereits konstruierte Wörter hintereinandergeschrieben,
entstehen neue Wörter, \dh sind $v, w \in \Sigma^*$, dann ist auch ihre
Verkettung (Konkatenation) $v, w \in \Sigma^*$ .

\item $\varepsilon$ (epsilon), das leere Wort, ist ein Wort über (jedem
Alphabet) $\Sigma$, \dh es gilt immer $\varepsilon \in \Sigma^*$.
$\varepsilon$ ist ein definiertes Wort ohne „Ausdehnung“. Es hat die
Eigenschaft: $\varepsilon w = w \varepsilon = w$ für alle $w \in
\Sigma^*$. Manchmal wird für das leere Wort auch das Symbol $\lambda$
(lambda) verwendet.
\end{enumerate}

Mit $\Sigma^+$ bezeichnen wir die Menge aller Wörter über $\Sigma$ ohne
das leere Wort, \dh $\Sigma^+ = \Sigma^* - \{ \varepsilon \}$.
\footcite[Seite 16]{vossen}
\end{description}

\noindent
Sei $\Sigma$ ein Alphabet. Eine formale Sprache $L$ ist eine Teilmenge
aller Wörter über $\Sigma$:

\begin{displaymath}
L \subseteq \Sigma^*
\end{displaymath}

%-----------------------------------------------------------------------
%
%-----------------------------------------------------------------------

\section{Grammatik}

Eine Grammatik ist ein 4-Tupel mit \liGrammatik{} und besteht aus:

\begin{itemize}
\item Einer endlichen Menge $V$ von \memph{Variablen} (Nonterminale)

\item Dem endlichen \memph{Terminalalphabet} $\Sigma$ mit $\Sigma \cap V
= \emptyset$

\item Der endlichen Menge an \memph{Produktionen}

\item Und einer \memph{Startvariablen} $S$ mit $S \in V$
\footcite[Seite 17]{theo:fs:1}
\end{itemize}

%%
%
%%

\begin{liKasten}
\subsection{Beispiel}

\liGrammatik{} mit

\begin{itemize}
\item $V = \{Z, A\}$

\item \liAlphabet{0, 1}

\item \begin{liProduktionsRegeln}
Z -> 1 A | 1,
A -> 1 A | 0 A | 1
\end{liProduktionsRegeln}
\end{itemize}

\bigskip

oder

\bigskip
\noindent
\liGrammatik{variablen={Z, A}, alphabet={0, 1}, produktionen={Z -> 1 A | 1,
A -> 1 A | 0 A | 1}}
\end{liKasten}

Eine reguläre Sprache wird durch eine reguläre Grammatik erzeugt, \dh
eine Grammatik mit Produktionsregeln der Form:

\liProduktionen{A -> EPSILON} oder
\liProduktionen{A -> 1} oder
\liProduktionen{A -> 0 A}

\begin{description}
\item[linke Seite:]

ein Nonterminal

\item[rechte Seite:]

$\varepsilon$ oder ein Terminal oder ein Terminal gefolgt von einem
Nonterminal
\end{description}

Man spricht hierbei auch von rechtslinearer Grammatik (es gibt auch
linkslineare Grammatiken)\footcite[Seite 18]{theo:fs:1}

%%
%
%%

\subsection{Syntax-Baum}

Gegebene Grammatik: \liGrammatik{variablen={Z, A}, alphabet={0, 1}, produktionen={Z -> 1 A | 1,
A -> 1 A | 0 A | 1}}

\noindent
Syntaxbaum zum Wort „100101“:

\begin{center}
\begin{tikzpicture}[level distance=0.7cm]
\Tree [.Z
  [.1 ] [.A
    [.0 ] [.A
      [.0 ] [.A
        [.1 ] [.A
          [.0 ] [.A
            [.1 ]
          ]
        ]
      ]
    ]
  ]
]
\end{tikzpicture}
\end{center}

%-----------------------------------------------------------------------
%
%-----------------------------------------------------------------------

\section{Reguläre Grammatik $\rightarrow$ NEA}

Wir können aus einer regulären Grammatik einen NEA machen: Jedes
Non-Terminal (Variable) wird zu einem Zustand konvertiert. Dazu kommt
der Zustand $q_{F}$, welcher der Endzustand ist. Existiert in der
Grammatik eine Regel $S \rightarrow \varepsilon$, wobei $S$ Startzustand
ist, so ist $S$ auch ein Endzustand. Dadurch wird erreicht, dass auch
das leere Wort akzeptiert wird. Wenn es eine Regel $A \rightarrow a$
gibt, d. h. eine Regel auf deren rechten Seite nur ein Terminal zu
finden ist, dann erzeugen wir eine Übergangsfunktion $\delta$ (zeichnen
einen Pfeil) von dem Zustand $A$ nach $q_{F}$ mit der Inschrift $a$.
Wenn es eine Regel $A\rightarrow aB$ gibt, dann erzeugen wir eine
Übergangsfunktion (zeichnen einen Pfeil) von dem Zustand $A$ in den
Zustand $B$ mit der Inschrift $a$.
\liFussnoteUrl{http://www.informatikseite.de/theorie/node50.php}

\subsection{Eindeutige Grammatik}

Eine Grammatik $G$ ist dann eindeutig, wenn für jedes Wort aus $L(G)$
genau eine mögliche Ableitung aus dem Startsymbol existiert.
\liFussnoteUrl{http://u-helmich.de/inf/kursQ21/27/folge-27-2.html}

Eine Sprache $L$ heißt eindeutig, wenn es für $L$ eine eindeutige
Grammatik gibt. Ansonsten heißt $L$ mehrdeutig.
\liFussnoteUrl{http://wwwmayr.in.tum.de/lehre/2008WS/ads-ei/2008-11-24.pdf}

%-----------------------------------------------------------------------
%
%-----------------------------------------------------------------------

\section{Äquivalenzklassen (nach Myhill, Nerode)
\footcite[Seite 60]{theo:fs:1}}

Zwei Wörter $x, y \in \Sigma^*$ sind äquivalent bzgl. einer Sprache $L$,
wenn für alle $z \in \Sigma^*$ gilt:

\begin{displaymath}
x z \in L \Leftrightarrow y z \in L
\end{displaymath}
\footcite{wiki:nerode-relation}

Eine Sprache $L$ ist genau dann regulär, wenn sie \memph{endlich viele
Äquivalenzklassen} erzeugt.\footcite{wiki:satz-myhill-nerode}

Die \memph{Anzahl der Äquivalenzklassen} einer Sprache entspricht der
\memph{minimalen Anzahl von Zustanden} eines deterministen-endlichen
Automaten, der diese Sprache erkennt. Darauf fußt der
Minimalisierungsalgorithmus.

\begin{liKasten}
\subsection{Beispiel zu Äquivalenzklassen}

$(1|11|111)0^*$ sind alle Binärzahlen, die mit einer, zwei oder drei
Einsen beginnen, gefolgt von beliebig vielen Nullen.

\begin{enumerate}
\item Sind $x=1$ und $y=10$ in der selben Äquivalenzklasse?

Nein, denn anfügen von $z=1$:

\begin{itemize}
\item $xz = 11$ erlaubtes Wort

\item $yz = 101$ unerlaubtes Wort
\end{itemize}

\item Sind $x=1$ und $y=11$ in der selben Äquivalenzklasse?

Nein, denn anfügen von $z=11$:

\begin{itemize}
\item $xz = 111$ erlaubtes Wort
\item $yz = 1111$ unerlaubtes Wort
\end{itemize}

\item Sind $x=10$ und $y=110$ in der selben Äquivalenzklasse?

Ja, fügt man nur $0$en an, sind die Wörter in $L$, fügt man auch $1$en
an, sind sie nicht in $L$.
\end{enumerate}
\end{liKasten}

\subsection{Äquivalente Zustände
\footcite[Seite 62]{theo:fs:1}}

Zwei Zustände $z$, $z^\prime$ eines deterministischen Automaten sind
äquivalent, wenn für alle möglichen Eingaben der Automat entweder in
\memph{beiden Fallen} in einen (nicht notwendig gleichen)
\memph{Endzustand} oder in beiden Fallen in einen (nicht notwendig
gleichen) \memph{Nicht-Endzustand} übergeht.
%
Äquivalente Zustände können \memph{zu einem Zustand verschmolzen}
werden.

%-----------------------------------------------------------------------
%
%-----------------------------------------------------------------------

\section{Abschlusseigenschaften reguläre Sprachen}

\begin{description}
\item[Vereinigung:]

Die Vereinigung $L = L_1 \cup L_2$ zweier regulärer Sprachen $L_1$ und
$L_2$ ist regulär.

\item[Schnitt:]

Der Schnitt $L = L_1 \cup L_2$ zweier regulärer Sprachen $L_1$ und $L_2$
ist regulär.

\item[Komplement:]

Das Komplement $\bar L = \Sigma^* \setminus L$ einer regulären Sprache
$L$ ist regulär.

Sei $A$ ein deterministischer endlicher Automat, der $L$
erkennt. Der Automat $A$ erreicht für jedes Wort $\omega$ Element $L$
einen Endzustand und für jedes Wort $\omega$ nicht Element L einen
Nicht-Endzustand. Indem in $A$ alle Endzustände zu Nicht-Endzuständen
gemacht werden und umgekehrt, entsteht ein deterministischer endlicher
Automat $A$, der $L$ erkennt. Also ist $L$ regulär.
\liFussnoteUrl{https://www.inf.hs-flensburg.de/lang/theor/regulaer-abgeschlossen.htm}

\section{Komplement}

Um zu zeigen, dass für jede reguläre Sprache $L$ auch das Komplement $L
= \Sigma^* \string\ L$ regulär ist, gehen wir in zwei Schritten vor: Im
ersten Schritt konstruieren wir einen endlichen Automaten $A$ mit $L (A)
= L$. Dass ein solcher Automat für jede reguläre Sprache existieren
muss, haben wir im vorherigen Abschnitt herausgearbeitet. Im zweiten
Schritt konstruieren wir aus $A$ den Komplementärautomaten $A$, der die
Sprache $L (A)$ erzeugt. Dass der Komplementärautomat die Sprache $L
(A)$ erzeugt, ist leicht einzusehen. Da wir die Menge der Endzustände
invertiert haben, wird ein Wort $\omega$ genau dann von $A$ akzeptiert,
wenn es von $A$ zurückgewiesen wird. Mit $L$ wird damit immer auch $L$
von einem Automaten akzeptiert, so dass die Menge der regulären Sprachen
in Bezug auf das Komplement abgeschlossen ist.\footcite[Seite
218-219]{hoffmann}

\item[Produkt:]

Das Produkt \liAusdruck[]{uv}{u \in L_1 \lor v \in L_2} zweier regulärer
Sprachen $L_1$ und $L_2$ ist regulär.

\item[Kleene-Stern:]

Der Kleene-Stern $L^*$ einer regulären Sprache $L$, \dh die beliebig
häufige Konkatenation von Wörter aus der Sprache $L$ vereinigt mit dem
leeren Wort, ist regulär.
\footcite[Seite 68]{theo:fs:1}
\end{description}

%-----------------------------------------------------------------------
%
%-----------------------------------------------------------------------

\section{Entscheidungsprobleme}

\begin{description}

\item[Wortproblem:]
$\Rightarrow$ für
reguläre Sprachen entscheidbar!

\item[Leerheitsproblem:]
$\Rightarrow$ für reguläre Sprachen entscheidbar!

\item[Endlichkeitsproblem:]
$\Rightarrow$ für reguläre Sprachen entscheidbar!

\item[Äquivalenzproblem:]

$\Rightarrow$ für reguläre Sprachen entscheidbar!
\footcite[Seite 70-71]{theo:fs:1}
\end{description}

\literatur

\end{document}
