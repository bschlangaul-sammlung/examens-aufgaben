\documentclass{bschlangaul-theorie}


\begin{document}

%%%%%%%%%%%%%%%%%%%%%%%%%%%%%%%%%%%%%%%%%%%%%%%%%%%%%%%%%%%%%%%%%%%%%%%%
% Theorie-Teil
%%%%%%%%%%%%%%%%%%%%%%%%%%%%%%%%%%%%%%%%%%%%%%%%%%%%%%%%%%%%%%%%%%%%%%%%

\section{Minimierungsalgorithmus\footcite[Seite 47-57]{vossen}}

\footcite[Seite 51-62]{theo:fs:1}

Wir erstellen eine \memph{Zustandspaartabelle} mit allen möglichen
Zustandspaaren. Dabei spielt es \memph{keine Rolle}, ob die
Zustände im Automaten \memph{verbunden sind oder nicht}.

Die Diagonale der Tabelle wird komplett gestrichen, da Zustände nicht
mit sich selbst überprüft werden müssen. Außerdem werden die
\memph{Paare nur in einer Richtung} betrachtet. Die \memph{obere Hälfte}
der Tabelle über der Diagonalen \memph{kann gestrichen werden}.

Wir markieren die Zustandspaare, in denen ein Zustand \memph{ein
Endzustand und der andere kein Endzustand} ist.

Als letztes werden mit Hilfe einer \memph{Übergangstabelle} die noch
nicht markierten Zustandspaare auf alle möglichen Übergangsmöglichkeiten
überprüft. Entsteht hierbei ein \memph{bereits gestrichenes Paar} so
wird das aktuell überprüfte Paar \memph{ebenfalls gestrichen}.

Wir testen jetzt für jedes Zustandspaar die Folgezustände bei der
Eingabe der Zeichen des Alphabets.

Wir suchen uns nun die Zustandspaare, die bei einer Eingabe ein
Zustandspaar ergeben, das bereits gestrichen ist.

Im Folgenden wird der Algorithmus zur Markierung der nicht-äquivalenten
Zustände anhand einer Dreieckstabelle durchgeführt, die alle fraglichen
Zustandspaare enthält.

Minimierungstabelle (Table filling)

Der Algorithmus trägt in seinem Verlauf eine Markierung in alle
diejenigen Zellen der Tabelle ein, die zueinander nicht äquivalente
Zustände bezeichnen. Die Markierung „$x_n$“ in einer Tabellenzelle ($i$,
$j$) bedeutet dabei, dass das Zustandspaar ($i$, $j$) in der $k$-ten
Iteration des Algorithmus markiert wurde und die Zustände $i$ und $j$
somit zueinander ($k - 1$)-äquivalent, aber nicht $k$-äquivalent und
somit insbesondere nicht äquivalent sind. Bleibt eine Zelle bis zum Ende
unmarkiert, sind die entsprechenden Zustände zueinander äquivalent.
\footcite[Seite 19]{koenig}

\bFussnoteUrl{https://studyflix.de/informatik/dea-minimieren-1212}

\literatur

\end{document}
