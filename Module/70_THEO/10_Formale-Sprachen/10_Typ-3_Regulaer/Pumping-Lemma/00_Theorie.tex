\documentclass{lehramt-informatik-haupt}
\liLadePakete{mathe}

\begin{document}

\chapter{Pumping-Lemma für reguläre Sprachen}

%
\noindent
Sei $L$ eine reguläre Sprache. Dann gibt es eine Zahl $j$, sodass für
alle Wörter $\omega \in L$  mit $|\omega| \geq j$ (jedes Wort $\omega$
in $L$ mit Mindestlänge $j$) jeweils eine Zerlegung $\omega = uvw$
existiert, sodass die folgenden Eigenschaften erfüllt sind:

\begin{enumerate}
\item $|v| \geq 1$
(Das Wort $v$ ist nicht leer.)

\item $|uv| \leq j$
(Die beiden Wörter $u$ und $v$ haben zusammen höchstens die Länge $j$.)

\item Für alle $i = 0, 1, 2, \dots$ gilt $uv^iw \in L$
(Für jede natürliche Zahl (mit 0) $i$ ist das Wort $uv^{i}w$ in der
Sprache $L$)
\end{enumerate}

Die kleinste Zahl $j$, die diese Eigenschaften erfüllt, wird
Pumping-Zahl der Sprache $L$ genannt.\footcite{wiki:pumping-lemma}

Die einzelnen Bestandteile der Zerlegung des Wortes $\omega$ heißen
Anfangsteil $u$, Endteil $w$ und Schleifenteil $v$.%
\liFussnoteUrl{https://studyflix.de/informatik/pumping-lemma-1445}

\noindent
Das Pumping-Lemma wird verwendet, um zu zeigen, dass eine
Sprache nicht regulär ist (Widerspruchsbeweis).\footcite[Seite 63]{theo:fs:1}

%%
%
%%

\subsection{Beispiel $L = \{a^n b^n | n \in \mathbb{N})$}

Ich behaupte, $L$ sei regulär.

\begin{enumerate}
\item Also gibt es eine Pumpzahl. Sie sei $j$.

\item (Wähle geschickt ein „langes“ Wort...)
$a^j b^j$ ist ein Wort aus $L$, das sicher länger als $j$ ist.

\item Da $L$ regulär ist, muss es nach dem Pumping-Lemma auch für dieses
Wort eine Zerlegung geben:
\end{enumerate}

\begin{center}
$a^j b^j = uvw$ mit $|v| \geq 1$ und $|uv| \leq j$
\end{center}

\noindent
Weil $uv$ höchstens $j$ lang ist, kann es im Fall von $a^j b^j$ nur aus
$a$‘s bestehen. Da $v$ mindestens ein Zeichen enthält, ist das
mindestens ein $a$. Pumpen führt nun zu mehr $a$‘s als $b$‘s und also zu
einem Wort, das nicht in der Sprache ist. (Widerspruch!)
\footcite{wiki:pumping}

$\Rightarrow$ Die Behauptung war falsch!

$\Rightarrow$ $L$ ist nicht regulär!
\footcite[Seite 63-64]{theo:fs:1}

hier ausführlich beschrieben https://www.informatik.hu-berlin.de/de/forschung/gebiete/algorithmenII/Lehre/ws13/einftheo/einftheo-skript.pdf

\literatur

\end{document}
