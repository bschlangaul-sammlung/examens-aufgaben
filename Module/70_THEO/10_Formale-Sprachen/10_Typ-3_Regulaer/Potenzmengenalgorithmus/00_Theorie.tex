\documentclass{bschlangaul-theorie}
\liLadePakete{mathe,syntaxbaum,automaten,formale-sprachen}

\begin{document}

%%%%%%%%%%%%%%%%%%%%%%%%%%%%%%%%%%%%%%%%%%%%%%%%%%%%%%%%%%%%%%%%%%%%%%%%
% Theorie-Teil
%%%%%%%%%%%%%%%%%%%%%%%%%%%%%%%%%%%%%%%%%%%%%%%%%%%%%%%%%%%%%%%%%%%%%%%%

%-----------------------------------------------------------------------
%
%-----------------------------------------------------------------------

\section{Potenzmengenalgorithmus NEA $\rightarrow$ DEA\footcite[Seite 35-47]{theo:fs:1}}

\begin{itemize}
\item Starte im Anfangszustand (in der Menge der Anfangszustände).

\item Gib für jedes Zeichen die Menge der erreichbaren Zustände an.

\item Wiederhole diesen Schritt für jede neu erreichte Menge an
Zuständen.

\item Die Zustandsmengen sind die Zustände des DEA.

\item Mengen, die „alte“ Endzustände enthalten, sind Endzustände des
neuen DEA.\footcite{wiki:potenzmengenkonstruktion}
\end{itemize}

%-----------------------------------------------------------------------
%
%-----------------------------------------------------------------------

\section{Erweiterter Potenzmengenalgorithmus $\varepsilon$-NEA zum
DEA\footcite[Seite 48-49]{theo:fs:1}}

Wir wählen als \memph{neuen Anfangszustand} den alten Anfangszustand und
alle Zustände aus, die vom alten aus mit \memph{$\varepsilon$
erreichbar} sind. Nun führen wir den Potenzmengenalgorithmus mit diesem
neuen, möglicherweise aus mehreren Zuständen sich zusammensetztend
Startzustand aus. In jeden weiteren Schritt des Algortihmus fügen wir
auch die \memph{Zustände hinzu}, die über einen oder mehrere zusätzliche
$\varepsilon$-Übergänge erreichbar sind. Wir wandern unter Umständen
über mehrer Zustände hinweg.

\literatur

\end{document}
