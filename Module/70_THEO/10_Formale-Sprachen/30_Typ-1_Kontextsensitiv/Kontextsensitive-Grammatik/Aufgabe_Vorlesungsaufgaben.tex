\documentclass{bschlangaul-aufgabe}
\bLadePakete{formale-sprachen,mathe}
\begin{document}
\bAufgabenTitel{Vorlesungsaufgaben kontextsensitive Grammatiken}
\section{Übung kontextsensitive Grammatiken
\index{Kontextsensitive Grammatik}
\footcite\footcite[Seite 8]{theo:fs:3}}

% Info_2021-04-23-2021-04-23_09.35.52.mp4 37min

Gegen sei folgende Grammatik:

\bGrammatik{} mit
$V = \{S, B, C\}$,
\bAlphabet{a, b, c},
$S=S$ und

\begin{liProduktionsRegeln}
S -> aSBC | aBC,
CB -> BC,
aB -> ab,
bB -> bb,
bC -> bc,
cC -> cc,
\end{liProduktionsRegeln}

\begin{enumerate}

%%
%
%%

\item Geben Sie die Sprache an, die die folgende Grammatik erzeugt:

\begin{bAntwort}
\bAusdruck{a^n b^b c^n}{n \in \mathbb{N}}
\end{bAntwort}

%%
%
%%

\item Gib Sie eine Ableitung mit der folgenden Grammatik für das Wort
aaabbbccc an.

% Info_2021-04-23-2021-04-23_09.35.52.mp4 54min

\begin{bAntwort}
\bAbleitung{aSBC ->
aaSBCBC ->
aaaBCBCBC ->
aaaBBCCBC ->
aaaBBCBCC ->
aaaBBBCCC ->
aaabBBCCC ->
aaabbBCCC ->
aaabbbCCC ->
aaabbbcCC ->
aaabbbccC ->
aaabbbccc}
\end{bAntwort}

\end{enumerate}
\end{document}
