\documentclass{bschlangaul-aufgabe}
\liLadePakete{syntax}
\begin{document}
\liAufgabenTitel{LOOP-Fakultät}
\section{
\index{LOOP-berechenbar}
\footcite[Aufgabe 1]{theo:ab:4}}

\begin{enumerate}

%%
%
%%

\item Geben Sie ein LOOP-Programm an, das die Funktion $f(n) = n!$
berechnet.

\begin{liAntwort}
\begin{minted}{md}
x_2 := 1;
LOOP x1 DO
  x_3 := x_3 + 1;
  x_2 := x_2 * x_3;
END
x_3 := 0;
RETURN x_2;
\end{minted}
\end{liAntwort}

%%
% (b)
%%

\item

Beweisen Sie:

Ist $f : N \rightarrow N$ LOOP-berechenbar, so ist auch
$g : N \rightarrow N$ mit $g(n) = f (i)$
LOOP-berechenbar.

\begin{liAntwort}
Bei einem LOOP-Programm der Form LOOP $x_i$ DO P END wird das Programm
$P$ so oft ausgeführt, wie der Wert der Variablen $x_i$ zu Beginn
angibt. Beweis:

\begin{minted}{md}
x_0 := 0;
i := 0;
LOOP n DO
  i := i + 1;
  y := f(i);
  x_0 := x_0 + y;
END
RETURN x_0;
\end{minted}

ist LOOP-berechenbar, da $f(n)$ LOOP-berechenbar ist.
\end{liAntwort}

\end{enumerate}
\end{document}
