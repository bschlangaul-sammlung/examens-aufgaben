\documentclass{lehramt-informatik-aufgabe}
\liLadePakete{syntax,mathe}
\begin{document}
\liAufgabenTitel{Vorlesungsaufgaben}
\section{Vorlesungsaufgaben WHILE-Programm
\index{Berechenbarkeit}
\footcite[Seite 16]{theo:fs:4}}

Geben Sie ein WHILE-Programm an, dass

% Info_2021-05-07-2021-05-07_09.32.08.mp4 2h29

\begin{itemize}
\item $2^{x_i}$

\begin{liAntwort}
Ausnutzen der 2er-Potenzeigenschaft:

$2^1 = 1 + 1 = 2$

$2^2 = 2 + 2 = 4$

$2^3 = 4 + 4 = 8$

Hier werden nur die elementaren Bestandteile der WHILE-Sprache ausgenutzt.

\begin{minted}{md}
x_2 := 1;
WHILE x_1 DO
  x_3 := x_2;
  WHILE x_3 DO
    x_2 := x_2 + 1;
    x_3 := x_3 - 1;
  END
  x_1 := x_1 - 1;
END
x_0 := x_2;
\end{minted}
\end{liAntwort}

\item $\text{ggt}(x_i, x_j)$

\begin{liAntwort}
Zusätzliche Voraussetzungen:

\begin{minted}{md}
x_1 > x_2;
MOD(x_1, x_2);
\end{minted}

\begin{minted}{md}
x_0 := MOD(x_1, x_2);
WHILE x_0 DO
  x_1 := x_2;
  x_2 := x_0;
  x_0 := MOD(x_1, x_2);
END
\end{minted}
\end{liAntwort}
\end{itemize}

berechnet.

\end{document}
