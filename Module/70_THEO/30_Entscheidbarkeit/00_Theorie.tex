\documentclass{bschlangaul-theorie}
\bLadePakete{mathe}

\begin{document}

%%%%%%%%%%%%%%%%%%%%%%%%%%%%%%%%%%%%%%%%%%%%%%%%%%%%%%%%%%%%%%%%%%%%%%%%
% Theorie-Teil
%%%%%%%%%%%%%%%%%%%%%%%%%%%%%%%%%%%%%%%%%%%%%%%%%%%%%%%%%%%%%%%%%%%%%%%%

\chapter{Entscheidbarkeit}

\begin{liQuellen}
\item \cite[Seite 309-312]{hoffmann}
\item \cite{wiki:entscheidbar}
\item \cite[Kapitel 19.2.3.1 Entscheidbarkeit und Semientscheidbarkeit, Seite 596-597]{schneider}
\end{liQuellen}

\begin{description}
\item[Entscheidbarkeit]

Eine Sprache $L \subseteq \Sigma^*$ heißt entscheidbar, wenn die
charakteristische Funktion $\chi_{L} : \Sigma^* \rightarrow \{0,1\}$ mit

\begin{equation*}
\chi _{L}(\omega)=
\begin{cases}
1,& \text{falls } \omega \in L,\\
0,& \text{falls } \omega \notin L
\end{cases}
\end{equation*}

berechenbar ist.

\item[Semi-Entscheidbarkeit]

Eine Sprache $L \subseteq \Sigma^*$ heißt semi-entscheidbar, wenn die
partielle charakteristische Funktion $\chi_{L}' : \Sigma^* \rightarrow
\{ 1 \}$ mit

\begin{equation*}
\chi _{L}'(\omega)=
\begin{cases}
1,& \text{falls } \omega \in L,\\
\bot,& \text{falls } \omega \notin L
\end{cases}
\end{equation*}

berechenbar ist.
\end{description}

\begin{liExkurs}[Charakteristische Funktion]
Die Indikatorfunktion (auch charakteristische Funktion genannt) ist eine
Funktion in der Mathematik, die sich dadurch auszeichnet, dass sie
nur einen oder zwei Funktionswerte annimmt. Sie ermöglicht es,
komplizierte Mengen mathematisch präzise zu fassen.
\footcite{wiki:charakteristische-funktion}
\end{liExkurs}

\section{Wichtige Definitionen}

Jede nichtdeterministische Turing-Maschine kann durch eine
deterministische Turing-Maschine simuliert werden.

Zu jeder Typ-0-Sprache $L$ existiert eine Turing-Maschine, die $L$
akzeptiert.

Zu jeder Typ-1-Sprache $L$ existiert eine nichtdeterministische, linear
beschränkte Turing-Maschine, die $L$ akzeptiert.
\footcite[Seite 35]{theo:fs:4}

%-----------------------------------------------------------------------
%
%-----------------------------------------------------------------------

\section{Sprache und Abzählbarkeit}

Eine Sprache $L$ heißt

\begin{description}
\item[rekursiv aufzählbar,] falls

\begin{itemize}
\item eine Turing-Maschine $T$ existiert, die $L$ akzeptiert oder

\item $L = \emptyset$ oder

\item eine surjektive und berechenbare Abbildung $f: \mathbb{N}
\rightarrow L$ existiert.
\end{itemize}

\item[abzählbar,] falls eine bijektive Abbildung $f: \mathbb{N}
\rightarrow L$ existiert.

\item[rekursiv oder entscheidbar,] falls eine
Turing-Maschine $T$ existiert, die $L$ akzeptiert und zusätzlich für
jede Eingabe terminiert.
\end{description}
\footcite[Seite 37]{theo:fs:4}

\begin{liExkurs}[Abzählbarkeit / Abzählbare Menge]
In der Mengenlehre wird eine Menge $A$ als abzählbar unendlich
bezeichnet, wenn sie die gleiche Mächtigkeit hat wie die Menge der
natürlichen Zahlen $\mathbb{N}$. Dies bedeutet, dass es eine Bijektion
zwischen $A$ und der Menge der natürlichen Zahlen gibt, die Elemente der
Menge $A$ also „durchnummeriert“ werden können.
\bFussnoteUrl{https://de.wikipedia.org/wiki/Abzählbare_Menge}
\end{liExkurs}

%-----------------------------------------------------------------------
%
%-----------------------------------------------------------------------

\section{Entscheidbarkeit - Feststellungen}

Eine Sprache $L$ ist genau dann entscheidbar, wenn sowohl $L$ als auch
$\overline{L}$ semi-entscheidbar sind.

Eine Sprache ist genau dann aufzählbar, wenn sie semi-entscheidbar ist.

Eine Sprache ist genau dann entscheidbar, wenn $L$ und $\overline{L}$
aufzählbar sind.
\footcite[Seite 37]{theo:fs:4}

Die Klasse der Typ-0-Sprachen ist mit der von allg. Turing-Maschinen
akzeptierten Sprachen identisch.

Die Klasse der kontextsensitiven (Typ-1) Sprachen ist mit der Klasse der
Sprachen identisch, die von linear beschränkten Turing-Maschinen
akzeptiert werden.
\footcite[Seite 38]{theo:fs:4}

%-----------------------------------------------------------------------
%
%-----------------------------------------------------------------------

\section{Unentscheidbarkeit}

Zu jedem Alphabet $\Sigma$ gibt es (unendlich viele) unentscheidbare
Sprachen.

Dieser Satz zeigt uns harte theoretische Grenzen auf!
\footcite[Seite 39]{theo:fs:4}

• Allgemeines Halteproblem:

Gegeben:
Gefragt:
Turingmaschine 𝑇 und das Eingabewort 𝜔
Terminiert 𝑇 unter der Eingabe 𝜔 ?
• Beweis von Turing (1936):
Das allgemeine Halteproblem ist unentscheidbar.

Beweis Halteproblem

Wir listen alle Wörter der Sprache in einer Tabelle in den Spalten auf.
Für jedes Wort 𝜔 𝑖 geben wir an, ob die Turing-Maschine 𝑇 𝑗 hält

Wir nehmen an, dass das Halteproblem entscheidbar ist.
\footcite[Seite 41]{theo:fs:4}

Dann würde eine TM H existieren, die für ein Paar (𝑇, 𝜔)
entscheidet, T bei der Eingabe von 𝜔 terminiert.
Zudem würde eine TM H‘ existieren,
• die das Element (i,j) analog zu H bestimmt
• gilt (i,j) = ja, dann geht H‘ in eine Endlosschleife über,
terminiert also nie!
• gilt (i,j) = nein, terminiert H‘ in einem Endzustand.
Da H‘ eine TM ist, muss sie in der Tabelle vorkommen.
Aber: Keine Zeile der Tabelle entspricht der TM H‘
Widerspruch zur Annahme der Existenz von H‘
Widerspruch zur Annahme der Entscheidbarkeit.
\footcite[Seite 42]{theo:fs:4}

\section{Satz von Rice:}

Mit $E$ sei eine nichttriviale funktionale Eigenschaft von
Turingmaschinen gegeben.\footcite[Seite 324]{hoffmann}

Dann ist das folgende Problem unentscheidbar:
\begin{description}
\item[Gegeben:] Turingmaschine T

\item[Gefragt:] Besitzt T die Eigenschaft E?
\end{description}
\footcite[Seite 43]{theo:fs:4}

%-----------------------------------------------------------------------
%
%-----------------------------------------------------------------------

\section{Reduktionen:}

$L_1$ heißt reduzierbar auf $L_2$ ($L_1 \leq L_2$), wenn es eine totale,
berechenbare Funktion $f \colon \Sigma^* \rightarrow \Gamma^*$ gibt, so
dass gilt:

\begin{displaymath}
\omega \in L_1 \Leftrightarrow f(\omega) \in L_2
\end{displaymath}

Wenn man $L_2$ lösen könnte, dann kann man auch $L_1$ lösen. Damit $L_2$
mindestens so unentscheidbar ist wie $L_1$, nimm für $L_1$ ein bekanntes
unentscheidbares Problem!
\footcite[Seite 44]{theo:fs:4}

\subsection{Vorgehen Reduktionsbeweise:}

Es gibt eine Funktion, die Instanzen von $L_1$ auf Instanzen von $L_2$
„umbaut“.

\begin{itemize}
\item Die Funktion ist total.
\item Die Funktion ist berechenbar.
\item Korrektheit der Reduktion.
\end{itemize}
\end{document}
