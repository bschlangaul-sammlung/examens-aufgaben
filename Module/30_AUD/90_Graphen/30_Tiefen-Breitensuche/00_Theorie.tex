\documentclass{lehramt-informatik-haupt}
\liLadePakete{graph,syntax}

\begin{document}

%%%%%%%%%%%%%%%%%%%%%%%%%%%%%%%%%%%%%%%%%%%%%%%%%%%%%%%%%%%%%%%%%%%%%%%%
% Theorie-Teil
%%%%%%%%%%%%%%%%%%%%%%%%%%%%%%%%%%%%%%%%%%%%%%%%%%%%%%%%%%%%%%%%%%%%%%%%

\chapter{Tiefensuche, Breitensuche}

\section{Tiefesuche}

\begin{liQuellen}
\item \cite[Seite 39-52 (PDF 32-45)]{aud:fs:6}
\item \cite{wiki:tiefensuche}
\item \cite[Kapitel 6.2.2.2 Graphalgorithmen, Seite 185]{schneider}
\end{liQuellen}

Die Tiefensuche (englisch \memph{depth-first search}, \memph{DFS}) ist
in der Informatik ein Verfahren zum Suchen von Knoten in einem Graphen.

Der Tiefendurchlauf ist das Standardverfahren zum Durchlaufen eines
Graphen bei dem \memph{jeder Knoten mindestens einmal} und \memph{jede
Kante genau einmal} besucht wird. Man geht vom jeweiligen Knoten
\memph{erst zu einem nicht besuchten Nachbarknoten} und setzt den
Algorithmus dort \memph{rekursiv} fort. Bei schon besuchten Knoten wird
abgebrochen. Als Hilfsstruktur wird ein \memph{Stack} (Stapelspeicher,
Keller) verwendet. Der konkrete Durchlauf hängt von der Reihenfolge der
Knoten in den Adjazenzlisten bzw. in der Adjazenzmatrix ab.
\footcite[Seite 40]{aud:fs:6}

\begin{liGraphenFormat}
A: 1 5
B: 0 4
C: 0 2
D: 1 0
E: 2 2
F: 3 1
G: 6 2
H: 5 3
I: 4 5
J: 5 5
K: 6 4
L: 7 3
M: 8 0
A -- B
A -- C
A -- E
B -- C
C -- E
C -- I
D -- E
D -- F
F -- G
F -- H
F -- I
F -- M
I -- J
L -- K
M -- L
\end{liGraphenFormat}

\begin{center}
\begin{tikzpicture}[li graph]
\node (A) at (1,5) {A};
\node (B) at (0,4) {B};
\node (C) at (0,2) {C};
\node (D) at (1,0) {D};
\node (E) at (2,2) {E};
\node (F) at (3,1) {F};
\node (G) at (6,2) {G};
\node (H) at (5,3) {H};
\node (I) at (4,5) {I};
\node (J) at (5,5) {J};
\node (K) at (6,4) {K};
\node (L) at (7,3) {L};
\node (M) at (8,0) {M};

\path (A) edge node {} (B);
\path (A) edge node {} (C);
\path (A) edge node {} (E);
\path (B) edge node {} (C);
\path (C) edge node {} (E);
\path (C) edge node {} (I);
\path (D) edge node {} (E);
\path (D) edge node {} (F);
\path (F) edge node {} (G);
\path (F) edge node {} (H);
\path (F) edge node {} (I);
\path (F) edge node {} (M);
\path (I) edge node {} (J);
\path (L) edge node {} (K);
\path (M) edge node {} (L);
\end{tikzpicture}
\end{center}

\def\TmpBesuch#1{\strut\par\textbf{Besuchte Knoten:} #1}
\def\TmpStack#1{\par\textbf{Stack:} #1}

Startknoten A:

\begin{description}

\item[A:]
\TmpBesuch{A, B}
\TmpStack{A}

\item[B:]
\TmpBesuch{A, B, C}
\TmpStack{A, B}

\item[C:]
\TmpBesuch{A, B, C, E}
\TmpStack{A, B, C}

\item[B:]
A, B, C
\TmpStack{A, B, C}

\item[C:]
A, B, C, E
\TmpStack{A, B, C, E}

\item[E:]
A, B, C, E, D
\TmpStack{A, B, C, E}

\item[E:]
A und C schon
markiert
A, B, C, E, D
\TmpStack{A, B, C, E, D}

\item[D:]
A, B, C, E, D, F
\TmpStack{A, B, C, E, D}

\item[F:]
A, B, C, E, D, F, M
\TmpStack{A, B, C, E, D, F}

\item[M:]
A, B, C, E, D, F, M, L
\TmpStack{A, B, C, E, D, F, M}

\item[L:]
A, B, C, E, D, F, M, L, K
\TmpStack{A, B, C, E, D, F, M, L}

\item[K:] keine Kinder
A, B, C, E, D, F, M, L, K
\TmpStack{A, B, C, E, D, F, M, L}

\item[L:] alle Kinder schon markiert
A, B, C, E, D, F, M, L, K
\TmpStack{A, B, C, E, D, F, M}

\item[M:] alle Kinder schon markiert
A, B, C, E, D, F, M, L, K
\TmpStack{A, B, C, E, D, F}

\item[F:]
A, B, C, E, D, F, M, L, K, G
\TmpStack{A, B, C, E, D, F}

\item[G:] keine Kinder

\item[F:]
A, B, C, E, D, F, M, L, K, G, H
\TmpStack{A, B, C, E, D, F}

\item[H:] keine Kinder

\item[F:]
A, B, C, E, D, F, M, L, K, G, H, I
\TmpStack{A, B, C, E, D, F}

\item[I:]
A, B, C, E, D, F, M, L, K, G, H, I, J
\TmpStack{A, B, C, E, D, F, I}

\item[J:] keine Kinder
I: alle Kinder markiert
F: alle Kinder markiert
D: alle Kinder markiert
E: alle Kinder markiert
C: alle Kinder markiert
B: alle Kinder markiert
A: alle Kinder markiert

A, B, C, E, D, F, M, L, K, G, H, I, J
\TmpStack{leer}

\end{description}

ENDE

\subsection{Implementierung der Tiefensuche}

\begin{itemize}
\item Eine Möglichkeit, abzuspeichern, welche Knoten bereits besucht
wurden $\rightarrow$ Boolean-Array

\item Eine Methode, die für uns diese Markierung der Knoten als besucht
übernimmt (und somit die eigentliche Tiefensuche durchführt)
$\rightarrow$  Knoten als besucht eintragen, existierende Nachbarknoten
suchen und prüfen, ob diese bereits besucht wurden, falls nicht: diese
durch rekursiven Aufruf besuchen

\item Eine Methode, um die Tiefensuche zu starten $\rightarrow$  Wenn
ein übergebnisener Startknoten existiert, dann müssen erst alle Knoten
als nicht besucht markiert werden und dann vom Startknoten aus das
Besuchen der Knoten gestartet werden
\end{itemize}
\footcite[Seite 45]{aud:fs:6}

\section{Pseudocode: Tiefensuche mit explizitem Stack\footcite[Seite 51 (PDF 45)]{aud:fs:6}}

\begin{minted}{md}
funktion dfs(G: Graph, k: Startknoten in G) {
  S := leerer Stack;
  lege k oben auf S;
  markiere k;

  solange S nicht leer ist fuehre aus {
    a := entferne oberstes Element von S;
    bearbeite Knoten a;

    fuer alle Nachfolger n von a fuehre aus {
      falls n noch nicht markiert fuehre aus {
        lege n oben auf S;
        markiere n;
      }
    }
  }
}
\end{minted}

\liJavaDatei[firstline=34,lastline=74]{graph/algorithmen/TiefenSucheStapel}

%-----------------------------------------------------------------------
%
%-----------------------------------------------------------------------

\section{Breitensuche}

\begin{liQuellen}
\item \cite[Seite 53-64 (PDF 46-57)]{aud:fs:6}
\item \cite[Kapitel 6.2.2.2 Graphalgorithmen, Seite 185]{schneider}
\item \cite{wiki:breitensuche}
\end{liQuellen}

Der Breitendurchlauf (englisch \memph{breadth-first search},
\memph{BFS})\footcite{wiki:breitensuche} ist Verfahren zum Durchlaufen
eines Graphen bei dem jeder Knoten genau einmal besucht wird. Man geht
von einem Knoten \memph{erst zu allen Nachbarknoten} \memph{bevor deren
Nachbarn} besucht werden. Bei schon besuchten Knoten wird abgebrochen.
Als Hilfsstruktur wird eine \memph{Queue (Warteschlange)} verwendet.

Der konkrete Durchlauf hängt von der Reihenfolge der Knoten in den
Adjazenzlisten ab.

\begin{center}
\begin{tikzpicture}[li graph]
\node (A) at (1,5) {A};
\node (B) at (0,4) {B};
\node (C) at (0,2) {C};
\node (D) at (1,0) {D};
\node (E) at (2,2) {E};
\node (F) at (3,1) {F};
\node (G) at (6,2) {G};
\node (H) at (5,3) {H};
\node (I) at (4,5) {I};
\node (J) at (5,5) {J};
\node (K) at (6,4) {K};
\node (L) at (7,3) {L};
\node (M) at (8,0) {M};

\path (A) edge node {} (B);
\path (A) edge node {} (C);
\path (A) edge node {} (E);
\path (B) edge node {} (C);
\path (C) edge node {} (E);
\path (C) edge node {} (I);
\path (D) edge node {} (E);
\path (D) edge node {} (F);
\path (F) edge node {} (G);
\path (F) edge node {} (H);
\path (F) edge node {} (I);
\path (F) edge node {} (M);
\path (I) edge node {} (J);
\path (L) edge node {} (K);
\path (M) edge node {} (L);
\end{tikzpicture}
\end{center}

\begin{description}

\item[A:]
A, B, C, E
w: B, C, E

\item[B:] alle Kinder markiert

A, B, C, E
w: C, E

\item[C:] A, B und E markiert

A, B, C, E, I
w: E, I

\item[E:] A und C markiert

A, B, C, E, I, D
w: I, D

\item[I:] C markiert

A, B, C, E, I, D, F, J
w: D, F, J

\item[D:] E, F sind markiert
A, B, C, E, I, D, F, J
w: F, J

\item[F:] I ist markiert
A, B, C, E, I, D, F, J, M
w: J, M, G, H

\item[J:] keine Kinder

A, B, C, E, I, D, F, J, M,
G, H
w: M, G, H

\item[M:]

A, B, C, E, I, D, F, J, M, G, H, L
w: G, H, L

\item[G:] keine Kinder
A, B, C, E, I, D, F, J, M, G, H, L
w: H, L

\item[H:] keine Kinder
A, B, C, E, I, D, F, J, M, G, H, L
w: L

\item[L:]
A, B, C, E, I, D, F, J, M, G, H, L, K
w: K

\item[K:] keine Kinder
A, B, C, E, I, D, F, J, M, G, H, L, K
w:
\end{description}

ENDE

\section{Pseudocode Breitensuche mit Queue
\footcite[Seite 64 PDF (56)]{aud:fs:6}}

\begin{minted}{md}
funktion bfs(G: Graph, k: Startknoten in G) {
  Q := leere Queue:
  fuege k in Q ein;
  markiere k;

  solange Q nicht leer ist fuehre aus {
    a := entferne vorderstes Element aus Q;
    bearbeite Knoten a;

    fuer alle Nachfolger n von a fuehre aus {
      falls n noch nicht markiert fuehre aus {
        fuege n hinten in Q ein;
        markiere n;
      }
    }
  }
}
\end{minted}

\liJavaDatei[firstline=37,lastline=77]{graph/algorithmen/BreitenSucheWarteschlange}

\literatur

\end{document}
