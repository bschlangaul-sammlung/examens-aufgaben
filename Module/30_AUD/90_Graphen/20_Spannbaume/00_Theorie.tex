\documentclass{bschlangaul-haupt}
\liLadePakete{graph,pseudo,syntax,tabelle}

\begin{document}

%%%%%%%%%%%%%%%%%%%%%%%%%%%%%%%%%%%%%%%%%%%%%%%%%%%%%%%%%%%%%%%%%%%%%%%%
% Theorie-Teil
%%%%%%%%%%%%%%%%%%%%%%%%%%%%%%%%%%%%%%%%%%%%%%%%%%%%%%%%%%%%%%%%%%%%%%%%

\chapter{Spannbäume}

%-----------------------------------------------------------------------
%
%-----------------------------------------------------------------------

\section{Definition „Spannbaum“}

Es sei $G$ ein zusammenhängender Graph und $S$ ein zusammenhängender
Teilgraph von $G$. $S$ ist ein Spannbaum von $G$, falls $S$ \memph{alle
Knoten von $G$ enthält} und $S$ \memph{zyklenfrei} ist.

%-----------------------------------------------------------------------
%
%-----------------------------------------------------------------------

\section{Definition „Minimaler Spannbaum“}

$S$ ist ein minimaler Spannbaum, falls $S$ ein Spannbaum von $G$ ist,
und die Summe der \memph{Kantengewichte} in $S$ \memph{kleiner oder
gleich der aller anderen möglichen Spannbäume} $S^\prime$ von $G$ ist
\footcite[Seite 29]{aud:fs:6}.

%-----------------------------------------------------------------------
%
%-----------------------------------------------------------------------

\section{Algorithmus von Kruskal}

Durch den Algorithmus von Kruskal wird ein \memph{minimaler Spannbaum}
eines ungerichteten, zusammenhängenden und kantengewichteten Graphen
bestimmt.
\footcite[Seite 30 (PDF 24)]{aud:fs:6}

Der Algorithmus von Kruskal nutzt die Kreiseigenschaft minimaler
Spannbäume. Dazu werden die Kanten in der \memph{ersten Phase}
\memph{aufsteigend nach ihrem Gewicht sortiert}. In der zweiten Phase
wird \memph{über die sortierten Kanten iteriert}. Wenn eine
\memph{Kante} zwei Knoten verbindet, die \memph{noch nicht} durch einen
Pfad vorheriger Kanten \memph{verbunden} sind, wird diese Kante zum
minimalen Spannbaum \memph{hinzugenommen}.
\footcite{wiki:kruskal}

\begin{algorithm}[H]
\KwData{$G = (V,E,w)$: ein zusammenhängender, ungerichteter,
kantengewichteter Graph kruskal(G)}

$E'\leftarrow \emptyset $\;
$L\leftarrow E$\;

Sortiere die Kanten in L aufsteigend nach ihrem Kantengewicht.\;
\While{$L \neq \emptyset $}{
  wähle eine Kante $e\in L$ mit kleinstem Kantengewicht\;
  entferne die Kante e aus L\;

  \If{der Graph $(V, E' \cup \lbrace e\rbrace)$ keinen Kreis enthält}{
    $E'\leftarrow E'\cup \lbrace e\rbrace $\;
  }
}

\KwResult{$M = (V,E')$ ist ein minimaler Spannbaum von G.}
\caption{Minimaler Spannbaum nach Kruskal\footcite{wiki:kruskal}}
\end{algorithm}

\begin{liGraphenFormat}
A: 0 5
B: 2 4
C: 5 5
D: 0 2
E: 4 2
F: 2 1
G: 5 0
A -- B: 7
A -- D: 5
B -- C: 8
B -- D: 9
B -- E: 7
C -- E: 5
D -- E: 15
D -- F: 6
E -- F: 8
E -- G: 9
F -- G: 11
\end{liGraphenFormat}

\begin{tikzpicture}[li graph]
\node (A) at (0,5) {A};
\node (B) at (2,4) {B};
\node (C) at (5,5) {C};
\node (D) at (0,2) {D};
\node (E) at (4,2) {E};
\node (F) at (2,1) {F};
\node (G) at (5,0) {G};

\path (A) edge node {7} (B);
\path (A) edge node {5} (D);
\path (B) edge node {8} (C);
\path (B) edge node {9} (D);
\path (B) edge node {7} (E);
\path (C) edge node {5} (E);
\path (D) edge node {15} (E);
\path (D) edge node {6} (F);
\path (E) edge node {8} (F);
\path (E) edge node {9} (G);
\path (F) edge node {11} (G);
\end{tikzpicture}

\begin{minipage}{7cm}
\begin{liGraphenFormat}
A: 0 5
B: 2 4
C: 5 5
D: 0 2
E: 4 2
F: 2 1
G: 5 0
A -- B*: 7
A -- D*: 5
B -- C: 8
B -- D: 9
B -- E*: 7
C -- E*: 5
D -- E: 15
D -- F*: 6
E -- F: 8
E -- G*: 9
F -- G: 11
\end{liGraphenFormat}

\begin{tikzpicture}[li graph]
\node (A) at (0,5) {A};
\node (B) at (2,4) {B};
\node (C) at (5,5) {C};
\node (D) at (0,2) {D};
\node (E) at (4,2) {E};
\node (F) at (2,1) {F};
\node (G) at (5,0) {G};

\path[li markierung] (A) edge node {7} (B);
\path[li markierung] (A) edge node {5} (D);
\path (B) edge node {8} (C);
\path (B) edge node {9} (D);
\path[li markierung] (B) edge node {7} (E);
\path[li markierung] (C) edge node {5} (E);
\path (D) edge node {15} (E);
\path[li markierung] (D) edge node {6} (F);
\path (E) edge node {8} (F);
\path[li markierung] (E) edge node {9} (G);
\path (F) edge node {11} (G);
\end{tikzpicture}
\end{minipage}
\begin{minipage}{4cm}
\begin{center}
\begin{tabular}{|l|l|r|}
\hline
Kante & & Gewicht\\\hline\hline
AD, CE & $2 \times 5$ & $10$\\
DF     &              & $6$\\
AB, BE & $2 \times 7$ & $14$\\
EG     &              & $9$\\\hline
       &              & $39$\\\hline
\end{tabular}
\end{center}
\end{minipage}

\liJavaDatei{graph/algorithmen/MinimalerSpannbaumKruskal}

%-----------------------------------------------------------------------
%
%-----------------------------------------------------------------------

\section{Algorithmus von Prim\footcite{wiki:prim}}

Der Algorithmus von Prim dient ebenfalls der Bestimmung eines minimalen
Spannbaums, erreicht jedoch sein Ziel durch eine unterschiedliche
Vorgehensweise. Er wird auch nach dem \memph{Entdecker} Jarník genannt.
\footcite{wiki:prim}

Der \memph{Teilgraph} $T$ wird \memph{schrittweise vergrößert}, bis
dieser ein Spannbaum ist. Der Algorithmus benötigt einen
\memph{konkreten Startpunkt}. Es wird die \memph{günstigste} von
einem Teilgraphen $T$ \memph{ausgehende Kante ausgewählt} und zu diesem
Spannbaum hinzugefügt. Der Algorithmus fügt jede Kante und deren
Endknoten zur Lösung hinzu, die mit allen zuvor gewählten Kanten keinen
Kreis bildet.
\footcite[Seite 32, (PDF 26)]{aud:fs:6}

\begin{minipage}{7cm}
\begin{liGraphenFormat}
A: 0 5
B: 2 4
C: 5 5
D: 0 2
E: 4 2
F: 2 1
G: 5 0
A -- B*: 7
A -- D*: 5
B -- C: 8
B -- D: 9
B -- E*: 7
C -- E*: 5
D -- E: 15
D -- F*: 6
E -- F: 8
E -- G*: 9
F -- G: 11
\end{liGraphenFormat}

\begin{tikzpicture}[li graph]
\node (A) at (0,5) {A};
\node (B) at (2,4) {B};
\node (C) at (5,5) {C};
\node (D) at (0,2) {D};
\node (E) at (4,2) {E};
\node (F) at (2,1) {F};
\node (G) at (5,0) {G};

\path[li markierung] (A) edge node {7} (B);
\path[li markierung] (A) edge node {5} (D);
\path (B) edge node {8} (C);
\path (B) edge node {9} (D);
\path[li markierung] (B) edge node {7} (E);
\path[li markierung] (C) edge node {5} (E);
\path (D) edge node {15} (E);
\path[li markierung] (D) edge node {6} (F);
\path (E) edge node {8} (F);
\path[li markierung] (E) edge node {9} (G);
\path (F) edge node {11} (G);
\end{tikzpicture}

\end{minipage}
\begin{minipage}{4cm}
\begin{center}
\begin{tabular}{|l|r|}
\hline
Kante & Gewicht\\\hline\hline
AD & $5$\\
DF & $6$\\
AB & $7$\\
BE & $7$\\
EC & $5$\\
EG & $9$\\\hline
   & $39$\\\hline
\end{tabular}
\end{center}
\end{minipage}

\liJavaDatei{graph/algorithmen/MinimalerSpannbaumPrim}

%%
%
%%

\section{Vergleich Kruskal und Prim}

\noindent
\begin{tabularx}{\linewidth}{r||X|X}
&
\textbf{Kruskal} &
\textbf{Prim} \\\hline\hline

Arbeitsweise &
sortiert Kanten nach Gewichten &
Startknoten $\rightarrow$ Nachbarknoten \\

Kanten-Sichtweise &
globale Sicht &
lokale Sicht \\

Zyklen-Vermeidung &
aktiv &
passiv \\

Gierigkeit &
greedy &
greedy \\

Teilgraph &
mehrere Teilgraphen möglich &
nur ein Teilgraph möglich \\

Datenstrukturen &
Union-Find &
Min-Heap\\

Laufzeit &
$\mathcal{O}\bigl(\left|E\right| \cdot \log(\left|E\right|)\bigr)$ &
$\mathcal{O}\bigl(\left|E\right| + \left|V\right| \cdot \log(\left|V\right|)\bigr)$ \\
\end{tabularx}

\literatur

\end{document}
