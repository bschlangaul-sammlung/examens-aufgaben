\documentclass{bschlangaul-theorie}
\bLadePakete{graph}
\begin{document}

%%%%%%%%%%%%%%%%%%%%%%%%%%%%%%%%%%%%%%%%%%%%%%%%%%%%%%%%%%%%%%%%%%%%%%%%
% Theorie-Teil
%%%%%%%%%%%%%%%%%%%%%%%%%%%%%%%%%%%%%%%%%%%%%%%%%%%%%%%%%%%%%%%%%%%%%%%%

\chapter{Graphen}

\begin{bQuellen}
\item \cite[Seite 445-481]{saake}
\item \cite{wiki:graph}
\end{bQuellen}

\noindent
Ein Graph ist eine \bEmph{dynamische Datenstruktur}, die die
\bEmph{Speicherung beliebig vieler Elemente} erlaubt. Ein Graph besteht
aus \bEmph{Knoten} und \bEmph{Kanten}, die die Beziehungen zwischen den
Knoten repräsentieren.

\section{Arten von Graphen}

\begin{compactitem}
\item gerichtete / ungerichtete Graphen
\item gewichtete / ungewichtete Graphen
\end{compactitem}

\section{Grad}

\begin{compactitem}
\item Eingangsgrad (Anzahl der eingehenden Kanten)
\item Ausgangsgrad (Anzahl der ausgehenden Kanten)
\item Bei ungerichteten Graphen: Grad
\end{compactitem}

\footcite[Seite 3]{aud:fs:6}

\noindent
Ein Graph ist ein Paar $G=(V, E)$. $V$ bezeichnet die Menge der Knoten
(\bEmph{vertex}) und $E \subseteq V \times V$ die Menge der Kanten
(\bEmph{edge}).
%
Eine Kante $(a, b) \in E$ ist im \emph{gerichteten} Graphen eine Kante
von $a$ nach $b$. Im \emph{ungerichteten} Graphen schreiben wir $[a,b]$.
Dabei ist $(a, b) \in E$ und $(b, a) \in E$.
%
Im gerichteten Graphen gibt es analog zu
Listen und Bäumen Vorgänger- und Nachfolgerknoten. Jeder Kante $e \in
E$ kann ein \bEmph{Kantengewicht} $c(e)$ zugeordnet sein. Kantengewichte
entsprechen den „Kosten“ für die Traversierung dieser Kante. Die
Darstellung von Graphen ist auf unterschiedliche Weise möglich:

\begin{compactitem}
\item graphische Darstellung
\item Mengenschreibweise
\item Adjazenzmatrix
\item Adjazenzlisten
\end{compactitem}
\footcite[Seite 4]{aud:fs:6}

%-----------------------------------------------------------------------
%
%-----------------------------------------------------------------------

\section{Adjazenz-Matrix}

Eine \textbf{Adjazenzmatrix} (von lat. \emph{adiacere} = bei oder neben
etwas liegen, angrenzen) eines Graphen ist eine Matrix, die speichert,
welche Knoten des Graphen durch eine Kante verbunden sind. Sie besitzt
für \bEmph{jeden Knoten} eine \bEmph{Zeile} und eine \bEmph{Spalte},
woraus sich für $n$ Knoten eine $n \times n$-Matrix ergibt. Ein Eintrag
in der $i$-ten Zeile und $j$-ten Spalte gibt hierbei an, ob eine Kante
von dem $i$-ten zu dem $j$-ten Knoten führt
\footcite{wiki:adjazenzmatrix}. Man geht \bEmph{von der Zeile zur
Spalte}.

%%
%
%%

\subsection{Die Adjazenzmatrix eines ungerichteten Graphen}

\begin{bGraphenFormat}
A: 0 0
B: 0 -3
C: 2.5 -2
D: 2 0
E: 5 0
F: 4 -3
A -- B
B -- D
B -- C
B -- F
C -- D
C -- F
D -- E
D -- A
E -- C
E -- F
\end{bGraphenFormat}

\begin{center}
\begin{tikzpicture}[li graph]
\node (A) at (0,0) {A};
\node (B) at (0,-3) {B};
\node (C) at (2.5,-2) {C};
\node (D) at (2,0) {D};
\node (E) at (5,0) {E};
\node (F) at (4,-3) {F};

\path (A) edge node {} (B);
\path (B) edge node {} (C);
\path (B) edge node {} (D);
\path (B) edge node {} (F);
\path (C) edge node {} (D);
\path (C) edge node {} (F);
\path (D) edge node {} (A);
\path (D) edge node {} (E);
\path (E) edge node {} (C);
\path (E) edge node {} (F);
\end{tikzpicture}
\end{center}

\noindent
Eine Verbindung (Kante) zwischen zwei Knoten in einem ungerichteten
Graphen ist in beide Richtungen gültig. Überträgt man die Verbindungen
in die Matrix, erhält man eine \bEmph{symmetrische, an der Diagonalen
gespiegelte Adjazenzmatrix}.

\[
\begin{blockarray}{ccccccc}
   & A & B & C & D & E & F \\
\begin{block}{c(cccccc)}
 A & * & 1 & - & 1 & - & - \\
 B & 1 & * & 1 & 1 & - & 1 \\
 C & - & 1 & * & 1 & 1 & 1 \\
 D & 1 & 1 & 1 & * & 1 & - \\
 E & - & - & 1 & 1 & * & 1 \\
 F & - & 1 & 1 & - & 1 & * \\
\end{block}
\end{blockarray}
\]

%%
%
%%

\subsection{Die Adjazenzmatrix eines gerichteten Graphen}

\begin{bGraphenFormat}
A: 0 0
B: 0 -3
C: 2.5 -2
D: 2 0
E: 5 0
F: 4 -3
A -> B
B -> D
B -> C
B -> F
C -> D
C -> F
D -> E
D -> A
E -> C
E -> F
\end{bGraphenFormat}

\begin{center}
\begin{tikzpicture}[li graph]
\node (A) at (0,0) {A};
\node (B) at (0,-3) {B};
\node (C) at (2.5,-2) {C};
\node (D) at (2,0) {D};
\node (E) at (5,0) {E};
\node (F) at (4,-3) {F};

\path[->] (A) edge node {} (B);
\path[->] (B) edge node {} (C);
\path[->] (B) edge node {} (D);
\path[->] (B) edge node {} (F);
\path[->] (C) edge node {} (D);
\path[->] (C) edge node {} (F);
\path[->] (D) edge node {} (A);
\path[->] (D) edge node {} (E);
\path[->] (E) edge node {} (C);
\path[->] (E) edge node {} (F);
\end{tikzpicture}
\end{center}

\noindent
Beim gerichteten Graphen sind die Verbindungen zwischen den Knoten
jeweils nur in einer Richtung gültig. Im Graphen erfolgt die Darstellung
der \bEmph{Richtung mit einem Pfeil}. Für die Adjazenzmatrix haben die
gerichteten Kanten zur Folge, dass \bEmph{keine Symmetrie mehr} besteht.

\[
\begin{blockarray}{ccccccc}
   & A & B & C & D & E & F \\
\begin{block}{c(cccccc)}
 A & * & 1 & - & - & - & - \\
 B & - & * & 1 & 1 & - & 1 \\
 C & - & - & * & 1 & - & 1 \\
 D & 1 & - & - & * & 1 & - \\
 E & - & - & 1 & - & * & 1 \\
 F & - & - & - & - & - & * \\
\end{block}
\end{blockarray}
\]

%%
%
%%

\subsection{Die Adjazenzmatrix eines gewichteten Graphens}

\begin{bGraphenFormat}
A: 0 0
B: 0 -3
C: 2.5 -2
D: 2 0
E: 5 0
F: 4 -3
A -> B: 20
B -> D: 10
B -> C: 50
B -> F: 60
C -> D: 20
C -> F: 10
D -> E: 20
D -> A: 30
E -> C: 10
E -> F: 30
\end{bGraphenFormat}

\begin{center}
\begin{tikzpicture}[li graph]
\node (A) at (0,0) {A};
\node (B) at (0,-3) {B};
\node (C) at (2.5,-2) {C};
\node (D) at (2,0) {D};
\node (E) at (5,0) {E};
\node (F) at (4,-3) {F};

\path[->] (A) edge node {20} (B);
\path[->] (B) edge node {50} (C);
\path[->] (B) edge node {10} (D);
\path[->] (B) edge node {60} (F);
\path[->] (C) edge node {20} (D);
\path[->] (C) edge node {10} (F);
\path[->] (D) edge node {30} (A);
\path[->] (D) edge node {20} (E);
\path[->] (E) edge node {10} (C);
\path[->] (E) edge node {30} (F);
\end{tikzpicture}
\end{center}

\noindent
In gewichteten Graphen haben die Kanten sogenannte \bEmph{Kosten} oder
\bEmph{Gewichte}. In einer Matrix für gewichtete Graphen werden die
\bEmph{Werte für das jeweilige Gewicht} der Kante zwischen den Knoten
eingetragen.
\footnote{\url{https://www.bigdata-insider.de/was-ist-eine-adjazenzmatrix-a-845891/}}

\[
\begin{blockarray}{ccccccc}
    &  A &  B &  C &  D &  E &  F \\
\begin{block}{c(cccccc)}
  A &  * & 20 &  - &  - &  - &  - \\
  B &  - &  * & 50 & 10 &  - & 60 \\
  C &  - &  - &  * & 20 &  - & 10 \\
  D & 30 &  - &  - &  * & 20 &  - \\
  E &  - &  - & 10 &  - &  * & 30 \\
  F &  - &  - &  - &  - &  - &  * \\
\end{block}
\end{blockarray}
\]

\section{Adjazenzliste}

\begin{center}
\begin{tabular}{llll}
A: & $\xrightarrow{~20~}$ B \\
B: & $\xrightarrow{~50~}$ C & $\xrightarrow{~10~}$ D & $\xrightarrow{~60~}$ F \\
C: & $\xrightarrow{~20~}$ D & $\xrightarrow{~10~}$ F \\
D: & $\xrightarrow{~30~}$ A & $\xrightarrow{~20~}$ E \\
E: & $\xrightarrow{~10~}$ C & $\xrightarrow{~30~}$ F \\
F: \\
\end{tabular}
\end{center}

\noindent
Wir speichern zu jedem Knoten $V$ eines Graphen seine Nachbarknoten in
einem Feld oder einer Liste der Länge $|V|$. Zur Repräsentation von
gewichteten Graphen werden in den Listen Tupel von Knoten und
Kantengewichten gespeichert. Ein Vorteil von Adjazenzlisten ist der
geringe Platzbedarf von $O(|V|+|E|)$. Außerdem sind die $n$ Nachfolger
eines Knotens in $\mathcal{O}(n)$ erreichbar, unabhängig von der
Gesamtzahl der Knoten. Jedoch ist nicht mehr in $\mathcal{O}(1)$
bestimmbar, ob eine bestimmte Kante existiert.\footcite[Seite 7]{aud:fs:6}

\literatur
\end{document}
