\documentclass{lehramt-informatik-aufgabe}
\liLadePakete{sortieren,quicksort,syntax}
\begin{document}
\liAufgabenTitel{Händisches Sortieren}

\section{Händisches Sortieren\footcite[Seite 1]{aud:pu:1}}

Gegeben sei ein Array $a$, welches die Werte $5,7,9,3,6,1,2,8$ enthält.
Sortieren Sie das Array händisch mit:

\begin{enumerate}

%%
% (a)
%%

\item Bubblesort\index{Bubblesort}

erster Durchgang a

\liVertauschen{5 7 9 >3 <6 1 2 8}

\liVertauschen{5 7 9 6 >3 <1 2 8}

\liVertauschen{5 7 9 6 1 >3 <2 8}

Zweiter Durchgang

\liVertauschen{5 7 9 >6 <1 2 3 8}

\liVertauschen{5 7 9 1 >6 <2 3 8}

\liVertauschen{5 7 9 1 2 >6 <3 8}

Dritter Durchgang

\liVertauschen{5 7 >9 <1 2 3 6 8}

\liVertauschen{5 7 1 >9 <2 3 6 8}

\liVertauschen{5 7 1 2 >9 <3 6 8}

\liVertauschen{5 7 1 2 3 >9 <6 8}

\liVertauschen{5 7 1 2 3 6 >9 <8}

Vierter Durchgang

\liVertauschen{5 >7 <1 2 3 6 8 9}

\liVertauschen{5 1 >7 <2 3 6 8 9}

\liVertauschen{5 1 2 >7 <3 6 8 9}

\liVertauschen{5 1 2 3 >7 <6 8 9}

Fünfter Durchgang

\liVertauschen{>5 <1 2 3 6 7 8 9}

\liVertauschen{1 >5 <2 3 6 7 8 9}

\liVertauschen{1 2 >5 <3 6 7 8 9}

fertig

\liVertauschen{1 2 3 5 6 7 8 9}

%%
% (b)
%%

\item Mergesort\index{Mergesort}

\begin{forest}
  /tikz/arrows=->, /tikz/>=latex, /tikz/nodes={draw},
  for tree={delay={sort}}, sort level=2
[5 7 9 3 6 1 2 8
  [5 7 9 3
    [5 7
      [5]
      [7]
    ]
    [9 3
      [9]
      [3]
    ]
  ]
  [6 1 2 8
    [6 1
      [6]
      [1]
    ]
    [2 8
      [2]
      [8]
    ]
  ]
]
%
\coordinate (m) at (!|-!\forestOnes);
\myNodes
\end{forest}

%%
% (c)
%%

\item Quicksort\index{Quicksort}

\QSinitialize{5,7,9,3,6,1,2,8}

\loop
\QSpivotStep
\ifnum\value{pivotcount}>0
  \QSsortStep
\repeat

\end{enumerate}

\end{document}
