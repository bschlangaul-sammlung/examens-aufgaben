\documentclass{lehramt-informatik-haupt}
\liLadePakete{mathe,master-theorem}

\begin{document}
\let\O=\liO
\let\o=\liOmega
\let\T=\liT
\let\t=\liTheta

%%%%%%%%%%%%%%%%%%%%%%%%%%%%%%%%%%%%%%%%%%%%%%%%%%%%%%%%%%%%%%%%%%%%%%%%
% Theorie-Teil
%%%%%%%%%%%%%%%%%%%%%%%%%%%%%%%%%%%%%%%%%%%%%%%%%%%%%%%%%%%%%%%%%%%%%%%%

\section{Master-Theorem}

Der Hauptsatz der Laufzeitfunktionen – oder oft auch aus dem Englischen
als Master-Theorem entlehnt – bietet eine \memph{schnelle Lösung} für
die Frage, \memph{in welcher Laufzeitklasse} eine gegebene
\memph{rekursiv definierte Funktion} liegt. Mit dem Master-Theorem kann
allerdings \memph{nicht jede rekursiv definierte Funktion} gelöst
werden. Lässt sich keiner der \memph{drei möglichen Fälle} des
Master-Theorems auf die Funktion $T$ anwenden, so muss man die
Komplexitätsklasse der Funktion anderweitig berechnen.

\liMasterVariablen

\liMasterFaelle

\literatur
\end{document}
