\documentclass{bschlangaul-theorie}

\begin{document}

%%%%%%%%%%%%%%%%%%%%%%%%%%%%%%%%%%%%%%%%%%%%%%%%%%%%%%%%%%%%%%%%%%%%%%%%
% Theorie-Teil
%%%%%%%%%%%%%%%%%%%%%%%%%%%%%%%%%%%%%%%%%%%%%%%%%%%%%%%%%%%%%%%%%%%%%%%%

\chapter{Divide-and-Conquer (Teile und herrsche)}

\begin{bQuellen}
\item \cite[Seite 7-9]{aud:fs:3}
\item \cite[Seite 218-222 (PDF 236-240)]{saake}
\item \cite{wiki:teile-und-herrsche-verfahren}
\end{bQuellen}

Das Prinzip „Divide-and-conquer“, deutsch „Teile und herrsche,“ basiert
darauf, in einem Schritt eine \bEmph{große Aufgabe} in mehrere
\bEmph{kleinere Aufgaben zu teilen} und diese rekursiv zu bearbeiten –
also ein klassischer Einsatz des \bEmph{Rekursionsprinzips}.
\footcite[Seite 219 (PDF 237)]{saake}

\literatur

\end{document}
