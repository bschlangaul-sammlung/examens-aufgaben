\documentclass{lehramt-informatik-aufgabe}
\liLadePakete{syntax}
\begin{document}
\liAufgabenMetadaten{
  Titel = {Backtracking: Das Damenproblem},
  Thematik = {Damenproblem},
  RelativerPfad = Module/30_AUD/60_Algorithmenmuster/50_Backtracking/Aufgabe_Damenproblem.tex,
  ZitatSchluessel = aud:fs:3,
  ZitatBeschreibung = {Seite 18 - 19},
  BearbeitungsStand = unbekannt,
  Korrektheit = unbekannt,
  Stichwoerter = {Backtracking, Implementierung in Java},
}

Implementieren sie mittels Backtracking einen Algorithmus, der acht
Damen auf einem Schachbrett so aufgestellt, dass keine zwei Damen
einander gemäß ihren in den Schachregeln definierten Zugmöglichkeiten
schlagen können. Für Damen heißt dies konkret: Es dürfen keine zwei
Damen auf derselben Reihe, Linie oder Diagonale stehen. Es gibt $92$
mögliche Lösungen für das $8 \times 8$ Feld.
\index{Implementierung in Java}\index{Backtracking}
\footcite[Seite 18 - 19]{aud:fs:3}

\begin{liAntwort}
\liJavaDatei[firstline=3,lastline=49]{aufgaben/aud/muster/backtracking/damenproblem/Damenproblem}
\end{liAntwort}

\begin{liAdditum}
\liJavaDatei{aufgaben/aud/muster/backtracking/damenproblem/Damenproblem}
\liJavaDatei{aufgaben/aud/muster/backtracking/damenproblem/Ausgabe}
\end{liAdditum}
\end{document}
