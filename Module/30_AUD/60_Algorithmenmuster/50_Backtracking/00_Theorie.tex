\documentclass{bschlangaul-theorie}
\liLadePakete{syntax,mathe}

\begin{document}

%%%%%%%%%%%%%%%%%%%%%%%%%%%%%%%%%%%%%%%%%%%%%%%%%%%%%%%%%%%%%%%%%%%%%%%%
% Theorie-Teil
%%%%%%%%%%%%%%%%%%%%%%%%%%%%%%%%%%%%%%%%%%%%%%%%%%%%%%%%%%%%%%%%%%%%%%%%

\chapter{Backtracking}

\begin{liQuellen}
\item \cite[Seite 15-21 (PDF 13-19)]{aud:fs:3}
\item \cite[Seite 222-230]{saake}
\item \cite{wiki:backtracking}
\end{liQuellen}

\noindent
Das Backtracking ist ein wichtiges Algorithmenmuster für Such- und
Optimierungsprobleme. Das Backtracking realisiert eine allgemeine
systematische Suchtechnik, die einen \memph{vorgegebenen Lösungsraum
komplett} bearbeitet.

Der Begriff \emph{Backtracking} kommt daher, dass man bei der Suche in
\memph{Sackgassen} gerät und dann wieder zur nächsten noch nicht
bearbeiteten Abzweigung zurückgeht, bis man alle Verzweigungen
abgearbeitet hat.
\footcite[Seite 222]{saake}

\section{Idee des Backtracking:}

Der Algorithmus wird so lange ausführen, bis man an eine Grenze kommt
Wenn das der Fall ist, geht man zurück zum letzten Schritt
und testet andere Folgeschritte. Man versucht, gültige Teillösung zu
finden und auf dieser den restlichen Weg zum Ziel aufzubauen. Wenn das
nicht möglich ist, versucht man andere Teillösung zu finden.
\footcite[Seite 18 - 19 (PDF 15-19)]{aud:fs:3}

\section{Funktionsweise\footcite[Seite 17 (PDF 15)]{aud:fs:3}} Das
Backtracking läuft fast immer nach demselben Schema ab. Wenn man sich
passende Methoden definiert, kann man für die Implementierung oft
dasselbe Grundgerüst verwenden. Die erforderlichen Methoden sind:

\begin{description}
\item[isFinal()] überprüft, ob eine Lösung gefunden wurde, \zB: ist
ein Sudoku vollständig gefüllt?

\item[getExtensions()] gibt alle möglichen Erweiterungen zurück, \zB:
alle erlaubte Zahlen für das aktuelle Feld

\item[apply()] verändert den aktuellen Zustand, \zB: schreibe die
aktuelle Zahl ins aktuelle Feld

\item[revert()] stellt den vorherigen Zustand wieder her, \zB: lösche
die letzte Zahl aus dem aktuellen Feld
\end{description}

\literatur

\end{document}
