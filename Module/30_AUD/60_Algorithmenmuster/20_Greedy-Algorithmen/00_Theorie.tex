\documentclass{bschlangaul-theorie}
\bLadePakete{mathe,syntax}

\begin{document}

%%%%%%%%%%%%%%%%%%%%%%%%%%%%%%%%%%%%%%%%%%%%%%%%%%%%%%%%%%%%%%%%%%%%%%%%
% Theorie-Teil
%%%%%%%%%%%%%%%%%%%%%%%%%%%%%%%%%%%%%%%%%%%%%%%%%%%%%%%%%%%%%%%%%%%%%%%%

\chapter{Greedy-Algorithmen}

\begin{bQuellen}
\item \cite[Seite 207-235 (PDF 225-253)]{saake}
\item \cite[Seite 5-6]{aud:fs:3}
\item \cite{wiki:greedy-algorithmus}
\end{bQuellen}

\bEmph{Greedy} steht hier für gierig. Das Prinzip gieriger Algorithmen
ist es, in \bEmph{jedem Teilschritt so viel wie möglich} zu erreichen.
\footcite[Seite 213 (PDF 231)]{saake}
Greedy-Algorithmen berechnen jeweils ein lokales Optimum in jedem Schritt
und können daher eventuell ein globales Optimum verpassen.
\footcite[Seite 214 (PDF 232)]{saake}

\literatur

\end{document}
