\documentclass{lehramt-informatik-aufgabe}
\liLadePakete{uml}
\begin{document}
\liAufgabenTitel{Maut}

\section{Aufgabe zu Heterogenen Listen
\index{Einfach-verkettete Liste}
\footcite[Seite 1, Aufgabe 1: Heterogene Liste]{aud:pu:4}}

Für die Umsetzung der Maut auf deutschen Autobahnen soll eine
Java-basierte Lösung entworfen werden. Dazu sollen alle Fahrzeuge, die
von einer Mautbrücken erfasst werden, in einer \emph{einfach verketteten
Liste} abgelegt werden. Um einen besseren Überblick über die Einnahmen
zu erhalten, soll zwischen \emph{LKWs}, \emph{PKWs} und
\emph{Motorrädern} unterschieden werden. Als Informatiker schlagen Sie
eine \emph{heterogene Liste} zur Realisierung vor. Notieren Sie unter
Verwendung des \emph{Entwurfsmusters Kompositum} ein entsprechendes
\emph{Klassendiagramm} zur Realisierung der Lösung für eine Mautbrücke.
Auf die Angabe von Attributen und Methoden kann verzichtet werden.
Kennzeichen Sie in Ihrem Klassendiagramm die \emph{abstrakten Klassen}
und benennen Sie die bestehenden \emph{Beziehungen}.
\index{Implementierung in Java}

\begin{center}
\begin{tikzpicture}
\umlsimpleclass[type=abstract,x=4.5,y=5]{Listenelement}
\umlsimpleclass[x=3,y=3]{Abschluss}
\umlsimpleclass[x=6,y=3]{Datenknoten}
\umlsimpleclass[x=0,y=5]{Mautbrücke}
\umlassoc[mult2=1,arg2=erster\liLeserichtungRechts{},pos2=0.7]{Mautbrücke}{Listenelement}
\umlinherit{Abschluss}{Listenelement}
\umlinherit{Datenknoten}{Listenelement}

\umlsimpleclass[type=abstract,x=5,y=2]{Fahrzeug}
\umlsimpleclass[x=3,y=0]{Motorrad}
\umlsimpleclass[x=5,y=0]{PKW}
\umlsimpleclass[x=7,y=0]{LKW}
\umlinherit{PKW}{Fahrzeug}
\umlinherit{Motorrad}{Fahrzeug}
\umlinherit{LKW}{Fahrzeug}

% \umlHVHassoc[
%   anchor1=east,
%   anchor2=east,
%   arm1=3cm,arg1=\liLeserichtungLinks{}enthält,pos1=0.3,mult1=1,
%   arg2=\liLeserichtungLinks{}nächster,pos2=2.7,mult2=1
% ]{Fahrzeug}{Listenelement}
\end{tikzpicture}
\end{center}
\end{document}
