\documentclass{bschlangaul-theorie}
\bLadePakete{syntax}

\begin{document}

%%%%%%%%%%%%%%%%%%%%%%%%%%%%%%%%%%%%%%%%%%%%%%%%%%%%%%%%%%%%%%%%%%%%%%%%
% Theorie-Teil
%%%%%%%%%%%%%%%%%%%%%%%%%%%%%%%%%%%%%%%%%%%%%%%%%%%%%%%%%%%%%%%%%%%%%%%%

\chapter{Warteschlange (Queue)}

\begin{liQuellen}
\item \cite[Kapitel 6.2.1.5, Seite 183]{schneider}
\item \cite{wiki:warteschlange}
\end{liQuellen}

\noindent
Eine Queue, auch \memph{Warteschlange} oder \memph{Puffer} genannt, ist
eine Datenstruktur, bei der die Elemente – ähnlich wie bei einem Stack –
\memph{als Folge „organisiert“} sind.
%
Ein neues Element kann \memph{nur auf einer Seite} der Elementfolge
\memph{eingefügt} werden.
%
Das \memph{Entfernen} eines Elements ist – im Gegensatz zum Stack – nur
\memph{auf der anderen Seite} der Elementfolge möglich (\memph{FIFO}:
\memph{First In – First Out}). Das Element, das als erstes in eine Queue
eingefügt wird, muss somit als erstes Element wieder entfernt werden.
%
Die Operation zum Hinzufügen eines Elements wird oft
\bJavaCode{insert()} bzw. \bJavaCode{enqueue()}, die Operation zum
Entfernen eines Elements oft \bJavaCode{remove()} bzw.
\bJavaCode{dequeue()} genannt.
\footcite[Seite 24 (PDF 21)]{aud:fs:4}

%%
%
%%

\section{Suchen in Queues}

Auch in einer Queue muss man bei der \memph{Suche} nach einem bestimmten
Element im schlimmsten Fall \memph{bis zum Ende} der Queue suchen.
%
Hierbei kann allerdings \memph{ohne Zerstörung der Datenstruktur}, also
auch ohne „Hilfsstruktur“, vorgegangen werden.
%
Wegen des \memph{möglichen linearen Komplettdurchlaufs} durch die
gesamte Datenstruktur ist auch bei Queues die Suche eines Elements nicht
optimal.
\footcite[Seite 25 (PDF 22)]{aud:fs:4}

\bJavaDatei{liste/WarteschlangeFehler}
\bJavaDatei{liste/Warteschlange}
\bJavaDatei{liste/FeldWarteschlange}

\literatur

\end{document}
