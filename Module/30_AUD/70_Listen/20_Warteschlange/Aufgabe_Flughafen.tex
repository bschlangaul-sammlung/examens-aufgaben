\documentclass{lehramt-informatik-aufgabe}
\liLadePakete{syntax}
\begin{document}
\liAufgabenTitel{Flughafen}

\section{Listen
\index{Warteschlange (Queue)}
\footcite[Aufgabe 3]{aud:e-klausur}}

Am Nürnberger Flughafen starten und landen täglich viele Flugzeuge. Der
Flughafen verfügt jedoch nur über eine einzige Start- bzw. Landebahn,
sodass Starts und Landungen gemeinsam koordiniert werden müssen. Für die
interne Verwaltung, welcher Flieger als nächstes bearbeitet werden soll,
wird eine neue Software entwickelt. Sie haben die Aufgabe einen Teil
dieser Software zu erstellen.

Das folgende UML-Klassendiagramm gibt einen Überblick über den für Sie
relevanten Teil der Software:

Implementieren Sie die beiden Methoden addNewFlight(Flug) und
addEmergency(Flug) der Klasse Warteschlange!

\begin{itemize}
\item In der Methode \liJavaCode{addNewFlight}

\begin{itemize}
\item soll ein neu übergebener Flug der Warteschlange hinzugefügt werden.
\end{itemize}

\item In der Methode \liJavaCode{addEmergency}

\begin{itemize}
\item soll für den übergebenen Flug überprüft werden, ob der Flug
bereits in der Warteschlange vorkommt oder ob es sich um einen komplet
neuen Flug handelt.

\item soll der Notfall die erste Priorität in der Warteschlange
erhalten, \dh zwingend als nächstes abgearbeitet werden.

\item soll die korrekte Funktionalität der Warteschlange weiterhin
gegeben sein.
\end{itemize}
\end{itemize}

Jeder Flug wird in der Warteschlange mit einem eigenen Ticket verwaltet.
Die Funktionalitäten der Tickets und der Flüge entnehmen Sie dem
Quelltext.

\begin{liAntwort}
\liPseudoUeberschrift{Klasse „Flug“}

\liJavaDatei[firstline=3]{aufgaben/aud/e_klausur/flughafen/Flug}

\liPseudoUeberschrift{Klasse „Ticket“}

\liJavaDatei[firstline=3]{aufgaben/aud/e_klausur/flughafen/Ticket}

\liPseudoUeberschrift{Klasse „Warteschlange“}

\liJavaDatei[firstline=3]{aufgaben/aud/e_klausur/flughafen/Warteschlange}
\end{liAntwort}
\end{document}
