\documentclass{bschlangaul-theorie}
\bLadePakete{syntax}

\begin{document}

%%%%%%%%%%%%%%%%%%%%%%%%%%%%%%%%%%%%%%%%%%%%%%%%%%%%%%%%%%%%%%%%%%%%%%%%
% Theorie-Teil
%%%%%%%%%%%%%%%%%%%%%%%%%%%%%%%%%%%%%%%%%%%%%%%%%%%%%%%%%%%%%%%%%%%%%%%%

\chapter{Stapel (Stack)}

\begin{bQuellen}
\item \cite[Seite 275-281 (PDF 291-297)]{saake}
\item \cite{wiki:stapelspeicher}
\item \cite[Kapitel 6.2.1.4 Seite 182]{schneider}
\end{bQuellen}

Ein Stack, auch \bEmph{Stapel} oder \bEmph{Keller} genannt, ist eine
Datenstruktur, bei der die Elemente \bEmph{als Folge „organisiert“}
sind. Elemente können nur an \bEmph{einem Ende der Folge eingefügt bzw.
ausgelesen / gelöscht} werden (\bEmph{LIFO}: \bEmph{Last In – First
Out}).
%
Das Element, das als \emph{letztes} Element in den Stack
\emph{eingefügt} wurde, muss also als \emph{erstes} Element wieder vom
Stack \emph{entfernt} werden.
%
Dementsprechend kann das als \emph{erstes} in den Stack
\emph{eingefügte} Element erst als \emph{letztes} Element wieder vom
Stack \emph{entfernt} werden.
%
Die Operation zum Hinzufügen eines Elements heißt in der Regel
\bJavaCode{push(Object o)}, die Operation zum Entfernen eines Stackelements
\bJavaCode{pop()}. \bJavaCode{top()} gibt das oberste Element des Stack zurück,
ohne es zu entfernen
\footcite[Seite 19 (PDF 17)]{aud:fs:4}

%%
%
%%

\section{Suchen im Stack}

Um in einem Stack ein Element zu suchen, muss man \bEmph{im schlimmsten
Fall bis zum Ende} des Stacks suchen.
%
Außerdem wird eine \bEmph{zweite Datenstruktur als „Zwischenlager“} der
von oben entfernten Elemente benötigt.
%
Und schließlich muss von dort aus \bEmph{alles wieder zurückgelegt}
werden, um den ursprünglichen Datenzustand wieder herzustellen.
\footcite[Seite 20 (PDF 18)]{aud:fs:4}

\bJavaDatei{liste/StapelFehler}
\bJavaDatei{liste/Stapel}
\bJavaDatei{liste/FeldStapel}
\bJavaDatei{liste/ListeStapel}

\literatur

\end{document}
