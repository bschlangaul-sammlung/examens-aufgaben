\documentclass{lehramt-informatik-aufgabe}
\liLadePakete{vollstaendige-induktion}
\begin{document}
\let\m=\liInduktionMarkierung
\let\e=\liInduktionErklaerung

\liAufgabenTitel{Gaußsche Summenformel}

\section{Gaußsche Summenformel
\index{Vollständige Induktion}}

Die Gaußsche Summenformel lautet: Für alle natürlichen Zahlen $n \geq 1$
gilt

\begin{flalign*}
\operatorname{A}(n):
\hspace{1cm}
1+2+\cdots+n = \frac{n(n+1)}{2}&&\\
\end{flalign*}

\noindent
Sie kann durch vollständige Induktion bewiesen werden.

%%
%
%%

\liInduktionAnfang

Der Induktionsanfang ergibt sich unmittelbar:

\begin{flalign*}
\operatorname{A}(1):
\hspace{1cm}
1 = \frac{1(1+1)}{2}&&\\
\end{flalign*}

\liInduktionVoraussetzung

\noindent
Im Induktionsschritt ist zu zeigen, dass aus der Induktionsvoraussetzung

\begin{flalign*}
\operatorname{A}(n):
\hspace{1cm}
1+2+\cdots+n = \frac{n(n+1)}{2}&&\\
\end{flalign*}

\liInduktionSchritt

\noindent
die Induktionsbehauptung

\begin{flalign*}
\operatorname{A}(n+1):
\hspace{1cm}
1+2+\cdots+n+(n+1) = \frac{(n+1)\bigl((n+1)+1\bigr)}{2} &&\text{für }n \geq 1\\
\end{flalign*}

\noindent
folgt. Dies gelingt folgendermaßen (Die Induktionsvoraussetzung ist rot
markiert.):

\begin{align*}
A(n+1)
& = \m{1 + 2 + \cdots + n} + (n + 1) \\
%
& = \m{\frac{n(n + 1)}{2}} + (n + 1) \\
%
&
= \frac{n(n+1)+2(n+1)}{\m{2}} &
\e{Hauptnenner $2$}\\
%
&
= \frac{(n+2)\m{(n+1)}}{2} &
\e{$(n+1)$ ausgeklammert} \\
%
&
= \frac{\m{(n+1)(n+2)}}{2} &
\e{Umgedreht nach Kommutativgesetz} \\
%
&
= \frac{\m{(n+1)}\bigl(\m{(n+1)}+1\bigr)}{2} &
\e{mit $(n+1)$ an der Stelle von $n$}\\
\end{align*}

\bigskip

\noindent
Schließlich der Induktionsschluss: Damit ist die Aussage
$\operatorname{A}(n)$ für alle $n \geq 1$ bewiesen.

\end{document}
