\documentclass{lehramt-informatik-aufgabe}
\liLadePakete{vollstaendige-induktion}
\begin{document}
\let\m=\liInduktionMarkierung
\let\e=\liInduktionErklaerung

\section{Geometrische Summenformel}

Gegeben sei folgende Methode:\footcite{sosy:e-klausur}
\index{Vollständige Induktion}

\liJavaDatei[firstline=3,lastline=11]{aufgaben/sosy/totale_korrektheit/GeoSum}

\noindent
Weisen Sie mittels vollständiger Induktion nach, dass

\begin{displaymath}
\text{geoSum}(n,q) = 1 - q^{n+1}
\end{displaymath}

\noindent
Dabei können Sie davon ausgehen, dass $q > 0$, $ n \in \mathbb{N}_0$

\begin{liAntwort}
\liInduktionAnfang

\begin{displaymath}
f(0): \text{geoSum}(0, q) = 1 - q^{0+1} = 1 - q^1 = 1 - q
\end{displaymath}

\liInduktionVoraussetzung

\begin{displaymath}
f(n): \text{geoSum}(n, q) = 1 - q^{n+1}
\end{displaymath}

\liInduktionSchritt

\begin{align*}
f(n)
& = \text{geoSum}(n, q)\\
%
& = \m{(1 - q) \cdot q^n +
    \text{geoSum}(n - 1, q)}
& \e{Java-Code in Mathe-Formel umgewandelt}\\
%
& = (1 - q) \cdot q^n +
    \m{1 - q^{(n - 1) + 1}}
& \e{für rekursiven Methodenaufruf gegebene Formel eingesetzt}\\
%
& = (1 - q) \cdot q^n +
    1 - q^{\m{n}}
& \e{Addition im Exponent}\\
\end{align*}

\begin{align*}
f(n+1)
& = \text{geoSum}(n + 1, q)\\
%
& = (1 - q) \cdot q^{n + 1} +
    1 - q^{n + 1}
& \e{von Java konvertierte Formel verwendet und $n + 1$ eingesetzt}\\
%
& = \m{q^{n + 1} - q^{(n + 1) + 1}} +
    1 - q^{n + 1}
& \e{ausmultipliziert}\\
%
& = - q^{(n + 1) + 1} + 1
& \e{$q^{n + 1} - q^{n + 1} = 0$}\\
%
& =  1 \m{- q^{(n + 1) + 1}}
& \e{Kommutativgesetz der Addition}\\
%
& =  1 - q^{(\m{n + 1}) + 1}
& \e{was zu zeigen war}\\
\end{align*}
\end{liAntwort}

\end{document}
