\documentclass{bschlangaul-haupt}
\liLadePakete{syntax,mathe}

\begin{document}

%%%%%%%%%%%%%%%%%%%%%%%%%%%%%%%%%%%%%%%%%%%%%%%%%%%%%%%%%%%%%%%%%%%%%%%%
% Theorie-Teil
%%%%%%%%%%%%%%%%%%%%%%%%%%%%%%%%%%%%%%%%%%%%%%%%%%%%%%%%%%%%%%%%%%%%%%%%

\chapter{Vollständige Induktion}

\begin{liQuellen}
\item \cite[Seite 115-120]{meinel}
\item \cite{wiki:vollständige-induktion}
\item \cite[Seite 22-27 (PDF 20-25)]{aud:fs:1}
\end{liQuellen}

\noindent
Die vollständige Induktion ist eine \memph{mathematische Beweismethode},
nach der eine Aussage für alle \memph{natürlichen Zahlen} bewiesen wird,
die \memph{größer oder gleich einem bestimmten Startwert} sind. Da es
sich um unendlich viele Zahlen handelt, kann eine Herleitung nicht für
jede Zahl einzeln erbracht werden.

Der Beweis, dass die Aussage
$\operatorname{A}(n)$ für alle
$n\geq n_{0}$
($n_0$ meist $1$ oder $0$) gilt, wird daher in zwei Etappen durchgeführt:

\begin{enumerate}
\item Im Induktionsanfang wird die Aussage $\operatorname{A}(n_0)$ für
eine kleinste Zahl $n_0$
hergeleitet.

\item Im Induktionsschritt wird für ein beliebiges $n > n_0$ die Aussage
$\operatorname{A}(n)$ aus der Aussage $\operatorname {A} (n-1)$
hergeleitet.

\end{enumerate}

\noindent
Oder weniger „mathematisch“ formuliert:

\begin{enumerate}
\item Induktionsanfang: Es wird bewiesen, dass die Aussage für die
kleinste Zahl, den Startwert, gilt.

\item Induktionsschritt: Folgendes wird bewiesen: Gilt die Aussage für
eine beliebige Zahl, so gilt sie auch für die Zahl eins größer.
\end{enumerate}

\noindent
Ausgehend vom Beweis für den Startwert erledigt der Induktionsschritt
den Beweis für alle natürlichen Zahlen oberhalb des Startwertes.

\literatur
\end{document}
