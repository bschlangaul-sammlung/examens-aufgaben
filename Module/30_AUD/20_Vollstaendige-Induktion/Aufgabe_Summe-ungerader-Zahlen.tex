\documentclass{lehramt-informatik-aufgabe}
\liLadePakete{vollstaendige-induktion}
\begin{document}
\let\m=\liInduktionMarkierung
\let\e=\liInduktionErklaerung

\liAufgabenTitel{Summe ungerader Zahlen (Maurolicus 1575)}
\section{Summe ungerader Zahlen (Maurolicus 1575)}

Die schrittweise Berechnung der Summe der ersten $n$ ungeraden Zahlen
legt die Vermutung nahe: Die Summe aller ungeraden Zahlen von $1$ bis
$2n-1$ ist gleich dem Quadrat von $n$:

\bigskip

$1 = 1 = 1^2$

$1 + 3 = 4 = 2^2$

$1 + 3 + 5 = 9 = 3^2$

$1 + 3 + 5 + 7 = 16 = 4^2$

\bigskip

\noindent
Folgende Java-Methode berechnet die Summer aller ungeraden Zahlen:

\liJavaDatei[firstline=8,lastline=13]{aufgaben/aud/induktion/Maurolicus}

\bigskip

\noindent
Beweisen Sie mittels vollständiger Induktion, dass der
Methodenaufruf \liJavaCode{oddSum(n)} die Summe aller ungeraden Zahlen von
$1$ bis nur $n$-ten ungeraden Zahl berechnet, wobei gilt:

\begin{displaymath}
\sum\limits^n_{i=1} (2i-1) = n^2
\end{displaymath}.

\begin{liAntwort}

\liInduktionAnfang

\begin{displaymath}
\sum\limits^1_{i=1} (2i-1) = 2 \cdot 1 - 1 = 1 = 1^2
\end{displaymath}

\begin{displaymath}
\texttt{oddSum(1)} = 1 = 1^2
\end{displaymath}

\liInduktionVoraussetzung

\begin{displaymath}
\sum\limits^n_{i=1} (2i-1) = n^2
\end{displaymath}

\begin{displaymath}
\texttt{oddSum(n)} = 2n - 1 + (n - 1)^2
\end{displaymath}

\liInduktionSchritt

\begin{align*}
\texttt{oddSum(n)}
& = 2(n+1) - 1 + ((n+1) - 1)^2
& \e{}\\
%
& = 2(n+1) - 1 + \m{n^2}
& \e{}\\
%
& = \m{2n + 2} + n^2 - 1
& \e{ausmultiplizieren}\\
%
& = 2n \m{+ 1} + n^2
& \e{$2-1 = 1$}\\
%
& = \m{n^2 + 2n + 1}
& \e{Kommutativgesetz} \\
%
& = (\m{n+1})^2
& \e{mit erster Binomischer Formel: $(a+b)^{2}=a^{2}+2ab+b^{2}$} \\
\end{align*}
\end{liAntwort}

\liJavaTestDatei{aufgaben/aud/induktion/MaurolicusTest}

\end{document}
