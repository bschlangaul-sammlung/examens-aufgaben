\documentclass{lehramt-informatik-aufgabe}
\liLadePakete{vollstaendige-induktion}
\begin{document}
\let\m=\liInduktionMarkierung
\let\e=\liInduktionErklaerung

\liAufgabenTitel{Summe ungerader Zahlen (Maurolicus 1575)}
\section{Summe ungerader Zahlen (Maurolicus 1575)}

Die schrittweise Berechnung der Summe der ersten $n$ ungeraden Zahlen
legt die Vermutung nahe: Die Summe aller ungeraden Zahlen von $1$ bis
$2n-1$ ist gleich dem Quadrat von $n$:

$1 = 1$

$1 + 3 = 4$

$1 + 3 + 5 = 9$

$1 + 3 + 5 + 7 = 16$

\liJavaDatei[firstline=8,lastline=13]{aufgaben/aud/induktion/Maurolicus}

\bigskip

\noindent
Der zu beweisende allgemeine Satz lautet: $\sum\limits^n_{i=1} (2i-1) = n^2$.

\liInduktionAnfang

\begin{flalign*}
\operatorname{A}(1):
\hspace{1cm}
\sum\limits^1_{i=1} (2i-1) = 2 \cdot 1 - 1 = 1 = 1^2&&\\
\end{flalign*}

\liInduktionVoraussetzung

\begin{flalign*}
\operatorname{A}(n):
\hspace{1cm}
\sum\limits^n_{i=1} (2i-1) = 1 + 3 + \cdots + (2n - 1) = n^2&&\\
\end{flalign*}

\liInduktionSchritt

\begin{flalign*}
\operatorname{A}(n+1):
\hspace{1cm}
\sum\limits^{n+1}_{i=1} (2i-1) = (n+1)^2&&\\
\end{flalign*}

\liPseudoUeberschrift{Beweis}

\noindent
Er ergibt sich über folgende Gleichungskette, bei der in der zweiten
Umformung die Induktionsvoraussetzung angewandt wird:

{\footnotesize
\begin{align*}
\sum^{n+1}_{i=1} (2i-1)
& = \m{1 + 3 + \cdots + (2n - 1)} +
    (2(n+1)-1)
& \e{Formel für die letzte Zahl ist: $2n - 1$, $n$ ist hier $n + 1$}\\
%
& = \m{\sum^n_{i=1} (2i-1)} +
     (2(n+1)-1)
& \e{andere Schreibweise mit dem Summenzeichen}\\
%
& = \m{n^2} +
    2(n + 1) - 1
& \e{Ersetzen des Summenzeichens mit dem Ergebnis der Formel}\\
%
& = n^2 +
    \m{2n + 2} - 1
& \e{ausmultiplizieren}\\
%
& = n^2 + 2n + \m{1}
& \e{subtrahiert $2 - 1 = 1$} \\
%
& = (n+1)^2
& \e{mit erster Binomischer Formel: $(a+b)^{2}=a^{2}+2ab+b^{2}$} \\
\end{align*}
}

\liJavaTestDatei{aufgaben/aud/induktion/MaurolicusTest}

\end{document}
