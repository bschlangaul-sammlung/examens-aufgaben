\documentclass{lehramt-informatik-aufgabe}
\liLadePakete{graph}
\begin{document}
\liAufgabenTitel{Minimaler Spannbaum A-H}

\section{Spannbaum
\index{Minimaler Spannbaum}
\footcite[Aufgabe 6]{aud:e-klausur}}

Ermitteln Sie einen minimalen Spannbaum des vorliegenden Graphen. Nutzen
Sie den \emph{Knoten A als Startknoten} in ihrem Algorithmus.

\begin{liGraphenFormat}
A: 0 5
B: 8 5
C: 4 4
D: 2 3
E: 6 3
F: 4 1
G: 0 0
H: 8 0
A -- B: 5
A -- C: 8
A -- D: 7
A -- G
B -- C: 2
B -- E: 7
B -- H: 2
C -- D
C -- E: 3
C -- F: 6
D -- F: 2
D -- G: 8
E -- H: 6
F -- E: 5
F -- H: 3
G -- F: 5
G -- H: 4
\end{liGraphenFormat}

\begin{tikzpicture}[li graph]
\node (A) at (0,5) {A};
\node (B) at (8,5) {B};
\node (C) at (4,4) {C};
\node (D) at (2,3) {D};
\node (E) at (6,3) {E};
\node (F) at (4,1) {F};
\node (G) at (0,0) {G};
\node (H) at (8,0) {H};

\path (A) edge node {5} (B);
\path (A) edge node {8} (C);
\path (A) edge node {7} (D);
\path (A) edge node {1} (G);
\path (B) edge node {2} (C);
\path (B) edge node {7} (E);
\path (B) edge node {2} (H);
\path (C) edge node {1} (D);
\path (C) edge node {3} (E);
\path (C) edge node {6} (F);
\path (D) edge node {2} (F);
\path (D) edge node {8} (G);
\path (E) edge node {6} (H);
\path (F) edge node {5} (E);
\path (F) edge node {3} (H);
\path (G) edge node {5} (F);
\path (G) edge node {4} (H);
\end{tikzpicture}

\begin{enumerate}
\item Welches Gewicht hat der Spannbaum insgesamt?

\begin{liAntwort}
\begin{liGraphenFormat}
A: 0 5
B: 8 5
C: 4 4
D: 2 3
E: 6 3
F: 4 1
G: 0 0
H: 8 0
A -- B: 5
A -- C: 8
A -- D: 7
A -- G*
B -- C*: 2
B -- E: 7
B -- H*: 2
C -- D*
C -- E*: 3
C -- F: 6
D -- F*: 2
D -- G: 8
E -- H: 6
F -- E: 5
F -- H: 3
G -- F: 5
G -- H*: 4
\end{liGraphenFormat}

\begin{tikzpicture}[li graph]
\node (A) at (0,5) {A};
\node (B) at (8,5) {B};
\node (C) at (4,4) {C};
\node (D) at (2,3) {D};
\node (E) at (6,3) {E};
\node (F) at (4,1) {F};
\node (G) at (0,0) {G};
\node (H) at (8,0) {H};

\path (A) edge node {5} (B);
\path (A) edge node {8} (C);
\path (A) edge node {7} (D);
\path[li markierung] (A) edge node {1} (G);
\path[li markierung] (B) edge node {2} (C);
\path (B) edge node {7} (E);
\path[li markierung] (B) edge node {2} (H);
\path[li markierung] (C) edge node {1} (D);
\path[li markierung] (C) edge node {3} (E);
\path (C) edge node {6} (F);
\path[li markierung] (D) edge node {2} (F);
\path (D) edge node {8} (G);
\path (E) edge node {6} (H);
\path (F) edge node {5} (E);
\path (F) edge node {3} (H);
\path (G) edge node {5} (F);
\path[li markierung] (G) edge node {4} (H);
\end{tikzpicture}

\begin{center}
\begin{tabular}{|l|l|r|}
\hline
Kante & & Gewicht\\\hline\hline
AG, CD     & $2 \times 1$ & $2$\\
BD, BH, DF & $3 \times 2$ & $6$\\
CE         & $1 \times 3$ & $3$\\
GH         & $1 \times 4$ & $4$\\\hline
           &              & $15$\\\hline
\end{tabular}
\end{center}
\end{liAntwort}

\item Welchen Algorithmus haben Sie zur Ermittlung eingesetzt?
\index{Algorithmus von Kruskal}
\begin{liAntwort}
Kruskal
\end{liAntwort}
\end{enumerate}
\end{document}
