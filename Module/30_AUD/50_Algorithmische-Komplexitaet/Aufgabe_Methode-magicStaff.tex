\documentclass{bschlangaul-aufgabe}
\bLadePakete{syntax,mathe}
\begin{document}
\bAufgabenTitel{Methode „magicStaff()“}

\section{Komplexität
\index{Algorithmische Komplexität (O-Notation)}
\footcite[Aufgabe 5]{aud:e-klausur}
}

Welche Komplexität hat das Programmfragment?

\bJavaDatei[firstline=5,lastline=17]{aufgaben/aud/komplexitaet/Komplexitaet}

\noindent
Bestimmen Sie in Abhängigkeit von $n$ die Komplexität des
Programmabschnitts im

\begin{enumerate}
\item Best-Case.

\begin{liAntwort}
$\mathcal{O}(1)$: Wenn die erste Zahl im Feld \bJavaCode{array} ohne Rest
durch 3 teilbar ist, wird sofort aus der for-Schleife ausgestiegen
(wegen der \bJavaCode{break} Anweisung).
\end{liAntwort}

\item Worst-Case.
\begin{liAntwort}
$\mathcal{O}(n^2)$: Wenn keine Zahl aus \bJavaCode{array} ohne Rest
durch 3 teilbar ist, werden zwei Schleifen (\bJavaCode{for} und
\bJavaCode{do while}) über die Anzahl \bJavaCode{n} der Elemente des
Felds durchlaufen.
\end{liAntwort}
\end{enumerate}

\end{document}
