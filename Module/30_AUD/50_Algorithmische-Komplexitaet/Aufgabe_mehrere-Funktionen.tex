\documentclass{bschlangaul-aufgabe}
\bLadePakete{mathe}
\begin{document}
\bAufgabenTitel{mehrere Funktionen}
\section{Aufgabe zur Komplexität
\index{Algorithmische Komplexität (O-Notation)}
\footcite[Seite 1-2, Aufgabe 2]{aud:pu:2}}

\noindent
Geben Sie die Komplexität folgender Funktionen in der
$\mathcal{O}$-Notation an!

\begin{enumerate}
\item $x(n) = 4 \cdot n$

\begin{liAntwort}
$\mathcal{O}(n)$
\end{liAntwort}

\item $a(n) = n^2$

\begin{liAntwort}
$\mathcal{O}(n^2)$
\end{liAntwort}

\item $k(n) = 5 + n$

\begin{liAntwort}
$\mathcal{O}(n)$
\end{liAntwort}

\item $p(n) = 4$

\begin{liAntwort}
$\mathcal{O}(1)$
\end{liAntwort}

\item $j(n) = 4^n$

\begin{liAntwort}
$\mathcal{O}(4^n)$
\end{liAntwort}

\item $b(n) = \frac{n}{8}$

\begin{liAntwort}
$\mathcal{O}(n)$
\end{liAntwort}

\item $m(n) = \frac{1}{n}$

\begin{liAntwort}
$\mathcal{O}(1)$
\end{liAntwort}
\end{enumerate}
\end{document}
