\documentclass{bschlangaul-aufgabe}
\bLadePakete{mathe}
\begin{document}
\bAufgabenMetadaten{
  Titel = {Polynome},
  Thematik = {Polynome f(n) in g(n)},
  RelativerPfad = Module/30_AUD/50_Algorithmische-Komplexitaet/Aufgabe_Polynome.tex,
  BearbeitungsStand = unbekannt,
  Korrektheit = unbekannt,
  Stichwoerter = {Algorithmische Komplexität (O-Notation)},
}

Gegeben sind die zwei Funktionen. Gilt $f(n) \in \Theta(g(n))$?
\index{Algorithmische Komplexität (O-Notation)}
\bFussnoteUrl{https://www.informatik.hu-berlin.de/de/forschung/gebiete/wbi/teaching/archive/SS17/ue_algodat/schaefer01.pdf}

\begin{align*}
f(n) &= 3n^5 + 4n^3 + 15\\
g(n) &= n^5\\
\end{align*}
\begin{liAntwort}
\begin{enumerate}

\item Zu zeigen: $f(n) \in \mathcal{O}(g(n)) \Leftrightarrow
(
  \exists c,n_0 > 0
  \forall n_0 \geq n_0:
  3n^5 + 4n^3 + 15 \leq c \cdot n^5
)$

Wähle \zB $c = 3 + 4 + 15 = 22$,

dann gilt $\forall n \geq 1:
3n^5 + 4n^3 + 15 \leq
3n^5 + 4n^5 + 15n^5 =
\leq 22n^5$

$\Rightarrow f(n) \in \mathcal{O}(n^5)$

\bigskip

\item Zu zeigen: $f(n) \in \Omega(g(n)) \Leftrightarrow
(
  \exists c', n_0 > 0
  \forall n_0 \geq n_0:
  3n^5 + 4n^3 + 15 \leq c' \cdot n^5
)$

Wähle \zB $c' = 3$,

dann gilt $\forall n \geq 1: 3n^5 = + 4n^3 + 15 \leq 3n^5$

$\Rightarrow f(n) \in \Omega(n^5)$

\end{enumerate}

\bigskip

$\Rightarrow f(n) \in \Theta(g(n))$
\end{liAntwort}
\end{document}
