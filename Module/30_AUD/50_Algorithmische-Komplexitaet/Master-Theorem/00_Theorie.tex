\documentclass{bschlangaul-theorie}
\liLadePakete{mathe,master-theorem}

\begin{document}
\let\O=\liO
\let\o=\liOmega
\let\T=\liT
\let\t=\liTheta

%%%%%%%%%%%%%%%%%%%%%%%%%%%%%%%%%%%%%%%%%%%%%%%%%%%%%%%%%%%%%%%%%%%%%%%%
% Theorie-Teil
%%%%%%%%%%%%%%%%%%%%%%%%%%%%%%%%%%%%%%%%%%%%%%%%%%%%%%%%%%%%%%%%%%%%%%%%

\section{Master-Theorem}

Der Hauptsatz der Laufzeitfunktionen – oder oft auch aus dem Englischen
als Master-Theorem entlehnt – bietet eine \memph{schnelle Lösung} für
die Frage, \memph{in welcher Laufzeitklasse} eine gegebene
\memph{rekursiv definierte Funktion} liegt. Mit dem Master-Theorem kann
allerdings \memph{nicht jede rekursiv definierte Funktion} gelöst
werden. Lässt sich keiner der \memph{drei möglichen Fälle} des
Master-Theorems auf die Funktion $T$ anwenden, so muss man die
Komplexitätsklasse der Funktion anderweitig berechnen.

\liMasterVariablen

\liMasterFaelle

\liPseudoUeberschrift{Merkhilfen:}

\noindent
Einsetzen von f(n)

\begin{enumerate}
\item $\mathcal{O} - \varepsilon$
\item $\Theta$
\item $\Omega + \varepsilon$ und  $a \cdot f(\textstyle {\frac {n}{b}})\leq c \cdot f(n)$
\end{enumerate}

\noindent
Ergebnisse

\begin{enumerate}
\item $\Theta(n^x)$
\item $\Theta(n^x \log n)$
\item $\Theta(f(n))$
\end{enumerate}

\literatur
\end{document}
