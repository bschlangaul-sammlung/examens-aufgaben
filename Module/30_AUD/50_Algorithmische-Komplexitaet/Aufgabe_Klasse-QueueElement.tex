\documentclass{lehramt-informatik-aufgabe}
\liLadePakete{syntax}
\begin{document}
\liAufgabenTitel{Klasse-QueueElement}

\section{Aufgabe zur Komplexität
\index{Algorithmische Komplexität (O-Notation)}
\footcite[Seite 3, Aufgabe 3]{aud:pu:2}}

\noindent
Der Konstruktor \liJavaCode{QueueElement(...)} und die Methode
\liJavaCode{setNext(...)} sowie \liJavaCode{getNext(...)} haben
$\mathcal{O}(1)$. Geben Sie die Zeitkomplexität der Methode
\liJavaCode{append(int content)} an, die einer Schlange ein neues
Element anhängt.

\begin{minted}{java}
public void append(int contents) {
QueueElement newElement = new QueueElement(contents) ;
  if (first == 0) {
    first = newElement;
    last = newElement;
  } else {
    // Ein neues Element hinten anhängen.
    last.setNext(newElement);
    // Das angehängte Element als Letztes setzen.
    last = last.getNext();
  }
}
\end{minted}

\begin{liAntwort}
Das Anhängen eines neuen Elements in die gegebene Warteschlange hat
die konstanten Rechenzeitbedarf von $\mathcal{O}(1)$, egal wie lange die
Schlange ist, da wir das letzte Element direkt ansprechen können.
\end{liAntwort}

\end{document}
