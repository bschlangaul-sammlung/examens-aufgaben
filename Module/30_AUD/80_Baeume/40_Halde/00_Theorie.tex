\documentclass{bschlangaul-theorie}
\usepackage{tabularx}
\usepackage{float}
\bLadePakete{baum,mathe,syntax}

\begin{document}

%%%%%%%%%%%%%%%%%%%%%%%%%%%%%%%%%%%%%%%%%%%%%%%%%%%%%%%%%%%%%%%%%%%%%%%%
% Theorie-Teil
%%%%%%%%%%%%%%%%%%%%%%%%%%%%%%%%%%%%%%%%%%%%%%%%%%%%%%%%%%%%%%%%%%%%%%%%

\chapter{Halden}

\begin{bQuellen}
\item \cite[Seite 25-32]{aud:fs:tafeluebung-11}
\item \cite{wiki:heap}
\item \cite[Seite 407-409 Kapitel 14.6.1]{saake}
\end{bQuellen}

\section{Visualisierungstools}

\begin{itemize}
\item Min-Heap: \url{https://www.cs.usfca.edu/~galles/visualization/Heap.html}
\item Max-Heap: \url{https://visualgo.net/en/heap}
\item Max-Heap: \url{http://btv.melezinek.cz/binary-heap.html}
\end{itemize}

\noindent
Ein Heap (englisch wörtlich: \bEmph{Haufen} oder \bEmph{Halde}) ist
eine auf Bäumen basierende \bEmph{abstrakte} und
\bEmph{dynamische} Datenstruktur.
%
Der Begriff Heap wird häufig als bedeutungsgleich zu einem
\bEmph{partiell geordneten Binärbaum} verstanden.
\footcite{wiki:heap}

Die Datenstruktur Heap bezeichnet einen binären Baum, der folgende
Eigenschaften erfüllt:\footcite[Seite 407, Kapitel 14.6.1]{saake}

\begin{itemize}
\item Der Baum ist vollständig, \dh, die Blattebene ist von links
nach rechts gefüllt.

\item Der Schlüssel eines jeden Knotens ist kleiner (oder gleich) als
die Schlüssel seiner Kinder. Diese partielle Ordnung wird auch als
Heap-Eigenschaft bezeichnet.
\end{itemize}

Es gibt zwei unterschiedlichen Arten von Halden, nämlich:

\begin{itemize}
\item ein \bEmph{Max-Heap}. Hier ist die Wurzel jedes Teilbaums ist das
\bEmph{Maximum} aller Knoten des Teilbaums.
%
\item eine \bEmph{Min-Heap}: Die Wurzel jedes Teilbaums ist das
\bEmph{Minimum} aller Knoten des Teilbaums.
\end{itemize}
%
\bEmph{Duplikate} sind in einem Heap \bEmph{erlaubt}.
%
Heaps erlauben einen schnellen Zugriff auf das \bEmph{größte} bzw.
\bEmph{kleinste} Element. Sie werden deshalb als
\bEmph{Prioritätswarteschlange} verwendent.
%
Heaps können als „klassischer“ Binärbaum implementiert werden. Eine
effiziente Speicherung ist in einem \bEmph{Array} möglich.
\footcite[Seite 22]{aud:fs:tafeluebung-11}

%-----------------------------------------------------------------------
%
%-----------------------------------------------------------------------

\section{Links-Vollständigkeit}

Oft (nicht immer) verstehen wir unter Heaps links-vollständige Bäume:
\dh alle „Ebenen“ (bis auf die unterste) sind \bEmph{voll besetzt}
und auf unterster „Ebene“ sitzen alle Knoten \bEmph{soweit links wie
möglich}. Dadurch ist eine \bEmph{lückenlose Darstellung} in einem
\bEmph{Array} möglich (sog. Feld-Einbettung). Nicht nur Halden können
linksvollständig sein, sondern jeder beliebige Baum.

%-----------------------------------------------------------------------
%
%-----------------------------------------------------------------------

\subsection{Berechnung der Indizes}

\begin{compactitem}
\item \emph{Wurzel} steht an Position $0$

\item \emph{Kinder} von Knoten an Position $i$ stehen an Stelle
$2 \cdot i + 1$ und $2 \cdot i + 2$

\item \emph{Elternknoten} von Knoten an Position $i$ steht an Stelle
$\frac{i - 1}{2}$
\footcite[Seite 26]{aud:fs:tafeluebung-11}
\end{compactitem}

\begin{center}
\begin{tikzpicture}[b binaer baum]
\Tree
[.1
  [.12
    [.15
      [.24 ]
      [.18 ]
    ]
    [.13
      [.17 ]
      [.19 ]
    ]
  ]
  [.2
    [.5
      [.28 ]
      \edge[blank]; \node[blank]{};
    ]
    [.42 ]
  ]
]
\end{tikzpicture}
\end{center}

%-----------------------------------------------------------------------
%
%-----------------------------------------------------------------------

\section{Operationen}

%%
%
%%

\subsection{Einfügen eines neuen Elements}

Ein neues Element wird an die \bEmph{nächste freie Position in der
untersten Ebene einfügt}. Falls die Ebene \bEmph{voll} ist, wird das
neue Element \bEmph{erster Knoten einer neuen Ebene}. Solange die
Halden-Eigenschaft in einem Teilbaum verletzt ist, lassen wir das neue
Element entsprechend der Halden-Eigenschaft \bEmph{nach oben wandern}.

%%
%
%%

\subsection{Löschen eines Elements}

Wir \bEmph{ersetzen} das zu löschende Element mit dem \bEmph{letzten
Element der untersten Ebene}.
%
Solange die Halden-Eigenschaft in einem Teilbaum verletzt ist, lassen
wir das neue Element entsprechend der Halden-Eigenschaft \bEmph{nach
unten sickern}.
%
Bei einer \bEmph{Min-Heap} wird mit dem \bEmph{kleineren Kind}
getauscht, bei einer \bEmph{Max-Heap} mit dem \bEmph{größeren Kind}.
\footcite[Seite 28]{aud:fs:tafeluebung-11}

\bJavaDatei[firstline=3]{baum/Halde}

\bJavaDatei[firstline=3]{sortier/Heap}

\literatur

\end{document}
