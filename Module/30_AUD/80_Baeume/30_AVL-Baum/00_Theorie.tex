\documentclass{bschlangaul-theorie}
\bLadePakete{mathe,baum,syntax,spalten,tabelle}
\usepackage{xparse}

\begin{document}

%%%%%%%%%%%%%%%%%%%%%%%%%%%%%%%%%%%%%%%%%%%%%%%%%%%%%%%%%%%%%%%%%%%%%%%%
% Theorie-Teil
%%%%%%%%%%%%%%%%%%%%%%%%%%%%%%%%%%%%%%%%%%%%%%%%%%%%%%%%%%%%%%%%%%%%%%%%

\section{AVL-Bäume}

\begin{liQuellen}
\item \cite{wiki:avl-baum}
\item \cite[Kapitel 14.4.2, Seite 378-386 (PDF 394-402)]{saake}
\item \cite[bst]{net:html:visualgo}
\end{liQuellen}

%-----------------------------------------------------------------------
%
%-----------------------------------------------------------------------

\section{Visualisierungstools}

\begin{itemize}
\item \url{https://www.cs.usfca.edu/~galles/visualization/AVLtree.html}
\item \url{https://visualgo.net/bn/bst} (oben in den Tabs umschalten auf AVL)
\end{itemize}

\noindent
Ein AVL-Baum ist ein binärer Suchbaum, der \memph{höhenbalanciert} ist,
\dh für jeden Knoten gilt:

\begin{center}
$|h_\text{ rechterTeilbaum} - h_\text{ linkerTeilbaum}| \leq 1$
\end{center}

\noindent
Die \memph{„Entartung“} des Baums wird so vermieden. Die Höhe eines
AVL-Baums ist $h \in \mathcal{O}(\log n)$. Beim Einfügen und Löschen von
Knoten muss die AVL-Eigenschaft durch \memph{Rotationen}
wiederhergestellt werden.
\footcite[Seite 22 (PDF 16)]{aud:fs:5}

\begin{liProjektSprache}{Baum}
baum binär (
  setze: 1 2 3 4 5;
  drucke;
)

baum avl (
  setze: 1 2 3 4 5;
  drucke;
)
\end{liProjektSprache}

\begin{multicols}{2}
\bPseudoUeberschrift{Binärer Suchbaum}

\begin{center}
\begin{tikzpicture}[li binaer baum]
\Tree
[.\node[label=+4]{1};
  \edge[blank]; \node[blank]{};
  [.\node[label=+3]{2};
    \edge[blank]; \node[blank]{};
    [.\node[label=+2]{3};
      \edge[blank]; \node[blank]{};
      [.\node[label=+1]{4};
        \edge[blank]; \node[blank]{};
        [.\node[label=0]{5}; ]
      ]
    ]
  ]
]
\end{tikzpicture}
\end{center}

\columnbreak

\bPseudoUeberschrift{AVL-Baum}

\begin{center}
\begin{tikzpicture}[li binaer baum]
\Tree
[.\node[label=+1]{2};
  [.\node[label=0]{1}; ]
  [.\node[label=0]{4};
    [.\node[label=0]{3}; ]
    [.\node[label=0]{5}; ]
  ]
]
\end{tikzpicture}
\end{center}
\end{multicols}

%-----------------------------------------------------------------------
%
%-----------------------------------------------------------------------

\section{Rotationsregeln}

\begin{center}
$b_\text{ Balance-Faktor} = h_\text{ rechter Teilbaum} - h_\text{ linker Teilbaum}$
\end{center}

\bPseudoUeberschrift{Erklärungen}

\begin{description}
\item[$bO$] Balance-Faktor des „oberen“ Wurzelknotens
\item[$bU$] Balance-Faktor des Kindknoten von $bO$
\end{description}

\begin{description}
\item[$-$] Rechtsdrehung

\item[$+$] Linksdrehung
\end{description}

\begin{center}
\noindent
\begin{tabularx}{\linewidth}{l|X}
$bU = -1$, $bO = -2$ &
Rechtsrotation (Rechtsdrehung um $bO$) \\

$bU = +1$, $bO = +2$ &
Linksrotation (Linksdrehung um $bO$) \\

$bU = +1$, $bO = -2$ &
Links-Rechts-Rotation (Linksdrehung um $bU$, dann Rechtsdrehung um $bO$) \\

$bU = -1$, $bO = +2$ &
Rechts-Links-Rotation (Rechtsdrehung um $bU$, dann Linksdrehung um $bO$) \\
\end{tabularx}
\end{center}

\noindent
Die Kindknoten, die nach der Drehung „im Weg stehen“, werden
„umgeklappt“, also vom linken zum rechten Kind der nächsten Ebene
„umgehängt“ und umgekehrt.

%-----------------------------------------------------------------------
%
%-----------------------------------------------------------------------

\section{Rotation (genaue Darstellung): Linksrotation von k.}

\begin{center}
\begin{tabular}{l|l|l}
 &  vorher & nachher \\\hline
 Wurzel                         & k               & k.links \\
 linkes	Kind der Wurzel         & k.links         & k.links.links \\
 rechte	Kind der Wurzel         & k.rechts        & k \\\hline

 li. Ki. des li. Ki. der Wurzel & k.links.links   & k.links.links.links \\
 re. Ki. des li. Ki. der Wurzel & k.links.rechts  & k.links.links.rechts \\\hline

 li. Ki. des re. Ki. der Wurzel & k.rechts.left   & k.links.rechts \\
 re. Ki. des re. Ki. der Wurzel & k.rechts.rechts & k.rechts\\\hline
\end{tabular}
\end{center}

\noindent
Die „nachher“-Position bezieht sich immer auf den den neuen
Wurzelknoten, der Pfad zu dem entsprechenden Knoten beginnt immer beim
alten Wurzelknoten k.
\bFussnoteLink{hs-mannheim.de}{https://services.informatik.hs-mannheim.de/~schramm/ads/files/Kapitel10_02.pdf}

%%
%
%%

\section{Linksrotation 1 2 3}

\begin{multicols}{3}
\begin{liDiagramm}{Einfügen von „1“}
\begin{tikzpicture}[li binaer baum]
\Tree
[.\node[label=0]{1}; ]
\end{tikzpicture}
\end{liDiagramm}

\begin{liDiagramm}{Einfügen von „2“}
\begin{tikzpicture}[li binaer baum]
\Tree
[.\node[label=+1]{1};
  \edge[blank]; \node[blank]{};
  [.\node[label=0]{2}; ]
]
\end{tikzpicture}
\end{liDiagramm}

\begin{liDiagramm}{Einfügen von „3“}
\begin{tikzpicture}[li binaer baum]
\Tree
[.\node[label=+2]{1};
  \edge[blank]; \node[blank]{};
  [.\node[label=+1]{2};
    \edge[blank]; \node[blank]{};
    [.\node[label=0]{3}; ]
  ]
]
\end{tikzpicture}
\end{liDiagramm}

\begin{liDiagramm}{Linksrotation}
\begin{tikzpicture}[li binaer baum]
\Tree
[.\node[label=0]{2};
  [.\node[label=0]{1}; ]
  [.\node[label=0]{3}; ]
]
\end{tikzpicture}
\end{liDiagramm}
\end{multicols}

\section{Rechtsrotation 3 2 1}

\begin{multicols}{3}
\begin{liDiagramm}{Einfügen von „3“}
\begin{tikzpicture}[li binaer baum]
\Tree
[.\node[label=0]{3}; ]
\end{tikzpicture}
\end{liDiagramm}

\begin{liDiagramm}{Einfügen von „2“}
\begin{tikzpicture}[li binaer baum]
\Tree
[.\node[label=-1]{3};
  [.\node[label=0]{2}; ]
  \edge[blank]; \node[blank]{};
]
\end{tikzpicture}
\end{liDiagramm}

\begin{liDiagramm}{Einfügen von „1“}
\begin{tikzpicture}[li binaer baum]
\Tree
[.\node[label=-2]{3};
  [.\node[label=-1]{2};
    [.\node[label=0]{1}; ]
    \edge[blank]; \node[blank]{};
  ]
  \edge[blank]; \node[blank]{};
]
\end{tikzpicture}
\end{liDiagramm}

\begin{liDiagramm}{Rechtsrotation}
\begin{tikzpicture}[li binaer baum]
\Tree
[.\node[label=0]{2};
  [.\node[label=0]{1}; ]
  [.\node[label=0]{3}; ]
]
\end{tikzpicture}
\end{liDiagramm}

\end{multicols}

%%
%
%%

\section{Links-Rechtsrotation 3 1 2}

\begin{multicols}{3}
\begin{liDiagramm}{Einfügen von „3“}
\begin{tikzpicture}[li binaer baum]
\Tree
[.\node[label=0]{3}; ]
\end{tikzpicture}
\end{liDiagramm}

\begin{liDiagramm}{Einfügen von „1“}
\begin{tikzpicture}[li binaer baum]
\Tree
[.\node[label=-1]{3};
  [.\node[label=0]{1}; ]
  \edge[blank]; \node[blank]{};
]
\end{tikzpicture}
\end{liDiagramm}

\begin{liDiagramm}{Einfügen von „2“}
\begin{tikzpicture}[li binaer baum]
\Tree
[.\node[label=-2]{3};
  [.\node[label=+1]{1};
    \edge[blank]; \node[blank]{};
    [.\node[label=0]{2}; ]
  ]
  \edge[blank]; \node[blank]{};
]
\end{tikzpicture}
\end{liDiagramm}

\begin{liDiagramm}{Linksrotation}
\begin{tikzpicture}[li binaer baum]
\Tree
[.\node[label=-2]{3};
  [.\node[label=-1]{2};
    [.\node[label=0]{1}; ]
    \edge[blank]; \node[blank]{};
  ]
  \edge[blank]; \node[blank]{};
]
\end{tikzpicture}
\end{liDiagramm}

\begin{liDiagramm}{Rechtsrotation}
\begin{tikzpicture}[li binaer baum]
\Tree
[.\node[label=0]{2};
  [.\node[label=0]{1}; ]
  [.\node[label=0]{3}; ]
]
\end{tikzpicture}
\end{liDiagramm}
\end{multicols}

%%
%
%%

\section{Rechts-Linksrotation 1 3 2}

\begin{multicols}{3}
\begin{liDiagramm}{Einfügen von „1“}
\begin{tikzpicture}[li binaer baum]
\Tree
[.\node[label=0]{1}; ]
\end{tikzpicture}
\end{liDiagramm}

\begin{liDiagramm}{Einfügen von „3“}
\begin{tikzpicture}[li binaer baum]
\Tree
[.\node[label=+1]{1};
  \edge[blank]; \node[blank]{};
  [.\node[label=0]{3}; ]
]
\end{tikzpicture}
\end{liDiagramm}

\begin{liDiagramm}{Einfügen von „2“}
\begin{tikzpicture}[li binaer baum]
\Tree
[.\node[label=+2]{1};
  \edge[blank]; \node[blank]{};
  [.\node[label=-1]{3};
    [.\node[label=0]{2}; ]
    \edge[blank]; \node[blank]{};
  ]
]
\end{tikzpicture}
\end{liDiagramm}

\begin{liDiagramm}{Rechtsrotation}
\begin{tikzpicture}[li binaer baum]
\Tree
[.\node[label=+2]{1};
  \edge[blank]; \node[blank]{};
  [.\node[label=+1]{2};
    \edge[blank]; \node[blank]{};
    [.\node[label=0]{3}; ]
  ]
]
\end{tikzpicture}
\end{liDiagramm}

\begin{liDiagramm}{Linksrotation}
\begin{tikzpicture}[li binaer baum]
\Tree
[.\node[label=0]{2};
  [.\node[label=0]{1}; ]
  [.\node[label=0]{3}; ]
]
\end{tikzpicture}
\end{liDiagramm}
\end{multicols}

\section{Löschen}

Um einen Schlüsselwert in einem AVL-Baum zu löschen, wird der
\memph{Löschalgorithmus der binären Suchbäume} angewendet. Wie beim
Einfügen in AVL-Bäumen muss der Suchbaum nun gegebenenfalls
\memph{rebalanciert} werden. Dazu wird der Weg vom Elternknoten des
gelöschten Knotens zur Wurzel zurückgelaufen und die Balance jedes
Knotens überprüft. Wenn der gelöschte Schlüsselwert in einem Knoten mit
zwei Nachfolgern gespeichert war, muss bei Elternknoten des
symmetrischen Nachfolgers begonnen werden. \bFussnoteLink{Universität
Würzburg}{https://www1.pub.informatik.uni-wuerzburg.de/databases/Binaere_Suchbaeume/avl_trees/dictionary/delete.html}

\section{Laufzeit bzw. Komplexität}

\begin{tabular}{lll}
         & Average-Case & Worst-Case \\
Suchen   & $\mathcal{O}(\log n)$     & $\mathcal{O}(\log n)$         \\
Einfügen & $\mathcal{O}(1)$          & $\mathcal{O}(\log n)$         \\
Löschen  & $\mathcal{O}(1)$          & $\mathcal{O}(\log n)$
\end{tabular}
\bFussnoteLink{studyflix.de}{https://studyflix.de/informatik/avl-baum-1434}

%%
%
%%

\subsection{Rechtsrotation}

\bJavaDatei[firstline=85,lastline=102]{baum/AVLBaum}

%%
%
%%

\subsection{Linksrotation}

\bJavaDatei[firstline=104,lastline=121]{baum/AVLBaum}

%-----------------------------------------------------------------------
%
%-----------------------------------------------------------------------

\section{Die Klasse AVLBaum}

\bJavaDatei[firstline=3]{baum/AVLBaum}

\literatur

\end{document}
