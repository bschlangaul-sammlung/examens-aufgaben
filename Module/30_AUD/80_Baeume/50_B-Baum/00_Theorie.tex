\documentclass{bschlangaul-haupt}
\liLadePakete{baum,syntax,spalten}

\begin{document}

%%%%%%%%%%%%%%%%%%%%%%%%%%%%%%%%%%%%%%%%%%%%%%%%%%%%%%%%%%%%%%%%%%%%%%%%
% Theorie-Teil
%%%%%%%%%%%%%%%%%%%%%%%%%%%%%%%%%%%%%%%%%%%%%%%%%%%%%%%%%%%%%%%%%%%%%%%%

% AUD_5_Teil1.m4v 43min
\chapter{B-Bäume}

\begin{liQuellen}
\item \cite{wiki:bbaum}
\item \cite[Kapitel 14.4.3, Seite 386-399 (PDF 402-415)]{saake}
\item \cite[Kapitel 13.5.4.2 Balancierte Mehrwegbäume, Seite 464, wird
nur erwähnt, nicht beschrieben]{schneider}
\item \cite[7.8 B-Bäume Seite 224-228]{kemper}
\end{liQuellen}

\url{https://www.cs.usfca.edu/~galles/visualization/BTree.html}
Max. Degree = 5 entspricht k = 2

\noindent
Eine ausgeglichene Baumstruktur ist der von R. Bayer und E. McCreight
entwickelte B-Baum. Hierbei \memph{steht der Name „B“ für balanciert},
breit, buschig oder auch Bayer, nicht jedoch für binär. Die Grundidee
des B-Baumes ist es gerade, dass der Verzweigungsgrad variiert, während
die Baumhöhe vollständig ausgeglichen ist.
\footcite[Seite 386]{saake}

%-----------------------------------------------------------------------
%
%-----------------------------------------------------------------------

\section{Definition}

Ein Baum heißt genau dann B-Baum, wenn gilt:
%

\begin{enumerate}
\item Jeder Knoten außer der Wurzel enthält zwischen \memph{$k$ und $2k$
Elemente} (Schlüsselwerte), $k$ wird als \memph{Ordnung} des B-Baums
bezeichnet.
%
\item Jeder \memph{Knoten ist entweder ein Blatt (ohne Kinder)} oder hat
\memph{mindestens $k + 1$ und höchstens $2k + 1$ Kind-Knoten}.
%
\item Der Wurzelknoten ist \memph{entweder ein Blatt oder hat mindestens
2 Nachfolger}.
%
\item Alle Blätter haben die \memph{gleiche Tiefe}, \dh alle Wege von
der Wurzel bis zu den Blättern sind gleich lang. Pfade haben die Länge
$h - 1$, wobei $h$ die Höhe des gesamten Baums ist.
\end{enumerate}
\footcite[Seite 32]{aud:fs:5}

%-----------------------------------------------------------------------
%
%-----------------------------------------------------------------------

\section{Einfügen}

Das Einfügen in einen B-Baum erfolgt \memph{nur in den Blattknoten}.
Wenn in einem Blattknoten die \memph{maximale Anzahl} von Elementen
($2k$) erreicht ist, findet ein \memph{Split} statt, \dh die
Elemente werden aufgeteilt und ein neuer Knoten entsteht. Das
\memph{mittlere Element} des ursprünglichen Knotens wird dabei \memph{in
den Elternknoten integriert}.
\footcite[Seite 32]{aud:fs:5}

%-----------------------------------------------------------------------
%
%-----------------------------------------------------------------------

\section{Suchen}

Beginnend mit dem Wurzelknoten werden die Knoten jeweils \memph{von
links nach rechts} durchsucht:
%
\memph{Stimmt} ein Element mit dem gesuchtem Schlüsselwert
\memph{überein}, ist der Satz \memph{gefunden}.
%
Ist das \memph{Element größer} als der gesuchte Wert, wird die
Suche im \memph{links} hängenden Unterbaum \memph{fortgesetzt}.
%
Ist das \memph{Element kleiner} als der gesuchte Wert, wird der
Vergleich mit dem \memph{nächsten Element der Wurzel wiederholt}.
%
Ist auch das letztes Element der Wurzel noch kleiner als der gesuchte
Wert, dann wird die Suche im \memph{rechten Unterbaum} des Elements
fortgesetzt.
%
Falls ein weiterer Abstieg in den Unterbaum nicht möglich ist
(\dh Blattknoten), wird die Suche abgebrechen. Dann ist kein Satz mit
dem gewünschten Schlüsselwert vorhanden.
\footcite[Seite 37]{aud:fs:5}

%-----------------------------------------------------------------------
%
%-----------------------------------------------------------------------

% AUD_5_Teil1.m4v 55min
\section{Löschen}

Suche den Knoten, in dem das zu löschende Element
liegt.\footcite[Seite 39]{aud:fs:5}

\subsection{Löschen aus Blattknoten}

\subsubsection{Verschiebung / Rotation / Ausgleichen}

Enthält der für den Abstieg ausgewählte Unterbaum nur die minimale
Schlüsselanzahl, aber ein vorausgehender oder nachfolgender
Geschwisterknoten hat genügend Schlüssel, wird ein Schlüssel in den
ausgewählten Knoten verschoben.
\footcite{wiki:bbaum}

\subsubsection{Verschmelzung / Mischen}

Enthalten sowohl der für den Abstieg ausgewählte Unterbaum  als auch
sein unmittelbar vorausgehender und nachfolgender Geschwisterknoten
\memph{genau die minimale Schlüsselanzahl}, ist eine Verschiebung nicht
möglich. In diesem Fall wird eine Verschmelzung des ausgewählten
Unterbaumes mit dem vorausgehenden oder nachfolgenden
Geschwisterknoten durchgeführt. Dazu wird der Schlüssel aus dem
\memph{Vaterknoten}, welcher die Wertebereiche der Schlüssel in den beiden zu
verschmelzenden Knoten trennt, als mittlerer Schlüssel in den
verschmolzenen Knoten verschoben. Die beiden Verweise auf die jetzt
verschmolzenen Kindknoten werden durch einen Verweis auf den neuen
Knoten ersetzt.
\footcite{wiki:bbaum}

Ein Unterlauf entsteht auf Blattebene. Der Unterlauf wird durch Mischen
des Unterlaufknotens mit seinem Nachbarknoten und dem darüberliegenden
Elemen  durchgeführt, dabei wird sozusagen ein Splitt rückwärts
durchgeführt. Wurde einmal mit dem Mischen auf Blattebene begonnen, muss
das evtl. nach oben fortgesetzt werden. Mischen wird so lange
fortgesetzt, bis kein Unterlauf mehr existiert oder die Wurzel erreicht
ist. Wird die Wurzel erreicht, kann der Baum in der Höhe um 1
schrumpfen. Beim Mischen kann auch wieder ein Überlauf entstehen, \dh
es muss wieder gesplittet werde.
\footcite[Seite 40]{aud:fs:5}

\subsection{Löschen aus inneren Knoten}

Wird der zu löschende Schlüssel bereits in einem inneren Knoten
gefunden, kann dieser nicht direkt gelöscht werden, weil er für die
Trennung der Wertebereiche seiner beiden Unterbäume benötigt wird. In
diesem Fall wird sein \memph{symmetrischer Vorgänger} (oder sein
symmetrischer \memph{Nachfolger}) gelöscht und \memph{an seine Stelle
kopiert}. Der symmetrische Vorgänger ist der größte Blattknoten im
linken Unterbaum, befindet sich also dort ganz rechts außen. Der
symmetrische Nachfolger ist entsprechend der kleinste Blattknoten im
rechten Unterbaum und befindet sich dort ganz links außen. Die
\memph{Entscheidung}, in welchen Unterbaum der Abstieg für die Löschung
stattfindet, wird davon abhängig gemacht, \memph{welcher genügend
Schlüssel enthält}. Haben beide nur die minimale Schlüsselanzahl, werden
die Unterbäume verschmolzen. Damit wird keine Trennung der Wertebereiche
mehr benötigt und der Schlüssel kann direkt gelöscht werden.
\footcite{wiki:bbaum}

Falls das Element $E$ in einem inneren Knoten liegt, dann
untersuche den linken und rechten Unterbaum von $E$:

\begin{itemize}
\item Betrachte den Blattknoten mit dem direkten Vorgänger $E'$ von $E$
und den Blattknoten mit direktem Nachfolger $E''$ von $E$.

\item Wähle den Blattknoten aus, der mehr Elemente hat. Falls beide
Blattknoten gleich viele Elemente haben, wähle zufällig einen der beiden
aus.

\item Ersetze das zu löschende Element $E$ durch $E'$ bzw. $E''$ aus dem
gewählten Blattknoten.

\item Lösche $E'$ bzw. $E''$ im gewählten Blattknoten und behandle ggf.
entstehenden Unterlauf in diesem Blattknoten.\footcite[Seite
39]{aud:fs:5}
\end{itemize}

%-----------------------------------------------------------------------
%
%-----------------------------------------------------------------------

\section{2-3-4 B-Baum}

ein Baum, für den folgendes gilt: Er besitzt in einem Knoten
max. 3 Schlüssel-Einträge und 4 Kindknoten und minimal einen Schlüssel
und 2 Nachfolger\footcite[Seite 7]{aud:pu:7}

\liJavaDatei{baum/BBaum}

\literatur

\end{document}
