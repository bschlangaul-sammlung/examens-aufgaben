\documentclass{bschlangaul-theorie}
\bLadePakete{syntax}

\begin{document}

%%%%%%%%%%%%%%%%%%%%%%%%%%%%%%%%%%%%%%%%%%%%%%%%%%%%%%%%%%%%%%%%%%%%%%%%
% Theorie-Teil
%%%%%%%%%%%%%%%%%%%%%%%%%%%%%%%%%%%%%%%%%%%%%%%%%%%%%%%%%%%%%%%%%%%%%%%%

\chapter{Radixsort}

\begin{bQuellen}
\item \cite[Seite 68-75]{aud:fs:tafeluebung-11}
\item \cite{wiki:radixsort}
\end{bQuellen}

RadixSort
; verallgemeinertes Sortieren durch Fachverteilen
• Voraussetzung:
• Elemente sind Zeichenfolgen über einem endlichen Alphabet
Σ
• auf den Zeichen des Alphabets ist eine Totalordnung definiert
• für jedes Zeichen des Alphabets steht ein Fach zur Verfügung
• einzelne Stellen aller Elemente von hinten nach vorne abarbeiten:
Partitionieren
; Elemente abhängig von der aktuellen Stelle in das passende Fach legen
Einsammeln
; Elemente wieder aus den Fächern rausnehmen
• vom Fach für das „kleinste“ Zeichen zum Fach für das “größte “ Zeichen
• relative Reihenfolge der Elemente in einem Fach muss erhalten bleiben!

\bJavaDatei[firstline=3,lastline=64]{sortier/Radix}

\literatur

\end{document}
