\documentclass{bschlangaul-theorie}
\bLadePakete{sortieren,syntax}

\begin{document}

%%%%%%%%%%%%%%%%%%%%%%%%%%%%%%%%%%%%%%%%%%%%%%%%%%%%%%%%%%%%%%%%%%%%%%%%
% Theorie-Teil
%%%%%%%%%%%%%%%%%%%%%%%%%%%%%%%%%%%%%%%%%%%%%%%%%%%%%%%%%%%%%%%%%%%%%%%%

\chapter{BubbleSort: Blasensortierung}

\begin{bQuellen}
\item \cite[Seite 43]{aud:fs:tafeluebung-11}
\item \cite[Seite 129-131 (PDF 147-149)]{saake}
\item \cite{wiki:bubblesort}
\end{bQuellen}

\bVertauschen{3 2 5 1 4}

1. Durchgang

\bVertauschen{>3 <2 5 1 4}

\bVertauschen{2 3 >5 <1 4}

\bVertauschen{2 3 1 >5 <4}

2. Durchgang

\bVertauschen{2 >3 <1 4 5}

\bVertauschen{2 1 3 4 5}

3. Durchgang

\bVertauschen{>2 <1 3 4 5}

fertig

\bVertauschen{1 2 3 4 5}

\section{Funktionsweise}

\begin{itemize}
\item solange zu sortierende Liste nicht vollständig sortiert ist:
\item iteriere von vorne über die Elemente
\item falls zwei aufeinanderfolgende Elemente nicht sortiert sind: vertauschen
\item fertig, wenn während einer Runde keine Vertauschung passiert ist
\end{itemize}

\section{Eigenschaften}

\begin{itemize}
\item Laufzeitkomplexität:

\begin{itemize}
\item $\mathcal{O}(n)$ (im Best-Case)
\item $\mathcal{O}(n^2)$ (im Average- und Worst-Case)
\end{itemize}

\item stabil
\item in-situ
\end{itemize}

\bJavaDatei[firstline=10,lastline=31]{sortier/BubbleMinimal}

\bJavaDatei[firstline=9,lastline=28]{sortier/BubbleIterativ}
\footcite[Seite 130-131]{saake}

\bJavaDatei[firstline=11,lastline=31]{sortier/BubbleRekursiv}

\literatur

\end{document}
