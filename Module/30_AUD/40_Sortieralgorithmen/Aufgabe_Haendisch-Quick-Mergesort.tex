\documentclass{lehramt-informatik-aufgabe}

\begin{document}
\liAufgabenMetadaten{
  Titel = {Aufgabe 3: Sortieren I},
  Thematik = {Händisch Quick- und Mergesort},
  RelativerPfad = Module/30_AUD/40_Sortieralgorithmen/Aufgabe_Haendisch-Quick-Mergesort.tex,
  ZitatSchluessel = aud:pu:7,
  ZitatBeschreibung = {entnommen aus Algorithmen und Datenstrukturen, Übungsblatt 2, Universität Würzburg, Aufgabe 3},
  Stichwoerter = {Mergesort, Quicksort},
}

\liAufgabenTitel{Händisch Quick- und Mergesort}

\section{Aufgabe 3: Sortieren I\footcite[entnommen aus Algorithmen und
Datenstrukturen, Übungsblatt 2, Universität Würzburg, Aufgabe 3]{aud:pu:7}}

Gegeben ist folgende Zahlenfolge:

35, 22, 5, 3, 28, 16, 8, 60, 17, 66, 4, 9, 82, 11, 10, 20

\begin{enumerate}
\item Sortiere Sie händisch mit Mergesort\index{Mergesort}. Orientieren
Sie sich beim Aufschreiben der Zwischenschritte an dieser Darstellung:

\item Sortiere Sie händisch mit Quicksort\index{Quicksort}. Wählen Sie
als Pivot-Element immer das Element in der Mitte - oder gegebenenfalls
das Element direkt links neben der Mitte. Orientieren Sie sich beim
Aufschreiben der Zwischenschritte an dieser Darstellung:
\end{enumerate}
\end{document}
