\documentclass{bschlangaul-theorie}
\bLadePakete{syntax}

\begin{document}

%%%%%%%%%%%%%%%%%%%%%%%%%%%%%%%%%%%%%%%%%%%%%%%%%%%%%%%%%%%%%%%%%%%%%%%%
% Theorie-Teil
%%%%%%%%%%%%%%%%%%%%%%%%%%%%%%%%%%%%%%%%%%%%%%%%%%%%%%%%%%%%%%%%%%%%%%%%

\chapter{Heapsort}

\begin{liQuellen}
\item \cite[Seite 45]{aud:fs:tafeluebung-11}
\item \cite{wiki:heapsort}
\item \cite[Seite 407-413 14.6.1 Sortieren mit Bäumen: HeapSort]{saake}
\end{liQuellen}

HeapSort
• HeapSort
; Haldensortierung
• füge alle Elemente in eine Max-Halde (bzw. Min-Halde) ein
• das Maximum (bzw. Minimum) eines jeden Teilbaums steht in der Wurzel
• solange noch Elemente in der Halde sind:
• entnimm die Wurzel der Halde
• füge sie vorne an die sortierte Liste an
(bzw. am hintersten freien Platz ins sortierte Array ein)
; alle Elemente werden dabei aufsteigend (bzw. absteigend) sortiert
• Eigenschaften von HeapSort:
• Laufzeitkomplexität:
$O( n \cdot \log n)$
• sortiert auf Grund der Eigenschaften der Halde instabil
• in-place-Implementierung möglich (s.u.)

\bJavaDatei[firstline=3,lastline=51]{sortier/Heap}

\literatur

\end{document}
