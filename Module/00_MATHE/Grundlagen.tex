\documentclass{lehramt-informatik-haupt}
\liLadePakete{mathe,formale-sprachen}

\begin{document}
%%%%%%%%%%%%%%%%%%%%%%%%%%%%%%%%%%%%%%%%%%%%%%%%%%%%%%%%%%%%%%%%%%%%%%%%
% Theorie-Teil
%%%%%%%%%%%%%%%%%%%%%%%%%%%%%%%%%%%%%%%%%%%%%%%%%%%%%%%%%%%%%%%%%%%%%%%%

\chapter{Mathematische Grundlagen}

\section{Zahlen}

\begin{displaymath}
\mathbb{R} \rightarrow
\mathbb{Q} \rightarrow
\mathbb{Z} \rightarrow
\mathbb{N}
\end{displaymath}

\subsection{$\mathbb{N}$: Natürliche Zahl}

Die Menge der natürlichen Zahlen wird mit $\mathbb{N}$ oder $\mathbf{N}$
bezeichnet. Die natürlichen Zahlen sind die beim \memph{Zählen
verwendeten Zahlen} $1$, $2$, $3$, $4$, $5$, $6$, $7$, $8$, $9$, $10$
usw. Je nach Definition kann auch die $0$ (\memph{Null}) zu den
natürlichen Zahlen gezählt werden.
\liFussnoteUrl{https://de.wikipedia.org/wiki/Natürliche_Zahl}

\subsection{$\mathbb{Z}$: Ganze Zahl}

Die Menge der ganzen Zahlen wird mit $\mathbb{Z}$ oder $\mathbf{Z}$
bezeichnet. Die ganzen Zahlen fügen den \memph{natürlichen Zahlen} die
negativen Zahlen hinzu.

\subsection{$\mathbb{Q}$: Rationale Zahl}

Die Menge der rationalen Zahlen wird mit $\mathbb{Q}$ oder $\mathbf{Q}$
bezeichnet. Sie umfasst alle Zahlen, die sich als \memph{Bruch} (engl.
fraction) darstellen lassen, der sowohl im \memph{Zähler als auch im
Nenner ganze Zahlen} enthält.
\liFussnoteUrl{https://de.wikipedia.org/wiki/Rationale_Zahl}

\subsection{$\mathbb{R} \setminus \mathbb{Q}$: Irrationale Zahl}

Kennzeichen einer irrationalen Zahl ist, dass sie \memph{nicht als
Quotient zweier ganzer Zahlen darstellbar} ist. Bekannte irrationale
Zahlen sind die Eulersche Zahl ${\rm e}$ und die Kreiszahl $\pi$. Auch
die Quadratwurzel aus Zwei 2 $\sqrt {2}$ und das
Teilungsverhältnis des Goldenen Schnitts sind irrationale Zahlen.

\liFussnoteUrl{https://de.wikipedia.org/wiki/Irrationale_Zahl}

\subsection{$\mathbb{R}$: Reelle Zahl}

Die reellen Zahlen umfassen die \memph{rationalen Zahlen und die
irrationalen Zahlen}.  Die Menge der reellen Zahlen wird mit
$\mathbb{R}$ oder $\mathbf{R}$ bezeichnet.
\liFussnoteUrl{https://de.wikipedia.org/wiki/Reelle_Zahl}

\section{Modulo / Division mit Rest}

Modulo berechnet den Rest $b$ der Division $n$ geteilt durch $m$.
\liFussnoteUrl{https://de.wikipedia.org/wiki/Division_mit_Rest\#Modulo}

\section{Rechengesetze}

%%
%
%%

\subsection{Kommutativgesetz\footcite{wiki:kommutativgesetz}}

\begin{align*}
a + b &= b + a \\
a \cdot b &= b \cdot a \\
\end{align*}

%%
%
%%

\subsection{Assoziativgesetz\footcite{wiki:assoziativgesetz}}

\begin{align*}
(a + b) + c &= a + (b + c) \\
(a \cdot b) \cdot c &= a \cdot (b \cdot c) \\
\end{align*}

%%
%
%%

\subsection{Distributivgesetz\footcite{wiki:distributivgesetz}}

\begin{align*}
a \cdot (b + c) &= (a \cdot b) + (a \cdot c)\\
(a+b) \cdot c & = (a \cdot c) + (b \cdot c)\\
\end{align*}

\section{Ausklammern:\footcite[/ausklammern]{net:html:mathebibel}}

Ausklammern dient dazu, aus einer Summe oder Differenz ein Produkt zu
machen.

\begin{displaymath}
\textcolor{red}{a}b + \textcolor{red}{a}c = \textcolor{red}{a}(b + c)
\end{displaymath}

\section{Ausmultiplizieren:\footcite[/ausmultiplizieren]{net:html:mathebibel}}

\begin{displaymath}
{\color{red}a} \cdot (b + c) = {\color{red}a}b + {\color{red}a}c
\end{displaymath}

%-----------------------------------------------------------------------
%
%-----------------------------------------------------------------------

\section{Binomische Formeln\footcite{wiki:binomische-formeln}}

Als binomische Formeln werden üblicherweise die folgenden drei
Umformungen bezeichnet:

\begin{enumerate}
\item $(a + b)^2 = a^2 + 2ab + b^2$
erste binomische Formel (Plus-Formel)

\item $(a - b)^2 = a^2 - 2ab + b^2$
zweite binomische Formel (Minus-Formel)

\item $(a + b) \cdot (a - b) = a^2 - b^2$
dritte binomische Formel (Plus-Minus-Formel)
\end{enumerate}

%-----------------------------------------------------------------------
%
%-----------------------------------------------------------------------

\section{Potenzgesetze}

%%
%
%%

\subsection{Multiplikation mit gleicher Basis}

Multipliziert man zwei Potenzen mit gleicher Basis miteinander, erhält
man das Ergebnis, indem man die Exponenten der Potenzen addiert.

\begin{displaymath}
x^a \cdot x^b = x^{a+b}
\end{displaymath}

%%
%
%%

\subsection{Division mit gleicher Basis}

Dividiert man zwei Potenzen mit gleicher Basis, erhält man das Ergebnis,
indem man die Exponenten der Potenzen voneinander subtrahiert.
\footcite[/potenzgesetze]{net:html:mathebibel}

\begin{displaymath}
x^a : x^b = \frac{x^a}{x^b} = x^{a-b}
\end{displaymath}

%-----------------------------------------------------------------------
%
%-----------------------------------------------------------------------

\section{Regel von L’Hospital}

Die Regel von de L’Hospital ist ein Hilfsmittel zum Berechnen von
Grenzwerten bei Brüchen $\frac{f}{g}$ von Funktionen $f$ und $g$, wenn
Zähler und Nenner entweder beide gegen $0$ oder beide gegen (+ oder -)
unendlich gehen. Wenn in einem solchen Fall auch der Grenzwert des
Bruches der Ableitungen existiert, so hat dieser denselben Wert wie der
ursprüngliche Grenzwert:
\footnote{\url{https://de.serlo.org/mathe/funktionen/grenzwerte-stetigkeit-differenzierbarkeit/grenzwert/regel-l-hospital}}

\begin{displaymath}
\lim_{x \to x_0} \frac{f(x)}{g(x)} = \lim_{x \to x_0} \frac{f'(x)}{g'(x)}
\end{displaymath}

%-----------------------------------------------------------------------
%
%-----------------------------------------------------------------------

\section{Binomialkoeffizient}

\begin{displaymath}
\binom nk = \frac{n!}{k! \cdot (n-k)!}
\end{displaymath}

\section{Fakultät}

\begin{displaymath}
n! = 1\cdot 2 \cdot 3 \dotsm n = \prod_{k=1}^n k
\end{displaymath}

\section{Mengen}

| „für die gilt“

\zB \liAusdruck[M]{x \in \mathbb{Q}}{x^2 - 4 = 0}
dabei bedeutet | „für die gilt“, also alle rationalen
Zahlen x, für die gilt, dass das Quadrat von x abzüglich 4 gleich 0
ist).\footcite[Seite 8]{foerster}

\section{Mengen}

\begin{tabular}{ll}
$M_1 \cup M_2$ &
Vereinigungsmenge von $M_1$ und $M_2$ \\

$M_1 \cap M_2$ &
Schnittmenge von $M_1$ und $M_2$\footcite[Seite 397]{hoffmann} \\
\end{tabular}
\literatur

\end{document}
