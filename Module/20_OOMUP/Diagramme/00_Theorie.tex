\documentclass{bschlangaul-theorie}

\begin{document}

%%%%%%%%%%%%%%%%%%%%%%%%%%%%%%%%%%%%%%%%%%%%%%%%%%%%%%%%%%%%%%%%%%%%%%%%
% Theorie-Teil
%%%%%%%%%%%%%%%%%%%%%%%%%%%%%%%%%%%%%%%%%%%%%%%%%%%%%%%%%%%%%%%%%%%%%%%%

\chapter{UML-Diagramme}

\begin{bQuellen}
\item \cite[8.3.3 Seite 233-240]{schneider}
\end{bQuellen}

\section{Überblick\footcite[Seite 8-9]{sosy:fs:2}}

\begin{itemize}
\item Strukturdiagramme (Structural diagrams)

\begin{itemize}
\item \textbf{Klassendiagramm} (Class diagram)
\item Kompositionsstrukturdiagramm (Composite structure diagram)
\item Komponentendiagramm (Component diagram)
\item Verteilungsdiagramm (Deployment diagram)
\item \textbf{Objektdiagramm} (Object diagram)
\item Paketdiagramm (Package diagram)
\item Profildiagramm (Profile diagram)
\end{itemize}

\item Verhaltensdiagramme (Behavorial diagrams)

\begin{itemize}
\item \textbf{Aktivitätsdiagramm} (Activity diagram)
\item \textbf{Anwendungsfalldiagramm} (Use case diagram)
\item \textbf{Zustandsdiagramm} (State diagram)

\item Interaktionsdiagramme:

\begin{itemize}
\item \textbf{Sequenzdiagramm} (Sequenz diagram)
\item \textbf{Kommunikationsdiagramm} (Communication diagram)
\item Interaktionsübersichtsdiagramm (Interaction overview diagram)
\item Zeitverlaufsdiagramm (Timing diagram)
\end{itemize}
\end{itemize}
\end{itemize}

\section{Fragen}

\begin{description}
\item[Anwendungsfalldiagramm]

„Was soll mein geplantes System eigentlich leisten?“

\item[Objektdiagramm]

„Wie sieht ein Schnappschuss meines Systems zur
Ausführungszeit aus?“

\item[Kommunikationsdiagramm]

„Welche Teile einer komplexen Struktur arbeiten wie zusammen, um eine
bestimmte Funktion zu erfüllen?“

\item[Zustandsdiagramm]

„Wie verhält sich das System in einem bestimmten Zustand bei gewissen
Ereignissen?“

\item[Klassendiagramm]

„Wie sind die Daten und das Verhalten meines Systems im Detail
strukturiert?“

\item[Aktivitätsdiagramm]

„Wie realisiert mein System ein bestimmtes Verhalten?“

\item[Sequenzdiagramm]

„Wie läuft die Kommunikation in meinem System ab?“
\end{description}

\literatur

\end{document}
