\documentclass{bschlangaul-theorie}

\begin{document}

%%%%%%%%%%%%%%%%%%%%%%%%%%%%%%%%%%%%%%%%%%%%%%%%%%%%%%%%%%%%%%%%%%%%%%%%
% Theorie-Teil
%%%%%%%%%%%%%%%%%%%%%%%%%%%%%%%%%%%%%%%%%%%%%%%%%%%%%%%%%%%%%%%%%%%%%%%%

\chapter{Kommunikationsdiagramm (communication diagram)}

\begin{bQuellen}
\item \cite[Seite 473-484]{rupp}
\item \cite{wiki:kommunikationsdiagramm}
\end{bQuellen}

Kommunikationsdiagramme (communication diagrams) zeigen die
unterschiedlichen Teile einer (komplexen) Struktur und ihre
Zusammenarbeit zur Erfüllung definierter Funktionalitäten. Die
Funktionsweise ist mit den Sequenzdiagrammen vergleichbar. Der
wesentliche Unterschied liegt in der Darstellung, wie die Objekte
verbunden sind und welche Nachrichten sie über diese Verbindungen in
einem spezifischen Szenario austauschen.
\footcite[Seite 167]{schatten}

Das Kommunikationsdiagramm (Communication Diagram) zeigt
\bEmph{Interaktionen} zwischen \bEmph{Teilen einer meist komplexen
Struktur.} Das Abstraktionsniveau ist so gewählt, dass das
\bEmph{Zusammenspiel} (Nachrichtenaustausch) zwischen den
Kommunikationspartnern und die \bEmph{Verantwortlichkeiten} (wer macht
was) herausgearbeitet werden.

\bEmph{Strikte zeitliche Abläufe}, Zustandswechsel, aber auch
strukturelle Zerlegungen, Parallelitäten oder Kontrollsequenzen (wie
Alter­nativen oder Schleifen) sind anders als in Sequenzdiagrammen
entweder \bEmph{nicht darstellbar} oder zumindest nicht in den
Vordergrund gestellt. Die Reihenfolge der Nachrichten wird lediglich
durch eine gesonderte Nummerierung angezeigt.
\footcite[Seite 474]{rupp}

\begin{description}

%%
%
%%

\item[Interaktionsrahmen] Auch das Kommunikationsdiagramm ist in einen
\bEmph{rechteckigen Rahmen} gefasst. Irreführen­ derweise wird auch hier
die Abkürzung \bEmph{sd} (eigentlich für sequence diagram) verwendet und
der Name der Interaktion in einem \bEmph{Fünfeck in der linken oberen
Ecke} eingetragen.
\footcite[Seite 478]{rupp}

%%
%
%%

\item[Lebenslinie] \strut
\begin{description}
\item[Notation] Eine Lebenslinie wird in Kommunikationsdiagrammen als
\bEmph{rechteckiger Kasten} modelliert. Die eigentliche (Lebens-)Linie
entfällt.

\item[Beschreibung] Die Lebenslinie (Lifeline) repräsentiert die im
Sequenzdiagramm beschriebenen \bEmph{Kommunikationspartner}. Allerdings
müssen bei der Modellierung in Kommunikationsdiagrammen auf viele
Konstrukte verzichtet werden.
\footcite[Seite 478]{rupp}
\end{description}

%%
%
%%

\item[Nachricht] \strut
\begin{description}
\item[Notation] Im Kommunikationsdiagramm modellieren Sie Nachrichten
als \bEmph{durchgezogene Linie}. An die Nachricht wird ein
\bEmph{Pfeil}, der die Richtung vom Sender zum Empfänger kennzeichnet,
angetragen. Der Nachricht wird beim Kommunikationsdiagramm zusätzlich
ein \bEmph{Sequenzbezeichner} vorangestellt.

\item[Beschreibung]
Nachrichten repräsentieren den Aufruf von Operationen und die
Übertragung von Signalen.
\footcite[Seite 480-481]{rupp}
\end{description}

\end{description}

\literatur

\end{document}
