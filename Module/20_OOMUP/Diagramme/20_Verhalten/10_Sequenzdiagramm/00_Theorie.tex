\documentclass{bschlangaul-theorie}
\liLadePakete{uml,syntax}

\begin{document}

%%%%%%%%%%%%%%%%%%%%%%%%%%%%%%%%%%%%%%%%%%%%%%%%%%%%%%%%%%%%%%%%%%%%%%%%
% Theorie-Teil
%%%%%%%%%%%%%%%%%%%%%%%%%%%%%%%%%%%%%%%%%%%%%%%%%%%%%%%%%%%%%%%%%%%%%%%%

\chapter{Sequenzdiagramm (Sequenz diagram)}

\begin{tikzpicture}
\begin{umlseqdiag}
\umlactor[class=A]{a}
\umldatabase[class=B]{b}
\umlmulti[class=C]{c}
\umlobject[class=D]{d}
\begin{umlcall}[op=opa(), type=synchron, return=0]{a}{b}
\begin{umlfragment}
\begin{umlcall}[op=opb(), type=synchron, return=1]{b}{c}
\begin{umlfragment}[type=alt, label=condition, inner xsep=8]
\begin{umlcall}[op=opc(), type=asynchron]{c}{d}
\end{umlcall}
\begin{umlcall}[type=return]{c}{b}
\end{umlcall}
\umlfpart[default]
\begin{umlcall}[op=opd(), type=synchron, return=3]{c}{d}
\end{umlcall}
\end{umlfragment}
\end{umlcall}
\end{umlfragment}
\begin{umlfragment}
\begin{umlcallself}[op=ope(), type=synchron, return=4]{b}
\begin{umlfragment}[type=assert]
\begin{umlcall}[op=opf(), type=synchron, return=5]{b}{c}
\end{umlcall}
\end{umlfragment}
\end{umlcallself}
\end{umlfragment}
\end{umlcall}
\umlcreatecall[class=E,x=8]{a}{e}
\begin{umlfragment}
\begin{umlcall}[op=opg(), name=test, type=synchron, return=6, dt=7]{a}{e}
\umlcreatecall[class=F, stereo=boundary, x=12]{e}{f}
\end{umlcall}
\begin{umlcall}[op=oph(), type=synchron, return=7]{a}{e}
\end{umlcall}
\end{umlfragment}
\end{umlseqdiag}
\end{tikzpicture}

\begin{liQuellen}
\cite[Seite 401-471]{rupp}
\cite[Zusammengefügtes PDF Seite 115-117 / Kapitel „Assoziationen“ 33-35]{brinda}
\end{liQuellen}

Sequenzdiagramme (sequence diagrams) stellen eine exemplarische Abfolge
von Nachrichten zwischen Objekten dar. Der Fokus liegt dabei auf der
zeitlichen Ordnung der Nachrichten. Sequenzdiagramme sind meist sehr
implementationsspezifisch, das heißt, sie sind ohne näheres Wissen
über den Sourcecode des Systems nur schwer nachvollziehbar.
\footcite[Seite 167]{schatten}

\begin{description}
\item[Kommunikationspartner als Lebenslinien]

Nachrichten in Form von Pfeile

\item[synchroner Operationsaufruf]
Pfeil mit ausgefülltem Dreieck, gestrichelter Pfeil zurück
\footcite[Seite 507 (PDF 523): \emph{Synchronous Messages typically
represent operation calls and are shown with a filled arrow head.}]{uml}

\begin{tikzpicture}
\begin{umlseqdiag}
\umlobject[class=A]{a}
\umlobject[class=B]{b}
\begin{umlcall}[op={call()},
return=return]{a}{b}
\end{umlcall}
\end{umlseqdiag}
\end{tikzpicture}

\item[asynchroner Signal-Operationsaufruf]
Normaler offener Pfeil
%
\footcite[Seite 507 (PDF 523): \emph{Asynchronous Messages have an open
arrow head.}]{uml}

\begin{tikzpicture}
\begin{umlseqdiag}
\umlobject[class=A]{a}
\umlobject[class=B]{b}
\begin{umlcall}[type=asynchron]{a}{b}
\end{umlcall}
\end{umlseqdiag}
\end{tikzpicture}

\item[Antwortnachricht]
gestrichelter Pfeil
%
\footcite[Seite 507 (PDF 523): \emph{The reply message from a method has a dashed
line.}]{uml}

\item[Erzeugunsaufruf]
gestrichelter Pfeil erzeugt neues Objekt
%
\footnote{\href{https://stackoverflow.com/a/2128192} The UML 2.2
specifications (superstructure) has an example on page 474, Figure 14.11
is their canonical syntax/notation reference. And on page 495 (UML 2.2)
(Seite 577 (PDF 619) UML 2.5.1) in the notation section it states
“Object creation Message has a dashed line with an open arrow.”}

\begin{tikzpicture}
\begin{umlseqdiag}
\umlobject[class=A]{a}
\umlcreatecall[class=B]{a}{b}
\end{umlseqdiag}
\end{tikzpicture}

\item[Sendeergebnis nicht bekannt]
kleiner Kreis am Pfeilanfang

\item[Empfangergebnis nicht bekannt]
kleiner Kreis am Pfeilende

\item[Fragmente] \strut

\begin{tikzpicture}
\begin{umlseqdiag}
\umlactor[class=A]{a}
\umlobject[class=B]{b}
\begin{umlfragment}[inner ysep=-1]
\begin{umlcall}[op=op(), type=synchron, return=1]{a}{b}
\end{umlcall}
\end{umlfragment}
\end{umlseqdiag}
\end{tikzpicture}

\end{description}

\literatur

\end{document}
