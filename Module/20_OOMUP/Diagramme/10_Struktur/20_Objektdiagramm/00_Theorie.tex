\documentclass{bschlangaul-theorie}
\begin{document}

%%%%%%%%%%%%%%%%%%%%%%%%%%%%%%%%%%%%%%%%%%%%%%%%%%%%%%%%%%%%%%%%%%%%%%%%
% Theorie-Teil
%%%%%%%%%%%%%%%%%%%%%%%%%%%%%%%%%%%%%%%%%%%%%%%%%%%%%%%%%%%%%%%%%%%%%%%%

\chapter{Objektdiagramm}

\begin{bQuellen}
\cite[Seite 183-192]{rupp}
\end{bQuellen}

Objektdiagramme (object diagrams) zeigen eine konkrete Ausprägung
des Systems zur Ausführungszeit.
\footcite[Seite 166]{schatten}

\literatur

\end{document}
