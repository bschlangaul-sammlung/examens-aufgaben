\documentclass{bschlangaul-haupt}
\liLadePakete{syntax}
\begin{document}

\chapter{Aufgabenblatt 3: Algorithmen implementieren I}

Alle Aufgaben auf diesem Blatt sind mit der Entwicklungsumgebung
Greenfoot zu bearbeiten. Es ist zu empfehlen, sich vor dem Arbeiten mit
Greenf oot in die Grundzüge des Themas Vererbung einzulesen.

%-----------------------------------------------------------------------
%
%-----------------------------------------------------------------------

\section{Aufgabe 1\footcite{oomup:ab:3}}

\begin{enumerate}

%%
% (a)
%%

\item Öffnen Sie Greenfoot, erstellen Sie ein neues Java-Szenario
Supermarkt und ändern Sie den Hintergund der Klasse \liJavaCode{MyWorld} auf
das Bild supermarket-background, das Sie im Materialordner finden.
Passen Sie die Größe der Spielwelt in der Klasse \liJavaCode{MyWorld} auf
600x566 an.

%%
% (b)
%%

\item Erzeugen Sie eine neue Unterklasse der Klasse Actor und nennen Sie
diese Einkauf swagen. Vergeben Sie für diese Klasse das entsprechende
Bild im Materialordner und positionieren Sie ein Objekt dieser Klasse
passend auf der Spielwelt. Speichern Sie die Welt, so dass sie bei jedem
reset wieder so aussieht, wie am Anfang. Zum Beispiel:

%%
% (c)
%%

\item Erstellen Sie in der Klasse Einkauf swagen eine Methode bewegen(),
die von der act-Methode aufgerufen wird. Diese soll dafür sorgen, dass
der Einkaufswagen sich durch die Pfeiltasten nach links und rechts
steuern lässt. Achten Sie auch darauf, dass das Bild passend gespiegelt
wird. In den nächsten Schritten fügen wir nun Lebensmittel am oberen
Rand des Spielfeldes ein, die nach unten fallen. Diese sollen mit dem
Einkaufswagen eingesammelt werden. Dabei geben gesunde Lebensmittel
(Obst) jeweils einen Punkt. Ungesunde Lebensmittel (Junk-Food) geben
einen Minuspunkt. Sobald man negative Gesamtpunkte hat, ist das Spiel
verloren.

%%
% (d)
%%

\item Füge eine Unterklasse Lebensmittel der Klasse Actor ein, die
wiederum die Unterklassen Obst und Junk-F ood besitzt. Erzeuge je drei
weitere Unterklassen, \zB Apfel, Birne und Banane für Obst und P
ommes, Muffin, Hamburger für JunkFood. Versehe die Unterklassen mit
passenden Bildern. (Klassenstruktur siehe Abb. rechts)

%%
% (e)
%%

\item Ergänzen Sie die Klasse M yW orld um eine Methoden
lebensmittelErzeugen(), die mit Hilfe von Zufallszahlen verschiedene
Lebensmittel am oberen Rand der Spielwelt erzeugt.

%%
% (f)
%%

\item Implementieren Sie in der Klasse Lebensmittel die Methode
protected void f allen(), die in den act()-Methoden der betroffenen
Klassen aufgerufen wird. Diese sorgt dafür, dass sich die Lebensmittel
in einem konstanten Tempo von oben nach unten bewegen.

%%
% (g)
%%

\item Ergänzen Sie die Klasse Einkauf swagen um ein Attribut punkte und
lassen Sie diesen an einer passenden Stelle der Spielwelt anzeigen.
Implementieren Sie außerdem eine Methode punkteAktualisieren(), die bei
Aufruf den Punktestand um 1 erhöht bzw. verringert, je nachdem, ob ein
Stück Obst oder ein Junk-FoodLebensmittel mit dem Einkaufswagen berührt
wird. Wird der Punktestand -1 erreicht, also wurden mehr
Junk-Food-Lebensmittel als Obst eingesammelt, so ist das Spiel verloren.
In der Mitte der Spielwelt soll die Anzeige GAME OVER erscheinen und
das Spiel soll stoppen.

\end{enumerate}

%-----------------------------------------------------------------------
%
%-----------------------------------------------------------------------

\section{Aufgabe 2\footcite{oomup:ab:3}}

Eine moderne Variante des altbekannten Spiels Frogger stellt das
Bier-Hol-Spiel dar, das momentan bei vielen beliebt ist.

\begin{enumerate}

%%
%
%%

\item Probieren Sie das Spiel aus (http://bier.drwuro.com) und
modellieren Sie es passend mit Hilfe eines Klassendiagramms. Gerne
können Sie den Kontext abändern. Die Funktionalitäten und der
Spielablauf sollen aber erhalten werden. Da das Spiel in Greenfoot
umgesetzt werden soll, denken Sie bitte auch an die Klassen World und
Actor.

Hinweis: Zur Vereinfachung können Sie sich auf Feinde beschränken, die
nur aus einer Fahrtrichtung kommen.

%%
%
%%

\item Setzen Sie das Projekt in \emph{Greenfoot} um und testen Sie es
ausführlich.

\begin{liAntwort}
\liJavaDatei[firstline=3]{aufgaben/oomup/ab_3/bier/Akteur}
\liJavaDatei[firstline=3]{aufgaben/oomup/ab_3/bier/BierWorld}
\liJavaDatei[firstline=3]{aufgaben/oomup/ab_3/bier/Gegner}
\liJavaDatei[firstline=3]{aufgaben/oomup/ab_3/bier/Kuh}
\liJavaDatei[firstline=3]{aufgaben/oomup/ab_3/bier/Sepp}
\liJavaDatei[firstline=3]{aufgaben/oomup/ab_3/bier/Traktor}
\end{liAntwort}
\end{enumerate}
\end{document}
