\documentclass{bschlangaul-theorie}

\begin{document}

%%%%%%%%%%%%%%%%%%%%%%%%%%%%%%%%%%%%%%%%%%%%%%%%%%%%%%%%%%%%%%%%%%%%%%%%
% Theorie-Teil
%%%%%%%%%%%%%%%%%%%%%%%%%%%%%%%%%%%%%%%%%%%%%%%%%%%%%%%%%%%%%%%%%%%%%%%%

\chapter{Vererbung (Generalisierung – Spezialisierung)}

„Vererbung (generalization) beschreibt eine Beziehung zwischen einer
allgemeinen Klasse (Basisklasse) und einer spezialisierten Klasse. Die
spezialisierte Klasse ist vollständig konsistent mit der Basisklasse,
enthält aber zusätzliche Informationen (Attribute, Operationen,
Assoziationen). Die allgemeine Klasse wird auch als Oberklasse (super
class), die spezialisierte als Unterklasse (sub class) bezeichnet.“
(vgl. Balzert 1999, S 51)

Was bedeutet Vererbung für die abgeleitete Klasse (Unterklasse)?

Eine abgeleitete Klasse erhält von der Basisklasse alle Attribute,
Methoden und Beziehungen:

\begin{itemize}
\item Alle Attribute der Basisklasse (Oberklasse) haben einschließlich
der Spezifikation auch in der abgeleiteten Klasse Gütligkeit.

\item Gleiches gilt auch für die Methoden der Oberklasse. Somit kann die
abgeleitete Klasse auch auf die vererbten Methoden zugreifen.

\item Hat die Basisklasse eine Beziehung zu anderen Klassen, so hat auch
die abgeleitete Klasse diese Beziehung.
\end{itemize}

%-----------------------------------------------------------------------
%
%-----------------------------------------------------------------------

\section{Vorteile}

\begin{description}
\item[Verfeinerung der Oberklasse]

Eine abgeleitete Klasse kann durch neue Bestandteile, wie weitere
Attribute, Methoden und Beziehungen, verfeinert werden. Auf diese Weise
können neue Typen von Objekten auf der Basis bereits vorhandener
Objekt-Definitionen festgelegt werden. Weiter können Methoden der
Basisklasse verändert werden. Dies wird erreicht, indem die Unterklasse
eine neue Methode erhält, die den gleichen Namen wie die der Oberklasse
und geänderte Inhalte hat. Dieser Vorgang heißt Redefinieren oder
Überschreiben.

\item[Redundanz bei der Definition von Klassen vermieden]

Durch die Vererbung entsteht eine Klassen-Hierarchie bzw. eine
Vererbungsstruktur. Vererbung ist über mehrere Abstraktionsstufen
möglich. Man erhält so schnell neue Klassen, ohne den gemeinsamen Code
neu schreiben zu müssen. Dies hat den Vorteil, dass Redundanz bei der
Definition von Klassen vermieden wird.

\item[Konsistenz des Programmcodes leichter sichergestellt]

Änderungen in der Oberklasse werden sofort für alle abgeleiteten Klassen
gültig. Auf diese Weise kann die Konsistenz des Programmcodes leichter
sichergestellt werden.

\item[Code mit geringerem Aufwand entwickelt]

Vererbung trägt dazu bei, dass der Code mit geringerem Aufwand
entwickelt und gewartet werden kann.
\footcite[Seite 11]{brinda}
\end{description}

%-----------------------------------------------------------------------
%
%-----------------------------------------------------------------------

\section{Nachteile}

\begin{description}
\item[Abhängigkeitsbeziehungen]

Mit Vererbung schafft man Abhängigkeitsbeziehungen, die dem
Modularisierungsgedanken nicht entsprechen.
\liFussnoteUrl{https://inf-schule.de/modellierung/ooppython/ampel/vererbung/konzept_vererbung}

\end{description}
\literatur

\end{document}
