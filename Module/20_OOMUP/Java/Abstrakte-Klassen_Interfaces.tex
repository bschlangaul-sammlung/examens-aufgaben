\documentclass{bschlangaul-theorie}
\bLadePakete{syntax,uml}
\usepackage{paralist}

\begin{document}

\chapter{Abstrakte Klassen / Interface}

\cite[Vererbung Seite 28-31 (PDF Seite 156-159)]{brinda}

\section{Abstrake Klassen und Methoden}

Eine abstrakte Klasse ist eine spezielle Klasse, von der keine Objekte
erstellt werden können, \dh die Klasse ist nicht \memph{nicht
instanziierbar}.

Das ist nützlich, wenn eine Menge von Klassen gemeinsame Methoden
benötigt, die in einer Oberklasse implementiert werden können, aber es
keinen Sinn mach, von der Oberklasse selbst Instanzen zu erstellen.

Es ist zudem möglich, Methoden, die alle Unterklassen haben sollen, in
einer abstrakten Oberklasse zu definieren, aber nicht zu implementieren,
weil die Funktionalität in den Unterklassen leicht unterschiedlich ist,
dazu werden abstrakte \memph{Methoden} verwendet. Diese \memph{müssen}
dann in der \memph{Unterklasse implementiert} werden.
\footcite[Seite 31]{oomup:fs:3}

\begin{minted}{java}
public abstract class Vierbeiner {
  private String vierbeinerName;

  public Vierbeiner(String name) {
    this.setName(name);
  }

  public void setName(String pName) {
    vierbeinerName = pName;
  }

  public abstract void rennen();
}
\end{minted}

\begin{minted}{java}
public class Katze extends Vierbeiner{
  public Katze(String name){
    super(name);
    rennen();
  }

  public void rennen(){
    System.out.println("Deine Katze rennt!");
  }
}
\end{minted}

\section{Interfaces (Schnittstellen)
\footcite[Vererbung Seite 32-38 (PDF Seite 160-166)]{brinda}}

„Eine Schnittstelle (interface) spezifiziert einen Ausschnitt aus dem
Verhalten einer Klasse. Eine Schnittstelle besteht nur aus den
Signaturen von Operationen, d. h. sie besitzt keine Implementierung,
keine Attribute oder Assoziationen. Eine Schnittstelle ist äquivalent zu
einer abstrakten Klasse, die ausschließlich abstrakte Operationen
besitzt.“ (vgl. Balzert, 2005)

Schnittstellen legen für die Objekte der implementierenden Klassen ein
bestimmtes, gleichartiges Verhalten fest. Durch das
Schnittstellenkonzept wird gewährleistet, dass jede Klasse, die die
Schnittstelle implementiert, die in der Schnittstelle deklarierten
Methoden zur Verfügung stellt. Eine Klasse kann mehrere Schnittstellen
implementieren.

\begin{tikzpicture}
\umlclass[x=2,type=abstract]
{Figur}
{rahmendicke\\farbe}
{berechnenFlaeche()}

\umlclass[x=-2,y=-3,anchor=north]
{Rechteck}
{laenge\\breite}
{berechnenFlaeche()}

\umlclass[x=2,y=-3,anchor=north]
{Kreis}
{radius}
{berechnenFlaeche()}

\umlclass[x=6,y=-3,anchor=north]
{Dreieck}
{grundlinie\\hohe}
{berechnenFlaeche()}

\umlVHVinherit[arm2=-2cm]{Rechteck}{Figur}
\umlVHVinherit[arm2=-2cm]{Kreis}{Figur}
\umlVHVinherit[arm2=-2cm]{Dreieck}{Figur}
\end{tikzpicture}

%-----------------------------------------------------------------------
%
%-----------------------------------------------------------------------

\bExamensAufgabeTTA 66116 / 2014 / 03 : Thema 2 Teilaufgabe 2 Aufgabe 1

\literatur

\end{document}
