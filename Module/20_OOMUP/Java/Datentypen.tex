\documentclass{bschlangaul-theorie}
\usepackage{paralist}
\begin{document}

\section{Datentypen}

\section{Primitive Datentypen}

\cite[Seite 37]{oomup:fs:2}

\begin{itemize}

%%
%
%%

\item ganze Zahlen

\begin{compactitem}
\item \verb|byte| (8 bit)
\item \verb|short| (16 bit)
\item \verb|int| (32 bit)
\item \verb|long| (64 bit)
\end{compactitem}

%%
%
%%

\item Fließkommazahlen, also Zahlen mit Nachkommastellen

\begin{compactitem}
\item \verb|float| (32 bit)
\item \verb|double| (64 bit)
\end{compactitem}

%%
%
%%

\item Wahrheitswerte

\begin{compactitem}
\item \verb|boolean| (1 bit)
\end{compactitem}

%%
%
%%

\item Zeichen

\begin{compactitem}
\item \verb|char|
\end{compactitem}

\end{itemize}

\literatur

\end{document}
