\documentclass{bschlangaul-theorie}
\begin{document}

\chapter{Aufgabenblatt 1: Grundlagen der Objektorientierung}

%-----------------------------------------------------------------------
%
%-----------------------------------------------------------------------



%-----------------------------------------------------------------------
%
%-----------------------------------------------------------------------

\section{Aufgabe 2\footcite{oomup:ab:1}}

Eine Hundeschule möchte zur Verwaltung der Hunde, Kunden und Trainer
eine Datenbank anlegen. Es soll davon ausgegangen werden, dass Trainer
mehrere Hunde trainieren, aber jeder Hund nur von einem Trainer geschult
werden kann. Darüber hinaus gibt es Verwaltungsangestellte, von denen
Name, Vorname, Geburtsdatum und Adresse gespeichert werden sollen, die
den Kunden für organisatorische Belange zur Verfügung stehen. Ein Hund,
von dem Name, Rasse und Alter bekannt sind, gehört zu genau einem
Kunden, von dem Name, Anschrift und Telefonnummer benötigt werden. Ein
Trainer hat einen Vornamen, einen Nachnamen und eine Adresse.
Modellieren Sie die gegebenen Informationen mit Hilfe eines
Klassendiagramms, wie es in der gymnasialen Mittelstufe normalerweise im
Bereich Datenbanken gemacht wird.

%-----------------------------------------------------------------------
%
%-----------------------------------------------------------------------

\section{Aufgabe 3\footcite{oomup:ab:1}}

Betrachten Sie die gegebene Klassenkarte der Klasse Auto mit allen
vorhandenen Methoden. Geben Sie in Punktnotation unter Beachtung der
passenden Parameter die Methodenaufrufe an, so dass die beiden Autos aus
Abb. 1 danach, wie in Abb. 2 zu sehen, angeordnet sind. Beachten Sie
dabei auch die Informationen in den Objektkarten! Geben Sie die zu Abb.
2 passenden Objektkarten der beiden Autos an. Hinweis 1 : In der
Informatik befindet sich der Ursprung des Koordinatensystems oben links.
Orientieren Sie sich anhand der gegebenen Objektkarten, wie die
Koordinaten der Autos zu bestimmen sind. Hinweis 2 : Beim Drehen des
Autos bleibt das Heck des Fahrzeugs im Kästenchen stehen und die Front
dreht sich.

%-----------------------------------------------------------------------
%
%-----------------------------------------------------------------------

\section{Aufgabe 4\footcite{oomup:ab:1}}

Ein vereinfachter Automat zum Verkauf von Schokoladentafeln funktioniert
folgendermaßen: Als Geldeinwurf werden 1- und 2-Euro-Stücke akzeptiert.
Mit zwei Druckknöpfen kann man zwischen einer großen und einer kleinen
Tafel Schokolade wählen. Eine große Tafel kostet 2 Euro, eine kleine 1
Euro. Bei Wahl einer kleinen Tafel und Einwurf eines 2-Euro-Stückes wird
mit der Schokolade 1 Euro Wechselgeld ausgegeben (ebenso bei Wahl einer
großen Tafel und Einwurf eines 1-Euro-Stückes, gefolgt von einem
2-Euro-Stück).

\renewcommand{\labelenumi}{(\alph{enumi})}
\begin{enumerate}

%%
% (a)
%%

\item Geben Sie eine geeignete Menge von Aktionen an, mit der das
Verhalten des Automaten beschrieben werden kann (Eine Aktion bezeichnet
eine Klasse gleichartiger Ereignisse).

%%
% (b)
%%

\item Geben Sie drei mögliche Abläufe (Ereignisfolgen) des Automaten an.

%%
% (c)
%%

\item Beschreiben Sie das Verhalten des o. g. Automaten durch ein
Zustandsübergangsdiagramm. Beachten Sie dabei den Zusammenhang zur
Aktionsmenge und überlegen Sie sich, mit welchen Informationen Sie die
einzelnen Zustände am besten repräsentieren können.

\end{enumerate}

\end{document}
