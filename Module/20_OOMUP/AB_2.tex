\documentclass{bschlangaul-theorie}

\renewcommand{\labelenumi}{(\alph{enumi})}
\begin{document}

\chapter{Aufgabenblatt 2: Algorithmen}

%-----------------------------------------------------------------------
%
%-----------------------------------------------------------------------

\section{Aufgabe 1\footcite{oomup:ab:2}}

\begin{enumerate}

%%
% (a)
%%

\item Geben Sie in eigenen Worten eine möglichst exakte Definition für
den Begriff Algorithmus an, die man auch im Informatikunterricht der
Unterstufe verwenden könnte.

%%
% (b)
%%

\item Geben Sie für den Schokoladenautomaten aus Blatt 1 Aufgabe 4
(umgangssprachlich) einen Algorithmus für sein Verhalten an. Verwenden
Sie beim Erstellen des Algorithmus typische Strukturelemente von
Kontrollstrukturen, wie \zB wenn... dann..., wiederhole solange...,
wiederhole f ür alle... usw.
\end{enumerate}

%-----------------------------------------------------------------------
%
%-----------------------------------------------------------------------

\section{Aufgabe 2 (Check-Up)\footcite{oomup:ab:2}}

Programmierung mit Snap!

In diesem Projekt erstellen Sie das Computerspiel Pong, das 1972 von
Atari entwickelt wurde.

\begin{enumerate}

%%
% (a)
%%

\item Erstellen Sie zunächst die Sprites für Paddel (gefülltes Rechteck)
und Ball (Kreis).

%%
% (b)
%%

\item Programmieren Sie die Steuerung des Paddels: Der Spieler soll es
mit den Pfeiltasten nach oben und nach unten steuern können. Achten Sie
dabei darauf, dass das Paddel nicht über den Spielfeldrand hinausgeht.

%%
% (c)
%%

\item Erstellen Sie nun den Code für den Ball: Dieser soll sich
selbstständig bewegen und am Rand des Spielfeldes sowie am Paddle
abprallen. Für das Abprallen am Spielfeldrand gibt es einen eigenen
Block in Snap!

%%
% (d)
%%

\item Ergänzen Sie nun einen Punktezähler: Der Spieler soll jedes mal,
wenn der Ball am Paddle abprallt einen Punkt bekommen. Berührt der Ball
allerdings die rechte Wand, so bekommt er einen MinusPunkt. Um den
Punktestand zu verwalten, bietet sich die Verwendung einer Variablen an.

%%
% (e)
%%

\item Erstellen Sie ein zweites Paddle auf der linken Seite des
Spielfeldes und passen Sie den Code so an, dass zwei Spieler
gegeneinander spielen können. Der zweite Spieler soll sein Paddle mit
Hilfe der Tasten w und s steuern können.

%%
% (f)
%%

\item Verändern Sie das Spiel nun so, dass das linke Paddle automatisch
durch den Computer gesteuert wird.

Hinweis: Um die Steuerung des zweiten Paddels zu automatisieren, kann
folgender Block aus dem Bereich Fühlen“benutzt werden:”Geben Sie den
Entwicklungsstand des Spiels sowohl nach Teilaufgabe (e) als auch nach
Teilaufgabe (f) ab.

\end{enumerate}
Diese Aufgabe entstammt den ”Beauty and Joy of Computing” - Materialien.
Diese sind unter Creative Commons BY-NC-SA durch die University of
California lizenziert. Das ursprüngliche Material wurde in der
vorliegenden Aufgabe frei übersetzt und adaptiert.
\end{document}
