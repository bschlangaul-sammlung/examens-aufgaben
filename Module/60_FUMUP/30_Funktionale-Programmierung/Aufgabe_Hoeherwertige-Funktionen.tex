\documentclass{bschlangaul-aufgabe}
\bLadePakete{syntax}
\begin{document}
\bAufgabenTitel{Höherwertige Funktionen}
\section{Höherwertige Funktionen
\index{Funktionale Programmierung mit Haskell}
\footcite{fumup:ab:3}}

Implementiere in der Datei switch.hs. Die Funktion
\bHaskellCode{switch} bekommt eine einwertige Entscheidungsfunktion
\bHaskellCode{s}, sowie ein Feld aus einwertigen Funktionen übergeben.
Ihr Ergebnis ist eine einwertige Funktion (mit Parameter
\bHaskellCode{x}). Diese Funktion interpretiert den Rückgabewert der
Entscheidungsfunktion (\bHaskellCode{s x}) als Index, mit dem eine
Funktion aus dem Feld auswählt wird. Diese gewählte Funktion wird dann
für \bHaskellCode{x} auswertet. Wenn der ermittelte Funktionsindex
nicht im Feld liegt, soll \bHaskellCode{x} unverändert zurückgegeben
werden.

\begin{liAntwort}
\bHaskellDatei{switch.hs}
\end{liAntwort}

\end{document}
