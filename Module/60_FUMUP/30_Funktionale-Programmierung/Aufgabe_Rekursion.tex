\documentclass{bschlangaul-aufgabe}
\bLadePakete{syntax}
\begin{document}
\bAufgabenTitel{Rekursion}
\section{Übungen zur Rekursion
\index{Funktionale Programmierung mit Haskell}
\footcite{fumup:ab:3}}

Implementiere in der Dateilists.hs eine Funktion
\bHaskellCode{mylength(liste)}, die die Länge der übergebenen Liste
berechnet. Die Funktion \bHaskellCode{myconcat(liste1, liste2)} soll
als Ergebnis die Konkatenation der beiden Listen liefern. Mit
\bHaskellCode{myappend(liste, elem)} soll das Element an das Ende der
Liste angehängt werden. Die Funktion \bHaskellCode{listSum(liste)} soll
die Summe aller Werte in der Liste zurückliefern. Verwenden Sie in
dieser Aufgabe keine spezialisierten Listenfunktionen (wie \zB den
\bHaskellCode{++}-Operator) außer dem \bHaskellCode{:}-Operator.

\begin{bAntwort}
\bHaskellDatei{Dateilists.hs}
\end{bAntwort}

\end{document}
