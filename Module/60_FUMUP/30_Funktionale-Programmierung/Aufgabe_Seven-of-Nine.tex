\documentclass{bschlangaul-aufgabe}
\liLadePakete{syntax}
\begin{document}
\liAufgabenTitel{Seven-of-Nine}
\section{Listen
\index{Funktionale Programmierung mit Haskell}
\footcite{fumup:ab:3}}

Implementiere in der Datei sevenOfNine.hs nachfolgende Funktion in
Haskell. Die parameterlose Funktion \liHaskellCode{sevenOfNine} liefert
eine (unendliche) Liste aller natürlicher Zahlen, die durch 7 oder 9
teilbar sind, zurück. \liHaskellCode{sevenOfNine :: [Int]}

\begin{liAntwort}
\liHaskellDatei{sevenOfNine.hs}
\end{liAntwort}

\end{document}
