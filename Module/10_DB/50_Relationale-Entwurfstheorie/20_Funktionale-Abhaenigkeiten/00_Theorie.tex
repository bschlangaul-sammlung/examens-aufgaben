\documentclass{bschlangaul-theorie}
\liLadePakete{normalformen}
\begin{document}
\let\fa=\liFunktionaleAbhaengigkeit
\let\FA=\liFunktionaleAbhaengigkeiten

\chapter{Funktionale Abhängigkeiten}

%-----------------------------------------------------------------------
%
%-----------------------------------------------------------------------

%%
%
%%

\subsection{Definition (Funktionale Abhängigkeit)}

Sei $R\{A_1,\dots,A_n\}$ ein Relationenschema. Seien $X, Y \subseteq
\{A_1,\dots,A_n\}$ Teilmengen der Attributmenge von $R$. $Y$ ist genau
dann funktional abhängig von $X$, wenn gleiche Werte für die Attribute
in $X$ auch gleiche Werte für die Attribute in $Y$ erzwingen. Man
schreibt: $X \rightarrow Y$. $X \rightarrow Y$ heißt funktionale
Abhängigkeit, englisch \emph{Functional Dependency}, kurz \textbf{FD},
für das Relationenschema $R\{A_1,\dots,A_n\}$ Dabei heißt eine
funktionale Abhängigkeit $X \rightarrow Y$

\begin{itemize}
\item \textbf{einfach}, wenn $Y$ nur aus einem Attribut besteht.
\item \textbf{trivial}, wenn $Y \subseteq X$ gilt.\footcite[Seite
168]{winter}
\end{itemize}

%%
%
%%

\subsection{Definition (Erfüllung funktionaler Abhängigkeiten)}

Sei $R\{A_1,\dots,A_n\}$ ein Relationenschema. Seien $X, Y \subseteq
R\{A_1,\dots,A_n\}$ Teilmengen der Attributmenge von R. Eine Instanz $R$
von $R\{A_1,\dots,A_n\}$ erfüllt die funktionale Abhängigkeit $X
\rightarrow Y$, wenn gilt: Stimmen zwei Tupel aus $R$ in den Werten der
Attribute aus $X$ überein, so stimmen sie auch in den Werten der
Attribute aus $Y$ überein. Man sagt, dass eine funktionale Abhängigkeit
$X \rightarrow Y$ für $R\{A_1,\dots,A_n\}$ \textbf{gilt}, wenn sie von
\textbf{jeder Instanz} von $R$ erfüllt wird. Eine Menge $F$ funktionaler
Abhängigkeiten gilt für $R\{A_1,\dots,A_n\}$, wenn jede Abhängigkeit aus
$F$ für $R\{A_1,\dots,A_n\}$ gilt.\footcite[Seite 169]{winter}

%%
%
%%

\subsection{Definition (Logische Folgerbarkeit)}

Für jede Instanz des Relationenschemas, die die FD \fa{PersNr ->
Name, Geschlecht, Wohnort, Geburtsjahr} erfüllt, gilt sicher auch
\fa{PersNr -> Name}. Man sagt, dass \fa{PersNr -> Name}
logisch aus $F$ folgt.\footcite[Seite 171]{winter}

%%
%
%%

\subsection{Definition (Hülle von F)}

Zu einer gegebenen Menge $F$ von FDs kann man die Menge aller aus $F$
logisch folgerbaren funktionalen Abhängigkeiten bestimmen. Diese Menge
heißt Hülle $F^+$.\footcite[Seite 171]{winter}

%-----------------------------------------------------------------------
%
%-----------------------------------------------------------------------

\section{Armstrong-Kalkül\footcite[Axiome von Armstrong]{wiki:funktionale-abhängigkeit}}

Mit Hilfe der Axiome von Armstrong (auch Armstrong-Axiome) lassen sich
aus einer Menge von funktionalen Abhängigkeiten, die auf einer Relation
gelten, \memph{weitere funktionale Abhängigkeiten ableiten}.
Die folgenden drei Regeln reichen aus, um alle funktionalen
Abhängigkeiten herzuleiten:\footcite[Seite 14]{db:fs:4}

%%
%
%%

\liPseudoUeberschrift{Axiome von Armstrong}

\begin{description}
\item[Reflexivität:]
%
Eine Menge von Attributen bestimmt eindeutig die Werte einer Teilmenge
dieser Attribute (triviale Abhängigkeit), das heißt,
$\beta \subseteq \alpha \Rightarrow \alpha \rightarrow \beta$ .

\item[Verstärkung:]
%
Gilt
$\alpha \rightarrow \beta$, so gilt auch
$\alpha \gamma \rightarrow \beta \gamma$
für jede Menge von Attributen
$\gamma$ der
Relation.

\item[Transitivität:]
%
Gilt
$\alpha \rightarrow \beta$
und
$\beta \rightarrow \gamma$,
so gilt auch
$\alpha \rightarrow \gamma$.
\end{description}
\footcite[Axiome von Armstrong]{wiki:funktionale-abhängigkeit}
Die Armstrong-Axiome sind \memph{korrekt} und \memph{vollständig}: Diese
Regeln sind gültig (korrekt) und alle anderen gültigen Regeln können von
diesen Regeln abgeleitet werden (vollständig).
\liFussnoteUrl{https://dbresearch.uni-salzburg.at/teaching/2019ss/db1/db1_06-handout-1x1.pdf}

Um Herleitungen einfacher zu gestalten, können zusätzlich die folgenden
(abgeleiteten) Regeln verwendet werden:
\footcite[Axiome von Armstrong]{wiki:funktionale-abhängigkeit}

%%
%
%%

\liPseudoUeberschrift{Abgeleitete Regeln}

\begin{description}
\item[Vereinigung:]
%
Gelten
$\alpha \rightarrow \beta$
$\alpha \rightarrow \gamma$ , so gilt auch
$\alpha \rightarrow \beta \gamma$.

\item[Dekomposition:]
%
Gilt
$\alpha \rightarrow \beta \gamma$,
so gelten auch
$\alpha \rightarrow \beta$ und
$\alpha \rightarrow \gamma$ .

\item[Pseudotransitivität:]
%
Gilt
$\alpha \rightarrow \beta$ und
$\beta \gamma \rightarrow \delta$,
so gilt auch
$\alpha \gamma \rightarrow \delta$
\end{description}

\subsection{Beispiel}

Geg.: Menge \FA{
  PersN -> Name, Wohnort;
  Fach -> Pflichtfach
}.

Dann sind (\zB) folgende FD mit Hilfe der Armstrong-Axiome
ableitbar:

\begin{enumerate}
\item Axiom 1 (Reflexivität): \fa{PersNr, Fach -> Fach}

\item Axiom 2 (Verstärkung): Da \fa{PersNr -> Name, Wohnort} und
${Fach} \subseteq {Fach}$ gilt, folgt die funktionale Abhängigkeit
\fa{PersNr, Fach -> Name, Wohnort, Fach}

\item Axiom 3 (Transitivität): Wegen \fa{PersNr, Fach -> Fach} und
\fa{Fach -> Pflichtfach} folgt die funktionale Abhängigkeit
\fa{PersNr, Fach -> Pflichtfach}
\end{enumerate}

\literatur

\end{document}
