\documentclass{bschlangaul-theorie}
\liLadePakete{mathe,normalformen,synthese-algorithmus}

\begin{document}
\let\schrittE=\liSyntheseUeberErklaerung

DB-normalizer \url{https://normalizer.db.in.tum.de/index.py}

\chapter{Synthesealgorithmus}

\begin{enumerate}
\item \schrittE{1}

\begin{enumerate}
\item \schrittE{1-1}
\item \schrittE{1-2}
\item \schrittE{1-3}
\item \schrittE{1-4}
\end{enumerate}

\item \schrittE{2}
\item \schrittE{3}
\item \schrittE{4}
\end{enumerate}

\begin{liLernkartei}{Synthesealgorithmus zur 3NF}

\begin{compactenum}
\item Reduktion (kanonische Überdeckung)

\begin{compactenum}
\item Linksreduktion:
$B \subseteq \textit{AttrHülle}(F, \alpha - A)$

\item Rechsreduktion:
$B \in \textit{AttrHülle}(F - (\alpha \rightarrow \beta) \cup (\alpha \rightarrow (\beta - B)), \alpha)$

\item Leere Klauseln streichen:
$\alpha \rightarrow \emptyset$

\item Vereinigung:
$\alpha \rightarrow \beta_1,...,\alpha \rightarrow \beta_n  = \alpha \rightarrow (\beta_1 \cup ... \cup \beta_n)$
\end{compactenum}

\item Neues Relationenschema
\item Hinzufügen einer Relation
\item Entfernen überflüssiger Teilschemata
\end{compactenum}
\end{liLernkartei}

\literatur

\end{document}
