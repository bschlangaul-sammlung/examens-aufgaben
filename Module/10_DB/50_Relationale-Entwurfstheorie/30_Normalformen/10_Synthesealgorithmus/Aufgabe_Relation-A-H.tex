\documentclass{lehramt-informatik-aufgabe}
\liLadePakete{normalformen,synthese-algorithmus}
\begin{document}
\let\ah=\liAttributHuelle
\let\ahl=\liAttributHuelleLinksReduktionInline
\let\fa=\liFunktionaleAbhaengigkeit
\let\FA=\liFunktionaleAbhaengigkeiten
\let\m=\liAttributMenge
\let\schrittE=\liSyntheseUeberErklaerung

\liAufgabenTitel{Relation A-H}

\section{Synthesealgorithmus
\index{Synthese-Algorithmus}
\footcite[Seite 1: Synthesealgorithmus, Aufgabe 2]{db:pu:4}
}

Überführen Sie das Relationenschema mit Hilfe des Synthesealgorithmus in
die 3. Normalform!

\begin{center}
\liRelation[R]{A, B, C, D, E, F, G, H}
\end{center}

\FA{
  F -> E;
  A -> B, D;
  A, E -> D;
  A -> E, F;
  A, G -> H;
}

\begin{liAntwort}

\begin{enumerate}

\item Kanonische Überdeckung

\begin{enumerate}

%%
%
%%

\item \schrittE{1-1}

Wir betrachten nur die zusammengesetzten Attribute:

\begin{itemize}
\item \fa{A, E -> D}:

\ahl{A, E}{E}{A, E, F, B, \textbf{D}} \\
\ahl{A, E}{A}{E}

\item \fa{A, G -> H}:

$\ah{F, \m{A}} = \m{A, E, F, B, D}$ \\
$\ah{F, \m{G}} = \m{G}$
\end{itemize}

\FA{F -> E;
A -> B, D;
A -> D;
A -> E, F;
AG -> H;
}

%%
%
%%

\item \schrittE{1-2}

Nur die Attribute betrachten, die rechts doppelt vorkommen:

$E$:

$\ah{F - \m{F \rightarrow E}, \m{F}} = \m{F}$ \\
$\ah{F - \m{A \rightarrow E}, \m{A}} = \m{A, B, D, F, \textbf{E}}$

$D$:

$\ah{F - \m{A \rightarrow D}, \m{A}} = \m{A, B, \textbf{D}, F, E}$

$A \rightarrow D$ kann wegen der Armstrongschen Dekompositionsregel
weggelassen werden. Wenn gilt $A \rightarrow B, D$, dann gilt auch $A
\rightarrow B$ und $A \rightarrow D$

\textbf{FDs}

\FA{%
  F -> E;
  A -> B, D;
  A -> NICHTS;
  A -> F;
  AG -> H;
}

\item \schrittE{1-3}

\FA{%
  F -> E;
  A -> B, D;
  A -> F;
  AG -> H;
}

\item \schrittE{1-4}

\FA{%
  F -> E;
  A -> B, D, F;
  AG -> H;
}

Jetzt die weiteren Hauptschritte:

\end{enumerate}

\item \schrittE{2}

\begin{compactitem}
\item \liRelation[R1]{F, E}
\item \liRelation[R2]{A, B, D, F}
\item \liRelation[R3]{A, G, H}
\end{compactitem}

\item \schrittE{3}

Schlüsselkandidaten hinzufügen, falls nicht vorhanden:
\liRelation[R4]{A, C, G}

\begin{compactitem}
\item \liRelation[R1]{F, E}
\item \liRelation[R2]{A, B, D, F}
\item \liRelation[R3]{A, G, H}
\item \liRelation[R4]{A, C, G}
\end{compactitem}

\item \schrittE{4}

\liNichtsZuTun{}

\end{enumerate}
\end{liAntwort}

\end{document}
