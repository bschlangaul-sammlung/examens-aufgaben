\documentclass{lehramt-informatik-aufgabe}
\liLadePakete{normalformen,synthese-algorithmus}
\begin{document}
\let\ah=\liAttributHuelle
\let\ahl=\liLinksReduktionInline
\let\ahr=\liRechtsReduktionInline
\let\fa=\liFunktionaleAbhaengigkeit
\let\FA=\liFunktionaleAbhaengigkeiten
\let\m=\liAttributMenge
\let\r=\liRelation
\let\u=\underline

\let\schrittE=\liSyntheseUeberErklaerung

\liAufgabenTitel{Relation A-H}

\section{Synthesealgorithmus
\index{Synthese-Algorithmus}
\footcite[Seite 1: Synthesealgorithmus, Aufgabe 2]{db:pu:4}
}

Überführen Sie das Relationenschema mit Hilfe des Synthesealgorithmus in
die 3. Normalform!

\begin{center}
\r[R]{A, B, C, D, E, F, G, H}
\end{center}

\FA{
  F -> E;
  A -> B, D;
  A, E -> D;
  A -> E, F;
  A, G -> H;
}

\begin{liAntwort}

\begin{enumerate}

\item Kanonische Überdeckung

\begin{enumerate}

%%
%
%%

\item \schrittE{1-1}

Wir betrachten nur die zusammengesetzten Attribute:

\begin{itemize}
\item \fa{A, E -> D}:

$D \in$ \ahl{A, E}{E}{A, E, F, B, \textbf{D}} \\
$D \notin$ \ahl{A, E}{A}{E}

\item \fa{A, G -> H}:

$H \notin$ \ahl{A, G}{G}{A, E, F, B, D} \\
$H \notin$ \ahl{A, G}{A}{G}
\end{itemize}

\FA{
  F -> E;
  A -> B, D;
  A -> D;
  A -> E, F;
  A, G -> H;
}

%%
%
%%

\item \schrittE{1-2}

Nur die Attribute betrachten, die rechts doppelt vorkommen:
\begin{description}

\item[$E$:] \strut

\ahr{F -> E}{}{F}{F} \\
\ahr{A -> E, F}{A -> E}{A}{A, B, D, F, \textbf{E}}

\item[$D$:] \strut

\ahr{A -> D}{}{A}{A, B, \textbf{D}, F, E}

\end{description}

\fa{A -> D} kann wegen der Armstrongschen Dekompositionsregel
weggelassen werden. Wenn gilt \fa{A -> B, D}, dann gilt auch \fa{A -> B}
und \fa{A -> D}

\FA{
  F -> E;
  A -> B, D;
  A -> NICHTS;
  A -> F;
  A, G -> H;
}

\item \schrittE{1-3}

\FA{
  F -> E;
  A -> B, D;
  A -> F;
  A, G -> H;
}

\item \schrittE{1-4}

\FA{
  F -> E;
  A -> B, D, F;
  A, G -> H;
}

\end{enumerate}

\item \schrittE{2}

\r[R1]{\u{F}, E}\\
\r[R2]{\u{A}, B, D, F}\\
\r[R3]{\u{A, G}, H}

\item \schrittE{3}

\r[R1]{\u{F}, E}\\
\r[R2]{\u{A}, B, D, F}\\
\r[R3]{\u{A, G}, H}\\
\r[R4]{\u{A, C, G}}

\item \schrittE{4}

\liNichtsZuTun{}

\end{enumerate}
\end{liAntwort}

\end{document}
