\documentclass{lehramt-informatik-aufgabe}
\liLadePakete{normalformen,synthese-algorithmus}
\begin{document}
\let\FA=\liFunktionaleAbhaengigkeiten
\let\fa=\liFunktionaleAbhaengigkeit
\let\m=\liAttributMenge
\let\ah=\liAttributHuelle
\let\schrittE=\liSyntheseUeberErklaerung

\liAufgabenTitel{Kanonische Überdeckung (Kemper)}

\section{Kanonische Überdeckung (kleines Beispiel aus Kemper)
\index{Kanonische Überdeckung}
\footcite[Seite 186]{kemper}
}

\FA{%
  A -> B;
  B -> C;
  A, B -> C;
}

\begin{liAntwort}

\begin{enumerate}
\item \schrittE{1-1}

$\ah{F, \m{A, B} - \m{B}} = \m{A, B, C}$

$\ah{F, \m{A, B} - \m{A}} = \m{C}$

\FA{%
  A -> B;
  B -> C;
  A -> C;
}

\item \schrittE{1-2}

\FA{%
  A -> B;
  B -> C;
  A -> NICHTS;
}

\item \schrittE{1-3}

\FA{%
  A -> B;
  B -> C;
}

\item \schrittE{1-4}

Nichts zu tun
\end{enumerate}

\end{liAntwort}
\end{document}
