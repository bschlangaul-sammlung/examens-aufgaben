\documentclass{bschlangaul-aufgabe}
\bLadePakete{normalformen}
\begin{document}
\let\ah=\bAttributHuelle
\let\fa=\bFunktionaleAbhaengigkeit
\let\FA=\bFunktionaleAbhaengigkeiten
\let\m=\bAttributMenge
\def\e{\text{ERG}}

\bAufgabenTitel{Schlüsselkandidat von R}

\section{Schlüsselkandidaten von Relation \emph{Abstrakt}
\footcite[Seite 1, Aufgabe 3]{db:ab:5}
\index{Schlüsselkandidat}
}

Gegeben sei die Relation \emph{Abstrakt} mit dem Schema
\bRelation[Abstrakt]{A, B, C, D, E}
und die Menge der funktionalen Abhängigkeiten

\FA[$F$]{
  A -> B, C;
  C, D -> E;
  A, C -> E;
  B -> D;
}.

\noindent
Bestimmen Sie die Schlüsselkandidaten von Abstrakt!

\begin{liAntwort}
Das Attribut $A$ kommt auf keiner rechten Seite der Funktionalen
Abhängigkeiten aus $F$ vor und kann deshalb in keinem Fall durch ein
anderes Attribut bestimmt werden. Damit muss $A$ in jedem
Schlüsselkandidaten von Abstrakt enthalten sein. Ist {A} bereits ein
Superschlüssel, ist die Menge folglich der (einzig mögliche)
Schlüsselkandidat. Wir überprüfen die Superschlüsseleigenschaft mit dem
Attributhüllenalgorithmus:

\bigskip

\noindent
\begin{tabular}{|l|l|}
\hline
ERG & Begründung \\\hline
$\e = \m{A}$ & Initialisierung \\\hline
$\e = \m{A} \cup \m{B, C}$ & \fa{A -> B, C} \\\hline
$\e = \m{A, B, C}$ & \fa{C, D -> E} \\\hline
$\e = \m{A, B, C} \cup \m{E}$ & \fa{A, C -> E} \\\hline
$\e = \m{A, B, C, E} \cup \m{D}$ & \fa{B -> D} \\\hline
$\e = \m{A, B, C, D, E}$ & \\\hline
\end{tabular}

\bigskip
\noindent
$\e = \m{A, B, C, D, E}$ kann bei einem zweiten Durchlauf nicht mehr
ändern, da die Menge bereits alle Attribute von Abstrakt enthält. Die
Attributhülle von $A$ über $F$ entspricht der Attributmenge von
\emph{Abstrakt}.

\bigskip
\noindent
$\ah{F, \m{A}} = \m{A, B, C, D, E} = R$

\bigskip
\noindent
$\rightarrow \m{A}$ ist der Schlüsselkandidat von \emph{Abstrakt}.

\end{liAntwort}

\end{document}
