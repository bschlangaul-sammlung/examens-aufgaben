\documentclass{bschlangaul-theorie}
\bLadePakete{mathe,normalformen,pseudo,syntax}

\begin{document}
\let\ah=\bAttributHuelle
\let\m=\bAttributMenge

%-----------------------------------------------------------------------
%
%-----------------------------------------------------------------------

\chapter{Schlüssel}

\section{Online-Tools zur Bestimmung von Schlüssel}

\begin{itemize}
\item \href{https://normalizer.db.in.tum.de/index.py}{DB->normalizer der TU München}

\item \href{http://www.ict.griffith.edu.au/normalization_tools/normalization/ind.php}
{Normalization Tool der Universität Griffith, Australien}
\end{itemize}

\section{Definitionen der wichtigsten Begriffe}

\begin{description}
\item[Superschlüssel]

(gelegentlich auch Oberschlüssel genannt) Attribut oder
Attributkombination, von der \bEmph{alle Attribute} einer Relation
funktional \bEmph{abhängen}.
\footcite[Seite 181 Kapitel 6.2 „Superschlüssel“]{kemper}

\item[Schlüsselkandidat]

(auch \bEmph{Kandidatenschlüssel} oder Alternativschlüssel genannt) ist
ein \bEmph{Minimaler} Superschlüssel. Keine Teilmenge dieses
Superschlüssels ist ebenfalls Superschlüssel.

\item[Primärschlüssel]

\bEmph{Unter allen Schlüsselkandidaten} einer Relation wird ein
sogenannter Primärschlüssel \bEmph{ausgewählt}.

\item[Schlüsselattribut]

Attribut, das \bEmph{Teil eines Schlüsselkandidaten} ist.

\item[Nicht-Schlüsselattribut]

Attribut, das an \bEmph{keinem} der Schlüsselkandidaten beteiligt ist.
\footcite[Seite 7]{db:fs:4}
\end{description}

%-----------------------------------------------------------------------
%
%-----------------------------------------------------------------------

\section{Bestimmung von Schlüsselkandidaten ohne Algorithmus}

Schlüsselkandidaten können manchmal auch ohne Anwendung des Algorithmus
gefunden werden. Bestimmen Sie die Menge X= {A 1 , ..., A n } der
Attribute, die auf keiner rechten Seite einer funktionalen Abhängigkeit
aus F (Menge der funktionalen Abhängigkeiten) vorkommen. Diese Attribute
kann man nicht durch Ableiten gewinnen. Sie müssen in jedem
Schlüsselkandidaten vorhanden sein sein. Auch Attribute, die in keiner
der Funktionalen Abhängigkeiten in irgendeiner Weise vorkommen, müssen
Teil jedes Schlüsselkandidaten sein.

Bestimmen Sie dann die Attributhülle von X, um zu prüfen, ob X
Superschlüssel von R ist. Falls ja, ist X einziger Schlüsselkandidat, da
man alle darin enthaltenen Attribute nicht aus Ableitungen gewinnen kann
und weil die Schlüsselkandidaten minimal sind. Falls nein, muss entweder
der Algorithmus zur Bestimmung von Schlüsselkandidaten angewendet werden
oder man findet die Schlüsselkandidaten durch „systematischen“ Aufbau,
ausgehend von der Menge X.
\footcite[Seite 11]{db:fs:4}

\subsection{Zusammenfassung}

\begin{itemize}
\item Attribute, die auf \bEmph{keiner rechten Seite} einer
Funktionalen Abhängigkeit vorkommen.

\item Attribute, die in \bEmph{keiner Funktionalen Abhängigkeit}
vorkommen.
\end{itemize}

%-----------------------------------------------------------------------
%
%-----------------------------------------------------------------------

\section{Algorithmus zur Bestimmung von Schlüsselkandidaten\footcite[Seite 8]{db:fs:4}}

\begin{algorithm}

\Begin{
  $Erg \longleftarrow \m{}$\;
  $Test \longleftarrow \m{\m{\textit{alle Attribute der Relation}}}$\;
  $K \longleftarrow \m{\textit{alle Attribute der Relation}}$\;
  \While{bis $\textit{Test}$ leer ist}{
    \tcc{Wähle ein Attribut aus $K$}
    \ForEach{$A \longleftarrow K$}{
      \tcc{Also die Menge $K$ ohne das Attribut $A$}
      \If{$\ah{F,K \SLASH \m{A}} = R$}{
        streiche $K$ aus $\textit{Test}$\;
        füge $K \SLASH \m{A}$ in $\textit{Test}$ ein\;
        \tcc{$K$ selbst bleibt unverändert}
      }
    }
  }

entferne ein anderes Attribut aus $K$, so lange, bis alle
Attribute reihum untersucht wurden.\;

wenn kein $K\m{A}$ Superschlüssel $\Rightarrow$
$K$ ist Schlüsselkandidat!
Füge $K$ zu $Erg$ hinzu und lösche $K$ aus $\textit{Test}$.\;

mache dasselbe mit allen Mengen, die jetzt in $\textit{Test}$ sind,
bis $\textit{Test}$ leer ist\;
}

\caption{Algorithmus zur Bestimmung von Schlüsselkandidaten}
\end{algorithm}

%%
%
%%

\bPseudoUeberschrift{Algorithmus zur Bestimmung aller Superschlüssel in Java}

\bJavaDatei[firstline=180,lastline=204]{db/AlgorithmenSammlung}

%%
%
%%

\bPseudoUeberschrift{Algorithmus zur Bestimmung aller Schlüsselkandidaten in Java}

\bJavaDatei[firstline=206,lastline=231]{db/AlgorithmenSammlung}

\literatur

\end{document}
