\documentclass{bschlangaul-haupt}
\liLadePakete{syntax,er,rmodell}
\usepackage{soul}

\begin{document}

%%%%%%%%%%%%%%%%%%%%%%%%%%%%%%%%%%%%%%%%%%%%%%%%%%%%%%%%%%%%%%%%%%%%%%%%
% Theorie-Teil
%%%%%%%%%%%%%%%%%%%%%%%%%%%%%%%%%%%%%%%%%%%%%%%%%%%%%%%%%%%%%%%%%%%%%%%%

\chapter{Entity-Relation-System}

\begin{liQuellen}
\item \cite{wiki:entity-relationship-modell}
\end{liQuellen}

\begin{itemize}
\item Datenmodell: Eignung zur Darstellung des konzeptuellen
Datenbankschemas
\item standardisierte graphische Notation: ER-Diagramm
\item Vorgang der Modellierung: ER-Entwurf
\item Resultat: ER-Modell
\item Vorteil: Kann leicht in Tabellen einer relationalen Datenbank
überführt werden
\item jedoch: nur Strukture der Daten, keine Datenmanipulation
\end{itemize}

\subsection{Begriffsklärung}

Miniwelt:

\begin{itemize}
\item Entity: Object
\item Relationships: Beziehungen zwischen den Werten
\item Attribute: Eigenschaften von Entities oder Relationships
\item Attributewerte: Werte der Attribute
\end{itemize}

Entity-Typen: Gleichartige Entities, mit den gleichen Eigenschaften
Relationship-Typen: Beziehungen gleicher Art

Dabei gilt:

\begin{enumerate}
\def\labelenumi{\arabic{enumi}.}
\item nicht disjunkt: ein Element mehrer Entity-Typen sein.
\item Stelligkeit eines Relationship-Typs
\item Entity-Typ kein mehrere Relationship-Typen haben
\item Schemaebene: Entity-Typ, Relationship-Typ, Attribute; Instanz: Entity,
Relationship, Attributwert
\item Modellierung auf der Schemaebene, keine konkreten Entites
\end{enumerate}

\subsection{Graphische Notation}

\begin{itemize}
\item Entity-Typen: Rechtecke
\item Relationship-Type: Rauten
\item Attribute: Ovale
\end{itemize}

\subsection{Rollennamen}

Zur genaueren Charakterisierung von Entity-Typen und Relationship-Typen

\subsection{Domäne}

zulässige Attributewerte

\begin{itemize}
\item extensional: Aufzählung alles zulässigen Werte
\item intensional: Angabe allgemein bekannter Mengen
\end{itemize}

Mehrwertiges Attribute (Doppelkreis): mehrere Telefonnummern
Abgeleitetes Attribute (gestrichelte Linie): Alter (vom Geburstag)

Relationshiptypen:

\begin{itemize}
\item Binäre zwischen 2 Entitytypen
\item Ternäre zwischen 3 Entitytypten
\end{itemize}

\section{Entity-Relation-System}

% https://usehardware.de/datenbanksysteme-iv-entity-relationship-modell-er-modell-datenbankdarstellungen-i/
\begin{tabular}{cl}

\begin{tikzpicture}
\node[entity] {Entität};
\end{tikzpicture} &
Entität
\\

\begin{tikzpicture}
\node[weak entity] {Entität};
\end{tikzpicture} &
schwache Entität
\\

\begin{tikzpicture}
\node[relationship] {Beziehung};
\end{tikzpicture} &
Beziehung / Relationship
\\

\begin{tikzpicture}
\node[attribute] {Attribut};
\end{tikzpicture} &
einfaches Attribut
\\

\begin{tikzpicture}[node distance=1.6cm]
\node[attribute] (att1) {Attribut};
\node[attribute] (att2) [above right of=att1] {Attribut} edge (att1);
\node[attribute] (att3) [above left of=att1] {Attribut} edge (att1);
\end{tikzpicture} &
zusammengesetztes Attribut
\\

\begin{tikzpicture}
\node[derived attribute] {Attribut};
\end{tikzpicture} &
abgeleitetes Attribut
\\

\begin{tikzpicture}
\node[multi attribute] {Attribut};
\end{tikzpicture} &
Mehrfachattribut
\\

\begin{tikzpicture}
\node[attribute] {\key{Attribut}};
\end{tikzpicture} &
Schlüsselattribut
\\

\begin{tikzpicture}
\node[attribute] {\discriminator{Attribut}};
\end{tikzpicture} &
schwaches Schlüsselattribut
\\

\begin{tikzpicture}
\path[link] (0,0) -- (3,0);
\end{tikzpicture} &
partielle Teilnahme
\\

\begin{tikzpicture}
\path[draw,weak] (0,0) -- (3,0);
\end{tikzpicture} &
totale Teilnahme
\\

\end{tabular}

\begin{itemize}
\item Datenmodell: Eignung zur Darstellung des konzeptuellen
Datenbankschemas
\item standardisierte graphische Notation: ER-Diagramm
\item Vorgang der Modellierung: ER-Entwurf
\item Resultat: ER-Modell
\item Vorteil: Kann leicht in Tabellen einer relationalen Datenbank
überführt werden
\item jedoch: nur Strukture der Daten, keine Datenmanipulation
\end{itemize}

\subsection{Begriffsklärung}

Miniwelt:

\begin{description}
\item[Entity:] Objekt

\item[Entity-Typen:] Gleichartige Entities, mit den gleichen
Eigenschaften

\begin{tikzpicture}
\node[entity] (schulklasse) {Schulklasse};
\end{tikzpicture}

\item[Relationships:] Beziehungen zwischen den Werten

\item[Relationship-Typen:] Beziehungen gleicher Art

\begin{tikzpicture}
\node[relationship,align=center] (hatKlassenleitung) {hatKlassen-\\leitungIn};
\end{tikzpicture}

\item[Attributewerte:] Werte der Attribute

\item[Attribute:] Eigenschaften von Entities oder Relationships

\begin{tikzpicture}
\node[attribute] (klassenzimmer) {Klassenzimmer};
\end{tikzpicture}
\end{description}

Dabei gilt:

\begin{enumerate}
\def\labelenumi{\arabic{enumi}.}
\item nicht disjunkt: ein Element mehrer Entity-Typen sein.
\item Stelligkeit eines Relationship-Typs
\item Entity-Typ kann mehrere Relationship-Typen haben
\item Schemaebene: Entity-Typ, Relationship-Typ, Attribute; Instanz: Entity,
Relationship, Attributwert
\item Modellierung auf der Schemaebene, keine konkreten Entites
\end{enumerate}

\subsection{Rollennamen}

Zur genaueren Charakterisierung von Entity-Typen und Relationship-Typen

\subsection{Domäne}

zulässige Attributewerte

\begin{itemize}
\item extensional: Aufzählung alles zulässigen Werte
\item intensional: Angabe allgemein bekannter Mengen
\end{itemize}

Mehrwertiges Attribut (Doppelkreis): mehrere Telefonnummern
Abgeleitetes Attribut (gestrichelte Linie): Alter (vom Geburstag)

Relationshiptypen:

\begin{itemize}
\item Binäre zwischen 2 Entitytypen
\item Ternäre zwischen 3 Entitytypten
\end{itemize}

%-----------------------------------------------------------------------
%
%-----------------------------------------------------------------------

\section{Generalisierung\footcite[Seite 27]{db:fs:1}}

Die Generalisierung ist eine \memph{Abstraktion auf Ebene der
Entitytypen}, um eine bessere Strukturierung zu erzielen.
%
Die \memph{Eigenschaften ähnlicher Entitytypen} werden einem
\memph{gemeinsamen Obertyp} zugeordnet. Die Eigenschaften (Attribute),
die nicht von den generalisierten Entitytypen geteilt werden, verbleiben
in den Untertypen. Der \memph{Untertyp stellt somit Spezialisierung} des
Obertyps dar.
%
Der \memph{Untertyp erbt alle Eigenschaften} des Obertyps. Die
Entitymenge des Untertyps ist eine Teilmenge der Entitymenge des
Obertyps.
%
Man spricht von einer \memph{disjunkten Spezialisierung}, wenn eine
Entity nur Mitglied von einer der Untertypen ist und es keine
Überschneidungen gibt.
%
\memph{Vollständige Spezialisierung} nennt man die Modellierung, wenn es
keine direkten Elemente des Obertyps gibt. Alle Entitys, die zur Menge
des Obertyps gehören, gehören auch zu einem Untertyp. Der Obertyp ist
Vereinigung der Untertypen.
% https://www.geeksforgeeks.org/generalization-specialization-and-aggregation-in-er-model/
Gezeichnet wird die Generalisierung meist durch ein \memph{Dreieck}.
In manchen Diagrammen stellt ein nach \memph{oben} zeigendes Dreieck die
\memph{Generalisierung}, ein nach unten gerichtetes Dreieick die
\memph{Spezialisierung} dar.

\begin{center}
\begin{tikzpicture}[node distance=1.5cm]
\node[entity] (Person) {Person};
\node[isa,below=of Person] (ISA) {ISA} edge (Person);
\node[entity,below left=of ISA] {Lehrer} edge (ISA);
\node[entity,below right=of ISA] {Schüler} edge (ISA);
\end{tikzpicture}
\end{center}

%-----------------------------------------------------------------------
%
%-----------------------------------------------------------------------

\section{Schwache Entitytypen}

Ein schwacher Entitytypen ist ein Entitytyp, der in seiner Existenz von
einem \memph{anderen Entitytyp abhängig} ist und oft \memph{nur in
Kombination} mit dem Schlüssel des \memph{übergeordneten Entitytyps}
eindeutig identifiziert werden kann.

Eine \memph{Totale Teilnahme} (auch \emph{totale Partizipation} oder
\emph{totale Beteiligung} genannt) liegt dann vor, wenn jede Entität
eines schwachen Entitättypen im Beziehung mit dem übergeordneten
Entitytyps steht.

Bespielsweise  kann es keine Entity des Entitytyps \emph{Klassenzimmer}
geben, die keine Beziehung zu einem Entity des Typs \emph{Schulgebäude}
hat. Das bedeutet auch, dass jedes \emph{Schulgebäude} mindestens einen
\emph{Klassenzimmer} hat. Die Nummer eines Klassenzimmers ist nur
innerhalb eines Schulgebäudes eindeutig. Der Schlüssel lautet dann:
\texttt{Klassenzimmers.Nummer} und \texttt{Schulgebäude.Nummer}.

Die Beziehung zwischen \emph{starken} und \emph{schwachem} Typ ist immer
eine \texttt{1:N}-Bezieh\-ung oder \texttt{1:1} in seltenen Fällen.

\footcite[Seite 26]{db:fs:1}

\begin{center}
\begin{tikzpicture}[node distance=1.5cm]
\node[entity] (Schulgebäude) at (0,0) {Schulgebäude};
\node[attribute] (Nummer) [above of=Schulgebäude] {\key{Nummer}} edge (Schulgebäude);
\node[attribute] (Höhe) [below of=Schulgebäude] {Höhe} edge (Schulgebäude);

\node[weak entity] (Klassenzimmer) at (6,0) {Klassenzimmer};
\node[attribute] (Nummer) [above of=Klassenzimmer] {\discriminator{Nummer}} edge (Klassenzimmer);
\node[attribute] (Größe) [below of=Klassenzimmer] {Größe} edge (Klassenzimmer);

\node[ident relationship] (liegtIn) at (3,0) {liegtIn}
  edge (Schulgebäude)
  edge[weak] (Klassenzimmer);
\end{tikzpicture}
\end{center}

%-----------------------------------------------------------------------
%
%-----------------------------------------------------------------------

\section{Funktionalitäten}

Die (min,max)-Notation zählt die Ausprägung von \emph{Beziehungen},
während die anderen Notationen \emph{Entitätstypausprägungen} zählen.
(Wikipedia)

\subsection{einfache / Chen-Notation}

Quellen
\footcite[2.7.1 Seite 41]{kemper}
\footcite[Seite 59]{brinda}

\begin{description}
\item[Schreibweise] 1:1, 1:n, n:1, n:m
\item[Bestimmung]

Auf zu bestimmenden Entitytyp zeigen und Frage formulieren:
%
\emph{„Wie viele X Entities sind in Relationship mit (einem) anderem/n
Entity/ies?“}
%
Das zu bestimmtende Entity ist diesen Fragensätzen \textbf{Objekt}.

\end{description}

\subsection{min-max-Notation / Kardinalitäten}

Quellen
\footcite[2.7.3 Seite 46]{kemper}
\footcite[Seite 62]{brinda}

\begin{description}
\item[Schreibweise] (0, *)
\item[Bestimmung]

Auf zu bestimmenden Entitytyp zeigen und Aussage formulieren:
%
\emph{„Ein Entity ist in Relationship mit X (min, max) anderem/n
Entity/ies“.}
%
Das zu bestimmtende Entity ist diesen Aussagesatz \textbf{Subjekt}.
\end{description}

\literatur
\end{document}
