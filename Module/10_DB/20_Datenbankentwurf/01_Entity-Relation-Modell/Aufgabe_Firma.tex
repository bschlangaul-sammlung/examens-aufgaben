\documentclass{bschlangaul-aufgabe}

\begin{document}
\bAufgabenMetadaten{
  Titel = {Aufgabe 9: Miniwelt},
  Thematik = {Firma},
  Referenz = DB.Datenbankentwurf.Entity-Relation-Modell.Firma,
  RelativerPfad = Module/10_DB/20_Datenbankentwurf/01_Entity-Relation-Modell/Aufgabe_Firma.tex,
  ZitatSchluessel = db:ab:klausurvorbereitung,
  BearbeitungsStand = mit Lösung,
  Korrektheit = unbekannt,
  Ueberprueft = {unbekannt},
  Stichwoerter = {Entity-Relation-Modell},
}

\section{Aufgabe 9: Miniwelt
\index{Entity-Relation-Modell}
\footcite{db:ab:klausurvorbereitung}}

Modellieren Sie folgende Miniwelt: Eine Firma mit Firmenname und
Firmensitz besteht aus Abteilungen (Abteilungsname, Abteilungsnummer).
Die Abteilungen haben Mitarbeiter, von denen Personalnummer, Büronummer
und die dienstliche Telefonnummer gespeichert werden sollen.
\end{document}
