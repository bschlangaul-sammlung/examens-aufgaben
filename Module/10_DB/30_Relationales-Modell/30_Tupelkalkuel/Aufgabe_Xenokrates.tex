\documentclass{bschlangaul-aufgabe}
\bLadePakete{mathe}
\begin{document}
\bAufgabenTitel{Xenokrates}

\section{Übungen der TU München
\index{Tupelkalkül}
\footcite{net:pdf:tum:db-tutor-uebung}
}

Lösen Sie die Aufgaben im Tupel- und Domänenkalkül:

\begin{enumerate}
\item Geben Sie alle Vorlesungen an, die der Student
\emph{Xenokrates} gehört hat.

\begin{bAntwort}
$\{
  v |
  v \in Vorlesungen \land
  \exists h \in hoeren(
    v.VorlNr = h.VorlNr \land
    \exists s \in Studenten(
      h.MatrNr = s.MatrNr \land s.Name = \mlq Xenokrates \mrq
    )
  )
\}
$
\end{bAntwort}

\item Geben Sie die Titel der direkten Voraussetzungen für die
Vorlesung Wissenschaftstheorie an.

\item Geben Sie Paare von Studenten(-Namen) an, die sich aus
der Vorlesung Grundzüge kennen.

\end{enumerate}
\end{document}
