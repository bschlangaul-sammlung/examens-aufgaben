\documentclass{bschlangaul-theorie}
\bLadePakete{er,rmodell}
\begin{document}

%-----------------------------------------------------------------------
%
%-----------------------------------------------------------------------

\chapter{Das Relationenmodell}

\begin{bQuellen}
\item \cite[Überführung in ein relationales Modell]{wiki:entity-relationship-modell}
\item \cite{wiki:relational-model}
\end{bQuellen}

%-----------------------------------------------------------------------
%
%-----------------------------------------------------------------------

\section{Grundbegriffe des Relationenmodells\footcite[Seite 76]{winter}}

Objekte und Beziehungen besitzen Eigenschaften, die \bEmph{Attribute}
genannt werden.
%
Diese Attribute sind für jedes Objekt bzw. jede Beziehung durch
\bEmph{Attributwerte} konkretisiert.
%
Die Menge aller möglichen Attributwerte eines Attributs heißt
\bEmph{Domäne}.
%
Objekte bzw. Beziehungen werden durch \bEmph{Tupel} \dh Listen
der entsprechenden Attributwerte, dargestellt.
%
\bEmph{Gleichartige Objekte} werden durch \bEmph{gleichartig
aufgebaute Tupel} repräsentiert. Diese Tupel werden zu einer Menge, der
sogenannten \bEmph{Relation}, zusammengefasst.
%
Während aber beim Entity-Relation\-ship-Modell zwei
Strukturierungskonzepte in Form von Entity- und Relationship-Typ zur
Verfügung gestellt werden, besitzt das Relationenmodell mit der Relation
lediglich \bEmph{ein Konzept}.

Eine Relation ist eine \bEmph{Menge gleichartig gebauter Tupel}. Diese
Eigenschaft unterscheidet eine Relation grundsätzlich von einer Tabelle.
Aus der Mengeneigenschaft der Relation ergeben sich folgende wichtige
Konsequenzen:
%
Es existieren \bEmph{keine Tupelduplikate}, \dh es gibt zu keinem
Zeitpunkt zwei Tupel mit identischen Attributwerten. Die Tupel sind
nicht geordnet, \dh es existiert \bEmph{keine festgelegte Reihenfolge}
der Elemente.\footcite[Seite 77]{winter}

%-----------------------------------------------------------------------
%
%-----------------------------------------------------------------------

\section{Vom Entity-Relationship-Modell zum Relationenmodell\footcite[Seite 33]{db:fs:1}}

Wir legen für jeden Entity-Typen ein \bEmph{Relationenschema} festgelegt.
%
Ebenso erstellen wir (zunächst) für jeden Relation\-ship-Typen ein
Relationenschema, das zusätzlich zu den Attributen des Relationship-Typs
die \bEmph{Schlüssel der zugehörigen Entity-Typen} enthält, welche
\bEmph{Fremdschlüssel} genannt werden. Das/die Schlüsselattribut/e
werden/wird unterstrichen.
%
Ein Relationenschema hat folgenden Aufbau:

\begin{center}
\texttt{Relationenname\{[Attribut1: Domäne1, Attribut2: Domäne2, ...])}
\end{center}

\noindent
Es gibt eine vereinfachte Notation:

\begin{center}
\texttt{Relationenname(Attribut1, Attribut2, Attribut3[Fremdrelation])}
\end{center}

%-----------------------------------------------------------------------
%
%-----------------------------------------------------------------------

\section{Das verfeinertes Relationenschema\footcite[Seite 34]{db:fs:1}}

Jedes Relationenmodell kann \bEmph{verfeinert} werden, indem Relationen,
deren Schlüssel aus den \bEmph{gleichen Attributen} besteht,
\bEmph{zusammengefasst} werden.
Grundsätzlich können \bEmph{nur} Relationen eliminiert werden,
die \bEmph{1:1 bzw. 1:N}-Beziehungen repräsentieren.

\begin{description}
\item[1:N-Beziehungen]

Integration in Relation der N-Seite (ansonsten ergeben sich Anomalien!).

\item[1:1-Beziehungen]

Integration in beide Relationen möglich, aber: Abwägen, wo mehr
„Lehrstellen“ in Form von NULL-Werten entstehen.
\end{description}

\noindent
Nur Relationen, die aus \bEmph{Relationship-Typen} entstanden sind,
dürfen \bEmph{eliminiert} werden! Diese Relationen dürfen nur dann
eliminiert werden, wenn die in ihnen enthaltene Information in eine
andere Relation vollständig integriert werden kann!

\section{Automatisierte Überführung}

Die genaue Überführung, die automatisiert werden kann, erfolgt in 7
Schritten:

\begin{description}
\item[Starke Entitätstypen:]

Für jeden starken Entitätstyp wird eine Relation mit seinen Attributen
und seinen Primärschlüssel erstellt.

\item[Schwache Entitätstypen:]

Für jeden schwachen Entitätstyp wird eine Relation erstellt. Der
Primärschlüssel des starken Entitätstyps wird als Fremdschlüssel
ergänzt, um den schwachen Entitätstyp zu identifizieren.

\item[1:1-Beziehungstypen:]

Für einen 1:1-Beziehungstyp zweier Entitätstypen wird eine der beiden
Relationen um den Fremdschlüssel für die jeweils andere Relation
erweitert.

\item[1:N-Beziehungstypen:]

Für den 1:N-Beziehungstyp zweier Entitätstypen wird der Fremdschlüssel
der „1“-Relation in die „N“-Relation geschrieben.

\item[N:M-Beziehungstypen:]

Für jeden N:M-Beziehungstyp wird eine neue Relation mit den
Fremschlüsseln der beiden Relation ergänzt.

\item[Mehrwertige Attribute:]

Für jedes mehrwertige Attribut wird eine Relation erstellt und als
Fremdschlüssel der Primärschlüssel des zugehörigen Entitätstyps
verwendet.

\item[n-äre Beziehungstypen:]

Für jeden Beziehungstyp mit einem Grad $n > 2$ wird eine neue Relation
erstellt.\footcite[Überführung in ein relationales
Modell]{wiki:entity-relationship-modell}
\end{description}

\literatur

\end{document}
