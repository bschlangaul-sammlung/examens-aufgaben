\documentclass{bschlangaul-theorie}

\begin{document}

%%%%%%%%%%%%%%%%%%%%%%%%%%%%%%%%%%%%%%%%%%%%%%%%%%%%%%%%%%%%%%%%%%%%%%%%
% Theorie-Teil
%%%%%%%%%%%%%%%%%%%%%%%%%%%%%%%%%%%%%%%%%%%%%%%%%%%%%%%%%%%%%%%%%%%%%%%%

\chapter{Einführung in relationale Datenbanksysteme \&
Datenmodellierung}

%-----------------------------------------------------------------------
%
%-----------------------------------------------------------------------

\section{Grundlagen von Datenbanksystemen}

%%
%
%%

\subsection{Begriffsklärung\footcite[Seite 13]{winter}}

\begin{itemize}
\item DB = Datenbank
\item Menge der gespeicherten Daten
\item DBMS = Datenbankmanagementsystem
\item Gesamtheit aller Programme für den Umgang mit den Daten
\item DBS = Datenbanksystem
\item System zur Beschreibung, Verwaltung, Speicherung und Wiedergewinnung
großer Datenmengen
\item Möglichkeit der gleichzeitigen Nutzung durch mehrer
Anwendungsprogramme
\item Informationssystem
\item Bereitstellung von Informationen zur Steuerung eines Aufgabenbereichs
\item Zugriff auf elektronisch gespeicherte Daten
\item realisiert durch DBMS
\end{itemize}

%%
%
%%

\subsection{Begriffshierachie}

\begin{itemize}
\item Informationssystem
\item DBS = Datenbanksystem

\begin{itemize}
\item DB = Datenbank
\item DBMS = Datenbankmanagementsystem
\end{itemize}
\end{itemize}

%%
%
%%

\subsection{Aufgaben des Datenbankmanagementsystems
\footcite[Seite 13]{winter}}

\begin{itemize}
\item Persistenz
\item Datenaustausch
\item Zugriffskontrolle
\item Effizienz
\end{itemize}

%%
%
%%

\subsection{grundlegendes Prinzip bei Datenbanksystemen
(DBS)}

\begin{itemize}
\item strikte Trennung von Daten und Datenbearbeitung
\item Benutzer unabhängig von der eigentlichen Organisation der Daten
\item Zurverfügungstellung einer Datenschnittstelle
\end{itemize}

%%
%
%%

\subsection{Datenbanken in der Praxis
\footcite[Seite 13]{winter}}

\begin{itemize}
\item Einzelbenutzersystem: Datenbestand von einem Anwender eingetragen und
gepflegt

\item Mehrbenutzersystem: gleichzeitiger Zugriff von unterschiedlichen
Benutzergruppen, Arbeitsumgebung für einzelnen Benutzer wie
Einzelbenutzersystem

\item Verteilte Datenbankensysteme: DB und DBS auf verschiedene Rechner
verteilt, durch Netzwerk erscheint sie für den Benutzer als zentrale,
einheitliche Datenbank \zB Client-Server-Datenbanksysteme

\item Multidatenbanksystem
\end{itemize}

\literatur

\end{document}
