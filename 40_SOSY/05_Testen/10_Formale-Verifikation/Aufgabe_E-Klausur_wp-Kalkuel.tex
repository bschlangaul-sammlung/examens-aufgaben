\documentclass{lehramt-informatik-aufgabe}
\liLadePakete{syntax,mathe,wpkalkuel}
\begin{document}
\let\wp=\liWpKalkuel
\let\equivalent=\liWpEquivalent
\let\erklaerung=\liWpErklaerung

\section{Totale Korrektheit IV
\footcite[Totale Korrektheit IV]{sosy:e-klausur}}

Gegeben sei die folgende Methode:

\liJavaDatei[firstline=4,lastline=11]{aufgaben/sosy/totale_korrektheit/Blub}

\noindent
Berechnen Sie hierzu das folgende Kalkül:

\begin{displaymath}
\wp{R}{a \geq 0}
\end{displaymath}

\begin{liAntwort}
$\wp{R}{a \geq 0}$

\equivalent{
  \wp{if (a > 15) \{a = a - 42;\} else \{a = -a;\}}{a \geq 0}
}

%%
%
%%

\erklaerung{Aufteilung der Verzweigung in:\\ \liWpErklaerungVerzweigung}

\equivalent{
  (a > 15 \land \wp{a = a - 42}{a \geq 0})
  \lor
  (a \leq 15 \land \wp{a = -a}{a \geq 0})
}

\erklaerung{Code einsetzen}

\equivalent{
  (a > 15 \land \wp{}{a - 42 \geq 0})
  \lor
  (a \leq 15 \land \wp{}{-a \geq 0})
}

%%
%
%%

\erklaerung{Kein Code mehr verhanden. Wir lassen „wp“ weg.}

\equivalent{
  (a > 15 \land a - 42 \geq 0)
  \lor
  (a \leq 15 \land -a \geq 0)
}

%%
%
%%

\erklaerung{Wir bringen 42 nach rechts und multiplizieren $-a \geq 0$
mit $-1$.}

\equivalent{
  (a > 15 \land a \geq 42)
  \lor
  (a \leq 15 \land a \leq 0)
}

%%
%
%%

\erklaerung{Die Aussagen reduzieren, redundante Aussagen weglassen}

\equivalent{
  (a \geq 42)
  \lor
  (a \leq 0)
}

%%
%
%%

\erklaerung{Weglassen der Klammern}

\equivalent{
  a \geq 42
  \lor
  a \leq 0
}

\end{liAntwort}

\end{document}
