\documentclass{lehramt-informatik-aufgabe}
\liLadePakete{checkbox}
\begin{document}
\liAufgabenTitel{Multiple-Choice Allgemeine Software-Technologie}

\section{Allgemeine Software-Technologie, Vorgehensmodelle und Requirements Engineering
\footcite{sosy:e-klausur}}

Kreuzen Sie bei der folgenden Multiple-Choice-Frage die richtige(n)
Antwort(en) an. Auf falsch gesetzte Kreuze gibt es je einen Minuspunkt.
Die Aufgabe wird nicht mit weniger als 0 Punkten gewertet.

\begin{enumerate}

%%
%
%%

\item  Welche Vorgehensmodelle sind für Projekte mit häufigen Änderungen
gedacht?

\begin{itemize}
\liRichtig EXtreme Programming (XP)\index{EXtreme Programming}
\liFalsch Das V-Modell 97\index{V-Modell}
\liFalsch Wasserfallmodell\index{Wasserfallmodell}
\liRichtig Scrum\index{SCRUM}
\end{itemize}

%%
%
%%

\item Welche der folgenden Aussagen ist korrekt?

\begin{itemize}
\liFalsch Mittels Prototyping\index{Prototyping} versucht man die Anzahl
an nötigen Unit-Tests\index{Unit-Test} zu reduzieren.

\liRichtig Ein Ziel von Prototyping ist die Erhöhung der Qualität
während der Anforderungsanalyse\index{Anforderungsanalyse}.

\liRichtig Mit Prototyping versucht man sehr früh Feedback von
Stakeholdern zu erhalten.
\end{itemize}

%%
%
%%

\item Welche der folgenden Aussagen ist korrekt?

\begin{itemize}
\liFalsch Das Wasserfallmodell sollte nur für große Projekte eingesetzt
werden, da der Einarbeitungsaufwand sehr groß ist.

\liFalsch Eine gute Anforderungsspezifikation muss vor allem für
Ingenieure verständlich sein,  da die Anforderungsspezifikation die
Grundlage der Systementwicklung bildet.

\liFalsch Verifikation ist der Prozess der Beurteilung eines Systems
mit dem Ziel festzustellen, ob die spezifizierten Anforderungen erfüllt
sind.

\liRichtig Durch Validierung kann überprüft werden, ob das Produkt den
Erwartungen des Kunden entspricht.

\liFalsch Mit Hilfe eines Black-Box-Tests kann man die Korrektheit eines
Programmcodes beweisen.
\end{itemize}
\end{enumerate}
\end{document}
