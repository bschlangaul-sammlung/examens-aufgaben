\documentclass{lehramt-informatik}
\InformatikPakete{syntax,uml}
\usepackage{paralist}

\begin{document}

\chapter{Abstrakte Klassen / Interface}

\cite[Vererbung Seite 28-31 (PDF Seite 156-159)]{brinda}

\section{Abstrake Klassen und Methoden}

Eine abstrakte Klasse ist eine spezielle Klasse, von der keine Objekte
erstellt werden können, d.\,h. die Klasse ist nicht \memph{nicht
instanziierbar}.

Das ist nützlich, wenn eine Menge von Klassen gemeinsame Methoden
benötigt, die in einer Oberklasse implementiert werden können, aber es
keinen Sinn mach, von der Oberklasse selbst Instanzen zu erstellen.

Es ist zudem möglich, Methoden, die alle Unterklassen haben sollen, in
einer abstrakten Oberklasse zu definieren, aber nicht zu implementieren,
weil die Funktionalität in den Unterklassen leicht unterschiedlich ist,
dazu werden abstrakte \memph{Methoden} verwendet. Diese \memph{müssen}
dann in der \memph{Unterklasse implementiert} werden.
\footcite[Seite 31]{oomup:fs:3}

\begin{minted}{java}
public abstract class Vierbeiner {
  private String vierbeinerName;

  public Vierbeiner(String name) {
    this.setName(name);
  }

  public void setName(String pName) {
    vierbeinerName = pName;
  }

  public abstract void rennen();
}
\end{minted}

\begin{minted}{java}
public class Katze extends Vierbeiner{
  public Katze(String name){
    super(name);
    rennen();
  }

  public void rennen(){
    System.out.println("Deine Katze rennt!");
  }
}
\end{minted}

\section{Interfaces (Schnittstellen)
\footcite[Vererbung Seite 32-38 (PDF Seite 160-166)]{brinda}}

„Eine Schnittstelle (interface) spezifiziert einen Ausschnitt aus dem
Verhalten einer Klasse. Eine Schnittstelle besteht nur aus den
Signaturen von Operationen, d. h. sie besitzt keine Implementierung,
keine Attribute oder Assoziationen. Eine Schnittstelle ist äquivalent zu
einer abstrakten Klasse, die ausschließlich abstrakte Operationen
besitzt.“ (vgl. Balzert, 2005)

Schnittstellen legen für die Objekte der implementierenden Klassen ein
bestimmtes, gleichartiges Verhalten fest. Durch das
Schnittstellenkonzept wird gewährleistet, dass jede Klasse, die die
Schnittstelle implementiert, die in der Schnittstelle deklarierten
Methoden zur Verfügung stellt. Eine Klasse kann mehrere Schnittstellen
implementieren.

\begin{tikzpicture}
\umlclass[x=2,type=abstract]
{Figur}
{rahmendicke\\farbe}
{berechnenFlaeche()}

\umlclass[x=-2,y=-3,anchor=north]
{Rechteck}
{laenge\\breite}
{berechnenFlaeche()}

\umlclass[x=2,y=-3,anchor=north]
{Kreis}
{radius}
{berechnenFlaeche()}

\umlclass[x=6,y=-3,anchor=north]
{Dreieck}
{grundlinie\\hohe}
{berechnenFlaeche()}

\umlVHVinherit[arm2=-2cm]{Rechteck}{Figur}
\umlVHVinherit[arm2=-2cm]{Kreis}{Figur}
\umlVHVinherit[arm2=-2cm]{Dreieck}{Figur}
\end{tikzpicture}

%-----------------------------------------------------------------------
%
%-----------------------------------------------------------------------

\section{Aufgabe zu Abstrakten Klassen und Interfaces (Parkhaus)}

\begin{quellen}
\item \cite{aud:ab:1}
\item \cite[Seite 11, Thema 2, Teilaufgabe 2, Aufgabe 1]{examen:66116:2014:03}
\end{quellen}

In dieser Aufgabe werden Sie Datentypen für die Verwaltung eines
Parkhauses mit Hilfe objektorientierter Methoden definieren. Bearbeiten
Sie die folgenden Teilaufgaben in einer objektorientierten
Programmiersprache Ihrer Wahl (geben Sie diese an)! Solange nicht anders
definiert, sollen Eigenschaften und Methoden \emph{uneingeschränkt
sichtbar} sein.

\renewcommand{\labelenumi}{(\alph{enumi})}
\renewcommand{\labelenumii}{(\roman{enumii})}
\begin{enumerate}

%%
% (a)
%%

\item Erzeugen Sie eine \emph{Klasse} \java{Fahrzeug}, deren Instanzen
folgende Eigenschaften besitzen (wählen Sie geeignete Typen):

\begin{compactitem}
\item Ein amtliches Kennzeichen (Buchstaben- und Zahlenkombination).

\item Die Dimensionen des Fahrzeugs (Länge, Breite, Höhe) in Metern.

\item Das Datum der Erstzulassung. Definieren Sie hierfür entweder einen
eigenen Datentyp oder machen Sie Gebrauch von der Standardbibliothek
Ihrer gewählten Programmiersprache.
\end{compactitem}

Die Eigenschaften sollen für \emph{Unterklassen nicht sichtbar} sein.
Schreiben Sie außerdem einen Konstruktor, der eine Instanz erzeugt und
die Eigenschaften setzt!

\begin{antwort}
\inputcode[firstline=3]{aufgaben/aud/ab_1/parkhaus/Fahrzeug}
\end{antwort}

%%
% (b)
%%

\item Schreiben Sie eine \emph{Klasse} \emph{Parkplatz}, in der
ebenfalls Eigenschaften für die Dimension (Länge, Breite, Höhe) in
Metern vorgesehen sind! Die Eigenschaften sollen ebenfalls im
\emph{Konstruktor} initialisiert werden können. Außerdem soll eine
Objektmethode hinzugefügt werden, die prüft, ob ein gegebenes Fahrzeug
in den Parkplatz passt.

\begin{antwort}
\inputcode[firstline=3]{aufgaben/aud/ab_1/parkhaus/Parkplatz}
\end{antwort}

%%
% (c)
%%

\item Ein \emph{Interface} \java{Parkhaus} soll Objektmethoden für
folgende Anwendungsfälle deklarieren:

\begin{enumerate}

%%
% (i)
%%

\item Alle freien Parkplätze sollen (z.B. als \emph{Array} oder als
Instanz einer in der Standardbibliothek Ihrer verwendeten Sprache
definierten Kollektionsklasse) zurückgegeben werden.

%%
% (ii)
%%

\item Der erste freie Parkplatz, der zu einem gegebenen Fahrzeug passt,
soll zurückgegeben werden.

%%
% (iii)
%%

\item Ein gegebener Parkplatz soll für ein gegebenes Fahrzeug reserviert
werden. Deklarieren Sie das Interface und geben Sie geeignete Signaturen
die \emph{Objektmethoden} an!
\end{enumerate}

\begin{antwort}
\inputcode[firstline=3]{aufgaben/aud/ab_1/parkhaus/Parkhaus}
\end{antwort}

\item Schreiben Sie eine \emph{abstrakte Klasse}, die das Interface
\java{Parkhaus} partiell implementiert! Geben Sie außerdem eine
geeignete \emph{Implementierung} für die Objektmethode (c.ii) unter
Verwendung der existierenden Objektmethoden aus (b) und (c.i) an!

\begin{antwort}
\inputcode[firstline=3]{aufgaben/aud/ab_1/parkhaus/MeinParkhaus}
\end{antwort}

\end{enumerate}

\literatur

\end{document}
