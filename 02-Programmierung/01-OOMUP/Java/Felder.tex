\documentclass{lehramt-informatik}
\InformatikPakete{syntax}
\begin{document}

%%%%%%%%%%%%%%%%%%%%%%%%%%%%%%%%%%%%%%%%%%%%%%%%%%%%%%%%%%%%%%%%%%%%%%%%
% Theorie-Teil
%%%%%%%%%%%%%%%%%%%%%%%%%%%%%%%%%%%%%%%%%%%%%%%%%%%%%%%%%%%%%%%%%%%%%%%%

\chapter{Felder}

\section{Initialsierung
\footcite[Seite 46]{oomup:fs:3}}

Oft sind die zu speichernden Werte erst zur Laufzeit bekannt. In diesem
Fall kann man Platz für eine festgelegte Anzahl an Elementen
reservieren.

\texttt{<Datentyp>[] <Bezeichner> = new <Datentype>[<Anzahl>];}

Die Anzahl kann ein beliebiger Ausdruck sein, der zu einem \java{int}
ausgewerte wird. So angelegte Felder werden mit \memph{0-wertigen Einträgen}
gefüllt. Zum Beispiel:

\begin{minted}{java}
boolean[] korrigiert = new boolean[23];
// -> {false, false, ..., false}
\end{minted}

\begin{minted}{java}
boolean[] noten = new int[6];
// -> {0, 0, 0, 0, 0, 0}
\end{minted}

Die Größe eines einmal angelegten Feldes kann nicht mehr verändert
werden. Wieviele Elemente in einem Feld Platz haben, kann mit dem
Attribut \texttt{.length} abgefragt werden.

%%%%%%%%%%%%%%%%%%%%%%%%%%%%%%%%%%%%%%%%%%%%%%%%%%%%%%%%%%%%%%%%%%%%%%%%
% Aufgaben
%%%%%%%%%%%%%%%%%%%%%%%%%%%%%%%%%%%%%%%%%%%%%%%%%%%%%%%%%%%%%%%%%%%%%%%%

\chapter{Aufgaben}

\section{Aktionen mit Feldern}

\begin{itemize}

%%
%
%%

\item Mittelwert berechnen

\begin{minted}{java}
public double berechneMittelwert() {
  double[] feld = { 1.7, 5.2, 8.4, 2.2 };
  double summe = 0;
  for (int i = 0; i < feld.length; i++) {
    summe = summe + feld[i];
  }
  return summe / feld.length;
}
\end{minted}

%%
%
%%

\item Maximalwert berechnen

\begin{minted}{java}
public double berechneMaximalwert() {
  double[] feld = { 1.7, 5.2, 8.4, 2.2 };
  double max = feld[0];
  for (int i = 0; i < feld.length; i++) {
    double aktuelleZahl = feld[i];
    if (aktuelleZahl > max) {
      max = aktuelleZahl;
    }
  }
  return max;
}
\end{minted}

%%
%
%%

\item Minimalwert berechnen

\begin{minted}{java}
public double berechneMinimalwert() {
  double[] feld = { 1.7, 5.2, 8.4, 2.2 };
  double min = feld[0];
  for (int i = 0; i < feld.length; i++) {
    double aktuelleZahl = feld[i];
    if (aktuelleZahl < min) {
      min = aktuelleZahl;
    }
  }
  return min;
}
\end{minted}

%%
%
%%

\item Wörter sortieren

\begin{minted}{java}
public void sortiereWoerter() {
  String[] namen = { "Meier", "Bleisteim", "Adam", "Müller", "Zeitlhöfler", "Celtic" };
  boolean aenderung = false;
  do {
    aenderung = false;
    for (int i = 0; i < namen.length - 1; i++) {
      String name1 = namen[i];
      String name2 = namen[i + 1];
      if (name1.compareTo(name2) > 0) {
        namen[i] = name2;
        namen[i + 1] = name1;
        aenderung = true;
      }
    }
  } while (aenderung);

  for (int i = 0; i < namen.length; i++) {
    System.out.println(namen[i]);
  }
}
\end{minted}

%-----------------------------------------------------------------------
%
%-----------------------------------------------------------------------

\section{Temperaturmessung\footcite[Seite 61, Klett, Informatik 3, S. 68]{oomup:fs:3}}

Steffi will ein Jahr lang jeden Tag um 15 Uhr die Temperatur auf ihrem
Balkon messen und die Ergebnisse auswerten. Dazu definiert sie eine
Klasse \java{Tempmessung}.

\begin{enumerate}

%%
% a)
%%

\item Lege ein Feld \java{temperatur} an, welches die reellen Werte für
jeden Tag eines Jahres aufnehmen kann. Definiere eine Methode, um das
Feld mit zufälligen Temperaturwerten zu belegen.

%%
% b)
%%

\item Nach genau einem Jahr sollen mithilfe dreier Methoden der Tag mit
dem höchsten Temperaturwert, die niedrigste gemessene Temperatur und der
Durchschnittswert aller Messwerte bestimmt werden. Implementiere
geeignete Methoden.

\begin{antwort}
\inputcode{aufgaben/oomup/pu_3/Tempmessung}
\end{antwort}

\end{enumerate}

\end{itemize}

\literatur

\end{document}
