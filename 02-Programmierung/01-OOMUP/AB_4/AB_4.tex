\documentclass{lehramt-informatik}
\InformatikPakete{syntax}

\begin{document}

\chapter{Aufgabenblatt 4: Algorithmen implementieren II}

Alle Aufgaben auf diesem Blatt sind mit der Entwicklungsumgebung BlueJ
zu bearbeiten.

%-----------------------------------------------------------------------
%
%-----------------------------------------------------------------------

\section{Aufgabe 1\footcite{oomup:ab:4}}

Eine gute Möglichkeit, um erstes Arbeiten mit Methoden und
Kontrollstrukturen zu üben, stellen Turtlegrafiken dar, die sowohl in
der bereits bekannten Programmierumgebung Greenfoot als auch in BlueJ
genutzt werden können. Bevor Sie mit dem Programmieren starten können,
gehen Sie bitte auf \url{http://www.java-online.ch/lego/turtleGrafik}
und laden sich die Klassenbibliothek \verb|aplu5.jar| herunter.
Integrieren Sie diese in BlueJ folgendermaßen:

\begin{itemize}
\item Starten Sie BlueJ.

\item Im Menü oben gehen Sie auf Werkzeuge (Windows / Linux) bzw. auf
BlueJ (OS) $\rightarrow$ Einstellungen $\rightarrow$ Bibliotheken
$\rightarrow$ Hinzufügen aplu5.jar auswählen $\rightarrow$ ok
$\rightarrow$ ok

\item Reset der Virtuellen Maschine mit Rechtsklick auf den Balken unten
rechts oder über Menüpunkt.

\end{itemize}

Lesen Sie sich auf der oben verlinkten Website in die Anwendung der
Turtlegrafik ein und bearbeiten Sie dann die folgenden Teilaufgaben:

\renewcommand{\labelenumi}{(\alph{enumi})}
\begin{enumerate}

%%
% (a)
%%

\item Implementieren Sie unter Verwendung der Wiederholung mit fester
Anzahl die Methode \verb|sechseckZeichnen()|.

\begin{antwort}
\inputcode[firstline=13,lastline=18]{aufgaben/oomup/ab_4/turtle/Blume}
\end{antwort}

%%
% (b)
%%

\item Ergänzen Sie Ihr Projekt um eine Methode \verb|blumeZeichnen()|,
die mit Hilfe der Methode \verb|sechseckZeichnen()| folgende Zeichnung
erstellt:

\begin{antwort}
\inputcode[firstline=20,lastline=28]{aufgaben/oomup/ab_4/turtle/Blume}
\end{antwort}

%%
% (c)
%%

\item Zeichnen Sie eine Europa-Flagge, indem Sie zuerst eine Methode
\verb|star()|, die einen Stern zeichnet, schreiben. Die blaue
Hintergrundfarbe erhält man mit der Methode \verb|clear(Color.blue)|.

\begin{antwort}
\inputcode[firstline=6]{aufgaben/oomup/ab_4/turtle/EuropaFlagge}
\end{antwort}

Die vorangegangenen Aufgabenstellungen und die verwendete Bibliothek
sind dem Material des Forschungsprojekts PHBern mit freundlicher
Genehmigung durch Frau Jarka Arnold entnommen.

\end{enumerate}

%-----------------------------------------------------------------------
%
%-----------------------------------------------------------------------

\section{Aufgabe 2\footcite{oomup:ab:4}}

Modellierung und Implementierung des Spiels \emph{„No Risk No Money“}

\subsection{Spielidee:}

\begin{itemize}
\item Der Spieler kann so oft würfeln, wie er will.

\item Mit jedem Wurf erhöht sich sein Gewinn um eins, sofern die
gewürfelte Zahl nicht vorher schon einmal gewürfelt wurde.

\item Wenn man eine Zahl das zweite Mal würfelt, ist der Gewinn weg.
\end{itemize}

\subsection{Ablauf:}

Der Spieler würfelt einmal. Hat er eine Zahl gewürfelt, die noch nicht
gewürftelt wurde, erhöht sich sein Gewinn um eins. Dies wiederholt der
Spieler so oft er möchte. Er kann sich jederzeit ausgeben lasse, welche
Zahlen er bereits gewürfelt hat.

\subsection{Spielende:}

Der Spieler kann entscheiden wie oft er spielen möchte. Hört er auf,
obwohl er noch spielen könnte, dann darf er den Gewinn behalten. Würfelt
er eine Zahl das zweite mal, dann ist das Spiel beendet und der Spieler
erhält keinen Gewinn.

\renewcommand{\labelenumi}{(\alph{enumi})}
\begin{enumerate}

%%
% (a)
%%

\item Begründen Sie, weshalb sich zur Verwaltung der bereits gewürfelten
Zahlen ein Feld vom Typ \verb|boolean| eignet.

%%
% (b)
%%

\item Erstellen Sie eine zur obigen Spielbeschreibung passende
Klassenkarte.

%%
% (c)
%%

\item Implementieren Sie das Spiel wie oben beschrieben und unter
Beachtung der Modellierung aus Teilaufgabe (b). Dabei können Sie
zunächst davon ausgehen, dass immer mit einem normalen
\emph{„sechsseitigen“} Würfel gewürfelt wird. Achten Sie darauf, dass
dem Spieler immer passende Ausgaben zum aktuellen Spielstand angezeigt
werden.

%%
% (d)
%%

\item Beschreiben Sie in kurzen Sätzen, welche Fälle beim Testen des
Spieles beachtet werden müssen, und testen Sie ihr Spiel danach
ausführlich.

%%
% (e)
%%

\item Ändern Sie die Implementierung des Spieles so ab, dass der Spieler
beim Start selbst entscheiden kann, mit welcher Art Würfel er spielen
möchte. Dabei muss er mindestens einen vierseitigen Würfel wählen.
\end{enumerate}

\begin{antwort}
\inputcode[firstline=3]{aufgaben/oomup/ab_4/risk/Spiel}
\end{antwort}

Diese Aufgabe entstammt den Materialien von Inf-schule.de. Diese sind
unter Creative Commons BY-SA 4.0 lizenziert. Das ursprüngliche Material
wurde in der vorliegenden Aufgabe adaptiert.

%-----------------------------------------------------------------------
%
%-----------------------------------------------------------------------

\section{Aufgabe 3 (Check-Up)\footcite{oomup:ab:4}}

Das Kino der Kleinstadt Cinehausen möchte die Kartenreservierung ab
sofort digital ermöglichen. Es besteht aus 5 Kinosälen, die jeweils 180
Sitzplätze beinhalten. Jeden Tag gibt es pro Kinosaal nur eine
Vorstellung um 20.15 Uhr. Die gezeigten Filme können in den Sälen
unterschiedlich sein, so dass Besucher bei der Reservierung angeben
müssen, welchen Sitzplatz sie in welchem Saal buchen möchten. Die
Reservierung ist immer nur für den gleichen Tag möglich. Jeden Tag
starten die vergebenen Buchungsnummern wieder bei 1 und werden dann nach
jeder vorgenommenen Reservierung um eins erhöht. Wenn das Ticket vom
Besucher abgeholt wurde, wird die Buchungsnummer des Sitzplatzes
gelöscht.

Weitere Informationen über die gewünschte Modellierung können Sie dem
folgenden Klassendiagramm entnehmen:

Setzen Sie das gegebene Szenario passend in einem BlueJ-Projekt um.
Schreiben Sie außerdem eine Testklasse, die alle geforderten Methoden
ausgiebig testet. Dabei sollen auch Fälle abgefangen werden, bei denen
zum Beispiel bei der Ticketabholung eine ungültige bzw. nicht vorhandene
Buchungsnummer angegeben wird oder ein Sitzplatz reserviert werden soll,
der bereits vergeben oder im Kino nicht vorhanden ist. Achten Sie auch
darauf, dass dem Benutzer passende Textausgaben angezeigt werden.

\begin{antwort}
\inputcode[firstline=3]{aufgaben/oomup/ab_4/kino/Kino}
\inputcode[firstline=3]{aufgaben/oomup/ab_4/kino/Kinosaal}
\inputcode[firstline=3]{aufgaben/oomup/ab_4/kino/Reservierung}
\inputcode[firstline=3]{aufgaben/oomup/ab_4/kino/Sitzplatz}
\end{antwort}

\end{document}
