\documentclass{lehramt-informatik}
\InformatikPakete{syntax,uml}

\begin{document}

%%%%%%%%%%%%%%%%%%%%%%%%%%%%%%%%%%%%%%%%%%%%%%%%%%%%%%%%%%%%%%%%%%%%%%%%
% Theorie-Teil
%%%%%%%%%%%%%%%%%%%%%%%%%%%%%%%%%%%%%%%%%%%%%%%%%%%%%%%%%%%%%%%%%%%%%%%%

\chapter{Objektorientierung}

%%%%%%%%%%%%%%%%%%%%%%%%%%%%%%%%%%%%%%%%%%%%%%%%%%%%%%%%%%%%%%%%%%%%%%%%
% Aufgaben
%%%%%%%%%%%%%%%%%%%%%%%%%%%%%%%%%%%%%%%%%%%%%%%%%%%%%%%%%%%%%%%%%%%%%%%%

\chapter{Aufgaben}

\section{Aufgabe 3 (Objektorientierte Implementierung)\footcite{oomup:pu:4}}

Für die nächste Fußballweltmeisterschaft möchte ein Wettbüro ein
Programm zur Verwaltung von Spielern, Vereinen und
(National-)Mannschaften entwickeln. Dazu wurde bereits das folgende
UML-Klassendiagramm entworfen.
\footcite{examen:46116:2018:03}

\begin{center}
\begin{tikzpicture}
\umlclass{Mannschaft}{
  -land: String\\
  -anzahlSpieler: Integer
}{
  +Mannschaft(land: String)\\
  +hinzufuegen(s: Spieler)\\
  +torschuetzenkoenig():String
}

\umlclass[right=1.5cm of Mannschaft]{Spieler}{
  -name: String
  -tore: Integer
}{
  +Spieler(name: String, verein: Verein)\\
  +getName(): String\\
  +getTore(): Integer\\
  +schiesseTor()\\
  +wechsleVerein(neuerVerein: Verein)
}

\umlclass[below=1cm of Spieler]{Verein}{
  -name: String
}{
  +Verein(name:String)\\
  +getName():String
}

\umluniassoc[mult2=0..20,arg2=-team,pos2=0.6]{Mannschaft}{Spieler}
\umluniassoc[mult2=0..20,arg2=-team]{Spieler}{Verein}
\end{tikzpicture}
\end{center}

\noindent
Es kann angenommen werden, dass die Klasse \java{Verein} bereits
implementiert ist. In den folgenden Implementierungsaufgaben können Sie
eine objektorientierte Programmiersprache Ihrer Wahl verwenden. Die
verwendete Sprache ist anzugeben. Zu beachten sind jeweils die im
Klassendiagramm angegebenen Sichtbarkeiten von Attributen, Rollennamen,
Konstruktorn und Operationen.

\begin{enumerate}

%%
% a)
%%

\item Es ist eine Implementierung der Klasse \java{Spieler} anzugeben.
Der Konstruktor soll die Instanzvariablen mit den gegebenen Parametern
initialisieren, wobei die. Anzahl der Tore nach der Objekterzeugung
gleich 0 sein soll. Ansonsten kann die Funktionalität der einzelnen
Operationen aus deren Namen geschlossen werden.

\begin{antwort}
\inputcode[firstline=3]{aufgaben/oomup/pu_4/verein/Spieler}
\end{antwort}

%%
% b)
%%

\item Es ist eine Implementierung der Klasse \java{Mannschaft}
anzugeben. Der Konstruktor soll das Land initialisieren, die Anzahl der
Spieler auf 0 setzen und das Team mit einem noch „leeren“ Array der
Länge 20 initialisieren. Das Team soll mit der Methode
\java{hinzufuegen} um einen Spieler erweitert werden. Die Methode
\java{torschuetzenkoenig} soll den Namen eines Spielers aus dem Team
zurückgeben, der die meisten Tore für die Mannschaft geschossen hat. Ist
das für mehrere Spieler der Fall, dann kann der Name eines beliebigen
solchen Spielers zurückgegeben werden. Ist noch kein Spieler im Team,
dann soll der String \java{"Kein Spieler vorhanden"} zurückgegeben
werden.

\begin{antwort}
\inputcode[firstline=3]{aufgaben/oomup/pu_4/verein/Mannschaft}
\end{antwort}

%%
% c)
%%

\item Schreiben Sie den Rumpf einer \java{main}-Methode, so dass nach
Ausführung der Methode eine deutsche Mannschaft existiert mit zwei
Spielern Namens \java{"Hugo Meier"} und \java{"Frank Huber"}. Beide
Spieler sollen zum selben Verein \java{"FC Staatsexamen"} gehören.
\java{"Hugo Meier"} soll nach Aufnahme in die deutsche Mannschaft genau
ein Tor geschossen haben, während \java{"Frank Huber"} noch kein Tor
erzielt hat. (Wir abstrahieren hier von der Realität, in der ein 2-er
Team noch gar nicht spielbereit ist.)

\begin{antwort}
\inputcode[firstline=3]{aufgaben/oomup/pu_4/verein/Verein}
\end{antwort}
\end{enumerate}

%-----------------------------------------------------------------------
%
%-----------------------------------------------------------------------

\section{Aufgabe 1: Vererbung und Abstrakte Klassen\footcite[(entnommen
aus Algorithmen und Datenstrukturen, 3. \& 4. Übungsblatt, Universität
Bayreuth)]{aud:pu:7}}

\def\TmpHinweis#1{{\footnotesize[#1]}}

In einer Anforderungsanalyse für ein
Banksystem wird der folgende Sachverhalt beschrieben:
\footcite[Objektorientierte Modellierung (aus Stex vertieft
Automatentheorie, Komplexität, Algorithmen, Herbst 02, Thema Nr. 1,
Aufgabe 4)]{examen:66112:2002:09}

Eine Bank hat einen Namen und sie führt Konten. Jedes Konto hat eine
Kontonummer, einen Kontostand und einen Besitzer. Der Besitzer hat einen
Namen und eine Kundennummer. Ein Konto ist entweder ein Sparkonto oder
ein Girokonto. Ein Sparkonto hat einen Zinssatz, ein Girokonto hat einen
Kreditrahmen und eine Jahresgebühr.

\begin{enumerate}

%%
% (a)
%%

\item Deklarieren Sie geeignete Klassen in Java [oder in C++], die die
oben beschriebenen Anforderungen widerspiegeln! Nutzen Sie dabei das
Vererbungskonzept aus, wo es sinnvoll ist! Gibt es Klassen, die als
abstrakt zu verstehen sind?

\TmpHinweis{Hinweis: Geben Sie sowohl ein Klassendiagramm als auch den
Quellcode für die beteiligten Klassen inkl. Attributen, ohne Methoden
an! Achtung: in Teilaufgabe b) werden anschließend die benötigten
Konstruktoren verlangt; Um die verschiedenen Konten in einer Bank zu
verwalten, eignet sich das Array NICHT als Datenstruktur. Informieren
Sie sich über die Datenstruktur „ArrayList“ und verwenden Sie diese.}

\begin{antwort}
Die Klasse Konto ist abstrakt, da ein Konto immer entweder ein Spar-
\emph{oder} ein Girokonto
ist. Ein Objekt der Klasse Konto ist deshalb nicht sinnvoll.
\end{antwort}

%%
% (b)
%%

\item Geben Sie für alle nicht abstrakten Klassen benutzerdefinierte
Konstruktoren an mit Parametern zur Initialisierung der folgenden Werte:
der Name einer Bank, die Kontonummer, der Kontostand, der Besitzer und
der Zinssatz (bzw. Kreditrahmen und Jahresgebühr) eines Sparkontos (bzw.
Girokontos), der Name und die Kundennummer eines Kontobesitzers.

Ergänzen Sie die Klassen um Methoden für die folgenden Aufgaben! Nutzen
Sie wann immer möglich das Vererbungskonzept aus und verwenden Sie ggf.
abstrakte (bzw. virtuelle) Methoden!

\TmpHinweis{Achtung: Ergänzen Sie ggf. alle benötigten Getter und Setter
in den beteiligten Klassen!}

\begin{antwort}
Konstruktoren ergänzen:

Bemerkung: Auch eine abstrakte Klasse kann einen Konstruktor besitzen,
dieser kann nur nicht ausgeführt werden. In den abgeleiteten Klassen
kann dieser Super-Konstruktor aber verwendet werden.
\end{antwort}

%%
% (c)
%%

\item Auf ein Konto soll ein Betrag eingezahlt und ein Betrag abgehoben
werden können. Soll von einem Sparkonto ein Betrag abgehoben werden,
dann darf der Kontostand nicht negativ werden. Bei einer Abhebung von
einem Girokonto darf der Kreditrahmen nicht überzogen werden.

\begin{antwort}
Bemerkung: Die Methode \java{einzahlen} ist in der Klasse \java{Konto}
implementiert, da sie sich für Spar- und Girokonten nicht unterscheidet
im Gegensatz zur Methode \java{abheben}, die in beiden Klassen
unterschiedlich implementiert ist. \TmpHinweis{Sie wird als abstrakte
Methode in der Klasse Konto angegeben, um ihre Implementierung
(Überschreiben) zu gewährleisten.} Kreditrahmen wird als negativer Wert
gespeichert. Die Methoden zum Abheben liefern zusätzlich zur Änderung
des Kontostandes eine Rückmeldung bezüglich Erfolg oder Misserfolg der
Abbuchung.
\end{antwort}

%%
% (d)
%%

\item Ein Konto kann eine Jahresabrechnung durchführen. Bei der
Jahresabrechnung eines Sparkontos wird der Zinsertrag gutgeschrieben,
bei der Jahresabrechnung eines Girokontos wird die Jahresgebühr
abgezogen (auch wenn dadurch der Kreditrahmen überzogen wird).

\begin{antwort}
Anmerkung: Im Folgenden nur noch Angabe der gesuchten Methoden, alle
vorherigen Implementierungen (Attribute, Konstruktoren, Methoden, s. o.)
wurden nicht nochmals aufgeführt.
\end{antwort}

%%
% (e)
%%

\item Eine Bank kann einen Jahresabschluss durchführen. Dieser bewirkt,
dass für jedes Konto der Bank eine Jahresabrechnung durchgeführt wird.

%%
% (f)
%%

\item Eine Bank kann ein Sparkonto eröffnen. Die Methode soll die
folgenden fünf Parameter haben: den Namen und die Kundennummer des
Kontobesitzers, die Kontonummer, den (anfänglichen) Kontostand und den
Zinssatz des Sparkontos. Alle Parameter sind als String-Objekte oder als
Werte eines Grunddatentyps zu übergeben! Das Sparkonto muss nach seiner
Eröffnung in den Kontenbestand der Bank aufgenommen sein.

\TmpHinweis{Hinweis: Falls der Kunde schon mit einem Konto in der Bank
geführt ist, können die Werte für das \java{Besitzer}-Objekt übernommen
werden. Schreiben Sie daher eine Hilfsmethode \java{Besitzer
schonVorhanden(String name, int kunr)}, die prüft, ob der Kunde mit
\java{name} und \java{kunr} schon in der Liste vorhanden ist und diesen
bzw. andernfalls einen neu erzeugten Besitzer zurückgibt.}

\TmpHinweis{Sinnvolle/notwendige Methoden der Klasse ArrayList für diese
Aufgabe:
\\
\java{public int size()}: Returns the number of elements in this
list.
\\
\java{public boolean isEmpty()}: Returns true if this list contains no
elements.
\\
\java{public E get(int index)}: Returns the element at the specified
position in this list.
\\
\java{public boolean add(E e)}: Appends the specified element to the end of this list.
\\
(siehe auch Java-Api: https://docs.oracle.com/javase/7/docs/api/java/util/ArrayList.html)}

\begin{antwort}
\inputcode[firstline=3]{aufgaben/oomup/examen_66112_2002_09/Bank}
\inputcode[firstline=3]{aufgaben/oomup/examen_66112_2002_09/Konto}
\inputcode[firstline=3]{aufgaben/oomup/examen_66112_2002_09/Sparkonto}
\inputcode[firstline=3]{aufgaben/oomup/examen_66112_2002_09/Girokonto}
\inputcode[firstline=3]{aufgaben/oomup/examen_66112_2002_09/Besitzer}
\end{antwort}

%-----------------------------------------------------------------------
%
%-----------------------------------------------------------------------

\section{Aufgabe 2: Vererbung und Abstrakte Klassen II
\footcite[entnommen aus Staatexamen DB/ST 66116 , Frühjahr 2015, TA2, A3 a]{aud:pu:7}}

In dieser Aufgabe implementieren Sie ein konzeptionelles Datenmodell für
eine Firma, die Personendaten von Angestellten und Kunden verwalten
möchte. Gegeben seien dazu folgende Aussagen:
\footcite[Seite 5]{examen:66116:2015:03}

\begin{itemize}
\item Eine Person hat einen Namen und ein Geschlecht (männlich oder
weiblich).

\item Ein Angestellter ist eine Person, zu der zusätzlich das monatliche
Gehalt gespeichert wird.

\item Ein Kunde ist eine Person, zu der zusätzlich eine Kundennummer
hinterlegt wird.
\end{itemize}

\begin{enumerate}

%%
% (a)
%%

\item Geben Sie in einer objektorientierten Programmiersprache Ihrer
Wahl (geben Sie diese an) eine Implementierung des aus den obigen
Aussagen resultierenden konzeptionellen Datenmodells in Form von Klassen
und Interfaces an. Gehen Sie dabei wie folgt vor:

\begin{itemize}
\item Schreiben Sie ein Interface \java{Person} sowie zwei davon erbende
Interfaces \java{Angestellter} und \java{Kunde}. Die Interfaces sollen
jeweils lesende Zugriffsmethoden (Getter) die entsprechenden Attribute
(Name, Geschlecht, Gehalt, Kundennummer) deklarieren.

\begin{antwort}
\inputcode[firstline=3]{aufgaben/oomup/examen_66116_2015_03/Person}
\inputcode[firstline=3]{aufgaben/oomup/examen_66116_2015_03/Angestellter}
\inputcode[firstline=3]{aufgaben/oomup/examen_66116_2015_03/Kunde}
\end{antwort}

\item Schreiben Sie eine abstrakte Klasse \java{PersonImpl}, die das
Interface \java{Person} implementiert. Für jedes Attribut soll ein
Objektfeld angelegt werden. Außerdem soll ein Konstruktor definiert
werden, der alle Objektfelder initialisiert.

\begin{antwort}
\inputcode[firstline=3]{aufgaben/oomup/examen_66116_2015_03/PersonImpl}
\end{antwort}

\item Schreiben Sie zwei Klassen \java{AngestellterImpl} und
\java{KundeImpl}, die von \java{PersonImpl} erben und die jeweils
dazugehörigen Interfaces implementieren. Es sollen wiederum
Konstruktoren definiert werden, die alle Objektfelder initialisieren und
dabei auf den Konstruktor der Basisklasse \java{PersonImpl} Bezug
nehmen.

\begin{antwort}
\inputcode[firstline=3]{aufgaben/oomup/examen_66116_2015_03/AngestellterImpl}
\inputcode[firstline=3]{aufgaben/oomup/examen_66116_2015_03/KundeImpl}
\end{antwort}
\end{itemize}
\end{enumerate}
\end{enumerate}

\literatur

\end{document}
