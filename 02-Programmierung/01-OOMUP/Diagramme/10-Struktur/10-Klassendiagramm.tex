\documentclass{lehramt-informatik-haupt}
\InformatikPakete{uml,syntax}

\begin{document}

%%%%%%%%%%%%%%%%%%%%%%%%%%%%%%%%%%%%%%%%%%%%%%%%%%%%%%%%%%%%%%%%%%%%%%%%
% Theorie-Teil
%%%%%%%%%%%%%%%%%%%%%%%%%%%%%%%%%%%%%%%%%%%%%%%%%%%%%%%%%%%%%%%%%%%%%%%%

\chapter{Klassendiagramm}

\begin{quellen}
\item \cite[Seite 108-169]{rupp}
\item \cite[Zusammengefügtes PDF Seite 30-32 / Kapitel „Klassen“ 8-10]{brinda}
\item \cite{wiki:klassendiagramm}
\end{quellen}

Klassendiagramme (class diagrams) beschreiben die Entitäten eines
Systems und welche Beziehungen sie untereinander eingehen können
(Struktur der Daten). Neben Paketdiagrammen werden Klassendiagramme bzw.
deren Notation wahrscheinlich am häufigsten eingesetzt.
\footcite[Seite 166]{schatten}

\begin{description}

%%
%
%%

\item[Klasse]
Klassen werden durch Rechtecke dargestellt, die entweder nur den Namen
der Klasse (fett gedruckt) tragen oder zusätzlich auch Attribute und
Methoden spezifiziert haben.
Dabei werden diese drei Rubriken (engl. compartment) – Klassenname,
Attribute, Methoden – jeweils durch eine horizontale
Linie getrennt.
\footcite{wiki:klassendiagramm}
\footcite[Seite 49-50 (PDF 65-66)]{uml}

\begin{tikzpicture}
\umlclass{Lehrer}{Attribute}{Methoden}
\umlsimpleclass[x=3]{Lehrer}
\umlclass[x=8]{Lehrer}{
- alter: int\\
+ name: String\\
+ guteLaune: Boolean = false
}{
+ frageAb()
}
\end{tikzpicture}

%%
%
%%

\item[Abstrakte Klassen]
Kursiv oder mit Untertitle \{abstract\} oder mit <<>>

\begin{tikzpicture}
\umlsimpleclass[type=abstract,x=0]{Lehrer}
\umlsimpleclass[tags=abstract,x=3]{Lehrer}
\umlsimpleclass[type=abstrakt,x=6]{Lehrer}
\end{tikzpicture}

%%
%
%%

\item[Sichtbarkeit]
$-$: private, $+$: public, $\#$: protected, \~{}: package
\footcite[Seite 141-142 (PDF 158-159)]{uml}

\begin{tikzpicture}
\umlclass{Lehrer}{
  - geburtstag \\
  + name \\
  \# spitzname \\
  \~{} hobby \\
}{}
\end{tikzpicture}

\item[Klassenattribute]
Klassenattribute (statische Attribute) werden unterstrichen
\footcite[Seite 118,121]{rupp}

\begin{tikzpicture}
\umlclass{Lehrer}{
  \umlstatic{- gehalt}
}{}
\end{tikzpicture}

%%
%
%%

\item[Multiplizität]
z. B. 0..1, 1..1, 1, 0..*, *, n..m
\footcite[Seite 98 (PDF 115)]{uml}

%%
%
%%

\item[Methoden/Operationsdeklaration]
z. B. \verb|-div(divident : Int, divisor : Int) : double|

%%
%
%%

\item[Assoziation] Eine Assoziation beschreibt eine Beziehung zwischen
zwei oder mehr Klassen. An den Enden von Assoziationen sind häufig
Multiplizitäten vermerkt.

\begin{tikzpicture}
\umlsimpleclass{Fach}
\umlsimpleclass[x=4]{Lehrer}
\umlassoc[mult1=1..*,mult2=1]{Fach}{Lehrer}
\end{tikzpicture}

%%
%
%%

\item[Generalisierung (Inheritance)] Eine Generalisierung in der UML ist
eine \emph{gerichtete} Beziehung zwischen einer \emph{generelleren} und
einer \emph{spezielleren} Klasse. Der Pfeil zeigt von der spezielleren
Klasse zur generellen Klasse.
\footcite[Kapitel 6.4.6 Generalisierung, Seite 135]{rupp}

\begin{tikzpicture}
\umlsimpleclass[y=1.4,tags=generelle Klasse]{Mensch}
\umlsimpleclass[tags=spezielle Klasse]{Lehrer}
\umlinherit{Lehrer}{Mensch}
\end{tikzpicture}

\item[Realisierung (Realization)] Gestrichelte Linie mit nicht
ausgefüllten Dreieck als Pfeilspitze. \latex{\umlreal{A}{B}}
\footcite[Kapitel 6.4.13, Seite 164]{rupp}

\begin{tikzpicture}
\umlsimpleclass[]{Implementierung}
\umlsimpleclass[x=4cm]{Spezifikation}
\umlreal{Implementierung}{Spezifikation}
\end{tikzpicture}

%%
%
%%

\item[Aggregation (Aggregation)]
Teil-Ganzes-Beziehung, nicht-ausgefüllte Raute am „Ganzen“-Objekt,
in Java keine entsprechung

\begin{tikzpicture}
\umlsimpleclass{Unterrichtsstunde}
\umlsimpleclass[x=6]{Schüler}
\umlaggreg[mult1=20..*,mult2=12..*]{Unterrichtsstunde}{Schüler}
\end{tikzpicture}

%%
%
%%

\item[Komposition (Composition)]
starke Aggregation, ausgefüllte Raute, Teilobjekte existenzabhängig von
Aggregationsobjekt: z. B. Bank, Konto

\begin{tikzpicture}
\umlsimpleclass{Schule}
\umlsimpleclass[x=4]{Klassenzimmer}
\umlcompo[mult1=1,mult2=1..*]{Schule}{Klassenzimmer}
\end{tikzpicture}

\item[Assoziationsklasse]
beschreibt Assoziation zwischen anderen Klassen beschreibt, wird
aufgelöst: entweder Verteilung der Attribute und Methoden oder
eigenständige Klasse. Als „Kochrezept“: „Was bei den Multiplizitäten
vorher bei den einzelnen Klassen steht, hüpft jeweils auf die andere
Seite der aufgelösten Assoziationsklasse.“

\item[Abhängigkeitsbeziehung (Dependency)]

Gestrichelte Line mit offenen Pfeil. Der Pfeil zeigt \emph{vom
abhängigen auf das unabhängige} Modellelement. \latex{\umldep{A}{B}}
\footcite[Kapitel 6.4.10 Abhängigkeitsbeziehung, Seite 159]{rupp}

\begin{tikzpicture}
\umlsimpleclass[tags=abhängig]{Schüler}
\umlsimpleclass[x=4,tags=unabhängig]{Lehrer}
\umldep{Schüler}{Lehrer}
\end{tikzpicture}
\end{description}

%-----------------------------------------------------------------------
%
%-----------------------------------------------------------------------

\section{Arten von Assoziationen}

\begin{description}
\item[Ungerichtete Assoziation]
Linie mit Beschriftung

\begin{tikzpicture}
\umlsimpleclass{A}
\umlsimpleclass[x=4]{B}
\umlassoc{A}{B}
\end{tikzpicture}

Assoziation mit sich selbst.

\begin{tikzpicture}
\umlsimpleclass{A}
\umlassoc[angle1=-90,angle2=0,loopsize=3cm,arg1=next,arg2=previous,mult1=1,mult2=1]{A}{A}
\end{tikzpicture}

\item[Gerichtete Assoziation]
Linie mit Beschriftung und Leserichtungspfeil

\begin{tikzpicture}
\umlsimpleclass{A}
\umlsimpleclass[x=4]{B}
\umluniassoc{A}{B}
\end{tikzpicture}

\item[Reflexive Assoziation]
Pfeil mit Beschriftung

%-----------------------------------------------------------------------
%
%-----------------------------------------------------------------------

\item[Navigierbare Assoziation]

Die Navigierbarkeit von Assoziationen wird durch eine Pfeilspitze am
Ende einer Assoziation ausgedrückt. Die Pfeilrichtung zeigt an, dass die
Instanzen der Klasse A die Instanzen der Klasse B in Pfeilrichtung
„kennen“.
\footcite[Seite 150]{rupp}

%%
%
%%

\begin{description}
\item[Unspezifizierte Navigierbarkeit] \latex{\umlassoc{A}{B}}

\begin{tikzpicture}
\umlsimpleclass{A}
\umlsimpleclass[x=4]{B}
\umlassoc{A}{B}
\end{tikzpicture}

%%
%
%%

\item[Unidirektionale Navigierbarkeit] \latex{\umluniassoc{A}{B}}

\begin{tikzpicture}
\umlsimpleclass{A}
\umlsimpleclass[x=4]{B}
\umluniassoc{A}{B}
\end{tikzpicture}
\end{description}
\end{description}

%-----------------------------------------------------------------------
%
%-----------------------------------------------------------------------

\section{Übungen}

\begin{enumerate}

%-----------------------------------------------------------------------
%
%-----------------------------------------------------------------------

\item Modellieren Sie die folgende Situation als Klassendiagramm:
\cite[Seite 1, Aufgae 2]{net:html:tu-dortmund:uebung-softwaretechnik}

Fußballmannschaften einer Liga bestreiten während einer
Meisterschaftsrunde Spiele gegen andere Mannschaften. Dabei werden in
jeder Mannschaft Spieler für einen bestimmten Zeitraum (in Minuten)
eingesetzt, die dabei eventuell Tore schießen.

Die Modellierung soll es ermöglichen, festzustellen, welcher Spieler in
welchem Spiel wie lange auf dem Feld war und wie viele Tore geschossen
hat. Ebenso soll es möglich sein, für jede Mannschaft festzustellen,
gegen welche Mannschaft welche Ergebnisse erzielt wurden.

\begin{center}
\begin{tikzpicture}
\umlsimpleclass[y=5]{Liga}
\umlsimpleclass[y=3]{Mannschaft}
\umlsimpleclass[y=1]{Spieler}
\umlsimpleclass[x=5,y=3]{Spiel}

\umlassoc[mult2=*]{Liga}{Mannschaft}
\umlassoc[mult1=1,mult2=*]{Mannschaft}{Spieler}
\umlassoc[mult1=2,mult2=*,name=mannschaftspiel]{Mannschaft}{Spiel}
\umlHVassoc[mult1=*,mult2=*,pos2=1.7,name=spielerspiel]{Spieler}{Spiel}
\node[below=0cm of spielerspiel-2,font=\scriptsize] {nimmt teil};

\umlassocclass[x=2.5,y=5]{Saison}{mannschaftspiel-1}{- spieljahr}{}
\umlassocclass[x=8,y=1]{Teilnahme}{spielerspiel-2}{- minuten: int\\- tore: int}{}
\end{tikzpicture}
\end{center}

%-----------------------------------------------------------------------
%
%-----------------------------------------------------------------------

\item Gegeben ist der folgende Sachverhalt.
\cite[Seite 4]{net:pdf:uzh-zuerich:uebung-4}

Jede \textbf{Person} hat einen \emph{Namen}, eine \emph{Telefonnummer}
und \emph{E-Mail}.

Jede \textbf{Wohnadresse} wird von nur einer Person bewohnt. Es kann
aber sein, dass einige Wohnadressen nichtbewohnt sind. Den Wohnadressen
sind je eine \emph{Strasse}, eine \emph{Stadt}, eine \emph{PLZ} und ein
\emph{Land} zugeteilt. Alle Wohnadressen können bestätigt werden und als
Beschriftung (für Postversand) gedruckt werden.

Es gibt zwei Sorten von \textbf{Personen}: \textbf{Student}, welcher
sich für ein \emph{Modul einschreiben} kann und \textbf{Professor},
welcher einen \emph{Lohn} hat. Der Student besitzt eine
\emph{Matrikelnummer} und eine \emph{Durchschnittsnote}.

Modellieren Sie diesen Sachverhalt mit einem UML Klassendiagramm.

\begin{tikzpicture}
\umlclass[x=2,y=4]{Person}{
  + name: String\\
  + telefonNummer: String\\
  + eMail: String
}{}
\umlclass[]{Professor}{
  + lohn: double
}{}
\umlclass[x=5]{Student}{
  + matrikelNummer: int\\
  + durchschnittsNote: double
}{
  + einschreibenFuerModule()
}
\umlclass[x=10,y=4]{Wohnadresse}{
  + strasse: String \\
  + stadt: String \\
  + plz: int \\
  + land: String \\
}{
  + bestaetigen()\\
  + drucken()
}

\umlVHVinherit{Professor}{Person}
\umlVHVinherit{Student}{Person}
\umlassoc[arg1=adresse,arg2=person,stereo=wohnt\LeserichtungRechts,mult1=0..1,mult2=1,pos1=0.15,pos2=0.85]{Person}{Wohnadresse}
\end{tikzpicture}

%-----------------------------------------------------------------------
%
%-----------------------------------------------------------------------

% https://upload.wikimedia.org/wikiversity/de/b/b9/Klassendiagramme.pdf

\item Essen in Gasthausen
\cite[Seite 4-5]{net:pdf:wikiversity:klassendiagramm}

In diesem kleinen Städtchen gibt es fünf Restaurants, aber egal, ob es
sich um ein 2*-Restaurant, oder um ein 5*-Restaurant handelt, alle haben
Gemeinsamkeiten. In jedem Restaurant sind mehrere \textbf{Mitarbeiter}
angestellt. Im kleinsten Restaurant nur zwei, aber im größten verdienen
16 Mitarbeiter ihr Geld. Die Mitarbeiter bekommen in jedem Restaurant
ein anderes \emph{Gehalt}, aber alle machen ihre Arbeit. Die
\textbf{Kellner}
\emph{nehmen bestellungen entgegen}, \emph{servieren das Essen} und \emph{kassieren das Geld}
von den Gästen. Jeder Kellner gibt die Bestellungen der Gäste an die
Köche weiter. Die \textbf{Köche} kennen verschiedene \textbf{Rezepte},
nach denen sie die \textbf{Menüs} dann zubereiten. Jedes Menü hat einen
anderen \emph{Preis}, aber es besteht nach Tradition immer aus einem
\textbf{Salat} und einer \textbf{Hauptspeise}. An manchen Tagen ist kein
\textbf{Gast} da. Dann werden Sie in ganz familiärem Rahmen bewirtet.
Aber auch an guten Tagen gibt es, bei 100 Plätzen im größten Restaurant,
sicher für jeden hungrigen Gast einen \textbf{Platz}. Wer in Gasthausen
Essen geht sollte hungrig sein! Natürlich hat jeder Gast unterschiedlich
viel Geld zur Verfügung, aber bei entsprechender Wahl des Restaurants
reicht es sicher für sein Lieblingsessen.

\begin{tikzpicture}[scale=0.8, transform shape]
\umlclass[x=0,y=5]{Restaurant}{- anzahlSterne: int}{}

\umlclass[x=8,y=5]{Mitarbeiter}{
- gehalt: double
}{}

\umlclass
[below left=0.5cm and -1cm of Mitarbeiter]
{Kellner}{}
{
  + nehmeBestellungenAuf()\\
  + serviereEssen()\\
  + kassiereGeld()
}

\umlclass[below right=0.5cm and -1cm of Mitarbeiter]{Koch}{}{}

\umlclass[below left=0.5cm and 0cm of Restaurant]{Sitzplatz}{}{}

\umlclass[x=8cm,y=0cm]{Menü}{- preis: float}{}
\umlsimpleclass[below left=0.5cm of Menü]{Salat}
\umlsimpleclass[below right=0.5cm of Menü]{Hauptspeise}
\umlcompo[mult1=1,mult2=1]{Menü}{Salat}
\umlcompo[mult1=1,mult2=1]{Menü}{Hauptspeise}

\umlclass{Gast}{
  hungrig: boolean = true\\
  lieblingsEssen: String\\
  geld: double
}{}

\umlassoc[mult1=2..16,mult2=5,stereo=beschäftigt]{Mitarbeiter}{Restaurant}

\umlassoc[mult1=0..100,mult2=0..5,stereo=besucht]{Gast}{Restaurant}

\umlassoc[stereo=bedient]{Kellner}{Gast}

\umlassoc[stereo=bestellt]{Gast}{Menü}

\umlassoc[stereo=Bestellung weiter geben]{Kellner}{Koch}

\end{tikzpicture}

\end{enumerate}

%-----------------------------------------------------------------------
%
%-----------------------------------------------------------------------

\ExamensAufgabeTTA 46116 / 2012 / 03 : Thema 2 Teilaufgabe 2 Aufgabe 1

\literatur

\end{document}
