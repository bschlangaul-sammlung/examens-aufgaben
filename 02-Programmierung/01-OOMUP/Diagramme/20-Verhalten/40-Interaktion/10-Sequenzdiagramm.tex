\documentclass{lehramt-informatik}
\InformatikPakete{uml,syntax}

\begin{document}

%%%%%%%%%%%%%%%%%%%%%%%%%%%%%%%%%%%%%%%%%%%%%%%%%%%%%%%%%%%%%%%%%%%%%%%%
% Theorie-Teil
%%%%%%%%%%%%%%%%%%%%%%%%%%%%%%%%%%%%%%%%%%%%%%%%%%%%%%%%%%%%%%%%%%%%%%%%

\chapter{Sequenzdiagramm}

\begin{quellen}
\cite[Seite 401-471]{rupp}
\cite[Zusammengefügtes PDF Seite 115-117 / Kapitel „Assoziationen“ 33-35]{brinda}
\end{quellen}

Sequenzdiagramme (sequence diagrams) stellen eine exemplarische Abfolge
von Nachrichten zwischen Objekten dar. Der Fokus liegt dabei auf der
zeitlichen Ordnung der Nachrichten. Sequenzdiagramme sind meist sehr
implementationsspezifisch, das heißt, sie sind ohne näheres Wissen
über den Sourcecode des Systems nur schwer nachvollziehbar.
\footcite[Seite 167]{schatten}

\begin{description}
\item[Kommunikationspartner als Lebenslinien]

Nachrichten in Form von Pfeile

\item[synchroner Operationsaufruf]
Pfeil mit ausgefülltem Dreieck, gestrichelter Pfeil zurück
\footcite[Seite 507 (PDF 523): \emph{Synchronous Messages typically
represent operation calls and are shown with a filled arrow head.}]{uml}

\begin{tikzpicture}
\begin{umlseqdiag}
\umlobject[class=A]{a}
\umlobject[class=B]{b}
\begin{umlcall}[op={call()},
return=return]{a}{b}
\end{umlcall}
\end{umlseqdiag}
\end{tikzpicture}

\item[asynchroner Signal-Operationsaufruf]
Normaler offener Pfeil
%
\footcite[Seite 507 (PDF 523): \emph{Asynchronous Messages have an open
arrow head.}]{uml}

\begin{tikzpicture}
\begin{umlseqdiag}
\umlobject[class=A]{a}
\umlobject[class=B]{b}
\begin{umlcall}[type=asynchron]{a}{b}
\end{umlcall}
\end{umlseqdiag}
\end{tikzpicture}

\item[Antwortnachricht]
gestrichelter Pfeil
%
\footcite[Seite 507 (PDF 523): \emph{The reply message from a method has a dashed
line.}]{uml}

\item[Erzeugunsaufruf]
gestrichelter Pfeil erzeugt neues Objekt
%
\footnote{\href{https://stackoverflow.com/a/2128192} The UML 2.2
specifications (superstructure) has an example on page 474, Figure 14.11
is their canonical syntax/notation reference. And on page 495 (UML 2.2)
(Seite 577 (PDF 619) UML 2.5.1) in the notation section it states
“Object creation Message has a dashed line with an open arrow.”}

\begin{tikzpicture}
\begin{umlseqdiag}
\umlobject[class=A]{a}
\umlcreatecall[class=B]{a}{b}
\end{umlseqdiag}
\end{tikzpicture}

\item[Sendeergebnis nicht bekannt]
kleiner Kreis am Pfeilanfang

\item[Empfangergebnis nicht bekannt]
kleiner Kreis am Pfeilende

\end{description}

%%%%%%%%%%%%%%%%%%%%%%%%%%%%%%%%%%%%%%%%%%%%%%%%%%%%%%%%%%%%%%%%%%%%%%%%
% Aufgaben
%%%%%%%%%%%%%%%%%%%%%%%%%%%%%%%%%%%%%%%%%%%%%%%%%%%%%%%%%%%%%%%%%%%%%%%%

\chapter{Aufgaben}

\section{Aufgabe 1: (Objektorientierte Programme und Reverse Engineering)
\footcite[Seite 15-16]{examen:66116:2018:03}}

Gegeben sei das folgende Java-Programm:

\begin{minted}{java}
class M {
  private boolean b;
  private F f;
  private A a;

  public void m() {
    f = new F();
    a = new A(f);
    b = true;
  }
}

class A {
  private R r;
  public A(I i) {
    r = i.createX();
  }
}

interface I {
  public X createX();
}

class F implements I {
  public X createX() {
    return new X(0, 0);
  }
}

abstract class R {
  protected int v;
}

class X extends R {
  private int v, w;
  public X(int v, int w) {
    this.v = v;
    this.w = w;
  }
}
\end{minted}

\begin{enumerate}

%%
% a)
%%

\item Das Subtypprinzip der objektorientierten Programmierung wird in
obigem Programmcode zweimal ausgenutzt. Erläutern Sie wo und wie dies
geschieht.

%%
% b)
%%

\item Zeichnen Sie ein UML-Klassendiagramm, das die statische Struktur
des obigen Programms modelliert. Instanzvariablen mit einem Klassentyp
sollen durch gerichtete Assoziationen mit Rollennamen und Multiplizität
am gerichteten Assoziationsende modelliert werden. Alle aus dem
Programmcode ersichtlichen statischen Informationen (insbesondere
Interfaces, abstrakte Klassen, Zugriffsrechte, benutzerdefinierte
Konstruktoren und Methoden) sollen in dem Klassendiagramm abgebildet
werden.

\begin{tikzpicture}
\umlclass[x=1,y=5]{M}{- boolean b}{+ m(): void}
\umlsimpleclass[x=0,y=2.5]{F}
\umlsimpleclass[x=5,y=5]{A}
\umlclass[x=7,y=2.5,type=abstract]{R}{\# int v}{}
\umlclass[x=3,y=0,type=interface]{I}{}{+ createX(): X}
\umlclass[x=8,y=0]{X}{- int v\\- int w}{}

\umluniassoc[mult=1]{A}{R}
\umluniassoc[mult=1]{M}{A}
\umluniassoc[mult=1]{M}{F}
\umluniassoc[mult=1]{I}{X}
\umldep[mult=1,pos=0.9]{A}{I}
\umlinherit{X}{R}
\umlreal{F}{I}
\end{tikzpicture}

%%
% c)
%%

\item Es wird angenommen, dass ein Objekt der Klasse M existiert, für
das die Methode \java{m()} aufgerufen wird. Geben Sie ein
Instanzendiagramm (Objektdiagramm) an, das alle nach der Ausführung der
Methode \java{m} existierenden Objekte und deren Verbindungen (Links)
zeigt.

%%
% d)
%%

\item Wie in Teil c) werde angenommen, dass ein Objekt der Klasse M
existiert, für das die Methode \java{m()} aufgerufen wird. Diese
Situation wird in Abb. 1 dargestellt. Zeichnen Sie ein Sequenzdiagramm,
das Abb. 1 so ergänzt, dass alle auf den Aufruf der Methode \java{m()}
folgenden Objekterzeugungen und Interaktionen gemäß der im Programmcode
angegebenen Konstruktor- und Methodenrümpfe dargestellt werden.
Aktivierungsphasen von Objekten sind durch längliche Rechtecke deutlich
zu machen.

\begin{tikzpicture}
\begin{umlseqdiag}
\umlobject[class=M]{m}
%,create text={new F()}
\umlcreatecall[class=F]{m}{f}
%,create text={new A(f)
\umlcreatecall[class=A]{m}{a}
\begin{umlcall}[op={createX()},return={:X}]{a}{f}
% ,create text={new X(0,0)
\umlcreatecall[class=X]{f}{x}
\end{umlcall}
\end{umlseqdiag}
\end{tikzpicture}
\end{enumerate}

%-----------------------------------------------------------------------
%
%-----------------------------------------------------------------------

\section{Aufgabe 2: Modellierung von Interaktionen durch Sequenzdiagramme
\footcite{examen:46116:2016:09}}

Die Fahrtrichtung der ersten elektrischen Spielzeugeisenbahnen wurde
häufig durch Stromunterbrechung gesteuert. Dazu betrachten wir das
Anwendungsfall-Diagramm.

\begin{center}
\begin{tikzpicture}
\umlactor{Spieler}
\begin{umlsystem}[x=5]{}
\umlusecase[name={Steuere Lokomotive}]{Steuere Lokomotive}
\end{umlsystem}
\umlassoc{Spieler}{Steuere Lokomotive}
\end{tikzpicture}
\end{center}

\noindent
An dem Anwendungsfall \emph{„Steuere Lokomotive“} sind ein Spieler, als
Aktor außerhalb des Systems, und jeweils ein Objekt der Klassen
\emph{Stromschalter}, \emph{Lokomotive}, \emph{Scheinwerfer} und
\emph{Rad}, als Objekte innerhalb des Systems, beteiligt. Zur
Vereinfachung wird nur ein Objekt der Klasse \emph{Rad} stellvertretend
für alle vier Räder modelliert. Der Anwendungsfall zum Steuern einer
Lokomotive wird durch folgendes Szenario beschrieben.

\begin{enumerate}

\item Der Spieler schaltet den Stromschalter ein, woraufhin der Schalter
der Lokomotive \emph{Strom zuführt}.

\item Die Lokomotive schickt nun den Rädern ein Signal um
\emph{vorwärts} zu fahren.

\item Dann schaltet der Spieler den Stromschalter aus, woraufhin der
Schalter die Stromzufuhr bei der Lokomotive \emph{abstellt}.

\item Daraufhin schickt die Lokomotive das Signal \emph{stop} an die
Räder.

\item Der Spieler schaltet jetzt den Stromschalter wieder ein, woraufhin
der Schalter der Lokomotive \emph{Strom zuführt}.

\item Die Lokomotive schickt den Rädern ein Signal um \emph{rückwärts}
zu fahren.

\item Nun schaltet der Spieler den Stromschalter wieder aus, woraufhin
der Schalter die Stromzufuhr bei die Lokomotive \emph{abstellt}.

\item Daraufhin schickt die Lokomotive wieder das Signal \emph{stop} an
die Räder.
\end{enumerate}

\noindent
Geben Sie ein Sequenzdiagramm an, das die oben beschriebenen
Interaktionen zwischen Spieler, Stromschalter, Lokomotive und Rädern
beschreibt.

\begin{tikzpicture}
\begin{umlseqdiag}
\umlactor{Spieler}
\umlobject[class=Stromschalter]{schalter}
\umlobject[class=Lokomotive]{lok}
\umlobject[class=Rad]{rad}

\begin{umlcall}[op=einschalten()]{Spieler}{schalter}
\begin{umlcall}[op=stromZufuehren()]{schalter}{lok}
\begin{umlcall}[op=vorwaertsFahren()]{lok}{rad}
\end{umlcall}
\end{umlcall}
\end{umlcall}

\begin{umlcall}[op=ausschalten(),with return]{Spieler}{schalter}
\begin{umlcall}[op=stromAbstellen(),with return]{schalter}{lok}
\begin{umlcall}[op=stop(),with return]{lok}{rad}
\end{umlcall}
\end{umlcall}
\end{umlcall}

\begin{umlcall}[op=einschalten()]{Spieler}{schalter}
\begin{umlcall}[op=stromZufuehren()]{schalter}{lok}
\begin{umlcall}[op=rueckwaertsFahren()]{lok}{rad}
\end{umlcall}
\end{umlcall}
\end{umlcall}

\begin{umlcall}[op=ausschalten(),with return]{Spieler}{schalter}
\begin{umlcall}[op=stromAbstellen(),with return]{schalter}{lok}
\begin{umlcall}[op=stop(),with return]{lok}{rad}
\end{umlcall}
\end{umlcall}
\end{umlcall}

\end{umlseqdiag}
\end{tikzpicture}

\literatur

\end{document}
