\documentclass{lehramt-informatik}
\InformatikPakete{uml}
\begin{document}

%%%%%%%%%%%%%%%%%%%%%%%%%%%%%%%%%%%%%%%%%%%%%%%%%%%%%%%%%%%%%%%%%%%%%%%%
% Theorie-Teil
%%%%%%%%%%%%%%%%%%%%%%%%%%%%%%%%%%%%%%%%%%%%%%%%%%%%%%%%%%%%%%%%%%%%%%%%

\chapter{Zustandsdiagramm}

\begin{quellen}
\item \cite[Seite 329-400]{rupp}
\item \cite{net:pdf:zustandsdiagramm}
\end{quellen}

\noindent
Zustandsautomaten (state diagrams) beschreiben die Systemzustände bei
definierten Ereignissen. Das bedeutet, ein Zustandsautomat bildet die
verschiedenen Zustände ab, die ein Objekt während seiner Lebenszeit
durchläuft.
\footcite[Seite 166]{schatten}

\begin{description}

%%
%
%%

\item[Einfacher Zustand (simple state))] \strut

\begin{description}
\item[Notation]

Ein Rechteck mit abgerundeten Ecken, das durch eine waagrechte Linie
unterteilt sein kann, symbolisiert einen Zustand.

Im zweiten Abschnitt wird angegeben, welches interne Verhalten und
welche internen Transitionen in diesem Zustand ausgeführt werden können.

In der UML sind die folgenden Arten von Verhalten eines Zustandes mit
ihren Auslösern definiert. Die Auslöser gelten als Schlüsselwörter und
dürfen demnach nicht in einem an­ deren Kontext verwendet werden.

\begin{description}
\item[Eintrittsverhalten:] entry / Verhalten
\item[Austrittsverhalten:] exit / Verhalten
\item[Zustandsverhalten:] do / Verhalten
\end{description}

\item[Beschreibung]

Ein einfacher Zustand (simple state) bildet eine Situation ab, in deren
Verlauf eine spezielle Bedingung gilt.
\footcite[Seite 338]{rupp}

\begin{center}
\begin{tikzpicture}
\begin{umlstate}[name=substate, entry=Verhalten, exit=Verhalten, do=Verhalten]{Zustand}
\end{umlstate}
\end{tikzpicture}
\end{center}
\end{description}

%%
%
%%

\item[Transition / Zustandsübergängen] \strut

\begin{description}
\item[Notation] \strut

Eine Transition wird durch eine durchgezogene, gerichtete und
üblicherweise beschriftete Kante abgebildet. Die Beschriftung beinhaltet
die folgenden Elemente (Ereignis(Argumente) [Bedingung] / Aktivität):

\begin{description}
\item[Trigger] der Auslöser für die Transition. Die einzelnen Trigger
werden durch Kommas voneinander getrennt. Erläuterungen zu den
verschiedenen Typen von Triggern finden Sie in der Infobox im
nachfolgenden Abschnitt.

\item[Guard] eine Bedingung, die wahr sein muss, damit die Transition
bei Erhalt des Triggers durchlaufen wird. Die Guard wird in eckigen
Klammern notiert.

\item[Verhalten] Das Verhalten, das beim Durchlaufen der Transition
ausgeführt wird. Es wird durch den Namen des gewünschten Verhaltens
angegeben.
\end{description}

\item[Beschreibung] Transitionen schaffen einen Übergang von einem
Ausgangs- zu einem Zielzustand.
\footcite[Seite 340-341]{rupp}

\end{description}

\begin{center}
\begin{tikzpicture}
\draw[tikzuml transition style] (0,0) -- (5,0) node[auto,pos=0.5]{Trigger[Guard] / Verhalten};
\end{tikzpicture}
\end{center}

%%
%
%%

\item[Startzustand]
Ein Startzustand wird als ausgefüllter Kreis dargestellt.

\begin{center}
\tikz \umlstateinitial ;
\end{center}

%%
%
%%

\item[Endzustand]
Ein Endzustand wird als ein kleiner ausgefüllter Kreis, umgeben von
einem unausgefüllten Kreis, dargestellt.

\begin{center}
\tikz \umlstatefinal ;
\end{center}
\end{description}

%-----------------------------------------------------------------------
%
%-----------------------------------------------------------------------

\section{Übungen}

\section{Aufgabe 3 (Verhaltens-Modellierung mit Zustandsdiagrammen)
\footcite{examen:46116:2018:09}}

Eine Digitaluhr kann alternativ entweder die Zeit (Stunden und Minuten)
oder das Datum (Tag, Monat und Jahr) anzeigen. Zu Beginn zeigt die Uhr
die Zeit an. Sie besitzt drei Druckknöpfe \textbf{A}, \textbf{B} und
\textbf{C}. Mit Knopf \textbf{A} kann zwischen Zeit- und Datumsanzeige
hin und her gewechselt werden.

Wird die Zeit angezeigt, dann kann mit Knopf \textbf{B} der Reihe nach
erst in einen Stundenmodus, dann in einen Minutenmodus und schließlich
zurück zur Zeitanzeige gewechselt werden. Im Stundenmodus blinkt die
Stundenanzeige. Mit Drücken des Knopfes \textbf{C} können dann die
Stunden schrittweise inkrementiert werden. Im Minutenmodus blinkt die
Minutenanzeige und es können mit Hilfe des Knopfes \textbf{C} die
Minuten schrittweise inkrementiert werden.

Die Datumsfunktionen sind analog. Wird das Datum angezeigt, dann kann
mit Knopf \textbf{B} der Reihe nach in einen Tagesmodus, Monatsmodus,
Jahresmodus und schließlich zurück zur Datumsanzeige gewechselt werden.
Im Tagesmodus blinkt die Tagesanzeige. Mit Drücken des Knopfes
\textbf{C} können dann die Tage schrittweise inkrementiert werden.
Analog blinken mit Eintritt in den entsprechenden Einstellmodus der
Monat oder das Jahr, die dann mit Knopf \textbf{C} schrittweise
inkrementiert werden können.

Wenn sich die Uhr in einem Einstellmodus befindet, hat das Betätigen des
Knopfes \textbf{A} keine Wirkung. Ebenso wirkungslos ist Knopf
\textbf{C}, wenn gerade
Zeit oder Datum angezeigt wird.

Beschreiben Sie das Verhalten der Digitaluhr durch ein
UML-Zustandsdiagramm. Dabei muss - gemäß der UML-Notation -
unterscheidbar sein, was Ereignisse und was Aktionen sind. Deren
Bedeutung soll durch die Verwendung von sprechenden Namen klar sein. Für
die Inkrementierung von Stunden, Minuten, Tagen etc. brauchen keine
konkreten Berechnungen angegeben werden. Der kontinuierliche
Zeitfortschritt des Uhrwerks ist nicht zu modellieren.

Zustände sind, wie in der UML üblich, durch abgerundete Rechtecke
darzustellen. Sie können unterteilt werden in eine obere und eine untere
Hälfte, wobei der Name des Zustands in den oberen Teil und eine in dem
Zustand auszuführende Aktivität in den unteren Teil einzutragen ist.
\footcite{oomup:pu:2}

\begin{tikzpicture}[font=\scriptsize]
\def\TmpRekursiv#1#2{
\umltrans[recursive=-10|10|0.5cm,recursive direction=right to right,name=#1-trans]{#1}{#1}
\node[anchor=west,text width=2cm,font=\scriptsize] at (#1-trans-3){#2};
}

%%
% Zeit
%%

\begin{umlstate}{Zeitanzeige}
\end{umlstate}

\begin{umlstate}[y=-3,do={Stundenanzeige blinkt}]{Stundenmodus}
\end{umlstate}

\begin{umlstate}[y=-6,do={Minutenanzeige blinkt}]{Minutenmodus}
\end{umlstate}

\umltrans[arg={drücke Knopf B},pos=0.5]{Zeitanzeige}{Stundenmodus}
\TmpRekursiv{Stundenmodus}{drücke Knopf C / Stunden um eins erhöhen}

\umltrans[arg={drücke Knopf B},pos=0.5]{Stundenmodus}{Minutenmodus}
\TmpRekursiv{Minutenmodus}{drücke Knopf C / Minuten um eins erhöhen}

\umlHVHtrans[arm1=-3cm,arg={drücke Knopf B},pos=1.75,swap]{Minutenmodus}{Zeitanzeige}

%%
% Datum
%%

\begin{umlstate}[x=7,y=0]{Datumsanzeige}
\end{umlstate}

\begin{umlstate}[x=7,y=-3,do={Tagesanzeige blinkt}]{Tagesmodus}
\end{umlstate}
\TmpRekursiv{Tagesmodus}{drücke Knopf C / Tag um eins erhöhen}

\begin{umlstate}[x=7,y=-6,do={Monatsanzeige blinkt}]{Monatsmodus}
\end{umlstate}
\TmpRekursiv{Monatsmodus}{drücke Knopf C / Monat um eins erhöhen}

\begin{umlstate}[x=7,y=-9,do={Jahresanzeige blinkt}]{Jahresmodus}
\end{umlstate}
% \TmpRekursiv{Jahresmodus}{drücke Knopf C / Jahr um eins erhöhen}
\umltrans[recursive=170|190|0.5cm,recursive direction=left to left,name=Jahresmodus-trans]{Jahresmodus}{Jahresmodus}
\node[anchor=east,text width=2cm,font=\scriptsize] at (Jahresmodus-trans-3){drücke Knopf C / Jahr um eins erhöhen};

\umltrans[arg={drücke Knopf C},pos=0.5]{Datumsanzeige}{Tagesmodus}
\umltrans[arg={drücke Knopf C},pos=0.5]{Tagesmodus}{Monatsmodus}
\umltrans[arg={drücke Knopf C},pos=0.5]{Monatsmodus}{Jahresmodus}
\umlHVHtrans[arm1=5cm,arg={drücke Knopf C},pos=1.5]{Jahresmodus}{Datumsanzeige}

\umlVHVtrans[arm1=1.5cm,arg={drücke Knopf A},pos=1.5]{Zeitanzeige}{Datumsanzeige}
\umltrans[arg={drücke Knopf B},pos=0.5]{Datumsanzeige}{Zeitanzeige}

\end{tikzpicture}

%-----------------------------------------------------------------------
%
%-----------------------------------------------------------------------

\section{Übung von Wikiversity\footcite{net:pdf:wikiversity:zustandsdiagramm}}

Max und Sarah haben ein Ziel. Jeder Gast, der ihr Restaurant, das
„fleißige Bienchen“ besucht, soll so schnell wie möglich satt werden. Um
das zu erreichen arbeiten sie auf Hochtouren. Wenn ein Gast sein Essen
bestellt, wird das sofort in der Küche gehört. Max fängt direkt an das
Gericht zuzubereiten. Man hat es noch nicht bewiesen, aber
wahrscheinlich ist er der schnellste Koch der Welt und dementsprechend
dauert das Kochen nur wenige Minuten. Ist das Gericht fertig rennt Sarah
zu den Gästen und serviert es. Manchmal verliert sie dabei, weil sie so
schnell rennt, Teile des Essens. Dann muss sie zurück in die Küche, in
der das Gericht neu zubereitet wird. Wenn aber alles gut gegangen ist,
kann der Gast schnell essen. Gäste mit kleinem Hunger sind nach dem
Essen satt. Hungrige Gäste bestellen noch einen Nachtisch der schon
bereit steht, damit keine Zeit verloren geht. Ist dieser aufgegessen
sind auch die hungrigsten Gäste satt. Sarah und Max haben
herausgefunden, dass die Gäste beim Bezahlen der Rechnung viel mehr
Trinkgeld geben, wenn sie innerhalb von einer halben Stunde satt sind.
Aber egal, wie viel Geld sie bekommen. Max und Sarah sind glücklich über
jeden Gast der sie besucht.

% https://www.isf.cs.tu-bs.de/cms/teaching/2012w/se1/solution6.pdf

\literatur

\end{document}
