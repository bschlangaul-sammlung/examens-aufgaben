\documentclass{lehramt-informatik-haupt}
\InformatikPakete{uml}
\begin{document}

%%%%%%%%%%%%%%%%%%%%%%%%%%%%%%%%%%%%%%%%%%%%%%%%%%%%%%%%%%%%%%%%%%%%%%%%
% Theorie-Teil
%%%%%%%%%%%%%%%%%%%%%%%%%%%%%%%%%%%%%%%%%%%%%%%%%%%%%%%%%%%%%%%%%%%%%%%%

\chapter{Zustandsdiagramm}

\begin{quellen}
\item \cite[Seite 329-400]{rupp}
\item \cite{net:pdf:zustandsdiagramm}
\end{quellen}

\noindent
Zustandsautomaten (state diagrams) beschreiben die Systemzustände bei
definierten Ereignissen. Das bedeutet, ein Zustandsautomat bildet die
verschiedenen Zustände ab, die ein Objekt während seiner Lebenszeit
durchläuft.
\footcite[Seite 166]{schatten}

\begin{description}

%%
%
%%

\item[Einfacher Zustand (simple state))] \strut

\begin{description}
\item[Notation]

Ein Rechteck mit abgerundeten Ecken, das durch eine waagrechte Linie
unterteilt sein kann, symbolisiert einen Zustand.

Im zweiten Abschnitt wird angegeben, welches interne Verhalten und
welche internen Transitionen in diesem Zustand ausgeführt werden können.

In der UML sind die folgenden Arten von Verhalten eines Zustandes mit
ihren Auslösern definiert. Die Auslöser gelten als Schlüsselwörter und
dürfen demnach nicht in einem an­ deren Kontext verwendet werden.

\begin{description}
\item[Eintrittsverhalten:] entry / Verhalten
\item[Austrittsverhalten:] exit / Verhalten
\item[Zustandsverhalten:] do / Verhalten
\end{description}

\item[Beschreibung]

Ein einfacher Zustand (simple state) bildet eine Situation ab, in deren
Verlauf eine spezielle Bedingung gilt.
\footcite[Seite 338]{rupp}

\begin{center}
\begin{tikzpicture}
\begin{umlstate}[name=substate, entry=Verhalten, exit=Verhalten, do=Verhalten]{Zustand}
\end{umlstate}
\end{tikzpicture}
\end{center}
\end{description}

%%
%
%%

\item[Transition / Zustandsübergängen] \strut

\begin{description}
\item[Notation] \strut

Eine Transition wird durch eine durchgezogene, gerichtete und
üblicherweise beschriftete Kante abgebildet. Die Beschriftung beinhaltet
die folgenden Elemente (Ereignis(Argumente) [Bedingung] / Aktivität):

\begin{description}
\item[Trigger] der Auslöser für die Transition. Die einzelnen Trigger
werden durch Kommas voneinander getrennt. Erläuterungen zu den
verschiedenen Typen von Triggern finden Sie in der Infobox im
nachfolgenden Abschnitt.

\item[Guard] eine Bedingung, die wahr sein muss, damit die Transition
bei Erhalt des Triggers durchlaufen wird. Die Guard wird in eckigen
Klammern notiert.

\item[Verhalten] Das Verhalten, das beim Durchlaufen der Transition
ausgeführt wird. Es wird durch den Namen des gewünschten Verhaltens
angegeben.
\end{description}

\item[Beschreibung] Transitionen schaffen einen Übergang von einem
Ausgangs- zu einem Zielzustand.
\footcite[Seite 340-341]{rupp}

\end{description}

\begin{center}
\begin{tikzpicture}
\draw[tikzuml transition style] (0,0) -- (5,0) node[auto,pos=0.5]{Trigger[Guard] / Verhalten};
\end{tikzpicture}
\end{center}

%%
%
%%

\item[Startzustand]
Ein Startzustand wird als ausgefüllter Kreis dargestellt.

\begin{center}
\tikz \umlstateinitial ;
\end{center}

%%
%
%%

\item[Endzustand]
Ein Endzustand wird als ein kleiner ausgefüllter Kreis, umgeben von
einem unausgefüllten Kreis, dargestellt.

\begin{center}
\tikz \umlstatefinal ;
\end{center}
\end{description}

%-----------------------------------------------------------------------
%
%-----------------------------------------------------------------------

\capter{Übungen}

\ExamensAufgabeTTA 46116 / 2018 / 09 : Thema 1 Teilaufgabe 1 Aufgabe 3

%-----------------------------------------------------------------------
%
%-----------------------------------------------------------------------

\section{Übung von Wikiversity\footcite{net:pdf:wikiversity:zustandsdiagramm}}

Max und Sarah haben ein Ziel. Jeder Gast, der ihr Restaurant, das
„fleißige Bienchen“ besucht, soll so schnell wie möglich satt werden. Um
das zu erreichen arbeiten sie auf Hochtouren. Wenn ein Gast sein Essen
bestellt, wird das sofort in der Küche gehört. Max fängt direkt an das
Gericht zuzubereiten. Man hat es noch nicht bewiesen, aber
wahrscheinlich ist er der schnellste Koch der Welt und dementsprechend
dauert das Kochen nur wenige Minuten. Ist das Gericht fertig rennt Sarah
zu den Gästen und serviert es. Manchmal verliert sie dabei, weil sie so
schnell rennt, Teile des Essens. Dann muss sie zurück in die Küche, in
der das Gericht neu zubereitet wird. Wenn aber alles gut gegangen ist,
kann der Gast schnell essen. Gäste mit kleinem Hunger sind nach dem
Essen satt. Hungrige Gäste bestellen noch einen Nachtisch der schon
bereit steht, damit keine Zeit verloren geht. Ist dieser aufgegessen
sind auch die hungrigsten Gäste satt. Sarah und Max haben
herausgefunden, dass die Gäste beim Bezahlen der Rechnung viel mehr
Trinkgeld geben, wenn sie innerhalb von einer halben Stunde satt sind.
Aber egal, wie viel Geld sie bekommen. Max und Sarah sind glücklich über
jeden Gast der sie besucht.

% https://www.isf.cs.tu-bs.de/cms/teaching/2012w/se1/solution6.pdf

\literatur

\end{document}
