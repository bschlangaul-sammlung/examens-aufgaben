\documentclass{lehramt-informatik}
\usepackage{tikz}
\usepackage{tikz-uml}
\usepackage{tikz-uml-activity}

\begin{document}

%%%%%%%%%%%%%%%%%%%%%%%%%%%%%%%%%%%%%%%%%%%%%%%%%%%%%%%%%%%%%%%%%%%%%%%%
% Theorie-Teil
%%%%%%%%%%%%%%%%%%%%%%%%%%%%%%%%%%%%%%%%%%%%%%%%%%%%%%%%%%%%%%%%%%%%%%%%

\chapter{Aktivitätsdiagramm (activity diagram)}

\begin{quellen}
\item \cite[Seite 263-328]{rupp}
\item \cite{wiki:aktivitaetsdiagramm}
\end{quellen}

Aktivitätsdiagramme (activity diagrams) zeigen eine bestimmte Sicht auf
die Dynamik des Systems (Abläufe). Sie beschreiben das konkrete
Verhalten eines Systems. Aktivitätsdiagramme sind meist sehr
domänenspezifisch und ohne weiteren Kontext schwer zu verstehen.
\footcite[Seite 166]{schatten}

\begin{description}

%%
%
%%

\item[Aktion (action)] \strut

\begin{description}
\item[Notation]
Eine Aktion wird durch ein Rechteck mit abgerundeten Ecken dargestellt.
Dieses kann mit einem Namen oder einem ausführlichen Text für die Aktion
versehen werden.

\item[Beschreibung]
Eine Aktion (action) steht für den Aufruf eines Verhaltens oder die
Bearbeitung von Daten, die innerhalb einer Aktivität nicht weiter
zerlegt wird. Sie beschreibt einen Einzelschritt, der zur Realisierung
des durch die Aktivität beschriebenen Verhaltens beiträgt.
\footcite[Seite 274]{rupp}
\end{description}

\begin{center}
\tikz \node[action]{Aktionsname};
%
\tikz \node[action,text width=2.5cm,align=center]{Schulaufgabe \\erstellen};
\end{center}

%%
%
%%

\item[Aktivität (activity)] \strut

\begin{description}
\item[Notation]
Eine Aktivität wird nach außen durch ein Rechteck mit abgerundeten Ecken
abgegrenzt. Mögliche Parameter werden durch Objektknoten auf der Grenze
dargestellt.

\item[Beschreibung]
Aktivität bezeichnet die die gesamte Einheit, die in einem
Aktivitätsmodell modelliert wird. Dieses Modell besteht aus einer Folge
von Aktionen und weiteren Elementen.
\footcite[Seite 278-279]{rupp}
\end{description}

%%
%
%%

\item[Objektknoten (object node)] \strut

\begin{description}
\item[Notation]
Ein Objektknoten wird durch ein Rechteck gezeichnet.

\item[Beschreibung]
Ein Objektknoten (object node) innerhalb einer Aktivität repräsentiert
Ausprägungen eines  bestimmten Typs. In den meisten Fällen sind das
Werte oder Objekte von Klassen. Objektknoten bilden das logische Gerüst,
um Daten und Werte innerhalb einer Aktivität während eines Ablaufs zu
transportieren.
\footcite[Seite 281]{rupp}
\end{description}

\begin{center}
\tikz \node[object]{Objektknoten};
\end{center}

%%
%
%%

\item[Kanten (activity edge)] \strut

\begin{description}
\item[Notation]
Kanten werden als Pfeile gezeichnet.

\item[Beschreibung]
Kanten sind Übergänge zwischen zwei Knoten (zum Beispiel Aktionen,
Objektknoten, ...). Die Kanten sind immer gerichtet und können zum
besseren Verständnis mit einem Namen versehen werden.
\footcite[Seite 287]{rupp}
\end{description}

\begin{center}
\tikz \draw[arrow] (0,0) -- (1.5,0);
\hspace{0.5cm}
\tikz \draw[arrow] (0,0) -- (1.5,0) node[auto,pos=0.5]{Name};
\end{center}

%%
%
%%

\item[Startknoten (initial node)] \strut

\begin{description}
\item[Notation]
Ein Startknoten wird durch einen ausgefüllten schwarzen Punkt notiert.

\item[Beschreibung]
Ein Startknoten (initial node) markiert den Startpunkt eines Ablaufs bei
Aktivierung einer Aktivität. Eine Aktivität darf beliebig viele
Startknoten besitzen. Jeder Startknoten darf wiederum beliebig viele
wegführende Kanten besitzen. Von einem Startknoten führen aller­dings
nur Kontrollkanten weg.
\footcite[Seite 292]{rupp}
\end{description}

\begin{center}
\tikz{
  \node[initial] at (1,0) (node) {};
  \draw[arrow] (0,0) -- (node);
}
\end{center}

\item[Endknoten für Aktivitäten (flow final node)]

\tikz{
  \node[flow final] at (1,0) (node) {};
  \draw[arrow] (0,0) -- (node);
}

\item[Endknoten für Kontrollflüsse (activity final node)]

\tikz{
  \node[activity final] at (1,0) (node) {};
  \draw[arrow] (0,0) -- (node);
}

\item[Verzweigungsknoten]

\tikz{
  \node[decision,scale=1.5] at (1,0) (node) {};
  \draw[arrow] (0,0) -- (node);
  \draw[arrow] (node) -- (2,0.5);
  \draw[arrow] (node) -- (2,0);
  \draw[arrow] (node) -- (2,-0.5);
}

\item[Verbindungsknoten]

\tikz{
  \node[decision,scale=1.5] at (1,0) (node) {};
  \draw[arrow] (0,0.5) -- (node);
  \draw[arrow] (0,0) -- (node);
  \draw[arrow] (0,-0.5) -- (node);
  \draw[arrow] (node) -- (2,0);
}

\end{description}

%%%%%%%%%%%%%%%%%%%%%%%%%%%%%%%%%%%%%%%%%%%%%%%%%%%%%%%%%%%%%%%%%%%%%%%%
% Aufgaben
%%%%%%%%%%%%%%%%%%%%%%%%%%%%%%%%%%%%%%%%%%%%%%%%%%%%%%%%%%%%%%%%%%%%%%%%

\chapter{Aufgaben}

\ExamensAufgabe 66116 / 2017 / 09 : Thema 2 Teilaufgabe 2 Aufgabe 2

\literatur

\end{document}
