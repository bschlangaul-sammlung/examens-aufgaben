\documentclass{lehramt-informatik}
\InformatikPakete{uml}

\begin{document}

%%%%%%%%%%%%%%%%%%%%%%%%%%%%%%%%%%%%%%%%%%%%%%%%%%%%%%%%%%%%%%%%%%%%%%%%
% Theorie-Teil
%%%%%%%%%%%%%%%%%%%%%%%%%%%%%%%%%%%%%%%%%%%%%%%%%%%%%%%%%%%%%%%%%%%%%%%%

\chapter{Anwendungsfalldiagramm / Nutzfalldiagramm (use case diagram)}

\begin{quellen}
\item \cite[Seite 241-262]{rupp}
\item \cite{wiki:anwendungsfalldiagramm}
\end{quellen}

Anwendungsfalldiagramme (use-case diagrams) stellen die externe
Anwendersicht auf das System dar und beschreiben, welche funktionalen
Anforderungen durch das System umgesetzt werden.
\footcite[Seite 166]{schatten}

Nützliches Werkzeug am Beginn einer Systementwicklung.

Ziele:
Überblick über die Funktionalität des Endprodukts für Kunden und
Entwickler

Ermöglicht offene und zielgerichtete Kommunikation mit Kunden in der
Frühphase der Entwicklung

\section{Notationselemente}

\begin{description}

%%
%
%%

\item[Use-Case]
\begin{description} \strut

\item In aller Regel wird ein Use-Case durch eine Ellipse
dargestellt. Der Name des Use-Cases wird dabei inner- oder unterhalb der
Ellipse notiert.\footcite[Seite 246]{rupp}
\end{description}

\begin{center}
\begin{tikzpicture}[
  every label/.style={
    font=\small,
  }
]
\umlusecase{Use-Case-Name}
\umlusecase[x=4,label=below:Use-Case-Name]{\strut\hspace{1.5cm}\strut}
\end{tikzpicture}
\end{center}

Neben der Modellierung von einzelnen, autarken Use-Cases dürfen Sie
Use-Cases auch in Beziehung zueinander setzen. Dadurch verknüpfen Sie
die Abläufe der einzelnen Use-Cases zu einem Ablauf.\footcite[Seite 248]{rupp}

Durch die Generalisierungsbeziehung (Generalisation) können
hierarchische Zusammenhänge zwischen Anwendungsfälle beschrieben werden.
Generellere Use Cases werden durch konkretere verfeinert. Ebenso können
Anwendungsfälle abstrakt sein (Name ist kursiv geschrieben) und erst
durch konkretere Anwendungsfälle „ausführbar“ werden.
\footnote{\url{https://www.sparxsystems.de/ressourcen/literatur/leseprobe-zu-projektabwicklung-mit-uml-und-enterprise-architect/anwendungsfalldiagramm-use-case-diagram/}}

\begin{center}
\begin{tikzpicture}
\umlusecase[x=3,y=0,name={Service nutzen}]{Service nutzen}
\umlusecase[x=0,y=-2,name={Geld abheben},text width=1.5cm]{Geld abheben}
\umlusecase[x=3,y=-2,name={Geld einzahlen},text width=1.5cm]{Geld einzahlen}
\umlusecase[x=6,y=-2,name={Kontostand abfragen},text width=2cm]{Kontostand abfragen}
\umlVHVinherit{Geld abheben}{Service nutzen}
\umlVHVinherit{Geld einzahlen}{Service nutzen}
\umlVHVinherit{Kontostand abfragen}{Service nutzen}
\end{tikzpicture}
\end{center}

%%
%
%%

\item[System (Betrachtungsgegenstand)]

Das System wird als rechteckiger Kasten abgebildet, wobei die Kanten des
Systems die Systemgrenzen darstellen. Der Name des Systems wird
innerhalb des Kastens angegeben.\footcite[Seite 249]{rupp}

\begin{tikzpicture}
\begin{umlsystem}{Systemname}
\end{umlsystem}
\end{tikzpicture}

%%
%
%%

\item[Akteur]

Die gebräuchlichste Notation für einen Akteur ist das Strichmännchen,
mit dem Namen des Akteurs oberhalb oder unterhalb. Allerdings erlaubt
Ihnen die UML, den Namen auch rechts oder links vom Strichmännchen zu
notieren. Vorgeschrieben ist nur, dass der Name in der Umgebung des
Strichmännchens stehen muss.\footcite[Seite 252]{rupp}

\begin{tikzpicture}
\umlactor{Name des Akteurs}
\end{tikzpicture}

%%
%
%%

\item[«include»-Beziehung] Die «include»-Beziehung wird durch eine
unterbrochene gerichtete Kante mit dem Schlüs­selwort «include»
dargestellt.\footcite[Seite 256]{rupp}

Ausführung des inkludierenden Anwendungsfalls führt immer auch alle
inkludierten Anwendungsfälle aus.\footnote{Aus Referats-Handout}

\begin{tikzpicture}
\umlusecase[name=useinclude-1]{Use-Case-1}
\umlusecase[name=useinclude-2,x=5]{Use-Case-2}
\umlinclude{useinclude-1}{useinclude-2}
\end{tikzpicture}

%
%%

\item[«extend»-Beziehung]

\begin{description}
\item[Notation]

Eine «extend»-Beziehung ist eine unterbrochene gerichtete Kante mit der
Bezeichnung «extend» vom erweiternden Use-Case zum erweiterten Use-Case.
\footcite[Seite 258]{rupp}

\item[Beschreibung]

Ausführung des erweiterten Anwendungsfall \emph{kann} den erweiternden
Anwendungsfall ausführen.\footnote{Aus Referats-Handout}
\end{description}

\begin{tikzpicture}
\umlusecase[name=useextend-1]{Use-Case-1}
\umlusecase[name=useextend-2,x=5]{Use-Case-2}
\umlextend{useextend-1}{useextend-2}
\end{tikzpicture}

Neben dem Erweiterungspunkt können Sie zudem eine \memph{Bedingung für die
Erweiterung} angeben. Die Bedingung wird bei Erreichen des
Erweiterungspunktes geprüft. Ist die Bedingung wahr, wird der
Ablauf erweitert, das heißt, der entsprechende referenzierte Use-Case
durchlaufen durchlaufen. Ist die Bedingung nicht erfüllt, läuft der
Use-Case-Ablauf „normal“ weiter.

Die entsprechenden Bedingungen und der zugehörige Erweiterungspunkt
werden als \memph{Notiz­zettel} (condition) an die «extend»-Beziehung
notiert.

\begin{center}
\begin{tikzpicture}
\umlusecase[name=Authentifizierung]{Authentifizierung}
\umlusecase[x=8,name={Karte sperren}]{Karte sperren}
\umlextend[name=extend-sperren]{Authentifizierung}{Karte sperren}
\umlnote[x=4,y=-2]{extend-sperren-1}{Karte und PIN konten 3x nicht erfolgreich geprüft werden.}
\end{tikzpicture}
\end{center}

\end{description}

%%%%%%%%%%%%%%%%%%%%%%%%%%%%%%%%%%%%%%%%%%%%%%%%%%%%%%%%%%%%%%%%%%%%%%%%
% Aufgaben
%%%%%%%%%%%%%%%%%%%%%%%%%%%%%%%%%%%%%%%%%%%%%%%%%%%%%%%%%%%%%%%%%%%%%%%%

\chapter{Aufgaben}

%-----------------------------------------------------------------------
%
%-----------------------------------------------------------------------

\section{Aufgabe 3\footcite{sosy:ab:3}}

Im Folgenden sehen Sie ein fehlerhaftes Use-Case-Diagramm für das System
„Geldautomat”. Geben Sie ein korrektes Use-Case-Diagramm (Beziehungen
und Beschriftungen) entsprechend der folgenden Beschreibung an. Sollte
es mehrere Möglichkeiten geben, begründen Sie Ihre Entscheidung kurz.

Kunden können am Geldautomat verschiedene Services nutzen, es kann Geld
abgehoben und eingezahlt sowie der Kontostand abgefragt werden. Für
jeden dieser Services ist eine Authentifizierung notwendig. Diese
besteht aus der Prüfung der Bankkarte und des eingegebenen PINs. Manche
Bankautomaten können Karten bei zu vielen Fehlversuchen (3) bei der
Anmeldung sperren, andere geben Fehlermeldungen aus, falls die Karte
gesperrt wurde oder nicht mehr genügend Geld auf dem Konto oder im
Bankautomaten vorhanden ist. Mitarbeiter bzw. Mitarbeiterinnen der Bank
können sich ebenfalls über PIN und Karte authentifizieren, um dann den
Bankautomaten neu zu starten oder mit Geld aufzufüllen. Bevor Geld
abgehoben werden kann, ist sicherzustellen, dass auf dem Konto und im
Bankautomaten genügend Geld vorhanden ist.
\footcite[StEx Realschule Frühjahr 2017, Thema 1, TA 1, A2]{examen:46116:2017:03}

\begin{antwort}
\begin{tikzpicture}[scale=0.7,transform shape]
\begin{umlsystem}{Geldautomat}

\umlusecase[x=2,y=0,name={Service nutzen}]{Service nutzen}
\umlusecase[x=8,y=0,name=Authentifizierung]{Authentifizierung}
\umlinclude{Service nutzen}{Authentifizierung}

\umlusecase[x=13,y=2,name={Karte prüfen}]{Karte prüfen}
\umlusecase[x=13,y=0,name={PIN prüfen}]{PIN prüfen}
\umlusecase[x=13,y=-2,name={Karte sperren}]{Karte sperren}

\umlinclude{Authentifizierung}{Karte prüfen}
\umlinclude{Authentifizierung}{PIN prüfen}
\umlextend[name=extend-sperren]{Authentifizierung}{Karte sperren}
\umlnote[x=13,y=-4]{extend-sperren-1}{Karte und PIN konten 3x nicht erfolgreich geprüft werden.}

\umlusecase[x=9,y=-4,name={Bankautomat neustarten},text width=2cm]{Bankautomat neustarten}
\umlusecase[x=5,y=-4,name={Bankautomat auffüllen},text width=2cm]{Bankautomat auffüllen}
\umlinclude{Bankautomat neustarten}{Authentifizierung}
\umlinclude{Bankautomat auffüllen}{Authentifizierung}
\end{umlsystem}

\umlactor[x=7,y=-6.5,name=Mitarbeiter]{Mitarbeiter}
\umlVHVassoc{Mitarbeiter}{Bankautomat neustarten}
\umlVHVassoc{Mitarbeiter}{Bankautomat auffüllen}

\umlactor[x=-1,y=0,name=Kunde]{Kunde}
\umlassoc{Kunde}{Service nutzen}

\umlusecase[x=0,y=2,name={Geld abheben},text width=1.5cm]{Geld abheben}
\umlusecase[x=3,y=2,name={Geld einzahlen},text width=1.5cm]{Geld einzahlen}
\umlusecase[x=6,y=2,name={Kontostand abfragen},text width=2cm]{Kontostand abfragen}
\umlVHVinherit{Geld abheben}{Service nutzen}
\umlVHVinherit{Geld einzahlen}{Service nutzen}
\umlVHVinherit{Kontostand abfragen}{Service nutzen}

\umlusecase[x=0,y=4,name={Kontostand / Automat prüfen},text width=1.5cm]{Kontostand / Automat prüfen}
\umlusecase[x=2,y=-2,name={Fehlermeldung},text width=1.5cm]{Fehlermeldung}

\end{tikzpicture}
\end{antwort}

\literatur

\end{document}
