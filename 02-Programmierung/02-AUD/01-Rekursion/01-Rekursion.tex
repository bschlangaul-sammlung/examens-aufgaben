\documentclass{lehramt-informatik}
\InformatikPakete{mathe,syntax}
\usepackage{paralist}

\begin{document}

%%%%%%%%%%%%%%%%%%%%%%%%%%%%%%%%%%%%%%%%%%%%%%%%%%%%%%%%%%%%%%%%%%%%%%%%
% Theorie-Teil
%%%%%%%%%%%%%%%%%%%%%%%%%%%%%%%%%%%%%%%%%%%%%%%%%%%%%%%%%%%%%%%%%%%%%%%%

\chapter{Rekursion}

\begin{quellen}
\item \cite[Seite 27-30]{saake}
\item \cite[Seite 178 6.1.2.5]{schneider}
\item \cite{wiki:rekursive-programmierung}
\end{quellen}

\noindent
Bei der \emph{rekursiven Programmierung} ruft sich eine
\memph{Prozedur}, \memph{Funktion} oder \memph{Methode} in einem
Computerprogramm \memph{selbst wieder auf} (d. h. enthält eine
Rekursion). Auch der gegenseitige Aufruf stellt eine Rekursion dar.

Wichtig bei der rekursiven Programmierung ist eine
\memph{Abbruchbedingung} in dieser Funktion, weil sich das rekursive
Programm sonst (theoretisch) unendlich oft selbst aufrufen würde.
\footcite{wiki:rekursive-programmierung}

%-----------------------------------------------------------------------
%
%-----------------------------------------------------------------------

\section{Ein Beispiel: Das Begrüßungsproblem\footcite[Seite 12-14]{aud:fs:1}}

\begin{minted}{java}
int anzahlBegruessungen(int anzahlPersonen) {
  if (anzahlPersonen < 2) {
    return 0;
  } else {
    return anzahlBegruessungen(anzahlPersonen - 1) + (anzahlPersonen - 1);
  }
}
\end{minted}

%-----------------------------------------------------------------------
%
%-----------------------------------------------------------------------

\section{Typen von Rekursion\footcite{net:html:gehaxelt:rekursionsarten}}

\begin{itemize}
\item \textbf{Indirekte Rekursion}

Bei der indirekten Rekursion ruft die Funktion eine andere Funktion auf,
welche wiederum die aufrufende Funktion aufruft.

\item \textbf{Direkte Rekursion}

Bei der direkten Rekursion ruft sich eine Funktion wieder selbst auf.

\begin{itemize}

%%
%
%%

\item \textbf{Repetitive Rekursion}

Bei der repetitiven Rekursion ruft sich die Funktion mit einem
\emph{veränderten Parameter} auf:

\begin{minted}{java}
public int f(int x, int y) {
  if (x == 0) return y
  y = y + x
  return func(x - 1, y)
}
\end{minted}

%%
%
%%

\item \textbf{Lineare Rekursion}

Bei der linearen Rekursion wird der \emph{übergebene Parameter} mit dem
Rekursion\emph{sergebnis verrechnet}:

\begin{minted}{java}
public int f(int x) {
  if (x == 0) return 1
  return x * f(x - 1)
}
\end{minted}

%%
%
%%

\item \textbf{Baumartige Rekursion}

Die baumartige Rekursion kommt zum Einsatz, wenn man das Ergebnis aus
\emph{zwei verschiedenen Rekursionsaufrufen} berechnet.

\begin{minted}{java}
public int f(int x) {
  if (x == 1) return 0
  return f(x - 1) + f(x - 2)
}
\end{minted}

%%
%
%%

\item \textbf{Geschachtelte Rekursion}

Bei der geschachtelten Rekursion ist das \emph{Ergebnis} des
Rekursionsaufrufes \emph{Parameter eines Rekursionsaufrufes}.

\begin{minted}{java}
public int f(int x) {
  if (x == 1) return 0
  return f(x - f(x-1))
}
\end{minted}

%%
%
%%

\item \textbf{Verschränkte Rekursion}

Bei der verschränkten Rekursion \emph{rufen} sich \emph{zwei Funktionen
gegenseitig} auf.

\begin{minted}{java}
public int f(int x) {
  if (x == 0) return 1
  return g(x - 1)
}

public int g(int x) {
  if (x == 1) return 0
  return f(x - 1)
}
\end{minted}
\end{itemize}
\end{itemize}

%-----------------------------------------------------------------------
%
%-----------------------------------------------------------------------

\section{Iterativ vs. Rekursiv}

\cite[Seite 16-19 (Gedruckte Seitenzahlen stimmen nicht)]{aud:fs:1}

Interativ

\begin{minted}{java}
public static void pizzaIterativ (int anzahlStuecke){
  for (int i = 1; i< anzahlStuecke; i++) {
    schneiden();
  }
  essen();
}
\end{minted}

Rekursiv

\begin{minted}{java}
public static pizzaRekursiv (int anzahlStuecke)
  if (anzahlStuecke == 1) {
    essen();
  } else {
    schneiden();
    pizzaRekursiv(anzahlStuecke - 1);
  }
}
\end{minted}

\inputcode[firstline=3]{aufgaben/aud/pu_1/Rater}

\inputcode[firstline=3]{aufgaben/aud/pu_1/Rekursion}

%%%%%%%%%%%%%%%%%%%%%%%%%%%%%%%%%%%%%%%%%%%%%%%%%%%%%%%%%%%%%%%%%%%%%%%%
% Aufgaben
%%%%%%%%%%%%%%%%%%%%%%%%%%%%%%%%%%%%%%%%%%%%%%%%%%%%%%%%%%%%%%%%%%%%%%%%

\chapter{Aufgaben}

%-----------------------------------------------------------------------
%
%-----------------------------------------------------------------------

\section{Aufgabe 2: Rekursion\footcite[Seite 2]{aud:ab:1}}

Gegeben ist folgende Methode.

\inputcode[firstline=5,lastline=12]{aufgaben/aud/ab_1/Rekursion}
\begin{enumerate}

%%
% (a)
%%

\item Beschreiben Sie kurz, woran man erkennt, dass es sich bei der
gegebenen Methode um eine rekursive Methode handelt. Gehen Sie dabei auf
wichtige Bestandteile der rekursiven Methode ein.

\begin{antwort}
Die Methode mit dem Namen \java{function} ruft sich in der letzten
Code-Zeile selbst auf. Außerdem gibt es eine Abbruchbedingung
(\java{if (e == 1) { return b * 1; }}), womit verhindert wird, dass die
Rekursion unendlich weiter läuft.
\end{antwort}

%%
% (b)
%%

\item Geben Sie die Rekursionsvorschrift für die Methode \java{function}
an. Denken Sie dabei an die Angabe der Zahlenbereiche!

\begin{antwort}
\begin{equation*}
\text{int function(int b, int e)} =
\begin{cases}
\text{return b*1}, & \text{falls e = 1}.\\
\text{return b*function(b,e-1)}, & \text{falls e > 1}.
\end{cases}
\end{equation*}
\end{antwort}

%%
% (c)
%%

\item Erklären Sie kurz, was die Methode \java{function} berechnet.

\begin{antwort}
Die Methode \java{function} berechnet die Potenz $b^e$.
\end{antwort}

%-----------------------------------------------------------------------
%
%-----------------------------------------------------------------------

\section{Aufgabe 1: Arrays und Rekursion\footcite[Diese Aufgabe stammt
aus der Vorlesung Konzepte der Programmierung von Prof. Bernhard
Westfechtel der Universität Bayreuth, WS 2017/18, Übungsblatt 8 und
wurde dankenswerterweise zur Verwendung in diesem Aufgabenblatt zur
Verfügung gestellt.]{aud:ab:2}}

\begin{enumerate}

%%
%
%%

\item Erstellen Sie eine neue Klasse \java{ArrayInvertierer} mit einer
rekursiven Methode, die den Inhalt eines ihr übergebenen 1D-Arrays
gefüllt mit Strings invertiert. Auf diese Weise kann z.B. ein deutscher
Satz im Array gespeichert und dann verkehrt herum ausgegeben werden.

\emph{Wichtig:} Nicht das übergebene Array soll verändert werden,
sondern ein Neues erstellt und von der Methode zurückgegeben werden.

\emph{Tipp:} Sie dürfen dafür gerne auch rekursive Hilfsmethoden
benutzen.

%%
%
%%

\item Implementieren Sie dann eine main-Methode, in der Sie zwei
verschieden lange \java{String}-Arrays erzeugen und die Wortreihenfolge
umkehren lassen. Das Ergebnis soll auf der Konsole ausgegeben werden
und könnte z.B. wie folgt aussehen.

\bigskip

{
\ttfamily
Den Satz\\
Ich find dich einfach klasse!\\
wuerde Meister Yoda so aussprechen:\\
klasse! einfach dich find Ich\\

Den Satz\\
Das war super einfach/schwer\\
wuerde Meister Yoda so aussprechen:\\
einfach/schwer super war Das
}

\bigskip

[optional] Wenn das ursprüngliche \java{String}-Array selbst verändert
werden soll, braucht die rekursive Methode keine Rückgabe. Versuchen
Sie, diese Aufgabe ohne das Nutzen einer Hilfsmethode zu lösen.

\begin{antwort}
\inputcode[firstline=3]{aufgaben/aud/ab_2/ArrayInvertierer}
\end{antwort}
\end{enumerate}
\end{enumerate}

\ExamensAufgabeTA 46115 / 2014 / 03 : Thema 2 Aufgabe 4

\literatur

\end{document}
