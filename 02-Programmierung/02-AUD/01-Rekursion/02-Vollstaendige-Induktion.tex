\documentclass{lehramt-informatik-haupt}
\liLadePakete{syntax,mathe}

\begin{document}

%%%%%%%%%%%%%%%%%%%%%%%%%%%%%%%%%%%%%%%%%%%%%%%%%%%%%%%%%%%%%%%%%%%%%%%%
% Theorie-Teil
%%%%%%%%%%%%%%%%%%%%%%%%%%%%%%%%%%%%%%%%%%%%%%%%%%%%%%%%%%%%%%%%%%%%%%%%

\chapter{Vollständige Induktion}

\begin{quellen}
\item \cite[Seite 115-120]{meinel}
\item \cite{wiki:vollständige-induktion}
\item \cite[Seite 22-27 (PDF 20-25)]{aud:fs:1}
\end{quellen}

\noindent
Die vollständige Induktion ist eine \memph{mathematische Beweismethode},
nach der eine Aussage für alle \memph{natürlichen Zahlen} bewiesen wird,
die \memph{größer oder gleich einem bestimmten Startwert} sind. Da es
sich um unendlich viele Zahlen handelt, kann eine Herleitung nicht für
jede Zahl einzeln erbracht werden.

Der Beweis, dass die Aussage
$\operatorname{A}(n)$ für alle
$n\geq n_{0}$
($n_0$ meist $1$ oder $0$) gilt, wird daher in zwei Etappen durchgeführt:

\begin{enumerate}
\item Im Induktionsanfang wird die Aussage $\operatorname{A}(n_0)$ für
eine kleinste Zahl $n_0$
hergeleitet.

\item Im Induktionsschritt wird für ein beliebiges $n > n_0$ die Aussage
$\operatorname{A}(n)$ aus der Aussage $\operatorname {A} (n-1)$
hergeleitet.

\end{enumerate}

\noindent
Oder weniger „mathematisch“ formuliert:

\begin{enumerate}
\item Induktionsanfang: Es wird bewiesen, dass die Aussage für die
kleinste Zahl, den Startwert, gilt.

\item Induktionsschritt: Folgendes wird bewiesen: Gilt die Aussage für
eine beliebige Zahl, so gilt sie auch für die Zahl eins größer.
\end{enumerate}

\noindent
Ausgehend vom Beweis für den Startwert erledigt der Induktionsschritt
den Beweis für alle natürlichen Zahlen oberhalb des Startwertes.

%-----------------------------------------------------------------------
%
%-----------------------------------------------------------------------

\section{Beispiele}

\subsection{Gaußsche Summenformel}

Die Gaußsche Summenformel lautet: Für alle natürlichen Zahlen $n \geq 1$
gilt

\begin{flalign*}
\operatorname{A}(n):
\hspace{1cm}
1+2+\cdots+n = \frac{n(n+1)}{2}&&\\
\end{flalign*}

\noindent
Sie kann durch vollständige Induktion bewiesen werden.
Der Induktionsanfang ergibt sich unmittelbar:

\begin{flalign*}
\operatorname{A}(1):
\hspace{1cm}
1 = \frac{1(1+1)}{2}&&\\
\end{flalign*}

\noindent
Im Induktionsschritt ist zu zeigen, dass aus der Induktionsvoraussetzung

\begin{flalign*}
\operatorname{A}(n):
\hspace{1cm}
1+2+\cdots+n = \frac{n(n+1)}{2}&&\\
\end{flalign*}

\noindent
die Induktionsbehauptung

\begin{flalign*}
\operatorname{A}(n+1):
\hspace{1cm}
1+2+\cdots+n+(n+1) = \frac{(n+1)\bigl((n+1)+1\bigr)}{2} &&\text{für }n \geq 1\\
\end{flalign*}

\noindent
folgt. Dies gelingt folgendermaßen (Die Induktionsvoraussetzung ist rot
markiert.):

\begin{align*}
{\color{red} 1 + 2 + \cdots + n} + (n + 1)
& = {\color{red} \frac{n(n + 1)}{2}} + (n + 1) \\
%
&
= \frac{n(n+1)+2(n+1)}{2} &&
\text{(Hauptnenner 2)}\\
%
&
= \frac{(n+2)(n+1)}{2} &&
\text{(Ausklammern von }(n+1)\text{)} \\
%
&
= \frac{(n+1)(n+2)}{2} &&
\text{(Umdrehen nach Kommutativgesetz)} \\
%
&
= \frac{(n+1)\bigl((n+1)+1\bigr)}{2} &&
\text{(mit } (n+1) \text{ an der Stelle von } n \text{)}\\
\end{align*}

\bigskip

\noindent
Schließlich der Induktionsschluss: Damit ist die Aussage
$\operatorname{A}(n)$ für alle $n \geq 1$ bewiesen.

\section{Summe ungerader Zahlen (Maurolicus 1575)}

Die schrittweise Berechnung der Summe der ersten $ n $ ungeraden Zahlen
legt die Vermutung nahe: Die Summe aller ungeraden Zahlen von $1$ bis
$2n-1$ ist gleich dem Quadrat von $n$:

$1 = 1$

$1 + 3 = 4$

$1 + 3 + 5 = 9$

$1 + 3 + 5 + 7 = 16$

\bigskip

\noindent
Der zu beweisende allgemeine Satz lautet: $\sum\limits^n_{i=1} (2i-1) = n^2$.

\liPseudoUeberschrift{Induktionsanfang}

\begin{flalign*}
\operatorname{A}(1):
\hspace{1cm}
\sum\limits^1_{i=1} (2i-1) = 2 \cdot 1 - 1 = 1 = 1^2&&\\
\end{flalign*}

\liPseudoUeberschrift{Induktionsvoraussetzung}

\begin{flalign*}
\operatorname{A}(n):
\hspace{1cm}
\sum\limits^n_{i=1} (2i-1) = 1 + 3 + \cdots + (2n - 1) = n^2&&\\
\end{flalign*}

\liPseudoUeberschrift{Induktionsbehauptung}

\begin{flalign*}
\operatorname{A}(n+1):
\hspace{1cm}
\sum\limits^{n+1}_{i=1} (2i-1) = (n+1)^2&&\\
\end{flalign*}

\liPseudoUeberschrift{Beweis}

\noindent
Er ergibt sich über folgende Gleichungskette, bei der in der zweiten
Umformung die Induktionsvoraussetzung angewandt wird:

{\footnotesize
\begin{align*}
\sum^{n+1}_{i=1} (2i-1)
  &= {\color{red}1 + 3 + \cdots + (2n - 1)} + (2(n+1)-1) && \text{Formel für die letzte Zahl ist: } 2n - 1 \text{, } n \text{ ist hier } n + 1\\
  &= {\color{red} \sum^n_{i=1} (2i-1)} + (2(n+1)-1) && \text{andere Schreibweise mit dem Summenzeichen}\\
  &= {\color{red} n^2} + 2(n + 1) - 1 && \text{Ersetzen des Summenzeichens mit dem Ergebnis der Formel}\\
  &= {\color{red} n^2} + 2n + 2 - 1 && \text{ausmultiplizieren}\\
  &= {\color{red} n^2} + 2n + 1 && \text{mit erster Binomischer Formel: } (a+b)^{2}=a^{2}+2ab+b^{2} \\
  &= (n+1)^2
\end{align*}
}
(Die Induktionsvoraussetzung ist rot markiert.)

%%%%%%%%%%%%%%%%%%%%%%%%%%%%%%%%%%%%%%%%%%%%%%%%%%%%%%%%%%%%%%%%%%%%%%%%
% Aufgaben
%%%%%%%%%%%%%%%%%%%%%%%%%%%%%%%%%%%%%%%%%%%%%%%%%%%%%%%%%%%%%%%%%%%%%%%%

\chapter{Aufgaben}

%-----------------------------------------------------------------------
%
%-----------------------------------------------------------------------

\ExamensAufgabeTA 66115 / 2014 / 03 : Thema 1 Aufgabe 1

%-----------------------------------------------------------------------
%
%-----------------------------------------------------------------------

\ExamensAufgabeTA 66112 / 2003 / 09 : Thema 2 Aufgabe 5

%-----------------------------------------------------------------------
%
%-----------------------------------------------------------------------

\ExamensAufgabeTA 66115 / 2017 / 03 : Thema 1 Aufgabe 4

\literatur
\end{document}
