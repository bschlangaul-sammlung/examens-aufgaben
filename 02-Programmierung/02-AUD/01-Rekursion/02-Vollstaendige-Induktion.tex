\documentclass{lehramt-informatik}
\InformatikPakete{syntax,mathe}

\begin{document}

%%%%%%%%%%%%%%%%%%%%%%%%%%%%%%%%%%%%%%%%%%%%%%%%%%%%%%%%%%%%%%%%%%%%%%%%
% Theorie-Teil
%%%%%%%%%%%%%%%%%%%%%%%%%%%%%%%%%%%%%%%%%%%%%%%%%%%%%%%%%%%%%%%%%%%%%%%%

\chapter{Vollständige Induktion}

\begin{quellen}
\item \cite[Seite 115-120]{meinel}
\item \cite{wiki:vollständige-induktion}
\item \cite[Seite 22-27 (PDF 20-25)]{aud:fs:1}
\end{quellen}

\noindent
Die vollständige Induktion ist eine \memph{mathematische Beweismethode},
nach der eine Aussage für alle \memph{natürlichen Zahlen} bewiesen wird,
die \memph{größer oder gleich einem bestimmten Startwert} sind. Da es
sich um unendlich viele Zahlen handelt, kann eine Herleitung nicht für
jede Zahl einzeln erbracht werden.

Der Beweis, dass die Aussage
$\operatorname{A}(n)$ für alle
$n\geq n_{0}$
($n_0$ meist $1$ oder $0$) gilt, wird daher in zwei Etappen durchgeführt:

\begin{enumerate}
\item Im Induktionsanfang wird die Aussage $\operatorname{A}(n_0)$ für
eine kleinste Zahl $n_0$
hergeleitet.

\item Im Induktionsschritt wird für ein beliebiges $n > n_0$ die Aussage
$\operatorname{A}(n)$ aus der Aussage $\operatorname {A} (n-1)$
hergeleitet.

\end{enumerate}

\noindent
Oder weniger „mathematisch“ formuliert:

\begin{enumerate}
\item Induktionsanfang: Es wird bewiesen, dass die Aussage für die
kleinste Zahl, den Startwert, gilt.

\item Induktionsschritt: Folgendes wird bewiesen: Gilt die Aussage für
eine beliebige Zahl, so gilt sie auch für die Zahl eins größer.
\end{enumerate}

\noindent
Ausgehend vom Beweis für den Startwert erledigt der Induktionsschritt
den Beweis für alle natürlichen Zahlen oberhalb des Startwertes.

%-----------------------------------------------------------------------
%
%-----------------------------------------------------------------------

\section{Beispiele}

\subsection{Gaußsche Summenformel}

Die Gaußsche Summenformel lautet: Für alle natürlichen Zahlen $n \geq 1$
gilt

\begin{flalign*}
\operatorname{A}(n):
\hspace{1cm}
1+2+\cdots+n = \frac{n(n+1)}{2}&&\\
\end{flalign*}

\noindent
Sie kann durch vollständige Induktion bewiesen werden.
Der Induktionsanfang ergibt sich unmittelbar:

\begin{flalign*}
\operatorname{A}(1):
\hspace{1cm}
1 = \frac{1(1+1)}{2}&&\\
\end{flalign*}

\noindent
Im Induktionsschritt ist zu zeigen, dass aus der Induktionsvoraussetzung

\begin{flalign*}
\operatorname{A}(n):
\hspace{1cm}
1+2+\cdots+n = \frac{n(n+1)}{2}&&\\
\end{flalign*}

\noindent
die Induktionsbehauptung

\begin{flalign*}
\operatorname{A}(n+1):
\hspace{1cm}
1+2+\cdots+n+(n+1) = \frac{(n+1)\bigl((n+1)+1\bigr)}{2} &&\text{für }n \geq 1\\
\end{flalign*}

\noindent
folgt. Dies gelingt folgendermaßen (Die Induktionsvoraussetzung ist rot
markiert.):

\begin{align*}
{\color{red} 1 + 2 + \cdots + n} + (n + 1)
& = {\color{red} \frac{n(n + 1)}{2}} + (n + 1) \\
%
&
= \frac{n(n+1)+2(n+1)}{2} &&
\text{(Hauptnenner 2)}\\
%
&
= \frac{(n+2)(n+1)}{2} &&
\text{(Ausklammern von }(n+1)\text{)} \\
%
&
= \frac{(n+1)(n+2)}{2} &&
\text{(Umdrehen nach Kommutativgesetz)} \\
%
&
= \frac{(n+1)\bigl((n+1)+1\bigr)}{2} &&
\text{(mit } (n+1) \text{ an der Stelle von } n \text{)}\\
\end{align*}

\bigskip

\noindent
Schließlich der Induktionsschluss: Damit ist die Aussage
$\operatorname{A}(n)$ für alle $n \geq 1$ bewiesen.

\section{Summe ungerader Zahlen (Maurolicus 1575)}

Die schrittweise Berechnung der Summe der ersten $ n $ ungeraden Zahlen
legt die Vermutung nahe: Die Summe aller ungeraden Zahlen von $1$ bis
$2n-1$ ist gleich dem Quadrat von $n$:

$1 = 1$

$1 + 3 = 4$

$1 + 3 + 5 = 9$

$1 + 3 + 5 + 7 = 16$

\bigskip

\noindent
Der zu beweisende allgemeine Satz lautet: $\sum\limits^n_{i=1} (2i-1) = n^2$.

\ueberschrift{Induktionsanfang}

\begin{flalign*}
\operatorname{A}(1):
\hspace{1cm}
\sum\limits^1_{i=1} (2i-1) = 2 \cdot 1 - 1 = 1 = 1^2&&\\
\end{flalign*}

\ueberschrift{Induktionsvoraussetzung}

\begin{flalign*}
\operatorname{A}(n):
\hspace{1cm}
\sum\limits^n_{i=1} (2i-1) = 1 + 3 + \cdots + (2n - 1) = n^2&&\\
\end{flalign*}

\ueberschrift{Induktionsbehauptung}

\begin{flalign*}
\operatorname{A}(n+1):
\hspace{1cm}
\sum\limits^{n+1}_{i=1} (2i-1) = (n+1)^2&&\\
\end{flalign*}

\ueberschrift{Beweis}

\noindent
Er ergibt sich über folgende Gleichungskette, bei der in der zweiten
Umformung die Induktionsvoraussetzung angewandt wird:

{\footnotesize
\begin{align*}
\sum^{n+1}_{i=1} (2i-1)
  &= {\color{red}1 + 3 + \cdots + (2n - 1)} + (2(n+1)-1) && \text{Formel für die letzte Zahl ist: } 2n - 1 \text{, } n \text{ ist hier } n + 1\\
  &= {\color{red} \sum^n_{i=1} (2i-1)} + (2(n+1)-1) && \text{andere Schreibweise mit dem Summenzeichen}\\
  &= {\color{red} n^2} + 2(n + 1) - 1 && \text{Ersetzen des Summenzeichens mit dem Ergebnis der Formel}\\
  &= {\color{red} n^2} + 2n + 2 - 1 && \text{ausmultiplizieren}\\
  &= {\color{red} n^2} + 2n + 1 && \text{mit erster Binomischer Formel: } (a+b)^{2}=a^{2}+2ab+b^{2} \\
  &= (n+1)^2
\end{align*}
}
(Die Induktionsvoraussetzung ist rot markiert.)

%%%%%%%%%%%%%%%%%%%%%%%%%%%%%%%%%%%%%%%%%%%%%%%%%%%%%%%%%%%%%%%%%%%%%%%%
% Aufgaben
%%%%%%%%%%%%%%%%%%%%%%%%%%%%%%%%%%%%%%%%%%%%%%%%%%%%%%%%%%%%%%%%%%%%%%%%

\chapter{Aufgaben}

%-----------------------------------------------------------------------
%
%-----------------------------------------------------------------------

\section{Aufgabe 1: „Rekursion und Induktion“}

\begin{quellen}
\item \cite[Seite 25]{aud:fs:1}
\item \cite[Seite 2-3, Thema 1, Aufgabe 1b]{examen:66115:2014:03}
\end{quellen}

\begin{enumerate}
\item Gegeben sei die Methode \verb|BigInteger lfBig(int n)| zur
Berechnung der eingeschränkten Linksfakultät:

\begin{minted}{java}
import java.math.Biginteger;
import static java.math.BigInteger.*;

public class LeftFactorial {
  // returns the left factorial !n
  BigInteger lfBig(int n) {
    if (n <= 0 || n >= Short.MAX_VALUE) {
      return ZERO;
    } else if (n == 1) {
    return ONE;
    } else {
      return sub(mul(n, lfBig(n - 1)), mul(n - 1, lfBig(n - 2)));
    }
  }
}
\end{minted}

Implementieren Sie unter Verwendung des Konzeptes der \emph{dynamischen
Programmierung} die Methode \verb|BigInteger dpBig(int n)|, die jede
$!n$
auch bei mehrfachem Aufrufen mit dem gleichen Parameter höchstens einmal
rekursiv berechnet. Sie dürfen der Klasse \verb|LeftFactorial| genau ein
Attribut beliebigen Datentyps hinzufügen und die in \verb|lfBig(int)|
verwendeten Methoden und Konstanten ebenfalls nutzen.

\item Betrachten Sie nun die Methode \verb|lfLong(int)| zur Berechnung
der vorangehend definierten Linksfakultät ohne obere Schranke. Nehmen
Sie im Folgenden an, dass der Datentyp \verb|long| unbeschränkt ist und
daher kein Überlauf auftritt.

\begin{minted}{java}
long lfLong(int n) {
  if (n <= 0) {
    return 0;
  } else if (n == 1) {
    return 1;
  } else {
    return n * lfLong(n - 1) - (n - 1) * lfLong(n - 2);
  }
}
\end{minted}

Beweisen Sie \emph{formal} mittels \emph{vollständiger Induktion}:

\def\lf#1{\text{lfLong}(#1)}
\def\sk#1{\sum^{#1}_{k=0}k!}

\begin{displaymath}
\forall n \geq 0: \lf{n} \equiv \sk{n-1}
\end{displaymath}

\begin{antwort}

IA:

\begin{displaymath}
n=1 \Rightarrow
\lf{1} =
1 =
\sk{n-1} =
0! =
1
\end{displaymath}

\begin{displaymath}
n=2 \Rightarrow
\lf{2} =
2 \cdot \lf{1} - 1 \cdot \lf{0} =
2 =
\sk{1} =
1! + 0! =
1 + 1 =
2
\end{displaymath}

IV:

\begin{displaymath}
\lf{n} = \sk{n-1}
\end{displaymath}

gilt:

IS:

\begin{equation}
\begin{split}
\lf{n+1} & = (n+1) \cdot \lf{n} - n \cdot \lf{n-1} \\
& = (n+1) \cdot \sk{n-1} - n \cdot \sk{n-2} \\
& = (n+1) \cdot \Big((n-1)! + \sk{n-2}\Big) - n \cdot \sk{n-2} \\
& = (n+1)(n-1)! + (n+1) \cdot \sk{n-2} - n \cdot \sk{n-2} \\
& = (n+1)(n-1)! \cdot \sk{n-2} + n \cdot \sk{n-2} - n \cdot \sk{n-2} \\
& = (n+1)(n-1)! + \sk{n-2} \\
& = n \cdot (n-1)! + (n-1)! + \sk{n-2} \\
& = n \cdot (n-1)! + \sk{n-1} \\
& = n! + \sk{n-1} \\
& = \sk{n} \\
& = \sk{(n+1)-1}
\end{split}
\end{equation}
\end{antwort}
\end{enumerate}

%-----------------------------------------------------------------------
%
%-----------------------------------------------------------------------

\section{Herbst 2003, Thema 2 Aufgabe 5}

5. a) Zeigen Sie mit Hilfe vollständiger Induktion, dass das folgende
Programm bzgl. der Vorbedingung $x > 0$ und der Nachbedingung drei\_hoch
$x =  3^x$ partiell korrekt ist!
\footcite[Thema 2 Aufgabe 5]{examen:66112:2003:09}

\begin{minted}{lisp}
(define (drei_hoch x)
  (cond ((= x 0) 1)
    (else (* 3 (drei_hoch (- x 1))))
  )
)
\end{minted}

\begin{antwort}
% IA: x=1
% drei_hoch 1 = 3*(drei_hoch 0) = 3*1=3 = 3 1
% IV: für alle x < x 0 gilt drei_hoch x = 3 x
% IS: x->x+1

\begin{align*}
\text{drei\_hoch} (x + 1)
& = 3 \cdot (\text{drei\_hoch} (- (x + 1) 1))\\
& = 3 \cdot (\text{drei\_hoch} x)\\
& = 3 \cdot 3^x\\
& = 3^{x+1}
\end{align*}
\end{antwort}

%-----------------------------------------------------------------------
%
%-----------------------------------------------------------------------

\section{Aufgabe 4: Vollständige Induktion\footcite{sosy:ab:8}}

Sie dürfen im Folgenden davon ausgehen, dass keinerlei Under- oder
Overflows auftreten.\footcite[nach Frühjahr 2017 (66115) - Thema 1,
Aufgabe 4]{examen:66115:2017:03}

\noindent
Gegeben sei folgende rekursive Methode für $n \geq 0$:

\begin{minted}{java}
long sumOfSquares (long n) {
  if (n == 0)
    return 0;
  else
    return n * n + sumOfSquares(n - 1);
}
\end{minted}

\begin{enumerate}

%%
% (a)
%%

\item Beweisen Sie formal mittels vollständiger Induktion:

\begin{displaymath}
\forall n \in \mathbb{N} : \texttt{sumOfSquares(n)} =
\frac{n(n + 1)(2n + 1)}{6}
\end{displaymath}

\begin{antwort}
Sei $f(n): \frac{n(n + 1)(2n + 1)}{6}$

%%
%
%%

\ueberschrift{Induktionsanfang}

Für $n = 0$ gilt:

$\texttt{sumOfSquares(0)} \overset{\texttt{if}}{=} 0 = f(0)$

%%
%
%%

\ueberschrift{Induktionshypothese}

Für ein festes $n \in \mathbb{N}$ gelte:

$\texttt{sumOfSquares(0)} = f(n)$

%%
%
%%

\ueberschrift{Induktionsschritt}

% https://mathcs.org/analysis/reals/infinity/answers/sm_sq_cb.html

$n \rightarrow n + 1$

$\texttt{sumOfSquares(n+1)} \overset{\texttt{else}}{=}$

$\texttt{(n+1)*(n+1)*sumOfSquares(n)} \overset{\texttt{I.H.}}{=}$

$(n + 1) \cdot (n + 1) + f(n)$

$(n + 1)^2 + \frac{n(n + 1)(2n + 1)}{6}$

$\frac{6(n + 1)^2}{6} + \frac{n(n + 1)(2n + 1)}{6}$

$\frac{6(n + 1)^2 + n(n + 1)(2n + 1)}{6}$

$\frac{(n + 1) \cdot (6(n + 1) + n(2n + 1))}{6}$

$\frac{(n + 1) \cdot (6n + 6 + 2n^2 + n))}{6}$

$\frac{(n + 1) \cdot (2n^2 + 7n + 6))}{6}$
$\frac{(n + 1) \cdot (n + 2) (2n + 3))}{6}$

Neben $2n^2 + 7n + 6 = (n + 2) (2n + 3) = n \cdot 2n + 2 \cdot 2n + 3 \cdot n + 2 \cdot 6$

% $\frac{6(n^2 + 2 \cdot n \cdot 1 + 1^2) + n(n + 1)(2n + 1)}{6}$

% $\frac{6n^2 + 12n + 6 + n(n + 1)(2n + 1)}{6}$

% $\frac{6n^2 + 12n + 6 + (n^2 + n)(2n + 1)}{6}$

% $\frac{6n^2 + 12n + 6 + (n^2 + n)2n + (n^2 + n)1}{6}$

% $\frac{6n^2 + 12n + 6 + 2n^3 + 2n^2 + n^2 + n}{6}$

% $\frac{6n^2 + 13n + 6 + 2n^3 + 2n^2 + n^2 + n}{6}$

% $\frac{9n^2 + 6 + 2n^3 + n}{6}$

\end{antwort}

%%
% (b)
%%

\item Beweisen Sie die Terminierung von \java{sumOfSquares(n)} für alle
$n \geq 0$.

\begin{antwort}
Sei $T (n) = n$. Die Funktion $T (n)$ ist offenbar ganzzahlig. In jedem
Rekursionsschritt wird $n$ um eins verringert, somit ist $T (n)$ streng
monoton fallend. Durch die Abbruchbedingung \java{n==0} ist $T (n)$
insbesondere nach unten beschränkt. Somit ist $T$ eine gültige
Terminierungsfunktion.
\end{antwort}

\end{enumerate}

\literatur
\end{document}
