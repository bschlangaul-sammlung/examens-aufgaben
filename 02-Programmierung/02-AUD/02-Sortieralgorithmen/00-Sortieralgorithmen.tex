\documentclass{lehramt-informatik-haupt}
\liLadePakete{sortieren,quicksort,syntax}
\usepackage{multicol}
\begin{document}

%%%%%%%%%%%%%%%%%%%%%%%%%%%%%%%%%%%%%%%%%%%%%%%%%%%%%%%%%%%%%%%%%%%%%%%%
% Theorie-Teil
%%%%%%%%%%%%%%%%%%%%%%%%%%%%%%%%%%%%%%%%%%%%%%%%%%%%%%%%%%%%%%%%%%%%%%%%

\chapter{Sortieralgorithmen}

\begin{quellen}
\cite{wiki:sortierverfahren}
\end{quellen}

%%%%%%%%%%%%%%%%%%%%%%%%%%%%%%%%%%%%%%%%%%%%%%%%%%%%%%%%%%%%%%%%%%%%%%%%
% Aufgaben
%%%%%%%%%%%%%%%%%%%%%%%%%%%%%%%%%%%%%%%%%%%%%%%%%%%%%%%%%%%%%%%%%%%%%%%%

\chapter{Aufgaben}

\section{Händisches Sortieren\footcite[Seite 1]{aud:pu:1}}

Gegeben sei ein Array $a$, welches die Werte $5,7,9,3,6,1,2,8$ enthält.
Sortieren Sie das Array händisch mit:

\begin{enumerate}

%%
% (a)
%%

\item Bubblesort

erster Durchgang a

\sortieren{5 7 9 >3 <6 1 2 8}

\sortieren{5 7 9 6 >3 <1 2 8}

\sortieren{5 7 9 6 1 >3 <2 8}

Zweiter Durchgang

\sortieren{5 7 9 >6 <1 2 3 8}

\sortieren{5 7 9 1 >6 <2 3 8}

\sortieren{5 7 9 1 2 >6 <3 8}

Dritter Durchgang

\sortieren{5 7 >9 <1 2 3 6 8}

\sortieren{5 7 1 >9 <2 3 6 8}

\sortieren{5 7 1 2 >9 <3 6 8}

\sortieren{5 7 1 2 3 >9 <6 8}

\sortieren{5 7 1 2 3 6 >9 <8}

Vierter Durchgang

\sortieren{5 >7 <1 2 3 6 8 9}

\sortieren{5 1 >7 <2 3 6 8 9}

\sortieren{5 1 2 >7 <3 6 8 9}

\sortieren{5 1 2 3 >7 <6 8 9}

Fünfter Durchgang

\sortieren{>5 <1 2 3 6 7 8 9}

\sortieren{1 >5 <2 3 6 7 8 9}

\sortieren{1 2 >5 <3 6 7 8 9}

fertig

\sortieren{1 2 3 5 6 7 8 9}

%%
% (b)
%%

\item Mergesort

\begin{forest}
  /tikz/arrows=->, /tikz/>=latex, /tikz/nodes={draw},
  for tree={delay={sort}}, sort level=2
[5 7 9 3 6 1 2 8
  [5 7 9 3
    [5 7
      [5]
      [7]
    ]
    [9 3
      [9]
      [3]
    ]
  ]
  [6 1 2 8
    [6 1
      [6]
      [1]
    ]
    [2 8
      [2]
      [8]
    ]
  ]
]
%
\coordinate (m) at (!|-!\forestOnes);
\myNodes
\end{forest}

%%
% (c)
%%

\item Quicksort

\QSinitialize{5,7,9,3,6,1,2,8}

\loop
\QSpivotStep
\ifnum\value{pivotcount}>0
  \QSsortStep
\repeat

\end{enumerate}

%-----------------------------------------------------------------------
%
%-----------------------------------------------------------------------

\section{Aufgabe 2: Sortieren\footcite{aud:ab:2}}

Für diese Aufgabe wird die Vorlage Sortieralgorithmen benötigt, die auf
dem Beiblatt genauer erklärt wird.\footnote{Diese Aufgabe stammt aus dem
Übungsblatt 1 zu Algorithmen und Datenstrukturen von Prof. Dr. Martin
Hennecke und Rainer Gall an der Universität Würzburg und wurde
dankenswerterweise zur Verwendung in diesem Aufgabenblatt zur Verfügung
gestellt.}

Die fertigen Methoden sollen in der Lage sein, beliebige Arrays zu
sortieren. Im gelben Textfeld des Eingabefensters soll dabei wieder
ausführlich und nachvollziehbar angezeigt werden, wie die jeweilige
Methode vorgeht. Beispielsweise so:

{
\tiny
\begin{multicols}{2}
\begin{verbatim}
Führe die Methode selectionSort() aus:
Sortiere dieses Feld: 5, 3, 17, 7, 42, 23
Der Marker liegt bei: 5
Das Maximum liegt bei: 4
Diese beiden Elemente werden nun vertauscht.
Ergebnis dieser Runde: 5, 3, 17, 7, 23, 42
Der Marker liegt bei: 4
Das Maximum liegt bei: 4
Diese beiden Elemente werden nun vertauscht.
Ergebnis dieser Runde: 5, 3, 17, 7, 23, 42
Der Marker liegt bei: 3
Das Maximum liegt bei: 2
Diese beiden Elemente werden nun vertauscht.
Ergebnis dieser Runde: 5, 3, 7, 17, 23, 42
Der Marker liegt bei: 2
Das Maximum liegt bei: 2
Diese beiden Elemente werden nun vertauscht.
Ergebnis dieser Runde: 5, 3, 7, 17, 23, 42
Der Marker liegt bei: 1
Das Maximum liegt bei: 0
Diese beiden Elemente werden nun vertauscht.
Ergebnis dieser Runde: 3, 5, 7, 17, 23, 42
Der Marker liegt bei: 0
Das Maximum liegt bei: 0
Diese beiden Elemente werden nun vertauscht.
Ergebnis dieser Runde: 3, 5, 7, 17, 23, 42
\end{verbatim}

\begin{verbatim}
Führe die Methode bubbleSort() aus:
Sortiere dieses Feld: 5, 3, 17, 7, 42, 23
Das Element an der Stelle 0 ist größer als sein Nachfolger.
Diese beiden werden nun vertauscht.
Ergebnis dieser Runde: 3, 5, 17, 7, 42, 23
Das Element an der Stelle 2 ist größer als sein Nachfolger.
Diese beiden werden nun vertauscht.
Ergebnis dieser Runde: 3, 5, 7, 17, 42, 23
Das Element an der Stelle 4 ist größer als sein Nachfolger.
Diese beiden werden nun vertauscht.
Ergebnis dieser Runde: 3, 5, 7, 17, 23, 42
\end{verbatim}
\end{multicols}
}

\begin{enumerate}

%%
% (a)
%%

\item Vervollständige die Methode \java{selectionSort()}.

\begin{antwort}
\inputcode[firstline=65,lastline=91]{aufgaben/aud/ab_2/Sortieralgorithmen}
\end{antwort}

%%
% (b)
%%

\item Vervollständige die Methode \java{bubbleSort()}.

\begin{antwort}
\inputcode[firstline=93,lastline=118]{aufgaben/aud/ab_2/Sortieralgorithmen}
\end{antwort}
\end{enumerate}

%-----------------------------------------------------------------------
%
%-----------------------------------------------------------------------

\section{Aufgabe 3: Sortieren I\footcite[entnommen aus Algorithmen und
Datenstrukturen, Übungsblatt 2, Universität Würzburg]{aud:pu:7}}

Gegeben ist folgende Zahlenfolge:

35, 22, 5, 3, 28, 16, 8, 60, 17, 66, 4, 9, 82, 11, 10, 20

\begin{enumerate}
\item Sortiere Sie händisch mit Mergesort. Orientieren Sie sich beim
Aufschreiben der Zwischenschritte an dieser Darstellung:

\item Sortiere Sie händisch mit Quicksort. Wählen Sie als Pivot-Element
immer das Element in der Mitte - oder gegebenenfalls das Element direkt
links neben der Mitte. Orientieren Sie sich beim Aufschreiben der
Zwischenschritte an dieser Darstellung:

\end{enumerate}

\literatur

\end{document}
