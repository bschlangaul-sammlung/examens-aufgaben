\documentclass{lehramt-informatik}
\InformatikPakete{syntax}
\usepackage{multicol}
\begin{document}

%%%%%%%%%%%%%%%%%%%%%%%%%%%%%%%%%%%%%%%%%%%%%%%%%%%%%%%%%%%%%%%%%%%%%%%%
% Theorie-Teil
%%%%%%%%%%%%%%%%%%%%%%%%%%%%%%%%%%%%%%%%%%%%%%%%%%%%%%%%%%%%%%%%%%%%%%%%

\chapter{Suchalgorithmen}

%-----------------------------------------------------------------------
%
%-----------------------------------------------------------------------

\section{Suchen Allgemein}

Eine der wichtigsten und häufigsten Aufgaben in der Informatik ist das
Suchen.
%
Die Suche in unsortierter Datenmenge ist langwierig. Es gibt keine
Möglichkeit, das ideal zu gestalten.
%
Das Suchen in sortierten Daten ist sinnvoll, denn es kann optimiert
werden.

\subsection{Möglichkeiten:}

\begin{description}
\item[Sequenzielle Suche:] Es wird eine Folge vom ersten Element an
durchlaufen. Die Suche ist beendet, wenn das gesuchte Element gefunden
ist oder die gesamte Folge ohne Ergebnis durchlaufen wurde.

\item[Binäre Suche:] Wir „verkleinern“ die zu durchsuchenden Menge durch
Herausgreifen eines Elements aus der Folge und vergleichen, ob mein
gesuchtes Element vor oder nach diesem Element in der Folge liegt. Es
muss nur noch eine Hälfte durchsucht werden. In dieser fahren wir analog
fort.
\end{description}

Ein Bewertungskriterium für das Suchverfahren ist der
Berechnungsaufwand. Wie viele Schritte sind durchschnittlich nötig, um
eine Folge zu durchlaufen?\footcite[Seite 16 (PDF 8)]{aud:fs:2}

%-----------------------------------------------------------------------
%
%-----------------------------------------------------------------------

\section{Binäre Suche}

Die binäre Suche ist ein Algorithmus, der auf einem Feld (also meist „in
einer Liste“) sehr effizient ein gesuchtes Element findet bzw. eine
zuverlässige Aussage über das Fehlen dieses Elementes liefert.
Voraussetzung ist, dass die Elemente in dem Feld \emph{sortiert} sind.
Der Algorithmus basiert auf einer einfachen Form des Schemas \emph{Teile
und Herrsche}, zugleich stellt er auch einen \emph{Greedy-Algorithmus}
dar. Ordnung und spätere Suche müssen sich auf denselben Schlüssel
beziehen.
\footcite{wiki:binaere-suche}

\cite[Seite 17 (PDF 9)]{aud:fs:2}

\section{Interativer Ansatz}

\begin{minted}{java}
public int binsearch(int[] a, int lo, int hi, int x){
  while (lo <= hi) {
    int m = lo + (hi - lo) / 2;
    if (x < a[m]){
      hi = m - 1;
    } else if (x > a[m]) {
      lo = m + 1;
    } else {
      return m;
    }
  }
  return -1;
}
\end{minted}

\section{Rekursiver Ansatz}

\begin{minted}{java}
public int binsearch(int[] a, int lo, int hi, int x){
  if (lo > hi){
    return -1;
  }
  int m = lo + (hi - lo) / 2;
  if (x < a[m]) {
    return binsearch(a, lo, m - 1, x);
  }
  if (x > a[m]) {
    return binsearch(a, m + 1, hi, x);
  }
  return m;
}
\end{minted}

%%%%%%%%%%%%%%%%%%%%%%%%%%%%%%%%%%%%%%%%%%%%%%%%%%%%%%%%%%%%%%%%%%%%%%%%
% Aufgaben
%%%%%%%%%%%%%%%%%%%%%%%%%%%%%%%%%%%%%%%%%%%%%%%%%%%%%%%%%%%%%%%%%%%%%%%%

\chapter{Aufgaben}

\section{Aufgabe 3: Suchen\footcite{aud:ab:2}}

Für diese Aufgabe wird die Vorlage Suchalgorithmen benötigt, die auf dem
Beiblatt genauer erklärt wird.\footcite{Diese Aufgabe stammt aus dem
Übungsblatt 1 zu Algorithmen und Datenstrukturen von Prof. Dr. Martin
Hennecke und Rainer Gall an der Universität Würzburg und wurde
dankenswerterweise zur Verwendung in diesem Aufgabenblatt zur Verfügung
gestellt.}

Vervollständigen Sie die Methode \java{sucheBinaer()}. Die fertige
Methode soll in der Lage sein, beliebige Werte in beliebigen sortierten
Arrays zu suchen. Im gelben Textfeld des Eingabefensters soll dabei
ausführlich und nachvollziehbar angezeigt werden, wie die Methode
vorgeht. Beispielsweise so:

{
\tiny
\begin{multicols}{2}
\begin{verbatim}
Führe die Methode sucheSequenziell() aus:
Suche in diesem Feld: 3, 5, 7, 17, 42, 23
Überprüfen des Werts an der Position 0
Überprüfen des Werts an der Position 1
Überprüfen des Werts an der Position 2
Überprüfen des Werts an der Position 3
Fertig :-)
\end{verbatim}

\begin{verbatim}
Führe die Methode sucheBinaer() aus:
Suche in diesem Feld: 3, 5, 7, 17, 42, 23
Suchbereich: 0 bis 5
Mitte: 2, also neuer Bereich: 3 bis 5
Mitte: 4, also neuer Bereich: 3 bis 3
Mitte: 3, Treffer : -)
\end{verbatim}
\end{multicols}
}

\begin{enumerate}
\item Sequenzielle Suche
\begin{antwort}
\inputcode[firstline=22,lastline=44]{aufgaben/aud/ab_2/Suchalgorithmen}
\end{antwort}

\item Binäre Suche

\begin{antwort}
\inputcode[firstline=46,lastline=76]{aufgaben/aud/ab_2/Suchalgorithmen}
\end{antwort}

\end{enumerate}

\literatur

\end{document}
