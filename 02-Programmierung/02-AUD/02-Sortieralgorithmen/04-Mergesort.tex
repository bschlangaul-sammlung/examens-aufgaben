\documentclass{lehramt-informatik-haupt}
\liLadePakete{syntax,sortieren,mathe}

\begin{document}

%%%%%%%%%%%%%%%%%%%%%%%%%%%%%%%%%%%%%%%%%%%%%%%%%%%%%%%%%%%%%%%%%%%%%%%%
% Theorie-Teil
%%%%%%%%%%%%%%%%%%%%%%%%%%%%%%%%%%%%%%%%%%%%%%%%%%%%%%%%%%%%%%%%%%%%%%%%

\chapter{MergeSort: Sortieren durch Verschmelzen}

\begin{quellen}
\item \cite[Seite 50]{aud:fs:tafeluebung-11}
\item \cite[Seite 131-135(PDF 149-153)]{saake}
\item \cite{wiki:mergesort}
\item \cite[6.4.2 Effiziente Sortierverfahren, Seite 192]{schneider}
\end{quellen}

\begin{center}
\def\myNodes{}
\begin{forest}
  /tikz/arrows=->, /tikz/>=latex, /tikz/nodes={draw},
  for tree={delay={sort}}, sort level=2
[38 27 43 3 9 82 10
  [38 27 43 3
    [38 27 [39][27]]
    [43 3 [43][3]]
  ]
  [9 82 10
    [9 82 [9] [82]]
    [10 [10]]
  ]
]
%
\coordinate (m) at (!|-!\forestOnes);
\myNodes
\end{forest}
\end{center}

\begin{itemize}
\item Funtionsweise:

\begin{itemize}
\item Listen mit maximal \emph{einem} Element sind \emph{trivialerweise sortiert}
\item falls zu sortierende Liste mehr als ein Element beinhaltet:

\begin{itemize}
\item \emph{teile} Liste in zwei kleinere Listen auf
\item verfahre \emph{rekursiv} mit den beiden Teillisten
\item \emph{verschmelze} die zwei rekursiv sortierten Listen
\item dabei Sortierung in verschmolzener Liste beibehalten
\end{itemize}
\end{itemize}

$\rightarrow$ Teile-Und-Herrsche
\item Eigenschaften von MergeSort:

\begin{itemize}
\item Laufzeitkomplexität:
$\mathcal{O}(n \cdot log(n))$ (im Best-, Average- und Worst-Case)
\item bei geeigneter „Verschmelzung“ \emph{stabile} Sortierung
\item durch Rekursion wachsender Aufrufstapel
$\rightarrow$ \emph{out-of-place}
\end{itemize}

\end{itemize}

\inputcode[firstline=14,lastline=61]{sortier/MergeSort}
\footcite[Seite 134 (PDF 152)]{saake}

%%%%%%%%%%%%%%%%%%%%%%%%%%%%%%%%%%%%%%%%%%%%%%%%%%%%%%%%%%%%%%%%%%%%%%%%
%
%%%%%%%%%%%%%%%%%%%%%%%%%%%%%%%%%%%%%%%%%%%%%%%%%%%%%%%%%%%%%%%%%%%%%%%%

\chapter{Aufgaben}

\section{Implementierung der \java{merge}-Methode, Berechnung der
Zeitkomplexität
\footcite[Seite 2, Aufgabe 3: Mergesort]{aud:ab:7}
}

Das Sortierverfahren \emph{Mergesort}, das nach der Strategie
\emph{Divide-and-Conquer} arbeitet, sortiert eine Sequenz, indem die
Sequenz in zwei Teile zerlegt wird, die dann einzeln sortiert und wieder
zu einer sortierten Sequenz zusammengemischt werden (to merge =
zusammenmischen, verschmelzen).

\begin{enumerate}

%%
% (a)
%%

\item Gegeben seien folgende Methoden:

\inputcode[firstline=13,lastline=31]{aufgaben/aud/ab_7/mergesort/Mergesort}

Schreiben Sie die Methode \java{public int[] merge (int[] s, int[] r)},
die die beiden aufsteigend sortierten Sequenzen \java{s} und \java{r} zu
einer aufsteigend sortierten Sequenz zusammenmischt.

\begin{antwort}
\inputcode[firstline=33,lastline=60]{aufgaben/aud/ab_7/mergesort/Mergesort}
\end{antwort}

%%
% (b)
%%

\item Analysieren Sie die Zeitkomplexität von \java{mergesort}.

\begin{antwort}
$O(n \cdot \log n)$

\ueberschrift{Erklärung}

Mergesort ist ein stabiles Sortierverfahren, vorausgesetzt der
Merge-Schritt ist korrekt implementiert. Seine Komplexität beträgt im
Worst-, Best- und Average-Case in Landau-Notation ausgedrückt stets $O(n
\cdot \log n)$. Für die Laufzeit $T(n)$ von Mergesort bei $n$ zu
sortierenden Elementen gilt die Rekursionsformel

\begin{align*}
T(n) & = \\
     & T\left(\left\lfloor\frac{n}{2}\right\rfloor\right) + && \text{Aufwand, 1. Teil zu sortieren}\\
     & T\left(\left\lceil\frac{n}{2}\right\rceil\right) + && \text{Aufwand, 2. Teil zu sortieren}\\
     & \mathcal{O}(n) && \text{Aufwand, beide Teile zu verschmelzen}\\
\end{align*}

mit dem Rekursionsanfang $T(1) = 1$.

Nach dem Master-Theorem kann die Rekursionsformel durch

\begin{displaymath}
2T\left(\left\lfloor\frac{n}{2}\right\rfloor\right) + n
\end{displaymath}

bzw.

\begin{displaymath}
2T\left(\left\lceil\frac{n}{2}\right\rceil\right) + n
\end{displaymath}

approximiert werden mit jeweils der Lösung $T(n) = O(n \cdot \log n)$.
\end{antwort}

\end{enumerate}

%-----------------------------------------------------------------------
%
%-----------------------------------------------------------------------

\ExamensAufgabeTA 46115 / 2018 / 03 : Thema 1 Aufgabe 6

\literatur

\end{document}
