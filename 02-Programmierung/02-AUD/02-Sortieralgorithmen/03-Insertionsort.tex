\documentclass{lehramt-informatik}
\InformatikPakete{syntax}
\usepackage{tikz}
\usetikzlibrary{chains,fit,shapes,shapes.multipart}

\begin{document}

%%%%%%%%%%%%%%%%%%%%%%%%%%%%%%%%%%%%%%%%%%%%%%%%%%%%%%%%%%%%%%%%%%%%%%%%
% Theorie-Teil
%%%%%%%%%%%%%%%%%%%%%%%%%%%%%%%%%%%%%%%%%%%%%%%%%%%%%%%%%%%%%%%%%%%%%%%%

\chapter{InsertionSort: Sortieren durch Einfügen}

\begin{quellen}
\item \cite[Seite 41]{aud:fs:tafeluebung-11}
\item \cite{wiki:insertionsort}
\item \cite[Seite 125-127 (PDF 143-145)]{saake}
\end{quellen}

\begin{itemize}
\item Funktionsweise:

\begin{itemize}
\item solange zu sortierende Liste mehr als ein Element beinhaltet:

\begin{itemize}
\item lösche das \emph{erste Element} aus der Liste
\item füge \emph{gemäß Sortierordnung} in die Ergebnisliste ein
\item wiederhole, bis Eingangsliste leer
\end{itemize}

\end{itemize}

\item Eigenschaften von InsertionSort:

\begin{itemize}
\item Laufzeitkomplexität:

\begin{itemize}
\item $\mathcal{O}(n)$ (im Best-Case)
\item $\mathcal{O}(n^2)$ (im Average- und Worst-Case)
\end{itemize}

\item \emph{stabil}
\item bei Arrays \emph{in-situ}
\end{itemize}

\end{itemize}

\inputcode[firstline=8,lastline=25]{sortier/InsertionSort}
\footcite[Seite 125 - 127]{saake}

\section{Zustand des Eingabefelds an einem Beispiel}

\begin{minted}{md}
for: i=1
for (Anfang) 7 4 9 2 3
while        7 7 9 2 3
for (Ende)   4 7 9 2 3
for: i=2
for (Anfang) 4 7 9 2 3
for (Ende)   4 7 9 2 3
for: i=3
for (Anfang) 4 7 9 2 3
while        4 7 9 9 3
while        4 7 7 9 3
while        4 4 7 9 3
for (Ende)   2 4 7 9 3
for: i=4
for (Anfang) 2 4 7 9 3
while        2 4 7 9 9
while        2 4 7 7 9
while        2 4 4 7 9
for (Ende)   2 3 4 7 9
\end{minted}

%-----------------------------------------------------------------------
%
%-----------------------------------------------------------------------

\section{Aufgabe zum Insertionsort}

\begin{quellen}
\item \cite[Seite 1]{aud:pu:2}
\item \cite[Herbst 2017 (RS), Thema 2, A7]{examen:46115:2017:09}
\end{quellen}

\begin{enumerate}
\item Führen Sie \emph{„Sortieren durch Einfügen“} lexikographisch
aufsteigend und \emph{in-situ} (\emph{in-place}) so in einem
Schreibtischlauf auf folgendem Feld (Array) aus, dass gleiche Elemente
ihre relative Abfolge jederzeit beibehalten (also dass z.\,B. $A_1$
stets vor $A_2$ im Feld steht). Jede Zeile stellt den Zustand des Feldes
dar, nachdem das jeweils nächste Element in die Endposition verschoben
wurde. Der bereits sortierte Teilbereich steht vor |||. Gleiche Elemente
tragen zwecks Unterscheidung ihre \emph{„Objektidentität“} als Index
(z.\,B. \java{"A1".equals("A2")} aber \java{"A1" != "A2"})

\tikzset{
  sorted/.style = {
    draw,
    line width=0.6pt,
    inner sep=0mm
  },
  unsorted/.style = {
    draw,
    font=\tiny,
    rectangle split,
    rectangle split horizontal,
    rectangle split parts=6,
    text centered,
    align=center
  },
}

\def\TmpSchreibtischLauf#1#2#3#4#5#6#7{
  \noindent
  \begin{tikzpicture}
  \node (qs) [unsorted] {
  \nodepart{one} #1
  \nodepart{two} #2
  \nodepart{three} #3
  \nodepart{four} #4
  \nodepart{five} #5
  \nodepart{six} #6
  };

  \node[sorted, fit=(qs.north west) (qs.south west) (qs.#7 split)] {};
  \end{tikzpicture}
}

\begin{center}
\TmpSchreibtischLauf{$L$}{$A_1$}{$B_1$}{$F$}{$A_2$}{$B_2$}{one}

\TmpSchreibtischLauf{$A_1$}{$L$}{$B_1$}{$F$}{$A_2$}{$B_2$}{two}

\TmpSchreibtischLauf{$A_1$}{$B_1$}{$L$}{$F$}{$A_2$}{$B_2$}{three}

\TmpSchreibtischLauf{$A_1$}{$B_1$}{$F$}{$L$}{$A_2$}{$B_2$}{four}

\TmpSchreibtischLauf{$A_1$}{$A_2$}{$B_1$}{$F$}{$L$}{$B_2$}{five}

\begin{tikzpicture}
\node (qs) [unsorted] {
\nodepart{one} $A_1$
\nodepart{two} $A_2$
\nodepart{three} $B_1$
\nodepart{four} $B_2$
\nodepart{five} $F$
\nodepart{six} $L$
};

\node[sorted, fit=(qs.north west) (qs.south west) (qs.south east)] {};
\end{tikzpicture}
\end{center}

\item Ergänzen Sie die folgende Methode so, dass sie die Zeichenketten
im Feld \java{a} lexikographisch aufsteigend durch Einfügen sortiert.
Sie muss zum vorangehenden Ablauf passen, d.\,h. sie muss
\emph{iterativ} sowie \emph{in-situ} (\emph{in-place}) arbeiten und die
relative Reihenfolge gleicher Elemente jederzeit beibehalten. Sie dürfen
davon ausgehen, dass kein Eintrag im Feld null ist.

\begin{minted}{java}
void sortierenDurchEinfuegen(String[] a) {
  // Hilfsvariable:
  String tmp;
}
\end{minted}

\inputcode[firstline=5,lastline=16]{aufgaben/aud/examen_46115_2017_09/InsertionSort}

\end{enumerate}

\literatur

\end{document}
