\documentclass{lehramt-informatik}
\InformatikPakete{syntax,mathe}

\begin{document}

%%%%%%%%%%%%%%%%%%%%%%%%%%%%%%%%%%%%%%%%%%%%%%%%%%%%%%%%%%%%%%%%%%%%%%%%
% Theorie-Teil
%%%%%%%%%%%%%%%%%%%%%%%%%%%%%%%%%%%%%%%%%%%%%%%%%%%%%%%%%%%%%%%%%%%%%%%%

\chapter{Selectionsort: Sortieren durch Auswählen\footcite[Seite 39]{aud:fs:tafeluebung-11}}

\begin{quellen}
\item \cite[Seite 39]{aud:fs:tafeluebung-11}
\item \cite{wiki:selectionsort}
\item \cite[Seite 127-129 (PDF 145-147)]{saake}
\item \cite[6.4.1 Naive Sortierverfahren, Seite 191]{schneider}
\end{quellen}

\begin{itemize}
\item Funktionsweise

\begin{itemize}
\item solange zu sortierende Liste mehr als ein Element beinhaltet:

\begin{itemize}
\item lösche das \emph{Maximum / Minimum} aus der Liste
\item füge es ans Ende der Ergebnisliste
\item wiederhole, bis Eingangsliste leer
\end{itemize}

\end{itemize}

\item \emph{Eigenschaften} von Selectionsort:

\begin{itemize}
\item Laufzeitkomplexität:
$\mathcal{O}(n^2)$ (im \emph{Best}-, \emph{Average}- und
\emph{Worst-Case})

\item Stabilität leicht erreichbar
\item bei Zahlen \emph{in-situ}
\end{itemize}

\end{itemize}

%%
%
%%

\section{Iterativ}

\inputcode[firstline=5,lastline=29]{sortier/SelectionSort}
\footcite[Seite 128 (PDF 146)]{saake}

\section{Halb Rekursiv}

\inputcode[firstline=31,lastline=52]{sortier/SelectionSort}

\section{Rekursiv}

\inputcode[firstline=54,lastline=77]{sortier/SelectionSort}

%%%%%%%%%%%%%%%%%%%%%%%%%%%%%%%%%%%%%%%%%%%%%%%%%%%%%%%%%%%%%%%%%%%%%%%%
% Aufgaben
%%%%%%%%%%%%%%%%%%%%%%%%%%%%%%%%%%%%%%%%%%%%%%%%%%%%%%%%%%%%%%%%%%%%%%%%

\chapter{Aufgaben}

%-----------------------------------------------------------------------
%
%-----------------------------------------------------------------------

\section{Herbst 2014 (66115) - Thema 2, Aufgabe 6\footcite[Aufgabe 4,
Seite 4]{aud:ab:3}}

Gegeben sei ein einfacher Sortieralgorithmus, der ein gegebenes Feld $A$
dadurch sortiert, dass er das \emph{Minimum} $m$ von $A$ \emph{findet},
dann das Minimum von $A$ ohne das Element $m$ usw.
\footcite[Thema 2, Aufgabe 6, Seite 5]{examen:66115:2014:09}

\begin{enumerate}

%%
% (a)
%%

\item Geben Sie den Algorithmus in Java an. Implementieren Sie den
Algorithmus \emph{in situ}, d.\,h. so, dass er außer dem Eingabefeld nur
konstanten Extraspeicher benötigt. Es steht eine Testklasse zur
Verfügung.

\begin{antwort}
\inputcode[firstline=3]{aufgaben/aud/examen_66115_2014_09/SortierungDurchAuswaehlen}
\end{antwort}

%%
% (b)
%%

\item Analysieren Sie die Laufzeit Ihres Algorithmus.

\begin{antwort}
Beim ersten Durchlauf des \emph{Selectionsort}-Algorithmus muss $n - 1$
mal das Minimum durch Vergleich ermittel werden, beim zweiten Mal
$n - 2$.
Mit Hilfe der \emph{Gaußschen Summenforme}l kann die Komplexität
gerechnet werden:

\begin{displaymath}
(n-1)+(n-2)+\dotsb+3+2+1 =
\frac{(n-1)\cdot n}{2} =
\frac{n^2}{2}-\frac{n}{2} \approx
\frac{n^2}{2} \approx
n^2
\end{displaymath}

Da es bei der Berechnung des Komplexität um die Berechnung der
asymptotischen oberen Grenze geht, können Konstanten und die Addition,
Subtraktion, Multiplikation und Division mit Konstanten z. b.
$\frac{n^2}{2}$ vernachlässigt werden.

Der \emph{Selectionsort}-Algorithmus hat deshalb die Komplexität
$\mathcal{O}(n^2)$, er ist von der Ordnung
$\mathcal{O}(n^2)$.
\end{antwort}

%%
% (c)
%%

\item Geben Sie eine Datenstruktur an, mit der Sie Ihren Algorithmus
beschleunigen können.

\begin{antwort}
Der \emph{Selectionsort}-Algorithmus kann mit einer Min- (in diesem
Fall) bzw. einer Max-Heap beschleunigt werden. Mit Hilfe dieser
Datenstruktur kann sehr schnell das Minimum gefunden werden. So kann auf
die viele Vergleiche verzichtet werden. Die Komplexität ist dann
$\mathcal{O}(n \log n)$.
\end{antwort}

%-----------------------------------------------------------------------
%
%-----------------------------------------------------------------------

\section{Sortieren\footcite[Aufgabe 2]{aud:e-klausur}}

In dieser Aufgabe soll ein gegebenes Integer Array mit Hilfe von
\textbf{Selection Sort} sortiert werden. Es soll eine iterative und eine
rekursive Methode geschrieben werden.

Verwenden Sie zur Implementierung jeweils die Methodenköpfe
\java{selectionSortIterativ()} und \java{selectionSortRekursiv()}. Eine
\java{swap}-Methode, die für ein gegebenes Array und zwei Indizes die
Einträge an den jeweiligen Indizes des Arrays vertauscht, ist gegeben
und muss nicht implementiert werden.

Es müssen keine weiteren Methoden geschrieben werden!

\begin{antwort}
\ueberschrift{iterativ}

\inputcode[firstline=11,lastline=21]{aufgaben/aud/e_klausur/SelectionSort}

\ueberschrift{rekursiv}

\inputcode[firstline=23,lastline=35]{aufgaben/aud/e_klausur/SelectionSort}
\end{antwort}

\end{enumerate}

\literatur

\end{document}
