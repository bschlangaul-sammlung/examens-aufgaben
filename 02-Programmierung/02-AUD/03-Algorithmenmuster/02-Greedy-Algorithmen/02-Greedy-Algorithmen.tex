\documentclass{lehramt-informatik}
\InformatikPakete{mathe,syntax}

\begin{document}

%%%%%%%%%%%%%%%%%%%%%%%%%%%%%%%%%%%%%%%%%%%%%%%%%%%%%%%%%%%%%%%%%%%%%%%%
% Theorie-Teil
%%%%%%%%%%%%%%%%%%%%%%%%%%%%%%%%%%%%%%%%%%%%%%%%%%%%%%%%%%%%%%%%%%%%%%%%

\chapter{Greedy-Algorithmen}

\begin{quellen}
\item \cite[Seite 207-235 (PDF 225-253)]{saake}
\item \cite[Seite 5-6]{aud:fs:3}
\item \cite{wiki:greedy-algorithmus}
\end{quellen}

\memph{Greedy} steht hier für gierig. Das Prinzip gieriger Algorithmen
ist es, in \memph{jedem Teilschritt so viel wie möglich} zu erreichen.
\footcite[Seite 213 (PDF 231)]{saake}
Greedy-Algorithmen berechnen jeweils ein lokales Optimum in jedem Schritt
und können daher eventuell ein globales Optimum verpassen.
\footcite[Seite 214 (PDF 232)]{saake}

%%%%%%%%%%%%%%%%%%%%%%%%%%%%%%%%%%%%%%%%%%%%%%%%%%%%%%%%%%%%%%%%%%%%%%%%
% Aufgaben
%%%%%%%%%%%%%%%%%%%%%%%%%%%%%%%%%%%%%%%%%%%%%%%%%%%%%%%%%%%%%%%%%%%%%%%%

\chapter{Aufgaben}

%-----------------------------------------------------------------------
%
%-----------------------------------------------------------------------

\section{Wechselgeldalgorithmus
\footcite{net:html:wikiversity:wechselgeld}}

Als Beispiel nehmen wir die Herausgabe von Wechselgeld auf Beträge unter
1€. Verfügbar sind die Münzen mit den Werten 50ct, 10ct, 5ct, 2ct, 1ct.
Unser Ziel ist, so wenig Münzen wie möglich in das Portemonnaie zu
bekommen.
%
Ein Beispiel: $78ct = 50 + 2 \cdot 10 + 5 + 2 + 1$
%
Es wird jeweils immer die größte Münze unter dem Zielwert genommen und
von diesem abgezogen. Das wird so lange durchgeführt, bis der Zielwert
Null ist.

%%
%
%%

\subsection{Formalisierung}

Gesucht ist ein Algorithmus der folgende Eigenschaften beschreibt.
Bei der \emph{Eingabe} muss gelten:

\bigskip

\begin{compactenum}
\item dass die eingegebene Zahl eine natürliche Zahl ist, also
$\text{betrag} > 0$

\item dass eine Menge von Münzwerten zur Verfügung steht $
\text{münzen}=\{c_1,...,c_n\}$ z.\,B. $\{1,2,5,10,20,50\}$
\end{compactenum}

\bigskip

\noindent
Die \emph{Ausgabe} besteht dann aus ganzen Zahlen
$\text{wechselgeld}[1], \ldots ,\text{wechselgeld}[n]$.
Dabei ist $\text{wechselgeld}[i] $ die Anzahl der Münzen
des Münzwertes für $ c_i $ für $ i=1,\ldots,n $ und haben die
Eigenschaften:

\bigskip

\begin{compactenum}
\item $\text{wechselgeld}[1] \cdot c_1 + \ldots +
\text{wechselgeld}[n] \cdot c_n = \text{betrag}$

\item $\text{wechselgeld}[1] + \ldots + \text{wechselgeld}[n] $
ist minimal unter allen Lösungen für 1.
\end{compactenum}

%%
%
%%

\begin{antwort}
\inputcode[firstline=3]{muster/Wechselgeld}
\end{antwort}

%-----------------------------------------------------------------------
%
%-----------------------------------------------------------------------

\section{Greedy-Münzwechsler
\footcite[Seite 1, Aufgabe 1: Greedy-Münzwechsler]{aud:ab:3}}

\begin{enumerate}

%%
% (a)
%%

\item Nehmen Sie an, es stehen beliebig viele 5-Cent, 2-Cent und
1-Cent-Münzen zur Verfügung. Die Aufgabe besteht darin, für einen
gegebenen Cent-Betrag möglichst wenig Münzen zu verbrauchen. Entwerfen
Sie eine Methode

\begin{minted}{java}
public void wechselgeld (int n)
\end{minted}

die diese Aufgabe mit einem Greedy-Algorithmus löst und für den Betrag
von $n$ Cent die Anzahl $c5$ der 5-Cent-Münzen, die Anzahl $c2$ der
2-Cent-Münzen und die Anzahl $c1$ der 1-Cent-Münzen berechnet und diese
auf der Konsole ausgibt. Sie können dabei den Operator \texttt{/} für
die ganzzahlige Division und den Operator $\%$ für den Rest bei der
ganzzahligen Division verwenden.
\footnote{Quelle möglicherweise von \url{https://www.yumpu.com/de/document/read/17936760/ubungen-zum-prasenzmodul-algorithmen-und-datenstrukturen}}

\begin{antwort}
\inputcode[firstline=3]{aufgaben/aud/ab_3/GreedyMuenzwechsler}
\end{antwort}

%%
% (b)
%%

\item Es kann gezeigt werden, dass der Greedy-Algorithmus für den obigen
Fall der Münzwerte 5, 2 und 1 optimal ist, d.\,h. dass er immer die
Gesamtzahl der Münzen minimiert. Nehmen Sie nun an, es gibt die
Münzwerte 5 und 1. Ist es dann möglich, einen dritten Münzwert so zu
wählen, dass der Greedy-Algorithmus mit den drei Münzen nicht mehr
optimal ist? Begründen Sie Ihre Antwort.

\begin{antwort}
Falls der dritte Münzwert 4 ist, ist der Greedy-Algorithmus nicht mehr
optimal. Der Greedy-Algorithmus benutzt zunächst so viele 5-Cent-Münzen
wie möglich und dann so viele 4-Cent-Münzen wie möglich. Ein Betrag von
8 Cent wird also in eine 5-Cent und drei 1-Cent-Münzen aufgeteilt.
Optimal ist aber die Aufteilung in zwei 4-Cent-Münzen.
\end{antwort}
\end{enumerate}

%-----------------------------------------------------------------------
%
%-----------------------------------------------------------------------

\ExamensAufgabeTA 66115 / 2007 / 03 : Thema 2 Aufgabe 1

%-----------------------------------------------------------------------
%
%-----------------------------------------------------------------------

\ExamensAufgabeTA 66115 / 2019 / 03 : Thema 1 Aufgabe 6

%-----------------------------------------------------------------------
%
%-----------------------------------------------------------------------

\ExamensAufgabeTA 66115 / 2009 / 09 : Thema 2 Aufgabe 6

%-----------------------------------------------------------------------
%
%-----------------------------------------------------------------------

\ExamensAufgabeTA 66115 / 2017 / 09 : Thema 1 Aufgabe 8

\literatur

\end{document}
