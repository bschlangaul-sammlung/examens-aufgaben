\documentclass{lehramt-informatik}
\usepackage{blkarray}
\usepackage{tikz}
\usepackage{lehramt-informatik-graph}

\usetikzlibrary{arrows.meta}

\begin{document}

%%%%%%%%%%%%%%%%%%%%%%%%%%%%%%%%%%%%%%%%%%%%%%%%%%%%%%%%%%%%%%%%%%%%%%%%
% Theorie-Teil
%%%%%%%%%%%%%%%%%%%%%%%%%%%%%%%%%%%%%%%%%%%%%%%%%%%%%%%%%%%%%%%%%%%%%%%%

\chapter{Spannbäume}

%-----------------------------------------------------------------------
%
%-----------------------------------------------------------------------

\section{Definition}

Es sei $G$ ein zusammenhängender Graph und $S$ ein zusammenhängender
Teilgraph von G. $S$ ist ein Spannbaum von $G$, falls $S$ \memph{alle
Knoten von $G$ enthält} und $S$ \memph{zyklenfrei} ist.

%-----------------------------------------------------------------------
%
%-----------------------------------------------------------------------

\section{Minimaler Spannbaum}

$S$ ist ein minimaler Spannbaum, falls $S$ ein Spannbaum von $G$ ist,
und die Summe der \memph{Kantengewichte} in $S$ \memph{kleiner oder
gleich der aller anderen möglichen Spannbäume} $S‘$ von $G$ ist
\footcite[Seite 29 (PDF 23)]{aud:fs:6}.

%-----------------------------------------------------------------------
%
%-----------------------------------------------------------------------

\section{Algorithmus von Kruskal}

Durch den Algorithmus von Kruskal wird ein \memph{minimaler Spannbaum}
eines ungerichteten, zusammenhängenden und kantengewichteten Graphen
bestimmt. Sollen bespielsweise Rohrleitungen so verlegt werden, dass
jeder Punkt erreicht wird, kann dieser Algorithmus zum Einsatz kommen.
Wie viele Meter Rohr benötigt man mindestens? Auf diese Frage liefert
der Algorithmus eine Antwort.
\footcite[Seite 30 (PDF 24)]{aud:fs:6}

Der Algorithmus von Kruskal nutzt die Kreiseigenschaft minimaler
Spannbäume. Dazu werden die Kanten in der \memph{ersten Phase}
\memph{aufsteigend nach ihrem Gewicht sortiert}. In der zweiten Phase
wird \memph{über die sortierten Kanten iteriert}. Wenn eine
\memph{Kante} zwei Knoten verbindet, die \memph{noch nicht} durch einen
Pfad vorheriger Kanten \memph{verbunden} sind, wird diese Kante zum
minimaler Spannbäume \memph{hinzugenommen}.
\footcite{wiki:kruskal}

\def\TmpGraph#1{
  \graph knoten {
    \knoten{A}(0,5)
    \knoten{B}(2,4)
    \knoten{C}(5,5)
    \knoten{D}(0,2)
    \knoten{E}(4,2)
    \knoten{F}(2,1)
    \knoten{G}(5,0)
  } kanten {
    #1
  }
}

\TmpGraph{
  \kante(A-B){7}
  \kante(A-D){5}
  \kante(B-C){8}
  \kante(B-D){9}
  \kante(B-E){7}
  \kante(C-E){5}
  \kante(D-E){15}
  \kante(D-F){6}
  \kante(E-F){8}
  \kante(E-G){9}
  \kante(F-G){11}
}

\begin{minipage}{7cm}
\TmpGraph{
  \KANTE(A-B){7}
  \KANTE(A-D){5}
  \kante(B-C){8}
  \kante(B-D){9}
  \KANTE(B-E){7}
  \KANTE(C-E){5}
  \kante(D-E){15}
  \KANTE(D-F){6}
  \kante(E-F){8}
  \KANTE(E-G){9}
  \kante(F-G){11}
}
\end{minipage}
\begin{minipage}{4cm}
\begin{center}
\begin{tabular}{|l|l|r|}
\hline
Kante & & Gewicht\\\hline\hline
AD, CE & $2 \times 5$ & $10$\\
DF     &              & $6$\\
AB, BE & $2 \times 7$ & $14$\\
EG     &              & $9$\\\hline
       &              & $39$\\\hline
\end{tabular}
\end{center}
\end{minipage}

%-----------------------------------------------------------------------
%
%-----------------------------------------------------------------------

\section{Algorithmus von Prim\footcite{wiki:prim}}

Der Algorithmus von Prim dient ebenfalls der Bestimmung eines minimalen
Spannbaums, erreicht jedoch sein Ziel durch eine unterschiedliche
Vorgehensweise.

Der \memph{Teilgraphen} $T$ wird \memph{schrittweise vergrößert}, bis
dieser ein Spannbaum ist. Der Algorithmus benötigt einen
\memph{konkreten Startpunkt}. Es wird die die \memph{günstigste} von
einem Teilgraphen $T$ \memph{ausgehende Kante ausgewählt} und zu diesem
Spannbaum hinzugefügt. Der Algorithmus fügt jede Kante und deren
Endknoten zur Lösung hinzu, die mit allen zuvor gewählten Kanten keinen
Kreis bildet.
\footcite[Seite 32, (PDF 26)]{aud:fs:6}

\begin{minipage}{7cm}
\TmpGraph{
  \KANTE(A-B){7}
  \KANTE(A-D){5}
  \kante(B-C){8}
  \kante(B-D){9}
  \KANTE(B-E){7}
  \KANTE(C-E){5}
  \kante(D-E){15}
  \KANTE(D-F){6}
  \kante(E-F){8}
  \KANTE(E-G){9}
  \kante(F-G){11}
}
\end{minipage}
\begin{minipage}{4cm}
\begin{center}
\begin{tabular}{|l|r|}
\hline
Kante & Gewicht\\\hline\hline
AD & $5$\\
DF & $6$\\
AB & $7$\\
BE & $7$\\
EC & $5$\\
EG & $9$\\\hline
   & $39$\\\hline
\end{tabular}
\end{center}
\end{minipage}

%%%%%%%%%%%%%%%%%%%%%%%%%%%%%%%%%%%%%%%%%%%%%%%%%%%%%%%%%%%%%%%%%%%%%%%%
% Aufgaben
%%%%%%%%%%%%%%%%%%%%%%%%%%%%%%%%%%%%%%%%%%%%%%%%%%%%%%%%%%%%%%%%%%%%%%%%

\chapter{Aufgaben}

\section{Frühjahr 2014 (46115) - Thema 1, Aufgabe 8
\footcite[Seite 1-2, Aufgabe 2: Spannbäume]{aud:ab:6}}

Bestimmen Sie einen minimalen Spannbaum für einen ungerichteten Graphen,
der durch die nachfolgende Entfernungsmatrix gegeben ist! Die Matrix ist
symmetrisch und $\infty$ bedeutet, dass es keine Kante gibt. Zeichnen
Sie den Graphen und geben Sie die Spannbaumkanten ein
\footcite[Seite 5 (PDF 4)]{examen:46115:2014:03}!

\[
\begin{blockarray}{ccccccccc}
  & A      & B      & C      & D      & E      & F      & G      & H      \\
\begin{block}{c(cccccccc)}
A & 0      & 8      & -1     & \infty & 8      & \infty & 7      & \infty \\
B & 8      & 0      & \infty & 2      & \infty & \infty & \infty & 9      \\
C & -1     & \infty & 0      & 5      & 8      & 1      & 7      & \infty \\
D & \infty & 2      & 5      & 0      & 6      & 6      & \infty & \infty \\
E & 8      & \infty & 8      & 6      & 0      & 6      & 3      & \infty \\
F & \infty & \infty & 1      & 6      & 6      & 0      & 11     & 4      \\
G & 7      & \infty & 7      & \infty & 3      & 11     & 0      & 5      \\
H & \infty & 9      & \infty & \infty & \infty & 4      & 5      & 0      \\
\end{block}
\end{blockarray}
\]

% http://graphonline.ru/en/?graph=JACsZrMiExxkFxBK

% 0,8,-1,0,8,0,7,0
% 8,0,0,2,0,0,0,9
% -1,0,0,5,8,1,7,0
% 0,2,5,0,6,6,0,0
% 8,0,8,6,0,6,3,0
% 0,0,1,6,6,0,11,4
% 7,0,7,0,3,11,0,5
% 0,9,0,0,0,4,5,0

\begin{minipage}{8cm}
\graph knoten {
  \knoten{A}(2,5)
  \knoten{B}(5,6)
  \knoten{C}(5,1)
  \knoten{D}(7,0)
  \knoten{E}(2,0)
  \knoten{F}(5,3)
  \knoten{G}(0,1)
  \knoten{H}(3,4.5)
} kanten {
  \kante(A-B){8}
  \KANTE(A-C){-1}
  \kante(A-E){8}
  \kante(A-G){7}
  \KANTE(B-D){2}
  \kante(B-H){9}
  \KANTE(C-D){5}
  \kante(C-E){8}
  \KANTE(C-F){1}
  \kante(C-G){7}
  \kante(D-E){6}
  \kante(D-F){6}
  \kante(E-F){6}
  \KANTE(E-G){3}
  \kante(F-G){11}
  \KANTE(F-H){4}
  \KANTE(G-H){5}
}
\end{minipage}
\begin{minipage}{4cm}
\begin{tabular}{lr}
Kante & Gewicht \\
\hline
AC & -1 \\
BD & 2 \\
CF & 1 \\
EG & 3 \\
FH & 4 \\
GH & 5 \\
CD & 5 \\\hline
& \textbf{19} \\
\end{tabular}
\end{minipage}

Nach dem Algorithmus von Kruskal wählt man aus den noch nicht gewählten
Kanten immer die kürzeste, die keinen Kreis mit den bisher gewählten
Kanten bildet.

%-----------------------------------------------------------------------
%
%-----------------------------------------------------------------------

\section{Aufgabe 12: Graphen III
\footcite[(entnommen aus Algorithmen und Datenstrukturen, Übungsblatt 7, Universität Würzburg)]{aud:pu:7}}

Sei G der folgende Graph.
\footcite[Staatsexamen Theoretische Informatik, Algorithmen und Datenstrukturen, Realschulen, Frühjahr 2018, Thema 2, Aufgabe 4 (gekürzt)]{examen:46115:2018:03}

\graph knoten {
  \knoten{a}(0,4)
  \knoten{b}(2,4)
  \knoten{c}(0,2)
  \knoten{d}(2,2)
  \knoten{e}(0,0)
  \knoten{f}(2,0)
} kanten {
  \kante(a-b){10}
  \kante(a-c){8}
  \kante(b-c){4}
  \kante(b-d){2}
  \kante(c-d){12}
  \kante(c-e){3}
  \kante(c-f){5}
  \kante(d-f){2}
  \kante(e-f){8}
}

\begin{enumerate}

%%
% (a)
%%

\item Der Algorithmus von Prim ist ein Algorithmus zur Bestimmung des
minimalen Spannbaums in einem Graphen. Geben Sie einen anderen
Algorithmus zur Bestimmung des minimalen Spannbaums an.

\begin{antwort}
Zum Beispiel der Algorithmus von Kruskal
\end{antwort}

%%
% (b)
%%

\item Führen Sie den Algorithmus von Prim schrittweise auf $G$ aus.
Ausgangsknoten soll der Knoten $a$ sein. Ihre Tabelle sollte wie folgt
beginnen:

\begin{tabular}{|l|l|l|l|l|l|l|}
\hline
a &
b &
c &
d &
e &
f &
Warteschlange\\\hline
\end{tabular}

Die Einträge der Tabelle geben an, wie weit der angegebene Knoten
vom aktuellen Baum entfernt ist.

\begin{antwort}
\graph knoten {
  \knoten{a}(0,4)
  \knoten{b}(2,4)
  \knoten{c}(0,2)
  \knoten{d}(2,2)
  \knoten{e}(0,0)
  \knoten{f}(2,0)
} kanten {
  \kante(a-b){10}
  \KANTE(a-c){8}
  \KANTE(b-c){4}
  \KANTE(b-d){2}
  \kante(c-d){12}
  \KANTE(c-e){3}
  \kante(c-f){5}
  \KANTE(d-f){2}
  \kante(e-f){8}
}

\begin{tabular}{|l|l|l|l|l|l|l|}
\hline
a &
b &
c &
d &
e &
f &
Warteschlange\\\hline\hline

0 & % a
$\infty$ & % b
$\infty$ & % c
$\infty$ & % d
$\infty$ & % e
$\infty$ & % f
a \\\hline % Warteschlange

0 & % a
10 & % b
8 & % c
$\infty$ & % d
$\infty$ & % e
$\infty$ & % f
c, b\\\hline % Warteschlange

0 & % a
4 & % b
0 & % c
12 & % d
3 & % e
5 & % f
e, b, f, d\\\hline % Warteschlange

0 & % a
4 & % b
0 & % c
12 & % d
0 & % e
5 & % f
b, f, d \\\hline % Warteschlange

0 & % a
0 & % b
0 & % c
2 & % d
0 & % e
5 & % f
d, f \\\hline % Warteschlange

0 & % a
0 & % b
0 & % c
0 & % d
0 & % e
2 & % f
f \\\hline % Warteschlange

0 & % a
0 & % b
0 & % c
0 & % d
0 & % e
0 & % f
\\\hline % Warteschlange
\end{tabular}

\end{antwort}

%%
% (c)
%%

\item Erklären Sie, warum der Kürzeste-Wege-Baum (also das gezeichnete
Ergebnis des Dijkstra-Algorithmus) und der minimale Spannbaum nicht
notwendigerweise identisch sind.

\begin{antwort}
Die Wahl der nächsten Kante erfolgt nach völlig verschiedenen Kriterien:

\begin{itemize}
\item Beim Kürzeste-Wege-Baum orientiert sie sich an der Entfernung der
einzelnen Knoten vom Startknoten.

\item Beim Spannbaum orientiert sie sich an der Entfernung der einzelnen
Knoten vom bereits erschlossenen Teil des Spannbaums.
\end{itemize}
\end{antwort}
\end{enumerate}

%-----------------------------------------------------------------------
%
%-----------------------------------------------------------------------

\section{Spannbaum\footcite[Aufgabe 6]{aud:e-klausur}}

Ermitteln Sie einen minimalen Spannbaum des vorliegenden Graphen. Nutzen
Sie den \emph{Knoten A als Startknoten} in ihrem Algorithmus.

\graph knoten {
  \knoten{A}(0,5)
  \knoten{B}(8,5)
  \knoten{C}(4,4)
  \knoten{D}(2,3)
  \knoten{E}(6,3)
  \knoten{F}(4,1)
  \knoten{G}(0,0)
  \knoten{H}(8,0)
} kanten {
  \kante(A-B){5}
  \kante(A-C){8}
  \kante(A-D){7}
  \kante(A-G){1}
  \kante(B-C){2}
  \kante(B-E){7}
  \kante(B-H){2}
  \kante(C-D){1}
  \kante(C-E){3}
  \kante(C-F){6}
  \kante(D-F){2}
  \kante(D-G){8}
  \kante(E-H){6}
  \kante(F-E){5}
  \kante(F-H){3}
  \kante(G-F){5}
  \kante(G-H){4}
}

\begin{enumerate}
\item Welches Gewicht hat der Spannbaum insgesamt?

\begin{antwort}
\graph knoten {
  \knoten{A}(0,5)
  \knoten{B}(8,5)
  \knoten{C}(4,4)
  \knoten{D}(2,3)
  \knoten{E}(6,3)
  \knoten{F}(4,1)
  \knoten{G}(0,0)
  \knoten{H}(8,0)
} kanten {
  \kante(A-B){5}
  \kante(A-C){8}
  \kante(A-D){7}
  \KANTE(A-G){1}
  \KANTE(B-C){2}
  \kante(B-E){7}
  \KANTE(B-H){2}
  \KANTE(C-D){1}
  \KANTE(C-E){3}
  \kante(C-F){6}
  \KANTE(D-F){2}
  \kante(D-G){8}
  \kante(E-H){6}
  \kante(F-E){5}
  \kante(F-H){3}
  \kante(G-F){5}
  \KANTE(G-H){4}
}

\begin{center}
\begin{tabular}{|l|l|r|}
\hline
Kante & & Gewicht\\\hline\hline
AG, CD     & $2 \times 1$ & $2$\\
BD, BH, DF & $3 \times 2$ & $6$\\
CE         & $1 \times 3$ & $3$\\
GH         & $1 \times 4$ & $4$\\\hline
           &              & $15$\\\hline
\end{tabular}
\end{center}
\end{antwort}

\item Welchen Algorithmus haben Sie zur Ermittlung eingesetzt?

\begin{antwort}
Kruskal
\end{antwort}
\end{enumerate}

\literatur

\end{document}
