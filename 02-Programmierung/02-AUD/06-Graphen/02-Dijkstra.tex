\documentclass{lehramt-informatik}
\InformatikPakete{syntax,mathe,graph}
\usepackage{blkarray}

\begin{document}

%%%%%%%%%%%%%%%%%%%%%%%%%%%%%%%%%%%%%%%%%%%%%%%%%%%%%%%%%%%%%%%%%%%%%%%%
% Theorie-Teil
%%%%%%%%%%%%%%%%%%%%%%%%%%%%%%%%%%%%%%%%%%%%%%%%%%%%%%%%%%%%%%%%%%%%%%%%

\chapter{Der Algorithmus von Dijkstra}

\noindent
Der Algorithmus von Dijkstra bestimmt den \memph{kürzesten Pfad}
zwischen einem Startknoten und einem oder mehreren Zielknoten in einem
gewichteten Graphen.
%
Es handelt sich um einen \memph{gierigen} Algorithmus: in jedem Schritt
wird der am nächsten gelegene Knoten besucht. Der untersuchte Graph kann
\memph{gerichtet oder ungerichtet} sein. Die \memph{Kantengewichte}
müssen jedoch \memph{alle positiv} sein.
\footcite[Seite 11]{aud:fs:6}

Laufzeit: $\mathcal{O}(\log(|V|) \times |V| + |E|)$

\subsection{Pseudocode}

für alle Knoten gespeichern: Distanz, Vorgänger und „besucht“-Markierung

\begin{itemize}
\item Initialisierung:

\begin{itemize}
\item Startknoten mit Distanz $0$, alle anderen Knoten mit Distanz
$\infty$

\item markiere alle Knoten als unbesucht
\end{itemize}

\item solange es noch unbesuchte Knoten im Graphen gibt:

\begin{itemize}
\item wähle den Knoten $v$ mit der geringsten Distanz (zum Startknoten)
aus (Min-Heap)

\item markiere diesen Knoten $v$ als besucht

\item für alle unbesuchten Nachbarn $u$ des aktuellen Knoten $v$

\begin{itemize}
\item berechne die Länge des Pfades über $v$ nach $u$: Gewicht $=$ Pfad
vom Startknoten nach v $+$ Gewicht der Kante ($v$, $u$)

\item falls das berechnete Gewicht kleiner als die gespeicherte Distanz
ist, dann

\begin{enumerate}
\item aktualisiere die Distanz auf den neuen Wert
\item speichere Knoten $v$ als Vorgänger von $u$
\footcite[Seite 12]{aud:fs:6}
\end{enumerate}
\end{itemize}
\end{itemize}
\end{itemize}

\inputcode[firstline=15]{graph/Dijkstra}

%%%%%%%%%%%%%%%%%%%%%%%%%%%%%%%%%%%%%%%%%%%%%%%%%%%%%%%%%%%%%%%%%%%%%%%%
%
%%%%%%%%%%%%%%%%%%%%%%%%%%%%%%%%%%%%%%%%%%%%%%%%%%%%%%%%%%%%%%%%%%%%%%%%

\chapter{Aufgaben}

%-----------------------------------------------------------------------
%
%-----------------------------------------------------------------------

\section{Flugverbindung zwischen sieben Städten
\footcite[Seite 1, Aufgabe 1: Dijkstra]{aud:ab:6}}

Nehmen Sie an, es gibt sieben Städte A, B, C, D, E, F und G. Sie wohnen
in der Stadt A und möchten zu jeder der anderen Städte die preiswerteste
Flugverbindung finden (einfach ohne Rückflug). Sie sind dazu bereit,
beliebig oft umzusteigen. Folgende Direktflüge stehen Ihnen zur
Verfügung:

\begin{center}
\begin{tabular}{l|r}
Städte & Preis \\\hline
A $\leftrightarrow$ B & 300 € \\
A $\leftrightarrow$ F & 1000 € \\
B $\leftrightarrow$ C & 1000 € \\
B $\leftrightarrow$ D & 700 € \\
B $\leftrightarrow$ E & 300 € \\
B $\leftrightarrow$ G & 1000 € \\
C $\leftrightarrow$ F & 200 € \\
D $\leftrightarrow$ F & 400 € \\
E $\leftrightarrow$ F & 300 € \\
F $\leftrightarrow$ G & 300 € \\
\end{tabular}
\end{center}

\noindent
Der Preis p in einer Zeile

\begin{center}
\begin{tabular}{l|r}
Städte & Preis \\\hline
x $\leftrightarrow$ y & p \\
\end{tabular}
\end{center}

\noindent
gilt dabei sowohl für einen einfachen Flug von x nach y als auch für
einen einfachen Flug von y nach x. Bestimmen Sie mit dem Algorithmus von
Dijkstra (führen Sie den Algorithmus händisch durch!) die Routen und die
Preise für die preiswertesten Flugverbindungen von der Stadt A zu jeder
der anderen Städte.

\graph knoten {
  \knoten{A}(6,6)
  \knoten{B}(8,4)
  \knoten{C}(8,2)
  \knoten{D}(6,0)
  \knoten{E}(1,0)
  \knoten{F}(0,4)
  \knoten{G}(2,6)
} kanten {
  \kante(A-B){$300$}
  \kante(A-F){$1000$}
  \kante(B-C){$1000$}
  \kante(B-D){$700$}
  \kante(B-E){$300$}
  \kante(B-G){$1000$}
  \kante(C-F){$200$}
  \kante(D-F){$400$}
  \kante(E-F){$300$}
  \kante(F-G){$300$}
}

\noindent
\begin{tabular}{|l|l|l|l|l|l|l|l|l|}
\hline
Schritt & besuchte Knoten & A & B & C & D & E & F & G \\\hline\hline

Init &
& % besuchte Knoten
0 & % A
$\infty$ & % B
$\infty$ & % C
$\infty$ & % D
$\infty$ & % E
$\infty$ & % F
$\infty$ \\\hline% G

1 &
\textbf{A} & % besuchte Knoten
0 & % A
\textbf{300,A} & % B
$\infty$ & % C
$\infty$ & % D
$\infty$ & % E
1000,F & % F
$\infty$ \\\hline% G

2 &
A,\textbf{B} & % besuchte Knoten B 300
0 & % A
| & % B
1300,B & % C
1000,B & % D
\textbf{600,B} & % E
1000,F & % F
1300,B \\\hline% G

3 &
A,B,\textbf{E} & % besuchte Knoten E 600
0 & % A
| & % B
1300,B & % C
1000,B & % D
| & % E
\textbf{900,E} & % F
1300,B \\\hline% G

4 &
A,B,E,\textbf{F} & % besuchte Knoten F 900
0 & % A
| & % B
1100,F & % C
\textbf{1000,B} & % D
| & % E
| & % F
1200,F \\\hline% G

5 &
A,B,E,F,\textbf{D} & % besuchte Knoten D 1000
0 & % A
| & % B
\textbf{1100,F} & % C
| & % D
| & % E
| & % F
1200,F \\\hline% G

6 &
A,B,E,F,D,\textbf{C} & % besuchte Knoten C 1100
0 & % A
| & % B
| & % C
| & % D
| & % E
| & % F
\textbf{1200,F} \\\hline% G

7 &
A,B,E,F,D,C,\textbf{G} & % besuchte Knoten G 1200
0 & % A
| & % B
| & % C
| & % D
| & % E
| & % F
| \\\hline% G
\end{tabular}

\noindent
\begin{tabular}{|l|l|}
\hline
Städte & Preis\\\hline\hline
A $\rightarrow$ B & 300 \\\hline
A $\rightarrow$ B $\rightarrow$ E $\rightarrow$ F $\rightarrow$ C & 1100 \\\hline
A $\rightarrow$ B $\rightarrow$ D & 1000 \\\hline
A $\rightarrow$ B $\rightarrow$ E & 600 \\\hline
A $\rightarrow$ B $\rightarrow$ E $\rightarrow$ F & 900 \\\hline
A $\rightarrow$ B $\rightarrow$ E $\rightarrow$ F $\rightarrow$ G & 1200 \\\hline
\end{tabular}

%-----------------------------------------------------------------------
%
%-----------------------------------------------------------------------

\section{gerichteter Distanzgraph angegeben durch Adjazenzmatrix
\footcite[Seite 3, Aufgabe 5 (Check-Up)]{aud:ab:6}}

Ein gerichteter Distanzgraph sei durch seine Adjazenzmatrix gegeben (in
einer Zeile stehen die Längen der von dem Zeilenkopf ausgehenden Wege.)
\footcite[Seite 3]{aud:ab:6}

\[
\begin{blockarray}{cccccc}
& M & A & P & R & N \\
\begin{block}{c(ccccc)}
  M & - & 5 & 10& - & - \\
  A & - & - & 3 & 9 & 2 \\
  P & - & 2 & - & 1 & - \\
  R & - & - & - & - & 4 \\
  N & 7 & - & - & 6 & - \\
\end{block}
\end{blockarray}
\]

\begin{enumerate}

%%
% (a)
%%

\item Stellen Sie den Graph in der üblichen Form dar.

\paragraph{Lösung gezeichnet mit \href{https://www.ctan.org/pkg/pgf}{TikZ}}

\begin{center}
\begin{tikzpicture}
\begin{scope}[every node/.style={circle,draw}]
  \node (M) at (-1,0) {M};
  \node (A) at (0,-3) {A};
  \node (P) at (2,2) {P};
  \node (R) at (5,0) {R};
  \node (N) at (4,-3) {N};
\end{scope}

\begin{scope}[>={Stealth[black]},
              every node/.style={fill=white,circle},
              every edge/.style={draw=black}]
  \path [->] (A) edge node {$5$} (M);
  \path [->] (P) edge node {$10$} (M);
  \path [->, bend angle=20, bend right] (P) edge node {$3$}  (A);
  \path [->] (R) edge node {$9$} (A);
  \path [->] (N) edge node {$2$} (A);
  \path [->, bend angle=10, bend right] (A) edge node {$2$} (P);
  \path [->] (R) edge node {$1$} (P);
  \path [->, bend angle=20, bend right] (N) edge node {$4$} (R);
  \path [->] (M) edge node {$7$} (N);
  \path [->, bend angle=20, bend right] (R) edge node {$6$} (N);
\end{scope}
\end{tikzpicture}
\end{center}

\paragraph{Lösung gezeichnet online auf
\href{http://graphonline.ru/en/?graph=uhkywkYEzLukCUxf}{graphonline.ru}}

\par

\url{http://graphonline.ru/en/?graph=uhkywkYEzLukCUxf}

\begin{center}
\includegraphics[width=0.6\linewidth]{Aufgabe-5_Check-Up_graphonline.ru.png}
\end{center}

%%
% (b)
%%

\item Bestimmen Sie mit dem Algorithmus von Dijkstra ausgehend von $M$
die kürzeste Wege zu allen anderen Knoten.

\paragraph{Nach der Methode von
\href
{https://www.youtube.com/watch?v=4pBP2hbnGso}
{Prof. Dr. Oliver Lazar}}

\begin{tabular}{|l|l|l|l|l|l|}
\hline
Schritt & Betrachteter Knoten & Kosten A & Kosten P & Kosten R & Kosten N  \\
Initial & M                   &          &          &          & 7         \\
1       & N                   & 9        &          & 11       & 7         \\
2       & A                   & 9        & 11       & 11       & 7         \\
3       & P                   & 9        & 11       & 11       & 7         \\
4       & R                   & 9        & 11       & 11       & 7         \\
\end{tabular}

\paragraph{Nach der Methode aus der Vorlesung}

\textbf{Besuchte Knoten:} M

\begin{tabular}{l||l|l|l|l|l}
Knoten-Name & M    & A        & P        & R        & N  \\
Distanz     & 0    & $\infty$ & $\infty$ & $\infty$ & $\infty$ \\
Vorgänger   & null & null     & null     & null     & null \\
\end{tabular}

\textbf{Besuchte Knoten:} M, N

\begin{tabular}{l||l|l|l|l|l}
Knoten-Name & M    & A        & P        & R        & N  \\
Distanz     & 0    & $\infty$ & $\infty$ & $\infty$ & 7 \\
Vorgänger   & null & null     & null     & null     & M \\
\end{tabular}

\textbf{Besuchte Knoten:} M, N, A

\begin{tabular}{l||l|l|l|l|l}
Knoten-Name & M    & A        & P        & R        & N  \\
Distanz     & 0    & 9        & $\infty$ & $\infty$ & 7 \\
Vorgänger   & null & N        & null     & null     & M \\
\end{tabular}

\textbf{Besuchte Knoten:} M, N, A, P

\begin{tabular}{l||l|l|l|l|l}
Knoten-Name & M    & A        & P        & R        & N  \\
Distanz     & 0    & 9        & 11       & $\infty$ & 7 \\
Vorgänger   & null & N        & A        & null     & M \\
\end{tabular}

\textbf{Besuchte Knoten:} M, N, A, P, R

\begin{tabular}{l||l|l|l|l|l}
Knoten-Name & M    & A        & P        & R        & N  \\
Distanz     & 0    & 9        & 11       & 11       & 7 \\
Vorgänger   & null & N        & A        & N        & M \\
\end{tabular}

\paragraph{Ergebnis}

$M \rightarrow N = 7$

$M \rightarrow N \rightarrow A = 9$

$M \rightarrow N \rightarrow A \rightarrow P = 11$

$M \rightarrow N \rightarrow R = 11$

%%
% (c)
%%

\item Beschreiben Sie wie ein Heap als Prioritätswarteschlange in diesem
Algorithmus verwendet werden kann.

\begin{antwort}
Die Prioritätswarteschlange kann dazu verwendet werden, den Knoten mit
der kürzesten Distanz schnell zu finden. Eine Prioritätswarteschlange
kann zum Beispiel durch eine Min-Heap realisiert werden. Wenn eine
Min-Heap aufgebaut wird, ist das Minium immer das Wurzelelement. Es kann
sehr einfach und schnell entnommen werden. Der Aufbau einer Min-Heap
geht mit linearem Zeitaufwand $\mathcal{O}(n)$ vonstatten. Die
Entnahme des Miniums schlägt im schlechtesten Fall mit einem Aufwand
von $\mathcal{O}(\log n)$ zu Buche schlagen.
\end{antwort}

%%
% (d)
%%

\item Geben Sie die Operation \emph{„Entfernen des Minimums“} für einen
Heap an. Dazu gehört selbstverständlich die Restrukturierung des Heaps.

\begin{antwort}
Bei der Entnahme des Minimums wird an dessen Stelle das am Ende
der Halde sich befindende Element gesetzt.

Das neue Minimum verletzt unter Umständen die Heap-Eigenschaften, wenn
eines oder beide seiner Kind-Knoten kleiner sind. Es muss mit dem
kleinsten Kind-Knoten getauscht werden. Diese Prozedur wird rekursiv so
lange ausgeführt, bis die Heap-Eigenschaften wieder hergestellt sind.
Man nennt diesen Vorgang auch \texttt{heapify}.

Es kann aber auch sein das das verschobene Element kleiner ist. Dann
muss die gegenteilige Operation von \texttt{heapify} ausführt werden,
die \texttt{decrease} genannt wird.
\end{antwort}
\end{enumerate}

%-----------------------------------------------------------------------
%
%-----------------------------------------------------------------------

\ExamensAufgabeTA 66112 / 2004 / 03 : Thema 1 Aufgabe 5

%-----------------------------------------------------------------------
%
%-----------------------------------------------------------------------

\ExamensAufgabeTA 46115 / 2013 / 03 : Thema 2 Aufgabe 5

%-----------------------------------------------------------------------
%
%-----------------------------------------------------------------------

\section{Dijkstra-Algorithmus\footcite[Aufgabe 4]{aud:e-klausur}}

Führen Sie auf dem gegebenen Graphen die Suche nach der kürzesten
Distanz aller Knoten zum Startknoten A mit dem Algorithmus von Dijkstra
durch. Tragen Sie die Abarbeitungsreihenfolge, den unmittelbaren
Vorgängerknoten, sowie die ermittelte kürzeste Distanz für jeden Knoten
ein! Bei gleichen Distanzen arbeiten Sie die Knoten in lexikalischer
Reihenfolge ab.

\graph knoten {
  \knoten{A}(1,4)
  \knoten{B}(3,5)
  \knoten{C}(3,3)
  \knoten{D}(0,2)
  \knoten{E}(5,5)
  \knoten{F}(5,1)
  \knoten{G}(3,0)
  \knoten{H}(6,3)
  \knoten{I}(8,4)
} kanten {
  \kante(A-B){2}
  \kante(A-C){5}
  \kante(A-D){2}
  \kante(B-C){3}
  \kante(B-E){1}
  \kante(C-D){3}
  \kante(C-E){1}
  \kante(C-F){1}
  \kante(C-H){1}
  \kante(D-G){2}
  \kante(E-I){7}
  \kante(F-G){2}
  \kante(F-H){3}
  \kante(F-H){3}
  \kante(H-I){1}
}

\literatur

\end{document}
