\documentclass{lehramt-informatik}
\InformatikPakete{baum}

\begin{document}

%%%%%%%%%%%%%%%%%%%%%%%%%%%%%%%%%%%%%%%%%%%%%%%%%%%%%%%%%%%%%%%%%%%%%%%%
% Theorie-Teil
%%%%%%%%%%%%%%%%%%%%%%%%%%%%%%%%%%%%%%%%%%%%%%%%%%%%%%%%%%%%%%%%%%%%%%%%

\chapter{Bäume}

\section{Grundlagen}

\begin{quellen}
\item \cite[Kapitel 6.2.2.3, Seite 186]{schneider}
\item \cite[Seite 345-419]{saake}
\end{quellen}

Bäume sind eine der \memph{wichtigsten dynamischen Datenstrukturen} in
der Informatik. Es können nicht nur Daten, sondern \memph{auch
Beziehungen }(z. B. Ordnungen) der Daten gespeichert werden. Bäume sind
aus Knoten aufgebaut, die durch (gerichtete) Kanten verbunden sind. Die
Daten werden in der Regel in den Knoten gespeichert. Die Wurzel eines
Baums besitzt nur auslaufende Kanten. Blätter sind Knoten mit nur einer
einlaufenden Kante.
\footcite[Seite 2]{aud:fs:5}

%%
%
%%

\subsection{Definition (Baum - rekursiv)}

Ein Baum ist leer oder er besteht aus einer Wurzel und einer leeren oder
nichtleeren endlichen Menge disjunkter Bäume (sogenannte Teilbäume).

%%
%
%%

\subsection{Definition (Binärbaum)}

Ein Binärbaum ist ein Baum, bei dem jeder Knoten genau zwei
Verzweigungsmöglichkeiten besitzt.

Diese Variante des Baumes wird sehr häufig verwendet. Es ist eine
Hierarchische Datenstruktur. Jedes Baumelement besitzt einen linken und
einen rechten Teilbaum. Diese Datenstruktur ist gut geeignet zum
Sortieren.
\footcite[Seite 3]{aud:fs:5}

%%%%%%%%%%%%%%%%%%%%%%%%%%%%%%%%%%%%%%%%%%%%%%%%%%%%%%%%%%%%%%%%%%%%%%%%
% Aufgaben
%%%%%%%%%%%%%%%%%%%%%%%%%%%%%%%%%%%%%%%%%%%%%%%%%%%%%%%%%%%%%%%%%%%%%%%%

\chapter{Aufgaben}

%-----------------------------------------------------------------------
%
%-----------------------------------------------------------------------

\section{Staatsexamenaufgabe zu Binärbaum, Halde, AVL
\footcite[Thema 2, Aufgabe 8]{examen:66115:2017:09}}

\begin{enumerate}

%%
% (a)
%%

\item Fügen Sie die Zahlen $13$, $12$, $42$, $3$, $11$ in der gegebenen
Reihenfolge in einen zunächst leeren binären Suchbaum mit aufsteigender
Sortierung ein. Stellen Sie nur das Endergebnis dar.
\footcite[Seite 12]{examen:66115:2017:09}

\begin{center}
\begin{tikzpicture}[binaerer baum]
\Tree
[.13
  [.12
    [.3
      \edge[blank]; \node[blank]{};
      [.11 ]
    ]
    \edge[blank]; \node[blank]{};
  ]
  [.42 ]
]
\end{tikzpicture}
\end{center}

%%
% (b)
%%

\item Löschen Sie den Wurzelknoten mit Wert $42$ aus dem folgenden
\emph{binären} Suchbaum mit aufsteigender Sortierung und ersetzen Sie
ihn dabei durch einen geeigneten Wert aus dem \emph{rechten} Teilbaum.
Lassen Sie möglichst viele Teilbäume unverändert und erhalten Sie die
Suchbaumeigenschaft.
\footcite[Seite 12]{examen:66115:2017:09}

% 42,13,66,28,50,99,61,55

\begin{minipage}{0.5\linewidth}
\begin{tikzpicture}[binaerer baum]
\Tree
[.\node(root){\textbf{42}};
  [.13
    \edge[blank]; \node[blank]{};
    [.28 ]
  ]
  [.\node(66){66};
    [.\node(replacement){\textbf{50}};
      \edge[blank]; \node[blank]{};
      [.\node(61){61};
        [.55 ]
        \edge[blank]; \node[blank]{};
      ]
    ]
    [.99 ]
  ]
]
\draw[dotted,->] (replacement) -- (root);
\draw[dotted,<-] (66) -- (61);
\end{tikzpicture}
\end{minipage}
\begin{minipage}{0.5\linewidth}
\begin{tikzpicture}[binaerer baum]
\Tree
[.\textbf{50}
  [.13
    \edge[blank]; \node[blank]{};
    [.28 ]
  ]
  [.66
    [.61
      [.55 ]
      \edge[blank]; \node[blank]{};
    ]
    [.99 ]
  ]
]
\end{tikzpicture}
\end{minipage}

%%
% (c)
%%

\item Fügen Sie einen neuen Knoten mit dem Wert 13 in die folgende
Min-Halde ein und stellen Sie anschließend die Halden-Eigenschaft vom
neuen Blatt aus beginnend wieder her, wobei möglichst viele Knoten der
Halde unverändert bleiben und die Halde zu jedem Zeitpunkt
links-vollständig sein soll. Geben Sie nur das Endergebnis an.

%%
% (d)
%%

\item Geben Sie für die ursprüngliche Min-Halde aus Teilaufgabe c)(d.h.
ohne den neu eingefügten Knoten mit dem Wert 13) die Feld-Einbettung
(Array-Darstellung) an.

%%
% (e)
%%

\item Löschen Sie den Wurzelknoten mit Wert 42 aus der folgenden
Max-Halde und stellen Sie anschließend die Halden-Eigenschaft ausgehend
von einer neuen Wurzel wieder her, wobei möglichst viele Knoten der
Halde unverändert bleiben und die Halde zu jedem Zeitpunkt
links-vollständig sein soll. Geben Sie nur das Endergebnis an.

%%
% (f)
%%

\item Fügen Sie in jeden der folgenden AVL-Bäume mit aufsteigender
Sortierung jeweils einen neuen Knoten mit dem Wert 13 ein und führen Sie
anschließend bei Bedarf die erforderliche(n) Rotation(en) durch.
Zeichnen Sie den Baum vor und nach den Rotationen.

\begin{enumerate}

%%
% i)
%%

\item AVL-Baum A

%%
% ii)
%%

\item AVL-Baum B
\end{enumerate}
\end{enumerate}

\literatur

\end{document}
