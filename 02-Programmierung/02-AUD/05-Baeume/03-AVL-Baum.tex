\documentclass{lehramt-informatik}
\InformatikPakete{mathe,baum,syntax}
\usepackage{xparse}

\begin{document}

%%%%%%%%%%%%%%%%%%%%%%%%%%%%%%%%%%%%%%%%%%%%%%%%%%%%%%%%%%%%%%%%%%%%%%%%
% Theorie-Teil
%%%%%%%%%%%%%%%%%%%%%%%%%%%%%%%%%%%%%%%%%%%%%%%%%%%%%%%%%%%%%%%%%%%%%%%%

\section{AVL-Bäume}

\begin{quellen}
\item \cite{wiki:avl-baum}
\item \cite[Kapitel 14.4.2, Seite 378-386 (PDF 394-402)]{saake}
\item \cite[bst]{net:html:visualgo}
\end{quellen}

%-----------------------------------------------------------------------
%
%-----------------------------------------------------------------------

\section{Visualisierungstools}

\begin{itemize}
\item \url{https://www.cs.usfca.edu/~galles/visualization/AVLtree.html}
\item \url{https://visualgo.net/bn/bst} (oben in den Tabs umschalten auf AVL)
\end{itemize}

\noindent
Ein AVL-Baum ist ein binärer Suchbaum, der \memph{höhenbalanciert} ist,
d.\,h. für jeden Knoten gilt:

\begin{center}
$|h(\text{rechterTeilbaum}) - h(\text{linkerTeilbaum})| \leq 1$
\end{center}

\noindent
Die \memph{„Entartung“} des Baums wird so vermieden. Die Höhe eines
AVL-Baums ist $h \in \mathcal{O}(\log n)$. Beim Einfügen und Löschen von
Knoten muss die AVL-Eigenschaft durch \memph{Rotationen}
wiederhergestellt werden.
\footcite[Seite 22 (PDF 16)]{aud:fs:5}

%-----------------------------------------------------------------------
%
%-----------------------------------------------------------------------

\section{Rotationsregeln}

\begin{center}
Balance-Faktor $=$ Knotenzahl rechter Teilbaum $-$ Knotenzahl linker Teilbaum
\end{center}

Rotationsregeln:
bO= „oberer“ Wurzelknoten, bU = Kindknoten von bO

\begin{itemize}
\item bO = -2, bU = -1  Rechtsrotation (Rechtsdrehung um bO)
\item bO = +2, bU = +1  Linksrotation (Linksdrehung um bO)
\item bO = -2, bU = +1  Links-Rechts-Rotation (Linksdrehung um bU, dann Rechtsdrehung um bO)
\item bO = +2, bU = -1  Rechts-Links-Rotation (Rechtsdrehung um bU, dann Linksdrehung um bO)
\end{itemize}

%%
%
%%

\subsection{Linksrotation}

\inputcode[firstline=147,lastline=152]{baum/saake/AVLBaum}

%%
%
%%

\subsection{Rechtsrotation}

\inputcode[firstline=160,lastline=165]{baum/saake/AVLBaum}

Kindknoten, die nach der Drehung „im Weg stehen“, werden „umgeklappt“,
also vom linken zum rechten Kind der nächsten Ebene und umgekehrt

\section{Aufgabe Herbst 2006, Thema 1\footcite{examen:66115:2006:09}}

\begin{enumerate}
\item Gegeben sei die folgende Folge ganzer Zahlen: 6, 13, 4, 8, 11, 9, 10.

\begin{enumerate}
\item Fügen Sie obige Zahlen der Reihe nach in einen anfangs leeren
AVL-Baum ein und stellen Sie den Baum nach jedem Einfügeschritt dar!

\item Löschen Sie das Wurzelelement des entstandenen AVL-Baums und
stellen Sie die AVL-Eigenschaft wieder her!
\end{enumerate}
\end{enumerate}

\NewDocumentCommand{\tmpEinfuegen}{ m }{
  \bigskip
  \noindent
  \textbf{Einfügen von $#1$:}
}

\NewDocumentCommand{\tmpAvl}{ m m } {
  \bigskip
  \noindent
  \begin{minipage}{6cm}
  \begin{tikzpicture}[binaerer baum]
   #1
  \end{tikzpicture}
  \end{minipage}
  \begin{minipage}{6cm}
  #2
  \end{minipage}
}

%-----------------------------------------------------------------------
%
%-----------------------------------------------------------------------

\tmpEinfuegen{6}

%%
%
%%

\tmpAvl
{\Tree[.6 ]}
{$6$ als Wurzel eingefügt}

%-----------------------------------------------------------------------
%
%-----------------------------------------------------------------------

\tmpEinfuegen{13}

%%
%
%%

\tmpAvl
{\Tree
[.6
  \edge[blank]; \node[blank]{};
  \edge[]; {13}
]
}
{$13 > 6$ deshalb rechts angehängt}

%-----------------------------------------------------------------------
%
%-----------------------------------------------------------------------

\tmpEinfuegen{4}

%%
%
%%

\tmpAvl
{\Tree
[.6
  [.4 ]
  [.13 ]
]}
{$4 < 6$ deshalb links angehängt}

%-----------------------------------------------------------------------
%
%-----------------------------------------------------------------------

\tmpEinfuegen{8}

%%
%
%%

\tmpAvl
{\Tree
[.6
  [.4 ]
  [.13
    \edge[]; {8}
    \edge[blank]; \node[blank]{};
  ]
]}
{$8 > 6$ und $8 < 13$}

%-----------------------------------------------------------------------
%
%-----------------------------------------------------------------------

\tmpEinfuegen{11}

%%
%
%%

\tmpAvl
{\Tree
[.6
  [.4 ]
  [.13
    [.8
      \edge[blank]; \node[blank]{};
      \edge[]; {11}
    ]
    \edge[blank]; \node[blank]{};
  ]
]}
{
  $11 > 6$, $11 < 13$ und $11 > 8$;
  die Balancierung ist jetzt verletzt, mit unterschiedlichen Vorzeichen ($+1$, $-2$)
  deshalb Doppelrotation links-rechts
}

%%
%
%%

\tmpAvl
{\Tree
[.6
  [.4 ]
  [.13
    [.11
      \edge[]; {8}
      \edge[blank]; \node[blank]{};
    ]
    \edge[blank]; \node[blank]{};
  ]
]}
{
  Balancierung wäre nach Linksrotation weiter verletzt, deshalb zusätzliche Rechtsrotation
}

%%
%
%%

\tmpAvl
{\Tree
[.6
  [.4 ]
  [.11
    [.8 ]
    [.13 ]
  ]
]}
{Balancierung wiederhergestellt}

%-----------------------------------------------------------------------
%
%-----------------------------------------------------------------------

\tmpEinfuegen{9}

%%
%
%%

\tmpAvl
{\Tree
[.6
  [.4 ]
  [.11
    [.8
      \edge[blank]; \node[blank]{};
      \edge[]; {9}
    ]
    [.13 ]
  ]
]}
{
  $9 > 8$;
  Balancierung verletzt mit unterschiedlichen Vorzeichen ($-1$, $+2$),
  deshalb Doppelrotation rechts-links
}

%%
%
%%

\tmpAvl
{\Tree
[.6
  [.4 ]
  [.8
    [.9 ]
    [.11
      \edge[blank]; \node[blank]{};
      \edge[]; {13}
    ]
  ]
]}
{
  Balancierung wäre nach Rechtsrotation weiter verletzt,
  deshalb zusätzliche Linksrotation
}

%%
%
%%

\tmpAvl
{\Tree
[.8
  [.6
    \edge[]; {4}
    \edge[blank]; \node[blank]{};
  ]
  [.11
    [.9 ]
    [.13 ]
  ]
]}
{
  Balancierung wiederhergestellt
}

%-----------------------------------------------------------------------
%
%-----------------------------------------------------------------------

\tmpEinfuegen{10}

%%
%
%%

\tmpAvl
{\Tree
[.8
  [.6
    \edge[]; {4}
    \edge[blank]; \node[blank]{};
  ]
  [.11
    [.9
      \edge[blank]; \node[blank]{};
      \edge[]; {10}
    ]
    [.13 ]
  ]
]}
{
  $10 > 8$
}

%%%%%%%%%%%%%%%%%%%%%%%%%%%%%%%%%%%%%%%%%%%%%%%%%%%%%%%%%%%%%%%%%%%%%%%%
% Aufgaben
%%%%%%%%%%%%%%%%%%%%%%%%%%%%%%%%%%%%%%%%%%%%%%%%%%%%%%%%%%%%%%%%%%%%%%%%

\chapter{Aufgaben}

\section{Einfügen von Knoten
\footcite[Seite 3, Aufgabe 5: AVL-Baum]{aud:ab:7}}

Gegeben sei folgender AVL-Baum:

Dabei sind die Blätter der Übersichtlichkeit halber weggelassen worden
und über jedem Knoten v ist folgender Wert angegeben:

\begin{center}
Höhe linker Teilbaum von v – Höhe rechter Teilbaum von v
\end{center}

\begin{enumerate}

%%
% (a)
%%

\item Fügen Sie in diesen Baum den Schlüssel 1 ein.

%%
% (b)
%%

\item Fügen Sie in diesen Baum den Schlüssel 11 ein.

\section{AVL-Baum\footcite{aud:e-klausur}}

Fügen Sie die Zahlen 2, 8, 10, 1, 4, 5, 11 in der vorgegebenen
Reihenfolge in einen AVL-Baum ein. Wie sieht der finale AVL-Baum aus?

\begin{antwort}
\begin{description}
\item[2:] ist die Wurzel
\item[8:] rechts an 2
\item[10:] rechts an 8, dann Links-Rotation
\item[1:] links an 2
\item[4:] rechts an 2
\item[5:] rechts an 4, dann Links-Rechts-Rotation.
\item[11:] rechts an 10
\end{description}

\begin{tikzpicture}[binaerer baum]
\Tree
[.4
  [.2
    [.1 ]
    \edge[blank]; \node[blank]{};
  ]
  [.8
    [.5 ]
    [.10
      \edge[blank]; \node[blank]{};
      [.11 ]
    ]
  ]
]
\end{tikzpicture}
\end{antwort}

\end{enumerate}

\literatur

\end{document}
