\documentclass{lehramt-informatik-haupt}
\InformatikPakete{mathe,baum,syntax}
\usepackage{xparse}

\begin{document}

%%%%%%%%%%%%%%%%%%%%%%%%%%%%%%%%%%%%%%%%%%%%%%%%%%%%%%%%%%%%%%%%%%%%%%%%
% Theorie-Teil
%%%%%%%%%%%%%%%%%%%%%%%%%%%%%%%%%%%%%%%%%%%%%%%%%%%%%%%%%%%%%%%%%%%%%%%%

\section{AVL-Bäume}

\begin{quellen}
\item \cite{wiki:avl-baum}
\item \cite[Kapitel 14.4.2, Seite 378-386 (PDF 394-402)]{saake}
\item \cite[bst]{net:html:visualgo}
\end{quellen}

%-----------------------------------------------------------------------
%
%-----------------------------------------------------------------------

\section{Visualisierungstools}

\begin{itemize}
\item \url{https://www.cs.usfca.edu/~galles/visualization/AVLtree.html}
\item \url{https://visualgo.net/bn/bst} (oben in den Tabs umschalten auf AVL)
\end{itemize}

\noindent
Ein AVL-Baum ist ein binärer Suchbaum, der \memph{höhenbalanciert} ist,
d.\,h. für jeden Knoten gilt:

\begin{center}
$|h(\text{rechterTeilbaum}) - h(\text{linkerTeilbaum})| \leq 1$
\end{center}

\noindent
Die \memph{„Entartung“} des Baums wird so vermieden. Die Höhe eines
AVL-Baums ist $h \in \mathcal{O}(\log n)$. Beim Einfügen und Löschen von
Knoten muss die AVL-Eigenschaft durch \memph{Rotationen}
wiederhergestellt werden.
\footcite[Seite 22 (PDF 16)]{aud:fs:5}

%-----------------------------------------------------------------------
%
%-----------------------------------------------------------------------

\section{Rotationsregeln}

\begin{center}
Balance-Faktor $=$ Knotenzahl rechter Teilbaum $-$ Knotenzahl linker Teilbaum
\end{center}

Rotationsregeln:
bO= „oberer“ Wurzelknoten, bU = Kindknoten von bO

\begin{itemize}
\item bO = -2, bU = -1  Rechtsrotation (Rechtsdrehung um bO)
\item bO = +2, bU = +1  Linksrotation (Linksdrehung um bO)
\item bO = -2, bU = +1  Links-Rechts-Rotation (Linksdrehung um bU, dann Rechtsdrehung um bO)
\item bO = +2, bU = -1  Rechts-Links-Rotation (Rechtsdrehung um bU, dann Linksdrehung um bO)
\end{itemize}

%%
%
%%

\subsection{Linksrotation}

\inputcode[firstline=147,lastline=152]{baum/AVLBaum}

%%
%
%%

\subsection{Rechtsrotation}

\inputcode[firstline=160,lastline=165]{baum/AVLBaum}

Kindknoten, die nach der Drehung „im Weg stehen“, werden „umgeklappt“,
also vom linken zum rechten Kind der nächsten Ebene und umgekehrt

\ExamensAufgabeTA 66115 / 2006 / 09 : Thema 1 Aufgabe 4

%%%%%%%%%%%%%%%%%%%%%%%%%%%%%%%%%%%%%%%%%%%%%%%%%%%%%%%%%%%%%%%%%%%%%%%%
% Aufgaben
%%%%%%%%%%%%%%%%%%%%%%%%%%%%%%%%%%%%%%%%%%%%%%%%%%%%%%%%%%%%%%%%%%%%%%%%

\chapter{Aufgaben}

\section{Einfügen von Knoten
\footcite[Seite 3, Aufgabe 5: AVL-Baum]{aud:ab:7}}

Gegeben sei folgender AVL-Baum:

Dabei sind die Blätter der Übersichtlichkeit halber weggelassen worden
und über jedem Knoten v ist folgender Wert angegeben:

\begin{center}
Höhe linker Teilbaum von v – Höhe rechter Teilbaum von v
\end{center}

\begin{enumerate}

%%
% (a)
%%

\item Fügen Sie in diesen Baum den Schlüssel 1 ein.

%%
% (b)
%%

\item Fügen Sie in diesen Baum den Schlüssel 11 ein.

\section{AVL-Baum\footcite{aud:e-klausur}}

Fügen Sie die Zahlen 2, 8, 10, 1, 4, 5, 11 in der vorgegebenen
Reihenfolge in einen AVL-Baum ein. Wie sieht der finale AVL-Baum aus?

\begin{antwort}
\begin{description}
\item[2:] ist die Wurzel
\item[8:] rechts an 2
\item[10:] rechts an 8, dann Links-Rotation
\item[1:] links an 2
\item[4:] rechts an 2
\item[5:] rechts an 4, dann Links-Rechts-Rotation.
\item[11:] rechts an 10
\end{description}

\begin{tikzpicture}[binaerer baum]
\Tree
[.4
  [.2
    [.1 ]
    \edge[blank]; \node[blank]{};
  ]
  [.8
    [.5 ]
    [.10
      \edge[blank]; \node[blank]{};
      [.11 ]
    ]
  ]
]
\end{tikzpicture}
\end{antwort}

\end{enumerate}

\literatur

\end{document}
