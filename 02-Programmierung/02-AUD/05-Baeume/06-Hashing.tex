\documentclass{lehramt-informatik-haupt}
\liLadePakete{syntax,mathe}
\usepackage{amsmath}

\begin{document}

%%%%%%%%%%%%%%%%%%%%%%%%%%%%%%%%%%%%%%%%%%%%%%%%%%%%%%%%%%%%%%%%%%%%%%%%
% Theorie-Teil
%%%%%%%%%%%%%%%%%%%%%%%%%%%%%%%%%%%%%%%%%%%%%%%%%%%%%%%%%%%%%%%%%%%%%%%%

\chapter{Hashing}

\begin{quellen}
\item \cite[Kapitel 6.3.3 Streutabellen (Hashing), Seite
188-189]{schneider}
\item \cite[Seite 22-39]{aud:fs:tafeluebung-10}
\item \cite[Seite 419-439]{saake}
\item \cite{wiki:hashtabelle}
\end{quellen}

%-----------------------------------------------------------------------
%
%-----------------------------------------------------------------------

\section{Grundlagen}
Das Ziel von Hashing ist einerseits einen \memph{extrem großen
Schlüsselraum} auf einen vernünftig \memph{kleinen Bereich von ganzen
Zahlen} abzubilden und andererseits dass \memph{zwei Schlüssel auf die
selbe Zahl} abgebildet werden, soll möglichst \memph{unwahrscheinlich}
sein.\footnote{Seite 4
\url{https://moves.rwth-aachen.de/wp-content/uploads/SS15/dsal/lec13.pdf}}

Daten werden in einer \memph{Streu(wert)tabelle} (hash table) abgelegt.
Aufgrund des wahlfreien Zugriffs eignen sich Felder zum Abspeichern der
Daten.
%
Eine \memph{Hashfunktion} $h$ bildet ein Datenelement auf einen
\memph{Hashwert} ab. Das Datenelement benötigt dazu einen
\memph{Schlüssel} (key), der das Element eindeutig identifiziert. Der
Hashwert wird als Index in dem Feld verwendet. Das Datenelement wird
im entsprechenden \memph{Bucket} der Tabelle gespeichert.
%
Bei der \memph{Suche} nach einem Element mit \memph{bekanntem Schlüssel}
wird der Index mittels der Hashfunktion bestimmt. Dies geschieht mit
\memph{konstantem Aufwand}.
%
Der Aufwand des Nachschlagens an entsprechender Stelle ist abhängig von
der \emph{Organisationsform}.
\footcite[Seite 25]{aud:fs:tafeluebung-10}

%-----------------------------------------------------------------------
%
%-----------------------------------------------------------------------

\section{Kollisionen}

Da Hashfunktionen im Allgemeinen  \memph{nicht eindeutig (injektiv)}
sind, können zwei unterschiedliche Schlüssel zum selben Hash-Wert, also
zum selben Feld in der Tabelle, führen. Dieses Ereignis wird als
\memph{Kollision} bezeichnet. In diesem Fall muss die Hashtabelle
mehrere Werte in demselben Bucket aufnehmen.

Eine Kollision benötigt bei der Suche eine spezielle Behandlung durch
das Verfahren: Zunächst wird aus einem Suchschlüssel wieder ein Hashwert
berechnet, der den Bucket des gesuchten Datenobjektes bestimmt; dann
muss noch durch direkten Vergleich des Suchschlüssels mit den Objekten
im Bucket das gesuchte Ziel bestimmt werden.

Zur Behandlung von Kollisionen werden kollidierte Daten nach einer
\memph{Ausweichstrategie in alternativen Feldern} oder in einer Liste
gespeichert. Schlimmstenfalls können Kollisionen zu einer Entartung der
Hashtabelle führen, wenn wenige Hashwerte sehr vielen Objekten
zugewiesen wurden, während andere Hashwerte unbenutzt bleiben.
Kollisionsauflösungsstrategien

Um das Kollisions-Problem zu handhaben, gibt es diverse
Kollisionsauflösungsstrategien.

%%
%
%%

\subsection{geschlossenes Hashing mit offener Adressierung}

Wenn dabei ein Eintrag an einer schon belegten Stelle in der
Tabelle abgelegt werden soll, wird stattdessen eine \memph{andere freie
Stelle genommen}. Häufig werden drei Varianten unterschieden:

\subsubsection{lineares Sondieren}

es wird um ein \memph{konstantes Intervall} verschoben nach einer freien
Stelle gesucht. Meistens wird die Intervallgröße auf 1 festgelegt.

\subsubsection{quadratisches Sondieren}

Nach jedem erfolglosen Suchschritt wird das \memph{Intervall quadriert}.

\ueberschrift{Beispiele}

\begin{itemize}

%%
%
%%

\item \ueberschrift{nach Foliensatz der TU Braunschweig}
\footnote{Seite 25\url{https://www.ibr.cs.tu-bs.de/courses/ws0708/aud/skript/hash.np.pdf}}

\ueberschrift{Formel}

$h(k, i) := h'(k) + (-1)^{i+1} \cdot \left\lfloor \frac{i+1}{2}\right\rfloor ^2 \mod m$

$k$, $k+1^2$, $k-1^2$, $k+2^2$, $k-2^2$,
$\ldots$
$k+(\frac{m-1}{2})^2$, $k-(\frac{m-1}{2})^2 \bmod m$

\ueberschrift{Werte}

$m=19$, d. h. das Feld (die Tabelle) hat die Index-Nummern 0 bis 18.
$k = h(x) = 7$

\ueberschrift{Sondierungsfolgen}

\def\tmp#1{{\tiny#1}}

\begin{tabular}{|l|l|l|l|}
i & Rechnung & Ergebnis & Index in der Tabelle\\\hline\hline
0 & $7 + 0^2$ & 7 & 7\\
1 & $7 + 1^2$ & 8 & 8\\
1 & $7 - 1^2$ & 6 & 6\\
2 & $7 + 2^2$ & 11 & 11\\
2 & $7 - 2^2$ & 3 & 2 \\
3 & $7 + 3^2 = 7 + 9$  & 16 & 16 \\

3 & $7 - 3^2 = 7 - 9$  & -2 &
17 \tmp{($19-2=10$) oder ($0 \rightarrow 0$, $-1 \rightarrow 18$, $-2 \rightarrow 17$)}
\\

4 & $7 + 4^2 = 7 + 16$  & 23 &
4 \tmp{($23-19=4$) oder ($19 \rightarrow 0$, $20 \rightarrow 1$, $21 \rightarrow 2$, $22 \rightarrow 3$, $23 \rightarrow 4$)}
\\

4 & $7 - 4^2 = 7 - 16$  & -9 &
10 \tmp{($19-9=10$) oder ($0 \rightarrow 0$, $-1 \rightarrow 18$, $-2 \rightarrow 17$, $\cdots$, $-9 \rightarrow 10$)}
\\

5 & $7 + 5^2 = 7 + 25$  & 32 & 13 \tmp{($32-19=13$)} \\
5 & $7 - 5^2 = 7 - 25$  & -18 & 1 \tmp{($19-18=1$)}\\
\end{tabular}

%%
%
%%

\item \ueberschrift{nach Foliensatz der RWTH Aachen}
\footnote{Seite 19 \url{https://moves.rwth-aachen.de/wp-content/uploads/SS15/dsal/lec13.pdf}}

$h'(k) = k \mod 11$

$h(k, i) = (h'(k) + i + 3i^2) \mod 11$

$h'(17) = 17 \mod 11 = 6$

\ueberschrift{Sondierungsfolgen}

$h(17, 0) = (17 + 0 + 3 \cdot 0^2) \mod 11 = 6$

$h(17, 1) = (17 + 1 + 3 \cdot 1^2) \mod 11 = 21 \mod 11 = 10$

$h(17, 2) = (17 + 2 + 3 \cdot 2^2) \mod 11 = 31 \mod 11 = 31 - 2 \cdot 11 = 9$

$h(17, 3) = (17 + 3 + 3 \cdot 3^2) \mod 11 = 47 \mod 11 = 47 - 4 \cdot 11 = 3$

\end{itemize}

\subsubsection{doppeltes Hashen}

eine \memph{weitere Hash-Funktion} liefert das Intervall.

%%
%
%%

\subsection{offenes Hashing mit geschlossener Adressierung}

Anstelle der gesuchten Daten enthält die Hashtabelle hier
\memph{Behälter} (englisch \memph{Buckets}), die alle Daten mit gleichem
Hash-Wert aufnehmen. Es müssen die \memph{Elemente im Behälter
durchsucht werden}. Oft wird die Verkettung durch eine lineare Liste pro
Behälter realisiert.
\footcite{wiki:hashtabelle}

%-----------------------------------------------------------------------
%
%-----------------------------------------------------------------------

\section{Die Divisionsrestmethode\footcite{wiki:divisionsrestmethode}}

Die Divisionsrestmethode - auch Modulo genannt liefert eine
Hashfunktion. Die Funktion lautet:  $h(k)=k{\bmod {m}}$. $m$ ist die
Größe der Hashtabelle. Die Hash-Funktion kann sehr schnell berechnet
werden. Die Wahl der Tabellengröße $m$ beeinflusst die
Kollisionswahrscheinlichkeit der Funktionswerte von $h$. Für
praxisrelevante Anwendungen liefert die Wahl einer Primzahl für $m$.

\section{Belegungsfaktor\footcite[Seite 29]{aud:fs:tafeluebung-10}}

$\text{Belegungsfaktor} =
\frac{\text{Anzahl tatsächlich eingetragenen Schlüssel}}
{\text{Anzahl Hashwerte}}$

%%%%%%%%%%%%%%%%%%%%%%%%%%%%%%%%%%%%%%%%%%%%%%%%%%%%%%%%%%%%%%%%%%%%%%%%
% Aufgaben
%%%%%%%%%%%%%%%%%%%%%%%%%%%%%%%%%%%%%%%%%%%%%%%%%%%%%%%%%%%%%%%%%%%%%%%%

\chapter{Aufgaben}

\section{Aufgabe zum Hashing\footcite[Seite 1]{aud:ab:5}}

\begin{enumerate}

%%
% (a)
%%

\item Ist $h(k) = k^2 \mod 11$ eine gut gewählte Hashfunktion? Begründen
Sie Ihre Antwort.

Tipp: Berechnen Sie zunächst $h(k)$ für $0 \leq k < 11$. Überlegen Sie
dann, welche Werte $h(k')$ für $k' = a \cdot 11 + k$ mit $a > 0$ und $0
\leq k < 11$ annehmen kann.

\begin{antwort}

\begin{center}
\begin{tabular}{|c|c|}
\hline
$k$ & $h(k)$\\\hline\hline
0 & 0\\\hline
1 & 1\\\hline
2 & 4\\\hline
3 & 9\\\hline
4 & 5\\\hline
5 & 3\\\hline
6 & 3\\\hline
7 & 5\\\hline
8 & 9\\\hline
9 & 4\\\hline
10 & 1\\\hline
\end{tabular}
\end{center}

Nein, $h$ ist keine gute Hashfunktion. Betrachten wir zunächst die
Wertetabelle von $h$ für $0 \leq k < 11$ . Wir erhalten

Wir sehen, dass nie die Werte $2$, $6$, $7$, $8$ und $10$ eingenommen
werden. Man könnte nun noch hoffen, dass das vielleicht für irgendein
größeres $k$ der Fall ist, dem ist jedoch nicht so. Wir können uns
leicht davon überzeugen, dass für ein beliebiges $k' = a \cdot 11 + k$
mit $a > 0$ und $0 \leq k < 11$ folgendes gilt:

\begin{align*}
h(k')
  &= (k')^2 mod 11\\
  &= (a \cdot 11 + k)^2 mod 11\\
  &= (a^2 \cdot 11^2 + 2ak \cdot 11 + k^2) mod 11\\
  &= (k^2) mod 11\\
  &= h(k)
\end{align*}

Somit haben wir die Berechnung des Hashwertes für ein beliebiges $k'$
auf die Berechnung des Hashwertes für ein $k < 11$ zurückgeführt, was
impliziert, dass kein Schlüssel jemals auf etwas anderes als $0$, $1$,
$3$, $4$, $5$ oder $9$ abgebildet werden kann.
\end{antwort}

%%
% (b)
%%

\item  Die Schlüssel $23$, $57$, $26$, $6$, $77$, $43$, $74$, $60$, $9$,
$91$ sollen in dieser Reihenfolge mit der Hashfunktion $h(k) = k \mod
17$ in eine Hashtabelle der Länge $17$ eingefügt werden.

\begin{antwort}
\begin{exkurs}[Sondieren]
\begin{description}
\item[separate Verkettung]
Kollisionsauflösung durch Verkettung (separate chaining): Jedes Bucket
speichert mit Hilfe einer dynamischen Datenstruktur (Liste, Baum,
weitere Streutabelle, ...) alle Elemente mit dem entsprechenden
Hashwert.\footcite[Seite 32]{aud:fs:tafeluebung-10}

\item[lineares Sondieren]
es wird um ein konstantes Intervall verschoben nach einer freien Stelle
gesucht. Meistens wird die Intervallgröße auf 1 festgelegt.

\item[quadratisches Sondieren]
Nach jedem erfolglosen Suchschritt wird das Intervall
quadriert.\footcite{wiki:hashtabelle}
\end{description}
\end{exkurs}
\end{antwort}

\begin{enumerate}

%%
% 1.
%%

\item Verwenden Sie separate Verkettung zur Kollisionsauflösung.

\begin{antwort}
Nebenrechnung:

$17 \cdot 1 = 17$\\
$17 \cdot 2 = 34$\\
$17 \cdot 3 = 51$\\
$17 \cdot 4 = 68$\\
$17 \cdot 5 = 85$

Modulo-Berechnung der gegebenen Zahlen:

$23 \mod 17 = \textbf{6} \text{ da } 23 : 17 = 1 \text{, Rest } 6 \text{ da } 23 = 1 \cdot 17 + 6$\\
$57 \mod 17 = 57 - 3 \cdot 17 = 57 - 51 = \textbf{6}$\\
$26 \mod 17 = 26 - 17 = \textbf{9}$\\
$6 \mod 17 = 6 - 0 \cdot 17 = \textbf{6}$\\
$77 \mod 17 = 77 - 4 \cdot 17 = 77 - 68 = \textbf{9}$\\
$43 \mod 17 = \textbf{9}$\\
$74 \mod 17 = \textbf{6}$\\
$60 \mod 17 = \textbf{9}$\\
$9 \mod 17 = \textbf{9}$\\
$91 \mod 17 = \textbf{6}$\\

{
\setlength{\tabcolsep}{2pt}
\footnotesize
\begin{tabular}{r|ccccccccccccccccc}
Index & 0 & 1 & 2 & 3 & 4 & 5 & 6 & 7 & 8 & 9 & 10 & 11 & 12 & 13 & 14 & 15 & 16 \\\hline
Schlüssel &&&&&&&23&&&26&&&&&&&\\
          &&&&&&&57&&&77&&&&&&&\\
          &&&&&&&6 &&&43&&&&&&&\\
          &&&&&&&74&&&60&&&&&&&\\
          &&&&&&&91&&&9 &&&&&&&\\
\end{tabular}
}
\end{antwort}

%%
% 2.
%%

\item Verwenden Sie lineares Sondieren zur Kollisionsauflösung.

\def\tmp#1{{\tiny($#1$)}}

\begin{antwort}
\def\tmp#1{{\footnotesize#1}}

Die Hashfunkion lautet:

\tmp{$h'(k) = k \mod 17$}

Die verwendete Hashfunktion beim linearen Sondieren:

\tmp{$h(k, i) = (h'(k) - i) \mod 17$}

\bigskip

Es ergeben sich folgende Sondierungsfolgen:
{
\setlength{\tabcolsep}{2pt}
\footnotesize

\begin{tabular}{r|ccccccccc}
Schlüssel & \multicolumn{6}{l}{Index}\\\hline
23 & 6\\
57 & 6 & 5\\
26 & 9\\
6  & 6 & 5 & 4\\
77 & 9 & 8\\
43 & 9 & 8 & 7\\
74 & 6 & 5 & 4 & 3 \\
60 & 9 & 8 & 7 & 6 & 5 & 4 & 3 & 2\\
9  & 9 & 8 & 7 & 6 & 5 & 4 & 3 & 2 & 1 \\
91 & 6 & 5 & 4 & 3 & 2 & 1\\
\end{tabular}

Damit ergibt sich folgende Hashtabelle:

\begin{tabular}{r|ccccccccccccccccc}
Index & 0 & 1 & 2 & 3 & 4 & 5 & 6 & 7 & 8 & 9 & 10 & 11 & 12 & 13 & 14 & 15 & 16 \\\hline
Schlüssel &91&9&60&74&6&57&23&43&77&26&&&&&&&\\
\end{tabular}
}
\end{antwort}

%%
% 3.
%%

\item Verwenden Sie quadratisches Sondieren zur Kollisionsauflösung.

\begin{antwort}

\def\tmp#1{{\footnotesize#1}}

Die Hashfunkion lautet:

\tmp{$h'(k) = k \mod 17$}

Die verwendete Hashfunktion beim quadratischen Sondieren:

\tmp{$h(k, i) = (h'(k) + i^2) \mod 17$}

\bigskip

Am Beispiel von zwei Schlüsseln werden die Sondierungsfolgen berechnet:

\bigskip

\ueberschrift{$h'(23) = 6$}

\begin{enumerate}
\item Sondierungsfolge: \\
\tmp{$h(23, 0) = (h'(23) + 0^2) \mod 17 = (6 + 0) \mod 17 = 6 \mod 17 = \textbf{6}$}

\item Sondierungsfolge: \\
\tmp{$h(23, 1) = (h'(23) + 1^2) \mod 17 = (6 + 1) \mod 17 = 7 \mod 17 = \textbf{7}$}

\item Sondierungsfolge: \\
\tmp{$h(23, 2) = (h'(23) + 2^2) \mod 17 = (6 + 4) \mod 17 = 10 \mod 17 = \textbf{10}$}

\item Sondierungsfolge: \\
\tmp{$h(23, 3) = (h'(23) + 3^2) \mod 17 = (6 + 9) \mod 17 = 15 \mod 17 = \textbf{15}$}

\item Sondierungsfolge: \\
\tmp{$h(23, 4) = (h'(23) + 4^2) \mod 17 = (6 + 16) \mod 17 = 22 \mod 17 = \textbf{5}$}
\end{enumerate}

\bigskip

\ueberschrift{$h'(26) = 9$}

\begin{enumerate}
\item Sondierungsfolge: \\
\tmp{$h(26, 0) = (h'(26) + 0^2) \mod 17 = (9 + 0) \mod 17 = 9 \mod 17 = \textbf{9}$}

\item Sondierungsfolge: \\
\tmp{$h(26, 1) = (h'(26) + 1^2) \mod 17 = (9 + 1) \mod 17 = 10 \mod 17 = \textbf{10}$}

\item Sondierungsfolge: \\
\tmp{$h(26, 2) = (h'(26) + 2^2) \mod 17 = (9 + 4) \mod 17 = 13 \mod 17 = \textbf{13}$}

\item Sondierungsfolge: \\
\tmp{$h(26, 3) = (h'(26) + 3^2) \mod 17 = (9 + 9) \mod 17 = 18 \mod 17 = \textbf{1}$}

\item Sondierungsfolge: \\
\tmp{$h(26, 4) = (h'(26) + 4^2) \mod 17 = (9 + 16) \mod 17 = 25 \mod 17 = \textbf{8}$}

\item Sondierungsfolge: \\
\tmp{$h(26, 5) = (h'(26) + 5^2) \mod 17 = (9 + 25) \mod 17 = 34 \mod 17 = \textbf{0}$}
\end{enumerate}

\bigskip

Es ergeben sich folgende Sondierungsfolgen:

{
\setlength{\tabcolsep}{2pt}
\footnotesize

\begin{tabular}{r|cccccc}
Schlüssel & \multicolumn{6}{l}{Index}\\\hline
23 & 6\\
57 & 6 & 7\\
26 & 9\\
6  & 6 & 7 & 10 \\
77 & 9 & 10 & 13 \\
43 & 9 & 10 & 13 & 1 \\
74 & 6 & 7 & 10 & 15\\
60 & 9 & 10 & 13 & 1 & 8 \\
9  & 9 & 10 & 13 & 1 & 8 & 0\\
91 & 6 & 7 & 10 & 15 & 5\\
\end{tabular}

Damit ergibt sich folgende Hashtabelle:

\begin{tabular}{r|ccccccccccccccccc}
Index & 0 & 1 & 2 & 3 & 4 & 5 & 6 & 7 & 8 & 9 & 10 & 11 & 12 & 13 & 14 & 15 & 16 \\\hline
Schlüssel &9&43&&&&91&23&57&60&26&6&&&77&&74&\\
\end{tabular}
}
\end{antwort}
\end{enumerate}
\end{enumerate}

%-----------------------------------------------------------------------
%
%-----------------------------------------------------------------------

\ExamensAufgabeTA 66112 / 2005 / 03 : Thema 2 Aufgabe 8

%-----------------------------------------------------------------------
%
%-----------------------------------------------------------------------

\ExamensAufgabeTA 46115 / 2013 / 03 : Thema 2 Aufgabe 4

%-----------------------------------------------------------------------
%
%-----------------------------------------------------------------------

\ExamensAufgabeTA 66115 / 2019 / 09 : Thema 2 Aufgabe 9

%-----------------------------------------------------------------------
%
%-----------------------------------------------------------------------

\ExamensAufgabeTA 46115 / 2015 / 09 : Thema 2 Aufgabe 1

%-----------------------------------------------------------------------
%
%-----------------------------------------------------------------------

\ExamensAufgabeTA 66115 / 2016 / 03 : Thema 2 Aufgabe 4

%-----------------------------------------------------------------------
%
%-----------------------------------------------------------------------

\ExamensAufgabeTA 66115 / 2010 / 09 : Thema 2 Aufgabe 3

\literatur

\end{document}
