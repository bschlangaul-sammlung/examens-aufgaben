\documentclass{lehramt-informatik-haupt}
\InformatikPakete{syntax,uml,baum}
\usepackage{amsmath}
\usepackage{tabularx}

\begin{document}

%%%%%%%%%%%%%%%%%%%%%%%%%%%%%%%%%%%%%%%%%%%%%%%%%%%%%%%%%%%%%%%%%%%%%%%%
% Theorie-Teil
%%%%%%%%%%%%%%%%%%%%%%%%%%%%%%%%%%%%%%%%%%%%%%%%%%%%%%%%%%%%%%%%%%%%%%%%

\chapter{Binärer Suchbaum}

\begin{quellen}
\item \cite{wiki:binaerbaum}
\item \cite{wiki:binaerer-suchbaum}
\item \cite[Kapitel 14.2.1, Seite 348-358 (PDF 364-374)]{saake}
\item \cite[Seite 4-14]{aud:fs:5}
\end{quellen}

%-----------------------------------------------------------------------
%
%-----------------------------------------------------------------------

\section{Visualisierungstools}

\begin{itemize}
\item \url{http://btv.melezinek.cz/binary-search-tree.html} Auch für Postorder etc
\item \url{https://visualgo.net/bn/bst}
\end{itemize}

\noindent
Ein binärer Suchbaum - häufig abgekürzt als \textbf{BST} (von englisch
\emph{Binary Search Tree}) - ist ein binärer Baum, bei dem die Knoten
\emph{„Schlüssel“} tragen, und die Schlüssel des \memph{linken
Teilbaums} eines Knotens nur \memph{kleiner (oder gleich)} und die des
\memph{rechten} Teilbaums nur \memph{größer}\footcite[Seite 5]{aud:fs:5}
als der Schlüssel des Knotens selbst sind.
\footcite{wiki:binaerer-suchbaum}

\begin{center}
\begin{tabular}{ll}
$S$ &
Wert eines Knotens \\

$S_\text{links}$ &
Wert der Wurzel des linken Teilbaumes \\

$S_\text{rechts}$ &
Wert der Wurzel des rechten Teilbaumes \\
\end{tabular}
\end{center}

Bei einem binären Suchbaum gilt:
$\text{S}_\text{links} \leq \text{S} < \text{S}_\text{recht}$

\section{Beispiel}

Anordnung von \verb|7, 3, 6, 9, 2, 9, 5, 1| in einem int-Baum
\footcite[Seite 5]{aud:fs:5}

\begin{center}
\begin{tikzpicture}[binaerer baum]
\Tree
[.7
  [.3
    [.2
      \edge[]; {1}
      \edge[blank]; \node[blank]{};
    ]
    [.6
      \edge[]; {5}
      \edge[blank]; \node[blank]{};
    ]
  ]
  [.9
    \edge[]; {9}
    \edge[blank]; \node[blank]{};
  ]
]
\end{tikzpicture}
\end{center}

%%
%
%%

\subsection{Traversierung eines Baums (7, 3, 6, 9, 2, 8, 5, 1)}

Es gibt drei verschiedene Möglichkeiten einen binären Baum zu
traversieren:

\begin{itemize}
\item \texttt{preorder}:
besuche die Wurzel, dann den linken Unterbaum, dann den rechten
Unterbaum; auch: WLR

\item \texttt{inorder}:
besuche den linken Unterbaum, dann die Wurzel, dann den rechten
Unterbaum;
auch: LWR
\item \texttt{postorder}:
besuche den linken Unterbaum, dann den rechten, dann die Wurzel;
auch: LRW
\end{itemize}

\begin{center}
\begin{tikzpicture}[binaerer baum]
\Tree
[.7
  [.3
    [.2
      [.1 ]
      \edge[blank]; \node[blank]{};
    ]
    [.6
      [.5 ]
      \edge[blank]; \node[blank]{};
    ]
  ]
  [.9
    [.8 ]
    \edge[blank]; \node[blank]{};
  ]
]
\end{tikzpicture}
\end{center}

\begin{center}
\begin{tabular}{ll}
\texttt{preorder} & \texttt{7 3 2 1 6 5 9 8} \\
\texttt{inorder} & \texttt{1 2 3 5 6 7 8 9} \\
\texttt{postorder} & \texttt{1 2 5 6 3 8 9 7} \\
\end{tabular}
\end{center}

%-----------------------------------------------------------------------
%
%-----------------------------------------------------------------------

\section{Entfernen eines Elements}

%%
%
%%

\subsection{Knoten ohne Nachfolger\footcite[Seite 10]{aud:fs:5}}

\begin{compactenum}
\item Suche den zu löschenden Knoten
\item Lösche die Referenz vom Vorgängerkonten auf den zu löschenden Knoten
\end{compactenum}

\begin{minipage}{0.5\linewidth}
\begin{tikzpicture}[binaerer baum]
\Tree
[.7
  [.3
    [.2
      [.1 ]
      \edge[blank]; \node[blank]{};
    ]
    [.6
      [.5 ]
      \edge[blank]; \node[blank]{};
    ]
  ]
  [.11
    \edge[dotted]; \node[dotted]{9};
    \edge[blank]; \node[blank]{};
  ]
]
\end{tikzpicture}
\end{minipage}
\begin{minipage}{0.5\linewidth}
\begin{tikzpicture}[binaerer baum]
\Tree
[.7
  [.3
    [.2
      [.1 ]
      \edge[blank]; \node[blank]{};
    ]
    [.6
      [.5 ]
      \edge[blank]; \node[blank]{};
    ]
  ]
  [.11 ]
]
\end{tikzpicture}
\end{minipage}

%%
%
%%

\subsubsection{Knoten mit genau einem Nachfolger\footcite[Seite 11]{aud:fs:5}}

\begin{compactenum}
\item Suche den zu löschenden Knoten
\item Referenz von Vorgängerkonten auf den (einzigen) Nachfolgerknoten
des zu löschenden Knotens
\end{compactenum}

\begin{minipage}{0.5\linewidth}
\begin{tikzpicture}[binaerer baum]
\Tree
[.7
  [.\node(3){3};
    [.2
      [.1 ]
      \edge[blank]; \node[blank]{};
    ]
    [.\node[dotted]{6};
      [.\node(5){5}; ]
      \edge[blank]; \node[blank]{};
    ]
  ]
  [.11
    [.9 ]
    \edge[blank]; \node[blank]{};
  ]
]
\draw (3) -- (5);
\end{tikzpicture}
\end{minipage}
\begin{minipage}{0.5\linewidth}
\begin{tikzpicture}[binaerer baum]
\Tree
[.7
  [.3
    [.2
      [.1 ]
      \edge[blank]; \node[blank]{};
    ]
    [.5 ]
  ]
  [.11
    [.9 ]
    \edge[blank]; \node[blank]{};
  ]
]
\end{tikzpicture}
\end{minipage}
%%
%
%%

\subsection{Knoten mit zwei Nachfolgern\footcite[Seite 12]{aud:fs:5}}

\begin{compactenum}
\item Suche den zu löschenden Knoten
\item Suche den „kleinsten“ Knoten im rechten Teilbaum
\item Ersetze den zu löschenden Knoten durch den „kleinsten“ Knoten
\end{compactenum}

\begin{minipage}{0.5\linewidth}
\begin{tikzpicture}[binaerer baum]
\Tree
[.\node(7){7};
  [.\node[dotted](3){3};
    [.\node(2){2};
      [.1 ]
      \edge[blank]; \node[blank]{};
    ]
    [.\node(6){6};
      [.\node(5){5}; ]
      \edge[blank]; \node[blank]{};
    ]
  ]
  [.11
    [.9 ]
    \edge[blank]; \node[blank]{};
  ]
]
\draw[dashed] (7) ..controls +(south east:5.1).. (5);
\draw[dashed] (5) ..controls +(east:0.7).. (6);
\draw[dashed] (5) ..controls +(south west:0.7).. (2);
\end{tikzpicture}
\end{minipage}
\begin{minipage}{0.5\linewidth}
\begin{tikzpicture}[binaerer baum]
\Tree
[.7
  [.5
    [.2
      [.1 ]
      \edge[blank]; \node[blank]{};
    ]
    [.6 ]
  ]
  [.11
    [.9 ]
    \edge[blank]; \node[blank]{};
  ]
]
\end{tikzpicture}
\end{minipage}

%%%%%%%%%%%%%%%%%%%%%%%%%%%%%%%%%%%%%%%%%%%%%%%%%%%%%%%%%%%%%%%%%%%%%%%%
% Aufgaben
%%%%%%%%%%%%%%%%%%%%%%%%%%%%%%%%%%%%%%%%%%%%%%%%%%%%%%%%%%%%%%%%%%%%%%%%

\chapter{Aufgaben}

%-----------------------------------------------------------------------
%
%-----------------------------------------------------------------------

\section{Aufgabe 1: Binärbaum\footcite[Seite 1]{aud:pu:5}}

\begin{enumerate}

%%
% (a)
%%

\item Erstellen Sie ein Klassendiagramm für einen Binärbaum.

\begin{antwort}
\ueberschrift{einfacher Binärbaum}

\begin{tikzpicture}
\umlsimpleclass{Binärbaum}
\umlclass[x=4]{Knoten}{wert}{}
\umlaggreg[mult=0..1,arg=wurzel,pos=0.6]{Binärbaum}{Knoten}
\umlaggreg[mult=0..2,pos=0.8]{Knoten}{Knoten}
\end{tikzpicture}

\ueberschrift{Binärbaum mit Kompositum}

\begin{tikzpicture}
\umlsimpleclass[x=-1,y=5]{Binärbaum}
\umlsimpleclass[x=3,y=5,type=abstrakt]{Baumelement}
\umlsimpleclass[x=1,y=3]{Abschluss}
\umlsimpleclass[x=5,y=3]{Datenknoten}
\umlsimpleclass[x=5,y=1]{Inhalt}

\umlassoc[mult2=1,stereo=wurzel \LeserichtungRechts]{Binärbaum}{Baumelement}
\umlVHVinherit{Abschluss}{Baumelement}
\umlVHVinherit{Datenknoten}{Baumelement}

\umlaggreg[mult=1]{Datenknoten}{Inhalt}
\umlVHaggreg[mult2=2,pos2=1.8,anchor1=50]{Datenknoten}{Baumelement}
\end{tikzpicture}
\end{antwort}

%%
% (b)
%%

\item Entwerfen Sie eine mögliche Implementierung zur Erzeugung eines
binären Baumes in Java.

\begin{antwort}
\ueberschrift{einfacher Binärbaum}

\inputcode[firstline=3]{baum/einfach/Baum}
\inputcode[firstline=3]{baum/einfach/Knoten}

\ueberschrift{Binärbaum mit Kompositum}

\inputcode[firstline=3]{baum/kompositum/Ahnenbaum}
\inputcode[firstline=3]{baum/kompositum/Baumelement}
\inputcode[firstline=3]{baum/kompositum/Abschluss}
\inputcode[firstline=3]{baum/kompositum/Datenknoten}
\inputcode[firstline=3]{baum/kompositum/Person}
\end{antwort}
\end{enumerate}

%-----------------------------------------------------------------------
%
%-----------------------------------------------------------------------

\ExamensAufgabeTA 66112 / 2003 / 09 : Thema 2 Aufgabe 8

%-----------------------------------------------------------------------
%
%-----------------------------------------------------------------------

\ExamensAufgabeTA 46115 / 2014 / 09 : Thema 2 Aufgabe 3

\literatur

\end{document}
