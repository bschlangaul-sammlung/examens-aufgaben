\documentclass{lehramt-informatik}
\InformatikPakete{syntax,uml,baum}
\usepackage{amsmath}
\usepackage{tabularx}

\begin{document}

%%%%%%%%%%%%%%%%%%%%%%%%%%%%%%%%%%%%%%%%%%%%%%%%%%%%%%%%%%%%%%%%%%%%%%%%
% Theorie-Teil
%%%%%%%%%%%%%%%%%%%%%%%%%%%%%%%%%%%%%%%%%%%%%%%%%%%%%%%%%%%%%%%%%%%%%%%%

\chapter{Binärer Suchbaum}

\begin{quellen}
\item \cite{wiki:binaerbaum}
\item \cite{wiki:binaerer-suchbaum}
\item \cite[Kapitel 14.2.1, Seite 348-358 (PDF 364-374)]{saake}
\item \cite[Seite 4-14]{aud:fs:5}
\end{quellen}

%-----------------------------------------------------------------------
%
%-----------------------------------------------------------------------

\section{Visualisierungstools}

\begin{itemize}
\item \url{http://btv.melezinek.cz/binary-search-tree.html} Auch für Postorder etc
\item \url{https://visualgo.net/bn/bst}
\end{itemize}

\noindent
Ein binärer Suchbaum - häufig abgekürzt als \textbf{BST} (von englisch
\emph{Binary Search Tree}) - ist ein binärer Baum, bei dem die Knoten
\emph{„Schlüssel“} tragen, und die Schlüssel des \memph{linken
Teilbaums} eines Knotens nur \memph{kleiner (oder gleich)} und die des
\memph{rechten} Teilbaums nur \memph{größer}\footcite[Seite 5]{aud:fs:5}
als der Schlüssel des Knotens selbst sind.
\footcite{wiki:binaerer-suchbaum}

\begin{center}
\begin{tabular}{ll}
$S$ &
Wert eines Knotens \\

$S_\text{links}$ &
Wert der Wurzel des linken Teilbaumes \\

$S_\text{rechts}$ &
Wert der Wurzel des rechten Teilbaumes \\
\end{tabular}
\end{center}

Bei einem binären Suchbaum gilt:
$\text{S}_\text{links} \leq \text{S} < \text{S}_\text{recht}$

\section{Beispiel}

Anordnung von \verb|7, 3, 6, 9, 2, 9, 5, 1| in einem int-Baum
\footcite[Seite 5]{aud:fs:5}

\begin{center}
\begin{tikzpicture}[binaerer baum]
\Tree
[.7
  [.3
    [.2
      \edge[]; {1}
      \edge[blank]; \node[blank]{};
    ]
    [.6
      \edge[]; {5}
      \edge[blank]; \node[blank]{};
    ]
  ]
  [.9
    \edge[]; {9}
    \edge[blank]; \node[blank]{};
  ]
]
\end{tikzpicture}
\end{center}

%%
%
%%

\subsection{Traversierung eines Baums (7, 3, 6, 9, 2, 8, 5, 1)}

Es gibt drei verschiedene Möglichkeiten einen binären Baum zu
traversieren:

\begin{itemize}
\item \texttt{preorder}:
besuche die Wurzel, dann den linken Unterbaum, dann den rechten
Unterbaum; auch: WLR

\item \texttt{inorder}:
besuche den linken Unterbaum, dann die Wurzel, dann den rechten
Unterbaum;
auch: LWR
\item \texttt{postorder}:
besuche den linken Unterbaum, dann den rechten, dann die Wurzel;
auch: LRW
\end{itemize}

\begin{center}
\begin{tikzpicture}[binaerer baum]
\Tree
[.7
  [.3
    [.2
      [.1 ]
      \edge[blank]; \node[blank]{};
    ]
    [.6
      [.5 ]
      \edge[blank]; \node[blank]{};
    ]
  ]
  [.9
    [.8 ]
    \edge[blank]; \node[blank]{};
  ]
]
\end{tikzpicture}
\end{center}

\begin{center}
\begin{tabular}{ll}
\texttt{preorder} & \texttt{7 3 2 1 6 5 9 8} \\
\texttt{inorder} & \texttt{1 2 3 5 6 7 8 9} \\
\texttt{postorder} & \texttt{1 2 5 6 3 8 9 7} \\
\end{tabular}
\end{center}

%-----------------------------------------------------------------------
%
%-----------------------------------------------------------------------

\section{Entfernen eines Elements}

%%
%
%%

\subsection{Knoten ohne Nachfolger\footcite[Seite 10]{aud:fs:5}}

\begin{compactenum}
\item Suche den zu löschenden Knoten
\item Lösche die Referenz vom Vorgängerkonten auf den zu löschenden Knoten
\end{compactenum}

\begin{minipage}{0.5\linewidth}
\begin{tikzpicture}[binaerer baum]
\Tree
[.7
  [.3
    [.2
      [.1 ]
      \edge[blank]; \node[blank]{};
    ]
    [.6
      [.5 ]
      \edge[blank]; \node[blank]{};
    ]
  ]
  [.11
    \edge[dotted]; \node[dotted]{9};
    \edge[blank]; \node[blank]{};
  ]
]
\end{tikzpicture}
\end{minipage}
\begin{minipage}{0.5\linewidth}
\begin{tikzpicture}[binaerer baum]
\Tree
[.7
  [.3
    [.2
      [.1 ]
      \edge[blank]; \node[blank]{};
    ]
    [.6
      [.5 ]
      \edge[blank]; \node[blank]{};
    ]
  ]
  [.11 ]
]
\end{tikzpicture}
\end{minipage}

%%
%
%%

\subsubsection{Knoten mit genau einem Nachfolger\footcite[Seite 11]{aud:fs:5}}

\begin{compactenum}
\item Suche den zu löschenden Knoten
\item Referenz von Vorgängerkonten auf den (einzigen) Nachfolgerknoten
des zu löschenden Knotens
\end{compactenum}

\begin{minipage}{0.5\linewidth}
\begin{tikzpicture}[binaerer baum]
\Tree
[.7
  [.\node(3){3};
    [.2
      [.1 ]
      \edge[blank]; \node[blank]{};
    ]
    [.\node[dotted]{6};
      [.\node(5){5}; ]
      \edge[blank]; \node[blank]{};
    ]
  ]
  [.11
    [.9 ]
    \edge[blank]; \node[blank]{};
  ]
]
\draw (3) -- (5);
\end{tikzpicture}
\end{minipage}
\begin{minipage}{0.5\linewidth}
\begin{tikzpicture}[binaerer baum]
\Tree
[.7
  [.3
    [.2
      [.1 ]
      \edge[blank]; \node[blank]{};
    ]
    [.5 ]
  ]
  [.11
    [.9 ]
    \edge[blank]; \node[blank]{};
  ]
]
\end{tikzpicture}
\end{minipage}
%%
%
%%

\subsection{Knoten mit zwei Nachfolgern\footcite[Seite 12]{aud:fs:5}}

\begin{compactenum}
\item Suche den zu löschenden Knoten
\item Suche den „kleinsten“ Knoten im rechten Teilbaum
\item Ersetze den zu löschenden Knoten durch den „kleinsten“ Knoten
\end{compactenum}

\begin{minipage}{0.5\linewidth}
\begin{tikzpicture}[binaerer baum]
\Tree
[.\node(7){7};
  [.\node[dotted](3){3};
    [.\node(2){2};
      [.1 ]
      \edge[blank]; \node[blank]{};
    ]
    [.\node(6){6};
      [.\node(5){5}; ]
      \edge[blank]; \node[blank]{};
    ]
  ]
  [.11
    [.9 ]
    \edge[blank]; \node[blank]{};
  ]
]
\draw[dashed] (7) ..controls +(south east:5.1).. (5);
\draw[dashed] (5) ..controls +(east:0.7).. (6);
\draw[dashed] (5) ..controls +(south west:0.7).. (2);
\end{tikzpicture}
\end{minipage}
\begin{minipage}{0.5\linewidth}
\begin{tikzpicture}[binaerer baum]
\Tree
[.7
  [.5
    [.2
      [.1 ]
      \edge[blank]; \node[blank]{};
    ]
    [.6 ]
  ]
  [.11
    [.9 ]
    \edge[blank]; \node[blank]{};
  ]
]
\end{tikzpicture}
\end{minipage}

%%%%%%%%%%%%%%%%%%%%%%%%%%%%%%%%%%%%%%%%%%%%%%%%%%%%%%%%%%%%%%%%%%%%%%%%
% Aufgaben
%%%%%%%%%%%%%%%%%%%%%%%%%%%%%%%%%%%%%%%%%%%%%%%%%%%%%%%%%%%%%%%%%%%%%%%%

\chapter{Aufgaben}

%-----------------------------------------------------------------------
%
%-----------------------------------------------------------------------

\section{Aufgabe 1: Binärbaum\footcite[Seite 1]{aud:pu:5}}

\begin{enumerate}

%%
% (a)
%%

\item Erstellen Sie ein Klassendiagramm für einen Binärbaum.

\begin{antwort}
\ueberschrift{einfacher Binärbaum}

\begin{tikzpicture}
\umlsimpleclass{Binärbaum}
\umlclass[x=4]{Knoten}{wert}{}
\umlaggreg[mult=0..1,arg=wurzel,pos=0.6]{Binärbaum}{Knoten}
\umlaggreg[mult=0..2,pos=0.8]{Knoten}{Knoten}
\end{tikzpicture}

\ueberschrift{Binärbaum mit Kompositum}

\begin{tikzpicture}
\umlsimpleclass[x=-1,y=5]{Binärbaum}
\umlsimpleclass[x=3,y=5,type=abstrakt]{Baumelement}
\umlsimpleclass[x=1,y=3]{Abschluss}
\umlsimpleclass[x=5,y=3]{Datenknoten}
\umlsimpleclass[x=5,y=1]{Inhalt}

\umlassoc[mult2=1,stereo=wurzel \LeserichtungRechts]{Binärbaum}{Baumelement}
\umlVHVinherit{Abschluss}{Baumelement}
\umlVHVinherit{Datenknoten}{Baumelement}

\umlaggreg[mult=1]{Datenknoten}{Inhalt}
\umlVHaggreg[mult2=2,pos2=1.8,anchor1=50]{Datenknoten}{Baumelement}
\end{tikzpicture}
\end{antwort}

%%
% (b)
%%

\item Entwerfen Sie eine mögliche Implementierung zur Erzeugung eines
binären Baumes in Java.

\begin{antwort}
\ueberschrift{einfacher Binärbaum}

\inputcode[firstline=3]{baum/einfach/Baum}
\inputcode[firstline=3]{baum/einfach/Knoten}

\ueberschrift{Binärbaum mit Kompositum}

\inputcode[firstline=3]{baum/kompositum/Ahnenbaum}
\inputcode[firstline=3]{baum/kompositum/Baumelement}
\inputcode[firstline=3]{baum/kompositum/Abschluss}
\inputcode[firstline=3]{baum/kompositum/Datenknoten}
\inputcode[firstline=3]{baum/kompositum/Person}
\end{antwort}

\end{enumerate}

%-----------------------------------------------------------------------
%
%-----------------------------------------------------------------------

\section{Aufgabe 2: Suchbaum\footcite[Seite 1]{aud:pu:5}}

\footcite[Abgewandelt von Herbst 2003, Thema 2, Aufgabe 8, Seite 8]{examen:66112:2003:09}

\begin{enumerate}

%%
% (a)
%%

\item Implementieren Sie in einer objektorientierten Sprache einen
binären Suchbaum für ganze Zahlen! Dazu gehören Methoden zum Setzen und
Ausgeben der Attribute \java{zahl}, \java{linker_teilbaum} und
\java{rechter_teilbaum}. Design: eine Klasse \java{Knoten} und eine
Klasse \java{BinBaum}. Ein Knoten hat einen linken und einen rechten
Nachfolger. Ein Baum verwaltet die Wurzel. Er hängt neue Knoten an und
löscht Knoten.

%%
% (b)
%%

\item Schreiben Sie eine Methode \java{fuegeEin(...)}, die eine Zahl in
den Baum einfügt!

%%
% (c)
%%

\item Schreiben Sie eine Methode \java{postOrder(...)}, die die Zahlen
in der Reihenfolge postorder ausgibt!

%%
% (d)
%%

\item Ergänzen Sie Ihr Programm um die rekursiv implementierte Methode
\java{summe(...)}, die die Summe der Zahlen des Unterbaums, dessen
Wurzel der Knoten ist, zurückgibt! Falls der Unterbaum leer ist, ist der
Rückgabewert 0!

\begin{minted}{java}
int summe (Knoten x)...
\end{minted}

%%
% (e)
%%

\item Schreiben Sie eine Folge von Anweisungen, die einen Baum mit Namen
BinBaum erzeugt und nacheinander die Zahlen 5 und 7 einfügt! In den
binären Suchbaum werden noch die Zahlen 4, 11, 6 und 2 eingefügt.
Zeichnen Sie den Baum, den Sie danach erhalten haben, und schreiben Sie
die eingefügten Zahlen in der Reihenfolge der Traversierungsmöglichkeit
\texttt{postorder} auf!

%%
% (f)
%%

\item Implementieren Sie eine Operation \java{isSorted(...)}, die für
einen (Teil-)baum feststellt, ob er sortiert ist.
\end{enumerate}

%-----------------------------------------------------------------------
%
%-----------------------------------------------------------------------

\section{Frühjahr 2014 (46115) - Thema 2, Aufgabe 3\footcite{examen:46115:2014:09}}

\begin{enumerate}

%%
% (a)
%%

\item Fügen Sie die Zahlen 17, 7, 21, 3, 10, 13, 1, 5 nacheinander in
der vorgegebenen Reihenfolge in einen binären Suchbaum ein und zeichnen
Sie das Ergebnis!

\begin{center}
\begin{tikzpicture}[binaerer baum]
\Tree
[.17
  [.7
    [.3
      [.1 ]
      [.5 ]
    ]
    [.10
      \edge[blank]; \node[blank]{};
      [.13 ]
    ]
  ]
  [.21 ]
]
\end{tikzpicture}
\end{center}

%%
% (b)
%%

\item Implementieren Sie in einer objektorientierten Programmiersprache
eine rekursiv festgelegte Datenstruktur, deren Gestaltung sich an
folgender Definition eines binären Baumes orientiert!

Ein binärer Baum ist entweder ein leerer Baum oder besteht aus einem
Wurzelelement, das einen binären Baum als linken und einen als rechten
Teilbaum besitzt. Bei dieser Teilaufgabe können Sie auf die
Implementierung von Methoden (außer ggf. notwendigen Konstruktoren)
verzichten!

\paragraph{Klasse Knoten}

\begin{minted}{java}
class Knoten {
  public int wert;
  public Knoten links;
  public Knoten rechts;
  public Knoten elternKnoten;

  Knoten(int wert) {
    this.wert = wert;
    links = null;
    rechts = null;
    elternKnoten = null;
  }

  public Knoten findeMiniumRechterTeilbaum() {
  }

  public void anhängen (Knoten knoten) {
  }
}
\end{minted}

\begin{minted}{java}

public class BinärerSuchbaum {
  public Knoten wurzel;

  BinärerSuchbaum(Knoten wurzel) {
    this.wurzel = wurzel;
  }

  BinärerSuchbaum() {
    this.wurzel = null;
  }

  public void einfügen(Knoten knoten) {
  }

  public void einfügen(Knoten knoten, Knoten elternKnoten) {
  }

  public Knoten suchen(int wert) {
  }

  public Knoten suchen(int wert, Knoten knoten) {
  }

}
\end{minted}

%%
% (c)
%%

\item Beschreiben Sie durch Implementierung in einer gängigen
objektorientierten Programmiersprache, wie bei Verwendung der obigen
Datenstruktur die Methode \java{loescheKnoten(w)} gestaltet sein muss,
mit der der Knoten mit dem Eintrag \java{w} aus dem Baum entfernt werden
kann, ohne die Suchbaumeigenschaft zu verletzen!

\begin{antwort}
\inputcode[firstline=64,lastline=93]{aufgaben/aud/examen_46115_2014_03/suchbaum/BinaererSuchbaum}
\end{antwort}
\end{enumerate}

\begin{antwort}
\ueberschrift{Klasse „BinärerSuchbaum“}

\inputcode{aufgaben/aud/examen_46115_2014_03/suchbaum/BinaererSuchbaum}

\ueberschrift{Klasse „Knoten“}

\inputcode{aufgaben/aud/examen_46115_2014_03/suchbaum/Knoten}
\end{antwort}

\literatur

\end{document}
