\documentclass{lehramt-informatik}
\usepackage{multicol}
\InformatikPakete{baum}

\begin{document}

%%%%%%%%%%%%%%%%%%%%%%%%%%%%%%%%%%%%%%%%%%%%%%%%%%%%%%%%%%%%%%%%%%%%%%%%
% Theorie-Teil
%%%%%%%%%%%%%%%%%%%%%%%%%%%%%%%%%%%%%%%%%%%%%%%%%%%%%%%%%%%%%%%%%%%%%%%%

\chapter{B-Bäume}

\begin{quellen}
\item \cite{wiki:bbaum}
\item \cite[Kapitel 14.4.3, Seite 386-399 (PDF 402-415)]{saake}
\item \cite[Kapitel 13.5.4.2 Balancierte Mehrwegbäume, Seite 464, wird
nur erwähnt, nicht beschrieben]{schneider}
\end{quellen}

\url{https://www.cs.usfca.edu/~galles/visualization/BTree.html}

%-----------------------------------------------------------------------
%
%-----------------------------------------------------------------------

\section{Definition}

Ein Baum heißt genau dann B-Baum, wenn gilt:
%

\begin{itemize}
\item Jeder Knoten außer der Wurzel enthält zwischen \memph{$k$ und $2k$
Elemente} (Schlüsselwerte), $k$ wird als \memph{Ordnung} des B-Baums
bezeichnet.
%
\item Jeder \memph{Knoten ist entweder ein Blatt (ohne Kinder)} oder hat
\memph{mindestens $k + 1$ und höchstens $2k + 1$ Kind-Knoten}.
%
\item Der Wurzelknoten ist \memph{entweder ein Blatt oder hat mindestens
2 Nachfolger}.
%
\item Alle Blätter haben die \memph{gleiche Tiefe}, d.\,h. alle Wege von
der Wurzel bis zu den Blättern sind gleich lang. Pfade haben die Länge
$h-1$, wobei $h$ die Höhe des gesamten Baums ist.
\end{itemize}

\footcite[Seite 32 (PDF 26)]{aud:fs:5}

%-----------------------------------------------------------------------
%
%-----------------------------------------------------------------------

\section{Einfügen}

Das Einfügen in einen B-Baum erfolgt \memph{nur in den Blattknoten}.
Wenn in einem Blattknoten die \memph{maximale Anzahl} von Elementen
($2k$) erreicht ist, findet ein \memph{Split} statt, d.\,h. die
Elemente werden aufgeteilt und ein neuer Knoten entsteht. Das
\memph{mittlere Element} des ursprünglichen Knotens wird dabei \memph{in
den Elternknoten integriert}.
\footcite[Seite 32 (PDF 26)]{aud:fs:5}

%-----------------------------------------------------------------------
%
%-----------------------------------------------------------------------

\section{Suchen}

\begin{enumerate}
\item Beginnend mit Wurzelknoten wird Knoten jeweils \memph{von links
nach rechts} durchsucht:

\item \memph{Stimmt} Element mit gesuchtem Schlüsselwert
\memph{überein}, ist der Satz \memph{gefunden}.

\item Ist das \memph{Element größer} als der gesuchte Wert, wird die
Suche im \memph{links} hängenden Unterbaum \memph{fortgesetzt}.

\item Ist das \memph{Element kleiner} als der gesuchte Wert, wird der
Vergleich mit dem \memph{nächsten Element der Wurzel wiederholt}.

\item Ist auch das letztes Element der Wurzel noch kleiner als der
gesuchte Wert, dann wird die Suche im rechten Unterbaum des Elements
fortgesetzt.

\item Falls ein weiterer Abstieg in den Unterbaum nicht möglich ist
(d.\,h. Blattknoten), wird die Suche abgebrechen. Dann ist kein Satz mit
dem gewünschten Schlüsselwert vorhanden.
\end{enumerate}

\footcite[Seite 37 (PDF 31)]{aud:fs:5}

%-----------------------------------------------------------------------
%
%-----------------------------------------------------------------------

\section{Löschen}

Suche den Knoten, in dem das zu löschende Element E liegt.

\begin{itemize}
\item Falls das Element $E$ im Blattknoten liegt, dann lösche $E$ dort
und behandle ggf. entstehenden Unterlauf durch Mischen.

\item Falls das Element $E$ in einem inneren Knoten liegt, dann
untersuche den linken und rechten Unterbaum von $E$:

\begin{itemize}
\item Betrachte den Blattknoten mit dem direkten Vorgänger $E'$ von $E$
und den Blattknoten mit direktem Nachfolger $E''$ von $E$.

\item Wähle den Blattknoten aus, der mehr Elemente hat. Falls beide
Blattknoten gleich viele Elemente haben, wähle zufällig einen der beiden
aus.

\item Ersetze das zu löschende Element $E$ durch $E'$ bzw. $E''$ aus dem
gewählten Blattknoten.

\item Lösche $E'$ bzw. $E''$ im gewählten Blattknoten und behandle ggf.
entstehenden Unterlauf in diesem Blattknoten.
\end{itemize}
\end{itemize}

\footcite[Seite 39 (PDF 32)]{aud:fs:5}

\subsection{Hinweise zum Unterlauf:}

Ein Unterlauf entsteht auf Blattebene. Der Unterlauf wird durch Mischen
des Unterlaufknotens mit seinem Nachbarknoten und dem darüberliegenden
Element beseitig.
\footcite[Seite 40 (PDF 33)]{aud:fs:5}

%%%%%%%%%%%%%%%%%%%%%%%%%%%%%%%%%%%%%%%%%%%%%%%%%%%%%%%%%%%%%%%%%%%%%%%%
% Aufgaben
%%%%%%%%%%%%%%%%%%%%%%%%%%%%%%%%%%%%%%%%%%%%%%%%%%%%%%%%%%%%%%%%%%%%%%%%

\chapter{Aufgaben}

\section{Übungsaufgabe B-Bäume
(Adaptiert von Herbst 2005, Thema 2, A6)
\footcite[Seite 34-37(PDF 28-31)]{aud:fs:5}}

\begin{enumerate}

%%
% a)
%%

\item Erzeugen Sie aus der gegebenen Folge einen B-Baum der Ordnung
$m=2$:

$22,10,19,20,1,13,11,12,7,8,5,42,33,21,52,48,50$

Fügen Sie dazu die einzelnen Elemente in gegebener Reihenfolge in einen
anfangs leeren B-Baum ein. Stellen Sie für jeden Wert die entsprechenden
Zwischenergebnisse und die angewendeten Operationen als Bäume dar!

\begin{itemize}

%%
%
%%

\item \fbox{+22} \fbox{+10} \fbox{+19} \fbox{+20} Einfügen der ersten
Zahlen bis zur kompletten Füllung der Wurzel:

\begin{tikzpicture}[bbaum]
\node {10 \nodepart{two} 19 \nodepart{three} 20 \nodepart{four} 22};
\end{tikzpicture}

%%
%
%%

\item \fbox{+1} Einfügen der 1 führt zum Überlauf, deshalb Aufspaltung:

\begin{tikzpicture}[bbaum]
\node {1 \nodepart{two} 10 \nodepart{three} 19 \nodepart{four} 20 \nodepart{five} 22};
\end{tikzpicture}

%%
%
%%

\item Übernahme des mittleren Elements (19) in die Wurzel:

\begin{tikzpicture}[
  bbaum,
  level 1/.append style={sibling distance=15mm},
]
\node {19}[->]
  child{node {1 \nodepart{two} 10}}
  child{node {20 \nodepart{two} 22}}
;
\end{tikzpicture}

%%
%
%%

\item \fbox{+13} Einfügen der 13:

\begin{tikzpicture}[
  bbaum,
  level 1/.append style={sibling distance=20mm},
]
\node {19}[->]
  child{node {1 \nodepart{two} 10 \nodepart{three} 11 \nodepart{four} 13}}
  child{node {20 \nodepart{two} 22}}
;
\end{tikzpicture}

%%
%
%%

\item \fbox{+12} Einfügen der 12 nicht möglich, also wieder Aufspaltung.
11 als mittleres Element wird nach oben geschrieben:

\begin{tikzpicture}[
  bbaum,
  level 1/.append style={sibling distance=15mm},
]
\node {11 \nodepart{two} 19}[->]
  child{node {1 \nodepart{two} 10}}
  child{node {12 \nodepart{two} 13}}
  child{node {20 \nodepart{two} 22}}
;
\end{tikzpicture}

%%
%
%%

\item \fbox{+7} \fbox{+8} Einfügen von 7 und 8:

\begin{tikzpicture}[
  bbaum,
  level 1/.append style={sibling distance=20mm},
]
\node {11 \nodepart{two} 19}[->]
  child{node {1 \nodepart{two} 7 \nodepart{three} 8 \nodepart{four} 10}}
  child{node {12 \nodepart{two} 13}}
  child{node {20 \nodepart{two} 22}}
;
\end{tikzpicture}

%%
%
%%

\item \fbox{+5} Einfügen von 5 nicht möglich, deshalb Aufspaltung, 7 als
mittleres Element wird nach oben geschrieben:

\begin{tikzpicture}[
  bbaum,
  level 1/.append style={sibling distance=15mm},
]
\node {7 \nodepart{two} 11 \nodepart{three} 19}[->]
  child{node {1 \nodepart{two} 5}}
  child{node {8 \nodepart{two} 10}}
  child{node {12 \nodepart{two} 13}}
  child{node {20 \nodepart{two} 22}}
;
\end{tikzpicture}

%%
%
%%

\item \fbox{+42} \fbox{+33} Einfügen von 42 und 33:

\begin{tikzpicture}[
  bbaum,
  level 1/.append style={sibling distance=20mm},
]
\node {7 \nodepart{two} 11 \nodepart{three} 19}[->]
  child{node {1 \nodepart{two} 5}}
  child{node {8 \nodepart{two} 10}}
  child{node {12 \nodepart{two} 13}}
  child{node {20 \nodepart{two} 22 \nodepart{three} 33 \nodepart{four} 42}}
;
\end{tikzpicture}

%%
%
%%

\item \fbox{+21} Einfügen von 21 nicht möglich, also Aufspaltung, 22
nach oben schieben

\begin{tikzpicture}[
  bbaum,
  level 1/.append style={sibling distance=15mm},
]
\node {7 \nodepart{two} 11 \nodepart{three} 19 \nodepart{four} 22}[->]
  child{node {1 \nodepart{two} 5}}
  child{node {8 \nodepart{two} 10}}
  child{node {12 \nodepart{two} 13}}
  child{node {20 \nodepart{two} 21}}
  child{node {33 \nodepart{two} 42}}
;
\end{tikzpicture}

\item \fbox{+52} \fbox{+48} Einfügen von 52 und 48

\begin{tikzpicture}[
  bbaum,
  level 1/.append style={sibling distance=20mm},
]
\node {7 \nodepart{two} 11 \nodepart{three} 19 \nodepart{four} 22}[->]
  child{node {1 \nodepart{two} 5}}
  child{node {8 \nodepart{two} 10}}
  child{node {12 \nodepart{two} 13}}
  child{node {20 \nodepart{two} 21}}
  child{node {33 \nodepart{two} 42 \nodepart{three} 48 \nodepart{four} 52}}
;
\end{tikzpicture}

\item \fbox{+50} Einfügen von 50 nicht möglich, daher splitten und 48 eine Ebene
nach oben schieben

\begin{tikzpicture}[
  bbaum,
  level 1/.append style={sibling distance=15mm},
]
\node {7 \nodepart{two} 11 \nodepart{three} 19 \nodepart{four} 22 \nodepart{five} 48}[->]
  child{node {1 \nodepart{two} 5}}
  child{node {8 \nodepart{two} 10}}
  child{node {12 \nodepart{two} 13}}
  child{node {20 \nodepart{two} 21}}
  child{node {33 \nodepart{two} 42}}
  child{node {48 \nodepart{two} 52}}
;
\end{tikzpicture}

\item Einfügen von 48 oben nicht möglich, da Knoten ebenfalls voll! ->
weiterer Splitt notwendig, der neue Ebene erzeugt!

\begin{tikzpicture}[
  bbaum,
  level 1/.append style={level distance=5mm,sibling distance=60mm},
  level 2/.append style={sibling distance=20mm}
]
\node{19}[->]
  child{node{7 \nodepart{two} 11}
      child{node {1 \nodepart{two} 5}}
      child{node {8 \nodepart{two} 10}}
      child{node {12 \nodepart{two} 13}}
  }
  child{node{22 \nodepart{two} 48}
    child{node {20 \nodepart{two} 21}}
    child{node {33 \nodepart{two} 42}}
    child{node {48 \nodepart{two} 52}}
  }
;
\end{tikzpicture}
\end{itemize}

%%
% b)
%%

\item In dem Ergebnisbaum suchen wir nun den Wert 17. Stellen Sie den
Ablauf des Suchalgorithmus an einer Zeichnung graphisch dar!

\begin{tikzpicture}[
  bbaum,
  level 1/.append style={level distance=5mm,sibling distance=60mm},
  level 2/.append style={sibling distance=20mm}
]
\node[name=eins]{19}[->]
  child{node[name=zwei]{7 \nodepart{two} 11}
      child{node {1 \nodepart{two} 5}}
      child{node {8 \nodepart{two} 10}}
      child{node[name=drei]{12 \nodepart{two} 13}}
  }
  child{node{22 \nodepart{two} 48}
    child{node {20 \nodepart{two} 21}}
    child{node {33 \nodepart{two} 42}}
    child{node {48 \nodepart{two} 52}}
  }
;
\end{tikzpicture}
\end{enumerate}

\section{Einfügen und Löschen-Operation
\footcite[Seite 3 und 4, Aufgabe 6: B-Baum (H 2015(66116) - Thema 2, TA1, A3, a)]{aud:ab:7}}

Gegeben ist der folgende B-Baum der Ordnung 3 (max. drei Kindknoten,
max. zwei Schlüssel pro Knoten):

\begin{tikzpicture}[
  bbaum,
  level 1/.style={level distance=10mm,sibling distance=50mm},
  level 2/.style={level distance=10mm,sibling distance=20mm}
]
\node {50} [->]
  child {node {10 \nodepart{two} 30}
    child {node {5 \nodepart{two} 8}}
    child {node {15 \nodepart{two} 20}}
    child {node {31 \nodepart{two} 33}}
  }
  child {node {70}
    child {node {60}}
    child {node {80}}
  };
\end{tikzpicture}

\noindent
Fügen Sie die Werte 9 und 45 ein. Löschen Sie anschließend die Werte 30
und 70. Zeichnen Sie den Baum nach jeder Einfüge- bzw. Lösch-Operation.

\begin{antwort}
\ueberschrift{9 einfügen}

\begin{tikzpicture}[
  scale=0.8,
  transform shape,
  bbaum,
  level 1/.style={level distance=10mm,sibling distance=35mm},
  level 2/.style={level distance=10mm,sibling distance=20mm},
]
\node {10 \nodepart{two} 50} [->]
  child {node {8}
    child {node {5}}
    child {node[thick,font=\bfseries] {9}}
  }
  child {node {30}
    child {node {15 \nodepart{two} 20}}
    child {node {31 \nodepart{two} 33}}
  }
  child {node {70}
    child {node {60}}
    child {node {80}}
  }
;
\end{tikzpicture}

\ueberschrift{45 einfügen}

\begin{tikzpicture}[
  scale=0.8,
  transform shape,
  bbaum,
  level 1/.style={level distance=10mm,sibling distance=45mm},
  level 2/.style={level distance=10mm,sibling distance=20mm},
]
\node {10 \nodepart{two} 50} [->]
  child {node {8}
    child {node {5}}
    child {node {9}}
  }
  child {node {30 \nodepart{two} 33}
    child {node {15 \nodepart{two} 20}}
    child {node {31}}
    child {node[thick,font=\bfseries]{45}}
  }
  child {node {70}
    child {node {60}}
    child {node {80}}
  }
;
\end{tikzpicture}

\ueberschrift{30 löschen}

\begin{tikzpicture}[
  scale=0.8,
  transform shape,
  bbaum,
  level 1/.style={level distance=10mm,sibling distance=42mm},
  level 2/.style={level distance=10mm,sibling distance=20mm},
]
\node {10 \nodepart{two} 50} [->]
  child {node {8}
    child {node {5}}
    child {node {9}}
  }
  child {node {20 \nodepart{two} 33}
    child {node {15}}
    child {node {31}}
    child {node {45}}
  }
  child {node {70}
    child {node {60}}
    child {node {80}}
  }
;
\end{tikzpicture}

\ueberschrift{70 löschen}

\begin{tikzpicture}[
  scale=0.8,
  transform shape,
  bbaum,
  level 1/.style={level distance=10mm,sibling distance=32mm},
  level 2/.style={level distance=10mm,sibling distance=20mm},
]
\node {10 \nodepart{two} 33} [->]
  child {node {8}
    child {node {5}}
    child {node {9}}
  }
  child {node {20}
    child {node {15}}
    child {node {31}}
  }
  child {node {50}
    child {node {45}}
    child {node {60 \nodepart{two} 80}}
  }
;
\end{tikzpicture}
\end{antwort}

%-----------------------------------------------------------------------
%
%-----------------------------------------------------------------------

\section{Aufgabe 9: Bäume\footcite[entnommen aus Algorithmen und
Datenstrukturen, Übungsblatt 6, Universität Würzburg]{aud:pu:7}}

\begin{enumerate}

%%
% (a)
%%

\item Fügen Sie in einen anfangs leeren 2-3-4-Baum (B-Baum der Ordnung
4)\footnote{ein Baum, für den folgendes gilt: Er besitzt in einem Knoten
max. 3 Schlüssel-Einträge und 4 Kindknoten und minimal einen Schlüssel
und 2 Nachfolger} der Reihe nach die folgenden Schlüssel ein:

\centerline{$1$, $2$, $3$, $5$, $7$, $8$, $9$, $4$, $11$, $12$, $13$, $6$.}

Dokumentieren Sie die Zwischenschritte so,
dass die Entstehung des Baumes und nicht nur das Endergebnis
nachvollziehbar ist. \footcite[Staatsexamen Theoretische Informatik,
Algorithmen und Datenstrukturen, Realschulen, Frühjahr 2011, Thema 1,
Aufgabe 3]{examen:46115:2011:03}

\begin{antwort}
\begin{multicols}{2}
\begin{enumerate}

%%
%
%%

\item 1, 2, 3 einfügen:

\begin{tikzpicture}[
  scale=0.8,
  transform shape,
  bbaum,
  level 1/.style={level distance=10mm,sibling distance=32mm},
  level 2/.style={level distance=10mm,sibling distance=20mm},
]
\node {1 \nodepart{two} 2 \nodepart{three} 3};
\end{tikzpicture}

%%
%
%%

\item 5 einfügen:

\begin{tikzpicture}[
  scale=0.8,
  transform shape,
  bbaum,
  level 1/.style={level distance=8mm,sibling distance=25mm},
]
\node {2} [->]
  child {node {1}}
  child {node {3 \nodepart{two} 5}}
;
\end{tikzpicture}

%%
%
%%

\item 7 einfügen:

\begin{tikzpicture}[
  scale=0.8,
  transform shape,
  bbaum,
  level 1/.style={level distance=8mm,sibling distance=25mm},
]
\node {2} [->]
  child {node {1}}
  child {node {3 \nodepart{two} 5 \nodepart{three} 7}}
;
\end{tikzpicture}

%%
%
%%

\item 8 einfügen:

\begin{tikzpicture}[
  scale=0.8,
  transform shape,
  bbaum,
  level 1/.style={level distance=10mm,sibling distance=15mm},
]
\node {2 \nodepart{two} 5} [->]
  child {node {1}}
  child {node {3}}
  child {node {7 \nodepart{two} 8}}
;
\end{tikzpicture}

%%
%
%%

\item 9 und 4 einfügen:

\begin{tikzpicture}[
  scale=0.8,
  transform shape,
  bbaum,
  level 1/.style={level distance=10mm,sibling distance=15mm},
]
\node {2 \nodepart{two} 5} [->]
  child {node {1}}
  child {node {3 \nodepart{two} 4}}
  child {node {7 \nodepart{two} 8 \nodepart{three} 9}}
;
\end{tikzpicture}

%%
%
%%

\item 11 einfügen:

\begin{tikzpicture}[
  scale=0.8,
  transform shape,
  bbaum,
  level 1/.style={level distance=10mm,sibling distance=15mm},
]
\node {2 \nodepart{two} 5 \nodepart{three} 8} [->]
  child {node {1}}
  child {node {3 \nodepart{two} 4}}
  child {node {7}}
  child {node {9 \nodepart{two} 11}}
;
\end{tikzpicture}

%%
%
%%

\item 12 einfügen:

\begin{tikzpicture}[
  scale=0.8,
  transform shape,
  bbaum,
  level 1/.style={level distance=10mm,sibling distance=15mm},
]
\node {2 \nodepart{two} 5 \nodepart{three} 8} [->]
  child {node {1}}
  child {node {3 \nodepart{two} 4}}
  child {node {7}}
  child {node {9 \nodepart{two} 11 \nodepart{three} 12}}
;
\end{tikzpicture}

%%
%
%%

\item 13 einfügen (zwei Splits):

\begin{tikzpicture}[
  scale=0.8,
  transform shape,
  bbaum,
  level 1/.style={level distance=7mm,sibling distance=31mm},
  level 2/.style={level distance=10mm,sibling distance=12mm},
]
\node {5} [->]
  child {node {2}
    child {node {1}}
    child {node {3 \nodepart{two} 4}}
  }
  child {node {8 \nodepart{two} 11}
    child {node {7}}
    child {node {9}}
    child {node {12 \nodepart{two} 13}}
  }
;
\end{tikzpicture}

%%
%
%%

\item 6 einfügen:

\begin{tikzpicture}[
  scale=0.8,
  transform shape,
  bbaum,
  level 1/.style={level distance=7mm,sibling distance=31mm},
  level 2/.style={level distance=10mm,sibling distance=12mm},
]
\node {5} [->]
  child {node {2}
    child {node {1}}
    child {node {3 \nodepart{two} 4}}
  }
  child {node {8 \nodepart{two} 11}
    child {node {6 \nodepart{two} 7}}
    child {node {9}}
    child {node {12 \nodepart{two} 13}}
  }
;
\end{tikzpicture}
\end{enumerate}
\end{multicols}
\end{antwort}

%%
% (b)
%%

\item Zeichnen Sie einen Rot-Schwarz-Baum oder einen AVL-Baum, der
dieselben Einträge enthält.

%%
% (c)
%%

\item Geben Sie eine möglichst gute untere Schranke (in
$\Omega$-Notation) für die Anzahl der Schlüssel in einem 2-3-4-Baum der
Höhe h an.

Hinweis: Überlegen Sie sich, wie ein 2-3-4-Baum mit Höhe $h$ und
möglichst wenigen Schlüsseln aussieht.

\begin{antwort}
Ein 2-3-4-Baum mit möglichst wenigen Schlüsseln sieht aus wie ein
Binärbaum:

\begin{itemize}
\item Ein Baum der Höhe $1$ hat $1$ Schlüssel.
\item Ein Baum der Höhe $2$ hat $3$ Schlüssel.
\item Ein Baum der Höhe $3$ hat $7$ Schlüssel.
\item $\cdots$
\item Ein Baum der Höhe $h$ hat $2^h - 1$ Schlüssel.
\end{itemize}

Also liegt die Untergrenze für die Anzahl der Schlüssel in
$\Omega(2^h)$.
\end{antwort}

%%
% (d)
%%

\item Geben Sie eine möglichst gute obere Schranke (in
$\mathcal{O}$-Notation) für die Anzahl der Schlüssel in einem 2-3-4-Baum
der Höhe h an.

\begin{antwort}
Ein 2-3-4-Baum mit möglichst vielen Schlüsseln hat in jedem Knoten drei
Schlüssel. Und jeder Knoten, der kein Blatt ist, hat vier Kinder:

\begin{itemize}
\item Ein Baum der Höhe $1$ hat $3$ Schlüssel.
\item Ein Baum der Höhe $2$ hat $15$ Schlüssel.
\item Ein Baum der Höhe $3$ hat $63$ Schlüssel.
\item $\cdots$
\item Ein Baum der Höhe $h$ hat $4^h - 1$ Schlüssel.
\end{itemize}

Also liegt die Obergrenze für die Anzahl der Schlüssel in
$\mathcal{O}(4^h)$.
\end{antwort}

\end{enumerate}

\literatur

\end{document}
