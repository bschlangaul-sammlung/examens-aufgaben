\documentclass{lehramt-informatik-haupt}
\usepackage{tabularx}

\begin{document}

%%%%%%%%%%%%%%%%%%%%%%%%%%%%%%%%%%%%%%%%%%%%%%%%%%%%%%%%%%%%%%%%%%%%%%%%
% Theorie-Teil
%%%%%%%%%%%%%%%%%%%%%%%%%%%%%%%%%%%%%%%%%%%%%%%%%%%%%%%%%%%%%%%%%%%%%%%%

\chapter{Entwurfsmuster / Design pattern}

\begin{quellen}
\item \cite{wiki:entwurfsmuster}
\end{quellen}

\section{Verwendete Überschriften}

\begin{enumerate}
\item Zweck
\item Szenario
\item UML-Diagramm
\item Akteure
\item Allgemeines Code-Beispiel
\end{enumerate}

Wiederkehrende, geprüfte, bewährte Lösungsschablonen für typische Probleme
→ Wiederverwendung
Kein Garant für gutes Design – können dieses aber gut unterstützen

\begin{itemize}
\item Erzeugungsmuster (Creational Patterns)

\begin{itemize}
\item Einzelstück (Singleton)
\item Abstrakte Fabrik (Abstract Factory)
\end{itemize}

\item Strukturmuster (Structural Patterns)

\begin{itemize}
\item Adapter
\item Dekorierer (Decorator)
\item Kompositum (Composite)
\item Stellvertreter (Proxy)
\end{itemize}

\item Verhaltensmuster (Behavioral Patterns)

\begin{itemize}
\item Beobachter (Observer)
\item Wiederholer (Iterator)
\item Schablone (Template)
\item Zustand (State)
\end{itemize}

\item Sonstige

\begin{itemize}
\item Modell-Präsentation-Steuerung (Model-View-Controller)
\end{itemize}
\end{itemize}

\footcite[Seite 39]{sosy:fs:3}

%%%%%%%%%%%%%%%%%%%%%%%%%%%%%%%%%%%%%%%%%%%%%%%%%%%%%%%%%%%%%%%%%%%%%%%%
% Aufgaben
%%%%%%%%%%%%%%%%%%%%%%%%%%%%%%%%%%%%%%%%%%%%%%%%%%%%%%%%%%%%%%%%%%%%%%%%

\chapter{Aufgaben}

\ExamensAufgabeTTA 46116 / 2017 / 09 : Thema 2 Teilaufgabe 1 Aufgabe 3

%-----------------------------------------------------------------------
%
%-----------------------------------------------------------------------

\section{Aufgabe 1\footcite{sosy:ab:6}}

\begin{enumerate}

%%
% (a)
%%

\item Erklären Sie, was man bei der Entwicklung von Softwaresystemen
unter einem Entwurfsmuster versteht und gehen Sie dabei auch auf die
Vorteile ein.

\begin{antwort}
Entwurfsmuster sind bewährte Lösungsschablonen für wiederkehrende
Entwurfsprobleme sowohl in der Architektur als auch in der
Softwarearchitektur und -entwicklung. Sie stellen damit eine
wiederverwendbare Vorlage zur Problemlösung dar, die in einem bestimmten
Zusammenhang einsetzbar ist.

Der primäre Nutzen eines Entwurfsmusters liegt in der Beschreibung einer
Lösung für eine bestimmte Klasse von Entwurfsproblemen. Weiterer Nutzen
ergibt sich aus der Tatsache, dass jedes Muster einen Namen hat. Dies
vereinfacht die Diskussion unter Entwicklern, da man abstrakt über eine
Struktur sprechen kann. So sind etwa Software-Entwurfsmuster zunächst
einmal unabhängig von der konkreten Programmiersprache. Wenn der Einsatz
von Entwurfsmustern dokumentiert wird, ergibt sich ein weiterer Nutzen
dadurch, dass durch die Beschreibung des Musters ein Bezug zur dort
vorhandenen Diskussion des Problemkontextes und der Vor- und Nachteile
der Lösung hergestellt wird.
\end{antwort}

%%
% (b)
%%

\item Nennen Sie die drei ursprünglichen Typen von Entwurfsmuster,
erklären Sie diese kurz und geben Sie zu jedem Typ drei Beispiele an.

\begin{antwort}

\begin{tabularx}{\linewidth}{p{2cm}|X|p{2cm}}
\textbf{Typ} & \textbf{Erlärung} & \textbf{Beispiele} \\\hline\hline

Erzeugungsmuster &
Dienen der Erzeugung von Objekten; diese wird dadurch gekapselt und
ausgelagert, um den Kontext der Objekterzeugung unabhängig von der
konkreten Implementierung zu halten &
abstrakte Fabrik, Singleton, Prototyp \\\hline

Strukturmuster &
Erleichtern den Entwurf von Software durch vorgefertigte Schablonen für
Beziehungen zwischen Klassen. &
Adapter, Kompositum, Stellvertreter \\\hline

Verhaltensmuster  &
Modellieren komplexes Verhalten der Software und erhöhen damit die
Flexibilität der Software hinsichtlich ihres Verhaltens. &
Beobachter, Interpreter, Iterator \\
\end{tabularx}
\end{antwort}
\end{enumerate}

%-----------------------------------------------------------------------
%
%-----------------------------------------------------------------------

\ExamensAufgabeTTA 66116 / 2016 / 03 : Thema 1 Teilaufgabe 2 Aufgabe 2

%%
%
%%

\section{Aufgabe 3\footcite{sosy:ab:6}}

Stellen Sie sich vor, Sie brauchen ein Grafiksystem. In diesem System
wollen Sie aus Linien, Rechtecken, und Text größere Abbildungen
darstellen. Diese Abbildungen sollen auch wieder andere Abbildungen
rekursiv enthalten können. Sie brauchen also primitive Objekte: die
Linie, das Rechteck und den Text. Zusätzlich brauchen Sie Behälter, die
weitere Behälter und primitive Objekte enthalten können.

Erklären Sie, mit welchem Entwurfsmuster man diese Struktur abbilden
kann und zeichnen Sie das dazugehörige Klassendiagramm.

%-----------------------------------------------------------------------
%
%-----------------------------------------------------------------------

\section{Aufgabe 4: (StEx H14, T2, Teilaufgabe 2 A2 (geänderte Aufgabenstellung))\footcite{sosy:ab:6}}

Sie sollen das Design für ein einfaches Wahlsystem entwerfen. Das System
soll dabei die Verteilung der Stimmen auf die einzelnen Parteien
ermöglichen. Zusätzlich soll es verschiedene Darstellungen dieser Daten
erlauben: Eine Tabelle, in der die Daten gelesen und auch eingegeben
werden können, und ein Diagramm als alternative Darstellung der
Informationen. Das System soll mit dem Model-View-Controller Muster
modelliert werden.

\begin{enumerate}

%%
% 1.
%%

\item Beschreiben Sie das Model-View-Controller Muster:

\begin{enumerate}

%%
% (a)
%%

\item Beschreiben Sie das Problem, welches das Muster adressiert.

%%
% (b)
%%

\item Das MVC-Muster ist aus drei anderen Entwurfsmustern
zusammengesetzt.
\end{enumerate}

\item Geben Sie diese an und beschreiben Sie die Aufgaben der einzelnen
Komponenten, die im MVC-Muster verwendet werden.

%%
% 2.
%%

\item Modellieren Sie das System unter Anwendung des Musters. Entwerfen
Sie hierfür ein UML-Klassendiagramm.

\end{enumerate}

\literatur

\end{document}
