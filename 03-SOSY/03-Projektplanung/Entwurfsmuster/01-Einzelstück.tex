\documentclass{lehramt-informatik-haupt}
\InformatikPakete{uml,syntax}
\begin{document}

%%%%%%%%%%%%%%%%%%%%%%%%%%%%%%%%%%%%%%%%%%%%%%%%%%%%%%%%%%%%%%%%%%%%%%%%
% Theorie-Teil
%%%%%%%%%%%%%%%%%%%%%%%%%%%%%%%%%%%%%%%%%%%%%%%%%%%%%%%%%%%%%%%%%%%%%%%%

\chapter{Einzelstück / Singleton}

\begin{quellen}
\item \cite{wiki:singleton}
\item \url{https://www.philipphauer.de/study/se/design-pattern/singleton.php}
\item \cite[Seite 109-115]{gof}
\item \cite[Kapitel 8.3.1, Seite 247-249]{schatten}
\item \cite[Kapitel 3.4 Seite 38-43]{eilebrecht}
\item \cite[Kapitel 1, Seite 1-17]{siebler}
\end{quellen}

\section{Zweck}

Stellt sicher, dass nur \memph{genau eine Instanz einer Klasse} erzeugt
wird.
\footcite[Seite 38]{eilebrecht}

%-----------------------------------------------------------------------
%
%-----------------------------------------------------------------------

\section{Klassendiagramm}

\begin{tikzpicture}
\umlclass{Singleton}{
\umlstatic{- instance: Singleton}\\
}{
\umlstatic{- Singleton()}\\
+ getInstance()
}
\end{tikzpicture}

%-----------------------------------------------------------------------
%
%-----------------------------------------------------------------------

\section{Allgemeines Code-Beispiel}

\inputcode[firstline=3]{entwurfsmuster/einzelstueck/allgemein/Einzelstueck}

%%%%%%%%%%%%%%%%%%%%%%%%%%%%%%%%%%%%%%%%%%%%%%%%%%%%%%%%%%%%%%%%%%%%%%%%
% Aufgaben
%%%%%%%%%%%%%%%%%%%%%%%%%%%%%%%%%%%%%%%%%%%%%%%%%%%%%%%%%%%%%%%%%%%%%%%%

\chapter{Aufgaben}

\literatur

\end{document}
