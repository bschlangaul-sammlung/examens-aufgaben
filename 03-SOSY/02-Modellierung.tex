\documentclass{lehramt-informatik}
\InformatikPakete{uml}

\begin{document}

%%%%%%%%%%%%%%%%%%%%%%%%%%%%%%%%%%%%%%%%%%%%%%%%%%%%%%%%%%%%%%%%%%%%%%%%
% Theorie-Teil
%%%%%%%%%%%%%%%%%%%%%%%%%%%%%%%%%%%%%%%%%%%%%%%%%%%%%%%%%%%%%%%%%%%%%%%%

\chapter{Modellierung}

%%%%%%%%%%%%%%%%%%%%%%%%%%%%%%%%%%%%%%%%%%%%%%%%%%%%%%%%%%%%%%%%%%%%%%%%
% Aufgaben
%%%%%%%%%%%%%%%%%%%%%%%%%%%%%%%%%%%%%%%%%%%%%%%%%%%%%%%%%%%%%%%%%%%%%%%%

\chapter{Aufgaben}

\section{Aufgabe 1
\footcite{sosy:ab:3}}

\begin{enumerate}
\item Beschreiben Sie den Unterschied zwischen einem Zustands- und einem
Aktivitätsdiagramm.

\item Der DVD-Automat für die Filmauswahl aus Blatt 2, Aufgabe 1 soll
als Zustandsdiagramm modelliert werden. Beachten Sie dabei die
angegebenen Funktionalitäten des Automaten.

\begin{enumerate}
\item Geben Sie die Ein- und Ausgaben des Automaten für die Filmauswahl
und für die Aus- und Rückgabe von Filmen an.

\item Geben Sie alle Zustandsattribute an, die für die Modellierung der
Automaten notwendig sind und beschreiben Sie deren Verwendungszweck.

\item Identifizieren Sie anhand der Zustandsattribute die Zustände der
obigen Automaten und geben Sie eine Charakterisierung der Zustände durch
Angabe der möglichen Wertebereiche der Zustandsattribute an. Welcher der
Zustände ist der Anfangszustand?

\item Zeichnen Sie die Zustandsübergangsdiagramme. Verwenden Sie hierzu
die Syntax mit Ein- und Ausgabe, Vor- und Nachbedingungen.
\end{enumerate}

\item Betrachten wir nun einen gewöhnlichen Video- und DVD-Verleih.
Beschreiben Sie das Szenario des Ausleihs eines Videos mit Hilfe eines
Aktivitätsdiagramms. Bei dem Videoverleih gelte:

\begin{itemize}
\item Der Kunde identifiziert sich beim Ausleihen mit seiner Kundenkarte
oder seinem Passwort. Hat der Kunde noch keine Karte, so muss der
Mitarbeiter ihn registrieren und ihm eine Kundenkarte ausstellen.

\item Filme können wie beim Automaten per Internet bis zu zwei Stunden
im Voraus reserviert werden.

\item Der Kunde hat kein Gehaltskonto, sondern bezahlt seine Gebühren
bei der Rückgabe des Videos in bar oder per Karte.

\item Ansonsten gelten die gleichen Bedingungen wie beim Automaten.
\end{itemize}
\end{enumerate}

%-----------------------------------------------------------------------
%
%-----------------------------------------------------------------------

\section{Aufgabe 2: (StEx Realschule Frühjahr 2017, Thema 1, TA 1, A3
(erweitert)) \footcite{sosy:ab:3}}

Betrachten Sie die folgenden UML-Diagrame. Sind diese korrekt? Falls
nein, begründen Sie warum nicht. Geben Sie in diesem Fall außerdem eine
korrigierte Version an. Falls ja, erklären Sie die inhaltliche Bedeutung
des Diagramms.\footcite{examen:46116:2017:03}

\begin{enumerate}
\item \strut

\begin{tikzpicture}
\umlactor{Person}
\umlactor[x=3]{Student}
\umlextend{Person}{Student}
\end{tikzpicture}

\begin{antwort}
\emph{Falsch}, den verwendeten „extends“-Pfeil gibt es nur zwischen
Anwendungsfällen.

\begin{tikzpicture}
\umlactor{Person}
\umlactor[x=3]{Student}
\umlinherit{Student}{Person}
\end{tikzpicture}
\end{antwort}

\item \strut

\begin{tikzpicture}
\umlsimpleclass[x=3]{Fahrzeug}
\umlsimpleclass[x=1,y=-2]{Landfahrzeug}
\umlsimpleclass[x=5,y=-2]{Wasserfahrzeug}
\umlsimpleclass[x=3,y=-4]{Amphibienfahrzeug}

\umlVHVinherit{Landfahrzeug}{Fahrzeug}
\umlVHVinherit{Wasserfahrzeug}{Fahrzeug}
\umlVHVinherit{Amphibienfahrzeug}{Wasserfahrzeug}
\umlVHVinherit{Amphibienfahrzeug}{Landfahrzeug}
\end{tikzpicture}

\begin{antwort}
Die dargestellte Modellierung ist \emph{korrekt}. Sowohl Land- als auch
Wasserfahrzeuge sind Fahrzeuge und erben somit von dieser Klasse. Da
ein Amphibienfahrzeug eine „Mischung“ aus beidem ist, erbt diese Klasse
auch von beiden Klassen. Diese Mehrfachvererbug kann allerdings nicht
in jeder Programmiersprache (z. B. nicht in Java) umgesetzt
werden.
\end{antwort}

\item \strut

\begin{tikzpicture}
\umlsimpleclass{Aquarium}
\umlsimpleclass[x=3]{Fisch}
\umlaggreg{Fisch}{Aquarium}
\end{tikzpicture}

\begin{antwort}
Bei der dargestellten Aggregation befindet sich die Raute an der
\emph{falschen} Seite der Beziehung. Das Diagramm würde bedeuten, dass
ein Fisch mehrere Aquarien enthält. Die Umkehrung ist aber korrekt:

\begin{tikzpicture}
\umlsimpleclass{Aquarium}
\umlsimpleclass[x=3]{Fisch}
\umlaggreg{Aquarium}{Fisch}
\end{tikzpicture}
\end{antwort}

\item \strut

\begin{tikzpicture}
\umlsimpleclass{Auto}
\umlsimpleclass[x=-2,y=-2]{Motor}
\umlsimpleclass[x=2,y=-2]{Rad}

\umlVHVcompo{Auto}{Motor}
\umlVHVcompo{Auto}{Rad}
\end{tikzpicture}

\begin{antwort}
Hier wurde für die Modellierung eine Komposition gewählt. Dies bedeutet,
dass die Existenz der Teile vom Ganzen abhängt. Einen Raum kann es z. B.
ohne ein Gebäude nicht geben. In diesem Fall ist die Darstellung
\emph{falsch}, da Motor und Rad auch ohne Auto existieren können. Die
Modellierung muss also mittels Aggregation erfolgen:

\begin{tikzpicture}
\umlsimpleclass{Auto}
\umlsimpleclass[x=-2,y=-2]{Motor}
\umlsimpleclass[x=2,y=-2]{Rad}

\umlVHVaggreg{Auto}{Motor}
\umlVHVaggreg{Auto}{Rad}
\end{tikzpicture}
\end{antwort}

\end{enumerate}

%-----------------------------------------------------------------------
%
%-----------------------------------------------------------------------

\section{Aufgabe 3: UML Diagramme in der Anwendung\footcite{sosy:pu:2}}

Gegeben sei folgender Sachverhalt:
\footcite[46116 – Herbst 2014 – Thema 2, TA 1, Aufgabe 3]{examen:46116:2014:09}

Für eine Verwaltungssoftware einer Behörde soll ein Bestellsystem
entwickelt werden. Dabei sollen die Nutzer ihre Raummaße eingeben
können. Anschließend können die Nutzer über ein Web-Interface das Büro
gestalten und Möbel (wie zum Beispiel Wandschränke) und andere
Einrichtungsgegenstände in einem virtuellen Büro platzieren. Aus dem
Web-Interface kann die Einrichtung dann direkt bestellt werden. Dazu
müssen die Nutzer ihre Büro-Nummer und den Namen und die Adresse der
Behörde eingeben und die Bestellung bestätigen.

Weiterhin können Nutzer auch Büromaterialien über das Web-Interface
bestellen. Dazu ist anstatt der Eingabe der Raummaße nur das Eingeben
von Büro-Nummer und des Namens und der Adresse der Behörde erforderlich.

Zusätzlich zum Standard-Nutzer können sich auch Administratoren im
System anmelden und Möbel zur Kollektion hinzufügen und aus der
Kollektion entfernen. Die Möbel können eindeutig durch ihre
Inventurnummer identifiziert werden.

Um jegliche Veränderungen im System protokollieren zu können müssen
Nutzer und Administratoren zur Bestätigung eingeloggt sein.

\begin{enumerate}

%%
% a)
%%

\item Erfassen Sie die drei Systemfunktionen \emph{Möbel bestellen},
\memph{Login} und \memph{Kollektion verwalten} in einem UML-konformen
Use Case Diagramm.

\begin{antwort}
% Nach Video gezeichnet

%%
%
%%

\begin{tikzpicture}
\begin{umlsystem}{Möbel bestellen}
\umlusecase[x=1,y=0,name=login]{Login}
\umlusecase[x=5,y=-1,text width=1.5cm,name=bestellung]{Bestellung bestätigen}
\umlusecase[x=4,y=-3,text width=1.5cm,name=buero]{Büro gestalten}
\umlusecase[x=1,y=-5,text width=1.5cm,name=raummass]{Raummaß eingeben}
\end{umlsystem}

\umlactor[x=-2,y=-2]{Nutzer}
\umlassoc{Nutzer}{login}
\umlassoc{Nutzer}{bestellung}
\umlassoc{Nutzer}{buero}
\umlassoc{Nutzer}{raummass}

\umlinclude{buero}{raummass}
\umlinclude{bestellung}{buero}
\umlinclude{bestellung}{login}
\end{tikzpicture}

%%
%
%%

\begin{tikzpicture}
\begin{umlsystem}{Login}
\umlusecase[x=1,y=-1,text width=1.5cm,name=bueronummer]{Büronummer eingeben}
\umlusecase[x=4,y=-3,name=name]{Name eingeben}
\umlusecase[x=1,y=-4,text width=1.5cm,name=adresse]{Adresse Behörde eingeben}
\end{umlsystem}

\umlactor[x=-2,y=-2]{Nutzer}
\umlassoc{Nutzer}{bueronummer}
\umlassoc{Nutzer}{name}
\umlassoc{Nutzer}{adresse}
\end{tikzpicture}

%%
%
%%

\begin{tikzpicture}
\begin{umlsystem}{Kollektion verwalten}
\umlusecase[x=1,y=-1,text width=1.5cm,name=hinzufuegen]{Möbel hinzufügen}
\umlusecase[x=4,y=-3,name=login]{Login}
\umlusecase[x=1,y=-4,text width=1.5cm,name=entfernen]{Möbel entfernen}
\end{umlsystem}

\umlactor[x=-2,y=-2]{Administrator}
\umlassoc{Administrator}{login}
\umlassoc{Administrator}{hinzufuegen}
\umlassoc{Administrator}{entfernen}

\umlinclude{hinzufuegen}{login}
\umlinclude{entfernen}{login}
\end{tikzpicture}
\end{antwort}

%%
% b)
%%

\item Erstellen Sie ein UML-Klassendiagramm, welches die Beziehungen und
sinnvolle Attribute der Klassen „Nutzer, Büro, Möbelstück und
Wandschrank“ darstellt.

%%
% c)
%%

\item Instanziieren Sie das Klassendiagramm in einem Objektdiagramm mit
den zwei Nutzern mit Namen Ernie und Bernd, einem Büro mit der Nummer
CAPITOL2 und zwei Schränken mit den Inventurnummern S1.88 und S1.77.

%%
% d)
%%

\item Geben Sie für ein Büromöbelstück ein Zustandsdiagramm an.
Überlegen Sie dazu, welche möglichen Zustände ein Möbelstück während des
Bestellvorgangs haben kann und finden Sie geeignete Zustandsübergänge.
\end{enumerate}

\literatur

\end{document}
