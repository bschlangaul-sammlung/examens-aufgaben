\documentclass{lehramt-informatik}
\usepackage{tabularx}

\begin{document}

%%%%%%%%%%%%%%%%%%%%%%%%%%%%%%%%%%%%%%%%%%%%%%%%%%%%%%%%%%%%%%%%%%%%%%%%
% Theorie-Teil
%%%%%%%%%%%%%%%%%%%%%%%%%%%%%%%%%%%%%%%%%%%%%%%%%%%%%%%%%%%%%%%%%%%%%%%%

\chapter{Anforderungsermittlung}

%-----------------------------------------------------------------------
%
%-----------------------------------------------------------------------

\section{Anforderungen und Spezifikationen\footcite[Seite 13]{sosy:fs:1}}

\subsection{Anforderungen (Requirements):}

\begin{itemize}
\item welches Problem soll gelöst werden?
\item welche Leistung soll das geplante Projekt erbringen?
\item Berücksichtigung möglichst aller beteiligten Personen (Stakeholder)
\end{itemize}

\subsection{Anforderungsanalyse (Anforderungsspezifikation):}

Kompromiss aller beteiligten Stakeholder im Hinblick auf das zu
erstellende Produkt\footcite[Seite 17-20]{schatten}

\subsection{Arten von Anforderungen:}\footcite[Seite 14]{sosy:fs:1}

\begin{description}
\item[Funktionale Anforderungen]
Systemverhalten; Funktionen des zu erstellenden Produkts

\item[Nichtfunktionale Anforderungen]
Qualitätsmerkmale wie z. B. Leistungsfähigkeit

\item[Designbedingungen]
Festlegung technischer Rahmenbedingungen

\item[Prozessbedingungen]
Rahmenbedingungen für die Vorgehensweise bei der Entwicklung eines
Software-Produkts\footcite[Seite 20-22]{text}
\end{description}

%-----------------------------------------------------------------------
%
%-----------------------------------------------------------------------

\subsection{Ermitteln der Anforderungen (requirements elicitation)}\footcite[Seite
15]{sosy:fs:1}

\begin{itemize}
\item (“Herauslocken“), Problem verstehen
\item Festlegen der Systemgrenzen
\item Definieren der Schnittstellen zwischen Software-System und Umgebung
\end{itemize}

\subsection{Techniken}

\begin{itemize}
\item Brainstorming
\item Fragebogen
\item Interview
\item Simulationsmodelle
\item Anforderungsreview
\item Workshop
\end{itemize}

%%%%%%%%%%%%%%%%%%%%%%%%%%%%%%%%%%%%%%%%%%%%%%%%%%%%%%%%%%%%%%%%%%%%%%%%
% Aufgaben
%%%%%%%%%%%%%%%%%%%%%%%%%%%%%%%%%%%%%%%%%%%%%%%%%%%%%%%%%%%%%%%%%%%%%%%%

\chapter{Aufgaben}

\section{Aufgabe\footcite{sosy:ab:2}}

Das Entwicklerteam von Teacher-Data hat einen neuen Auftrag bekommen.
Sie sollen für die Automaten-Videothek DVDRental die Software
entwickeln. Nach den ersten Gesprächen konnten folgende Informationen
zusammengetragen werden:

\begin{itemize}
\item DVD-Verleih-Automat:

\begin{itemize}
\item Der Kunde soll einfach, schnell und günstig am Automaten 24
Stunden, sieben Tage die Woche DVDs ausleihen können. Der Kunde hat des
Weiteren die Möglichkeit, Filme vorab über das Internet zu reservieren.
In der DVD-Verleihstation gibt es drei verschiedene Automaten (siehe
Zeichnung).

\item Um DVDs ausleihen zu können, müssen sich Neukunden während der
Service- und Anmeldezeiten der DVD-Verleihstation bei einem Mitarbeiter
anmelden. Dabei werden die benötigten Daten (\emph{Name},
\emph{Vorname}, \emph{Geburtsdatum}, \emph{Anschrift},
\emph{Personalausweisnummer}, \emph{Guthaben}, \emph{Kennwort}) des
Kunden aufgenommen und im System gespeichert. Zur persönlichen
Identifikation des Kunden wird anschließend der Fingerabdruck am
Automaten eingelesen. Der Kunde bekommt eine Magnetkarte ausgehändigt,
um sich an den Automaten anmelden zu können.

\item Beim DVD-Ausleih muss sich der Kunde mit der Magnetkarte und
seinem Fingerabdruck am Automaten für die Filmauswahl identifizieren. Am
Eingabe-Terminal hat der Kunde die Möglichkeit Filme nach \emph{Genre},
\emph{Schauspieler}, \emph{Titel}, \emph{TOP-10},
\emph{Neuerscheinungen} und \emph{Stichworten} zu suchen. Suchergebnisse
werden mit folgenden Informationen auf dem Display präsentiert:
\emph{Titel des Films}, \emph{Regisseur}, \emph{Hauptdarsteller},
\emph{Kurzbeschreibung}, \emph{Erscheinungsjahr}, \emph{Altersfreigabe}
und \emph{Verfügbarkeit}. Nach der Wahl eines Films, erhält der Kunde
seine Magnetkarte zurück und kann den Film am Automaten für die Ausund
Rückgabe nach erneuter Identifikation durch die Karte und den
Fingerabdruck abholen. Die DVD wird über den Ausgabeschacht ausgegeben.

\item Bei der Rückgabe einer DVD muss sich der Kunde am Automaten für
die Aus- und Rückgabe von Filmen mit der Magnetkarte und seinem
Fingerabdruck am Automaten identifizieren. Nach erfolgreicher
Identifikation führt der Kunde die DVD in den Rückgabeschacht ein. Der
Automat liest daraufhin den Barcode auf der DVD und sortiert diese
eigenständig wieder ein und bucht die Leihgebühr vom Guthaben auf der
Karte ab.

\item Das Guthaben kann am Automaten zum Aufladen des Guthabens
aufgeladen werden. Nach Eingabe der Kundenkarte können Münzen (5, 10,
20, 50 Cent, 1 und 2€) und Scheine (5, 10, 20, 50€) bar eingezahlt
werden. Bei einem Betrag von 20€ gibt es einen Bonus von 3€, bei 30€
einen von 5€ und bei einem von 50€ von 10€. Die Karte wird bei beendeter
Geldeingabe nach Drücken des roten Knopfes wieder ausgeworfen.

\item Reservierung von Filmen über das Internet: Nach erfolgreicher
Anmeldung mit Kundennummer und Geheimzahl auf der betreibereigenen
Homepage kann der Kunde nach \emph{Genre}, \emph{Schauspieler},
\emph{Titel}, \emph{TOP-10}, \emph{Neuerscheinungen} und
\emph{Stichworten} Filme suchen. Suchergebnisse werden mit folgenden
Informationen präsentiert: \emph{Titel des Films}, \emph{Regisseur},
\emph{Hauptdarsteller}, \emph{Kurzbeschreibung},
\emph{Erscheinungsjahr}, \emph{Altersfreigabe} und \emph{Verfügbarkeit}.
Der Kunde kann nach der Suche einzelne Filme reservieren, die er
innerhalb von zwei Stunden abholen muss. Sollte dies nicht in dem
Zeitraum geschehen, so werden die Filme nach zwei Stunden wieder zum
Ausleihen freigegeben.
\end{itemize}

\item Weitere Rahmenbedingungen:

\begin{itemize}

\item Wird die Kundenkarte im Automaten vergessen, so wird sie zur
Sicherheit automatisch eingezogen und kann während der Servicezeiten
wieder abgeholt werden.

\item Die maximale Leihdauer beträgt 10 Tage.

\item Hat der Kunde versehentlich den falschen Film ausgeliehen, so kann
er diese bis 10 Minuten nach Ausgabe der DVD ohne Berechnung einer
Leihgebühr wieder am Automaten zurückgeben.

\item Die Leihgebühr für jeden angefangenen Tag (24h) beträgt 2€.

\item Zu jedem Film gibt es mehrere Exemplare.

\item Die Ausgabe und Einsortierung von DVDs übernimmt ein
DVD-Archiv-Roboter, der über eine Schnittstelle angesteuert werden muss.
\end{itemize}

\end{itemize}

\begin{enumerate}

%%
% a)
%%

\item Identifizieren Sie die Aktoren des oben beschriebenen Systems.

\begin{antwort}
Kunde, Service-Mitarbeiter, Wartung, DVD-Archiv-Roboter
\end{antwort}

%%
% b)
%%

\item Identifizieren Sie die nicht-funktionalen und funktionalen
Anforderungen (als Use Cases).

\begin{antwort}
\begin{description}
\item[Nicht-funktional:] Bedienoberfläche Web / Automat gleich, einfache
Handhabbarkeit, schnell, Dauerbetrieb, Ausfallquote, Leihdauer

\item[Funktional:] Geldkarte aufladen, Film suchen, Film auswählen, Film
entnehmen, Film zurückgeben, Kunden aufnehmen, Film reservieren
\end{description}
\end{antwort}

\item Geben Sie eine formale Beschreibung für drei Use Cases an.
Orientieren können Sie sich an folgendem Beispiel:

\begin{antwort}
\ueberschrift{Geldkarte aufladen}

\begin{tabularx}{\linewidth}{p{3cm}X}
Use Case: &
Geldkarte aufladen \\

Ziel: &
Auf der Geldkarte befindet sich ein gewünschter Be-trag an Geld. \\

Kategorie: & primär \\

Vorbedingung: & Geldkarte befindet sich im Automaten. \\

Nachbedingung Erfolg: &
Auf der Geldkarte befindet sich der gewünschte Betrag. \\

Nachbedingung Fehlschlag: &
Betrag auf der Karte ist der gleiche wie davor  $\rightarrow$
Fehlermeldung \\

Akteure: & Kunde \\

Auslösendes Ereignis: &
Geldkarte wird in den Automaten zum Aufladen des Guthabens geschoben. \\
Beschreibung: &
\begin{enumerate}
\item Kunde schiebt Karte in den Automaten.
\item Kunde wirft Münzen oder gibt einen Geldschein ein.
\item Kunde drückt Knopf zum Auswerfen der Karte.
\item Karte wird ausgegeben.
\end{enumerate}
\\

Erweiterungen: &
Schein oder Münze wird nicht akzeptiert. $\rightarrow$ Fehlermeldung \\

Alternativen: & Mindest-Guthaben auf der Karte \\
\end{tabularx}

\ueberschrift{Film suchen}

\begin{tabularx}{\linewidth}{p{3cm}X}
\footnotesize
Use Case: & Film suchen \\

Ziel: & Anzeigen des gesuchten Films \\

Kategorie: & primär \\

Vorbedingung: &
Kunde hat sich mit Magnetkarte und Fingerabdruck identifiziert. \\

Nachbedingung Erfolg: & Eine Liste von Filmen wird angezeigt. \\

Nachbedingung Fehlschlag: &
Kein Film gefunden  $\rightarrow$ Fehlermeldung \\

Akteure: & Kunde \\

Auslösendes Ereignis: & Kunde will einen Film ausleihen. \\

Beschreibung: &
Kunde gibt Genre, Schauspieler, Titel oder Stichwort ein oder lässt sich
Neuerscheinungen und TOP10 auflisten. \\

Erweiterungen: & – \\
Alternativen: & Vorgaben von weiteren Suchkriterien
\end{tabularx}

\ueberschrift{Kunden aufnehmen}

\begin{tabularx}{\linewidth}{p{3cm}X}
Use Case: & Kunden aufnehmen \\

Ziel: & Kunde ist in der Kundendatei. \\

Kategorie: & primär \\

Vorbedingung: & Kunde ist noch nicht in der Kundendatei vorhanden. \\

Nachbedingung Erfolg: &
Kunde ist als Mitglied in der Kundendatei aufgenommen. \\

Nachbedingung Fehlschlag: &
Kunde kann nicht in die Datei aufgenommen werden. \\

Akteure: & Kunde, Service-Mitarbeiter \\

Auslösendes Ereignis: &
Kunde stellt zu den Öffnungszeiten bei einem Service-Mitarbeiter einen
Mitgliedsantrag. \\

Beschreibung: &

\begin{enumerate}
\item Kunde meldet sich zu den Öffnungszeiten bei einem
Service-Mitarbeiter an.

\item Service-Mitarbeiter erfasst nötige Daten (Name, Vorname,
Geburtsdatum, Anschrift, Personalausweisnummer, Kennwort).

\item Service-Mitarbeiter kassiert einen Startbetrag und erfasst ihn bei
den Kundendaten.

\item Der Fingerabdruck des Kunden wird erfasst und gespeichert.

\item Kunde bekommt eine Magnetkarte mit seinen Daten.
\end{enumerate}

\\
Erweiterungen: & - \\
Alternativen: &
Kunde ist bereits im System und lässt nur die Kundendaten ändern.
\end{tabularx}
\end{antwort}

%%
% d)
%%

\item Verfeinern Sie nun die Use Cases, indem Sie ähnliche Teile
identifizieren und daraus weitere Use Cases bilden.

\begin{antwort}
Identifizierung des Kunden an den Automaten, Filmsuche
\end{antwort}

%%
% e)
%%

\item Nennen Sie mindestens drei offene Fragen, die in einem weiteren
Gespräch mit dem Kunden noch geklärt werden müssten.\footnote{Quelle:
Universität Stuttgart, Institut für Automatisierungs- und
Softwaretechnik Prof. Dr.-Ing. Dr. h. c. P. Göhner}

\begin{antwort}
\begin{itemize}
\item Was passiert nach der maximlaen Leihdauer?
\item Was passiert, wenn der Finger nicht zur Karte passt?
\item Was passiert mit defekten DVDs?
\item Was passiert bei Rückgabe der falschen DVD?
\end{itemize}
\end{antwort}
\end{enumerate}

\literatur

\end{document}
