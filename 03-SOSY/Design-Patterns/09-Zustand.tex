\documentclass{lehramt-informatik}
\InformatikPakete{syntax,uml}

\begin{document}

%%%%%%%%%%%%%%%%%%%%%%%%%%%%%%%%%%%%%%%%%%%%%%%%%%%%%%%%%%%%%%%%%%%%%%%%
% Theorie-Teil
%%%%%%%%%%%%%%%%%%%%%%%%%%%%%%%%%%%%%%%%%%%%%%%%%%%%%%%%%%%%%%%%%%%%%%%%

\chapter{Zustand, Objekte für Zustände / State, Objects for states}

\begin{quellen}
\item \cite{wiki:zustand}
\item \url{https://www.philipphauer.de/study/se/design-pattern/state.php}
\item \cite[PDF Seite 258-265]{gof}
% \item \cite{schatten}
% \item \cite{eilebrecht}
\item \cite[Seite 69-81]{siebler}
\end{quellen}

\section{Zweck}

Das Zustandsmuster wird zur Kapselung unterschiedlicher,
zustandsabhängiger Verhaltensweisen eines Objektes eingesetzt.
\footcite{wiki:zustand}

%-----------------------------------------------------------------------
%
%-----------------------------------------------------------------------

\section{Klassendiagramm}

\begin{tikzpicture}
\umlclass[x=-1,y=3]{Context}{}{+request()}
\umlclass[x=3,y=3,type=interface]{State}{}{+handle()}
\umlclass[x=1,y=0]{ConcreteStateA}{}{+handle()}
\umlclass[x=5,y=0]{ConcreteStateB}{}{+handle()}

\umlVHVreal{ConcreteStateA}{State}
\umlVHVreal{ConcreteStateB}{State}

\umlaggreg{Context}{State}

\umlnote[x=-2,y=0,width=2cm]{Context}{state.handle()}
\end{tikzpicture}

Quelle: Englische Wikipedia, so ähnlich wie in GoF

%-----------------------------------------------------------------------
%
%-----------------------------------------------------------------------

\section{Allgemeines Code-Beispiel}

\def\TmpCode#1{\inputcode[firstline=3]{entwurfsmuster/zustand/allgemein/#1}}

\TmpCode{Kontext}
\TmpCode{Zustand}

%%%%%%%%%%%%%%%%%%%%%%%%%%%%%%%%%%%%%%%%%%%%%%%%%%%%%%%%%%%%%%%%%%%%%%%%
% Aufgaben
%%%%%%%%%%%%%%%%%%%%%%%%%%%%%%%%%%%%%%%%%%%%%%%%%%%%%%%%%%%%%%%%%%%%%%%%

\chapter{Aufgaben}

\section{Aufgabe 4: (Entwurfsmuster) (30 Punkte)\footcite{examen:66116:2019:09}}

Zu den Aufgaben eines Betriebssystems zählt die Verwaltung von
Prozessen. Jeder Prozess durchläuft verschiedene Zustände; Transitionen
werden durch Operationsaufrufe ausgelöst. Folgendes Zustandsdiagramm
beschreibt die Verwaltung von Prozessen:

\begin{center}
\begin{tikzpicture}
\umlstateinitial[x=0,y=5,name=initial]
\umlbasicstate[x=0,y=3]{Bereit}
\umlbasicstate[x=0,y=0]{Aktiv}
\umlbasicstate[x=6,y=0]{Suspendiert}
\umlstatefinal[x=0,y=-2, name=beendet]
\umlstatefinal[x=6,y=-2, name=abgebrochen]
\node [below=0cm of beendet] {\footnotesize{}Beendet};
\node [below=0cm of abgebrochen] {\footnotesize{}Abgebrochen};

\umltrans[arg=starten(),pos=0.5]{Bereit}{Aktiv}
\umltrans[arg=beenden(),pos=0.5]{Aktiv}{beendet}
\umltrans[arg=abbrechen(),pos=0.5]{Suspendiert}{abgebrochen}
\umltrans[pos=0.5]{initial}{Bereit}

\path[tikzuml transition style,draw,pos=0.5] (Aktiv.north east) -- node[auto]{\footnotesize{}suspendieren()} (Suspendiert.north west);
\path[tikzuml transition style,draw,pos=0.5] (Suspendiert.south west) -- node[auto]{\footnotesize{}fortsetzen()} (Aktiv.south east) ;
\end{tikzpicture}
\end{center}

\noindent
Implementieren Sie dieses Zustandsdiagramm in einer Programmiersprache
Ihrer Wahl mit Hilfe des Zustandsmusters; geben Sie die gewählte Sprache
an. Die Methoden für die Transitionen sollen dabei die Funktionalität
der Prozessverwaltung simulieren, indem der Methodenaufruf auf der
Standardausgabe protokolliert wird. Falls Transitionen im aktuellen
Zustand undefiniert sind, soll eine Fehlermeldung ausgegeben werden.

\begin{antwort}
\inputcode[firstline=3]{aufgaben/sosy/examen_66116_2019_09/Prozess}
\inputcode[firstline=3]{aufgaben/sosy/examen_66116_2019_09/ProzessZustand}
\inputcode[firstline=3]{aufgaben/sosy/examen_66116_2019_09/ZustandAbgebrochen}
\inputcode[firstline=3]{aufgaben/sosy/examen_66116_2019_09/ZustandAktiv}
\inputcode[firstline=3]{aufgaben/sosy/examen_66116_2019_09/ZustandBeendet}
\inputcode[firstline=3]{aufgaben/sosy/examen_66116_2019_09/ZustandBereit}
\inputcode[firstline=3]{aufgaben/sosy/examen_66116_2019_09/ZustandSuspendiert}
\end{antwort}

%-----------------------------------------------------------------------
%
%-----------------------------------------------------------------------

\section{Aufgabe 2: Klassendiagramme, STATE-Pattern, Zustandsdiagramme
\footcite{examen:66116:2018:09}}

Gegeben sei das Java-Programm:

\inputcode{aufgaben/sosy/examen_66116_2018_09/beatles/Music}

\begin{enumerate}

%%
% a)
%%

\item Zeichnen Sie ein UML-Klassendiagramm, das die statische Struktur
des Programms modelliert. Instanzvariablen mit einem Klassentyp sollen
durch gerichtete Assoziationen mit Rollennamen und passender
Multiplizität am gerichteten Assoziationsende modelliert werden. Alle
aus dem Programmcode ersichtlichen statischen Informationen sollen in
dem Klassendiagramm dargestellt werden.

\end{enumerate}

\noindent
Das Programm implementiert ein Zustandsdiagramm, das das
Verhalten von Objekten der Klasse \java{Music} beschreibt. Für die
Implementierung wurde das Design-Pattern \texttt{STATE} angewendet.

\begin{enumerate}
\setcounter{enumi}{2}

%%
% b)
%%

\item Geben Sie die statische Struktur des STATE-Patterns an und
erläutern Sie, welche Rollen aus dem Entwurfsmuster den Klassen des
gegebenen Programms dabei zufallen und welche Operationen aus dem
Entwurfsmuster durch (ggf. mehrere) Methoden in unserem Beispielprogramm
implementiert werden. Es ist von den z. B. im Design-Pattern-Katalog von
Gamma et al. verwendeten Namen auszugehen, das heißt von Klassen mit
Namen \java{Context}, \java{State}, \java{ConcreteStateA},
\java{ConcreteStateB} und von Operationen mit Namen \java{request} und
\java{handle}.

\item Zeichnen Sie das UML-Zustandsdiagramm (mit Anfangszustand), das
von dem Programm implementiert wird. Dabei muss - gemäß der UML-Notation
- unterscheidbar sein, was Ereignisse und was Aktionen sind. In dem
Diagramm kann zur Vereinfachung statt \java{System.out.println ("x")}
einfach \java{"x"} geschrieben werden.

\end{enumerate}

\literatur

\end{document}
