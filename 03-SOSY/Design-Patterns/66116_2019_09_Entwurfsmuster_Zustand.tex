\documentclass{article}
\usepackage{tikz-uml}
\usetikzlibrary{positioning}
\begin{document}

\section{Aufgabe 4: (Entwurfsmuster) (30 Punkte)}

Zu den Aufgaben eines Betriebssystems zählt die Verwaltung von
Prozessen. Jeder Prozess durchläuft verschiedene Zustände; Transitionen
werden durch Operationsaufrufe ausgelöst. Folgendes Zustandsdiagramm
beschreibt die Verwaltung von Prozessen:

\begin{center}
\begin{tikzpicture}
\umlstateinitial[x=0,y=5,name=initial]
\umlbasicstate[x=0,y=3]{Bereit}
\umlbasicstate[x=0,y=0]{Aktiv}
\umlbasicstate[x=6,y=0]{Suspendiert}
\umlstatefinal[x=0,y=-2, name=beendet]
\umlstatefinal[x=6,y=-2, name=abgebrochen]
\node [below=0cm of beendet] {\footnotesize{}Beendet};
\node [below=0cm of abgebrochen] {\footnotesize{}Abgebrochen};

\umltrans[arg=starten(),pos=0.5]{Bereit}{Aktiv}
\umltrans[arg=beenden(),pos=0.5]{Aktiv}{beendet}
\umltrans[arg=abbrechen(),pos=0.5]{Suspendiert}{abgebrochen}
\umltrans[pos=0.5]{initial}{Bereit}

\path[tikzuml transition style,draw,pos=0.5] (Aktiv.north east) -- node[auto]{\footnotesize{}suspendieren()} (Suspendiert.north west);
\path[tikzuml transition style,draw,pos=0.5] (Suspendiert.south west) -- node[auto]{\footnotesize{}fortsetzen()} (Aktiv.south east) ;
\end{tikzpicture}
\end{center}

\noindent
Implementieren Sie dieses Zustandsdiagramm in einer Programmiersprache
Ihrer Wahl mit Hilfe des Zustandsmusters; geben Sie die gewählte Sprache
an. Die Methoden für die Transitionen sollen dabei die Funktionalität
der Prozessverwaltung simulieren, indem der Methodenaufruf auf der
Standardausgabe protokolliert wird. Falls Transitionen im aktuellen
Zustand undefiniert sind, soll eine Fehlermeldung ausgegeben werden.

\end{document}
