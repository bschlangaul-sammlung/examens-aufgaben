\documentclass{lehramt-informatik}
\InformatikPakete{syntax,uml}

\begin{document}

%%%%%%%%%%%%%%%%%%%%%%%%%%%%%%%%%%%%%%%%%%%%%%%%%%%%%%%%%%%%%%%%%%%%%%%%
% Theorie-Teil
%%%%%%%%%%%%%%%%%%%%%%%%%%%%%%%%%%%%%%%%%%%%%%%%%%%%%%%%%%%%%%%%%%%%%%%%

\chapter{Adapter}

\begin{quellen}
\item \cite{wiki:adapter}
% \item \url{https://www.philipphauer.de/study/se/design-pattern/}
\item \cite[Seite 120-129]{gof}
\item \cite[Kapitel 8., Seite 255]{schatten}
\item \cite[Kapitel 5.1, Seite 77-79]{eilebrecht}
\item \cite[Kapitel 20, Seite 243]{siebler}
\end{quellen}

\section{Zweck}

Ein Adapter passt die Schnittstelle einer Klasse an eine andere von
ihren Klienten erwartete Schnittstelle an. Das Adaptermuster lässt
\memph{Klassen zusammenarbeiten}, die andernfalls dazu nicht in der Lage
wären.
\footcite[Seite 77]{eilebrecht}

%-----------------------------------------------------------------------
%
%-----------------------------------------------------------------------

\section{Klassendiagramm}

\begin{tikzpicture}
\umlsimpleclass[x=0,y=3]{Klient}{}{}
\umlclass[x=4,y=3]{Ziel}{}{agiere()}
\umlclass[x=4,y=0]{Adapter}{}{agiere()}
\umlclass[x=8,y=1.5]{Dienst}{}{agiereSpeziell()}

\umlreal{Adapter}{Ziel}
\umluniassoc{Klient}{Ziel}
\umlinherit{Adapter}{Dienst}

\umlnote[x=7,y=-1,width=2cm]{Adapter}{agiereSpeziell()}
\end{tikzpicture}

%-----------------------------------------------------------------------
%
%-----------------------------------------------------------------------

\section{Allgemeines Code-Beispiel}

\def\TmpCode#1{\inputcode[firstline=3]{entwurfsmuster/adapter/allgemein/#1}}

\TmpCode{Dienst}
\TmpCode{Ziel}
\TmpCode{Adapter}
\TmpCode{Klient}

%%%%%%%%%%%%%%%%%%%%%%%%%%%%%%%%%%%%%%%%%%%%%%%%%%%%%%%%%%%%%%%%%%%%%%%%
% Aufgaben
%%%%%%%%%%%%%%%%%%%%%%%%%%%%%%%%%%%%%%%%%%%%%%%%%%%%%%%%%%%%%%%%%%%%%%%%

\chapter{Aufgaben}

\literatur

\end{document}
