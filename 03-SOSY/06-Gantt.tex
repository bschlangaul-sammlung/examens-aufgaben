\documentclass{lehramt-informatik}
\InformatikPakete{gantt,cpm}
\begin{document}

%%%%%%%%%%%%%%%%%%%%%%%%%%%%%%%%%%%%%%%%%%%%%%%%%%%%%%%%%%%%%%%%%%%%%%%%
% Theorie-Teil
%%%%%%%%%%%%%%%%%%%%%%%%%%%%%%%%%%%%%%%%%%%%%%%%%%%%%%%%%%%%%%%%%%%%%%%%

\chapter{Gantt-Diagramm}

\begin{quellen}
\item \cite[Seite 194]{schatten}
\item \cite{wiki:gantt-diagramm}
\end{quellen}

Ein Gantt-Diagramm ist ein Instrument des Projektmanagements, das die
zeitliche Abfolge von Aktivitäten graphisch in Form von Balken auf einer
Zeitachse darstellt.

\section{Anordnungsbeziehungen im Gantt Diagramm\footcite[Seite 34]{sosy:fs:3}}

\begin{description}
\item[Normalfolge EA:] end-to-start relationship Anordnungsbeziehung vom
Ende eines Vorgangs zum Anfang seines Nachfolgers.

\begin{center}
\begin{ganttchart}{1}{9}
\ganttbar{}{2}{4} \\
\ganttbar{}{6}{8}
\ganttlink[link type=f-s]{elem0}{elem1}
\end{ganttchart}
\end{center}

\item[Anfangsfolge AA:] start-to-start relationship Anordnungsbeziehung
vom Anfang eines Vorgangs zum Anfang seines Nachfolgers.

\begin{center}
\begin{ganttchart}{1}{9}
\ganttbar{}{2}{4} \\
\ganttbar{}{6}{8}
\ganttlink[link type=s-s]{elem0}{elem1}
\end{ganttchart}
\end{center}

\item[Endefolge EE:] finish-to-finish relationship Anordnungsbeziehung
vom Ende eines Vorgangs zum Ende seines Nachfolgers.

\begin{center}
\begin{ganttchart}{1}{9}
\ganttbar{}{2}{4} \\
\ganttbar{}{6}{8}
\ganttlink[link type=f-f]{elem0}{elem1}
\end{ganttchart}
\end{center}

\item[Sprungfolge AE:] start-to-finish relationship Anordnungsbeziehung
vom Anfang eines Vorgangs zum Ende seines Nachfolgers

\begin{center}
\begin{ganttchart}{1}{9}
\ganttbar{}{2}{4} \\
\ganttbar{}{6}{8}
\ganttlink[link type=s-f]{elem0}{elem1}
\end{ganttchart}
\end{center}

\end{description}

\section{Von Gantt nach CPM}

\begin{enumerate}

\item Zeichne (falls gefordert bzw. nötig) für das Projekt einen Start-
und Endknoten

\item Zeichen für jede Aktivität (= Balken im Gantt Diagramm) einen
Start- und einen Endknoten

\item Zeichne zwischen Start- und Endknoten einen Vorgang ein. Seine
Dauer entspricht der Länge des Balkens (falls Pufferzeiten in Klammern
angegeben sind werden sie von der Balkenlänge abgezogen).

\item Zeichne für jede Anordnungsbeziehung die im Gantt Diagramm
eingezeichnet ist einen Vorgang ein. Ist der zeitliche Versatz zwischen
dem Balken der Aktivität und dem der Nachfolge-Aktivität

\begin{itemize}
\item positiv, so entspricht die Richtung des Pfeils der im Gantt
Diagramm ist.

\item negativ so wird die Richtung des Pfeils im CPM Netz umgekehrt.

\item null, so zeichne einen Scheinvorgang ein.
\end{itemize}

Der Betrag des zeitlichen Versatzes der Balken ist die Dauer des
Vorgangs (also immer positiv).
\end{enumerate}

\footcite[Seite 35]{sosy:fs:3}

\section{Von CPM nach Gantt}

\begin{enumerate}
\item Zeichne für jeden Vorgang einen Balken ein. Beginne beim FAZ und
ende beim SEZ des Vorgangs. Optional: Trage die Bezeichnung des Vorgangs
ein.

\item Optional (falls gefordert): Trage Dauer und Puffer ein.

\item Pfeile im CPM die zwischen Anfangs- und Endpunkt verschiedener
Vorgänge verlaufen werden zu Anordnungsbeziehungen (AOBs):

\begin{itemize}
\item identifiziere im CPM denjenigen Vorgang, der der Nachfolgevorgang
ist (z.B. Entwurf ist Nachfolger der Anforderungsanalyse).

\item identifiziere damit die Art der AOB (EA, AA, EE, AE)

\item Zeichne sie im Gantt Diagramm ein

\item Optional (falls gefordert): beschrifte sie mit der identifizierten
Art (EA, AA, EE, AE) und ggf. der

\begin{itemize}
\item Verzögerung = positiver Wert: Pfeil zeigt im CPM zum
Nachfolgevorgang

\item Überlappung = negativer Wert: Pfeil geht im CPM vom
Nachfolgevorgang aus
\end{itemize}
\end{itemize}
\end{enumerate}

\footcite[Seite 36]{sosy:fs:3}

%%%%%%%%%%%%%%%%%%%%%%%%%%%%%%%%%%%%%%%%%%%%%%%%%%%%%%%%%%%%%%%%%%%%%%%%
% Aufgaben
%%%%%%%%%%%%%%%%%%%%%%%%%%%%%%%%%%%%%%%%%%%%%%%%%%%%%%%%%%%%%%%%%%%%%%%%

\chapter{Aufgaben}

%-----------------------------------------------------------------------
%
%-----------------------------------------------------------------------

\section{Aufgabe 2: Gantt\footcite[Seite 1]{sosy:ab:5}}

In Gantt-Diagrammen unterscheidet man vier Anordnungsbeziehungen:
Normalfolge (EA), Anfangsfolge (AA), Endfolge (EE) und Sprungfolge
(AE). Ordnen Sie folgende Beispielen den Anordnungsbeziehungen (EA, AA,
EE, AE) zu.

\begin{enumerate}

%%
% (a)
%%

\item Tests durchführen und Dokumentation erstellen

\begin{antwort}
EE (Das Testen muss abgeschlossen sein bevor die Erstellung der
Dokumentation beendet werden kann.)
\end{antwort}

%%
% (b)
%%

\item Systementwurf und Implementierung

\begin{antwort}
AA (Der Beginn des Systementwurfs ist Voraussetzung für den Beginn der
Implementierung.)
\end{antwort}

%%
% (c)
%%

\item neue Anwendung in Betrieb nehmen und alte Anwendung abschalten

\begin{antwort}
AE (Die alte Anwendung kann erst abgeschaltet werden, wenn die neue
Anwendung in Betrieb genommen wurde.)
\end{antwort}

%%
% (d)
%%

\item Implementierung und Dokumentation

\begin{antwort}
EE (Die Implementierung muss abgeschlossen sein bevor die Dokumentation
beendet werden kann.)
\end{antwort}

%%
% (e)
%%

\item Abitur schreiben und Studieren

\begin{antwort}
EA (Das Abitur muss abgeschlossen sein bevor mit dem Studium begonnen
werden kann.)
\end{antwort}

%%
% (f)
%%

\item Führerschein machen und selbstständiges Autofahren

\begin{antwort}
EA (Der Führerschein muss bestanden sein bevor mit dem selbstständigen
Autofahren begonnen werden kann.)
\end{antwort}

%%
% (g)
%%

\item Studieren und Buch aus Uni-Bibliothek ausleihen

\begin{antwort}
AA (Der Beginn des Studiums ist Voraussetzung für das Ausleihen eines
Buches aus der Uni-Bibliothek.)
\end{antwort}
\end{enumerate}

%-----------------------------------------------------------------------
%
%-----------------------------------------------------------------------

\section{Aufgabe 3: Gantt (Staatsexamen H18 T2 TA1 A1, erweiterte
Aufgabenstellung*)\footcite[Seite 2]{sosy:ab:5}}

Ein Team von zwei Softwareentwicklern soll ein Projekt umsetzen, das in
sechs Arbeitspakete unterteilt ist. Die Dauer der Arbeitspakete und ihre
Abhängigkeiten können Sie aus folgender Tabelle entnehmen:
\footcite{examen:66116:2018:09}

\begin{center}
\begin{tabular}{|l|l|l|}
\hline
Name & Dauer in Wochen & Abhängig von\\\hline\hline
A1 & 2 & - \\\hline
A2 & 5 & - \\\hline
A3 & 2 & A1 \\\hline
A4 & 5 & A1,A2 \\\hline
A5 & 7 & A3 \\\hline
A6 & 4 & A4 \\\hline
\end{tabular}
\end{center}

\begin{enumerate}

%%
% (a)
%%

\item Zeichnen Sie ein Gantt-Diagramm, das eine kürzestmögliche
Projektabwicklung beinhaltet.

\begin{antwort}
\begin{center}
\begin{ganttchart}[x unit=0.7cm, y unit chart=0.6cm]{1}{14}
\gantttitlelist{1,...,14}{1} \\
\ganttbar[name=A1]{A1}{3}{5} \\
\ganttbar[name=A2]{A2}{1}{5} \\
\ganttbar[name=A3]{A3}{6}{7} \\
\ganttbar[name=A4]{A4}{6}{10} \\
\ganttbar[name=A5]{A5}{8}{14} \\
\ganttbar[name=A6]{A6}{11}{14}
\ganttlink{A1}{A3}
\ganttlink{A1}{A4}
\ganttlink[link type=f-s]{A2}{A4}
\ganttlink[link type=f-s]{A3}{A5}
\ganttlink[link type=f-s]{A4}{A6}
\end{ganttchart}
\end{center}
\end{antwort}

%%
% (b)
%%

\item Bestimmen Sie die Länge des kritischen Pfades und geben Sie an,
welche Arbeitspakete an ihm beteiligt sind.

\begin{antwort}
Auf dem kritischen Pfad befinden die Arbeitspakete A2, A4 und A6. Die
Länge des kritischen Pfades ist 14.
\end{antwort}

%%
% (c*)
%%

\item Wandeln Sie das Gantt-Diagramm in ein CPM-Netzplan um.

\begin{antwort}
\begin{tikzpicture}[font=\scriptsize,scale=0.95]
\ereignis{SP}(0,1.5)

\ereignis{1A}(1.5,3)
\ereignis{1E}(3,3)
\ereignis{2A}(1.5,0)
\ereignis{2E}(3,0)
\ereignis{3A}(4.5,3)
\ereignis{3E}(6,3)
\ereignis{4A}(4.5,0)
\ereignis{4E}(6,0)
\ereignis{5A}(7.5,3)
\ereignis{5E}(9,3)
\ereignis{6A}(7.5,0)
\ereignis{6E}(9,0)

\ereignis{EP}(10.5,1.5)

\vorgang(1A>1E){2}
\vorgang(2A>2E){5}
\vorgang(3A>3E){2}
\vorgang(4A>4E){5}
\vorgang(5A>5E){7}
\vorgang(6A>6E){4}

\scheinvorgang(1E>3A){}
\scheinvorgang(2E>4A){}
\scheinvorgang(1E>4A){}
\scheinvorgang(3E>5A){}
\scheinvorgang(4E>6A){}

\scheinvorgang(SP>1A){}
\scheinvorgang(SP>2A){}

\scheinvorgang(5E>EP){}
\scheinvorgang(6E>EP){}
\end{tikzpicture}
\end{antwort}

%%
% (d*)
%%

\item Berechnen Sie für jedes Ereignis den \emph{frühesten Termin} und
den \emph{spätesten Termin} sowie die \emph{Gesamtpufferzeiten}.

\begin{antwort}
{\scriptsize
\setlength{\tabcolsep}{5pt}
\begin{tabular}{|l|c|c|c|c|c|c|c|c|c|c|c|c|c|c|}
\hline
Ergebnis &SP&1A&1E&2A&2E&3A&3E&4A&4E&5A&5E&6A&6E&EP\\\hline\hline
$FZ_i$   &0 &0 &2 &0 &5 &2 &4 &5 &10&4 &11&10&14&14\\\hline
$SZ_i$   &0 &3 &5 &0 &5 &5 &7 &5 &10&7 &14&10&14&14\\\hline
$GP$     &0 &3 &3 &0 &0 &3 &3 &0 &0 &3 &3 &0 &0 &0 \\\hline
\end{tabular}
}
\end{antwort}
\end{enumerate}

%-----------------------------------------------------------------------
%
%-----------------------------------------------------------------------

\section{Projektmanagement}

Gegeben ist das folgende Gantt-Diagramm zur Planung eines
hypothetischen Softwareprojekts:
\footcite{examen:66116:2017:09}

\begin{enumerate}

%%
% a)
%%

\item Konvertieren Sie das Gantt-Diagramm in ein CPM-Netzwerk, das die
Aktivitäten und Abhängigkeiten äquivalent beschreibt. Gehen Sie von der
Zeiteinheit „Monate“ aus. Definieren Sie im CPM-Netzwerk je einen
globalen Start- und Endknoten. Der Start jeder Aktivität hängt dabei vom
Projektstart ab, das Projektende hängt vom Ende aller Aktivitäten ab.

%%
% b)
%%

\item Berechnen Sie für jedes Ereignis (d.\,h. für jeden Knoten Ihres
CPM-Netzwerks) die früheste Zeit, die späteste Zeit sowie die
Pufferzeit. Beachten Sie, dass die Berechnungsreihenfolge einer
topologischen Sortierung des Netzwerks entsprechen sollte.

%%
% c)
%%

\item Geben Sie einen kritischen Pfad durch das CPM-Netzwerk an.
Welche Aktivität darf sich demnach wie lange verzögern?
\end{enumerate}

%-----------------------------------------------------------------------
%
%-----------------------------------------------------------------------

\section{Projektmanagement\footcite{examen:66116:2016:03}}

\begin{enumerate}

%%
% (a)
%%

\item Erklären Sie in maximal zwei Sätzen den Unterschied zwischen
Knoten- und Kantennetzwerken im Kontext des Projektmanagements.

%%
% (b)
%%

\item Gegeben ist die folgende Tabelle zur Grobplanung eines
hypothetischen Softwareprojekts:

Aktivität
Anforderungs-
analyse

Minimale Dauer
2 Monate
Entwurf 4 Monate
Implemen-
tierung 5 Monate
Einschränkungen
Endet frühestens einen Monat nach dem Start der
Entwurfsphase.
Startet frühestens zwei Monate nach dem Start der
Anforderungsanalyse.
Endet frühestens drei Monate nach dem Endeder
Entwurfsphase. Darf erst starten, nachdem die
Anforderungsanalyse abgeschlossenist.

Geben Sie ein CPM-Netzwerkan, das die Aktivitäten und Abhängigkeit des
obigen Projektplans beschreibt. Gehen Sie von der Zeiteinheit „Monate“
aus. Das Projekt hat einen Start- und einen Endknoten.

Jede Aktivität wird auf einen Start- und einen Endknoten abgebildet. Die
Dauer der Aktivitäten sowie Abhängigkeiten sollen durch Kanten
dargestellt werden. Der Start jeder Aktivität hängt vom Projektstart ab,
das Projektende hängt vom Ende aller Aktivitäten ab. Modellieren Sie
diese Abhängigkeiten durch Pseudoaktivitäten mit Dauer null.

%%
% (c)
%%

\item Berechnen Sie für jedes Ereignis (d.h. für jeden Knoten) die
früheste Zeit sowie die späteste Zeit. Beachten Sie, dass die
Berechnungsreihenfolge einer topologischen Sortierung des Netzwerks
entsprechensollte.

%%
% (d)
%%

\item Geben Sie einen kritischen Pfad durch das CPM-Netzwerk an.
Möglicherweise sind hierfür weitere Vorberechnungen vonnöten. Welche
Aktivität sollte sich demnach auf keinen Fall verzögern?

%%
% (e)
%%

\item Geben Sie ein Gantt-Diagramm an, das den Projektplan visualisiert.
Gehen Sie davon aus, dass jede Aktivität zur frühesten Zeitihres
Startknotens beginnt und zur spätesten Zeit ihres Endknotens endet(s.
jeweils Teilaufgabe (c)). Geben Sie die minimale Dauer jeder Aktivität,
sowie die Pufferzeit (in Klammern) an. Beispiel: 4 (+2). Notieren Sie
alle Einschränkungen mit Hilfe geeigneter Abhängigkeitsbeziehungen.
Geben Sie eine absolute Zeitskala in Monaten an.

%%
% (f)
%%

\item Nennen Sie zwei weitere Aktivitäten, die in der obigen Tabelle
fehlen, jedoch typischerweise in Softwareentwicklungs-Prozessmodellen
wie etwa dem Wasserfallmodell vorkommen.

\end{enumerate}

%-----------------------------------------------------------------------
%
%-----------------------------------------------------------------------

\section{Aufgabe 5: Projektmanagement\footcite{examen:46116:2017:03}}

Betrachten Sie die folgende Tabelle zum Projektmanagement:

Arbeitspaket | Dauer (Tage) | Abhängig von
Al 5
A2 10
A3 5 Al
AA 15 A2, A3
AS 10 Al
A6 10 Al, A2
A7 10 A2, A4
A8 15 A4, AS

Tabelle 1: Übersicht Arbeitspakete

\begin{enumerate}

%%
% a)
%%

\item Erstellen Sie ein Gantt-Diagramm, das die in der Tabelle
angegebenen Abhängigkeiten berücksichtigt.

%%
% b)
%%

\item Wie lange dauert das Projekt mindestens?

%%
% c)
%%

\item Geben Sie den oder die kritischen Pfad(e) an.

%%
% d)
%%

\item Konstruieren Sie ein PERT-Chart zum obigen Problem.

\end{enumerate}

%-----------------------------------------------------------------------
%
%-----------------------------------------------------------------------

\section{3. Projektmanagement\footcite{examen:46116:2015:09}}

Betrachten Sie die folgende Tabelle zum Projektmanagement:

Arbeitspaket | Dauer (Tage) | abhängig von
Al 10
A2 5 Al
A3 15 Al
Aa 10 A2, AB
AS 15 Ai, A3
A6 10
AT 5 A2, A4
AS 10 A4, A5, A6

Tabelle 1: Ubersicht Arbeitspakete

\begin{enumerate}

%%
% a)
%%

\item Erstellen Sie ein Gantt-Diagramm, das die in der Tabelle
angegebenen Abhängigkeiten berücksichtigt.

%%
% b)
%%

\item Wie lange dauert das Projekt mindestens?

%%
% c)
%%

\item Geben Sie den oder die kritischen Pfad(e) an.

%%
% d)
%%

\item Konstruieren Sie ein PERT-Chart zum obigen Problem.

\end{enumerate}

\literatur

\end{document}
