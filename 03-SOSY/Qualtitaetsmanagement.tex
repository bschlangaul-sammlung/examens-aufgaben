\documentclass{lehramt-informatik}
\InformatikPakete{syntax}
\usepackage{tabularx}

\begin{document}

%%%%%%%%%%%%%%%%%%%%%%%%%%%%%%%%%%%%%%%%%%%%%%%%%%%%%%%%%%%%%%%%%%%%%%%%
% Theorie-Teil
%%%%%%%%%%%%%%%%%%%%%%%%%%%%%%%%%%%%%%%%%%%%%%%%%%%%%%%%%%%%%%%%%%%%%%%%

\chapter{Qualitätsmanagement}

%%%%%%%%%%%%%%%%%%%%%%%%%%%%%%%%%%%%%%%%%%%%%%%%%%%%%%%%%%%%%%%%%%%%%%%%
% Aufgaben
%%%%%%%%%%%%%%%%%%%%%%%%%%%%%%%%%%%%%%%%%%%%%%%%%%%%%%%%%%%%%%%%%%%%%%%%

\chapter{Aufgaben}

\section{Aufgabe 1\footcite{sosy:ab:9}}

\begin{itemize}
\item Nennen Sie jeweils einen Vorteil und einen Nachteil für
Qualitätssicherung durch „Testing“ bzw. durch „Model Checking“.\footcite[Herbst 2013 (66116) - Thema 1, Teilaufgabe 2, Aufgabe 3]{examen:66116:2013:09}

\item Definieren Sie den Begriff „Refactoring“.

\item Begründen Sie, warum bei der Entwicklung nach der Methode des
„eXtreme Programming“ langfristig gesehen Refactorings zwingend notwendig
werden.

\item Wie wird in der Praxis während und nach erfolgtem Refactoring
sichergestellt, dass keine neuen Defekte eingeführt werden bzw. wurden?

\item Worin besteht der Unterschied zwischen „White-Box-Testing“ und
„Black-Box-Testing“?

\item Nennen Sie vier Qualitätsmerkmale von Software.

\item Worin besteht der Unterschied zwischen funktionalen und
nicht-funktionalen Anforderungen?

\item Was verbirgt sich hinter dem Begriff „Continuous Integration“?

\item Nennen Sie sechs Herausstellungsmerkmale des
„eXtreme Programming“ Ansatzes.

\item Was versteht man unter einem Unit-Test? Begründen Sie, warum es
unzureichend ist, wenn eine Test-Suite ausschließlich Unit-Tests
enthält.

\item Nennen Sie jeweils eine Methodik, mit welcher in der Praxis die
Prozesse der „Validierung“ und der „Verifikation“ durchgeführt werden.

\item Grenzen Sie die Begriffe „Fault“ und „Failure“ voneinander ab.

\end{itemize}

%-----------------------------------------------------------------------
%
%-----------------------------------------------------------------------

\section{Aufgabe 2\footcite{sosy:ab:9}}

\begin{description}

%%
%
%%

\item[A] Allgemein\footcite[Herbst 2016 (66116) - Thema 2, Teilaufgabe 2, Aufgabe 1a]{examen:66116:2016:09}

\begin{description}
\item[A1] Im Software Engineering geht es vor allem darum qualitativ
hochwertige Software zu entwickeln.

\item[A2] Software Engineering ist gleichbedeutend mit Programmieren.
\end{description}

%%
%
%%

\item[B] Vorgehensmodelle

\begin{description}
\item[B1] Die Erhebung und Analyse von Anforderungen sind nicht Teil des
Software Engineerings.

\item[B2] Agile Methoden eignen sich besonders gut für die Entwicklung
komplexer und sicherer Systeme in verteilten Entwicklerteams.

\item[B3] Das Spiralmodell ist ein Vorläufer sogenannter Agiler
Methoden.
\end{description}

%%
%
%%

\item[C] Anforderungserhebung

\begin{description}
\item[C1] Bei der Anforderungserhebung dürfen in keinem Fall mehrere
Erhebungstechniken (z. B. Workshops, Modellierung) angewendet werden,
weil sonst Widersprüche in Anforderungen zu, Vorschein kommen könnten.

\item[C2] Ein Szenario beinhaltet eine Menge von Anwendungsfällen.

\item[C3] Nicht-funktionale Anforderungen sollten, wenn möglich, immer
quantitativ spezifiziert werden.
\end{description}

%%
%
%%

\item[D] Architekturmuster

\begin{description}
\item[D1] Schichtenarchitekturen sind besonders für Anwendungen
geeignet, in denen Performance eine wichtige Rolle spielt.

\item[D2] Das Black Board Muster ist besonders für Anwendungen geeignet,
in denen Performance eine wichtige Rolle spielt.

\item[D3] „Dependency Injection“ bezeichnet das Konzept, welches
Abhängigkeiten zur Laufzeit reglementiert.
\end{description}

%%
%
%%

\item[E] UML

\begin{description}
\item[E1] Sequenzdiagramme beschreiben Teile des Verhaltens eines
Systems.

\item[E2] Zustandsübergangsdiagramme beschreiben das Verhalten eines
Systems.

\item[E3] Komponentendiagramme beschreiben die Struktur eines Systems.
\end{description}

%%
%
%%

\item[F] Entwurfsmuster

\begin{description}
\item[F1] Das MVC Pattern verursacht eine starke Abhängigkeit zwischen
Datenmodell und Benutzeroberfläche.

\item[F2] Das Singleton Pattern stellt sicher, dass es zur Laufzeit von
einer bestimmten Klasse höchstens ein Objekt gibt.

\item[F3] Im Kommando Enwurfsmuster (engl. „Command Pattern“) werden
Befehle in einem sog. Kommando-Objekt gekapselt, um sie bei Bedarf
rückgängig zu machen.
\end{description}

%%
%
%%

\item[G] Testen

\begin{description}
\item[G1] Validation dient der Überprüfung von Laufzeitfehlern.

\item[G2] Testen ermöglicht sicherzustellen, dass ein Programm absolut
fehlerfrei ist.

\item[G3] Verifikation dient der Überprüfung, ob ein System einer
Spezifikation entspricht.
\end{description}

\end{description}

%-----------------------------------------------------------------------
%
%-----------------------------------------------------------------------

\section{Aufgabe 3\footcite[Seite 3]{sosy:ab:9}}

Die folgende Seite enthält Software-Quellcode, der einen Algorithmus zur
binären Suche implementiert. Dieser ist durch Inspektion zu überprüfen.
Im Folgenden sind die Regeln der Inspektion angegeben.
\footcite{examen:66116:2017:09}

\noindent
\begin{tabularx}{\linewidth}{|l|l|X|}
\hline
RM1 &
(Dokumentation) &
Jede Quellcode-Datei beginnt mit einem Kommentar, der
den Klassennamen, Versionsinformationen, Datum und
Urheberrechtsangaben enthält.
\\\hline

RM2 &
(Dokumentation) &
Jede Methode wird kommentiert. Der Kommentar enthält
eine vollständige Beschreibung der Signatur so wie eine
Design-by-Contract-Spezifikation.
\\\hline

RM3 &
(Dokumentation) &
Deklarationen von Variablen werden kommentiert.
\\\hline

RM4 &
(Dokumentation) &
Jede Kontrollstruktur wird kommentiert.
\\\hline

RM5 &
(Formatierung) &
Zwischen einem Schlüsselwort und einer Klammer steht
ein Leerzeichen.
\\\hline

RM6 &
(Formatierung) &
Zwischen binären Operatoren und den Operanden stehen
Leerzeichen.
\\\hline

RM7 &
(Programmierung) &
Variablen werden in der Anweisung initialisiert, in der sie
auch deklariert werden.
\\\hline

RM8 &
(Bezeichner) &
Klassennamen werden groß geschrieben, Variablennamen klein.
\\\hline
\end{tabularx}

\begin{enumerate}
\item Überprüfen Sie durch Inspektion, ob die obigen Regeln für den
Quellcode eingehalten wurden. Erstellen Sie eine Liste mit allen
Verletzungen der Regeln. Geben Sie für jede Verletzung einer Regel die
Zeilennummer, Regelnummer und Kommentar an, z.\,B.(07, RM4, while nicht
kommentiert).[...]

\item Entspricht die Methode ‘binarySearch’ ihrer Spezifikation, die
durch Vor-und Nachbedingungen angeben ist? Geben Sie gegebenenfalls
Korrekturen der Methode an.

\item Beschreiben alle Kommentare ab Zeile 24 die Semantik des Codes
korrekt? Geben Sie zu jedem falschen Kommentar einen korrigierten
Kommentar mit Zeilennummer an.

\item Geben Sie den Kontrollflussgraphen für die Methode ‘binarySearch’
an.

\item Geben Sie maximal drei Testfälle für die Methode ‘binarySearch’
an, die insgesamt eine vollständige Anweisungsüberdeckung leisten.

\inputcode{aufgaben/sosy/ab_9/BinarySearch}

\end{enumerate}

\literatur

\end{document}
