\documentclass{lehramt-informatik-minimal}
\InformatikPakete{syntax,kontrollflussgraph}
\begin{document}

\begin{minted}{java}
public int ggT(int a, int b) {
  int result = 1;
  for (int i = 1; i <= Math.min(a, b); i++) {
    if((a % 1 == 0) & (b % i == 0)) {
      result = i;
    }
  }
  return result;
}
\end{minted}

\begin{tikzpicture}[kontrollfluss,xscale=1,yscale=-1.2]
\node[knoten] at (0,0) (nS) {nS}; % public int ggT(int a, int b) {
\node[knoten] at (0,1) (n1) {n1}; % int result = 1;
\node[knoten] at (0,2) (n2) {n2}; % in der for: int i = 1;
\node[knoten] at (0,3) (n3) {n3}; % in der for: i <= Math.min(a, b);
\node[knoten] at (0,4) (n4) {n4}; % if((a % 1 == 0) & (b % i == 0)) {
\node[knoten] at (1,5) (n5) {n5}; % result = i;
\node[knoten] at (0,6) (n6) {n6}; % i++
\node[knoten] at (0,7) (nE) {nE}; % return result;

\draw[->] (nS) -- (n1);
\draw[->] (n1) -- (n2);
\draw[->] (n2) -- (n3);
\draw[->] (n3) -- node[name=n3n4]{} (n4);
\draw[->] (n4) -- node[name=n4n5]{} (n5);
\draw[->] (n4) -- node[name=n4n6]{} (n6);
\draw[->] (n5) -- (n6);
\draw[->] (n6) -- (nE);
\draw[->] (n3) -- (2,3) -- node[name=n3nE]{} (2,7) -- (nE);
\draw[->] (n6) -- (-2,6) -- (-2,3) -- (n3);

\node[usebox] at (2,0) {def(a), def(b)} edge[dashed] (nS);
\node[usebox] at (2,1) {def(result)} edge[dashed] (n1);
\node[usebox] at (2,2) {def(i)} edge[dashed] (n2);

\node[usebox] at (3,3) {p-use(i), p-use(a), p-use(b)} edge[dashed] (n3n4.center) edge[dashed] (n3nE.center);
\node[usebox] at (-5,4) {p-use(a), p-use(b)} edge[dashed] (n4n5.center) edge[dashed] (n4n6.center);

\node[usebox] at (3,5) {def(result), c-use(i)} edge[dashed] (n5);

\node[usebox] at (3,5) {def(result), c-use(i)} edge[dashed] (n5);
\node[usebox] at (3,6) {def(i), c-use(i)} edge[dashed] (n6);
\node[usebox] at (-1,8) {c-use(result)} edge[dashed] (nE);

\end{tikzpicture}
\end{document}
