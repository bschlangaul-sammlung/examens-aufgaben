\documentclass{lehramt-informatik}

\begin{document}

%%%%%%%%%%%%%%%%%%%%%%%%%%%%%%%%%%%%%%%%%%%%%%%%%%%%%%%%%%%%%%%%%%%%%%%%
% Theorie-Teil
%%%%%%%%%%%%%%%%%%%%%%%%%%%%%%%%%%%%%%%%%%%%%%%%%%%%%%%%%%%%%%%%%%%%%%%%

\chapter{Softwaretechnik}

%-----------------------------------------------------------------------
%
%-----------------------------------------------------------------------

\section{Umfangreiche Software-Systeme\footcite[Seite 6]{sosy:fs:1}}

\subsection{Programmieren im Großen}

\begin{itemize}
\item Auftraggeber $\neq$ Softwareentwickler

\item mehrere Personen, evtl. räumlich (gar weltweit) getrennt

\item u. U. kein Hintergrundwissen zum Anwendungsgebiet vorhanden,
deshalb Aufgabenstellung oft nicht vollständig verstanden

\item Komplexe Aufgabenstellung, evtl. nur anhand zu erwartender
Anwendungsfälle illustriert (nicht alle Szenarien vorhersehbar)

\item Lösung kann sehr umfangreich werden, etwa 50 – 100 KLOC
\footnote{\url{https://www.yumpu.com/de/document/read/30868823/ringvorlesung-vertiefungsrichtungen-in-mechatronik-lehrstuhl-fa-1-4-r-} Seite 5}
\end{itemize}

%-----------------------------------------------------------------------
%
%-----------------------------------------------------------------------

\section{Aufgaben, Bereiche, Ziele\footcite[Seite 8]{sosy:fs:1}}

Software-Lebenszyklus
Vorgehensmodelle
Requirement Engineering
Spezifikation / Objektorientierte Analyse
Entwurf / Objektorientiertes Design
Implementierung
Nachweisverfahren: Test / Reviews / Beweis
Wiederverwendung und Wartung
Ziel:
Entwicklung qualitativ hochwertiger komplexer Software
unter Berücksichtigung von Zeit und Budget
die den wirklichen Bedürfnissen der Kunden entspricht

%%%%%%%%%%%%%%%%%%%%%%%%%%%%%%%%%%%%%%%%%%%%%%%%%%%%%%%%%%%%%%%%%%%%%%%%
% Aufgaben
%%%%%%%%%%%%%%%%%%%%%%%%%%%%%%%%%%%%%%%%%%%%%%%%%%%%%%%%%%%%%%%%%%%%%%%%

\chapter{Aufgaben}

\literatur

\end{document}
