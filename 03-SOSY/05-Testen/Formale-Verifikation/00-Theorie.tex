\documentclass{lehramt-informatik-haupt}
\InformatikPakete{syntax,mathe,wpkalkuel}
\usepackage{array}
\usepackage{multicol}

\def\MatheEnv#1{
  \medskip

  \hspace{2em}#1

  \medskip
}

\def\Mathe#1{
  \MatheEnv{$#1$}
}

\def\MatheEquiv#1{
  \MatheEnv{$\equiv$\hspace{2em}$#1$}
}

\def\Erklaerung#1{
  \medskip
  {\footnotesize#1}
}

\begin{document}

%%%%%%%%%%%%%%%%%%%%%%%%%%%%%%%%%%%%%%%%%%%%%%%%%%%%%%%%%%%%%%%%%%%%%%%%
% Theorie-Teil
%%%%%%%%%%%%%%%%%%%%%%%%%%%%%%%%%%%%%%%%%%%%%%%%%%%%%%%%%%%%%%%%%%%%%%%%

\chapter{Formale Verifikation}

%-----------------------------------------------------------------------
%
%-----------------------------------------------------------------------

\section{Programmverifikation, Korrektheit\footcite[Seite 14]{sosy:fs:5}}

Verifikation dient dazu, die Korrektheit eines Programms
\memph{mathematisch} zu beweisen.

Ein Algorithmus ist dabei

\begin{itemize}

\item \memph{partiell korrekt}, wenn er bei gültiger Eingabe
(Vorbedingung) immer ein gültiges Ergebnis (Nachbedingung) liefert.

\item \memph{total korrekt}, wenn er partiell korrekt ist und terminiert.
\end{itemize}

%-----------------------------------------------------------------------
%
%-----------------------------------------------------------------------

\section{\text{wp}-Kalkül zur Verfikation\footcite[Seite 15]{sosy:fs:5}}

Der \text{wp}-Kalkül ist ein Kalkül in der Informatik zur Verifikation
eines imperativen Programmcodes. Die Abkürzung \text{wp} steht für
\memph{weakest precondition}, auf deutsch schwächste Vorbedingung. Bei
der Verifikation geht es nicht darum, die Funktion mit einer bestimmten
Menge an Eingabedaten auf korrekte Ergebnisse zu testen, sondern darum,
eine \memph{allgemeingültige Aussage über das korrekte Ablaufen} des
Programms zu erhalten.

Die Überprüfung der Korrektheit geschieht durch \memph{Rückwärtsanalyse}
des Programmcodes. Ausgehend von der Nachbedingung wird geprüft, ob
diese durch die Vorbedingung und den Programmcode garantiert wird.

Alternativ kann auch der \memph{Hoare-Kalkül} benutzt werden, bei dem im
Gegensatz zum \text{wp}-Kalkül eine \memph{Vorwärtsanalyse} stattfindet.

Der \text{wp}-Kalkül hilft gewisse Zusicherungen im Programm zu machen.
Eine Zusicherung ist eine prädikatenlogische Aussage über den Inhalt der
Variablen an der bestimmten Stelle. Eine Zusicherung vor einem
Programmtext heißt \memph{Vorbedingung P}, eine Zusicherung danach
\memph{Nachbedingung Q}.\footcite{wiki:wp-kalkuel}

%-----------------------------------------------------------------------
%
%-----------------------------------------------------------------------

\section{Beispiel}

$\wp{x += 5; y *= 2; z = z \% 4; y--;}{x + y = 17}$

$\wp{x += 5; y *= 2; z = z \% 4;}{x + y - 1 = 17}$

$\wp{x += 5; y *= 2; z = z \% 4;}{x + y = 18}$

$\wp{x += 5; y *= 2;}{x + y = 18}$

$\wp{x += 5;}{x + y \cdot 2 = 18}$

$x + 5 + y \cdot 2 = 18$

$x + y \cdot 2 = 13$\footcite[Seite 21-26]{sosy:fs:5}

\section{Verzweigung}

$\text{\text{wp}}(\text{„}\texttt{IF B THEN <}A_1\texttt{> ELSE <}A_2\texttt{>}\text{“}, Q)
\equiv
(B \land \text{\text{wp}}(\texttt{<}A_1\texttt{>}, Q))
\lor
(\neg B \land \text{\text{wp}}(\texttt{<}A_2\texttt{>}, Q))$

\section{Transformationen}

\begin{enumerate}
\item $\wp{}{Q} = Q$
Nichts passiert, die Vorbedingung bleibt gleich

\item $\wp{Fehler}{Q} = \text{falsch}$
Fehler dürfen nicht auftreten

\item $\wp{A}{Q} \land \wp{A}{R} = \wp{A}{Q \land R}$
Distributivität der Konjunktion

\item $\text{\text{wp}}(A, Q) \lor \text{\text{wp}}(A, R) = \text{\text{wp}}(A, Q \lor R)$
Distributivität der Disjunktion
\end{enumerate}

%-----------------------------------------------------------------------
%
%-----------------------------------------------------------------------

\section{Schleifen}

Die Behandlung von Schleifen ist etwas schwieriger als die von anderen
Konstrukten, da die Variablen innerhalb jedes einzelnen
Schleifendurchgangs verändert werden. Daher ist es nicht einfach möglich
eine starre Ersetzung vorzunehmen. Anstattdessen verwendet man eine Art
Vollständige Induktion um die Funktion der Schleife nachzuweisen.

Um die schwächste Vorbedingung eines Ausdrucks der Form „\java{while b
{ A }}“ zu finden verwendet man eine \memph{Schleifeninvariante}. Sie
ist ein Prädikat für das

\begin{displaymath}
\{ I \wedge b \} A \{ I \}
\end{displaymath}

\noindent
gilt. Die Schleifeninvariante gilt also sowohl vor, während und nach der
Schleife. Das Schema einer Schleife sieht dann wie folgt aus:

\begin{minted}{java}
// { I } - Invariante gilt vor der Schleife
while (b) {
  // { I && b} - Invariante gilt vor dem Schleifenkörper
  A;
  // { I } - Invariante gilt nach dem Schleifenkörper
}
// { I ∧ (¬b) }
\end{minted}

Nun gilt es nur noch folgende Schritte zu beweisen:

\begin{enumerate}
\item Die Invariante gilt vor Schleifeneintritt

\item $\{ I \wedge b \} A \{ I \}$, dass also $I$ wirklich eine
Invariante ist

\item $(I \wedge \neg b) \Rightarrow Q$, dass also bei der Terminierung
auch die Nachbedingung aus der Invariante folgt. Dass die Schleife
terminiert (mittels Schleifenvariante/Terminierungsfunktion)
\end{enumerate}

\subsection{Beispiel}

Dazu ein Beispiel, das die Fakultät einer Variable $x$ ausrechnet und in
der Variable $fak$ ausgibt

\begin{minted}{java}
int berechneFakultaet(int x) {
  int i = 1;
  int fak = 1;
  // I: fak = i!
  while (i < x) {
    // I: fak = i! && i < x
    i = i + 1;
    fak = fak * i;
    // I: fak = i!
  }
  // I: fak = i! && i >= x
  return x;
}
\end{minted}

\noindent
Die Schleifeninvariante ist hier $(\text{fak} = i!)$. Der Ausdruck $(x -
i)$ fällt streng monoton während der Schleifenausführung gegen $0$ und
ist die Abbruchbedingung.

\subsubsection{Am Beispiel von $x = 3$}

\begin{multicols}{2}
\ueberschrift{1. Durchgang}

\begin{minted}{java}
int i = 1;
int fak = 1;
// I: fak = i! -> 1 = 1
while (i < x) {
  // I: fak = i! && i < x -> 1 = 1 && 1 < 3
  i = i + 1;
  // i = 2
  fak = fak * i;
  // I: fak = i! -> 2 = 2
}
\end{minted}

\ueberschrift{2. Durchgang}

\begin{minted}{java}
while (i < x) {
  // I: fak = i! && i < x -> 2 = 2 && 2 < 3
  i = i + 1;
  // i = 3
  fak = fak * i;
  // I: fak = i! -> 6 = 3! -> 6 = 3
}
// I: fak = i! && i >= x -> 6 = 6 && 3 >= 3
\end{minted}

\end{multicols}

%%%%%%%%%%%%%%%%%%%%%%%%%%%%%%%%%%%%%%%%%%%%%%%%%%%%%%%%%%%%%%%%%%%%%%%%
% Aufgaben
%%%%%%%%%%%%%%%%%%%%%%%%%%%%%%%%%%%%%%%%%%%%%%%%%%%%%%%%%%%%%%%%%%%%%%%%

\chapter{Aufgaben}

\section{Aufgabe 1: Grundwissen\footcite[Seite 1]{sosy:ab:8}}

\begin{enumerate}

%%
% (a)
%%

\item Geben Sie zwei verschiedene Möglichkeiten der formalen
Verifikation an.

\begin{antwort}
\begin{itemize}
\item 1. Möglichkeit: formale Verifikation mittels \emph{vollständiger
Induktion} (eignet sich bei \emph{rekursiven} Programmen).

\item 2. Möglichkeit: formale Verifikation mittels \emph{\text{wp}-Kalkül} oder
\emph{Hoare-Kalkül} (eignet sich bei \emph{iterativen} Programmen).
\end{itemize}
\end{antwort}

%%
% (b)
%%

\item Erläutern Sie den Unterschied von partieller und totaler
Korrektheit.

\begin{antwort}
\begin{itemize}
\item partielle Korrektheit:

Das Programm verhält sich spezifikationsgemäß, \emph{falls} es
terminiert.

\item totale Korrektheit:

Das Programm verhält sich spezifikationsgemäß und es \emph{terminiert
immer}.
\end{itemize}
\end{antwort}

%%
% (c)
%%

\item Gegeben sei die Anweisungssequenz $A$. Sei $P$ die Vorbedingung
und $Q$ die Nachbedingung dieser Sequenz. Erläutern Sie, wie man die
(partielle) Korrektheit dieses Programmes nachweisen kann.

\begin{antwort}
\begin{tabular}{p{4cm}ll}
Vorgehen & Horare-Kalkül & \text{wp}-Kalkül \\

Wenn die Vorbedingung $P$ zutrifft, gilt nach der Ausführung der
Anweisungssequenz $A$ die Nachbedingung $Q$. &

$\{P\}A\{Q\}$ &

$P \Rightarrow \text{\text{wp}}(A,Q)$

\end{tabular}
\end{antwort}

%%
% (d)
%%

\item Gegeben sei nun folgendes Programm:

\begin{minted}{python}
A_1
while(b):
    A_2
A_3
\end{minted}

wobei $A_1$, $A_2$, $A_3$ Anweisungssequenzen sind. Sei $P$ die
Vorbedingung und $Q$ die Nachbedingung des Programms. Die
Schleifeninvariante der while-Schleife wird mit $I$ bezeichnet.
Erläutern Sie, wie man die (partielle) Korrektheit dieses Programmes
nachweisen kann.

\begin{antwort}
\begin{tabular}{>{\raggedright\arraybackslash}p{4cm}ll}
Vorgehen & Horare-Kalkül & \text{wp}-Kalkül \\\hline\hline

Die Invariante $I$ gilt vor Schleifeneintritt. &
$\{P\} A_1 \{I\}$ &
$P \Rightarrow \text{\text{wp}}(A_1,I)$\\\hline

$I$ ist invariant, d. h. $I$ gilt nach jedem Schleifendurchlauf.&
$\{I \land b\} A_2 \{I\}$ &
$I \land b \Rightarrow \text{\text{wp}}(A_2, I)$\\\hline

Die Nachbedingung $Q$ wird erfüllt.&
$\{I \land \neg b\} A_3 \{Q\}$ &
$I \land \neg b \Rightarrow \text{\text{wp}}(A_3, I)$\\
\end{tabular}
\end{antwort}

%%
% (e)
%%

\item Beschreiben Sie, welche Vorraussetzungen eine
Terminierungsfunktion erfüllen muss, damit die totale Korrektheit
gezeigt werden kann.

%-----------------------------------------------------------------------
%
%-----------------------------------------------------------------------

\section{Aufgabe 2: \text{wp}-Kalkül\footcite{sosy:ab:8}}

Gegeben sei folgendes Programm:

\begin{minted}{java}
int f(int x, int y) {
  /* P */ x = 2 * x + 1 + x * x;
  y += 7;
  if (x > 196) {
    y = 2 * y;
  } else {
    y -= 8;
    x *= 2;
  } /* Q */
  return x + y;
}
\end{minted}

Bestimmen Sie die schwächste Vorbedingung (weakest precondition), für
die die Nachbedingung $Q := (x \geq 8) \land (y \% 2 = 1)$ noch
zutrifft.
\end{enumerate}

\begin{antwort}
\setlength{\parindent}{0pt}

\def\code#1{„\texttt{#1}“}
\def\FarbeA#1{\textcolor{orange}{#1}}
\def\FarbeB#1{\textcolor{red}{#1}}
\def\FarbeC#1{\textcolor{blue}{#1}}

\def\NebenRechnung#1{\medskip\bigskip\par\textbf{Nebenrechnung: #1}\par\medskip}

\def\AnweisungI{x=2*x+1+x*x;}
\def\AnweisungII{y+=7;}
\def\AnweisungIII{y=2*y;}
\def\AnweisungIV{y-=8;}
\def\AnweisungV{x*=2;}

\def\A{\FarbeA{\AnweisungI\AnweisungII}}
\def\B{\FarbeB{\AnweisungIII}}
\def\C{\FarbeC{\AnweisungIV\AnweisungV}}
\def\Q{(x \geq 8) \land (y \% 2 = 1)}

Mit dem Distributivgesetz der Konjugation gilt: \bigskip

$\wp{A; if(b) B; else C;}{Q} \equiv$

$\wp{A;}{b} \land \wp{A;B;}{Q}$

$\lor$%

$\wp{A;}{\neg b} \land \wp{A;C;}{Q}$

\bigskip Der tatsächliche Programmcode wird eingesetzt:\bigskip

$\wp{\A{}if(x>196)\{\B\}else\{\C\}}{\Q} \equiv$

% A if(b)
$\wp{\A}{x > 196} \land$

% A;B; Q
$\wp{\A\B}{\Q}$

%
$\lor$

% A if(!b)
$\wp{\A}{x \leq 196} \land$

% A;C; Q
$\wp{\A\C}{\Q}$

=: P

%%
%
%%

\NebenRechnung{$\wp{A;}{b}$}

\Mathe{
  \wp{\A}{x > 196}
}

\Erklaerung{Wir lassen $y+=7$ weg, weil in der Nachbedingung kein y
vorkommt und setzen in den Term $x > 196$ für das $x$ die erste
Code-Zeile $2 \cdot x + 1 + x \cdot x$ ein.}

\MatheEquiv{
  \wp{}{2 \cdot x + 1 + x \cdot x > 196}
}

\Erklaerung{Nach der Transmformationsregel \textit{Nichts passiert, die
Vorbedingung bleibt gleich} kann das auch so geschrieben werden:}

\MatheEquiv{
  2 \cdot x + 1 + x \cdot x > 196
}

\Erklaerung{Die erste binomische Formel (Plus-Formel) lautet
$(a + b)^2 = a^2 + 2ab + b^2$.
Man kann die Formel auch umgedreht verwenden:
$a^2 + 2ab + b^2 = (a + b)^2$.
Die erste Code-Zeile $2 \cdot x + 1 + x \cdot x$
kann umformuliert werden in
$
1 + 2 \cdot 1 \cdot x + x \cdot x =
1^2 + 2 \cdot 1 \cdot x + x^2 =
(1 + x)^2 =
(x + 1)^2
$.
Wir haben für $a$ die Zahl $1$ und für $b$ den Buchstaben $x$
eingesetzt.}

\MatheEquiv{
  (x + 1)^2 > 196
}

%%
%
%%

\NebenRechnung{$\wp{A;B;}{Q}$}

\Mathe{
  \wp{\A\B}{\Q}
}

\Erklaerung{Für das $x$ in der Nachbedingung setzen wir die erste
Code-Zeile $2 \cdot x + 1 + x \cdot x$ ein.
%
Für das $y$ in der Nachbedingung setzen wir dritte Code-Zeile
\texttt{\AnweisungIII} ein und dann die zweite Code-Zeile
\texttt{\AnweisungII}. Das wp-Kalkül arbeitet den Code rückswärts ab.
%
in $y \% 2$ die dritte Anweisung $y = 2 \cdot y$ einfügen: $2 \cdot y \%
2$
%
dann in $2 \cdot y \% 2$ die zweite Anweisung $y = y + 7$ einfügen: $2
\cdot (y + 7) \% 2$}

\MatheEquiv{
  (x + 1)^2 \geq 8 \land 2(y + 7)\%2 = 1
}

\Erklaerung{Diese Aussage ist falsch, da $2(y + 7)$ immer eine gerade
Zahl ergibt und der Rest von einer Division durch zwei einer geraden
Zahl immer 0 ist und nicht 1.}

\MatheEquiv{
  (x + 1)^2 \geq 8 \land \text{falsch}
}

\MatheEquiv{
  \text{falsch}
}

%%
%
%%

\NebenRechnung{$\wp{A;}{\neg b}$}

\Mathe{
  \wp{\A}{x \leq 196}
}

\Erklaerung{Analog zu Nebenrechnung 1}

\MatheEquiv{
  (x + 1)^2 \leq 196
}

%%
%
%%

\NebenRechnung{$\wp{A;C;}{Q}$}

\Mathe{
  \wp{\A\C}{\Q}
}

\Erklaerung{\code{x*=2;}: $x \cdot 2$ für $x$ einsetzen:}

\MatheEquiv{
  \wp{x=2*x+1+x*x;y+=7;y-=8;}
  {(2 \cdot x \geq 8) \land (y \% 2 = 1)}
}

\Erklaerung{\code{y-=8;}: $y - 8$ für $y$ einsetzen:}

\MatheEquiv{
  \wp{x=2*x+1+x*x;y+=7;}
  {(2 \cdot x \geq 8) \land ((y - 8) \% 2 = 1)}
}

\Erklaerung{\code{y+=7}: $y + 7$ für $y$ einsetzen:}

\MatheEquiv{
  \wp{x=2*x+1+x*x;}
  {(2 \cdot x \geq 8) \land (((y + 7) - 8) \% 2 = 1)}
}

\Erklaerung{\code{x=2*x+1+x*x;}: $(x + 1)^2$ für $x$ einsetzen:}

\MatheEquiv{
\wp{}
  {(2 \cdot (x + 1)^2 \geq 8) \land (((y + 7) - 8) \% 2 = 1)}
}

\Erklaerung{Nur noch die Nachbedingung stehen lassen:}

\MatheEquiv{
  (2 \cdot (x + 1)^2 \geq 8) \land ((\textcolor{red}{(y + 7) - 8}) \% 2 = 1)
}

\Erklaerung{Subtraktion:}

\MatheEquiv{
  (2 \cdot (x + 1)^2 \geq 8) \land ((\textcolor{red}{y - 1}) \% 2 = 1)
}

\Erklaerung{Vereinfachen (links beide Seiten durch 2 teilen und rechts
von beiden Seiten $1$ abziehen)}

\MatheEquiv{
  (\frac{2 \cdot (x + 1)^2}{2} \geq \frac{8}{2}) \land (((y - 1) \% 2) - 1 = 1 - 1)
}

\Erklaerung{Zwischenergebnis:}

\MatheEquiv{
((x + 1)^2 \geq 4) \land y \% 2 = 0
}

\vspace{1cm}

\ueberschrift{Zusammenführung}

\Erklaerung{Die Zwischenergebnisse aus den Nebenrechnungen
zusammenfügen:}

\MatheEquiv{
  [
    (x + 1)^2 > 196 \land
    \text{falsch}
  ]
  \lor
  [
    (x + 1)^2 \leq 196 \land
    (x + 1)^2 \geq 4 \land y\%2 = 0
  ]
}

\Erklaerung{„falsch“ und eine Aussage verbunden mit logischem Und
„$\land$“ ist insgesamt falsch:}

\MatheEquiv{
  \text{falsch}
  \lor
  [
    (x + 1)^2 \leq 196 \land
    (x + 1)^2 \geq 4 \land y\%2 = 0
  ]
}

\Erklaerung{falsch verbunden mit oder weglassen:}

\MatheEquiv{
  (x + 1)^2 \leq 196
  \land
  (x + 1)^2 \geq 4 \land y\%2 = 0
}

\Erklaerung{Umgruppieren, sodass nur noch ein $(x + 1)^2$ geschrieben
werden muss:}

\MatheEquiv{
  4 \leq (x + 1)^2 \leq 196
  \land
  y\%2 = 0
}

$4 = 2^2$ und $196 = 14^2$

\MatheEquiv{
  2^2 \leq (x + 1)^2 \leq 14^2
  \land
  y\%2 = 0
}

\Erklaerung{Hoch zwei weg lassen: Betragsklammer $|x|$ oder auch
Betragsfunktion hinzufügen (Die Betragsfunktion ist festgelegt als
„Abstand einer Zahl von der Zahl Null“.}

\MatheEquiv{
  2 \leq | x + 1 | \leq 14 \land y\%2 = 0
}

\Erklaerung{Auf die Gleichung der linken Aussage $-1$ anwenden:}

\MatheEquiv{
  1 \leq |x| \leq 13 \land y\%2 = 0
}

\Erklaerung{Die Betragsklammer weg lassen:}

\MatheEquiv{
  (1 \leq x \leq 13 \lor -13 \leq x \leq -1) \land y\%2 = 0
}

\Mathe{=: P}

\end{antwort}

%-----------------------------------------------------------------------
%
%-----------------------------------------------------------------------

\ExamensAufgabeTTA 66116 / 2015 / 09 : Thema 2 Teilaufgabe 2 Aufgabe 3

%-----------------------------------------------------------------------
%
%-----------------------------------------------------------------------

\ExamensAufgabeTTA 66116 / 2017 / 03 : Thema 2 Teilaufgabe 2 Aufgabe 4

%-----------------------------------------------------------------------
%
%-----------------------------------------------------------------------

\ExamensAufgabeTTA 46116 / 2014 / 03 : Thema 2 Teilaufgabe 1 Aufgabe 1

\literatur

\end{document}
