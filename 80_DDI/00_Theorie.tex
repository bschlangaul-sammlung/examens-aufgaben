\documentclass{lehramt-informatik-haupt}

\begin{document}

%%%%%%%%%%%%%%%%%%%%%%%%%%%%%%%%%%%%%%%%%%%%%%%%%%%%%%%%%%%%%%%%%%%%%%%%
% Theorie-Teil
%%%%%%%%%%%%%%%%%%%%%%%%%%%%%%%%%%%%%%%%%%%%%%%%%%%%%%%%%%%%%%%%%%%%%%%%

\chapter{Didaktik}

\section{Allgemeinbildung nach Bussmann/Heymann [Heymann.1997]}

%%
%
%%

\subsection{1. Lebensvorbereitung}

„ .. so muß im Hinblick auf jedes Schulfach gefragt werden:

Inwieweit fallen in seinen Bereich Kenntnisse, Fähigkeiten und
Fertigkeiten, die im pragmatischen Sinne zu Bewältigung alltäglicher
Lebenssituationen beitragen und die ohne Schule nicht oder nicht in
hinreichendem Maße gelernt würden?“\footcite[Seite 5]{ddi:fs:1}

%%
%
%%

\subsection{2. Stiftung kultureller Kohärenz}

„Im Blick auf den Beitrag zur Allgemeinbildung ist unter dem soeben
erläuterten Aspekt für jedes Schulfach zu fragen:

Welche für unseren Kulturkreis, unser kulturelles und gesellschaftliches
Selbstverständnis zentralen kulturellen Errungenschaften werden in dem
betreffenden Fach tradiert?

Welche Bezüge zwischen der für das betreffende Fach (bzw. die
korrespondierenden Wissenschaften) charakteristischen Fachkultur und der
Gesamtkultur (bzw. anderen Subkulturen) gibt es?“\footcite[Seite 6]{ddi:fs:1}

%%
%
%%

\subsection{3. Weltorientierung}

„Deshalb ist mit Blick auf jedes Schulfach zu fragen:

Was von dem unüberschaubaren „Gebirge“ dessen, was man innerhalb dieses
Faches prinzipiell wissen könnte, ist so fundamental, so erhellend, so
beispielhaft, daß es dem Einzelnen helfen kann, eine Gesamtorientierung
zu finden, ein eigenes tragfähiges Weltbild aufzubauen?

Wie sind die Wissensinhalte, die die Lernenden sich in diesem Fach
aneignen sollten, untereinander und mit den Inhalten anderer Fächer
vernetzt?

Wo bieten sich Möglichkeiten, die Grenzen des Fachs zu überschreiten und
„Schlüsselprobleme“ zu thematisieren?

Wird das Exemplarische des Unterrichtsstoffs und seine Vernetzung mit
anderen Elementen des Weltwissens im herkömmlichen Fachunterricht
hinreichend deutlich? [..]“\footcite[Seite 7]{ddi:fs:1}

%%
%
%%

\subsection{4. Anleitung zum kritischen Vernunftsgebrauch}

„Jedes Schulfach hat sich somit den Fragen zu stellen:

\begin{itemize}
\item Bieten die üblichen Fachinhalte hinreichend Gelegenheit zum
kritischen Vernunftgebrauch?

\item Bietet die übliche Praxis des Fachunterrichts ein hinreichend
anregendes geistiges Klima, in dem sich das kritische Denken der
Schülerinnen und Schüler kultivieren läßt?

\item Welche Merkmale des traditionellen Fachunterrichts stehen der
Anleitung zum kritischen Vernunftgebrauch möglicherweise im Wege, und
wie könnte dem begegnet werden?“\footcite[Seite 8]{ddi:fs:1}
\end{itemize}

%%
%
%%

\subsection{5. Entfaltung von Verantwortungsbereitschaft}

An jedes Schulfach ist die Frage zu stellen:

\begin{itemize}
\item Bietet der betreffende Fachunterricht, die Art, wie mit den
Sachthemen und miteinander umgegangen wird, einen Rahmen für die
Entfaltung von Verantwortungsbereitschaft?

\item Gibt es Eigentümlichkeiten des Faches bzw. der fachspezifischen
Sozialisation, die diesem Ziel tendenziell im Wege stehen?\footcite[Seite 9]{ddi:fs:1}
\end{itemize}

%%
%
%%

\subsection{6. Einübung in Verständigung und Kooperation}

„Für die Schulfächer ergeben sich aus dem Ernstnehmen dieses Aspekts von
Allgemeinbildung Fragen, die sich vornehmlich auf die Methodik und die
„Unterrichtskultur“ richten:

\begin{itemize}
\item Bietet der übliche Fachunterricht hinreichend Gelegenheiten zur
Einübung in Verständigung und Kooperation?

\item Gibt es fachspezifische Besonderheiten, die die Einübung in
Verständigung und Kooperation eher behindern?“\footcite[Seite 10]{ddi:fs:1}
\end{itemize}

%%
%
%%

\subsection{7. Stärkung des Schüler-Ichs}

„Die Schulfächer haben sich also den Fragen zu stellen:

\begin{itemize}
\item Geben die herkömmlichen Inhalte und Unterrichtsmethoden genügend
Raum für die Förderung des Einzelnen im beschriebenen Sinne?

\item Welche Elemente der üblichen Fachkultur beeinträchtigen die
angestrebte Stärkung des Schüler-Ichs möglicherweise?“\footcite[Seite 11]{ddi:fs:1}
\end{itemize}

%-----------------------------------------------------------------------
%
%-----------------------------------------------------------------------

\section{Fundamentale Ideen}

Das Konzept der fundamentalen Ideen findet sich erstmals bei Bruner
[Bruner.1960].

Nach Schwill (1993) findet sich keine konkrete Definition, jedoch einige
charakterisierende Aussagen. So heißt es,…

\begin{itemize}
\item … dass die Grundlagen eines jeden Faches jedem Menschen in jedem
Alter in irgendeiner Form beigebracht werden können.

\item … dass die basalen Ideen […] ebenso einfach wie durchschlagend
sind.

\item … dass ein Begriff eine ebenso umfassende wie durchgreifende
Anwendbarkeit besitzt.\footcite[Seite 12]{ddi:fs:1}
\end{itemize}

Eine fundamentale Idee (bzgl. einer Wissenschaft) ist ein Denk-,
Handlungs-, Beschreibungs- oder Erklärungsschema, das

\begin{enumerate}

%%
% (1)
%%

\item in verschiedenen Bereichen (der Wissenschaft) vielfältig anwendbar
oder erkennbar ist (Horizontalkriterium),

%%
% (2)
%%

\item auf jedem intellektuellen Niveau aufgezeigt und vermittelt werden
kann (Vertikalkriterium),

%%
% (3)
%%

\item in der historischen Entwicklung (der Wissenschaft) deutlich
wahrnehmbar ist und längerfristig relevant bleibt (Zeitkriterium),

%%
% (4)
%%

\item eine Bezug zu Sprache und Denken des Alltags und der Lebenswelt
besitzt (Sinnkriterium).

%%
% (5)
%%

\item Das zur Annäherung an eine gewisse idealisierte Zielvorstellung
dient, die jedoch faktisch möglicherweise unerreichbar ist
(Zielkriterium)\footcite[Seite 13]{ddi:fs:1}

\end{enumerate}
\literatur

\end{document}
