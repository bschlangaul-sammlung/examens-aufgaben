\documentclass{lehramt-informatik}
\usepackage{fancyvrb}
\usepackage{soul}
\usepackage{multicol}
\def\TmpUeber#1{{\setul{-0.9em}{}\ul{#1}}}

\InformatikPakete{mathe,syntax}

\begin{document}

%%%%%%%%%%%%%%%%%%%%%%%%%%%%%%%%%%%%%%%%%%%%%%%%%%%%%%%%%%%%%%%%%%%%%%%%
% Theorie-Teil
%%%%%%%%%%%%%%%%%%%%%%%%%%%%%%%%%%%%%%%%%%%%%%%%%%%%%%%%%%%%%%%%%%%%%%%%

\chapter{Relationale Algebra}

\begin{quellen}
\item \cite{net:html:uni-innsbruck:relax}
\end{quellen}

\section{Symbole}
% https://dbai.tuwien.ac.at/education/dm/resources/symbols.html

\begin{tabular}{l|l|l|l}
\textbf{Name} & \textbf{Symbol} & \textbf{LaTeX} & \textbf{Alternativtext}\\\hline\hline
% σ
Selektion & $\sigma$ & \verb|\sigma| & SEL\\
% π
Projektion & $\pi$ & \verb|\pi| & PR\\
% ∪
Vereinigung & $\cup$ & \verb|\cup| & UNION\\
% ∩
Durchschnitt & $\cap$ & \verb|\cap| & INTERSECTION\\
% −
Mengendifferenz & $-$ & - & -\\
% ✕
kartesisches Produkt & $\times$ & \verb|\times| & X\\
% ρ ←
Umbenennung (+Zuweisung) & $\rho \leftarrow$ & \verb|\rho \leftarrow| & RENAME\\
% ÷
Division & $\div$ & \verb|\div| & DIV\\

% △
Symmetrische Differenz & $\bigtriangleup$ & \verb|\bigtriangleup| & \\

\hline

% ⋈
Join & $\bowtie$ & \verb|\bowtie| & JOIN\\

% ⟕
%Left Outer Join & {\tiny \textifsym{d|><|}} & \verb|{\tiny \textifsym{d|><|}}| & LOJOIN\\

% Right Outer Join & ⟖ & {\tiny \textifsym{|><|d}} & ROJOIN\\

% Full Outer Join & ⟗ & {\tiny \textifsym{d|><|d}} & FOJOIN\\

% ⋉ needs \usepackage{amssymb}
Left Semi Join & $\ltimes$ & \verb|\ltimes| & LSJOIN\\

% ⋊ needs \usepackage{amssymb}
Right Semi Join & $\rtimes$ & \verb|\rtimes| & RSJOIN\\

\hline

% ∧
Und & $\land$ & \verb|\land| & AND\\

% ∨
Oder & $\lor$ & \verb|\lor| & OR\\

% ¬
Negation & $\neg$ & \verb|\neg| & -\\

% ≥
Größer-gleich & $\geq$ & \verb|\geq| & >=\\

% ≤
Kleiner-gleich & $\leq$ & \verb|\leq| & <=\\

% ≠
Ungleich & $\neq$ & \verb|\neq| & =/=\\

% ≡
Äquivalenz & $\equiv$ & \verb|\equiv| & EQ\\

% ∃
Existenzquantor & $\exists$ & \verb|\exists| & EXISTS\\

% ∀
All-Quantor & $\forall$ & \verb|\forall| & FORALL\\
\end{tabular}

%-----------------------------------------------------------------------
%
%-----------------------------------------------------------------------

\section{Operationen der Relationen Algebra}

\subsection{Mengenoperation}

%%
%
%%

\subsubsection{Vereinigung}

\begin{description}
\item[Symbol-Schreibweise] $R \cup S$
\item[SQL] UNION
\end{description}

%%
%
%%

\subsubsection{Mengendifferenz}

\begin{description}
\item[Symbol-Schreibweise] $R - S$
\item[SQL] EXCEPT
\end{description}

%%
%
%%

\subsubsection{Mengendurchschnitt (Schnittmenge/Intersection)}

\begin{description}
\item[Symbol-Schreibweise] $R \cap S$
\item[SQL] INTERSECT
\end{description}

%%
%
%%

\subsubsection{Symmetrische Differenz}

\begin{description}
\item[Symbol-Schreibweise] $R \bigtriangleup S$
\item[SQL] INTERSECT
\end{description}

%-----------------------------------------------------------------------
%
%-----------------------------------------------------------------------

\subsection{Selektion}

\begin{description}
\item[Symbol-Schreibweise] $\sigma_{Ausdruck}(R)$
\item[lineare Schreibweise] $R[Ausdruck]$
\item[SQL] WHERE
\end{description}

%-----------------------------------------------------------------------
%
%-----------------------------------------------------------------------

\subsection{Projektion}

\begin{description}
\item[Symbol-Schreibweise] $\pi_{\beta}(R)$
\item[lineare Schreibweise] $R[\beta]$
\item[SQL] SELECT
\end{description}

%-----------------------------------------------------------------------
%
%-----------------------------------------------------------------------

\subsection{Kartesisches Produkt (Kreuzprodukt)}

\begin{description}
\item[Symbol-Schreibweise] $R \times S$
\item[lineare Schreibweise] $R~x~S$
\item[SQL] CROSS JOIN
\end{description}

%-----------------------------------------------------------------------
%
%-----------------------------------------------------------------------

\subsection{Umbenennung}

\begin{description}
\item[Symbol-Schreibweise] $\rho_{[{neu}\leftarrow alt]} (R)$
\item[lineare Schreibweise] $R[alt\rightarrow neu]$
\end{description}

%-----------------------------------------------------------------------
%
%-----------------------------------------------------------------------

\subsection{Division\footcite[Division]{relationale-algebra}}

\begin{description}
\item[Symbol-Schreibweise] $R \div S$
\end{description}

Die Division ist dann definiert durch:

$R\div S:=\pi_{{R'}}(R)-\pi_{{R'}}((\pi_{{R'}}(R)\times S)-R)$

\begin{minted}{sql}
SELECT distinct MatrNr
FROM hoert
WHERE MatrNr NOT IN(
  SELECT R.MatrNr
  FROM hoert R, Professor P, Vorlesung V
  WHERE P.Name = 'Sokrates'
  AND P.PersNr=V.gelesenVon
  AND (R.MatrNr, V.VorlNr) NOT IN (
    SELECT MatrNr, VorlNr
    FROM hoert
  )
);
\end{minted}

%-----------------------------------------------------------------------
%
%-----------------------------------------------------------------------

\section{Die 5 Grundoperationen der Relationalen Algebra
\footcite{net:pdf:lmu:dbs}
}

Mit diesen Grundoperationen lassen sich weitere Operationen (z. B. die
Schnittmenge) nachbilden.

\begin{itemize}
\item Verenigung $R = S \cup T$
\item Differenz $R = S - T$
\item Kartesisches Prokukt (Kreuzprodukt) $R = S \times T$
\item Selection $R = \sigma_F(S)$
\item Projektion $R = \pi_{A,B,...}(S)$
\end{itemize}

%%%%%%%%%%%%%%%%%%%%%%%%%%%%%%%%%%%%%%%%%%%%%%%%%%%%%%%%%%%%%%%%%%%%%%%%
% Aufgaben
%%%%%%%%%%%%%%%%%%%%%%%%%%%%%%%%%%%%%%%%%%%%%%%%%%%%%%%%%%%%%%%%%%%%%%%%

\chapter{Aufgaben}

%-----------------------------------------------------------------------
%
%-----------------------------------------------------------------------

\section{Staatsexamen Herbst 2018 - DB/ST (RS) - Thema 2. TA II. A2
\footcite[Herbst 2018 - DB/ST (RS) - Thema 2. TA II. A2]{examen:46116:2018:09}}

Geben Sie die Ergebnisrelation folgender Ausdrücke der relationalen
Algebra als Tabellen an. Begründen Sie Ihr Ergebnis, gegebenenfalls
durch Zwischenschritte. Gegeben seien folgende Relationen:
\footcite[Seite 1, Aufgabe 1: Herbst 2018 - DB/ST (RS) - Thema 2. TA II. A2]{db:ab:3}

\bigskip

\begin{minipage}[t]{5cm}
\subsection*{R}
\begin{tabular}{llllll}
A & B & C & D & E & F \\\hline
6 & 8 & 1 & 7 & 3 & 7 \\
5 & 3 & 4 & 4 & 5 & 7 \\
0 & 6 & 3 & 0 & 1 & 7
\end{tabular}
\end{minipage}
%
\begin{minipage}[t]{3.8cm}
\subsection*{S}
\begin{tabular}{llll}
A & C & X & Z \\\hline
7 & 8 & 6 & 1 \\
0 & 3 & 0 & 0 \\
2 & 3 & 0 & 5 \\
0 & 6 & 1 & 6 \\
6 & 7 & 1 & 7 \\
7 & 1 & 2 & 2 \\
1 & 8 & 8 & 0 \\
5 & 1 & 5 & 5 \\
7 & 3 & 0 & 2 \\
4 & 8 & 2 & 7 \\
\end{tabular}
\end{minipage}
%
\begin{minipage}[t]{2cm}
\subsection*{T}
\begin{tabular}{ll}
X & Y \\\hline
5 & 3 \\
0 & 5 \\
8 & 6 \\
3 & 6 \\
5 & 7 \\
2 & 8 \\
\end{tabular}
\end{minipage}

\begin{enumerate}

%%
% (a)
%%

\item $\sigma_{A>6}(S) \bowtie_{S.X=T.Y} \pi_Y(T)$

\begin{antwort}
\begin{tabular}{lllll}
A & C & X & Z & Y \\\hline
7 & 8 & 6 & 1 & 6 \\
\end{tabular}
\end{antwort}

%%
% (b)
%%

\item $\pi_{A,C}(S) - (\pi_A(R) \times \pi_C(\sigma_{x=1}(S)))$

\begin{antwort}
\begin{minipage}[t]{4cm}
$\sigma_{x=1}(S)$:

\bigskip
\begin{tabular}{llll}
A & C & X & Z \\\hline
0 & 6 & 1 & 6 \\
6 & 7 & 1 & 7 \\
\end{tabular}
\end{minipage}
%
\begin{minipage}[t]{3cm}
$\pi_C(\sigma_{x=1}(S))$:

\bigskip
\begin{tabular}{l}
C \\\hline
6 \\
7 \\
\end{tabular}
\end{minipage}
%
\begin{minipage}[t]{3cm}
$\pi_A(R)$:

\bigskip
\begin{tabular}{l}
A \\\hline
6 \\
5 \\
0 \\
\end{tabular}
\end{minipage}

\begin{minipage}[t]{5cm}
$(\pi_A(R) \times \pi_C(\sigma_{x=1}(S)))$

\bigskip
\begin{tabular}{ll}
A & C \\\hline
6 & 6 \\
5 & 6 \\
0 & 6 \\
6 & 7 \\
5 & 7 \\
0 & 7 \\
\end{tabular}
\end{minipage}
%
\begin{minipage}[t]{4cm}
$\pi_{A,C}(S)$

\bigskip
\begin{tabular}{llll}
A & C \\\hline
7 & 8 \\
0 & 3 \\
2 & 3 \\
0 & 6 \\
6 & 7 \\
7 & 1 \\
1 & 8 \\
5 & 1 \\
7 & 3 \\
4 & 8 \\
\end{tabular}
\end{minipage}

\begin{tabular}{llll}
A & C \\\hline
7 & 8 \\
0 & 3 \\
2 & 3 \\
7 & 1 \\
1 & 8 \\
5 & 1 \\
7 & 3 \\
4 & 8 \\
\end{tabular}
\end{antwort}

%%
% (c)
%%

\item $(\pi_D(R) \times \pi_E(R)) \div \pi_E(R)$

\begin{antwort}
\begin{minipage}[t]{3cm}
$\pi_D(R) \times \pi_E(R)$

\bigskip
\begin{tabular}{ll}
A & E \\\hline
7 & 3 \\
4 & 3 \\
0 & 3 \\
7 & 5 \\
4 & 5 \\
0 & 5 \\
7 & 1 \\
4 & 1 \\
0 & 1 \\
\end{tabular}
\end{minipage}
%
\begin{minipage}[t]{2cm}
$\pi_E(R)$

\bigskip
\begin{tabular}{l}
E \\\hline
3 \\
5 \\
1 \\
\end{tabular}
\end{minipage}
%
\begin{minipage}[t]{6cm}
$(\pi_D(R) \times \pi_E(R)) \div \pi_E(R)$

\bigskip
\begin{tabular}{l}
D \\\hline
7 \\
4 \\
0 \\
\end{tabular}
\end{minipage}
\end{antwort}
\end{enumerate}

%-----------------------------------------------------------------------
%
%-----------------------------------------------------------------------

\ExamensAufgabe 46116 / 2015 / 03 : Thema 1 Teilaufgabe 2 Aufgabe 1

%-----------------------------------------------------------------------
%
%-----------------------------------------------------------------------

\section{Bundesliga-Datenbank
\footcite[Aufgabe 4: Relationale Algebra und SQL]{db:ab:7}}

Gegeben sei die folgende Bundesliga-Datenbank, in der die Vereine,
Spiele, Trainer und Spieler mit ihren Einsätzen für die laufende Saison
verwaltet werden:

\begin{itemize}
\item VEREIN (VNAME, ORT, PRÄSIDENT)
\item SPIELE (HEIM, GAST, RESULTAT, ZUSCHAUER, TERMIN, SPIELTAG, H-TRAINER,
G-TRAINER)
\item SPIELER (SPNR, NAME, VORNAME, VEREIN, ALTER, GEHALT, GEB-ORT)
\item TRAINER (TRNR, NAME, VORNAME, VEREIN, ALTER, GEHALT, GEB-ORT)
\item EINSATZ (HEIM, GAST, SPNR, VON, BIS, TORE, KARTE)
\end{itemize}

\begin{enumerate}
\item Zeichnen Sie das „zugehörige“ ER-Modell!

\item Formulieren Sie folgende Anfragen in relationaler Algebra und in
SQL:

\begin{itemize}
\item Welche Spieler haben beim Spiel TSV 1860 München – FC Bayern
München mitgewirkt?

\begin{antwort}
$\pi_{\text{NAME,VORNAME}}(
  \text{Spieler}
  \bowtie
  (\sigma_{
    \text{HEIM} = \mlq \text{TSV 1860 München} \mrq \land
    \text{GAST} = \mlq \text{FC Bayern München} \mrq
  }(\text{Einsatz}))
)$

\begin{minted}{sql}
SELECT NAME, VORNAME FROM Spieler, Einsatz,
WHERE
  HEIM = 'TSV 1860 München' AND GAST = 'FC Bayern München' AND
  Einsatz.SPNR = Spieler.SPNR;
\end{minted}
\end{antwort}

%%
%
%%

\item Welche Spiele sind 2 : 0 ausgegangen?

\begin{antwort}
$\pi_{\text{HEIM},\text{GAST},\text{SPIELTAG}}(\sigma_{\text{RESULTAT} = \mlq 2 : 0 \mrq}(\text{SPIELE}))$

\begin{minted}{sql}
SELECT HEIM, GAST, SPIELTAG FROM SPIELE
WHERE RESULTAT = '2 : 0';
\end{minted}
\end{antwort}

%%
%
%%

\item Welche Spieler spielen in einem Verein ihres Geburtsortes?

\begin{antwort}
$\pi_{\text{NAME},\text{VORNAME}}(
  \text{VEREIN}
  \bowtie_{\text{VEREIN.ORT} = \text{SPIELER.GEB-ORT} \land \text{SPIELER.VEREIN} = \text{VEREIN.VNAME}}
  \text{SPIELER}
)$

\begin{minted}{sql}
SELECT NAME, VORNAME
FROM SPIELER, VEREIN
WHERE VEREIN.ORT = SPIELER.GEB-ORT AND SPIELER.VEREIN = VEREIN.VNAME;
\end{minted}
\end{antwort}

%%
%
%%

\item Welche Spieler vom 1. FC Köln haben alle Spiele mitgemacht?

\begin{antwort}
\def\tmpkoeln{\mlq \text{1. FC Köln} \mrq}

\begin{multline*}
\pi_{\text{NAME,VORNAME}}(\\
  \sigma_{\text{VEREIN} = \tmpkoeln}(\text{SPIELER})\\
  \bowtie\\
  (
    \pi_{\text{HEIM,GAST,SPNR}}(\sigma_{\text{HEIM} = \tmpkoeln \lor \text{GAST} = \tmpkoeln}(\text{EINSATZ}))\\
    \div\\
    \pi_{\text{HEIM,GAST}}(\sigma_{\text{HEIM} = \tmpkoeln \lor \text{GAST} = \tmpkoeln}(\text{SPIELE}))
  )
)
\end{multline*}

\begin{minted}{sql}
SELECT NAME, VORNAME
FROM SPIELER
WHERE VEREIN = '1. FC Koeln' AND SPNR IN (
  SELECT SPNR
  FROM EINSATZ
  WHERE HEIM = '1. FC Koeln' OR GAST = '1. FC Koeln'
  GROUP BY SPNR
  HAVING COUNT(*) = (
    SELECT COUNT(*)
    FROM SPIELE
    WHERE HEIM = '1. FC Koeln' OR GAST = '1. FC Koeln'
  )
);
\end{minted}
\end{antwort}

%%
%
%%

\item Wie heißen die Präsidenten der Vereine, die zur Zeit einen Trainer
beschäftigen, der jünger ist als der älteste Spieler, der beim Verein
beschäftigt ist?

\begin{antwort}
\begin{multline*}
\pi_{\text{PRÄSIDENT}}(\\
  \text{VEREIN}\\
  \bowtie\\
  \pi_{\text{VEREIN}}(\\
    \text{TRAINER}\\
    \bowtie_{
      \text{TRAINER.ALTER} < \text{SPIELTER.ALTER} \land
      \text{TRAINER.VEREIN} = \text{SPIELER.VEREIN}
    }\\
    \text{SPIELER}\\
  )\\
)
\end{multline*}

\begin{minted}{sql}
SELECT PRÄSIDENT
FROM VEREIN, SPIELER, TRAINER
WHERE
  VEREIN.VNAME = SPIELER.VEREIN AND
  VEREIN.VNAME = VEREIN.TRAINER AND
  TRAINER.ALTER < SPIELER.ALTER;
\end{minted}
\end{antwort}

%%
%
%%

\item Welche Spieler haben bisher noch nie gespielt?

\begin{antwort}
$\pi_{\text{SPNR}}(\text{SPIELER}) - \pi_{\text{SPNR}}(\text{EINSATZ})$

\begin{minted}{sql}
SELECT SPNR FROM SPIELER
EXCEPT
SELECT SPNR FROM EINSATZ;
\end{minted}
\end{antwort}

%%
%
%%

\item Welche Spieler haben bisher noch kein Tor geschossen?

\begin{antwort}
$\pi_{\text{SPNR}}(\text{SPIELER}) -
\pi_{\text{SPNR}}(\sigma_{\text{TORE} > 0}(\text{EINSATZ}))$

\begin{minted}{sql}
SELECT SPNR FROM SPIELER
EXCEPT
SELECT SPNR FROM EINSATZ WHERE TORE > 0;
\end{minted}
\end{antwort}

%%
%
%%

\item Welcher Trainer hat schon mehr als einen Verein trainiert? Welche
Vereine haben schon mehrere Trainer gehabt?

%%
%
%%

\begin{antwort}
Vereine:

\begin{minted}{sql}
SELECT HEIM FROM SPIELE l, SPIELE r
WHERE l.HEIM = r.HEIM AND NOT (l.H-TRAINER = r.H-TRAINER)
UNION
SELECT GAST FROM SPIELE l, SPIELE r
WHERE l.GAST = r.GAST AND NOT (l.G-TRAINER = r.G-TRAINER)
UNION
SELECT HEIM FROM SPIELE l, SPIELE r
WHERE l.HEIM = r.GAST AND NOT (l.H-TRAINER = r.G-TRAINER)
\end{minted}

Trainer:

\begin{minted}{sql}
SELECT H-TRAINER FROM SPIELE l, SPIELE r
WHERE l.H-TRAINER = r.HEIM AND NOT (l.HEIM = r.HEIM)
UNION
SELECT G-TRAINER FROM SPIELE l, SPIELE r
WHERE l.G-TRAINER = r.G-TRAINER AND NOT (l.GAST = r.GAST)
UNION
SELECT H-TRAINER FROM SPIELE l, SPIELE r
WHERE l.H-TRAINER = r.G-TRAINER AND NOT (l.H-TRAINER = r.G-TRAINER)
\end{minted}
\end{antwort}

%-----------------------------------------------------------------------
%
%-----------------------------------------------------------------------

\item Welche Spiele am 10. Spieltag hatten mehr als 30.000 Zuschauer?

\begin{antwort}
$\pi_{\text{HEIM,GAST}}(
  \sigma_{\text{ZUSCHAUER} > 30000 \land \text{SPIELTAG} = 10}(\text{SPIELE})
)$

\begin{minted}{sql}
SELECT HEIM, GAST
  FROM SPIELE
  WHERE ZUSCHAUER > 30000 AND SPIELTAG = 10;
\end{minted}
\end{antwort}
\end{itemize}
\end{enumerate}

%-----------------------------------------------------------------------
%
%-----------------------------------------------------------------------

\section{2. SQL und relationale Algebra\footcite{examen:66116:2016:09}}

Gegeben sei der folgende Ausschnitt aus dem Schema einer Schulverwaltung:
\footcite{db:ab:examen-gym-2016-09}

\newcommand{\tmpu}[1]{\underline{\texttt{#1}}}

\begin{Verbatim}[commandchars=+*~]
Person : {[
  +tmpu*ID~ : INTEGER,
  Name : VARCHAR(255),
  Wohnort : VARCHAR(255),
  Typ : CHAR(1)
]}

Unterricht : {[
  +tmpu*Klassenbezeichnung~ : VARCHAR(20),
  +tmpu*Schuljahr~ : INTEGER,
  +tmpu*Lehrer~ : INTEGER,
  +tmpu*Fach~ : VARCHAR(100)
]}

Klasse : {[
  +tmpu*Klassenbezeichnung~ : VARCHAR(20),
  +tmpu*Schuljahr~ : INTEGER,
  Klassenlehrer : INTEGER
]}

Klassenverband : {[
  +tmpu*Schüler~ : INTEGER,
  Klassenbezeichnung : VARCHAR(20),
  +tmpu*Schuljahr~ : INTEGER
]}
\end{Verbatim}

\noindent
Hierbei enthält die Tabelle \emph{Person} Informationen über Lehrer (Typ
’L’) und Schüler (Typ ’S’); andere Werte für Typ sind nicht zulässig.
\emph{Klasse} beschreibt die Klassen, die in jedem Schuljahr gebildet
wurden, zusammen mit ihrem Klassenlehrer. In \emph{Unterricht} wird
abgelegt, welcher Lehrer welches Fach in welcher Klasse unterrichtet; es
ist möglich, dass derselbe Lehrer mehr als ein Fach in einer Klasse
unterrichtet. \emph{Klassenverband} beschreibt die Zuordnung der Schüler
zu den Klassen.

\begin{enumerate}

%%
% a)
%%

\item Schreiben Sie eine SQL-Anweisung, die die Tabelle
\emph{Unterricht} mit allen ihren Constraints (einschließlich
Fremdschlüsselconstraints) anlegt.

\begin{antwort}
Ich habe \verb|REFERENCES| bei Unterricht Schuljahr vergessen, die
referenzierten Tabellen \verb|Person| und \verb|Klasse| wurden in der
Musterlösung auch nicht angelegt.

\begin{minted}{sql}
CREATE TABLE Person(
  ID INTEGER PRIMARY KEY,
  Name VARCHAR(255),
  Wohnort VARCHAR(255),
  Typ CHAR(1) CHECK(Typ in ('S', 'L'))
);

CREATE TABLE Klasse(
  Klassenbezeichnung VARCHAR(20),
  Schuljahr INTEGER,
  Klassenlehrer INTEGER REFERENCES Person(ID),
  PRIMARY KEY (Klassenbezeichnung, Schuljahr)
);

CREATE TABLE Unterricht (
  Klassenbezeichnung VARCHAR(20) REFERENCES Klasse(Klassenbezeichnung),
  Schuljahr INTEGER REFERENCES Klasse(Schuljahr),
  Lehrer INTEGER REFERENCES Person(ID),
  Fach VARCHAR(100),
  CONSTRAINT Unterricht_PK
    PRIMARY KEY (Klassenbezeichnung, Schuljahr, Lehrer, Fach)
);
\end{minted}
\end{antwort}

%%
% b)
%%

\item Definieren Sie ein geeignetes Constraint, das sicherstellt, dass
nur zulässige Werte im Attribut Typ der (bereits angelegten) Tabelle
\emph{Person} eingefügt werden können.

\begin{antwort}
Ich habe \verb|REFERENCES| bei Unterricht Schuljahr vergessen, die
referenzierten Tabellen \verb|Person| und \verb|Klasse| wurden in der
Musterlösung auch nicht angelegt.

\begin{minted}{sql}
ALTER TABLE Person
  ADD CONSTRAINT TypLS
    CHECK(Typ IN ('S', 'L'));
\end{minted}
\end{antwort}

%%
% c)
%%

\item Schreiben Sie eine SQL-Anweisung, die die Bezeichnung der Klassen
bestimmt, die im Schuljahr 2015 die meisten Schüler haben.

\begin{antwort}
Falsch: \verb|ORDER BY Anzahl;|. \verb|DESC| vergessen.

\begin{minted}{sql}
SELECT k.Klassenbezeichnung, COUNT(*) AS Anzahl
FROM Klasse k, Klassenverband v
WHERE
  k.Schuljahr = 2015 AND
  k.Klassenbezeichnung = v.Klassenbezeichnung
GROUP BY k.Klassenbezeichnung
ORDER BY COUNT(*) DESC;
\end{minted}
\end{antwort}

%%
% d)
%%

\item Schreiben Sie eine SQL-Anweisung, die die Namen aller Lehrer
bestimmt, die nur Schüler aus ihrem Wohnort unterrichtet haben.

\begin{antwort}
\begin{minted}{sql}
SELECT DISTINCT l.Name
FROM Person l
WHERE NOT EXISTS(
  SELECT DISTINCT *
  FROM Unterricht u, Klassenverband v, Person s
  WHERE
    u.Lehrer = l.ID AND
    u.Klassenbezeichnung = v.Klassenbezeichnung AND
    v.Schüler = s.ID AND
    l.Wohnort != s.Wohnort
);
\end{minted}
\end{antwort}

%%
% e)
%%

\item Schreiben Sie eine SQL-Anweisung, die die Namen aller Schüler
bestimmt, die immer den gleichen Klassenlehrer hatten.

\begin{antwort}
\begin{minted}{sql}
SELECT s.Name
FROM Person s, Klasse k, Klassenverband v
WHERE
  s.ID = v.Schueler AND
  v.Klassenbezeichnung = k.Klassenbezeichnung AND
  v.Schuler = s.ID
GROUP BY k.Klassenlehrer, v.Schueler
HAVING COUNT(*) = 1
\end{minted}
\end{antwort}

%%
% f)
%%

\item Schreiben Sie eine SQL-Anweisung, die alle Paare von Schülern
bestimmt, die mindestens einmal in der gleichen Klasse waren. Es genügt
dabei, wenn Sie die ID der Schüler bestimmen.

\begin{antwort}
Ich habe Schuljahr zum joinen vergessen.
\begin{minted}{sql}
SELECT DISTINCT s1.ID, s2.ID
FROM Klassenverband s1, Klassenverband s2
WHERE
  s1.Schuljahr = s2.Schuljahr AND
  s1.Schueler <> s2.Schueler AND
  s1.Klassenbezeichung = s2. Klassenbezeichnung;
\end{minted}
\end{antwort}

%%
% g)
%%

\item Formulieren Sie eine Anfrage in der relationalen Algebra, die die
ID aller Schüler bestimmt, die mindestens einmal von „Ludwig Lehrer“
unterrichtet wurden.

\begin{antwort}
\begin{multline*}
\pi_{\text{Schüler}} (\\
  \sigma_{\text{Name} = \mlq \text{Ludwig Lehrer} \mrq} (\text{Person})\\
  \bowtie_{\text{Person.ID} = \text{Unterricht.Lehrer}}\\
  \text{Unterricht}\\
  \bowtie_{\text{Unterricht.Klassenbezeichnung} = \text{Klassenverband.Klassenbezeichnung}}\\
  \text{Klassenverband}\\
)
\end{multline*}
\end{antwort}

%%
% h)
%%

\item Formulieren Sie eine Anfrage in der relationalen Algebra, die
Namen und ID der Schüler bestimmt, die von allen Lehrern unterrichtet
wurden.

\begin{antwort}
\begin{multline*}
\pi_{\text{Name,ID}}(\\
  (\\
    \pi_{\text{Lehrer,Schueler}}(
      \text{Unterricht}
      \bowtie
      \text{Klassenverband}
    )
    \div
    \pi_{\text{ID}}(
      \sigma_{\text{Typ} = \mlq \text{L} \mrq }(\text{Person})
    )\\
  )\\
  \bowtie\\
  \text{Person}\\
)
\end{multline*}
\end{antwort}

\end{enumerate}

\noindent
Beachten Sie bei der Formulierung der SQL-Anfragen, dass die
Ergebnisrelationen keine Duplikate enthalten dürfen. Sie dürfen
geeignete Views definieren.

%-----------------------------------------------------------------------
%
%-----------------------------------------------------------------------

\section{Aufgabe 2: Relationale Algebra\footcite{db:pu:wh}}

Gegeben sei das folgende relationale Schema mitsamt Beispieldaten für
eine Datenbank von Mitfahrgelegenheiten. Die Primärschlüssel-Attribute
sind jeweils unterstrichen, Fremdschlüssel sind überstrichen.
\footcite[DB/ST - Frühjahr 2014 (nicht vertieft -46116), Thema 2, A2]{examen:46116:2014:03}

{
\footnotesize
\begin{multicols}{2}
„Kunde":

\begin{tabular}{|l|l|l|l|}
\hline
\ul{KID} & Name & Vorname & \TmpUeber{Stadt}\\\hline\hline
K1 & Meier & Stefan & S3\\\hline
K2 & Müller & Peta & S3\\\hline
K3 & Schmidt & Christine & S2\\\hline
K4 & Schulz & Michael & S4\\\hline
\end{tabular}

„Stadt"

\begin{tabular}{|l|l|l|}
\hline
\ul{SID} & SName & Bundesland\\\hline\hline
S1 & Berlin & Berlin\\\hline
S2 & Nürn & Bayern\\\hline
S3 & Köln & Nordrhein-Wesffalen\\\hline
S4 & Stuttgart & Baden-Württemberg\\\hline
S5 & München & Bayer\\\hline
\end{tabular}
\end{multicols}

\begin{multicols}{2}
„Angebot":

\begin{tabular}{|l|l|l|l|l|}
\hline
\ul{KID} & \TmpUeber{Start} & \TmpUeber{Ziel} & \ul{Datum} & Plätze\\\hline\hline
K4 & S4 & S5 & 08.07.2011 & 3\\\hline
K4 & S5 & S4 & 10.07.2011 & 3\\\hline
K1 & S1 & S5 & 08.07.2011 & 3\\\hline
K3 & S2 & S3 & 15.07.2011 & 1\\\hline
K4 & S4 & S1 & 15.07.2011 & 3\\\hline
K1 & S5 & S5 & 09.07.2011 & 2\\\hline
\end{tabular}

„Anfrage":

\begin{tabular}{|l|l|l|l|}
\hline
\ul{KID} & \TmpUeber{Start} & \TmpUeber{Ziel} & \ul{Datum}\\\hline\hline
K2 & S4 & S5 & 08.07.2011\\\hline
K2 & S5 & S4 & 10.07.2011\\\hline
K3 & S2 & S3 & 08.07.2011\\\hline
K3 & S3 & S2 & 10.07.2011\\\hline
K2 & S4 & S5 & 05.07.2011\\\hline
K2 & S5 & S4 & 17.07.2011\\\hline
\end{tabular}
\end{multicols}
}

\renewcommand{\labelenumi}{\arabic{enumi}.}
\begin{enumerate}
\item Formulieren Sie die folgenden Anfragen auf das gegebene Schema in
relationaler Algebra:

\begin{itemize}
\item Finden Sie die Namen aller Städte in Bayern!

\begin{antwort}
$\pi_{\text{SName}}(\sigma_{\text{Bundesland} = \text{Bayern}}(\text{Stadt}))$
\end{antwort}

%%
%
%%

\item Finden Sie die SIDs aller Städte, für die weder als Start noch als
Ziel eine Anfrage vorliegt!

\begin{antwort}
$
\pi_{\text{SID}}(\text{Stadt}) - \pi_{\text{Start}}(\text{Anfage}) - \pi_{\text{Ziel}}(\text{Anfrage})
$
\end{antwort}

%%
%
%%

\item Finden Sie alle IDs von Kunden, welche eine Fahrt in ihrer
Heimatstadt starten.

\begin{antwort}
\begin{multline*}
\pi_{\text{KID}}(\\
  \text{Kunde} \bowtie_{\text{Kunde.KID} = \text{Anfrage.KID} \land \text{Kunde.Stadt} = \text{Anfrage.Stadt}} \text{Anfrage}
)\\
\land\\
\pi_{\text{KID}}(\\
  \text{Kunde} \bowtie_{\text{Kunde.KID} = \text{Angebot.KID} \land \text{Kunde.Stadt} = \text{Angebot.Stadt}} \text{Angebot}
  )
\end{multline*}
\end{antwort}

%%
%
%%

\item Geben Sie das Datum aller angebotenen Fahrten von München nach
Stuttgart aus!

\begin{antwort}
\begin{multline*}
\pi_{\text{Datum}}(\\
  (\text{Angebot} \bowtie_{\text{Start} = \text{SID} \land \text{SName} = \mlq\text{München}\mrq} \text{Stadt})\\
  \bowtie_{\text{Ziel} = \text{SID} \land \text{SName} = \mlq\text{Stuttgart}\mrq}\\
  \text{Stadt}\\
)
\end{multline*}
\end{antwort}

Variante 2:

\begin{antwort}
\begin{multline*}
\pi_{\text{Datum}}(\\
  \sigma_{
    \text{Sname} = \mlq\text{München}\mrq \land
    \text{Zname} = \mlq\text{Stuttgart}\mrq
  }(\\
    \rho_{
      \text{Zname} \leftarrow \text{Sname},
      \text{SID1} \leftarrow \text{SID}
    }(\text{Stadt})\\
    \bowtie_{\text{Ziel} = \text{SID1}}\\
    \text{Angebot}\\
    \bowtie_{\text{Start} = \text{SID}}\\
    \text{Stadt}
  )
)
\end{multline*}
\end{antwort}

%%
%
%%

\end{itemize}

\item Geben Sie das Ergebnis (bezüglich der Beispieldaten) der folgenden
Ausdrücke der relationalen Algebra als Tabellen an:

%%
%
%%

\begin{itemize}
\item $\pi_{\text{KID}} (\text{Angebot}) \bowtie \text{Kunde}$

\begin{antwort}
Zeile mit der Petra Müller fällt weg.

\begin{tabular}{|l|l|l|l|}
\hline
\ul{KID} & Name & Vorname & \TmpUeber{Stadt}\\\hline\hline
K1 & Meier & Stefan & S3\\\hline
K3 & Schmidt & Christine & S2\\\hline
K4 & Schulz & Michael & S4\\\hline
\end{tabular}
\end{antwort}

%%
%
%%

\item $
\pi_{(\text{KID},\text{Stadt})} (\text{Kunde})
\bowtie_{\text{Kunde.Stadt} = \text{Angebot.Ziel}}
\pi_{\text{Plaetze}} (\text{Angebot})$

\begin{antwort}

\begin{tabular}{|l|l|l|}
\hline
KID & Stadt & Plätze \\\hline\hline
K1 & S3 & 1 \\\hline
K2 & S3 & 1 \\\hline
K4 & S4 & 1 \\\hline
K4 & S4 & 2 \\\hline
\end{tabular}
\end{antwort}
\end{itemize}
\end{enumerate}

\literatur

\end{document}
