\documentclass{lehramt-informatik}
\usepackage{fancyvrb}
\usepackage{soul}
\usepackage{multicol}
\def\TmpUeber#1{{\setul{-0.9em}{}\ul{#1}}}

\InformatikPakete{mathe,syntax}

\begin{document}

%%%%%%%%%%%%%%%%%%%%%%%%%%%%%%%%%%%%%%%%%%%%%%%%%%%%%%%%%%%%%%%%%%%%%%%%
% Theorie-Teil
%%%%%%%%%%%%%%%%%%%%%%%%%%%%%%%%%%%%%%%%%%%%%%%%%%%%%%%%%%%%%%%%%%%%%%%%

\chapter{Relationale Algebra}

\begin{quellen}
\item \cite{net:html:uni-innsbruck:relax}
\end{quellen}

\section{Symbole}
% https://dbai.tuwien.ac.at/education/dm/resources/symbols.html

\begin{tabular}{l|l|l|l}
\textbf{Name} & \textbf{Symbol} & \textbf{LaTeX} & \textbf{Alternativtext}\\\hline\hline
% σ
Selektion & $\sigma$ & \verb|\sigma| & SEL\\
% π
Projektion & $\pi$ & \verb|\pi| & PR\\
% ∪
Vereinigung & $\cup$ & \verb|\cup| & UNION\\
% ∩
Durchschnitt & $\cap$ & \verb|\cap| & INTERSECTION\\
% −
Mengendifferenz & $-$ & - & -\\
% ✕
kartesisches Produkt & $\times$ & \verb|\times| & X\\
% ρ ←
Umbenennung (+Zuweisung) & $\rho \leftarrow$ & \verb|\rho \leftarrow| & RENAME\\
% ÷
Division & $\div$ & \verb|\div| & DIV\\

% △
Symmetrische Differenz & $\bigtriangleup$ & \verb|\bigtriangleup| & \\

\hline

% ⋈
Join & $\bowtie$ & \verb|\bowtie| & JOIN\\

% ⟕
%Left Outer Join & {\tiny \textifsym{d|><|}} & \verb|{\tiny \textifsym{d|><|}}| & LOJOIN\\

% Right Outer Join & ⟖ & {\tiny \textifsym{|><|d}} & ROJOIN\\

% Full Outer Join & ⟗ & {\tiny \textifsym{d|><|d}} & FOJOIN\\

% ⋉ needs \usepackage{amssymb}
Left Semi Join & $\ltimes$ & \verb|\ltimes| & LSJOIN\\

% ⋊ needs \usepackage{amssymb}
Right Semi Join & $\rtimes$ & \verb|\rtimes| & RSJOIN\\

\hline

% ∧
Und & $\land$ & \verb|\land| & AND\\

% ∨
Oder & $\lor$ & \verb|\lor| & OR\\

% ¬
Negation & $\neg$ & \verb|\neg| & -\\

% ≥
Größer-gleich & $\geq$ & \verb|\geq| & >=\\

% ≤
Kleiner-gleich & $\leq$ & \verb|\leq| & <=\\

% ≠
Ungleich & $\neq$ & \verb|\neq| & =/=\\

% ≡
Äquivalenz & $\equiv$ & \verb|\equiv| & EQ\\

% ∃
Existenzquantor & $\exists$ & \verb|\exists| & EXISTS\\

% ∀
All-Quantor & $\forall$ & \verb|\forall| & FORALL\\
\end{tabular}

%-----------------------------------------------------------------------
%
%-----------------------------------------------------------------------

\section{Operationen der Relationen Algebra}

\subsection{Mengenoperation}

%%
%
%%

\subsubsection{Vereinigung}

\begin{description}
\item[Symbol-Schreibweise] $R \cup S$
\item[SQL] UNION
\end{description}

%%
%
%%

\subsubsection{Mengendifferenz}

\begin{description}
\item[Symbol-Schreibweise] $R - S$
\item[SQL] EXCEPT
\end{description}

%%
%
%%

\subsubsection{Mengendurchschnitt (Schnittmenge/Intersection)}

\begin{description}
\item[Symbol-Schreibweise] $R \cap S$
\item[SQL] INTERSECT
\end{description}

%%
%
%%

\subsubsection{Symmetrische Differenz}

\begin{description}
\item[Symbol-Schreibweise] $R \bigtriangleup S$
\item[SQL] INTERSECT
\end{description}

%-----------------------------------------------------------------------
%
%-----------------------------------------------------------------------

\subsection{Selektion}

\begin{description}
\item[Symbol-Schreibweise] $\sigma_{Ausdruck}(R)$
\item[lineare Schreibweise] $R[Ausdruck]$
\item[SQL] WHERE
\end{description}

%-----------------------------------------------------------------------
%
%-----------------------------------------------------------------------

\subsection{Projektion}

\begin{description}
\item[Symbol-Schreibweise] $\pi_{\beta}(R)$
\item[lineare Schreibweise] $R[\beta]$
\item[SQL] SELECT
\end{description}

%-----------------------------------------------------------------------
%
%-----------------------------------------------------------------------

\subsection{Kartesisches Produkt (Kreuzprodukt)}

\begin{description}
\item[Symbol-Schreibweise] $R \times S$
\item[lineare Schreibweise] $R~x~S$
\item[SQL] CROSS JOIN
\end{description}

%-----------------------------------------------------------------------
%
%-----------------------------------------------------------------------

\subsection{Umbenennung}

\begin{description}
\item[Symbol-Schreibweise] $\rho_{[{neu}\leftarrow alt]} (R)$
\item[lineare Schreibweise] $R[alt\rightarrow neu]$
\end{description}

%-----------------------------------------------------------------------
%
%-----------------------------------------------------------------------

\subsection{Division\footcite[Division]{relationale-algebra}}

\begin{description}
\item[Symbol-Schreibweise] $R \div S$
\end{description}

Die Division ist dann definiert durch:

$R\div S:=\pi_{{R'}}(R)-\pi_{{R'}}((\pi_{{R'}}(R)\times S)-R)$

\begin{minted}{sql}
SELECT distinct MatrNr
FROM hoert
WHERE MatrNr NOT IN(
  SELECT R.MatrNr
  FROM hoert R, Professor P, Vorlesung V
  WHERE P.Name = 'Sokrates'
  AND P.PersNr=V.gelesenVon
  AND (R.MatrNr, V.VorlNr) NOT IN (
    SELECT MatrNr, VorlNr
    FROM hoert
  )
);
\end{minted}

%-----------------------------------------------------------------------
%
%-----------------------------------------------------------------------

\section{Die 5 Grundoperationen der Relationalen Algebra
\footcite{net:pdf:lmu:dbs}
}

Mit diesen Grundoperationen lassen sich weitere Operationen (z. B. die
Schnittmenge) nachbilden.

\begin{itemize}
\item Verenigung $R = S \cup T$
\item Differenz $R = S - T$
\item Kartesisches Prokukt (Kreuzprodukt) $R = S \times T$
\item Selection $R = \sigma_F(S)$
\item Projektion $R = \pi_{A,B,...}(S)$
\end{itemize}

%%%%%%%%%%%%%%%%%%%%%%%%%%%%%%%%%%%%%%%%%%%%%%%%%%%%%%%%%%%%%%%%%%%%%%%%
% Aufgaben
%%%%%%%%%%%%%%%%%%%%%%%%%%%%%%%%%%%%%%%%%%%%%%%%%%%%%%%%%%%%%%%%%%%%%%%%

\chapter{Aufgaben}

%-----------------------------------------------------------------------
%
%-----------------------------------------------------------------------

\ExamensAufgabeTTA 46116 / 2018 / 09 : Thema 2 Teilaufgabe 2 Aufgabe 2

%-----------------------------------------------------------------------
%
%-----------------------------------------------------------------------

\ExamensAufgabeTTA 46116 / 2015 / 03 : Thema 1 Teilaufgabe 2 Aufgabe 1

%-----------------------------------------------------------------------
%
%-----------------------------------------------------------------------

\section{Bundesliga-Datenbank
\footcite[Aufgabe 4: Relationale Algebra und SQL]{db:ab:7}}

Gegeben sei die folgende Bundesliga-Datenbank, in der die Vereine,
Spiele, Trainer und Spieler mit ihren Einsätzen für die laufende Saison
verwaltet werden:

\begin{itemize}
\item VEREIN (VNAME, ORT, PRÄSIDENT)
\item SPIELE (HEIM, GAST, RESULTAT, ZUSCHAUER, TERMIN, SPIELTAG, H-TRAINER,
G-TRAINER)
\item SPIELER (SPNR, NAME, VORNAME, VEREIN, ALTER, GEHALT, GEB-ORT)
\item TRAINER (TRNR, NAME, VORNAME, VEREIN, ALTER, GEHALT, GEB-ORT)
\item EINSATZ (HEIM, GAST, SPNR, VON, BIS, TORE, KARTE)
\end{itemize}

\begin{enumerate}
\item Zeichnen Sie das „zugehörige“ ER-Modell!

\item Formulieren Sie folgende Anfragen in relationaler Algebra und in
SQL:

\begin{itemize}
\item Welche Spieler haben beim Spiel TSV 1860 München – FC Bayern
München mitgewirkt?

\begin{antwort}
$\pi_{\text{NAME,VORNAME}}(
  \text{Spieler}
  \bowtie
  (\sigma_{
    \text{HEIM} = \mlq \text{TSV 1860 München} \mrq \land
    \text{GAST} = \mlq \text{FC Bayern München} \mrq
  }(\text{Einsatz}))
)$

\begin{minted}{sql}
SELECT NAME, VORNAME FROM Spieler, Einsatz,
WHERE
  HEIM = 'TSV 1860 München' AND GAST = 'FC Bayern München' AND
  Einsatz.SPNR = Spieler.SPNR;
\end{minted}
\end{antwort}

%%
%
%%

\item Welche Spiele sind 2 : 0 ausgegangen?

\begin{antwort}
$\pi_{\text{HEIM},\text{GAST},\text{SPIELTAG}}(\sigma_{\text{RESULTAT} = \mlq 2 : 0 \mrq}(\text{SPIELE}))$

\begin{minted}{sql}
SELECT HEIM, GAST, SPIELTAG FROM SPIELE
WHERE RESULTAT = '2 : 0';
\end{minted}
\end{antwort}

%%
%
%%

\item Welche Spieler spielen in einem Verein ihres Geburtsortes?

\begin{antwort}
$\pi_{\text{NAME},\text{VORNAME}}(
  \text{VEREIN}
  \bowtie_{\text{VEREIN.ORT} = \text{SPIELER.GEB-ORT} \land \text{SPIELER.VEREIN} = \text{VEREIN.VNAME}}
  \text{SPIELER}
)$

\begin{minted}{sql}
SELECT NAME, VORNAME
FROM SPIELER, VEREIN
WHERE VEREIN.ORT = SPIELER.GEB-ORT AND SPIELER.VEREIN = VEREIN.VNAME;
\end{minted}
\end{antwort}

%%
%
%%

\item Welche Spieler vom 1. FC Köln haben alle Spiele mitgemacht?

\begin{antwort}
\def\tmpkoeln{\mlq \text{1. FC Köln} \mrq}

\begin{multline*}
\pi_{\text{NAME,VORNAME}}(\\
  \sigma_{\text{VEREIN} = \tmpkoeln}(\text{SPIELER})\\
  \bowtie\\
  (
    \pi_{\text{HEIM,GAST,SPNR}}(\sigma_{\text{HEIM} = \tmpkoeln \lor \text{GAST} = \tmpkoeln}(\text{EINSATZ}))\\
    \div\\
    \pi_{\text{HEIM,GAST}}(\sigma_{\text{HEIM} = \tmpkoeln \lor \text{GAST} = \tmpkoeln}(\text{SPIELE}))
  )
)
\end{multline*}

\begin{minted}{sql}
SELECT NAME, VORNAME
FROM SPIELER
WHERE VEREIN = '1. FC Koeln' AND SPNR IN (
  SELECT SPNR
  FROM EINSATZ
  WHERE HEIM = '1. FC Koeln' OR GAST = '1. FC Koeln'
  GROUP BY SPNR
  HAVING COUNT(*) = (
    SELECT COUNT(*)
    FROM SPIELE
    WHERE HEIM = '1. FC Koeln' OR GAST = '1. FC Koeln'
  )
);
\end{minted}
\end{antwort}

%%
%
%%

\item Wie heißen die Präsidenten der Vereine, die zur Zeit einen Trainer
beschäftigen, der jünger ist als der älteste Spieler, der beim Verein
beschäftigt ist?

\begin{antwort}
\begin{multline*}
\pi_{\text{PRÄSIDENT}}(\\
  \text{VEREIN}\\
  \bowtie\\
  \pi_{\text{VEREIN}}(\\
    \text{TRAINER}\\
    \bowtie_{
      \text{TRAINER.ALTER} < \text{SPIELTER.ALTER} \land
      \text{TRAINER.VEREIN} = \text{SPIELER.VEREIN}
    }\\
    \text{SPIELER}\\
  )\\
)
\end{multline*}

\begin{minted}{sql}
SELECT PRÄSIDENT
FROM VEREIN, SPIELER, TRAINER
WHERE
  VEREIN.VNAME = SPIELER.VEREIN AND
  VEREIN.VNAME = VEREIN.TRAINER AND
  TRAINER.ALTER < SPIELER.ALTER;
\end{minted}
\end{antwort}

%%
%
%%

\item Welche Spieler haben bisher noch nie gespielt?

\begin{antwort}
$\pi_{\text{SPNR}}(\text{SPIELER}) - \pi_{\text{SPNR}}(\text{EINSATZ})$

\begin{minted}{sql}
SELECT SPNR FROM SPIELER
EXCEPT
SELECT SPNR FROM EINSATZ;
\end{minted}
\end{antwort}

%%
%
%%

\item Welche Spieler haben bisher noch kein Tor geschossen?

\begin{antwort}
$\pi_{\text{SPNR}}(\text{SPIELER}) -
\pi_{\text{SPNR}}(\sigma_{\text{TORE} > 0}(\text{EINSATZ}))$

\begin{minted}{sql}
SELECT SPNR FROM SPIELER
EXCEPT
SELECT SPNR FROM EINSATZ WHERE TORE > 0;
\end{minted}
\end{antwort}

%%
%
%%

\item Welcher Trainer hat schon mehr als einen Verein trainiert? Welche
Vereine haben schon mehrere Trainer gehabt?

%%
%
%%

\begin{antwort}
Vereine:

\begin{minted}{sql}
SELECT HEIM FROM SPIELE l, SPIELE r
WHERE l.HEIM = r.HEIM AND NOT (l.H-TRAINER = r.H-TRAINER)
UNION
SELECT GAST FROM SPIELE l, SPIELE r
WHERE l.GAST = r.GAST AND NOT (l.G-TRAINER = r.G-TRAINER)
UNION
SELECT HEIM FROM SPIELE l, SPIELE r
WHERE l.HEIM = r.GAST AND NOT (l.H-TRAINER = r.G-TRAINER)
\end{minted}

Trainer:

\begin{minted}{sql}
SELECT H-TRAINER FROM SPIELE l, SPIELE r
WHERE l.H-TRAINER = r.HEIM AND NOT (l.HEIM = r.HEIM)
UNION
SELECT G-TRAINER FROM SPIELE l, SPIELE r
WHERE l.G-TRAINER = r.G-TRAINER AND NOT (l.GAST = r.GAST)
UNION
SELECT H-TRAINER FROM SPIELE l, SPIELE r
WHERE l.H-TRAINER = r.G-TRAINER AND NOT (l.H-TRAINER = r.G-TRAINER)
\end{minted}
\end{antwort}

%-----------------------------------------------------------------------
%
%-----------------------------------------------------------------------

\item Welche Spiele am 10. Spieltag hatten mehr als 30.000 Zuschauer?

\begin{antwort}
$\pi_{\text{HEIM,GAST}}(
  \sigma_{\text{ZUSCHAUER} > 30000 \land \text{SPIELTAG} = 10}(\text{SPIELE})
)$

\begin{minted}{sql}
SELECT HEIM, GAST
  FROM SPIELE
  WHERE ZUSCHAUER > 30000 AND SPIELTAG = 10;
\end{minted}
\end{antwort}
\end{itemize}
\end{enumerate}

%-----------------------------------------------------------------------
%
%-----------------------------------------------------------------------

\ExamensAufgabeTTA 66116 / 2016 / 09 : Thema 2 Teilaufgabe 1 Aufgabe 2

%-----------------------------------------------------------------------
%
%-----------------------------------------------------------------------

\ExamensAufgabeTTA 46116 / 2014 / 03 : Thema 2 Teilaufgabe 2 Aufgabe 2

\literatur

\end{document}
