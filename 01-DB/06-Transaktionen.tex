\documentclass{lehramt-informatik}

\begin{document}

\section{Definition}

Unter einer Transaktion versteht man die „Bündelung“ mehrerer
Datenbankoperationen, die in einem Mehrbenutzersystem ohne unerwünschte
Einflüsse durch andere Transaktionen als Einheit fehlerfrei ausgeführt
werden sollen.

\section{ACID-Prinzip}

\begin{quellen}
\cite[Kapitel 9.5 „Eigenschaften Von Transaktionen“, Seite 299]{kemper}
\cite[Seite 1]{db:fs:5}
\cite{wiki:acid}
\end{quellen}

\begin{description}
\item[Atomicity]

Eine Transaktion ist atomar, d. h. von den vorgesehenen
Änderungsoperationen auf die Datenbank haben entweder alle oder keine
eine Wirkung auf die Datenbank.

\item[Consistency]

Eine Transaktion überführt einen korrekten (konsistenten)
Datenbankzustand wieder in einen korrekten (konsistenten)
Datenbankzustand.

\item[Isolation]

Eine Transaktion bemerkt das Vorhandensein anderer (parallel
ablaufender) Transaktionen nicht und beeinflusst auch andere
Transaktionen nicht.

\item[Durability]

Die durch eine erfolgreiche Transaktion vorgenommenen Änderungen sind
dauerhaft (persistent).

\end{description}

\section{Operationen auf Transaktions-Ebene}

\begin{quellen}
\item \cite[Seite 5]{db:fs:5}
\item \cite[Seite 211]{winter}
\end{quellen}

\begin{description}
\item[read \& write]

als Operationen zur Ausführung von Änderungen

\item[begin of transaction (BOT)]

Mit diesem Befehl wird der Beginn einer eine Transaktion darstellende
Befehlsfolge gekennzeichnet.

\item[commit]

Hierdurch wird die Beendigung der Transaktion eingeleitet. Alle
Änderungen der Datenbasis werden durch diesen Befehl festgeschrieben,
d.h. sie werden dauerhaft in die Datenbank eingebaut.

\item[abort]

Dieser Befehl führt zu einem Selbstabbruch der Transaktion. Das
Datenbanksystem muss sicherstellen, dass die Datenbasis wieder in den
Zustand zurückgesetzt wird, der vor Beginn der Transaktionsausführung
existierte.
\end{description}

Für den Abschluss einer Transaktion gibt es 2 Möglichkeiten:

\begin{itemize}
\item Den erfolgreichen Abschluss mit \texttt{commit}.
\item Den erfolglosen Abschluss mit \texttt{abort}
\end{itemize}

%-----------------------------------------------------------------------
%
%-----------------------------------------------------------------------

\section{Zustandsübergänge einer Transaktion}

\begin{quellen}
\item \cite[Kapitel 9.7, Seite 301]{kemper}
\item \cite[Seite 5]{db:fs:5}
\end{quellen}

\begin{description}
\item[potentiell]

Die Transaktion ist codiert und „wartet darauf", in den Zustand aktiv zu
wechseln. Diesen Übergang nennen wir inkarnieren.

\item[aktiv]

Die aktiven (d.h. derzeit rechnenden) Transaktionen konkurrieren
untereinander um die Betriebsmittel, wie z.B. Hauptspeicher, Rechnerkern
zur Ausführung von Operationen, etc.

\item[wartend]

Bei einer Überlast des Systems: (z.B. thrashing (Seitenflattern) des
Puffers) kann die Transaktionsverwaltung einige aktive Transaktionen in
den Zustand wartend verdrängen. Nach Behebung der Überlast werden diese
wartenden Transaktionen sukzessive wieder eingebracht, d.h. wieder
aktiviert.

\item[abgeschlossen]

Durch den commit-Befehl wird eine aktive Transaktion beendet. Die
Wirkung abgeschlossener TAS kann aber nicht gleich in der Datenbank
festgeschrieben werden. Vorher müssen noch möglicherweise verletzte
Konsistenzbedingungen überprüft werden.

\item[persistent]

Die Wirkungen abgeschlossener Transaktionen werden — wenn die
Konsistenzerhaltung sichergestellt ist — durch festschreiben dauerhaft
in die Datenbasis eingebracht. Damit ist die Transaktion persistent.
Dies ist einer von zwei möglichen Endzuständen einer
Transaktionsverarbeitung.

\item[gescheitert]

Transaktionen können aufgrund vielfältiger Ereignisse scheitern. Z.B.
kann der Benutzer selbst durch ein abort eine aktive Transaktion
abbrechen. Weiterhin können Systemfehler zum Scheitern aktiver oder
wartender Transaktionen führen. Bei abgeschlossenen Transaktionen können
auch Konsistenzverletzungen festgestellt werden, die ein Scheitern
veranlassen.

\item[wiederholbar]

Einmal gescheiterte Transaktionen sind u.U. wiederholbar. Dazu muss
deren Wirkung auf die Datenbasis zunächst zurückgesetzt werden. Danach
können sie durch Neustarten wiederum aktiviert werden.

\item[aufgegeben]

Eine gescheiterte Transaktion kann sich aber auch als „hoffnungslos"
herausstellen. In diesem Fall wird ihre Wirkung zurückgesetzt und die
Transaktionsverarbeitung geht in den Endzustand aufgegeben über.
\end{description}

%-----------------------------------------------------------------------
%
%-----------------------------------------------------------------------

\section{Fehler}

\section{“Einfacher” Fehlerfall}

Änderungen durch nicht zu Ende geführte Transaktionen müssen rückgängig
gemacht werden. Diesen Vorgang nennt man \emph{Rollback}. Dazu müssen
gegebenenfalls die zuletzt gültigen Werte der geänderten
Datenbankobjekte in die Datenbank geschrieben werden.

Es muss sichergestellt werden, dass die Änderungen erfolgreich zu Ende
geführter Transaktionen tatsächlich permanent gemacht werden. Dazu
müssen gegebenenfalls die neuen Werte der Datenbankobjekte erneut auf
die Datenbank geschrieben werden.

%%
%
%%

\subsection{Das Lost-Update-Problem\footcite{wiki:verlorenes-update}}

Änderungen in einer Datenbank, die durch eine Transaktion vorgenommen
wurden, gehen aufgrund unkontrollierter Parallelausführung verloren.

\emph{„Überschreiben“}

%%
%
%%

\subsection{Das Dirty-Read-Problem\footcite{wiki:schreib-lese-konflikt}}

Der Wert eines Datenobjekts, der noch nicht permanent gespeichert wurde,
wird gelesen.

\emph{„Zwischenstände lesen“}

%%
%
%%

\subsection{Das Unrepeatable-Read-Problem\footcite{wiki:nichtwiederholbares-lesen}}

Während einer Transaktion wird der Wert eines Datenobjekts zweimal
gelesen, wobei dieser Wert aber in der Zwischenzeit von einer anderen
Transaktion geändert wurde.

\emph{„Verschiedene Zwischenstände lesen“}

%-----------------------------------------------------------------------
%
%-----------------------------------------------------------------------

\section{Zweiphasige Abarbeitung mit Sperren}

\cite[Seite 15]{db:fs:5}

Die Abarbeitung von Transaktionen erfolgt zweiphasig oder genügt dem
2-Phasen-Sperrprotokoll, wenn keine Transaktion eine Sperre freigibt,
bevor sie alle benötigten Sperren angefordert hat.

\begin{itemize}
\item Ein strenges 2-Phasen-Sperrprotokoll wird dadurch realisiert, dass

\begin{itemize}
\item eine Transaktion sukzessive alle benötigten Sperren anfordert
(alle Elemente, die eine Transaktion lesen oder ändern möchte, werden
von dieser Transaktion reserviert und dürfen nicht von anderen
Transaktionen abgegriffen werden),

\item alle Lesesperren (\textbf{S}hared oder \textbf{R}ead) bis zum
Ende, d. h. bis zur Aktion commit, hält und

\item alle Schreibsperren (e\textbf{X}clusive oder \textbf{W}rite) bis
nach \texttt{commit} hält.
\end{itemize}

\item Eine bereits gesetzte Lesesperre (durch Angabe von \texttt{rlock})
kann durch \texttt{xlock} in eine \emph{Schreibsperre umgewandelt}
werden, ohne dass vorher die Lesesperre aufgehoben werden muss.

\item Reine \emph{Lesesperren} auf ein Element können \emph{mehrere}
Transaktionen gleichzeitig besitzen, dann ist jedoch \emph{keine
Schreibsperre} mehr möglich.
\end{itemize}

%%
%
%%

\subsection{2-Phasen-Sperrprotokoll\footcite[Seite 16]{db:fs:5}}

Die Serialisierbarkeit ist bei Einhaltung des folgenden
Zwei-Phasen-Sperrprotokolls durch den Scheduler gewährleistet. Bezogen
auf individuelle Transaktion wird folgendes verlangt:\footcite{wiki:sperrverfahren}

\begin{enumerate}
\item Jedes Objekt, das von einer Transaktion benutzt werden soll, muss
vorher entsprechend gesperrt werden.

\item Eine Transaktion fordert eine Sperre, die sie schon besitzt, nicht
erneut an.

\item Eine Transaktion muss die Sperren anderer Transaktionen auf dem
von ihr benötigten Objekt gemäß der Verträglichkeitstabelle beachten.
Wenn die Sperre nicht gewährt werden kann, wird die Transaktion in eine
entsprechende Warteschlange eingereiht — bis die Sperre gewährt werden
kann.

\item Jede Transaktion durchläuft zwei Phasen:

\begin{itemize}
\item Eine Wachstumsphase, in der sie Sperren anfordern, aber keine
freigeben darf und

\item Eine Schrumpfungsphase, in der sie ihre bisher erworbenen
Sperren freigibt, aber keine weiteren anfordern darf.
\end{itemize}

\item Bei \texttt{EOT} (Transaktionsende) muss eine Transaktion alle
ihre Sperren zurückgeben
\end{enumerate}

%%
%
%%

\subsection{Deadlock (bei Transaktionen)}

Deadlock bei den sperrbasierten Synchronisationsmethoden:

\begin{itemize}
\item Beide Transaktionen können zunächst lesen

\item Beim Umwandeln der Lesesperre in eine Schreibsperre in Schritt 9
kommt es zu einem Sperrkonflikt, da T2 noch die Lesesperre hält.

\item Die Transaktion T1 wird vom System verzögert.

\item In Schritt 10 möchte T2 ihre Lesesperre in eine Schreibsperre
umwandeln.

\item Es kommt erneut zu einem Sperrkonflikt, der wiederum zur
Verzögerung der zweiten Transaktion führt.
\end{itemize}

\begin{itemize}
\item Deadlock-Situationen müssen entweder vermieden oder vom DBMS
erkannt und aufgelöst werden.

\item Das DBMS kann in diesem Fall eine der beiden Transaktionen
abbrechen und diese von Neuem starten.
\end{itemize}

%-----------------------------------------------------------------------
%
%-----------------------------------------------------------------------

\section{Recovery-Klassen: Wiederherstellungsmechanismen des DBS nach
Fehlern}

\begin{description}

%%
%
%%

\item[Partial Undo (partielles Zurücksetzen / R1-Recovery)]

Nach Transaktionsfehler

\begin{itemize}
\item Isoliertes und vollständiges Zurücksetzen der veränderten Daten in
den Zustand zu Beginn der Transaktion

\item Beeinflusst andere Transaktionen nicht!
\end{itemize}

%%
%
%%

\item[Partial Redo (partielles Wiederholen / R2-Recovery)]

Nach Systemfehler (mit Verlust des Hauptspeicherinhalts)

\begin{itemize}
\item Wiederholung aller verlorengegangenen Änderungen (waren nur im
Puffer) von abgeschlossenen Transaktionen
\end{itemize}

%%
%
%%

\item[Global Undo (vollständiges Zurücksetzen / R3-Recovery)]

Nach Systemfehler (mit Verlust des Hauptspeicherinhalts), z.B.
Stromausfall

\begin{itemize}
\item Zurücksetzen aller durch den Ausfall abgebrochenen Transaktionen
\end{itemize}

%%
%
%%

\item[Global Redo (vollständiges Wiederholen / R4-Recovery)]

Nach Gerätefehler

\begin{itemize}
\item Einspielen einer Archivkopie auf neuen Datenträger und
Nachvollziehen aller beendeten Transaktionen, die nach der letzten
beendeten Transaktion auf der Archivkopie noch ausgeführt wurden
\end{itemize}

\end{description}

%-----------------------------------------------------------------------
%
%-----------------------------------------------------------------------

\section{Puffer-Verwaltung}

\cite[Seite 25]{db:fs:5}
\cite[Seite 305]{kemper}

Wann werden Änderungen, die von Transaktionen ausgelöst wurden, vom
temporären Puffer in die Datenbank geschrieben (permanent)?

%-----------------------------------------------------------------------
%
%-----------------------------------------------------------------------

\section{Verdrängungsstrategien}

Ersetzung „schmutziger“ Seiten im Puffer, d.h. Seite Puffer $\neq$ Seite
DB $\rightarrow$ Seite wurde verändert

\begin{itemize}

\item No-Steal

\begin{itemize}
\item Schmutzige Seiten dürfen nicht aus dem Puffer entfernt und in die
DB übertragen werden, solange die Transaktion noch aktiv ist

\item DB enthält keine Änderungen nicht-erfolgreicher TAs

\item UNDO-Recovery ist nicht erforderlich

\item Probleme bei langen Änderungs-TAs, da große Teile des Puffers
blockiert werden
\end{itemize}

\item Steal

\begin{itemize}
\item Schmutzige Seiten dürfen jederzeit ersetzt und in die DB
eingebracht werden

\item DB kann unbestätigte Änderungen enthalten

\item UNDO-Recovery ist erforderlich

\item Effektivere Puffernutzung bei langen TAs mit vielen Änderungen
\end{itemize}
\end{itemize}

%%
%
%%

\subsection{Ausschreibestrategien (EOT-Behandlung)}

\begin{itemize}

\item Force

\begin{itemize}
\item Alle geänderten Seiten werden spätestens bei EOT (vor COMMIT) in
die DB geschrieben

\item Keine REDO-Recovery erforderlich bei Systemfehler

\item Hoher I/O-Aufwand, da Änderungen jeder TA einzeln geschrieben
werden

\item Vielzahl an Schreibvorgängen führt zu schlechteren Antwortzeiten,
länger gehaltenen Sperren und damit zu mehr Sperrkonflikten

\item Große DB-Puffer werden schlecht genutzt
\end{itemize}

\item No-Force

\begin{itemize}
\item Änderungen können auch erst nach dem COMMIT in die DB geschrieben
werden, „Sammeln“ von Änderungen durch mehrere Transaktionen

\item Beim COMMIT werden lediglich REDO-Informationen in die Log-Datei
geschrieben

\item REDO-Recovery erforderlich bei Systemfehler

\item Änderungen auf einer Seite über mehrere TAs hinweg können
gesammelt werden
\end{itemize}
\end{itemize}

\literatur

\end{document}
