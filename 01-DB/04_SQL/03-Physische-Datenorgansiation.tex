\documentclass{lehramt-informatik-haupt}

\begin{document}

\chapter{Physische Datenorganisation}

\tableofcontents

(Foliensatz Seite 18)

\section{Definition „Physische Datenunabhängigkeit“:}

\begin{mdframed}
Änderungen an der physischen Speicher- oder der Zugriffsstruktur
(beispielsweise Anlegen einer Indexstruktur) haben keine Auswirkungen
auf die logische Struktur der Datenbasis, also auf das Datenbankschema
\end{mdframed}

%-----------------------------------------------------------------------
%
%-----------------------------------------------------------------------

\section{ANSI-SPARC-Architektur (auch Drei-Schema-Architektur)
\footcite[Seite 443, 13.1.3 Architektur eines Datenbanksystems]{schneider}
}

Trennung des Datenbankschemas in 3 verschiedene Beschreibungsebenen:

\begin{description}
\item[Externe Ebene]

stellt Benutzern und Anwendungen individuelle Benutzersichten bereit.
Bsp.: Formulare, Masken-Layouts, Schnittstellen

\item[Konzeptionelle Ebene]

beschriebt, welche Daten in der Datenbank gespeichert sind, sowie deren
Beziehungen. Designziel: vollständige und redundanzfreie Darstellung
aller zu speichernden Informationen. Hier findet die Normalisierung des
relationalen Datenbankschemas statt.

\item[Interne / physische Ebene]

stellt physische Sicht der Datenbank im Computer dar. Beschreibt, wie
und wo die Daten in der Datenbank gespeichert werden. Designziel:
effizienter Zugriff auf die gespeicherten Informationen, meist durch
bewusst in Kauf genommene Redundanz erreicht (z. B. Index speichert die
gleichen Daten, die auch schon in der Tabelle gespeichert sind).

\end{description}

\subsection{Vorteile des Modells:}

\begin{description}
\item[physischen Datenunabhängigkeit] da die interne von der
konzeptionellen und externen Ebene getrennt ist. Physische Änderungen,
z. B. des Speichermediums oder des Datenbankprodukts, wirken sich nicht
auf die konzeptionelle oder externe Ebene aus.

\item[logischen Datenunabhängigkeit] da die konzeptionelle und die
externe Ebene getrennt sind. Dies bedeutet, dass Änderungen an der
Datenbankstruktur (konzeptionelle Ebene) keine Auswirkungen auf die
externe Ebene, also die Masken-Layouts, Listen und Schnittstellen haben.
\end{description}

Allgemein: höhere Robustheit gegenüber Änderungen

%-----------------------------------------------------------------------
%
%-----------------------------------------------------------------------

\section{Physische Datenorganisation
\footcite[Seite 21]{db:fs:3}}

\begin{itemize}
\item Daten werden in Form von \emph{Sätzen} auf der Festplatte
abgelegt, um auf Sätze zugreifen zu können, verfügt jeder Satz über eine
\emph{eindeutige, unveränderliche Satzadresse}

\item TID = Tupel Identifier: dient zur Adressierung von Sätzen in einem
Segment und besteht aus zwei Komponenten

\begin{itemize}
\item Seitennummer (Seiten bzw. Blöcke sind größere Speichereinheiten
auf der Platte)

\item Relative Indexposition innerhalb der Seite
\end{itemize}

\item Satzverschiebung innerhalb einer Seite bleibt ohne Auswirkungen
auf TID, wird ein Satz auf eine andere Seite migriert, wird eine
„Stellvertreter-TID“ zum Verweis auf den neuen Speicherort verwendet.
Die eigentliche TID-Adresse bleibt stabil
\footcite[Seite 219]{kemper}
\end{itemize}

%%%%%%%%%%%%%%%%%%%%%%%%%%%%%%%%%%%%%%%%%%%%%%%%%%%%%%%%%%%%%%%%%%%%%%%%
% Aufgaben
%%%%%%%%%%%%%%%%%%%%%%%%%%%%%%%%%%%%%%%%%%%%%%%%%%%%%%%%%%%%%%%%%%%%%%%%

\chapter{Aufgaben}

\section{Aufgabe 3: Physische Datenorganisation, TID
\footcite{db:pu:3}}

DB/ST - Frühjahr 2017, Thema 1 Teilaufgabe 2 A4

\begin{enumerate}
\item Erläutern Sie in ein bis zwei Sätzen, aus welchen zwei Teilen
sich ein TID (Tupelldentifikator) zusammensetzt.

\begin{antwort}
Seitennummer (Seiten bzw. Blöcke sind größere Speichereinheiten auf der
Platte) Relative Indexposition innerhalb der Seite
\end{antwort}

\item Erläutern Sie in ein bis zwei Sätzen das Vorgehen, wenn ein
durch einen TID adressierter Satz innerhalb einer Seite verschoben
werden muss.

\begin{antwort}
Satzverschiebung innerhalb einer Seite bleibt ohne Auswirkungen auf TID,
\end{antwort}

\item Erläutern Sie in ein bis zwei Sätzen das Vorgehen, wenn ein
durch einen TID adressierter Satz erstmalig in eine andere Seite
verschoben werden muss.

\begin{antwort}
wird ein Satz auf eine andere Seite migriert, wird eine
„Stellvertreter-TID“ zum Verweis auf den neuen Speicherort verwendet.
Die eigentliche TID-Adresse bleibt stabil
\end{antwort}

\item Erläutern Sie in zwei bis drei Sätzen das Vorgehen, wenn ein
durch einen TID adressierter und bereits einmal über Seitengrenzen
hinweg verschobener Satz erneut in eine andere Seite verschoben werden
muss.

\begin{antwort}
Es wird eine neue stellvertreter TID aktualisiert.
\end{antwort}

\end{enumerate}

\literatur
\end{document}
