\documentclass{lehramt-informatik}
\usepackage{amsmath}
\begin{document}

%-----------------------------------------------------------------------
%
%-----------------------------------------------------------------------

\chapter{Schlüssel}

\begin{lernkartei}{Schlüssel}
\begin{description}
\item[Superschlüssel] Attribut oder Attributkombination, von
der alle Attribute einer Relation funktional abhängen

\item[Schlüsselkandidat] \textbf{Minimaler} Superschlüssel
(Keine Teilmenge dieses Superschlüssels ist ebenfalls Superschlüssel)

\item[Primärschlüssel] Unter allen Schlüsselkandidaten einer
Relation wird ein sogenannter \textbf{Primärschlüssel} ausgewählt.

\item[Schlüsselattribut] Attribut, das Teil eines
Schlüsselkandidaten ist

\item[Nicht-Schlüsselattribut] Attribut, das an keinem der
Schlüsselkandidaten beteiligt ist
\end{description}
\end{lernkartei}

%-----------------------------------------------------------------------
%
%-----------------------------------------------------------------------

\subsection{Bestimmung von Schlüsselkandidaten ohne Algorithmus}

\begin{itemize}
\item Attribute, die auf \textbf{keiner rechten Seite} einer FD
vorkommen.

\item Attribute, die in \textbf{keiner FD} vorkommen.
\end{itemize}

%-----------------------------------------------------------------------
%
%-----------------------------------------------------------------------

\subsection{Algorithmus zur Bestimmung von Schlüsselkandidaten}

\href{http://www.ict.griffith.edu.au/~jw/normalization/ind.php}
{Normalization Tool of the Griffith Univerity}

Foliensätze Seite 8

\begin{itemize}
\item
$Erg = \{\}$,
$Test = \{\{\textit{alle Attribute der Relation}\}\}$,
$K = \{\textit{alle Attribute der Relation}\}$

\begin{itemize}
\item Wähle ein Attribut aus $K$, ist
$\textit{AttrHüll}(F,K \SLASH \{A\}) = R$?
(Also die Menge $K$ ohne das Attribut $A$)

\item ja $\Rightarrow$ ändere $\textit{Test}$:
streiche $K$ aus $\textit{Test}$, füge $K \SLASH \{A\}$ in $\textit{Test}$ ein.

\item $K$ selbst bleibt unverändert!

\item entferne ein anderes Attribut aus $K$, so lange, bis alle
Attribute reihum untersucht wurden.
\end{itemize}

\item wenn kein $K\{A\}$ Superschlüssel $\Rightarrow$
$K$ ist Schlüsselkandidat!
Füge $K$ zu $Erg$ hinzu und lösche $K$ aus $\textit{Test}$.

\item mache dasselbe mit allen Mengen, die jetzt in $\textit{Test}$ sind,
bis $\textit{Test}$ leer ist.

\end{itemize}

\subsection{Am Beispiel einer Aufgabe}

Seite 12

Gegeben sei ein Relationenschema $R$ mit Attributen $A, B, C, D$. Für
dieses Relationenschema seien die folgenden Mengen an funktionalen
Abhängigkeiten (FDs) gegeben:

\begin{multline*}
F = \{ \\
  A \rightarrow B,\\
  B \rightarrow C,\\
  A \rightarrow D,\\
\}
\end{multline*}

%%
%
%%

\subsubsection{Abkürzung}

$A$ kommt auf keiner rechten Seite der FD’s vor.
Man kann es über FD's nicht erreichen. $A$ muss also Teil des
Schlüsselkandidaten sein.

\begin{equation*}
\textit{AttrHüll}(F, \{A\}) = R = \textit{Superschlüssel}
\end{equation*}

A ist minimal Schlüsselkandidat.
%
Jeder weitere Schlüsselkandidat muss ebenfalls minimal sein und zudem
A enthalten $\rightarrow$ A einziger Schlüsselkandidat.

%%
%
%%

\subsubsection{Mit Hilfe des Algorithmus:}

$Test = \{\{A, B, C, D\}\}$ $Erg = \{\}$

\begin{enumerate}

%%
% 1.
%%

\item $K = \{A, B, C, D\}$

$K \SLASH A: \textit{AttrHüll}(F, \{B, C, D\}) = \{B, C, D\}$ !\\

$K \SLASH B: \textit{AttrHüll}(F, \{A, C, D\}) = R$\\
$\rightarrow Test = \{\{A, C, D\}\}$ \\

$K \SLASH C: \textit{AttrHüll}(F, \{A, B, D\}) = R$\\
$\rightarrow Test = \{\{A, C, D\}, \{A, B, D\}\}$ \\

$K \SLASH D: \textit{AttrHüll}(F, \{A, B, C\}) = R$\\
$\rightarrow Test = \{\{A, C, D\}, \{A, B, D\}, \{A, B, C\}\}$ \\

%%
% 2.
%%

\item $K = \{A, C, D\}$

$K \SLASH A: \textit{AttrHüll}(F, \{C, D\}) = \{C, D\}$ !\\

$K \SLASH C: \textit{AttrHüll}(F, \{A, D\}) = R$\\
$\rightarrow Test = \{\{A, D\}, \{A, C, D\}, \{A, B, D\}, \{A, B, C\}\}$ \\

$K \SLASH C: \textit{AttrHüll}(F, \{A, C\}) = R$\\
$\rightarrow Test = \{\{A, C\}, \{A, D\}, \{A, C, D\}, \{A, B, D\}, \{A, B, C\}\}$ \\

%%
% 3.
%%

\item $K = \{A, C\}$

$K \SLASH A: \textit{AttrHüll}(F, \{C\}) = \{C\}$ !\\

$K \SLASH C: \textit{AttrHüll}(F, \{A\}) = R$\\
$\rightarrow Test = \{\{A\}, \{A, D\}, \{A, C, D\}, \{A, B, D\}, \{A, B, C\}\}$ \\

%%
% 4.
%%

\item $K = \{A\}$

$K \SLASH A$: ! $\rightarrow$ kein Superschlüssel ohne A mehr möglich\\
$\rightarrow$ dieses K wandert in Ergebnis und wird in Test gelöscht

$\rightarrow Test = \{\{A, D\}, \{A, C, D\}, \{A, B, D\}, \{A, B, C\}\}$ \\
$\rightarrow Erg = \{\{A\}\}$ \\
\end{enumerate}

analog verfahren wir mit den übrigen Mengen in Test, wie man bereits
sieht bleibt $\{A\}$ einziger Schlüsselkandidat.

\literatur

\end{document}
