\documentclass{lehramt-informatik}
\usepackage{amsmath}

\begin{document}

\chapter{Synthesealgorithmus}

\begin{lernkartei}{Synthesealgorithmus zur 3NF}

\begin{compactenum}
\item Reduktion (kanonische Überdeckung)

\begin{compactenum}
\item Linksreduktion:
$B \subseteq \textit{AttrHülle}(F, \alpha - A)$

\item Rechsreduktion:
$B \in \textit{AttrHülle}(F - (\alpha \rightarrow \beta) \cup (\alpha \rightarrow (\beta - B)), \alpha)$

\item Leere Klauseln streichen:
$\alpha \rightarrow \emptyset$

\item Vereinigung:
$\alpha \rightarrow \beta_1,...,\alpha \rightarrow \beta_n  = \alpha \rightarrow (\beta_1 \cup ... \cup \beta_n)$
\end{compactenum}

\item Neues Relationenschema
\item Hinzufügen einer Relation
\item Entfernen überflüssiger Teilschemata
\end{compactenum}
\end{lernkartei}

%-----------------------------------------------------------------------
%
%-----------------------------------------------------------------------

\section{Synthesealgorithmus (kleines Beispie aus Kemper)}

\cite[Seite 186]{kemper}

$F = \{A \rightarrow B, B \rightarrow C, AB \rightarrow C\}$

\begin{compactitem}
\item $A \rightarrow B$
\item $B \rightarrow C$
\item $AB \rightarrow C$
\end{compactitem}

\begin{enumerate}
\item Kanonische Überdeckung

\begin{enumerate}
\item Linksreduktion

AttrH(F, A) = {A, B, C}

\begin{compactitem}
\item $A \rightarrow B$
\item $B \rightarrow C$
\item $A \rightarrow C$
\end{compactitem}

\item Rechtsreduktion

\begin{compactitem}
\item $A \rightarrow B$
\item $B \rightarrow C$
\item $A \rightarrow \emptyset$
\end{compactitem}

\item Leere Klauseln leeren

\begin{compactitem}
\item $A \rightarrow B$
\item $B \rightarrow C$
\end{compactitem}

\item Vereinigung

\item Neues Relationenschema

\begin{compactitem}
\item $R1(A, B)$
\item $R2(B, C)$
\end{compactitem}

\item Hinzufügen einer Relation
\item Entfernen überflüssiger Teilschemata

\end{enumerate}
\end{enumerate}

%-----------------------------------------------------------------------
%
%-----------------------------------------------------------------------

\section{Staatsexamen Frühjahr 1994 - Aufgabe 7}

\cite[Seite 1, Aufgabe 2]{db:ab:6}
\cite[Seite 4, Aufgabe 7]{examen:66111:1994:03}

Betrachten Sie das relationale Schema

$R(Signatur, Titel, Fachgebiet, Art, ErschOrt, MatrNr, StudName, Gebdatum,
StudWohnort, StudFachrichtung, AutNr, AutName, AutWohnort, AutBuchHonorar)$

und die Menge

\begin{multline}
F = \{\\
Signatur \rightarrow Titel, Fachgebiet, Art, ErschOrt, \\
Signatur \rightarrow MatrNr, \\
MatrNr \rightarrow StudName, Gebdatum, StudWohnort, StudFachrichtung,\\
AutNr \rightarrow AutName, AutWohnort, \\
AutNr, Signatur \rightarrow AutBuchHonorar \\
\}
\end{multline}

Geben Sie eine abhängigkeitserhaltende und verlustfreie Zerlegung von R
in 3. Normalform an!

\begin{antwort}

\begin{enumerate}
\item Linksreduktion

$AttrHull(F, \{Autnr\}) = \{ Autnr, AutName, AutWohnort \}$

$AttrHull(F, \{Signatur\}) = \{ Signatur, Titel, Fachgebiet, Art, ErschOrt, MatrNr, StudName, Gebdatum, StudWohnort, StudFachrichtung, \}$

\item Rechtsreduktion

$AttrHull(F - \{Signatur \rightarrow MatrNr\}, \{Signatur\}) = \{ Signatur, Titel, Fachgebiet, Art, ErschOrt \}$ Es kann nichts weggelassen werden

\end{enumerate}

\end{antwort}

%-----------------------------------------------------------------------
%
%-----------------------------------------------------------------------

\section{Normalformen Einstieg
\footcite[Seite 1, Aufgabe 1: Normalformen Einstieg]{db:pu:4}
}

Es seien folgende Relationenschemata mit den jeweiligen Mengen
funktionaler Abhängigkeiten gegeben:

$S_1 \{P, Q, R\}$ mit
$F_1 = \{P Q \rightarrow R, P R \rightarrow Q, QR \rightarrow P \}$

$S_2 \{P, R, S, T \}$ mit
$F_2 = \{P S \rightarrow T \}$

$S_3 \{P, S, U \}$ mit
$F_3 = \{\}$

\begin{enumerate}

%%
% (a)
%%

\item Welche der drei Schemata sind in BCNF, welche in 3NF, welche in
2NF? Begründe!

\begin{antwort}
\begin{itemize}
\item $S_1$: BCNF

\item $S_2$: 1NF aber nicht 2NF

\item $S_3$: BCNF
\end{itemize}
\end{antwort}

%%
% (b)
%%

\item Wenden Sie auf ($S_2$, $F_2$) den Synthesealgorithmus an, und
bestimmen Sie auch die Mengen aller nichttrivialen einfachen
funktionalen Abhängigkeiten, die über den erhaltenen Teilrelationen
gelten. Ihr Lösungsweg muss nachvollziehbar sein.

\begin{antwort}
PS -> T (ist schon kanonische Überdeckung)

2. Schritt

R1 (P, S, T)

3. Schritt

R1 (P, S, T) mit F21 = {PS->}
R2 (P, S, R) mit F22 = {}

\end{antwort}

\end{enumerate}

%-----------------------------------------------------------------------
%
%-----------------------------------------------------------------------

\section{Synthesealgorithmus
\footcite[Seite 1, Aufgabe 2: Synthesealgorithmus]{db:pu:4}
}

Überführen Sie das Relationenschema mit Hilfe des Synthesealgorithmus in
die 3. Normalform!

$R(A, B, C, D, E, F, G, H)$

\begin{compactitem}
\item $F \rightarrow E$
\item $A \rightarrow B, D$
\item $AE \rightarrow D$
\item $A \rightarrow E, F$
\item $AG \rightarrow H$
\end{compactitem}

\begin{antwort}

\begin{enumerate}

\item Kanonische Überdeckung

\begin{enumerate}

%%
%
%%

\item 1. Linksreduktion:

Wir betrachten nur die zusammegesetzten Attribute:

\begin{itemize}
\item $AE \rightarrow D$:

$\textit{AttrHüll}(F, \{A\}) = \{A, E, F, B, \textbf{D}\}$ \\
$\textit{AttrHüll}(F, \{E\}) = \{E\}$

\item $AG \rightarrow H$:

$\textit{AttrHüll}(F, \{A\}) = \{A, E, F, B, D\}$ \\
$\textit{AttrHüll}(F, \{G\}) = \{G\}$
\end{itemize}

\textbf{FDs}

\begin{compactitem}
\item $F \rightarrow E$
\item $A \rightarrow B, D$
\item $A \rightarrow D$
\item $A \rightarrow E, F$
\item $AG \rightarrow H$
\end{compactitem}

%%
%
%%

\item 2. Rechtsreduktion:

Nur die Attribute betrachten, die rechts doppelt vorkommen:

$E$:

$\textit{AttrHüll}(F - \{F \rightarrow E\}, \{F\}) = \{F\}$ \\
$\textit{AttrHüll}(F - \{A \rightarrow E\}, \{A\}) = \{A, B, D, F, \textbf{E}\}$

$D$:

$\textit{AttrHüll}(F - \{A \rightarrow D\}, \{A\}) = \{A, B, \textbf{D}, F, E\}$

$A \rightarrow D$ kann wegen der Armstrongschen Dekompositionsregel
weggelassen werden. Wenn gilt $A \rightarrow B, D$, dann gilt auch $A
\rightarrow B$ und $A \rightarrow D$

\textbf{FDs}

\begin{compactitem}
\item $F \rightarrow E$
\item $A \rightarrow B, D$
\item $A \rightarrow \emptyset$
\item $A \rightarrow F$
\item $AG \rightarrow H$
\end{compactitem}

\item 3. Leere Klauseln streichen:

\begin{compactitem}
\item $F \rightarrow E$
\item $A \rightarrow B, D$
\item $A \rightarrow F$
\item $AG \rightarrow H$
\end{compactitem}

\item 4. Vereinigung

\begin{compactitem}
\item $F \rightarrow E$
\item $A \rightarrow B, D, F$
\item $AG \rightarrow H$
\end{compactitem}

Jetzt die weiteren Hauptschritte:

\end{enumerate}

\item Neues Relationenschema

\begin{compactitem}
\item $R1(F, E)$
\item $R2(A, B, D, F)$
\item $R3(A, G, H)$
\end{compactitem}

\item Hinzufügen einer Relation

Schlüsselkandidaten hinzufügen, falls nicht vorhanden:
$R4(A, C, G)$

\begin{compactitem}
\item $R1(F, E)$
\item $R2(A, B, D, F)$
\item $R3(A, G, H)$
\item $R4(A, C, G)$
\end{compactitem}

\item Entfernen überflüssiger Teilschemata

nichts zu tun

\end{enumerate}
\end{antwort}

%-----------------------------------------------------------------------
%
%-----------------------------------------------------------------------

\section{Zusatzaufgabe 1 (wird nicht in der Übung
besprochen)\footcite{db:pdf:tum:uebung-08}}

Betrachten Sie ein abstraktes Relationenschema $R = \{M, N, V, T, P,
PN\}$ mit den FDs

\begin{compactitem}
\item $M \rightarrow M$
\item $M \rightarrow N$
\item $V \rightarrow T, P, PN$
\item $P \rightarrow PN$
\end{compactitem}

\begin{enumerate}
\item Bestimmen Sie alle Kandidatenschlüssel.

\begin{antwort}
V kommt auf keiner rechten Seite der FDs vor.

$\text{AttrHuell}(R, \{V\}) = \{V, T, P, PN\} \neq  R$

$\text{AttrHuell}(R, \{V, M\}) = \{V, M, N, T, P, PN\} = R$

$\text{AttrHuell}(R, \{V, P\}) = \{V, P, T, PN\} \neq R$

$V, M$ ist Schlüsselkandidat
\end{antwort}

\item In welcher Normalform befindet sich die Relation?

\begin{antwort}
1NF weil nichtprimäre Attribute von einer echten Teilmenge des
Schlüsselkandidaten abhängen (z. B. $M \rightarrow N$).
\end{antwort}

\item Bestimmen Sie zu den gebenen FDs die kanonische Überdeckung.

\begin{antwort}

\begin{enumerate}
\item Linkreduktion bleibt aus

\item Rechtsreduktion: PN ist doppelt

$\text{AttrHuell}(R - (V \rightarrow T, P, PN) \cup (V \rightarrow T, P), \{V\}) = \{V, T, P, PN\}$

\begin{compactitem}
\item $M \rightarrow M$
\item $M \rightarrow N$
\item $V \rightarrow T, P$
\item $P \rightarrow PN$
\end{compactitem}
\item Leere Klausel streichen

\item Vereinigung

\begin{compactitem}
\item $M \rightarrow N$
\item $V \rightarrow T, P$
\item $P \rightarrow PN$
\end{compactitem}

\end{enumerate}

\end{antwort}

\item Falls nötig, überführen Sie die Relation verlustfrei und
abhängigkeitsbewahrend in die dritte Normalform.
\end{enumerate}

\literatur

\end{document}
