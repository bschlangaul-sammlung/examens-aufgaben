\documentclass{lehramt-informatik}

\begin{document}

\chapter{Funktionale Abhängigkeiten}

%-----------------------------------------------------------------------
%
%-----------------------------------------------------------------------

%%
%
%%

\subsection{Definition (Funktionale Abhängigkeit)}

Sei $R\{A_1,...,A_n\}$ ein Relationenschema. Seien $X, Y \subseteq
\{A_1,...,A_n\}$ Teilmengen der Attributmenge von $R$. $Y$ ist genau
dann funktional abhängig von $X$, wenn gleiche Werte für die Attribute
in $X$ auch gleiche Werte für die Attribute in $Y$ erzwingen. Man
schreibt: $X \rightarrow Y$. $X \rightarrow Y$ heißt funktionale
Abhängigkeit, englisch \emph{Functional Dependency}, kurz \textbf{FD},
für das Relationenschema $R\{A_1,...,A_n\}$ Dabei heißt eine funktionale
Abhängigkeit $X \rightarrow Y$

\begin{itemize}
\item \textbf{einfach}, wenn $Y$ nur aus einem Attribut besteht.
\item \textbf{trivial}, wenn $Y \subseteq X$ gilt.
\end{itemize}

%%
%
%%

\subsection{Definition (Erfüllung funktionaler Abhängigkeiten)}

Sei $R\{A_1,...,A_n\}$ ein Relationenschema. Seien $X, Y \subseteq
R\{A_1,...,A_n\}$ Teilmengen der Attributmenge von R. Eine Instanz $R$
von $R\{A_1,...,A_n\}$ erfüllt die funktionale Abhängigkeit $X
\rightarrow Y$, wenn gilt: Stimmen zwei Tupel aus $R$ in den Werten der
Attribute aus $X$ überein, so stimmen sie auch in den Werten der
Attribute aus $Y$ überein. Man sagt, dass eine funktionale Abhängigkeit
$X \rightarrow Y$ für $R\{A_1,...,A_n\}$ \textbf{gilt}, wenn sie von
\textbf{jeder Instanz} von $R$ erfüllt wird. Eine Menge $F$ funktionaler
Abhängigkeiten gilt für $R\{A_1,...,A_n\}$, wenn jede Abhängigkeit aus
$F$ für $R\{A_1,...,A_n\}$ gilt.

%%
%
%%

\subsection{Definition (Logische Folgerbarkeit)}

Für jede Instanz des Relationenschemas, die die FD $PersNr \rightarrow
Name, Geschlecht, Wohnort, Geburtsjahr$ erfüllt, gilt sicher auch
$PersNr \rightarrow Name$. Man sagt, dass $PersNr \rightarrow Name$
logisch aus $F$ folgt.

%%
%
%%

\subsection{Definition(Hülle von F)}

Zu einer gegebenen Menge $F$ von FDs kann man die Menge aller aus $F$
logisch folgerbaren funktionalen Abhängigkeiten bestimmen. Diese Menge
heißt Hülle $F^+$.

%-----------------------------------------------------------------------
%
%-----------------------------------------------------------------------

\section{Armstrong-Kalkül}

\href{https://de.wikipedia.org/wiki/Funktionale_Abh%C3%A4ngigkeit#Axiome_von_Armstrong}{Wikipedia}
Foliensatz Seite 14

Mit Hilfe der Axiome von Armstrong (auch Armstrong-Axiome) lassen sich
aus einer Menge von funktionalen Abhängigkeiten, die auf einer Relation
gelten, \textbf{weitere} funktionale Abhängigkeiten \textbf{ableiten}.
Die folgenden drei Regeln reichen aus, um alle funktionalen
Abhängigkeiten herzuleiten:

\begin{lernkartei}{Armstrong-Kalkül}

Zweck: \textbf{weitere} funktionale Abhängigkeiten \textbf{ableiten}.

\begin{compactenum}
\item \textbf{Reflexivität}:
%
Eine Menge von Attributen bestimmt
eindeutig die Werte einer Teilmenge dieser Attribute (triviale
Abhängigkeit), das heißt,
$\beta \subseteq \alpha \Rightarrow \alpha \rightarrow \beta$ .

\item \textbf{Verstärkung}:
%
Gilt
$\alpha \rightarrow \beta$, so gilt auch
$\alpha \gamma \rightarrow \beta \gamma$
für jede Menge von Attributen
$\gamma$ der
Relation.

\item \textbf{Transitivität}:
%
Gilt
$\alpha \rightarrow \beta$
und
$\beta \rightarrow \gamma$,
so gilt auch
$\alpha \rightarrow \gamma$.
\end{compactenum}

\noindent

(abgeleiteten) Regeln

\begin{compactenum}
\item \textbf{Vereinigung}:
%
Gelten
$\alpha \rightarrow \beta$
$\alpha \rightarrow \gamma$ , so gilt auch
$\alpha \rightarrow \beta \gamma$.

\item \textbf{Dekomposition}:
%
Gilt
$\alpha \rightarrow \beta \gamma$,
so gelten auch
$\alpha \rightarrow \beta$ und
$\alpha \rightarrow \gamma$ .

\item \textbf{Pseudotransitivität}:
%
Gilt
$\alpha \rightarrow \beta$ und
$\beta \gamma \rightarrow \delta$,
so gilt auch
$\alpha \gamma \rightarrow \delta$
\end{compactenum}
\end{lernkartei}

\subsection{Beispiel}

Geg.: Menge $F = \{PersNr \rightarrow Name Wohnort, Fach \rightarrow
Pflichtfach\}$.

Dann sind (z.B.) folgende FD mit Hilfe der Armstrong-Axiome
ableitbar:

\begin{enumerate}
\item Axiom 1 (Reflexivität): $PersNr Fach \rightarrow Fach$

\item Axiom 2 (Verstärkung): Da $PersNr \rightarrow Name Wohnort$ und
${Fach} \subseteq {Fach}$ gilt, folgt die funktionale Abhängigkeit
$PersNr Fach \rightarrow Name Wohnort Fach$

\item Axiom 3 (Transitivität): Wegen $PersNr Fach \rightarrow Fach$ und
$Fach \rightarrow Pflichtfach$ folgt die funktionale Abhängigkeit
$PersNr Fach \rightarrow Pflichtfach$
\end{enumerate}

\literatur

\end{document}
