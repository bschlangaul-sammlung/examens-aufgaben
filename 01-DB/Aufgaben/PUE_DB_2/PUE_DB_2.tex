\documentclass{lehramt-informatik-haupt}
\usepackage{amsmath}
\begin{document}

\chapter{Präsenzübung 2: SQL\footcite{db:pu:2}}

%-----------------------------------------------------------------------
%
%-----------------------------------------------------------------------

\section{Aufgabe 1: SQL Einstieg\footcite{db:pu:2}}

Gegeben ist folgende Datenbank: \bigskip

{
\ttfamily
\noindent
Album (Titel, Typ, Firma, Preis, \underline{ANr})\\
\noindent
herausgegeben (\underline{BName}, \underline{ANr}, Jahr)\\
\noindent
Band (\underline{BName}, Musikrichtung, Gruendungsjahr, Aktiv)\\
\noindent
Musiker (Vorname, Name, GebJahr, BName, \underline{ID})\\
}

\noindent
Beantworten Sie folgende Fragen durch geeignete SQL-Anfragen.

\renewcommand{\labelenumi}{(\alph{enumi})}
\begin{enumerate}

%%
% (a)
%%

\item Welche Alben wurden von der Firma „Col“ herausgegeben?

\begin{antwort}[muster]
\begin{minted}{sql}
SELECT a.Titel
FROM Album a
WHERE a.Firma = 'Col';
\end{minted}
\end{antwort}

%%
% (b)
%%

\item Welche Alben wurden 1990 von \emph{„Black Sabbath“}
veröffentlicht?

\begin{antwort}[muster]
\begin{minted}{sql}
SELECT a.Titel
FROM ALBUM a, herausgegeben h
WHERE
  a.ANr = h.ANr AND
  h.BName = 'Black Sabbath' AND
  h.Jahr = 1990;
\end{minted}
\end{antwort}

%%
% (c)
%%

\item Welche Band veröffentlichte das Album \emph{„Sin After Sin“}?

\begin{antwort}[muster]
\begin{minted}{sql}
SELECT h.BName
FROM herausgegeben h, Album a
WHERE
  a.ANr = h.ANr AND
  a.Titel = 'Sin After Sin';
\end{minted}
\end{antwort}

%%
% (d)
%%

\item In welcher Band spielt „Ozzy Osbourne“?

\begin{antwort}[muster]
\begin{minted}{sql}
SELECT BName
FROM Musiker
WHERE
  Name = 'Osbourne' AND
  Vorname = 'Ozzy';
\end{minted}
\end{antwort}

%%
% (e)
%%

\item Welche Bands wurden vor \emph{1980} gegründet, spielen
\emph{Hardrock} und sind nicht bei \emph{Col} unter Vertrag?

\begin{antwort}[muster]
Joins sind teuer: zuerst Bedingungen

\begin{minted}{sql}
SELECT DISTINCT b.BName
FROM Band b, herausgegeben h, Album a
WHERE
  b.Musikrichtung = 'Hardrock' AND
  a.Firma != 'Col' AND
  b.Gruendungsjahr < 1980 AND
  b.BName = h.BName AND
  h.ANr = a.ANr;
\end{minted}
\end{antwort}

%%
% (f)
%%

\item Wie viele Alben mit einem Preis von unter 10€ sind
gelistet?

\begin{antwort}[muster]
Nicht \verb|COUNT(a.Titel)|: doppelte werden nur 1 mal gezählt

\begin{minted}{sql}
SELECT COUNT(*) AS Anzahl
FROM Album a
WHERE a.Preis < 10;
\end{minted}
\end{antwort}

%%
% (g)
%%

\item Welche Musiker spielen in einer Hardrock Band
(alphabetisch nach Name)?

\begin{antwort}[muster]
\verb|DISTINCT|: keine Duplicate. \verb|DISTINCT| ist \verb|GROUP BY|
vorzuziehen

\begin{minted}{sql}
SELECT DISTINCT m.Name, m.Vorname
FROM Musiker m, Band b
WHERE
  m.BName = b.BName AND
  b.Musikrichtung = 'Hardrock'
ORDER BY m.Name, m.Vorname;
\end{minted}
\end{antwort}

%%
% (h)
%%

\item Wie viele Alben hat jede Band veröffentlicht (Bandname,
Anzahl der Alben)?

\begin{antwort}[muster]
\begin{minted}{sql}
SELECT BName, COUNT(*) AS AnzahlAlben
FROM herausgegeben
GROUP by BName;
\end{minted}
\end{antwort}

%%
% (i)
%%

\item Gib alle verschiedenen \emph{Namen} der \emph{Musiker} aufsteigend
sortiert aus, die in \emph{aktiven} Bands spielen.

\begin{antwort}[muster]
\begin{minted}{sql}
SELECT DISTINCT m.Name, m.Vorname
FROM Band b, Musiker m
WHERE
  b.BName = m.BName AND
  b.aktiv = 1
ORDER By m.Name, m.Vorname ASC;
\end{minted}
\end{antwort}

%%
% (j)
%%

\item Welche Musiker spielen in einer Band, die keine Alben vor
\emph{1970} veröffentlicht hat?

\begin{antwort}[muster]
\begin{minted}{sql}
SELECT DISTINCT m.Name
FROM Musiker m
WHERE
  m.BName NOT IN (SELECT BName FROM herausgegeben WHERE Jahr < 1970);
\end{minted}
\end{antwort}

%%
% k)
%%

\item Welche \emph{Musiker} spielen in einer Band, in der es
mindestens ein \emph{jüngeres} Bandmitglied gibt?

\begin{antwort}[muster]
\begin{minted}{sql}
SELECT DISTINCT a.Vorname, a.Name
FROM Musiker a, Musiker b
WHERE a.BName = b.BName
AND a.GebJahr < b.GebJahr;
\end{minted}
\end{antwort}

\end{enumerate}

%-----------------------------------------------------------------------
%
%-----------------------------------------------------------------------

\section{Aufgabe 2: SQL\footcite{db:pu:2}}

Gegeben seien die folgenden drei Relationen. Diese Relationen erfassen
die Mitarbeiterverwaltung eines Unternehmens. Schlüssel sind fett
dargestellt und Fremd- schlüssel sind kursiv dargestellt. So werden
Mitarbeiter, Abteilungen und Unternehmen jeweils durch ihre Nummer
identifiziert. AbtNr ist die Nummer der Abteilung, in der ein
Mitarbeiter arbeitet. Manager ist die Nummer des Mitarbeiters, der die
Abteilung leitet. \verb|UntNr| ist die Nummer des Unternehmens, dem eine
Abteilung zugeordnet ist.

\begin{minted}{md}
Mitarbeiter(Nummer, Name, Alter, Gehalt, AbtNr)
Abteilung(Nummer, Name, Budget, Manager, UntNr)
Unternehmen(Nummer, Name, Adresse)
\end{minted}
\begin{enumerate}

% a)
\item Wie hoch ist das Durchschnittsalter der Abteilung
„Personal Care“ im Unternehmen „Test.com“?

\begin{antwort}[muster]
\verb|GROUP BY| nicht nötig, \verb|AS| nicht vergessen.

\begin{minted}{sql}
SELECT AVG(m.Alter) AS Durchschnittsalter
FROM Unternehmen u, Abteiltung a, Mitarbeiter m
WHERE
  a.Name = 'Personal Care' AND
  u.Name = 'Test.com' AND
  u.Nummer = a.UntNr AND
  m.AbtNr = a.Nummer;
\end{minted}
\end{antwort}

%%
% b)
%%

\item Geben Sie für jedes Unternehmen das Durchschnittsalter
der Mitarbeiter an!

\begin{antwort}[muster]

Statt \verb|a.UntNr| kann \verb|u.Nummer| verwendet werden.
\verb|a.UntNr| nur deshalb, weil man dann eventuell den Join über die
Unternehmenstabelle sparen kann.

Alles was ausgegeben werden soll, muss auch in \verb|GROUP BY| enthalten
sein.

\begin{minted}{sql}
SELECT a.UntNr, u.Name, AVG(m.Alter) as Durchschnittsalter
FROM Unternehmen u, Abteiltung a, Mitarbeiter m
WHERE
  u.Nummer = a.UntNr AND
  m.AbtNr = a.Nummer
GROUP BY a.UntNr, u.Name;
\end{minted}
\end{antwort}

%%
% c)
%%

\item Wie viele Mitarbeiter im Unternehmen „Test.com“ sind
älter als ihr Chef? (D.h. sind älter als der Manager der Abteilung, in
der sie arbeiten.)

\begin{antwort}[muster]
\begin{minted}{sql}
SELECT COUNT(*)
FROM Mitarbeiter m, Abteilung a, Unternehmer u
WHERE
  m.AbtNr = a.Nummer AND
  a.UntNr = u.Nummer AND
  u.Name = 'Test.com'
AND m.Alter > (
  SELECT ma.Alter
  FROM Mitarbeiter ma, Abteilung ab
  WHERE
    ma.Nummer = ab.Manager AND
    a.Nummer = ab.Nummer
);
\end{minted}

oder einfacher:

\begin{minted}{sql}
SELECT COUNT(*)
FROM Mitarbeiter m, Abteilung a, Unternehmer u
WHERE
  m.AbtNr = a.Nummer AND
  a.UntNr = u.Nummer AND
  u.Name = 'Test.com'
AND m.Alter > (
  SELECT ma.Alter
  FROM Mitarbeiter ma
  WHERE ma.Nummer = a.Manager
);
\end{minted}

Alternativ Lösung ohne Unterabfragem, mit Self join:

\begin{minted}{sql}
SELECT  COUNT(*)
FROM Mitarbeiter m, Abteilung a, Unternehmer u, Mitarbeiter m2
WHERE
  m.AbtNr = a.Nummer AND
  a.UntNr = u.Nummer AND
  u.Name = 'Test.com' AND
  a.Manager = m2.Nummer AND
  m.Alter > m2.Atler;
\end{minted}
\end{antwort}

%%
% (d)
%%

\item Welche Abteilungen haben ein geringeres Budget als die
Summe der Gehälter der Mitarbeiter, die in der Abteilung arbeiten?

\begin{antwort}[muster]
\begin{minted}{sql}
SELECT a.Name, a.Nummer
FROM Abteilung a
WHERE a.Budget < (
  SELECT SUM(m.Gehalt)
  FROM Mitarbeiter m
  WHERE a.Nummer = m.AbtNr
);
\end{minted}

Ohne Unterabfrage

\begin{minted}{sql}
SELECT a.Name, a.Nummer
FROM Abteilung a, Mitarbeiter m
WHERE a.Nummer = m.AbtNr
GROUP BY a.Nummer, a.Name, a.Budget
HAVING a.Budget < SUM(m.Gehalt);
\end{minted}
\end{antwort}

%%
% (e)
%%

\item Versetzen Sie den Mitarbeiter „Wagner“ in die Abteilung „Personal
Care“!

\begin{antwort}[muster]
\begin{minted}{sql}
UPDATE Mitarbeiter m
SET AbtNr = (
  SELECT a.Nummer FROM
  Abteilung a
  WHERE a.Name = 'Personal Care'
)
WHERE m.Name = 'Wagner';
\end{minted}
\end{antwort}

%%
% (f)
%%

\item Löschen Sie die Abteilung „Personal Care“ mit allen ihren
Mitarbeitern!

\begin{antwort}[muster]
\begin{minted}{sql}
DELETE FROM Mitarbeiter
WHERE AbtNr = (
  SELECT a.Nummer
  FROM Abteilung a
  WHERE a.Name = 'Personal Care'
);

DELETE FROM Abteilung
WHERE Name = 'Personal Care';
\end{minted}
\end{antwort}

%%
% (g)
%%

\item Geben Sie den Managern aller Abteilungen, die ihr Budget nicht
überziehen, eine 10 Prozent Gehaltserhöhung. (Das Budget ist überzogen,
wenn die Gehälter der Mitarbeiter höher sind als das Budget der
Abteilung.) Zusatzfrage: Was passiert mit Mitarbeitern, die Manager von
mehreren Abteilungen sind?

\begin{antwort}[muster]
\begin{minted}{sql}
CREATE VIEW LowBudget AS (
  SELECT Nummer
  FROM Abteilung
  WHERE Nummer NOT IN (
    SELECT a.Nummer
    FROM Abteilung a
    WHERE
      a.Budget < (
        SELECT SUM(Gehalt)
        FROM Mitarbeiter m Abteilung A
        WHERE m.AbtNr = A.Nummer AND
        a.Nummer = A.Nummer
    )
  )
)

UPDATE Mitarbeiter
SET Gehalt = 1.1 * Gehalt
WHERE Nummer IN (
  SELECT Manager
  FROM LowBudget, Abteilung
  WHERE LowBudget.Manager = Abteilung.Nummer
)
\end{minted}
\end{antwort}

\end{enumerate}

%-----------------------------------------------------------------------
%
%-----------------------------------------------------------------------

\section{Aufgabe 3: SQL\footcite{db:pu:2}}

Gegeben sei folgendes relationales Schema, das eine
Universitätsverwaltung modelliert:

\begin{minted}{md}
Studenten {[MatrNr:integer, Name:string, Semester:integer]}
Vorlesungen {[VorlNr:integer, Titel:string, SWS:integer, gelesenVon:integer]}
Professoren {[PersNr:integer, Name:string, Rang:string, Raum:integer] }
hoeren {[MatrNr:integer, VorlNr:integer]}
voraussetzen {[VorgaengerVorlNr:integer, NachfolgerVorlNr:integer]}
pruefen {[MatrNr:integer, VorlNr:integer, PrueferPersNr:integer, Note:decimal]}
\end{minted}

\noindent
Formulieren Sie die folgenden Anfragen in SQL:

\begin{enumerate}

%%
% (a)
%%

\item Alle Studenten, die den Professor \emph{Kant} aus einer Vorlesung
kennen.

\begin{antwort}[muster]
\begin{minted}{sql}
SELECT DISTINCT s.Name
FROM Studenten s, Professoren p, hoeren h, Vorlesungen v
WHERE
  p.Name = 'Kant' AND
  v.gelesenVon = p.PersNr AND
  s.MatrNr = h.MatrNr AND
  h.VorlNr = h.VorlNr
\end{minted}
\end{antwort}

%%
% (b)
%%

\item Geben Sie eine Liste der Professoren (Name, PersNr) mit ihrem
Lehrdeputat (Summe der SWS der gelesenen Vorlesungen) aus. Ordnen Sie
diese Liste so, dass sie absteigend nach Lehrdeputat sortiert ist! Bei
gleicher Lehrtätigkeit dann noch aufsteigend nach dem Namen des
Professors/der Professorin.

\begin{antwort}[muster]
\begin{minted}{sql}
SELECT p.Name, p.PersNr, SUM(v.SWS) AS Lehrdeputat
FROM Vorlesung v, Professoren p
WHERE v.gelesenVon = p.PersNr
GROUP BY p.Name, p.PersNr
ORDER BY Lehrdeputat DESC, p.Name ASC;
\end{minted}
\end{antwort}

%%
% (c)
%%

\item Geben Sie eine Liste der Studenten (Name, MatrNr, Semester) aus,
die mindestens zwei Vorlesungen bei \emph{Kant} gehört haben.

\begin{antwort}[muster]
\subsection*{Mit einer VIEW}

\begin{minted}{sql}
CREATE VIEW hoertKant AS
  SELECT s.Name, s.MatrNr, s.Semester, v.VorlNr
  FROM Studenten s, hoeren h, Vorlesungen v, Professoren p
  WHERE
    s.MatrNr = h.MatrNr AND
    h.VorlNr = v.VorlNr AND
    v.gelesenVon = p.PersNr AND
    p.Name = 'Kant';
\end{minted}

\begin{minted}{sql}
SELECT DISTINCT h1.Name, h2.MatrNr, h1.Semester
FROM hoertKant h1, hoertKant h2
WHERE h1.MatrNr = h2.MatrNr AND h1.VorlNr <> h2.VorlNr;
\end{minted}

oder:

\begin{minted}{sql}
SELECT DISTINCT Name, MatrNr, Semester
FROM hoertKant
GROUP BY Name, MatrNr, Semester
HAVING COUNT(VorlNr) > 1;
\end{minted}

\subsection*{In einer Abfrage}

\begin{minted}{sql}
SELECT s.Name, s.MatrNr, s.Semester
FROM Studenten s, hoeren h, Vorlesungen v, Professoren p
WHERE
  s.MatrNr = h.MatrNr AND
  h.VorlNr = v.VorlNr AND
  v.gelesenVon = p.PersNr AND
  p.Name = 'Kant'
GROUP BY s.MatrNr, s.Name, s.Semster
HAVING COUNT(s.MatrNr) > 1;
\end{minted}
\end{antwort}

%%
% (d)
%%

\item Geben Sie eine Liste der Semesterbesten (MatrNr und
Notendurchschnitt) aus.

\begin{antwort}[muster]
\begin{minted}{sql}
CREATE VIEW Notenschnitte AS (
  SELECT p.MatrNr, s.Name, s.Semester, AVG(Note) AS Durchschnitt
  FROM Studenten s, pruefen p
  WHERE s.MatrNr = p.MatrNr
  GROUP BY p.MatrNr, s.Name, s.Semester
);

SELECT a.Durchschnitt, a.MatrNr, a.Semester
FROM Notenschnitte a, Notenschnitte b
WHERE
  a.Durchschnitt >= b.Durchschnitt
  a.Semster = b.Semster
GROUP BY a.Durchschnitt, a.MatrNr, a.Semester
HAVING COUNT(*) < 2;
\end{minted}
\end{antwort}

\end{enumerate}

%-----------------------------------------------------------------------
%
%-----------------------------------------------------------------------

\section{Aufgabe 4: Relationale Algebra Einstieg\footcite{db:pu:2}}

Gegeben ist folgende Datenbank-Anfrage:

\begin{math}
\pi_{\text{Bezeichnung}}(
  \sigma_{\text{SWS} = 2 \land\neg(\text{Name} = \mlq \text{Wassned} \mrq)}(
    \text{Vorlesung} \bowtie \text{Professor}
  )
)
\end{math}

\begin{enumerate}

%%
% (a)
%%

\item Geben Sie eine umgangssprachliche Formulierung der Anfrage an!

\begin{antwort}[muster]
Eine Liste mit den Bezeichnungen der Vorlesungen, die 2 Semester
Wochenstunden dauern und die nicht vom dem Professor „Wassned“ gelesen
werden.
\end{antwort}

%%
% (b)
%%

\item Geben Sie die Ergebnistabelle an!

\begin{antwort}[muster]

\begin{tabular}{|l|}
\hline
\textbf{Bezeichnung}\\
Japanische Malerei\\
Chinesische Schrift\\
\hline
\end{tabular}

\end{antwort}

\end{enumerate}

%-----------------------------------------------------------------------
%
%-----------------------------------------------------------------------

\section{Aufgabe 5: Relationale Algebra\footcite{db:pu:2}}

\begin{minted}{md}
Studenten {[MatrNr:integer, Name:string, Semester:integer]}
Vorlesungen {[VorlNr:integer, Titel:string, SWS:integer, gelesenVon:integer] }
Professoren {[PersNr:integer, Name:string, Rang:string, Raum:integer] }
hoeren {[MatrNr:integer, VorlNr:integer]}
voraussetzen {[VorgaengerVorlNr:integer, NachfolgerVorlNr:integer]}
pruefen {[MatrNr:integer, VorlNr:integer, PrueferPersNr:integer, Note:decimal]}
\end{minted}
\begin{enumerate}

%%
% (a)
%%

\item Geben Sie verbal an, welches Ergebnis folgende SQL-Anfrage
liefert:

\begin{antwort}[muster]
Liste mit zwei unterschiedliche Studenten, die in derselben
Vorlesung waren.
\end{antwort}

%
\item Geben Sie einen Relationenalgebra-Ausdruck für diese Anfrage an.
Dieser Ausdruck sollte keine Kreuzprodukte (nur Joins) enthalten.

\begin{antwort}[muster]
\begin{multline*}
\pi_{s_1.\text{Name},s_2.\text{Name}}\\
(\\
  (\rho_{s_1} (\text{Studenten}) \bowtie \rho_{h_1} (\text{hoeren}))\\
  \bowtie_{(h_1.\text{VorlNr} = h_2.\text{VorlNr} \land s_1.\text{MatrNr} <> s_2.\text{MatrNr})}\\
  (\rho_{s_2} (\text{Studenten}) \bowtie \rho_{h_2} (\text{hoeren}))\\
)
\end{multline*}
\end{antwort}

\end{enumerate}
\end{document}
