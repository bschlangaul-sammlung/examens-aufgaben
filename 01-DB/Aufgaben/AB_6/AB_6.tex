\documentclass{lehramt-informatik}
\usepackage{amsmath}
\renewcommand{\labelenumi}{(\alph{enumi})}
\begin{document}

\chapter{Aufgabenblatt 6: Normalformen \& Transaktionsmanagement}

%-----------------------------------------------------------------------
%
%-----------------------------------------------------------------------

\section{Aufgabe 3}

Die Transaktionen eines Transaktionsprogramms besteht aus SQL-Befehlen.
Die Transaktionen T1 und T2 arbeiten auf der Tabelle TAB.

\begin{tabular}{l}
Transaktion T1        \\
Bot                   \\
SELECT FROM TAB       \\
NEUF:-F+5             \\
UPDATE TAB SET F=NEUF \\
COMMIT WORK
\end{tabular}

\begin{tabular}{l}
Transaktion T2        \\
Bot                   \\
SELECT FROM TAB       \\
NEUF:-F*2             \\
UPDATE TAB SET F=NEUF \\
COMMIT WORK
\end{tabular}

Die quasiparallele Abarbeitung erfolgt in folgenden Schritten:

\begin{tabular}{lll}
Schritt & T1                    & T2                    \\
1       &                       & Bot                   \\
2       & BOT                   &                       \\
3       & SELCT F FROM TAB      &                       \\
4       &                       & SELECT F FROM TAB     \\
5       &                       & NEUF := F*2           \\
6       & NEUF := F+5           &                       \\
7       & UPDATE TAB SET F=NEUF &                       \\
8       & COMMIT WORK           &                       \\
9       &                       & UPDATE TAB SET F=NEUF \\
10      &                       & COMMIT WORK
\end{tabular}

\begin{enumerate}

%%
%
%%

\item Ist die (quasiparallele) Bearbeitung der Transaktionen korrekt?
Begründung!

%%
%
%%

\item Konstruieren Sie unter Verwendung von T1 und T2 einen
Dirty-Read-Fehlerfall.
\end{enumerate}

%-----------------------------------------------------------------------
%
%-----------------------------------------------------------------------

\section{Aufgabe 4: ACID}

Beurteilen Sie kurz folgende Aussagen oder Fragen unter
ACID-Gesichtspunkten!

\begin{enumerate}

%%
% (a)
%%

\item Seit dem Abort meiner Transaktion sind deren Änderungen überhaupt
nicht mehr vorhanden!

%%
% (b)
%%

\item Leider wurde die erfolgreich abgeschlossene Transaktion
zurückgesetzt, da das DBS abgestürzt ist.

%%
% (c)
%%

\item Eine andere Transaktion hat Änderungen meiner Transaktion
überschrieben. Darf ich jetzt meine Transaktion überhaupt noch beenden
oder muss ich sie abbrechen?
\end{enumerate}

%-----------------------------------------------------------------------
%
%-----------------------------------------------------------------------

\documentclass{lehramt-informatik-minimal}
\InformatikPakete{graph,mathe}
\begin{document}

\section{Städte gemischt gerichtet / ungerichtet
\index{Algorithmus von Dijkstra}
\footcite[Thema 1 Aufgabe 5]{examen:66112:2004:03}
}

Ein wichtiges Problem im Bereich der Graphalgorithmen ist die Berechnung
kürzester Wege. Gegeben sei der folgende Graph, in dem Städte durch
Kanten verbunden sind. Die Kantengewichte geben Fahrzeiten an. Außer den
durch Pfeile als nur in eine Richtung befahrbar gekennzeichneten Straßen
sind alle Straßen in beiden Richtungen befahrbar.
\footcite{aud:pu:6}

\graph knoten {
  \knoten{A}(2,5)
  \knoten{B}(3,3)
  \knoten{C}(0,3)
  \knoten{D}(5,1)
  \knoten{E}(4,5)
  \knoten{F}(7,1)
  \knoten{G}(1,0)
  \knoten{H}(6,4)
} kanten {
  \kante(A-B){$10$}
  \kante(A-E){$40$}
  \kante(B-G){$20$}
  \kante(B-C){$50$}
  \kante(B-D){$90$}
  \kante(B-H){$90$}
  \kante(C-G){$20$}
  \kante(D-G){$75$}
  \kante(D-F){$80$}
  \kante(E-H){$5$}
  \kante(F-H){$10$}
  \kanteR(A>C){$70$}
  \kanteR(H>D){$10$}
}

\begin{enumerate}

%%
% a)
%%

\item Geben Sie zu dem obigen Graphen zunächst eine Darstellung als
Adjazenzmatix an.

\[
\begin{blockarray}{ccccccccc}
  & A  & B  & C  & D  & E  & F  & G  & H  \\
\begin{block}{c(cccccccc)}
A & 0  & 10 & 70 & 0  & 40 & 0  & 0  & 0  \\
B & 10 & 0  & 50 & 90 & 0  & 0  & 20 & 90 \\
C & 0  & 50 & 0  & 0  & 0  & 0  & 20 & 0  \\
D & 0  & 90 & 0  & 0  & 0  & 80 & 75 & 0  \\
E & 40 & 0  & 0  & 0  & 0  & 0  & 0  & 5  \\
F & 0  & 0  & 0  & 80 & 0  & 0  & 0  & 10 \\
G & 0  & 20 & 20 & 75 & 0  & 0  & 0  & 0  \\
H & 0  & 90 & 0  & 10 & 5  & 10 & 0  & 0  \\
\end{block}
\end{blockarray}
\]

%%
% b)
%%

\item Berechnen Sie nun mit Hilfe des Algorithmus von Dijkstra die
kürzesten Wege vom Knoten A zu allen anderen Knoten.
\end{enumerate}

\end{document}


%-----------------------------------------------------------------------
%
%-----------------------------------------------------------------------

\documentclass{lehramt-informatik-aufgabe}
\liLadePakete{normalformen}
\begin{document}
\liAufgabenTitel{Synthese-Algorithmus bei Relationenschema A-F}

\section{Aufgabe 6: Normalformen\footcite{examen:66116:2018:03}}

Gegeben sei das Relationenschema R(A,B,C,D,E,F), sowie die Menge der
zugehörigen funktionalen Abhängigkeiten F' \footcite{db:pu:wh}

\begin{itemize}
\item \liFA C > B
\item \liFA B > A
\item \liFA CE > D
\item \liFA E > F
\item \liFA CE > F
\item \liFA C > A
\end{itemize}

\begin{enumerate}
\item Bestimmen Sie den Schlüsselkandidaten der Relation R und begründen
Sie, warum es keine weiteren Schlüsselkandidaten gibt.

\begin{liAntwort}
 C und E müssen immer Teil des Schlüsselkandidaten
AttrHull(F, \{C, E\} = \{C, E, B, A, D, F\})

-> Superschlüssel
-> Schlüsselkandidat, weil minimal denn C und E müssen immer Teil sein.
-> kein anderer SK möglich, weil C und E immer Teil sein müssen.

Sie selbst aber schon minimal sind.
\end{liAntwort}

\item Überführen Sie das Relationenschema R mit Hilfe des
Synthesealgorithmus\index{Synthese-Algorithmus} in die dritte
Normalform\index{Dritte Normalform}. Führen Sie hierfür jeden der vier
Schritte durch und kennzeichnen Sie Stellen, bei denen nichts zu tun
ist.
\end{enumerate}

\end{document}


\end{document}
