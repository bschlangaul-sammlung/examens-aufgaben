\documentclass{lehramt-informatik}

\begin{document}

\section{Aufgabensammlung zur Bundesliga-Datenbank}

Diese Aufgabe bezieht sich auf die Bundesliga-Datenbank der Uni
Bayreuth:

\url{https://dbup2date.uni-bayreuth.de/bundesliga.html}

Gegeben sei das folgende relationale Schema: (Fremdschlüssel sind durch
Überstreichen gekennzeichnet)

Erstelle SQL-Abfragen, die die folgenden Informationen liefern:

% Liga

% +----------------------+--------------+------+-----+---------+-------+
% | Field                | Type         | Null | Key | Default | Extra |
% +----------------------+--------------+------+-----+---------+-------+
% | Liga_Nr              | int(1)       | NO   | PRI | NULL    |       |
% | Verband              | varchar(255) | NO   |     | NULL    |       |
% | Erstaustragung       | date         | NO   |     | NULL    |       |
% | Meister              | int(11)      | YES  | MUL | NULL    |       |
% | Rekordspieler        | varchar(255) | YES  |     | NULL    |       |
% | Spiele_Rekordspieler | int(11)      | YES  |     | NULL    |       |
% +----------------------+--------------+------+-----+---------+-------+

% Spiel

% +-----------+---------+------+-----+---------+----------------+
% | Field     | Type    | Null | Key | Default | Extra          |
% +-----------+---------+------+-----+---------+----------------+
% | Spiel_ID  | int(11) | NO   | PRI | NULL    | auto_increment |
% | Spieltag  | int(11) | YES  |     | NULL    |                |
% | Datum     | date    | YES  |     | NULL    |                |
% | Uhrzeit   | time    | YES  |     | NULL    |                |
% | Heim      | int(11) | NO   | MUL | NULL    |                |
% | Gast      | int(11) | NO   | MUL | NULL    |                |
% | Tore_Heim | int(11) | YES  |     | NULL    |                |
% | Tore_Gast | int(11) | YES  |     | NULL    |                |
% +-----------+---------+------+-----+---------+----------------+

% Spieler

% +--------------+--------------+------+-----+---------+----------------+
% | Field        | Type         | Null | Key | Default | Extra          |
% +--------------+--------------+------+-----+---------+----------------+
% | Spieler_ID   | int(11)      | NO   | PRI | NULL    | auto_increment |
% | Vereins_ID   | int(11)      | NO   | MUL | NULL    |                |
% | Trikot_Nr    | int(11)      | YES  |     | NULL    |                |
% | Spieler_Name | varchar(255) | NO   |     | NULL    |                |
% | Land         | varchar(255) | YES  |     | NULL    |                |
% | Spiele       | int(11)      | YES  |     | NULL    |                |
% | Tore         | int(11)      | YES  |     | NULL    |                |
% | Vorlagen     | int(11)      | YES  |     | NULL    |                |
% +--------------+--------------+------+-----+---------+----------------+

% Verein

% +-------+--------------+------+-----+---------+----------------+
% | Field | Type         | Null | Key | Default | Extra          |
% +-------+--------------+------+-----+---------+----------------+
% | V_ID  | int(11)      | NO   | PRI | NULL    | auto_increment |
% | Name  | varchar(255) | NO   | UNI | NULL    |                |
% | Liga  | int(1)       | YES  | MUL | NULL    |                |
% +-------+--------------+------+-----+---------+----------------+

\subsection{Aufgaben}

\renewcommand{\labelenumi}{\alph{enumi})}
\begin{enumerate}
\item Gesucht sind die Nummern aller Ligen zusammen mit den Namen der
jeweiligen Rekordspieler.

\begin{minted}{sql}
SELECT Liga_Nr, Rekordspieler
FROM Liga;
\end{minted}

% +---------+--------------------+
% | Liga_Nr | Rekordspieler      |
% +---------+--------------------+
% |       1 | Karl-Heinz Körbel  |
% |       2 | Willi Landgraf     |
% |       3 | Tim Danneberg      |
% +---------+--------------------+

\item Gesucht sind die IDs und die Namen aller Vereine, die in der
ersten oder zweiten Liga spielen.

\begin{minted}{sql}
SELECT V_ID, Name
FROM Verein
WHERE Liga = 1 OR Liga = 2;
\end{minted}

%%
%
%%

\item Gesucht sind der vollständige Name und die aktuelle Liga aller
Mannschaften, deren Name das Wort Würzburg enthält.

\begin{minted}{sql}
SELECT Name, Liga
FROM Verein
WHERE Name LIKE '%Wuerzburg%';
\end{minted}

%%
%
%%

\item Gesucht sind die ID und das Datum aller Spiele, die vor dem 1.
Oktober 2018 stattgefunden haben.

\begin{minted}{sql}
SELECT Spiel_ID, Datum
FROM Spiel
WHERE Datum < '2018-10-01';
\end{minted}

%%
%
%%

\item Gesucht sind die ID und das Datum aller Spiele, die zwischen
dem 1. Januar 2019 und dem 31. Januar 2019 stattfinden werden.

\begin{minted}{sql}
SELECT Spiel_ID, Datum
FROM Spiel
WHERE Datum >= '2019-01-01'
  AND Datum <= '2019-01-31';
\end{minted}

%%
%
%%

\item Gesucht sind die ID und das Datum aller Spiele, die in der
Vergangenheit liegen und in denen kein einziges Tor gefallen ist.

\begin{minted}{sql}
SELECT Spiel_ID, Datum
FROM Spiel
WHERE
  Datum < CURDATE() AND
  Tore_Heim = 0 AND
  Tore_Gast = 0;
\end{minted}

Musterlösung: NOW() statt CURDATE()

\end{enumerate}

Hinweis:

Datumsangaben müssen innerhalb der Abfrage in Anführungszeichen stehen,
beispielsweise so: '2018-10-01'

%-----------------------------------------------------------------------
%
%-----------------------------------------------------------------------

\subsection{Weitere Aufgaben:}

\begin{enumerate}

%%
%
%%

\item Welche Vereine spielen in der ersten Liga?

\begin{minted}{sql}
SELECT *
FROM Verein
WHERE Liga = 1;
\end{minted}

%%
%
%%

\item Welche Spieler spielen für den Verein „FC Bayern München“? Ordnen
Sie die Ergebnisse aufsteigend nach der Trikotnummer.

\begin{minted}{sql}
SELECT *
FROM Verein v, Spieler s
WHERE
  v.Name = 'FC Bayern München'
  s.Vereins_ID = v.V_ID
ORDER BY s.Trikot_Nr ASC;
\end{minted}

%%
%
%%

\item An wie vielen Spielen haben die Rekordspieler aller drei Ligen
insgesamt teilgenommen?

\begin{minted}{sql}
SELECT SUM(Spiele_Rekordspieler) FROM Liga;
\end{minted}

%%
%
%%

\item Wie viele Spieler tragen die Trikotnummer 12? Benennen Sie die
Ergebnisspalte in „Anzahl“ um.

\begin{minted}{sql}
SELECT COUNT(*) AS Anzahl
FROM Spieler
WHERE Trikot_Nr = 12;
\end{minted}

%%
%
%%

\item Wie viele Tore wurden bisher durchschnittlich von den Spielern
geschossen, die schon an mehr als 10 Spielen teilgenommen und mehr als
drei Vorlagen geliefert haben?

\begin{minted}{sql}
SELECT AVG(Tore)
FROM Spieler
WHERE
  Spiele > 10 AND
  Vorlagen > 3;
\end{minted}

%%
%
%%

\item Welche Vereine (Name, Liga) haben bisher schon mindestens einmal
unentschieden gespielt? Ordnen Sie das Ergebnis aufsteigend nach der
Liga und absteigend nach dem Vereinsnamen.

\begin{minted}{sql}
SELECT DISTINCT v.Name, v.Liga
FROM Verein v, Spiel s
WHERE
  s.Tore_Heim = s.Tore_Gast AND
  (s.Heim = v.V_ID OR s.Gast = v.V_ID)
ORDER BY v.Liga ASC, v.Name DESC;
\end{minted}

falsch: kein OR nur nach Heim gesucht, DISTINCT vergessen
\end{enumerate}
\end{document}
