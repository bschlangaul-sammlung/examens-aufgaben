\documentclass{lehramt-informatik-haupt}
\liLadePakete{syntax,mathe,rmodell}

\begin{document}

\chapter{Aufgabenblatt 4: Wiederholung: Relationale Algebra, SQL, ER-Modell, Relationenschema}

%-----------------------------------------------------------------------
%
%-----------------------------------------------------------------------

\section{Aufgabe 1: Division}

Gegeben sind zwei Relationen, repräsentiert als Tabellen. Bestimmen Sie
die Ergebnisrelation $R1 \div R2$!

\begin{antwort}
\begin{tabular}{ll}
\textbf{a} & \textbf{c} \\
a & 4 \\
b & d
\end{tabular}
\end{antwort}

%-----------------------------------------------------------------------
%
%-----------------------------------------------------------------------

\ExamensAufgabeA 66111 / 1996 / 03 : Aufgabe 2

%-----------------------------------------------------------------------
%
%-----------------------------------------------------------------------

\section{Aufgabe 3: Olympische Spiele – Zur Wiederholung}

Geben Sie ein Entity-Relationship-Diagramm (mit Schlüssel und
Funktionalität) für folgendes Problem an:

Eine europäische Fachzeitschrift für Leichtathletik möchte in einer
relationalen Datenbank die folgenden Informationen zu den Olympischen
Sommerspielen speichern: Für jeden Sportler, der bei den Olympischen
Spielen teilgenommen hat, sollen der Name, die Nationalität, die Größe
und das Gewicht bekannt sein. Zusätzlich soll abgefragt werden können,
ob der betreffenden Sportler aus Europa stammt. Zu den Olympischen
Spielen sollen das Jahr der Austragungsort und die Teilnehmerzahl
gespeichert werden. Außerdem soll bekannt sein, welcher Sportler bei
welchen Spielen welchen Rekord erzielt hat. Zu jedem Rekord sollen die
Disziplin und die Art des Rekords (Weltrekord, Europarekord,...)
gespeichert werden. Darüber soll ohne Umweg abrufbar sein, ob es sich um
einen Rekord in der Leichtathletik handelt. Es sei vorausgesetzt, dass
jeder Sportler eindeutig durch seinen Namen identifizierbar ist und dass
Olympische Spiele in verschiedenen Jahren in derselben Stadt stattfinden
können. Ein Rekord soll eindeutig durch die Disziplin und die Art des
Rekords gekennzeichnet sein.

\begin{antwort}[falsch]
gemacht am 12.12 auf Schmierpapier. Route erzielt vergessen. Entity-Typen
können nicht direkt verbunden werden.
\end{antwort}

%-----------------------------------------------------------------------
%
%-----------------------------------------------------------------------

\documentclass{lehramt-informatik-aufgabe}
\liLadePakete{}
\begin{document}
\liAufgabenTitel{VertexCover}
\section{Aufgabe 4
\index{Polynomialzeitreduktion}
\footcite{46115:2016:09}}

Betrachte die beiden folgenden Probleme:

V ERT EXCOV ER

\begin{description}
\item[Gegeben:]

Ein ungerichteter Graph G = (v, E) und eine Zahl k ∈ {1, 2, 3, ...}.
\item[Frage:]

Gibt es eine Menge C ⊆ V mit |C| ≤ k, so dass für jede Kante (u, v) aus E
mindestens einer der Knoten u und v in C ist?
V ERT EXCOV ER3
\end{description}

\begin{description}
\item[Gegeben:]

Ein ungerichteter Graph G = (v, E) und eine Zahl k ∈ {3, 4, 5, ...}.

\item[Frage:]

Gibt es eine Menge C ⊆ V mit |C| ≤ k, so dass für jede Kante (u, v) aus E
mindestens einer der Knoten u und v in C ist?
Gib eine polynomielle Reduktion von V ERT EXCOV ER auf V ERT EXCOV ER3 an
und begründe anschließend, dass die Reduktion korrekt ist.
\end{description}
\footcite[Aufgabe 13, Seite 12]{theo:ab:4}

\begin{liAntwort}

V ERT EXCOV ER ≤ p V ERT EXCOV ER3

f fügt vier neue Knoten, von denen jeweils ein Paar verbunden ist. Außerdem erhöht f k um 2.

Total: Jeder Graph kann durch f so verändert werden. Korrektheit: Wenn VC für k in G exis-
tiert, dann existiert auch VC mit k + 2 Knoten in G 0 , da für den eingefügten Teilgraphen ein VC
mit k = 2 existiert.
In Polyzeit berechenbar: für Adjazenzmatrix müssen lediglich 4 neue Spalten/Zeilen ein-
gefügt werden und k+2 berechnet werden.

\end{liAntwort}

\end{document}


%-----------------------------------------------------------------------
%
%-----------------------------------------------------------------------

\documentclass{lehramt-informatik-minimal}
\InformatikPakete{graph,mathe}
\begin{document}

\section{Städte gemischt gerichtet / ungerichtet
\index{Algorithmus von Dijkstra}
\footcite[Thema 1 Aufgabe 5]{examen:66112:2004:03}
}

Ein wichtiges Problem im Bereich der Graphalgorithmen ist die Berechnung
kürzester Wege. Gegeben sei der folgende Graph, in dem Städte durch
Kanten verbunden sind. Die Kantengewichte geben Fahrzeiten an. Außer den
durch Pfeile als nur in eine Richtung befahrbar gekennzeichneten Straßen
sind alle Straßen in beiden Richtungen befahrbar.
\footcite{aud:pu:6}

\graph knoten {
  \knoten{A}(2,5)
  \knoten{B}(3,3)
  \knoten{C}(0,3)
  \knoten{D}(5,1)
  \knoten{E}(4,5)
  \knoten{F}(7,1)
  \knoten{G}(1,0)
  \knoten{H}(6,4)
} kanten {
  \kante(A-B){$10$}
  \kante(A-E){$40$}
  \kante(B-G){$20$}
  \kante(B-C){$50$}
  \kante(B-D){$90$}
  \kante(B-H){$90$}
  \kante(C-G){$20$}
  \kante(D-G){$75$}
  \kante(D-F){$80$}
  \kante(E-H){$5$}
  \kante(F-H){$10$}
  \kanteR(A>C){$70$}
  \kanteR(H>D){$10$}
}

\begin{enumerate}

%%
% a)
%%

\item Geben Sie zu dem obigen Graphen zunächst eine Darstellung als
Adjazenzmatix an.

\[
\begin{blockarray}{ccccccccc}
  & A  & B  & C  & D  & E  & F  & G  & H  \\
\begin{block}{c(cccccccc)}
A & 0  & 10 & 70 & 0  & 40 & 0  & 0  & 0  \\
B & 10 & 0  & 50 & 90 & 0  & 0  & 20 & 90 \\
C & 0  & 50 & 0  & 0  & 0  & 0  & 20 & 0  \\
D & 0  & 90 & 0  & 0  & 0  & 80 & 75 & 0  \\
E & 40 & 0  & 0  & 0  & 0  & 0  & 0  & 5  \\
F & 0  & 0  & 0  & 80 & 0  & 0  & 0  & 10 \\
G & 0  & 20 & 20 & 75 & 0  & 0  & 0  & 0  \\
H & 0  & 90 & 0  & 10 & 5  & 10 & 0  & 0  \\
\end{block}
\end{blockarray}
\]

%%
% b)
%%

\item Berechnen Sie nun mit Hilfe des Algorithmus von Dijkstra die
kürzesten Wege vom Knoten A zu allen anderen Knoten.
\end{enumerate}

\end{document}


\end{document}
