\documentclass{lehramt-informatik}
\InformatikPakete{syntax}

\newcommand{\tmptabelle}[1]{
\bigskip
\par
\noindent
\textbf{#1}:
\bigskip
\par
\noindent
}

\begin{document}

\chapter{Aufgabenblatt 2: Einstieg SQL}

%-----------------------------------------------------------------------
%
%-----------------------------------------------------------------------

\section{Aufgabe 1: Kaufhausdatenbank - einfacher Einstieg\footcite{db:ab:2}}

Die relationale Datenbank eines Kaufhauses enthält folgende Tabelle mit
dem Namen „Artikel“:

\begin{tabular}{llll}
ArtNr & Bezeichnung  & Verkaufspreis & Einkaufspreis \\
95    & Kamm         & 1.25          & 0.80          \\
97    & Kamm         & 0.99          & 0.75          \\
507   & Seife        & 3.93          & 2.45          \\
1056  & Zwieback     & 1.20          & 0.90          \\
1401  & Räucherlachs & 4.90          & 3.60          \\
2045  & Herrenhose   & 37.25         & 24.45         \\
2046  & Herrenhose   & 20.00         & 17.00         \\
2340  & Sommerkleid  & 94.60         & 71.50
\end{tabular}

Formulieren Sie mit Hilfe von SQL folgende Anfragen:

\renewcommand{\labelenumi}{(\alph{enumi})}
\begin{enumerate}

%%
% (a)
%%

\item Gesucht sind alle Informationen über Herrenhose und Sommerkleid!

\begin{antwort}[muster]
\begin{minted}{sql}
SELECT *
FROM Artikel
WHERE
  Bezeichnung = 'Herrenhose' OR
  Bezeichnung = 'Sommerkleid';
\end{minted}
\end{antwort}

%%
% (b)
%%

\item Welche Artikelnummer hat der Zwieback?

\begin{antwort}[muster]
\begin{minted}{sql}
SELECT ArtNr
FROM Artikel
WHERE
  Bezeichnung = 'Zwieback';
\end{minted}
\end{antwort}

%%
% (c)
%%

\item Welche Waren (Artikelnummer und Verkaufspreis) werden für mehr als
25€ verkauft?

\begin{antwort}[muster]
\begin{minted}{sql}
SELECT ArtNr, Verkaufspreis
FROM Artikel
WHERE Verkaufspreis > 25.00;
\end{minted}
\end{antwort}

%%
% (d)
%%

\item Welche Artikel (Angabe der Bezeichnung) bietet das Kaufhaus an?

\begin{antwort}[muster]
\begin{minted}{sql}
SELECT DISTINCT Bezeichnung
FROM Artikel;
\end{minted}
\end{antwort}

%%
% (e)
%%

\item Gesucht sind die Artikelnummern aller Artikel mit Ausnahme der
Artikelnummer 2046.

\begin{antwort}[richtig]
\begin{minted}{sql}
SELECT ArtNr
FROM Artikel
WHERE ArtNr != 2046;
\end{minted}
\end{antwort}

\begin{antwort}[muster]
\begin{minted}{sql}
SELECT ArtNr
FROM Artikel
WHERE NOT (ArtNr = 2046);
\end{minted}
\end{antwort}

%%
% (f)
%%

\item Gib die Artikelnummern und die Verkaufspreise aller Herrenhosen
aus, die für höchstens 25€ verkauft werden! Der Spaltenname für die
Verkaufspreise soll in der Ergebnistabelle „Sonderangebot“ heißen.

\begin{antwort}[muster]
\begin{minted}{sql}
SELECT ArtNr, Verkaufspreis AS Sonderangebot
FROM Artikel
WHERE Bezeichnung = 'Herrenhose' AND Verkaufspreis <= 25;
\end{minted}
\end{antwort}

%%
% (g)
%%

\item Gib Artikelnummer und Verkaufspreis aller Artikel aus, die im
Einkauf zwischen 80 Cent und 5€ kosten.

\begin{antwort}[richtig]
\begin{minted}{sql}
SELECT ArtNr, Verkaufspreis
FROM Artikel
WHERE Einkaufspreis >= 0.8 AND Einkaufspreis <= 5;
\end{minted}
\end{antwort}

\begin{antwort}[muster]
\begin{minted}{sql}
SELECT ArtNr, Verkaufspreis
FROM Artikel
WHERE Einkaufspreis BETWEEN 0.80 AND 5.00;
\end{minted}
\end{antwort}

\end{enumerate}

%-----------------------------------------------------------------------
%
%-----------------------------------------------------------------------

\section{Aufgabe 2: Kaufhaus reloaded\footcite{db:ab:2}}

Eine Kaufhausdatenbank besteht aus folgenden Tabellen:

\tmptabelle{Artikel}

\begin{tabular}{llll}
\textbf{ArtNr} & \textbf{Bezeichnung}  & \textbf{Verkaufspreis} & \textbf{Einkaufspreis} \\
95    & Kamm         & 1.25          & 0.80          \\
97    & Kamm         & 0.99          & 0.75          \\
507   & Seife        & 3.93          & 2.45          \\
1056  & Zwieback     & 1.20          & 0.90          \\
1401  & Räucherlachs & 4.90          & 3.60          \\
2045  & Herrenhose   & 37.25         & 24.45         \\
2046  & Herrenhose   & 20.00         & 17.00         \\
2340  & Sommerkleid  & 94.60         & 71.50
\end{tabular}

\tmptabelle{Abteilung}

\begin{tabular}{lll}
\textbf{Abteilungsname} & \textbf{Stockwerk} & \textbf{Abteilungsleiter} \\
Lebensmittel   & I         & Josef Kunz       \\
Lebensmittel   & EG        & Monika Stiehl    \\
Textilien      & II        & Monika Stiehl
\end{tabular}

\tmptabelle{Bestand}

\begin{tabular}{lll}
\textbf{Abteilungsname}  & \textbf{ArtNr} & \textbf{Vorrat} \\
Lebensmittel    & 1056  & 129    \\
Lebensmittel    & 1401  & 200    \\
Textilien       & 2045  & 14
\end{tabular}

Hinweis:

Basistabellen (Tabellen, die in der Datenbank definiert sind), werden im
Rahmen des Kurses graphisch dadurch gekennzeichnet, dass es eine
„nullte“ Spalte gibt, in der der Tabellenname angegeben wird. Die
Realisierung der Tabellen in der Datenbank enthält diese Spalte
natürlich nicht.

\renewcommand{\labelenumi}{\arabic{enumi}.}
\renewcommand{\labelenumii}{(\alph{enumii})}

\begin{enumerate}
\item Geben Sie die SQL-Befehle an, mit der die Tabellenschemata von
Artikel und Bestand erzeugt werden können. Wählen Sie dabei geeignete
Domänen.

Frage richtig lesen. Abteilung soll nicht erzeugt werden.

\begin{antwort}[falsch]
\begin{minted}{sql}
CREATE TABLE Artikel (
  ArtNr INTEGER PRIMARY KEY NOT NULL,
  Bezeichnung VARCHAR(100) NOT NULL,
  Verkaufspreis FLOAT(2),
  Einkaufspreis FLOAT(2)
);

CREATE TABLE Abteilung (
  Abteilungsname VARCHAR(60) NOT NULL,
  Stockwerk VARCHAR(10) NOT NULL,
  Abteilungsleiter VARCHAR(100),
  PRIMARY KEY (Abteilungsname, Stockwerk)
);

CREATE TABLE Bestand (
  Abteilungsname VARCHAR(100) REFERENCES Abteilung(Abteilungsname),
  ArtNr INTEGER PRIMARY KEY NOT NULL,
  Vorrat INTEGER
);
\end{minted}
\end{antwort}

\begin{antwort}[muster]
\begin{minted}{sql}
CREATE TABLE Artikel (
  ArtNr INTEGER PRIMARY KEY NOT NULL,
  Bezeichnung VARCHAR(100) NOT NULL,
  Verkaufspreis FLOAT(2),
  Einkaufspreis FLOAT(2)
);

CREATE TABLE Bestand (
  Abteilungsname VARCHAR(100) REFERENCES Abteilung(Abteilungsname),
  ArtNr INTEGER REFERENCES Artikel(ArtNr),
  Vorrat INTEGER,
  PRIMARY KEY (Abteilungsname, ArtNr)
);
\end{minted}
\end{antwort}

%%
% 2.
%%

\item Es treten nun nacheinander die folgenden Änderungen auf.
Aktualisieren Sie den Tabellenbestand mit den entsprechenden
SQL-Befehlen:

\begin{enumerate}

%%
% (a)
%%

\item Ein Sommerkleid mit der Artikelnummer 2341, dem Einkaufspreis 70€
und dem Verkaufspreis 90,75€ wird in das Artikelsortiment aufgenommen.

\begin{antwort}[muster]
\begin{minted}{sql}
INSERT INTO Artikel
VALUES (2341, 'Sommerkleid', 90.75, 70.00);
\end{minted}
\end{antwort}

%%
% (b)
%%

\item Der Artikel mit der Nummer 2341 wird wieder aus dem Sortiment
genommen, da er den Qualitätsstandards nicht entsprochen hat.

\begin{antwort}[muster]
\begin{minted}{sql}
DELETE FROM Artikel WHERE ArtNr = 2341;
\end{minted}
\end{antwort}

%%
% (c)
%%

\item Eine Bürste mit der Artikelnummer 2 wird in das Sortiment
aufgenommen. Einkaufs- bzw. Verkaufspreis sind noch nicht festgelegt.

\begin{antwort}[muster]
\begin{minted}{sql}
INSERT INTO Artikel (ArtNr, Bezeichnung)
VALUES (2, 'Bürste');
\end{minted}
\end{antwort}

%%
% (d)
%%

\item Eine Damenhose (Verkaufspreis 89€, Einkaufspreis: 60,50€) wird
neu in das Sortiment aufgenommen. Eine Artikelnummer wurde noch nicht
festgelegt.

\begin{antwort}[falsch]
Artikelnummer ist noch nicht festgelegt.
\begin{minted}{sql}
INSERT INTO Artikel
VALUES (2342, 'Damenhose', 89.00, 60.50);
\end{minted}
\end{antwort}

\begin{antwort}[muster]
ArtNr ist der Primärschlüssel der Tabelle Artikel. Bei Eingabe eines
neuen Datensatzes müssen mindestens die Werte aller Attribute, die zum
Primärschlüssel gehören, angegeben werden. Da aber im Fall der Damenhose
die Artikelnummer noch nicht festgelegt ist, ist eine Eingabe der
Damenhose-Daten in die Tabelle Artikel nicht möglich. Hinweis: Denken
Sie also immer daran, dass bei Einfügen von Datensätzen der
Primärschlüssel keine NULL-Werte enthalten darf!
\end{antwort}

%%
% (e)
%%

\item Die Herrenhosen werden aus dem Sortiment genommen und deshalb aus
der Tabelle Artikel gelöscht.

\begin{antwort}[falsch]
Artikel nicht aus dem Bestand herausgelöscht.
\begin{minted}{sql}
DELETE FROM Artikel WHERE Bezeichnung = 'Herrenhose';
\end{minted}
\end{antwort}

\begin{antwort}[muster]
\begin{minted}{sql}
DELETE FROM Bestand WHERE ArtNr = 2045;
DELETE FROM Artikel WHERE Bezeichnung = 'Herrenhose';
\end{minted}
\end{antwort}

%%
% (f)
%%

\item Die neue Abteilungsleiterin der Lebensmittelabteilung heißt Elvira
Sommer.

\begin{antwort}[muster]
\begin{minted}{sql}
UPDATE Abteilung
SET Abteilungsleiter = 'Elvira Sommer'
WHERE Abteilungsnahme = 'Lebensmittel';
\end{minted}
\end{antwort}

%%
% (g)
%%

\item Die Abteilung Feinkost hat einen Bestand von 150
Räucherlachspackungen.

\begin{antwort}[falsch]
vergessen Wert in Abteilung einzutragen
\begin{minted}{sql}
INSERT INTO Bestand VALUES ('Feinkost', 1401, 150);
\end{minted}
\end{antwort}

\begin{antwort}[muster]
Die Attribute ArtNr und Abteilungsname der Tabelle
Bestand sind Fremdschlüssel. Ein neuer Datensatz darf in die Tabelle nur
eingefügt werden, wenn die Fremdschlüsselwerte in den entsprechenden
(Primärschlüssel-)Attribute der referenzierten Tabelle auch existieren.
Die Abteilung Feinkost, genauer gesagt den Abteilungsnamen ’Feinkost’
gibt es in Abteilung aber noch nicht.

\begin{enumerate}
\item Lösungsmöglichkeit: Die Aktualisierung kann nicht durchgeführt
werden.

\item Lösungsmöglichkeit: Die entsprechende Abteilung Feinkost wird –
natürlich in „Absprache“ mit der Kaufhausleitung – eingeführt und ein
dementsprechender Datensatz in Abteilung eingefügt.
\end{enumerate}

\begin{minted}{sql}
INSERT INTO Abteilung (Abteilungsname) VALUES ('Feinkost');
INSERT INTO Bestand VALUES ('Feinkost', 1401, 150);
\end{minted}
\end{antwort}

\end{enumerate}

%%
% 3.
%%

\item Formulieren Sie folgende Anfragen in SQL:

\begin{enumerate}

%%
% (a)
%%

\item Gesucht sind Artikelnummer und Vorrat aller Artikel aus der
Textil-Abteilung.

\begin{antwort}[muster]
\begin{minted}{sql}
SELECT ArtNr, Vorrat FROM Bestand WHERE Abteilungsname = 'Textilien';
\end{minted}
\end{antwort}

%%
% (b)
%%

\item Gesucht sind alle Informationen über die Abteilungen, die im
zweiten Stock platziert sind oder von Frau Stiehl geleitet werden.

\begin{antwort}[muster]
\begin{minted}{sql}
SELECT * FROM Abteilung
WHERE Stockwerk = 'II' OR Abteilungsleiter = 'Monika Stiehl';
\end{minted}
\end{antwort}

\end{enumerate}

%%
% 4.
%%

\item Formulieren Sie folgende SQL-Anfragen umgangssprachlich:

\begin{enumerate}

%%
% (a)
%%

\item SQL-Anfrage:

\begin{minted}{sql}
SELECT DISTINCT Abteilungsleiter
FROM Abteilung
WHERE NOT (Abteilungsname = 'Kosmetik');
\end{minted}

\begin{antwort}[richtig]
Es werden alle Abteilungsleiter aus der Abteilungsleiter der
Kosmetik-Abteilung aufgelistet.
\end{antwort}

\begin{antwort}[muster]
Gesucht sind die Namen aller Abteilungsleiter mit Ausnahme der
Kosmetik-Abteilung. Duplikate sollen eliminiert werden.
\end{antwort}

%%
% (b)
%%

\item SQL-Anfrage:

\begin{minted}{sql}
SELECT ArtNr
FROM Bestand
WHERE Abteilungsname = "Lebensmittel" AND Vorrat <= 100;
\end{minted}

\begin{antwort}[richtig]
Gesucht sind die Artikelnummer aller Artikel von denen 100 oder weniger
Bestand vorhanden ist und die aus der Abteilung Lebensmittel stammen.
\end{antwort}

\begin{antwort}[muster]
Gesucht sind die Nummern der Artikel, von denen in der
Lebensmittelabteilung maximal 100 vorrätig sind.
\end{antwort}

\end{enumerate}

\item Interpretieren Sie nun die obigen Tabellen als Repräsentationen
der drei Relationen Artikel, Abteilung und Bestand. Bestimmen Sie die
Ergebnisrelationen folgender relationaler Ausdrücke:

\tmptabelle{Artikel}

\begin{tabular}{llll}
\textbf{ArtNr} & \textbf{Bezeichnung}  & \textbf{Verkaufspreis} & \textbf{Einkaufspreis} \\
95    & Kamm         & 1.25          & 0.80          \\
97    & Kamm         & 0.99          & 0.75          \\
507   & Seife        & 3.93          & 2.45          \\
1056  & Zwieback     & 1.20          & 0.90          \\
1401  & Räucherlachs & 4.90          & 3.60          \\
2045  & Herrenhose   & 37.25         & 24.45         \\
2046  & Herrenhose   & 20.00         & 17.00         \\
2340  & Sommerkleid  & 94.60         & 71.50
\end{tabular}

\tmptabelle{Abteilung}

\begin{tabular}{lll}
\textbf{Abteilungsname} & \textbf{Stockwerk} & \textbf{Abteilungsleiter} \\
Lebensmittel   & I         & Josef Kunz       \\
Lebensmittel   & EG        & Monika Stiehl    \\
Textilien      & II        & Monika Stiehl
\end{tabular}

\tmptabelle{Bestand}

\begin{tabular}{lll}
\textbf{Abteilungsname}  & \textbf{ArtNr} & \textbf{Vorrat} \\
Lebensmittel    & 1056  & 129    \\
Lebensmittel    & 1401  & 200    \\
Textilien       & 2045  & 14
\end{tabular}

\begin{enumerate}

%%
% (a)
%%

\item $\pi_{ArtNr,Bezeichnung}(Artikel)$

\begin{antwort}[muster]
\begin{tabular}{ll}
95    & Kamm          \\
97    & Kamm          \\
507   & Seife         \\
1056  & Zwieback      \\
1401  & Räucherlachs  \\
2045  & Herrenhose    \\
2046  & Herrenhose    \\
2340  & Sommerkleid
\end{tabular}
\end{antwort}

%%
% (b)
%%

\item $\pi_{Abteilungsname}(Bestand)$

\begin{antwort}[muster]
\begin{tabular}{l}
Lebensmittel   \\
Textilien
\end{tabular}
\end{antwort}

%%
% (c)
%%

\item $\sigma_{((Vorrat < 100 \land ArtNr > 1500) \lor ArtNr < 1100)}(Bestand)$

\begin{antwort}[muster]
\begin{tabular}{lll}
Lebensmittel    & 1056  & 129    \\
Textilien       & 2045  & 14
\end{tabular}
\end{antwort}

%%
% (d)
%%

\item $\sigma_{((Vorrat < 100 \land (ArtNr > 1500  \lor ArtNr < 1100)}(Bestand)$

\begin{antwort}[muster]
\begin{tabular}{lll}
Textilien       & 2045  & 14
\end{tabular}
\end{antwort}

%%
% (e)
%%

\item $\pi_{ArtNr}(\sigma_{Bezeichnung=Herrenhose}(Artikel))$

\begin{antwort}[muster]
\begin{tabular}{l}
2045 \\
2046
\end{tabular}
\end{antwort}

%%
% (f)
%%

\item $\pi_{Abteilungsname}(Abteilung) - \pi_{Abteilungsname} (Bestand)$

\begin{antwort}[muster]
\begin{tabular}{l}
Kosmetik
\end{tabular}
\end{antwort}

%%
% (g)
%%

\item

\begin{math}
\pi_{Bezeichnung,Einkaufspreis}(\sigma_{Einkaufspreis < 2.50} (Artikel))
\cup\\
\pi_{Bezeichnung,Einkaufspreis}(\sigma_{Einkaufsreis > 20.00} (Artikel))
\end{math}

\begin{antwort}[muster]
Die letzten Zeile ist nicht in der Musterlösung dabei. Ich glaube aber
es müsste so stimmen.

\begin{tabular}{llll}
Bezeichnung  & Einkaufspreis \\
Kamm         & 0.80          \\
Kamm         & 0.75          \\
Seife        & 2.45          \\
Zwieback     & 0.90          \\
Herrenhose   & 24.45         \\
Sommerkleid  & 71.50
\end{tabular}
\end{antwort}

Hinweis:

Die Aufgabe bezieht sich auf die oben angegebenen Tabellen. Die in
Aufgabe 3 durchgeführten Änderungen des Datenbestandes brauchen nicht
berücksichtigt werden.

\end{enumerate}
\end{enumerate}

%-----------------------------------------------------------------------
%
%-----------------------------------------------------------------------

\section{Aufgabe 4: SQL-Anfragen auf mehreren Tabellen\footcite{db:ab:2}}

(Quelle: Staatsexamen Softwaretechnologie/Datenbanksysteme, Thema Nr. 2,
Teilaufgabe II, Aufgabe 4, Herbst 2017 Realschule)

\noindent
Für die bayerische Meisterschaft im Turmspringen ist folgendes
Datenbankschema angelegt:

\begin{minted}{md}
Springer (Startnummer, Nachname, Vorname, Geburtsdatum, Körpergröße)
Sprung (SID, Beschreibung, Schwierigkeit)
springt (SID, Startnummer, Durchgang)
FK (SID) referenziert Sprung (SID)
FK (Startnummer) referenziert Springer (Startnummer)
\end{minted}

\noindent
Das Attribut Schwierigkeit kann die Werte 1 bis 10 annehmen, das
Attribut Durchgang ist positiv und ganzzahlig. Die Körpergröße der
Springer ist in Zentimeter angegeben.

\begin{enumerate}

%%
% (a)
%%

\item Welche Springer sind größer als 1,80 m? Schreiben Sie eine
SQL-Anweisung, welche in der Ausgabe mit dem größten Springer beginnt.

\begin{antwort}[muster]
\begin{minted}{sql}
SELECT Vorname, Nachname
FROM Springer
WHERE Koerpergroesse > 180
ORDER BY Koerpergroesse DESC;
\end{minted}
\end{antwort}

%%
% (b)
%%

\item Welche Springer haben im ersten Durchgang einen Sprung mit einer
Schwierigkeit von unter 6 gezeigt? Schreiben Sie eine SQL-Anweisung,
welche Startnummer und Nachname dieser Springer ausgibt.

\begin{antwort}[muster]
\begin{minted}{sql}
SELECT Springer.Startnummer, Springer.Nachname
FROM Springer, Sprung, springt
WHERE
  Sprung.SID = springt.SID AND
  Springer.Startnummer = springt.Startnummer AND
  springt.Durchgang = 1
  Sprung.Schwierigkeit < 6;
\end{minted}
\end{antwort}

%%
% (c)
%%

\item Formulieren Sie in Umgangssprache, aber trotzdem möglichst
präzise, wonach mit folgender Abfrage gesucht wird:

\begin{minted}{sql}
SELECT springt.Startnummer, s.Nachname, s.Vorname,
MAX (springt.Durchgang)
FROM springt, Springer s
WHERE springt.Startnummer = s.Startnummer
GROUP BY springt.Startnummer, s.Nachname, s.Vorname
\end{minted}

\begin{antwort}[richtig]
Die Abfrage gibt die Startnummer, den Nachnamen, den Vornamen und
die Anzahl der Sprünge, d. h. die Anzahl der Durchgänge der
einzelnen Springer an.
\end{antwort}

\begin{antwort}[muster]
Die Abfrage liefert die Startnummer, Vorname und Nachname sowie die
maximale Anzahl an Durchgängen jedes Springers.
\end{antwort}

\item Gesucht ist die „durchschnittliche Körpergröße“ all der Springer,
die vor dem 01.01.2000 geboren wurden. Formulieren Sie eine
[entsprechende] SQL-Anweisung, [...] wobei die Spalte mit der
durchschnittlichen Körpergröße genau diesen Namen „durchschnittliche
Körpergröße“ haben soll.

\begin{antwort}
Umlaute und Leerzeichen sind bei Spaltenbeschriftungen nicht erlaubt.

\begin{minted}{sql}
SELECT AVG(Koerpergroesse) AS durchschnittliche_Koerpergroesse
FROM SPRINGER
WHERE Geburtsdatum < DATE('2000-01-01');
\end{minted}
\end{antwort}

\begin{antwort}[muster]
\begin{minted}{sql}
SELECT AVG(Koerpergroesse) AS durchschnittliche_Koerpergroesse
FROM SPRINGER
WHERE Geburtsdatum < '01.01.2001';
\end{minted}
\end{antwort}

\end{enumerate}

%-----------------------------------------------------------------------
%
%-----------------------------------------------------------------------


\end{document}
