\documentclass{lehramt-informatik-haupt}
\usepackage{soul}
\usepackage{multicol}
\def\TmpUeber#1{{\setul{-0.9em}{}\ul{#1}}}

\liLadePakete{syntax,mathe,er}

\begin{document}

\chapter{Präsenzübung 6: Wiederholung}

\ExamensAufgabeTTA 46116 / 2014 / 03 : Thema 2 Teilaufgabe 2 Aufgabe 3

\section{Aufgabe 6: Normalformen\footcite{db:pu:wh}}

a) C und E müssen immer Teil des Schlüsselkandidaten
AttrHull(F, \{C, E\} = \{C, E, B, A, D, F\})

-> Superschlüssel
-> Schlüsselkandidat, weil minimal denn C und E müssen immer Teil sein.
-> kein anderer SK möglich, weil C und E immer Teil sein müssen.

Sie selbst aber schon minimal sind.

b)

\literatur

\end{document}
