\documentclass{lehramt-informatik-haupt}
\InformatikPakete{syntax,er,rmodell}
\usepackage{soul}

\begin{document}

%%%%%%%%%%%%%%%%%%%%%%%%%%%%%%%%%%%%%%%%%%%%%%%%%%%%%%%%%%%%%%%%%%%%%%%%
% Theorie-Teil
%%%%%%%%%%%%%%%%%%%%%%%%%%%%%%%%%%%%%%%%%%%%%%%%%%%%%%%%%%%%%%%%%%%%%%%%

\chapter{Entity-Relation-System}

\begin{quellen}
\item \cite{wiki:entity-relationship-modell}
\end{quellen}

\begin{itemize}
\item Datenmodell: Eignung zur Darstellung des konzeptuellen
Datenbankschemas
\item standardisierte graphische Notation: ER-Diagramm
\item Vorgang der Modellierung: ER-Entwurf
\item Resultat: ER-Modell
\item Vorteil: Kann leicht in Tabellen einer relationalen Datenbank
überführt werden
\item jedoch: nur Strukture der Daten, keine Datenmanipulation
\end{itemize}

\subsection{Begriffsklärung}

Miniwelt:

\begin{itemize}
\item Entity: Object
\item Relationships: Beziehungen zwischen den Werten
\item Attribute: Eigenschaften von Entities oder Relationships
\item Attributewerte: Werte der Attribute
\end{itemize}

Entity-Typen: Gleichartige Entities, mit den gleichen Eigenschaften
Relationship-Typen: Beziehungen gleicher Art

Dabei gilt:

\begin{enumerate}
\def\labelenumi{\arabic{enumi}.}
\item nicht disjunkt: ein Element mehrer Entity-Typen sein.
\item Stelligkeit eines Relationship-Typs
\item Entity-Typ kein mehrere Relationship-Typen haben
\item Schemaebene: Entity-Typ, Relationship-Typ, Attribute; Instanz: Entity,
Relationship, Attributwert
\item Modellierung auf der Schemaebene, keine konkreten Entites
\end{enumerate}

\subsection{Graphische Notation}

\begin{itemize}
\item Entity-Typen: Rechtecke
\item Relationship-Type: Rauten
\item Attribute: Ovale
\end{itemize}

\subsection{Rollennamen}

Zur genaueren Charakterisierung von Entity-Typen und Relationship-Typen

\subsection{Domäne}

zulässige Attributewerte

\begin{itemize}
\item extensional: Aufzählung alles zulässigen Werte
\item intensional: Angabe allgemein bekannter Mengen
\end{itemize}

Mehrwertiges Attribute (Doppelkreis): mehrere Telefonnummern
Abgeleitetes Attribute (gestrichelte Linie): Alter (vom Geburstag)

Relationshiptypen:

\begin{itemize}
\item Binäre zwischen 2 Entitytypen
\item Ternäre zwischen 3 Entitytypten
\end{itemize}

\section{Entity-Relation-System}

% https://usehardware.de/datenbanksysteme-iv-entity-relationship-modell-er-modell-datenbankdarstellungen-i/
\begin{tabular}{cl}

\begin{tikzpicture}
\node[entity] {Entität};
\end{tikzpicture} &
Entität
\\

\begin{tikzpicture}
\node[weak entity] {Entität};
\end{tikzpicture} &
schwache Entität
\\

\begin{tikzpicture}
\node[relationship] {Beziehung};
\end{tikzpicture} &
Beziehung / Relationship
\\

\begin{tikzpicture}
\node[attribute] {Attribut};
\end{tikzpicture} &
einfaches Attribut
\\

\begin{tikzpicture}[node distance=1.6cm]
\node[attribute] (att1) {Attribut};
\node[attribute] (att2) [above right of=att1] {Attribut} edge (att1);
\node[attribute] (att3) [above left of=att1] {Attribut} edge (att1);
\end{tikzpicture} &
zusammengesetztes Attribut
\\

\begin{tikzpicture}
\node[derived attribute] {Attribut};
\end{tikzpicture} &
abgeleitetes Attribut
\\

\begin{tikzpicture}
\node[multi attribute] {Attribut};
\end{tikzpicture} &
Mehrfachattribut
\\

\begin{tikzpicture}
\node[attribute] {\key{Attribut}};
\end{tikzpicture} &
Schlüsselattribut
\\

\begin{tikzpicture}
\node[attribute] {\discriminator{Attribut}};
\end{tikzpicture} &
schwaches Schlüsselattribut
\\

\begin{tikzpicture}
\path[link] (0,0) -- (3,0);
\end{tikzpicture} &
partielle Teilnahme
\\

\begin{tikzpicture}
\path[draw,weak] (0,0) -- (3,0);
\end{tikzpicture} &
totale Teilnahme
\\

\end{tabular}

\begin{itemize}
\item Datenmodell: Eignung zur Darstellung des konzeptuellen
Datenbankschemas
\item standardisierte graphische Notation: ER-Diagramm
\item Vorgang der Modellierung: ER-Entwurf
\item Resultat: ER-Modell
\item Vorteil: Kann leicht in Tabellen einer relationalen Datenbank
überführt werden
\item jedoch: nur Strukture der Daten, keine Datenmanipulation
\end{itemize}

\subsection{Begriffsklärung}

Miniwelt:

\begin{description}
\item[Entity:] Objekt

\item[Entity-Typen:] Gleichartige Entities, mit den gleichen
Eigenschaften

\begin{tikzpicture}
\node[entity] (schulklasse) {Schulklasse};
\end{tikzpicture}

\item[Relationships:] Beziehungen zwischen den Werten

\item[Relationship-Typen:] Beziehungen gleicher Art

\begin{tikzpicture}
\node[relationship,align=center] (hatKlassenleitung) {hatKlassen-\\leitungIn};
\end{tikzpicture}

\item[Attributewerte:] Werte der Attribute

\item[Attribute:] Eigenschaften von Entities oder Relationships

\begin{tikzpicture}
\node[attribute] (klassenzimmer) {Klassenzimmer};
\end{tikzpicture}
\end{description}

Dabei gilt:

\begin{enumerate}
\def\labelenumi{\arabic{enumi}.}
\item nicht disjunkt: ein Element mehrer Entity-Typen sein.
\item Stelligkeit eines Relationship-Typs
\item Entity-Typ kann mehrere Relationship-Typen haben
\item Schemaebene: Entity-Typ, Relationship-Typ, Attribute; Instanz: Entity,
Relationship, Attributwert
\item Modellierung auf der Schemaebene, keine konkreten Entites
\end{enumerate}

\subsection{Rollennamen}

Zur genaueren Charakterisierung von Entity-Typen und Relationship-Typen

\subsection{Domäne}

zulässige Attributewerte

\begin{itemize}
\item extensional: Aufzählung alles zulässigen Werte
\item intensional: Angabe allgemein bekannter Mengen
\end{itemize}

Mehrwertiges Attribut (Doppelkreis): mehrere Telefonnummern
Abgeleitetes Attribut (gestrichelte Linie): Alter (vom Geburstag)

Relationshiptypen:

\begin{itemize}
\item Binäre zwischen 2 Entitytypen
\item Ternäre zwischen 3 Entitytypten
\end{itemize}

%-----------------------------------------------------------------------
%
%-----------------------------------------------------------------------

\section{Generalisierung\footcite[Seite 27]{db:fs:1}}

Die Generalisierung ist eine \memph{Abstraktion auf Ebene der
Entitytypen}, um eine bessere Strukturierung zu erzielen.
%
Die \memph{Eigenschaften ähnlicher Entitytypen} werden einem
\memph{gemeinsamen Obertyp} zugeordnet. Die Eigenschaften (Attribute),
die nicht von den generalisierten Entitytypen geteilt werden, verbleiben
in den Untertypen. Der \memph{Untertyp stellt somit Spezialisierung} des
Obertyps dar.
%
Der \memph{Untertyp erbt alle Eigenschaften} des Obertyps. Die
Entitymenge des Untertyps ist eine Teilmenge der Entitymenge des
Obertyps.
%
Man spricht von einer \memph{disjunkten Spezialisierung}, wenn eine
Entity nur Mitglied von einer der Untertypen ist und es keine
Überschneidungen gibt.
%
\memph{Vollständige Spezialisierung} nennt man die Modellierung, wenn es
keine direkten Elemente des Obertyps gibt. Alle Entitys, die zur Menge
des Obertyps gehören, gehören auch zu einem Untertyp. Der Obertyp ist
Vereinigung der Untertypen.
% https://www.geeksforgeeks.org/generalization-specialization-and-aggregation-in-er-model/
Gezeichnet wird die Generalisierung meist durch ein \memph{Dreieck}.
In manchen Diagrammen stellt ein nach \memph{oben} zeigendes Dreieck die
\memph{Generalisierung}, ein nach unten gerichtetes Dreieick die
\memph{Spezialisierung} dar.

\begin{center}
\begin{tikzpicture}[node distance=1.5cm]
\node[entity] (Person) {Person};
\node[isa,below=of Person] (ISA) {ISA} edge (Person);
\node[entity,below left=of ISA] {Lehrer} edge (ISA);
\node[entity,below right=of ISA] {Schüler} edge (ISA);
\end{tikzpicture}
\end{center}

%-----------------------------------------------------------------------
%
%-----------------------------------------------------------------------

\section{Schwache Entitytypen}

Ein schwacher Entitytypen ist ein Entitytyp, der in seiner Existenz von
einem \memph{anderen Entitytyp abhängig} ist und oft \memph{nur in
Kombination} mit dem Schlüssel des \memph{übergeordneten Entitytyps}
eindeutig identifiziert werden kann.

Eine \memph{Totale Teilnahme} (auch \emph{totale Partizipation} oder
\emph{totale Beteiligung} genannt) liegt dann vor, wenn jede Entität eines
schwachen Entitättypen im Beziehung mit dem übergeordneten Entitytyps steht.

Bespielsweise  kann es keine Entity des Entitytyps \emph{Klassenzimmer}
geben, die keine Beziehung zu einem Entity des Typs \emph{Schulgebäude}
hat. Das bedeutet auch, dass jedes \emph{Schulgebäude} mindestens einen
\emph{Klassenzimmer} hat. Die Nummer eines Klassenzimmers ist nur
innerhalb eines Schulgebäudes eindeutig. Der Schlüssel lautet dann:
\texttt{Klassenzimmers.Nummer} und \texttt{Schulgebäude.Nummer}.

Die Beziehung zwischen \emph{starken} und \emph{schwachem} Typ ist immer
eine \texttt{1:N}-Bezieh\-ung oder \texttt{1:1} in seltenen Fällen.

\footcite[Seite 26]{db:fs:1}

\begin{center}
\begin{tikzpicture}[node distance=1.5cm]
\node[entity] (Schulgebäude) at (0,0) {Schulgebäude};
\node[attribute] (Nummer) [above of=Schulgebäude] {\key{Nummer}} edge (Schulgebäude);
\node[attribute] (Höhe) [below of=Schulgebäude] {Höhe} edge (Schulgebäude);

\node[weak entity] (Klassenzimmer) at (6,0) {Klassenzimmer};
\node[attribute] (Nummer) [above of=Klassenzimmer] {\discriminator{Nummer}} edge (Klassenzimmer);
\node[attribute] (Größe) [below of=Klassenzimmer] {Größe} edge (Klassenzimmer);

\node[ident relationship] (liegtIn) at (3,0) {liegtIn}
  edge (Schulgebäude)
  edge[weak] (Klassenzimmer);
\end{tikzpicture}
\end{center}

%-----------------------------------------------------------------------
%
%-----------------------------------------------------------------------

\section{Funktionalitäten}

Die (min,max)-Notation zählt die Ausprägung von \emph{Beziehungen},
während die anderen Notationen \emph{Entitätstypausprägungen} zählen.
(Wikipedia)

\subsection{einfache / Chen-Notation}

Quellen
\footcite[2.7.1 Seite 41]{kemper}
\footcite[Seite 59]{brinda}

\begin{description}
\item[Schreibweise] 1:1, 1:n, n:1, n:m
\item[Bestimmung]

Auf zu bestimmenden Entitytyp zeigen und Frage formulieren:
%
\emph{„Wie viele X Entities sind in Relationship mit (einem) anderem/n
Entity/ies?“}
%
Das zu bestimmtende Entity ist diesen Fragensätzen \textbf{Objekt}.

\end{description}

\subsection{min-max-Notation / Kardinalitäten}

Quellen
\footcite[2.7.3 Seite 46]{kemper}
\footcite[Seite 62]{brinda}

\begin{description}
\item[Schreibweise] (0, *)
\item[Bestimmung]

Auf zu bestimmenden Entitytyp zeigen und Aussage formulieren:
%
\emph{„Ein Entity ist in Relationship mit X (min, max) anderem/n
Entity/ies“.}
%
Das zu bestimmtende Entity ist diesen Aussagesatz \textbf{Subjekt}.
\end{description}

%%%%%%%%%%%%%%%%%%%%%%%%%%%%%%%%%%%%%%%%%%%%%%%%%%%%%%%%%%%%%%%%%%%%%%%%
% Aufgaben
%%%%%%%%%%%%%%%%%%%%%%%%%%%%%%%%%%%%%%%%%%%%%%%%%%%%%%%%%%%%%%%%%%%%%%%%

\chapter{Aufgaben}

%-----------------------------------------------------------------------
%
%-----------------------------------------------------------------------

\ExamensAufgabeTTA 66116 / 2018 / 03 : Thema 2 Teilaufgabe 1 Aufgabe 2

%-----------------------------------------------------------------------
% Freizeitparks
%-----------------------------------------------------------------------

%-----------------------------------------------------------------------
% Mitarbeiter / Abteilung / Projekt
%-----------------------------------------------------------------------

\section{Aufgabe 3: Relationenmodell Einstieg\footcite[Seite 2-3]{db:pu:1}}

\begin{tikzpicture}
\node[entity] (Abteilung) {Abteilung};
\node[attribute,above left=0.5cm and -0.5cm of Abteilung] {\key{AbtID}}
  edge (Abteilung.north);
\node[attribute,above right=0.5cm and -0.5cm of Abteilung] {AName}
  edge (Abteilung.north);

\node[relationship,right=of Abteilung] (gehörtZu) {gehörtZu}
  edge node[auto]{1} (Abteilung);

\node[entity,right=of gehörtZu] (Mitarbeiter) {Mitarbeiter}
  edge node[auto]{n} (gehörtZu);
\node[attribute,above right=0cm and 1cm of Mitarbeiter] {\key{PerID}}
  edge (Mitarbeiter.east);
\node[attribute,below right=0cm and 1cm of Mitarbeiter] {MName}
  edge (Mitarbeiter.east);

\node[relationship,above=1cm of Mitarbeiter] (istChefVon) {istChefVon}
  (istChefVon.east) edge node[auto,pos=0.1]{Untergebener} node[auto,swap,pos=0.4]{n} (Mitarbeiter.north east)
  (istChefVon.west) edge node[auto,swap,pos=0.3]{Chef} node[auto,pos=0.7]{1} (Mitarbeiter.north west);
\node[relationship,below left=1cm and 0cm of Mitarbeiter] (arbeitetMit) {arbeitetMit}
  (arbeitetMit.north) edge node[auto]{m} (Mitarbeiter.south);
\node[relationship,below right=1cm and 0cm of Mitarbeiter] (verantFür) {verantFür}
  (verantFür.north) edge node[auto]{1} (Mitarbeiter.south);

\node[entity,node distance=3cm,below=of Mitarbeiter] (Projekt) {Projekt}
edge node[auto]{n} (arbeitetMit.south) edge node[auto]{1} (verantFür.south);
\node[attribute,above left=-0.25cm and 1cm of Projekt] {\key{ProjID}}
  edge (Projekt.west);
\node[attribute,below left=-0.25cm and 1cm of Projekt] {PName}
  edge (Projekt.west);
\end{tikzpicture}

\begin{enumerate}

%%
% (a)
%%

\item Übertragen Sie das gegebene ER-Modell in ein
relationales Schema! Geben Sie in geeigneter Weise Schlüssel an.

%%
% (b)
%%

\item Verfeinern Sie das Relationenschema!

\begin{antwort}
\begin{minted}{md}
Abteilung (id:AbtId, A_Name)
Mitarbeiter (id:PersID, M_Name, AbtID[Abteilung], Chef[Mitarbeiter])
Projekt (id:ProjID, P_Name, Verantwortlicher (PersId)[Mitarbeiter])
arbeitet_mit (ProjId[Projekt], PersID[Mitarbeiter]) Schlüsselkandiaten: beide
\end{minted}
\end{antwort}

\end{enumerate}

%-----------------------------------------------------------------------
% Zirkus
%-----------------------------------------------------------------------

\section{Aufgabe 4: Relationenmodell}

\begin{enumerate}

%%
% (a)
%%

\item Überführen Sie das ER-Diagramm der Zirkusverwaltung in
ein verfeinertes Relationenschema. Geben Sie die Fremdschlüssel an.

\end{enumerate}

%-----------------------------------------------------------------------
%
%-----------------------------------------------------------------------

\ExamensAufgabeTTA 46116 / 2013 / 03 : Thema 1 Teilaufgabe 2 Aufgabe 1

\ExamensAufgabeTTA 46116 / 2018 / 09 : Thema 1 Teilaufgabe 2 Aufgabe 2

%-----------------------------------------------------------------------
% Aufgabe 3
%-----------------------------------------------------------------------

\section{Aufgabe 3: DBTec}

Die Firma \emph{DBTec} fertigt verschiedene Geräte. Für die betriebliche
Organisation dieser Firma soll eine relationale Datenbank eingesetzt
werden. Dabei gilt folgendes:
% Entity: Bauteil
Jedes \mpEntity{Bauteil}, das verwendet wird, hat eine eindeutige
\mpAttribute{Nummer} und eine \mpAttribute{Bezeichnung}, die allerdings
für mehrere verschiedene Bauteile gleich sein kann. Von jedem Teil
werden außerdem der \mpAttribute{Name des Herstellers}, der
\mpAttribute{Einkaufspreis} pro Stück und der am Lager vorhandene
\mpAttribute{Vorrat} gespeichert.

% Entity: Gerät
Jedes herzustellende \mpEntity{Gerät} hat eine eindeutige
\mpAttribute{Bezeichnung}. Auch von jedem schon gefertigten Gerätetyp
soll der aktuelle \mpAttribute{Lagerbestand} gespeichert werden, ebenso
wie der \mpAttribute{Verkaufspreis} des Gerätes. In unserem fiktiven
Betrieb gilt die Regelung, dass Maschinen, die mehr als 1000,- EUR
kosten, unentgeltlich an die Kunden ausgeliefert werden; für Geräte, die
weniger kosten, ist zusätzlich zum Preis eine gerätespezifische
\mpAttribute{Anliefergebühr} zu entrichten. In der Datenbank ist
ebenfalls zu speichern, welche Bauteile für welche Geräte
\mpRelationship{benötigt} werden. Es gibt Bauteile, die für mehrere
Geräte verwendet werden.

% Entity: Kunde
Von jedem \mpEntity{Kunden} werden der \mpAttribute{Name}, die
\mpAttribute{Adresse} und die \mpAttribute{Branche} gespeichert. Es kann
verschiedene Kunden mit demselben Namen oder derselben Adresse geben.
% Relationship: betreutKunde
Außerdem ist zu jedem Kunden vermerkt, wer aus unserer Firma für die
entsprechende \mpRelationship{Kundenbetreuung zuständig} ist.
% Relationship: beliefert
Natürlich ist auch zu speichern, welche Kunden mit welchen Geräten
\mpRelationship{beliefert} werden. Es kann sein, dass gewissen Kunden
für bestimmte Geräte \mpAttribute{Sonderkonditionen} eingeräumt worden
sind, dies soll ggf. ebenfalls in der Datenbank vermerkt werden.
\footcite{db:ab:1}

\begin{enumerate}

%%
% (a)
%%

\item Bestimmen Sie die Entity- und die Relationship-Typen mit ihren
Attributen und zeichnen Sie ein mögliches Entity-Relationship-Diagramm!

%%
% (b)
%%

\item Bestimmen Sie zu allen Entity-Typen einen Primärschlüssel und
tragen Sie diese in das Modell ein.

%%
% (c)
%%

\item Bestimmen Sie die Funktionalitäten (1:1, 1:n, n:m) der
Relationship-Typen und tragen Sie diese in das Modell ein.

%%
% (d)
%%

\item In der Firma wird ein neues Betreuungssystem eingeführt. Jeder
\mpEntity{Kundenbetreuer} ist für die Kunden eines festgelegten Bezirks
\mpRelationship{zuständig}. Die \mpEntity{Bezirke} sind
\mpAttribute{durchnummeriert}. Für jeden Bezirk existiert eine
\mpAttribute{Beschreibung}, die nicht näher festgelegt ist. Erweitern
Sie Ihr ER-Modell aus Teilaufgabe a) entsprechend. Bezirke werden nur
festgelegt, wenn es dazu auch Kunden gibt.

\begin{antwort}[falsch]
%\includegraphics[width=\linewidth]{Aufgabe_3}
\end{antwort}

\begin{antwort}[muster]
%\includegraphics[width=\linewidth]{Aufgabe_3_Muster}
\end{antwort}

\end{enumerate}

%-----------------------------------------------------------------------
%
%-----------------------------------------------------------------------

\section{Fahrzeugverwaltung}

Gegeben ist das folgende ER-Modell der Fahrzeugverwaltung einer Firma:
\footcite[Seite 2-3, Aufgabe 4: Fahrzeugverwaltung]{db:ab:1}

\begin{tikzpicture}[scale=0.8, transform shape]
\node[entity] (Fahrer) at (0,0) {Fahrer};
\node[entity] (Fahrzeug) at (5,0) {Fahrzeug};
\node[entity] (Abteilung) at (10,0) {Abteilung};
\node[entity] (Garage) at (5,-4) {Garage};

\node[relationship,align=center] (Fahrerlaubnis) at (2.5,0) {Fahrer-\\laubnis}
  edge (Fahrer)
  edge (Fahrzeug);

\node[relationship] (gehoert) at (7.5,0) {gehoert}
  edge (Fahrzeug)
  edge (Abteilung);

\node[relationship] (stehtIn) at (5,-2) {stehtIn}
  edge (Fahrzeug)
  edge (Garage);
\end{tikzpicture}

\noindent
Die Attribute wurden aus Einfachheitsgründen weggelassen. Es gelten
folgende Bedingungen:

\begin{itemize}

\item Jedes Fahrzeug gehört zu höchstens einer Abteilung, wobei aber
jede Abteilung mindestens ein Fahrzeug hat.

\item Für fast alle Fahrzeuge gibt es eine (fest zugeordnete)
Einzelgarage. Jede dieser Garagen ist belegt.

\item Für jedes Fahrzeug muss es mindestens drei Personen mit einer
entsprechenden Fahrerlaubnis geben. Ansonsten gibt es keine
Einschränkung.
\end{itemize}

\begin{enumerate}

%%
%
%%

\item Geben Sie gemäß obiger Bedingungen geeignete Funktionalitäten in
der (min, max)-Notation an.

\begin{tikzpicture}[scale=0.8, transform shape]
\node[entity] (Fahrer) at (0,0) {Fahrer};
\node[entity] (Fahrzeug) at (6,0) {Fahrzeug};
\node[entity] (Abteilung) at (12,0) {Abteilung};
\node[entity] (Garage) at (6,-4) {Garage};

\node[relationship,align=center] (Fahrerlaubnis) at (3,0) {Fahrer-\\laubnis}
  edge node[auto,swap] {(0,*)} (Fahrer)
  edge node[auto,swap] {(3,*)} (Fahrzeug);

\node[relationship] (gehoert) at (9,0) {gehoert}
  edge node[auto,swap] {(0,1)} (Fahrzeug)
  edge node[auto,swap] {(1,*)} (Abteilung);

\node[relationship] (stehtIn) at (6,-2) {stehtIn}
  edge node[auto,swap] {(0,1)} (Fahrzeug)
  edge node[auto,swap] {(1,1)} (Garage);
\end{tikzpicture}

%%
%
%%

\item Wie lauten die entsprechenden Funktionalitätsangaben (z.B. 1:1,
n:m etc.)?

\begin{tikzpicture}[scale=0.8, transform shape]
\node[entity] (Fahrer) at (0,0) {Fahrer};
\node[entity] (Fahrzeug) at (6,0) {Fahrzeug};
\node[entity] (Abteilung) at (12,0) {Abteilung};
\node[entity] (Garage) at (6,-4) {Garage};

\node[relationship,align=center] (Fahrerlaubnis) at (3,0) {Fahrer-\\laubnis}
  edge node[auto,swap] {n} (Fahrer)
  edge node[auto,swap] {m} (Fahrzeug);

\node[relationship] (gehoert) at (9,0) {gehoert}
  edge node[auto,swap] {n} (Fahrzeug)
  edge node[auto,swap] {1} (Abteilung);

\node[relationship] (stehtIn) at (6,-2) {stehtIn}
  edge node[auto,swap] {1} (Fahrzeug)
  edge node[auto,swap] {1} (Garage);
\end{tikzpicture}
\end{enumerate}

%-----------------------------------------------------------------------
%
%-----------------------------------------------------------------------

\ExamensAufgabeTTA 46116 / 2018 / 09 : Thema 2 Teilaufgabe 2 Aufgabe 5

%-----------------------------------------------------------------------
%
%-----------------------------------------------------------------------

\ExamensAufgabeTTA 66116 / 2016 / 03 : Thema 1 Teilaufgabe 1 Aufgabe 1

%-----------------------------------------------------------------------
%
%-----------------------------------------------------------------------

\ExamensAufgabeTTA 46116 / 2015 / 03 : Thema 1 Teilaufgabe 2 Aufgabe 3

\literatur
\end{document}
