\documentclass{lehramt-informatik}
\InformatikPakete{er,rmodell}
\begin{document}

%-----------------------------------------------------------------------
%
%-----------------------------------------------------------------------

\chapter{Das Relationenmodell}

\begin{quellen}
\item \cite[Überführung in ein relationales Modell]{wiki:entity-relationship-modell}
\item \cite{wiki:relational-model}
\end{quellen}

%-----------------------------------------------------------------------
%
%-----------------------------------------------------------------------

\section{Grundbegriffe des Relationenmodells\footcite[Seite 76]{winter}}

Objekte und Beziehungen besitzen Eigenschaften, die \memph{Attribute}
genannt werden.
%
Diese Attribute sind für jedes Objekt bzw. jede Beziehung durch
\memph{Attributwerte} konkretisiert.
%
Die Menge aller möglichen Attributwerte eines Attributs heißt
\memph{Domäne}.
%
Objekte bzw. Beziehungen werden durch \memph{Tupel} d.h. Listen
der entsprechenden Attributwerte, dargestellt.
%
\memph{Gleichartige Objekte} werden durch \memph{gleichartig
aufgebaute Tupel} repräsentiert. Diese Tupel werden zu einer Menge, der
sogenannten \memph{Relation}, zusammengefasst.
%
Während aber beim Entity-Relation\-ship-Modell zwei
Strukturierungskonzepte in Form von Entity- und Relationship-Typ zur
Verfügung gestellt werden, besitzt das Relationenmodell mit der Relation
lediglich \memph{ein Konzept}.

Eine Relation ist eine \memph{Menge gleichartig gebauter Tupel}. Diese
Eigenschaft unterscheidet eine Relation grundsätzlich von einer Tabelle.
Aus der Mengeneigenschaft der Relation ergeben sich folgende wichtige
Konsequenzen:
%
Es existieren \memph{keine Tupelduplikate}, d.\,h. es gibt zu keinem
Zeitpunkt zwei Tupel mit identischen Attributwerten. Die Tupel sind
nicht geordnet, d.h. es existiert \memph{keine festgelegte Reihenfolge}
der Elemente.\footcite[Seite 77]{winter}

%-----------------------------------------------------------------------
%
%-----------------------------------------------------------------------

\section{Vom ER-Modell zum Relationenmodell\footcite[Seite 33]{db:fs:1}}

Für jeden Entity-Typen wird ein \memph{Relationenschema} festgelegt.
%
Für jeden Relation\-ship-Typen wird (zunächst) ein Relationenschema
festgelegt, das zusätzlich zu den Attributen des Relationship-Typs die
\memph{Schlüssel der zugehörigen Entity-Typen} enthält, welche
\memph{Fremdschlüssel} genannt werden. Das/die Schlüsselattribut/e
werden/wird unterstrichen.
%
Ein Relationenschema hat folgenden Aufbau:

\begin{center}
\texttt{Relationenname\{[Attribut1: Domäne1, Attribut2: Domäne2, ...])}
\end{center}

\noindent
Es gibt eine vereinfachte Notation:

\begin{center}
\texttt{Relationenname(Attribut1, Attribut2, Attribut3[Fremdrelation])}
\end{center}

%-----------------------------------------------------------------------
%
%-----------------------------------------------------------------------

\section{Verfeinertes Relationenschema\footcite[Seite 34]{db:fs:1}}

Jedes Relationenmodell kann \memph{verfeinert} werden, indem Relationen,
deren Schlüssel aus den \memph{gleichen Attributen} besteht,
\memph{zusammengefasst} werden.
Grundsätzlich können \memph{nur} Relationen eliminiert werden,
die \memph{1:1 bzw. 1:N}-Beziehungen repräsentieren.

\begin{description}
\item[1:N-Beziehungen] Integration in Relation der N-Seite (ansonsten
ergeben sich Anomalien!).

\item[1:1-Beziehungen] Integration in beide Relationen möglich, aber:
Abwägen, wo mehr „Lehrstellen“ in Form von NULL-Werten entstehen.
\end{description}

\noindent
Nur Relationen, die aus \memph{Relationship-Typen} entstanden sind,
dürfen \memph{eliminiert} werden! Diese Relationen dürfen nur dann
eliminiert werden, wenn die in ihnen enthaltene Information in eine
andere Relation vollständig integriert werden kann!

Die genaue Überführung, die automatisiert werden kann, erfolgt in 7
Schritten:

\begin{description}
\item[Starke Entitätstypen:] Für jeden starken Entitätstyp wird eine
Relation mit seinen Attributen und seinen Primärschlüssel erstellt.

\item[Schwache Entitätstypen:] Für jeden schwachen Entitätstyp wird eine
Relation erstellt. Der Primärschlüssel des starken Entitätstyps wird als
Fremdschlüssel ergänzt, um den schwachen Entitätstyp zu identifizieren.

\item[1:1-Beziehungstypen:] Für einen 1:1-Beziehungstyp zweier
Entitätstypen wird eine der beiden Relationen um den Fremdschlüssel
für die jeweils andere Relation erweitert.

\item[1:N-Beziehungstypen:] Für den 1:N-Beziehungstyp zweier
Entitätstypen wird der Fremdschlüssel der „1“-Relation in die „N
“-Relation geschrieben.

\item[N:M-Beziehungstypen:] Für jeden N:M-Beziehungstyp wird eine neue
Relation mit den Fremschlüsseln der beiden Relation ergänzt.

\item[Mehrwertige Attribute:] Für jedes mehrwertige Attribut wird eine
Relation erstellt und als Fremdschlüssel der Primärschlüssel des
zugehörigen Entitätstyps verwendetn.1

\item[n-äre Beziehungstypen:] Für jeden Beziehungstyp mit einem Grad $n
> 2$ wird eine neue Relation erstellt.\footcite[Überführung in ein
relationales Modell]{wiki:entity-relationship-modell}

\end{description}

%%%%%%%%%%%%%%%%%%%%%%%%%%%%%%%%%%%%%%%%%%%%%%%%%%%%%%%%%%%%%%%%%%%%%%%%
% Aufgaben
%%%%%%%%%%%%%%%%%%%%%%%%%%%%%%%%%%%%%%%%%%%%%%%%%%%%%%%%%%%%%%%%%%%%%%%%

\chapter{Aufgaben}

%-----------------------------------------------------------------------
%
%-----------------------------------------------------------------------

\section{Aufgabe 5: ER-Tutor\footcite{db:ab:1}}

\begin{center}
\begin{tikzpicture}[er2]
\node[entity] (kurs) {Kurs};
\node[attribute,far,left of=kurs] {Kursname} edge(kurs);
\node[attribute,right of=kurs] {KursNr} edge(kurs);

\node[relationship,below of=kurs,node distance=6em] (betreut) {betreut} edge node[auto] {1} (kurs);
\node[attribute,far,left of=betreut]{Sprechstunde} edge(betreut);

\node[entity,below of=betreut,node distance=6em] (tutor) {Tutor} edge node[auto] {1} (betreut);
\node[attribute,left of=tutor] {PersNr} edge(tutor);
\node[attribute,right of=tutor] {Büro} edge(tutor);
\node[attribute,below of=tutor,node distance=4em] {Name} edge(tutor);
\end{tikzpicture}
\end{center}

Beim obenstehenden ER-Modell gilt:

\begin{itemize}
\item Ein Tutor kann durch seine Personalnummer oder durch die
Kombination aus Name und Büro(-nummer) identifiziert werden.

\item Bei den Kursen ist sowohl Kursname als auch Kursnummer eindeutig.
\end{itemize}

\begin{enumerate}

%%
% (a)
%%

\item Überführen Sie die Entity-Typen Kurs und Tutor in entsprechende
Relationen. Legen Sie für jede der Relationen einen Primärschlüssel
fest.

\begin{antwort}
\begin{rmodell}
Kurs (\primaer{KursNr}, Kursname)

Tutor (\primaer{PersNr}, Name, Buero)
\end{rmodell}
\end{antwort}

%%
% (b)
%%

\item Für die Konvertierung des Relationship-Typen betreut gibt es –
unabhängig von den gewählten Primärschlüsseln in Aufgabe a – mehrere
Möglichkeiten. Geben Sie alle möglichen Relationen (mit Festlegung des
Primärschlüssels) an.

\begin{antwort}
\begin{rmodell}
betreut (\primaer{PersNr}, KursNr, Sprechstunde)

betreut (PersNr, \primaer{KursNr}, Sprechstunde)

betreut (\primaer{PersNr}, Kursname, Sprechstunde)

betreut (PersNr, \primaer{Kursname}, Sprechstunde)

betreut (\primaer{Name, Büro}, KursNr, Sprechstunde)

betreut (Name, Büro, \primaer{KursNr}, Sprechstunde)

betreut (\primaer{Name, Büro}, Kursname, Sprechstunde)

betreut (Name, Büro, \primaer{Kursname}, Sprechstunde)
\end{rmodell}
\end{antwort}
\end{enumerate}

%-----------------------------------------------------------------------
%
%-----------------------------------------------------------------------

\section{Überführen Sie folgendes ER-Diagramm in ein (verfeinertes) Relationenschema!
\footcite{db:ab:7}}

\begin{tikzpicture}[nearer]
\node[entity] (Station) {Station};
\node[attribute,above left=0.5cm of Station] {\key{Nr}}
  edge (Station.west);
\node[attribute,left=0.5cm of Station] {Name}
  edge (Station.west);
\node[attribute,below left=0.5cm of Station] {Bettenanzahl}
  edge (Station.west);

\node[relationship,above right=of Station] (leitet) {leitet}
  edge node[auto]{(1,1)} (Station);

\node[relationship,below right=of Station] (arbeitetAuf) {arbeitetAuf}
  edge node[auto]{(1,n)} (Station);

\node[entity,far,right=of Station] (Arzt) {Arzt}
  edge node[auto]{(0,1)} (leitet)
  edge node[auto]{(1,1)} (arbeitetAuf);
  \node[attribute,above right=0.5cm of Arzt] {\key{PersNr}}
    edge (Arzt.east);
  \node[attribute,right=0.5cm of Arzt] {Name}
    edge (Arzt.east);
  \node[attribute,below right=0.5cm of Arzt] {Fachrichtung}
    edge (Arzt.east);

\node[relationship,below=of Station] (liegtAuf) {liegtAuf}
  edge node[auto]{(0,*)} (Station);

\node[entity,below=1cm of liegtAuf] (Patient) {Patient}
  edge node[auto]{(1,1)} (liegtAuf);
  \node[attribute,above left=0.5cm of Patient] {\key{PatNr}}
    edge (Patient.west);
  \node[attribute,left=0.5cm of Patient] {Name}
    edge (Patient.west);
  \node[attribute,below left=0.5cm of Patient] {Adresse}
    edge (Patient.west);
  \node[attribute,below=1cm of Patient,text width=2cm,align=center] {Kranken\-versicherung}
    edge (Patient);

\node[relationship,below=of Arzt] (behandelt) {behandelt}
  edge node[auto,swap]{(0,*)} (Arzt)
  edge node[auto,swap]{(1,*)} (Patient);

\node[entity,far,below=1cm of behandelt] (Behandlung) {Behandlung}
  edge node[auto]{(0,*)} (behandelt);
  \node[attribute,above right=0.5cm of Behandlung] {\key{Code}}
    edge (Behandlung.east);
  \node[attribute,right=0.5cm of Behandlung] {Name}
    edge (Behandlung.east);
  \node[attribute,below right=0.5cm of Behandlung] {Behandlungspunkte}
    edge (Behandlung.east);

\node[relationship,below left=0.8cm of Behandlung] (benötigt) {benötigt}
  edge node[auto]{(0,*)} (Behandlung);

\node[entity,far,below=1.5cm of Behandlung] (Gerät) {Gerät}
  edge node[auto]{(1,*)} (benötigt);
  \node[attribute,right=0.5cm of Gerät] {\key{Typ}}
    edge (Gerät);
  \node[attribute,below=0.5cm of Gerät] {Einsatzkosten}
    edge (Gerät);
\end{tikzpicture}

\literatur

\end{document}
