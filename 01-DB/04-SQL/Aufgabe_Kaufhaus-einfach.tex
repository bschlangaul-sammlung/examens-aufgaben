\documentclass{lehramt-informatik-aufgabe}
\liLadePakete{syntax}
\begin{document}
\liAufgabenTitel{Kaufhausdatenbank (einfach)}
\section{Kaufhausdatenbank - einfacher Einstieg
\index{SQL}
\footcite{db:ab:2}
}

Die relationale Datenbank eines Kaufhauses enthält folgende Tabelle mit
dem Namen „Artikel“:

\begin{tabular}{llll}
ArtNr & Bezeichnung  & Verkaufspreis & Einkaufspreis \\
95    & Kamm         & 1.25          & 0.80          \\
97    & Kamm         & 0.99          & 0.75          \\
507   & Seife        & 3.93          & 2.45          \\
1056  & Zwieback     & 1.20          & 0.90          \\
1401  & Räucherlachs & 4.90          & 3.60          \\
2045  & Herrenhose   & 37.25         & 24.45         \\
2046  & Herrenhose   & 20.00         & 17.00         \\
2340  & Sommerkleid  & 94.60         & 71.50
\end{tabular}

Formulieren Sie mit Hilfe von SQL folgende Anfragen:

\begin{enumerate}

%%
% (a)
%%

\item Gesucht sind alle Informationen über Herrenhose und Sommerkleid!

\begin{antwort}
\begin{minted}{sql}
SELECT *
FROM Artikel
WHERE
  Bezeichnung = 'Herrenhose' OR
  Bezeichnung = 'Sommerkleid';
\end{minted}
\end{antwort}

%%
% (b)
%%

\item Welche Artikelnummer hat der Zwieback?

\begin{antwort}[muster]
\begin{minted}{sql}
SELECT ArtNr
FROM Artikel
WHERE
  Bezeichnung = 'Zwieback';
\end{minted}
\end{antwort}

%%
% (c)
%%

\item Welche Waren (Artikelnummer und Verkaufspreis) werden für mehr als
25€ verkauft?

\begin{antwort}
\begin{minted}{sql}
SELECT ArtNr, Verkaufspreis
FROM Artikel
WHERE Verkaufspreis > 25.00;
\end{minted}
\end{antwort}

%%
% (d)
%%

\item Welche Artikel (Angabe der Bezeichnung) bietet das Kaufhaus an?

\begin{antwort}
\begin{minted}{sql}
SELECT DISTINCT Bezeichnung
FROM Artikel;
\end{minted}
\end{antwort}

%%
% (e)
%%

\item Gesucht sind die Artikelnummern aller Artikel mit Ausnahme der
Artikelnummer 2046.

\begin{antwort}
\begin{minted}{sql}
SELECT ArtNr
FROM Artikel
WHERE NOT (ArtNr = 2046);
\end{minted}
\end{antwort}

%%
% (f)
%%

\item Gib die Artikelnummern und die Verkaufspreise aller Herrenhosen
aus, die für höchstens 25€ verkauft werden! Der Spaltenname für die
Verkaufspreise soll in der Ergebnistabelle „Sonderangebot“ heißen.

\begin{antwort}
\begin{minted}{sql}
SELECT ArtNr, Verkaufspreis AS Sonderangebot
FROM Artikel
WHERE Bezeichnung = 'Herrenhose' AND Verkaufspreis <= 25;
\end{minted}
\end{antwort}

%%
% (g)
%%

\item Gib Artikelnummer und Verkaufspreis aller Artikel aus, die im
Einkauf zwischen 80 Cent und 5€ kosten.

\begin{antwort}
\begin{minted}{sql}
SELECT ArtNr, Verkaufspreis
FROM Artikel
WHERE Einkaufspreis BETWEEN 0.80 AND 5.00;
\end{minted}
\end{antwort}

\end{enumerate}

\begin{minted}{sql}
-- AB 2 AB 7

-- sudo mysql  < Kaufhaus.sql
-- DROP DATABASE IF EXISTS Kaufhaus;
-- CREATE DATABASE Kaufhaus;
-- USE Kaufhaus;

CREATE TABLE Artikel (
  ArtNr INTEGER PRIMARY KEY NOT NULL,
  Bezeichnung VARCHAR(100) NOT NULL,
  Verkaufspreis FLOAT(2),
  Einkaufspreis FLOAT(2)
);

CREATE TABLE Abteilung (
  Abteilungsname VARCHAR(60) NOT NULL,
  Stockwerk VARCHAR(10) NOT NULL,
  Abteilungsleiter VARCHAR(100),
  PRIMARY KEY (Abteilungsname, Stockwerk)
);

CREATE TABLE Bestand (
  Abteilungsname VARCHAR(100) REFERENCES Abteilung(Abteilungsname),
  ArtNr INTEGER REFERENCES Artikel(ArtNr),
  Vorrat INTEGER,
  PRIMARY KEY (Abteilungsname, ArtNr)
);

-- Artikel

INSERT INTO Artikel VALUES (95, 'Kamm', 1.25, 0.80);
INSERT INTO Artikel VALUES (97, 'Kamm', 0.99, 0.75);
INSERT INTO Artikel VALUES (507, 'Seife', 3.93, 2.45);
INSERT INTO Artikel VALUES (1056, 'Zwieback', 1.20, 0.90);
INSERT INTO Artikel VALUES (1401, 'Räucherlachs', 4.90, 3.60);
INSERT INTO Artikel VALUES (2045, 'Herrenhose', 37.25, 24.45);
INSERT INTO Artikel VALUES (2046, 'Herrenhose', 20.00, 17.00);
INSERT INTO Artikel VALUES (2340, 'Sommerkleid', 94.60, 71.50);

-- Abteilung

INSERT INTO Abteilung VALUES ('Lebensmittel', 'I', 'Josef Kunz');
INSERT INTO Abteilung VALUES ('Lebensmittel', 'EG', 'Monika Stiehl');
INSERT INTO Abteilung VALUES ('Textilien', 'II', 'Monika Stiehl');

-- Bestand

INSERT INTO Bestand VALUES ('Lebensmittel', 1056, 129);
INSERT INTO Bestand VALUES ('Lebensmittel', 1401, 200);
INSERT INTO Bestand VALUES ('Textilien', 2045, 14);
\end{minted}

\end{document}
