\documentclass{lehramt-informatik-haupt}

\begin{document}

%%%%%%%%%%%%%%%%%%%%%%%%%%%%%%%%%%%%%%%%%%%%%%%%%%%%%%%%%%%%%%%%%%%%%%%%
% Theorie-Teil
%%%%%%%%%%%%%%%%%%%%%%%%%%%%%%%%%%%%%%%%%%%%%%%%%%%%%%%%%%%%%%%%%%%%%%%%

\chapter{Lebenszyklus eines Software-Produktes}

\begin{quellen}
\item \cite[Kapitel 2, Seite 11-45]{schatten}
\end{quellen}

%-----------------------------------------------------------------------
%
%-----------------------------------------------------------------------

\section{Grundlegende Phasen des Software-Lebenszyklusses\footcite[Seite
19]{sosy:fs:1}}

vier grundlegenden Schritte des Lebenszyklusses\footcite[Seite
13]{schatten}

\begin{itemize}
\item Software-Spezifikation

\item Design und Implementierung

\item Software-Validierung

\item Software-Evolution
\end{itemize}

Die technischen Phasen

\begin{itemize}
\item Anforderungen und Spezifikationen
\item Planung
\item Entwurf und Design
\item Implementierung und Integration
\item Betrieb und Wartung
\item Stilllegung
\end{itemize}

%-----------------------------------------------------------------------
%
%-----------------------------------------------------------------------

\section{Übergreifende Aktivitäten\footcite[Seite 11]{sosy:fs:1}}

\begin{itemize}
\item Projektmanagement (PM):

Planung, Kontrolle und Steuerung von Projekten (Aufwand, Ressourcen,
Personen, Kosten) $\rightarrow$ organisatorischer Rahmen

\item Qualitätsmanagement (QM):

qualitativer Kundennutzen ist im Vordergrund:

\begin{itemize}
\item Erfüllung, der vom Kunden gewünschten Eigenschaften eines Produktes
(ob das richtige Produkt erstellt wurde) $\rightarrow$ Validierung

\item Erfüllung der spezifizierten Eigenschaften eines Produktes
(ob das Produkt richtig erstellt wurde) $\rightarrow$ Verifikation
\end{itemize}

\end{itemize}

\section{Qualitätsmerkmale:\footcite[Seite 12]{sosy:fs:1}}

(Qualitätsmerkmale nach ISO/IEC 9126-1)

\begin{description}

\item[Funktionalität]
Welche Aufgaben sollen durch zu erstellende Software erfüllt werden?

\item[Zuverlässigkeit]
Fähigkeit verlangte Funktionalität unter gegebenen Randbedingungen in
gegebener Zeit zu erfüllen

\item[Benutzbarkeit]
Verwendbarkeit durch den Endanwender, z. B. Benutzerführung,
Erlernbarkeit

\item[Effizienz]
Leistung, die ein System mit einem minimum an Ressourcen erbringen kann,
v.a. Zeitverhalten

\item[Änderbarkeit]
Durchführbarkeit von Änderungen und Erweiterungen am Software-Produkt

\item[Übertragbarkeit]
Fähigkeit, ein Software-System in einer anderen Umgebung, z. B. auf
einer anderen Plattform, einsetzen zu können\footcite[Seite
15-16]{schatten}
\end{description}

\subsection{Lasten- und Pflichtenheft\footcite[Seite 16]{sosy:fs:1}}

Der \memph{Auftraggeber} beschreibt im
\memph{Lastenheft}\footcite{wiki:lastenheft} möglichst präzise die
Gesamtheit der Anforderungen. (\memph{Was?})

Das \memph{Pflichtenheft}\footcite{wiki:pflichtenheft} beschreibt in
konkreter Form, wie der \memph{Auftragnehmer} die Anforderungen des
Auftraggebers zu lösen gedenkt. (\memph{Wie? Womit?})

Erst wenn der Auftraggeber das Pflichtenheft akzeptiert, sollte die
eigentliche Umsetzungsarbeit beim Auftragnehmer beginnen.

\subsubsection{Gliederung Pflichtenheft\footcite[Seite
17-18]{sosy:fs:1}}

\begin{enumerate}
\item Zielbestimmung

\begin{itemize}
\item Musskriterien
\item Wunschkriterien
\item Abgrenzungskriterien
\end{itemize}

\item Produkteinsatz

\begin{itemize}
\item Anwendungsbereiche
\item Zielgruppen
\item Betriebsbedingungen
\end{itemize}

\item Produktübersicht

\begin{itemize}
\item Übersicht über die wichtigsten Anwendungsfälle
\end{itemize}

\item Produktfunktionen

\begin{itemize}
\item Konkretisierung / Detaillierung der Anwendungsfälle
\end{itemize}

\item Produktdaten

\begin{itemize}
\item Beschreibung langfristig zu speichernder Daten
\end{itemize}

\item Produktleistungen

\begin{itemize}
\item Leistungsanforderungen bzgl. Zeit / Genauigkeit an Funktionen / Daten
\end{itemize}

\item Qualitätsanforderungen

\begin{itemize}
\item Qualitätsmerkmale / -stufen z.B. bzgl. definierter Standards
\end{itemize}

\item Benutzungsoberfläche

\begin{itemize}
\item grundlegende Anforderungen z.B. Fensterlayout, Dialogstruktur, Mausbedienung
\item Benutzerrollen ggfs. Zugriffsrechte
\end{itemize}

\item Nichtfunktionale Anforderungen

\begin{itemize}
\item einzuhaltende Gesetze
\item einzuhaltende Normen
\item Plattformabhängigkeiten
\end{itemize}

\item Technische Produktumgebung

\begin{itemize}
\item Software
\item Hardware
\item Orgware („organisatorische Randbedingungen“)
\item Produktschnittstellen („Schnittstellen zu anderen Produkten“)
\end{itemize}

\item Spezielle Anforderungen an die Entwicklungsumgebung

\begin{itemize}
\item Software
\item Hardware
\item Orgware
\item Entwicklungsschnittstellen
\end{itemize}

\item Gliederung in Teilprodukte

\begin{itemize}
\item sequentiell entwickelbare Teilprodukte
\end{itemize}

\item Ergänzungen
\end{enumerate}

%-----------------------------------------------------------------------
%
%-----------------------------------------------------------------------

\section{2. Projektplanung und -steuerung\footcite[Seite 19]{sosy:fs:1}}

Das Projektmanagement muss das Projekt initial planen (Ressourcen,
Arbeitspakete etc.) und diese Planung in regelmäßigen Abständen
überprüfen.

Je detaillierter diese Informationen sind, desto genauer und
plan-getriebener kann die Entwicklung erfolgen ($\rightarrow$
Wasserfall-, V-Modell).

Bei eher vagen Vorstellungen, ungenauen Projektaufträgen oder
gewünschter hoher Flexibilität erfolgt die Planung iterativ
($\rightarrow$ agiler Ansatz) Instrumente zur
Projektplanung:\footcite[Seite 25-27]{schatten}

\begin{itemize}
\item Gantt-Diagramm
\item CPM-Netzplan
\item Petri-Netze
\end{itemize}

%%%%%%%%%%%%%%%%%%%%%%%%%%%%%%%%%%%%%%%%%%%%%%%%%%%%%%%%%%%%%%%%%%%%%%%%
% Aufgaben
%%%%%%%%%%%%%%%%%%%%%%%%%%%%%%%%%%%%%%%%%%%%%%%%%%%%%%%%%%%%%%%%%%%%%%%%

\chapter{Aufgaben}

\literatur

\end{document}
