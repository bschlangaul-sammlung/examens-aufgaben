\documentclass{lehramt-informatik-aufgabe}
\liLadePakete{}
\begin{document}
\liAufgabenTitel{Grundwissen}

\section{Aufgabe 1: Grundwissen
\footcite{sosy:ab:7}}

\begin{enumerate}

%%
% (a)
%%

\item Nennen Sie die wesentlichen Unterschiede zwischen White-Box-Testen
und Black-Box-Testen und geben Sie jeweils zwei Beispiele an.
\index{White-Box-Testing}
\index{Black-Box-Testing}

\begin{antwort}
White-Box-Tests sind \emph{strukturorientiert}, z.\,B.
Kontrollflussorientiertes Testen oder Datenflussorientiertes Testen.
Black-Box-Test sind \emph{spe\-zifi\-kations- und funktionsorientiert} (nur
Ein- und Ausgabe relevant), z. B. Äquivalenzklassenbildung oder
Grenzwertanalyse.
\end{antwort}

%%
% (b)
%%

\item Geben Sie drei nicht-funktionalorientierte Testarten an.
\index{Funktionalorienteres Testen}

\begin{antwort}
Performanztest, Lasttest, Stresstest.
\end{antwort}

%%
% (c)
%%

\item Nennen Sie die vier verschiedenen Teststufen aus dem
V-Modell\index{V-Modell} und erläutern Sie deren Ziele.

\begin{antwort}
\begin{description}
\item[Komponenten-Test:]
Fehlerzustände in Modulen finden.

\item[Integrations-Test:]
Fehlerzustände in Schnittstellen und Interaktionen finden.

\item[System-Test:]
Abgleich mit Spezifikation.

\item[Abnahme-Test:]
Vertrauen in System und nicht-funktionale Eigenschaften gewinnen.
\footcite[Seite 50, Abbildung 3.2]{schatten}
\end{description}
\end{antwort}

%%
% (d)
%%

\item Nennen Sie fünf Aktivitäten des Testprozesses.

\begin{antwort}
\begin{enumerate}
\item Testplanung und Steuerung,

\item Testanalyse und Testentwurf,

\item Testrealisierung und Testdurchführung,

\item Bewertung von Endekriterien und Bericht,

\item Abschluss der Testaktivitäten\footcite[Kapitel „5.6.2 Der
traditionelle Testprozess“ Seite 135-138]{schatten}
\end{enumerate}
\end{antwort}

\end{enumerate}
\end{document}
