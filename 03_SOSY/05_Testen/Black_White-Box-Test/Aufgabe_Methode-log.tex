\documentclass{lehramt-informatik-aufgabe}
\liLadePakete{kontrollflussgraph,syntax,spalten}
\begin{document}
\let\c=\liKontrollCode

\liAufgabenTitel{Methode „log()“}

\section{Aufgabe 3: Schleifen-Inneres-Überdeckung\footcite{sosy:ab:7}}

Gegeben sei folgende Methode und ihr Kontrollflussgraph:
\index{Kontrollflussgraph}

\begin{multicols}{2}
\liJavaDatei[firstline=4,lastline=16]{aufgaben/sosy/ab_7/Aufgabe3}

\begin{liKontrollflussgraph}[xscale=1,yscale=-0.8]
\node[knoten] at (0,0) (S) {S};
\node[knoten] at (0,1) (1) {1};
\node[knoten] at (0,2) (2) {2};
\node[knoten] at (0,3) (3) {3};
\node[knoten] at (-1,4) (4) {4};
\node[knoten] at (1,4) (5) {5};
\node[knoten] at (0,5) (6) {6};
\node[knoten] at (0,6) (7) {7};
\node[knoten] at (0,7) (E) {E};

\path (S) -- (1);
\path (1) -- (2);
\path (2) -- (3);
\path (3) -- (4);
\path[falsch] (3) -- (5);
\path (4) -- (6);
\path (5) -- (6);
\path (6) -- (7);
\path (7) -- (-2,6) -- (-2,2) -- (2);
\path[falsch] (2) -- (2,2) -- (2,7) -- (E);
\end{liKontrollflussgraph}
\end{multicols}

\begin{enumerate}

%%
% (a)
%%

\item Begründen Sie, warum der Pfad \liKontrollTextzeileKnoten{S} -
\liKontrollTextzeileKnoten{1} - \liKontrollTextzeileKnoten{2} - \liKontrollTextzeileKnoten{3} -
\liKontrollTextzeileKnoten{5} - \liKontrollTextzeileKnoten{6} - \liKontrollTextzeileKnoten{7} -
\liKontrollTextzeileKnoten{2} - \liKontrollTextzeileKnoten{E}
infeasible (= nicht überdeckbar) ist, also weshalb es keine Eingabe
gibt, unter der dieser Pfad durchlaufen werden kann.
\index{Überdeckbarkeit}

\begin{liAntwort}
Damit dieser Pfad durchlaufen werden könnte, müsste die Eingabe $a$
gleichzeitig $2$ und ungerade sein.

\begin{liKontrollflussgraph}[xscale=1,yscale=-1]
\node[knoten] at (0,0) (S) {S};

\node[knoten,pin={
  \c{int x = y; int z = 0;}
}] at (0,1) (1) {1};

\node[knoten,pin={
    [pin distance=1cm]
    5:\c{while (x > 1)}
}] at (0,2) (2) {2};

\node[knoten,pin={
  [pin distance=2cm]
  \c{if (x \% 2 == 0)}
}] at (0,3) (3) {3};

\node[knoten,pin={
  [pin distance=1cm]
  180:\c{z++; x /= 2;}
}] at (-1,4) (4) {4};

\node[knoten,pin={
  [pin distance=1cm]
  \c{x--;}
}] at (1,4) (5) {5};

\node[knoten] at (0,5) (6) {6};
\node[knoten] at (0,6) (7) {7};
\node[knoten,pin={
  -90:\c{return z;}
}] at (0,7) (E) {E};

\path (S) -- (1);
\path (1) -- (2);
\path (2) -- (3) \liBedingung{right}{x > 1};
\path (3) -- (4);
\path[falsch] (3) -- (5);
\path (4) -- (6);
\path (5) -- (6);
\path (6) -- (7);
\path (7) -- (-2,6) -- (-2,2) -- (2);
\path[falsch] (2) -- (2,2) -- (2,7) \liBedingung{right}{x <= 1} -- (E) ;
\end{liKontrollflussgraph}
\end{liAntwort}

%%
% (b)
%%

\item Geben Sie eine minimale Menge von Pfaden an, mit der eine
vollständigen Schleifen-Inneres-Überdeckung erzielt werden kann, sowie
gegebenenfalls zu jedem Pfad eine Eingabe, unter der dieser Pfad
durchlaufen werden kann.
\index{C2b Schleife-Inneres-Pfadüberdeckung (Boundary-Interior Path Coverage)}

\end{enumerate}

\end{document}
