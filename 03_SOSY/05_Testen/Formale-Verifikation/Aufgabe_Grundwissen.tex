\documentclass{lehramt-informatik-aufgabe}
\liLadePakete{mathe,tabelle,syntax}
\begin{document}
\liAufgabenTitel{Grundwissen}

\section{Aufgabe 1: Grundwissen\footcite[Seite 1]{sosy:ab:8}}

\begin{enumerate}

%%
% (a)
%%

\item Geben Sie zwei verschiedene Möglichkeiten der formalen
Verifikation\index{Formale Verifikation} an.

\begin{liAntwort}
\begin{itemize}
\item 1. Möglichkeit: formale Verifikation mittels \emph{vollständiger
Induktion} (eignet sich bei \emph{rekursiven} Programmen).

\item 2. Möglichkeit: formale Verifikation mittels
\emph{wp-Kalkül}\index{wp-Kalkül} oder
\emph{Hoare-Kalkül}\index{Hoare-Kalkül} (eignet sich bei
\emph{iterativen} Programmen).
\end{itemize}
\end{liAntwort}

%%
% (b)
%%

\item Erläutern Sie den Unterschied von partieller\index{Partielle
Korrektheit} und totaler Korrektheit\index{Totale Korrektheit}.

\begin{liAntwort}
\begin{itemize}
\item partielle Korrektheit:

Das Programm verhält sich spezifikationsgemäß, \emph{falls} es
terminiert.

\item totale Korrektheit:

Das Programm verhält sich spezifikationsgemäß und es \emph{terminiert
immer}.
\end{itemize}
\end{liAntwort}

%%
% (c)
%%

\item Gegeben sei die Anweisungssequenz $A$. Sei $P$ die Vorbedingung
und $Q$ die Nachbedingung dieser Sequenz. Erläutern Sie, wie man die
(partielle) Korrektheit dieses Programmes nachweisen kann.

\begin{liAntwort}
\begin{tabular}{p{4cm}ll}
Vorgehen & Horare-Kalkül & \text{wp}-Kalkül \\

Wenn die Vorbedingung $P$ zutrifft, gilt nach der Ausführung der
Anweisungssequenz $A$ die Nachbedingung $Q$. &

$\{P\}A\{Q\}$ &

$P \Rightarrow \text{\text{wp}}(A,Q)$

\end{tabular}
\end{liAntwort}

%%
% (d)
%%

\item Gegeben sei nun folgendes Programm:

\begin{minted}{python}
A_1
while(b):
    A_2
A_3
\end{minted}

wobei $A_1$, $A_2$, $A_3$ Anweisungssequenzen sind. Sei $P$ die
Vorbedingung und $Q$ die Nachbedingung des Programms. Die
Schleifeninvariante\index{Invariante} der while-Schleife wird mit $I$
bezeichnet. Erläutern Sie, wie man die (partielle) Korrektheit dieses
Programmes nachweisen kann.

\begin{liAntwort}
\begin{tabular}{>{\raggedright\arraybackslash}p{4cm}ll}
Vorgehen & Horare-Kalkül & \text{wp}-Kalkül \\\hline\hline

Die Invariante $I$ gilt vor Schleifeneintritt. &
$\{P\} A_1 \{I\}$ &
$P \Rightarrow \text{\text{wp}}(A_1,I)$\\\hline

$I$ ist invariant, d. h. $I$ gilt nach jedem Schleifendurchlauf.&
$\{I \land b\} A_2 \{I\}$ &
$I \land b \Rightarrow \text{\text{wp}}(A_2, I)$\\\hline

Die Nachbedingung $Q$ wird erfüllt.&
$\{I \land \neg b\} A_3 \{Q\}$ &
$I \land \neg b \Rightarrow \text{\text{wp}}(A_3, I)$\\
\end{tabular}
\end{liAntwort}

%%
% (e)
%%

\item Beschreiben Sie, welche Vorraussetzungen eine
Terminierungsfunktion\index{Terminierungsfunktion} erfüllen muss, damit
die totale Korrektheit gezeigt werden kann.
\end{enumerate}
\end{document}
