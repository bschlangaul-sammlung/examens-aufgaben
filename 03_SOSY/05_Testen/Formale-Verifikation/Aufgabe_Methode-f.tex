\documentclass{lehramt-informatik-aufgabe}
\liLadePakete{mathe,wpkalkuel,syntax}
\begin{document}
\liAufgabenTitel{Methode „f()“}

\section{Aufgabe 2:
\index{wp-Kalkül}
\text{wp}-Kalkül\footcite{sosy:ab:8}}

Gegeben sei folgendes Programm:

\begin{minted}{java}
int f(int x, int y) {
  /* P */ x = 2 * x + 1 + x * x;
  y += 7;
  if (x > 196) {
    y = 2 * y;
  } else {
    y -= 8;
    x *= 2;
  } /* Q */
  return x + y;
}
\end{minted}

Bestimmen Sie die schwächste Vorbedingung (weakest precondition), für
die die Nachbedingung $Q := (x \geq 8) \land (y \% 2 = 1)$ noch
zutrifft.

\begin{liAntwort}
\setlength{\parindent}{0pt}

\def\code#1{„\texttt{#1}“}
\def\FarbeA#1{\textcolor{orange}{#1}}
\def\FarbeB#1{\textcolor{red}{#1}}
\def\FarbeC#1{\textcolor{blue}{#1}}

\def\NebenRechnung#1{\medskip\bigskip\par\textbf{Nebenrechnung: #1}\par\medskip}

\def\AnweisungI{x=2*x+1+x*x;}
\def\AnweisungII{y+=7;}
\def\AnweisungIII{y=2*y;}
\def\AnweisungIV{y-=8;}
\def\AnweisungV{x*=2;}

\def\A{\FarbeA{\AnweisungI\AnweisungII}}
\def\B{\FarbeB{\AnweisungIII}}
\def\C{\FarbeC{\AnweisungIV\AnweisungV}}
\def\Q{(x \geq 8) \land (y \% 2 = 1)}

Mit dem Distributivgesetz der Konjugation gilt: \bigskip

$\wp{A; if(b) B; else C;}{Q} \equiv$

$\wp{A;}{b} \land \wp{A;B;}{Q}$

$\lor$%

$\wp{A;}{\neg b} \land \wp{A;C;}{Q}$

\bigskip Der tatsächliche Programmcode wird eingesetzt:\bigskip

$\wp{\A{}if(x>196)\{\B\}else\{\C\}}{\Q} \equiv$

% A if(b)
$\wp{\A}{x > 196} \land$

% A;B; Q
$\wp{\A\B}{\Q}$

%
$\lor$

% A if(!b)
$\wp{\A}{x \leq 196} \land$

% A;C; Q
$\wp{\A\C}{\Q}$

=: P

%%
%
%%

\NebenRechnung{$\wp{A;}{b}$}

\Mathe{
  \wp{\A}{x > 196}
}

\liWpErklaerung{Wir lassen $y+=7$ weg, weil in der Nachbedingung kein y
vorkommt und setzen in den Term $x > 196$ für das $x$ die erste
Code-Zeile $2 \cdot x + 1 + x \cdot x$ ein.}

\liWpEquivalent{
  \wp{}{2 \cdot x + 1 + x \cdot x > 196}
}

\liWpErklaerung{Nach der Transmformationsregel \textit{Nichts passiert, die
Vorbedingung bleibt gleich} kann das auch so geschrieben werden:}

\liWpEquivalent{
  2 \cdot x + 1 + x \cdot x > 196
}

\liWpErklaerung{Die erste binomische Formel (Plus-Formel) lautet
$(a + b)^2 = a^2 + 2ab + b^2$.
Man kann die Formel auch umgedreht verwenden:
$a^2 + 2ab + b^2 = (a + b)^2$.
Die erste Code-Zeile $2 \cdot x + 1 + x \cdot x$
kann umformuliert werden in
$
1 + 2 \cdot 1 \cdot x + x \cdot x =
1^2 + 2 \cdot 1 \cdot x + x^2 =
(1 + x)^2 =
(x + 1)^2
$.
Wir haben für $a$ die Zahl $1$ und für $b$ den Buchstaben $x$
eingesetzt.}

\liWpEquivalent{
  (x + 1)^2 > 196
}

%%
%
%%

\NebenRechnung{$\wp{A;B;}{Q}$}

\Mathe{
  \wp{\A\B}{\Q}
}

\liWpErklaerung{Für das $x$ in der Nachbedingung setzen wir die erste
Code-Zeile $2 \cdot x + 1 + x \cdot x$ ein.
%
Für das $y$ in der Nachbedingung setzen wir dritte Code-Zeile
\texttt{\AnweisungIII} ein und dann die zweite Code-Zeile
\texttt{\AnweisungII}. Das wp-Kalkül arbeitet den Code rückswärts ab.
%
in $y \% 2$ die dritte Anweisung $y = 2 \cdot y$ einfügen: $2 \cdot y \%
2$
%
dann in $2 \cdot y \% 2$ die zweite Anweisung $y = y + 7$ einfügen: $2
\cdot (y + 7) \% 2$}

\liWpEquivalent{
  (x + 1)^2 \geq 8 \land 2(y + 7)\%2 = 1
}

\liWpErklaerung{Diese Aussage ist falsch, da $2(y + 7)$ immer eine gerade
Zahl ergibt und der Rest von einer Division durch zwei einer geraden
Zahl immer 0 ist und nicht 1.}

\liWpEquivalent{
  (x + 1)^2 \geq 8 \land \text{falsch}
}

\liWpEquivalent{
  \text{falsch}
}

%%
%
%%

\NebenRechnung{$\wp{A;}{\neg b}$}

\Mathe{
  \wp{\A}{x \leq 196}
}

\liWpErklaerung{Analog zu Nebenrechnung 1}

\liWpEquivalent{
  (x + 1)^2 \leq 196
}

%%
%
%%

\NebenRechnung{$\wp{A;C;}{Q}$}

\Mathe{
  \wp{\A\C}{\Q}
}

\liWpErklaerung{\code{x*=2;}: $x \cdot 2$ für $x$ einsetzen:}

\liWpEquivalent{
  \wp{x=2*x+1+x*x;y+=7;y-=8;}
  {(2 \cdot x \geq 8) \land (y \% 2 = 1)}
}

\liWpErklaerung{\code{y-=8;}: $y - 8$ für $y$ einsetzen:}

\liWpEquivalent{
  \wp{x=2*x+1+x*x;y+=7;}
  {(2 \cdot x \geq 8) \land ((y - 8) \% 2 = 1)}
}

\liWpErklaerung{\code{y+=7}: $y + 7$ für $y$ einsetzen:}

\liWpEquivalent{
  \wp{x=2*x+1+x*x;}
  {(2 \cdot x \geq 8) \land (((y + 7) - 8) \% 2 = 1)}
}

\liWpErklaerung{\code{x=2*x+1+x*x;}: $(x + 1)^2$ für $x$ einsetzen:}

\liWpEquivalent{
\wp{}
  {(2 \cdot (x + 1)^2 \geq 8) \land (((y + 7) - 8) \% 2 = 1)}
}

\liWpErklaerung{Nur noch die Nachbedingung stehen lassen:}

\liWpEquivalent{
  (2 \cdot (x + 1)^2 \geq 8) \land ((\textcolor{red}{(y + 7) - 8}) \% 2 = 1)
}

\liWpErklaerung{Subtraktion:}

\liWpEquivalent{
  (2 \cdot (x + 1)^2 \geq 8) \land ((\textcolor{red}{y - 1}) \% 2 = 1)
}

\liWpErklaerung{Vereinfachen (links beide Seiten durch 2 teilen und rechts
von beiden Seiten $1$ abziehen)}

\liWpEquivalent{
  (\frac{2 \cdot (x + 1)^2}{2} \geq \frac{8}{2}) \land (((y - 1) \% 2) - 1 = 1 - 1)
}

\liWpErklaerung{Zwischenergebnis:}

\liWpEquivalent{
((x + 1)^2 \geq 4) \land y \% 2 = 0
}

\vspace{1cm}

\liPseudoUeberschrift{Zusammenführung}

\liWpErklaerung{Die Zwischenergebnisse aus den Nebenrechnungen
zusammenfügen:}

\liWpEquivalent{
  [
    (x + 1)^2 > 196 \land
    \text{falsch}
  ]
  \lor
  [
    (x + 1)^2 \leq 196 \land
    (x + 1)^2 \geq 4 \land y\%2 = 0
  ]
}

\liWpErklaerung{„falsch“ und eine Aussage verbunden mit logischem Und
„$\land$“ ist insgesamt falsch:}

\liWpEquivalent{
  \text{falsch}
  \lor
  [
    (x + 1)^2 \leq 196 \land
    (x + 1)^2 \geq 4 \land y\%2 = 0
  ]
}

\liWpErklaerung{falsch verbunden mit oder weglassen:}

\liWpEquivalent{
  (x + 1)^2 \leq 196
  \land
  (x + 1)^2 \geq 4 \land y\%2 = 0
}

\liWpErklaerung{Umgruppieren, sodass nur noch ein $(x + 1)^2$ geschrieben
werden muss:}

\liWpEquivalent{
  4 \leq (x + 1)^2 \leq 196
  \land
  y\%2 = 0
}

$4 = 2^2$ und $196 = 14^2$

\liWpEquivalent{
  2^2 \leq (x + 1)^2 \leq 14^2
  \land
  y\%2 = 0
}

\liWpErklaerung{Hoch zwei weg lassen: Betragsklammer $|x|$ oder auch
Betragsfunktion hinzufügen (Die Betragsfunktion ist festgelegt als
„Abstand einer Zahl von der Zahl Null“.}

\liWpEquivalent{
  2 \leq | x + 1 | \leq 14 \land y\%2 = 0
}

\liWpErklaerung{Auf die Gleichung der linken Aussage $-1$ anwenden:}

\liWpEquivalent{
  1 \leq |x| \leq 13 \land y\%2 = 0
}

\liWpErklaerung{Die Betragsklammer weg lassen:}

\liWpEquivalent{
  (1 \leq x \leq 13 \lor -13 \leq x \leq -1) \land y\%2 = 0
}

\Mathe{=: P}

\end{liAntwort}
\end{document}
