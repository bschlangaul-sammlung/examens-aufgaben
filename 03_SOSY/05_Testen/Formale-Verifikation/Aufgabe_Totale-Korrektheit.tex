\documentclass{lehramt-informatik-aufgabe}
\liLadePakete{syntax,mathe,wpkalkuel}
\begin{document}

\begin{enumerate}
\item Gegeben sei folgende Methode:\footcite{sosy:e-klausur}

\inputcode[firstline=3,lastline=11]{aufgaben/sosy/totale_korrektheit/GeoSum}

Weisen Sie mittels vollständiger Induktion nach, dass

\begin{displaymath}
\text{geoSum}(n,q) = 1 - q^{n+1}
\end{displaymath}

Dabei können Sie davon ausgehen, dass $q > 0$, $ n \in \mathbb{N}_0$

Manuelle Rückmeldung:
Beim Induktionsanfang fehlt ein f(0) oder Ähnliches (je nachdem, wie

Also ist

du die Funktion nennen möchtest) (-1 BE) Du musst ja zeigen, dass für
n=0 der Programmcode und die Formel auf das gleiche Ergebnis kommen.
Beim Induktionsschritt fehlt der "Start" mit dem Programmcode, der ist
nur in der Behauptung vorhanden. Du musst da näher am Programmcode
bleiben als bei mathematischen Beispielen (vgl. Aufgabe mit Türme von
Hanoi) (- 1 BE) Sonst sehr schön!

\begin{antwort}
\liPseudoUeberschrift{Induktionsanfang}

\begin{displaymath}
f(0): \text{geoSum}(0, q) = 1 - q^{0+1} = 1 - q^1 = 1 - q
\end{displaymath}

\liPseudoUeberschrift{Induktionsvoraussetzung}

\begin{displaymath}
f(n): \text{geoSum}(n, q) = 1 - q^{n+1}
\end{displaymath}

\liPseudoUeberschrift{Induktionsschritt}

\begin{align*}
f(n+1): \text{geoSum}(n + 1, q)
& = (1 - q)^{(n + 1) + 1} + \text{geoSum}(n, q) \\
& = (1 - q)^{n + 1 + 1} + (1 - q)^{n + 1} \\
& = 1-q^{n+1} + q^{n+1} \cdot (1-q) \\
& = 1-q^{n+1}+q^{n+1}-q^{n+2} \\
& = 1-q^{(n+1)+1}
\end{align*}

\end{antwort}

%%
%
%%

\item Gegeben sei die folgende Methode:

\inputcode[firstline=4,lastline=11]{aufgaben/sosy/totale_korrektheit/Blub}

Berechnen Sie hierzu das folgende Kalkül:

\begin{displaymath}
\wp{R}{a \geq 0}
\end{displaymath}

\begin{antwort}
$\wp{R}{a \geq 0}$

\MatheEquiv{
  \wp{if (a > 15) \{a = a - 42;\} else \{ a = -a;\}}{a \geq 0}
}

\MatheEquiv{
  (a > 15 \land \wp{a = a - 42}{a \geq 0})
  \lor
  (a \leq 15 \land \wp{a = -a}{a \geq 0})
}

\MatheEquiv{
  (a > 15 \land \wp{}{a - 42 \geq 0})
  \lor
  (a \leq 15 \land \wp{}{-a \geq 0})
}

\MatheEquiv{
  (a > 15 \land a - 42 \geq 0)
  \lor
  (a \leq 15 \land -a \geq 0)
}

\MatheEquiv{
  (a > 15 \land a \geq 42)
  \lor
  (a \leq 15 \land a \leq 0)
}

\MatheEquiv{
  (a \geq 42)
  \lor
  (a \leq 0)
}

\MatheEquiv{
  a \geq 42
  \lor
  a \leq 0
}

\end{antwort}
\end{enumerate}
\end{document}
