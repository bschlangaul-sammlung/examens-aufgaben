\documentclass{lehramt-informatik-haupt}
\liLadePakete{syntax,uml}

\begin{document}

%%%%%%%%%%%%%%%%%%%%%%%%%%%%%%%%%%%%%%%%%%%%%%%%%%%%%%%%%%%%%%%%%%%%%%%%
% Theorie-Teil
%%%%%%%%%%%%%%%%%%%%%%%%%%%%%%%%%%%%%%%%%%%%%%%%%%%%%%%%%%%%%%%%%%%%%%%%

\chapter{Zustand, Objekte für Zustände / (State, Objects for states)}

\begin{quellen}
\item \cite{wiki:zustand}
\item \url{https://www.philipphauer.de/study/se/design-pattern/state.php}
\item \cite[PDF Seite 258-265]{gof}
% \item \cite{schatten}
% \item \cite{eilebrecht}
\item \cite[Seite 69-81]{siebler}
\end{quellen}

\section{Zweck}

Das Zustandsmuster wird zur Kapselung unterschiedlicher,
zustandsabhängiger Verhaltensweisen eines Objektes eingesetzt.
\footcite{wiki:zustand}

%-----------------------------------------------------------------------
%
%-----------------------------------------------------------------------

\section{Klassendiagramm}

\begin{tikzpicture}
\umlclass[x=-1,y=3]{Context}{}{+request()}
\umlclass[x=3,y=3,type=interface]{State}{}{+handle()}
\umlclass[x=1,y=0]{ConcreteStateA}{}{+handle()}
\umlclass[x=5,y=0]{ConcreteStateB}{}{+handle()}

\umlVHVreal{ConcreteStateA}{State}
\umlVHVreal{ConcreteStateB}{State}

\umlaggreg{Context}{State}

\umlnote[x=-2,y=0,width=2cm]{Context}{state.handle()}
\end{tikzpicture}

Quelle: Englische Wikipedia, so ähnlich wie in GoF

%-----------------------------------------------------------------------
%
%-----------------------------------------------------------------------

\section{Allgemeines Code-Beispiel}

\def\TmpCode#1{\liJavaDatei[firstline=3]{entwurfsmuster/zustand/allgemein/#1}}

\TmpCode{Kontext}
\TmpCode{Zustand}

\literatur

\end{document}
