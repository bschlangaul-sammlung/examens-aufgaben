\documentclass{lehramt-informatik-haupt}

\begin{document}

%%%%%%%%%%%%%%%%%%%%%%%%%%%%%%%%%%%%%%%%%%%%%%%%%%%%%%%%%%%%%%%%%%%%%%%%
% Theorie-Teil
%%%%%%%%%%%%%%%%%%%%%%%%%%%%%%%%%%%%%%%%%%%%%%%%%%%%%%%%%%%%%%%%%%%%%%%%

\chapter{Modell-Präsentation-Steuerung (Model view controller (MVC))}

\begin{quellen}
\item \cite{wiki:mvc}
% \item \url{https://www.philipphauer.de/study/se/design-pattern/}
\item \cite[Kapitel „Observer“, Seite 256]{gof}
%\item \cite{schatten}
\item \cite[Kapitel 5.6, Seite 92-95]{eilebrecht}
\item \cite[Kapitel 3.8, Seite 48-49]{siebler}
\end{quellen}

%-----------------------------------------------------------------------
%
%-----------------------------------------------------------------------

\section{Zweck}

Die Verantwortlichkeiten beim Aufbau von Benutzerschnittstellen werden
auf \section{drei verschiedene Rollen verteilt}, um die
\memph{unterschiedliche Präsentation} derselben Information \memph{zu
erleichtern}.
\footcite[Seite 92]{eilebrecht}

%-----------------------------------------------------------------------
%
%-----------------------------------------------------------------------

\section{Szenario}

Sie entwickeln eine Software zur Medienverwaltung (CDs, DVDs etc.). Die
Benutzerschnittstelle (User Interface, UI) soll möglichst flexibel
gehalten werden. Neben konventionellen grafischen Benutzeroberflächen
kommen beispielsweise auch HTML-Seiten oder eine iOS-App infrage. Sie
müssen die grafische Darstellung unabhängig von anderen Systemteilen
austauschen können. Code-Redundanz wollen Sie dabei weitgehend
vermeiden.
\footcite[Seite 92]{eilebrecht}

\literatur

\end{document}
