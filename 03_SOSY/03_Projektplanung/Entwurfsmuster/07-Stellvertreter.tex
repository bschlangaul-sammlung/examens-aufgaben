\documentclass{lehramt-informatik-haupt}
\liLadePakete{syntax,uml}

\begin{document}

%%%%%%%%%%%%%%%%%%%%%%%%%%%%%%%%%%%%%%%%%%%%%%%%%%%%%%%%%%%%%%%%%%%%%%%%
% Theorie-Teil
%%%%%%%%%%%%%%%%%%%%%%%%%%%%%%%%%%%%%%%%%%%%%%%%%%%%%%%%%%%%%%%%%%%%%%%%

\chapter{Stellvertreter (Proxy)}

\begin{liQuellen}
\item \cite{wiki:stellvertreter}
% \item \url{https://www.philipphauer.de/study/se/design-pattern/}
\item \cite[Seite 176-185]{gof}
\item \cite[Kapitel 8.4.4, Seite 256-262]{schatten}
\item \cite[Kapitel 5.5, Seite 89-92]{eilebrecht}
\item \cite[Kapitel 21, Seite 241-268]{siebler}
\end{liQuellen}

%-----------------------------------------------------------------------
%
%-----------------------------------------------------------------------

\section{Zweck}

Ein Proxy stellt einen \memph{Platzhalter} für eine andere Komponente
(Objekt) dar und \memph{kontrolliert den Zugang zum echten Objekt}.
\footcite[Seite 89]{eilebrecht}

%-----------------------------------------------------------------------
%
%-----------------------------------------------------------------------

\section{Klassendiagramm}

\begin{tikzpicture}
\umlsimpleclass[x=-1,y=2]{Klient}

\umlclass[x=2,y=2]{Subjekt}{}{+ agiere()}
\umlclass[x=0,y=0]{KonkretesSubjekt}{}{+ agiere()}
\umlclass[x=4,y=0]{Stellvertreter}{}{+ agiere()}

\umlinherit{KonkretesSubjekt}{Subjekt}
\umlinherit{Stellvertreter}{Subjekt}
\umluniassoc{Stellvertreter}{KonkretesSubjekt}
\umluniassoc{Klient}{Subjekt}

\end{tikzpicture}
%-----------------------------------------------------------------------
%
%-----------------------------------------------------------------------

\section{Allgemeines Code-Beispiel}

\def\TmpCode#1{\liJavaDatei[firstline=3]{entwurfsmuster/stellvertreter/allgemein/#1}}

\TmpCode{Subjekt}
\TmpCode{KonkretesSubjekt}
\TmpCode{Stellvertreter}
\TmpCode{Klient}

\literatur

\end{document}
