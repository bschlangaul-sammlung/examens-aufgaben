\documentclass{lehramt-informatik-haupt}
\liLadePakete{syntax,uml}

\begin{document}

%%%%%%%%%%%%%%%%%%%%%%%%%%%%%%%%%%%%%%%%%%%%%%%%%%%%%%%%%%%%%%%%%%%%%%%%
% Theorie-Teil
%%%%%%%%%%%%%%%%%%%%%%%%%%%%%%%%%%%%%%%%%%%%%%%%%%%%%%%%%%%%%%%%%%%%%%%%

\chapter{Kompositum}

\begin{quellen}
\item \cite{wiki:kompositum}
%\item \url{https://www.philipphauer.de/study/se/design-pattern/}
\item \cite[Seite 139-147]{gof}
%\item \cite{schatten}
\item \cite[Seite 102-104]{eilebrecht}
\item \cite[Kapitel 14, Seite 173]{siebler}
\end{quellen}

\section{Das Kompositum}

Das Kompositum (englisch \emph{composite} oder \emph{whole-part}) ist
ein Entwurfsmuster aus dem Bereich der Softwareentwicklung. Das
Kompositionsmuster (\emph{composite pattern}) wird angewendet, um
\emph{Teil-Ganzes-Hierarchien} zu repräsentieren, indem Objekte zu
\emph{Baumstrukturen} zusammengefügt werden. Die Grundidee des
Kompositionsmusters ist, in einer \emph{abstrakten Klasse} sowohl
\emph{primitive Objekte} als auch ihre \emph{Behälter} zu
repräsentieren. Somit können sowohl einzelne Objekte als auch ihre
Kompositionen \emph{einheitlich behandelt} werden.
\footcite{aud:fs:4}

Ein anderes Beispiel sind die \emph{Klassendefinitionen} der
\emph{grafischen Benutzeroberfläche von Java}. Alle Elemente wie
Schaltflächen und Textfelder sind \emph{Spezialisierungen der Klasse
Component}. Die Behälter für diese Elemente sind aber ebenfalls
Spezialisierungen derselben Klasse. Mit anderen Worten: Alle
Standardelemente werden wesentlich durch eine einzige
(Kompositum-)Klasse definiert.
\footcite{wiki:kompositum}

\begin{center}
\begin{tikzpicture}
\umlclass[x=2.5,y=3,type=abstract]{Komponente}{}{+ agiere()}
\umlclass[x=0]{Blatt}{}{+ agiere()}
\umlclass[x=5]{Kompositum}{}{
  + agiere()\\
  + fügeKindHinzu()\\
  + entferneKind()\\
  + gibKind()
}

\umlVHVinherit{Kompositum}{Komponente}
\umlVHVinherit{Blatt}{Komponente}
\umlHVHaggreg[anchor1=east,arm1=1.5cm,arg1=eltern,mult1=1,arg2=kind,mult2=0..*,pos2=2.9,pos1=0.4]{Kompositum}{Komponente}
\end{tikzpicture}
\end{center}

Details zum Nachlesen: Klett: Informatik 4, S.27 - 29

\section{Zweck}

Das Composite-Muster ermöglicht die Gleichbehandlung von Einzelelementen
und Elementgruppierungen in einer verschachtelten Struktur (z. B. Baum),
sodass aus Sicht des Clients keine explizite Unterscheidung notwendig
ist.
\footcite[Seite 102]{eilebrecht}

%-----------------------------------------------------------------------
%
%-----------------------------------------------------------------------

Das Kompositum (englisch \emph{composite} oder \emph{whole-part}) ist
ein Entwurfsmuster aus dem Bereich der Softwareentwicklung. Das
Kompositionsmuster (\emph{composite pattern}) wird angewendet, um
\emph{Teil-Ganzes-Hierarchien} zu repräsentieren, indem Objekte zu
\emph{Baumstrukturen} zusammengefügt werden. Die Grundidee des
Kompositionsmusters ist, in einer \emph{abstrakten Klasse} sowohl
\emph{primitive Objekte} als auch ihre \emph{Behälter} zu
repräsentieren. Somit können sowohl einzelne Objekte als auch ihre
Kompositionen \emph{einheitlich behandelt} werden.
\footcite{aud:fs:4}

Ein anderes Beispiel sind die \emph{Klassendefinitionen} der
\emph{grafischen Benutzeroberfläche von Java}. Alle Elemente wie
Schaltflächen und Textfelder sind \emph{Spezialisierungen der Klasse
Component}. Die Behälter für diese Elemente sind aber ebenfalls
Spezialisierungen derselben Klasse. Mit anderen Worten: Alle
Standardelemente werden wesentlich durch eine einzige
(Kompositum-)Klasse definiert.
\footcite{wiki:kompositum}

\begin{center}
\begin{tikzpicture}
\umlclass[x=2.5,y=3,type=abstract]{Komponente}{}{+ agiere()}
\umlclass[x=0]{Blatt}{}{+ agiere()}
\umlclass[x=5]{Kompositum}{}{
  + agiere()\\
  + fügeKindHinzu()\\
  + entferneKind()\\
  + gibKind()
}

\umlVHVinherit{Kompositum}{Komponente}
\umlVHVinherit{Blatt}{Komponente}
\umlHVHaggreg[anchor1=east,arm1=1.5cm,arg1=eltern,mult1=1,arg2=kind,mult2=0..*,pos2=2.9,pos1=0.4]{Kompositum}{Komponente}

\end{tikzpicture}
\end{center}

Details zum Nachlesen: Klett: Informatik 4, S.27 - 29

\literatur

\end{document}
