\documentclass{lehramt-informatik-haupt}
\liLadePakete{syntax,uml}

\begin{document}

%%%%%%%%%%%%%%%%%%%%%%%%%%%%%%%%%%%%%%%%%%%%%%%%%%%%%%%%%%%%%%%%%%%%%%%%
% Theorie-Teil
%%%%%%%%%%%%%%%%%%%%%%%%%%%%%%%%%%%%%%%%%%%%%%%%%%%%%%%%%%%%%%%%%%%%%%%%

\chapter{Beobachter / Observer}

\begin{quellen}
\item \cite{wiki:beobachter}
\item \url{https://www.philipphauer.de/study/se/design-pattern/observer.php}
\item \cite[Seite 249-257]{gof}
\item \cite[Kapitel 8.5.1, Seite 269-274]{schatten}
\item \cite[Kapitel 4.7, Seite 70-75]{eilebrecht}
\item \cite[Seite 269]{siebler}
\end{quellen}

%-----------------------------------------------------------------------
%
%-----------------------------------------------------------------------

\section{Zweck}

Das Observer-Muster ermöglicht einem oder mehreren Objekten, automatisch
auf die \memph{Zustandsänderung} eines bestimmten Objekts \memph{zu
reagieren}, um den eigenen Zustand anzupassen.
\footcite[Seite 70]{eilebrecht}

%%
%
%%

\section{Szenario}

Zusätzlich zur historischen Darstellung von Verkaufszahlen soll Ihre
Software um eine Prognose-Ansicht erweitert werden. Diese soll zum einen
jede Viertelstunde die aktuellen Zahlen einbeziehen und zum anderen eine
durch den Benutzer gewählte Vorhersagestrategie. Im Gegensatz zur
Anzeige des historischen Verlaufs kann sich folglich die Prognose
kurzfristig ändern. Genauer gesagt, sobald neue Verkaufszahlen vorliegen
oder der Anwender die Strategie ändert. Sie möchten erreichen, dass sich
die Anzeige der Prognose automatisch anpasst, sobald sich im Hintergrund
Änderungen ergeben.
\footcite[Seite 70]{eilebrecht}

%%
%
%%

\section{UML-Diagramm}

\begin{tikzpicture}
\umlclass[x=0,y=0,type=abstract]{Gegenstand}{}{
  + registriere(beobachter: Beobachter): void\\
  + entferne(beobachter: Beobachter): void\\
  + benachrichtige(beobachter: Beobachter): void\\
}
\umlclass[x=0,y=-3]{KonkreterGegenstand}{}{
  + setzteZustand(zustand: Zustand): void
}
\umlinherit{KonkreterGegenstand}{Gegenstand}

\umlclass[x=8,y=0,type=interface]{Beobachter}{}{
  + aktualisiere(zustand: Zustand): void
}
\umlclass[x=8,y=-3]{KonkreterBeobachter}{}{
  + aktualisiere(zustand: Zustand): void
}
\umlreal{KonkreterBeobachter}{Beobachter}

\umlHVHaggreg[arg1=beobachter,pos1=0.5,mult2=*,pos2=2.5]{KonkreterGegenstand}{Beobachter}

\end{tikzpicture}

%%
%
%%

\section{Akteure}

\begin{description}
\item[Gegenstand / Subjekt (Subject / Observable)]

Ein Subjekt (beobachtbares Objekt, auf Englisch publisher, also
„Veröffentlicher“, genannt) hat eine Liste von Beobachtern, ohne deren
konkrete Typen zu kennen. Es bietet eine Schnittstelle zur An- und
Abmeldung von Beobachtern und eine Schnittstelle zur Benachrichtigung
von Beobachtern über Änderungen an.\footcite[Seite 251]{gof}

\item[Beobachter (Observer)]
Die Beobachter (auf Englisch auch subscriber, also „Abonnent“, genannt)
definieren eine Aktualisierungsschnittstelle.

\item[konkreter/s Gegenstand / Subjekt (ConcreteSubject / ConcreteObservable)]

Ein konkretes Subjekt (konkretes, beobachtbares Objekt) speichert den
relevanten Zustand und benachrichtigt alle Beobachter bei
Zustandsänderungen über deren Aktualisierungsschnittstelle. Es verfügt
über eine Schnittstelle zur Erfragung des aktuellen Zustands.

\item[Konkrete Beobachter (ConcreteObserver)]

Konkrete Beobachter verwalten die Referenz auf ein konkretes Subjekt,
dessen Zustand sie beobachten und speichern und dessen Zustand
konsistent ist. Sie implementieren eine Aktualisierungsschnittstelle
unter Verwendung der Abfrageschnittstelle des konkreten Subjekts.
\footcite{wiki:beobachter}
\end{description}

%-----------------------------------------------------------------------
%
%-----------------------------------------------------------------------

\section{Allgemeines Code-Beispiel}

\def\TmpCode#1{\inputcode[firstline=3]{entwurfsmuster/beobachter/allgemein/#1}}

\TmpCode{Gegenstand}
\TmpCode{KonkreterGegenstand}
\TmpCode{Beobachter}
\TmpCode{KonkreterBeobachterA}
\TmpCode{KonkreterBeobachterB}
\TmpCode{Klient}

%%%%%%%%%%%%%%%%%%%%%%%%%%%%%%%%%%%%%%%%%%%%%%%%%%%%%%%%%%%%%%%%%%%%%%%%
% Aufgaben
%%%%%%%%%%%%%%%%%%%%%%%%%%%%%%%%%%%%%%%%%%%%%%%%%%%%%%%%%%%%%%%%%%%%%%%%

\chapter{Aufgaben}

\ExamensAufgabeTTA 66116 / 2018 / 03 : Thema 2 Teilaufgabe 2 Aufgabe 2

\literatur

\end{document}
