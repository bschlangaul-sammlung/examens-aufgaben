\documentclass{lehramt-informatik-haupt}
\liLadePakete{syntax,uml}

\begin{document}

%%%%%%%%%%%%%%%%%%%%%%%%%%%%%%%%%%%%%%%%%%%%%%%%%%%%%%%%%%%%%%%%%%%%%%%%
% Theorie-Teil
%%%%%%%%%%%%%%%%%%%%%%%%%%%%%%%%%%%%%%%%%%%%%%%%%%%%%%%%%%%%%%%%%%%%%%%%

\chapter{Schablone Schablonenmethode / Template, Template method}

\begin{quellen}
\item \cite{wiki:schablone}
% \item \url{https://www.philipphauer.de/study/se/design-pattern/}
\item \cite[Seite 274-278]{gof}
% \item \cite{schatten}
\item \cite[Seite 68 - 70]{eilebrecht}
% \item \cite{siebler}
\end{quellen}

%-----------------------------------------------------------------------
%
%-----------------------------------------------------------------------

\section{Zweck}

Es wird die Struktur eines Algorithmus definiert, wobei \memph{einzelne,
konkrete Schritte in Unterklassen verlagert} werden. Das Muster erlaubt
es, bestimmte Operationen eines Algorithmus zu überschreiben, ohne
dessen Struktur zu ändern.
\footcite[Seite 68]{eilebrecht}

%-----------------------------------------------------------------------
%
%-----------------------------------------------------------------------

\section{Klassendiagramm}

\begin{tikzpicture}
\umlclass[x=0,y=3]{AbstrakteKlasse}{}{
  + schablonenMethode()\\
  \umlvirt{+ primitiveMethode1()}\\
  \umlvirt{+ primitiveMethode2()}
}
\umlclass[x=0,y=0]{KonkreteKlasse}{}{
  + primitiveMethode1()\\
  + primitiveMethode2()
}

\umlinherit{KonkreteKlasse}{AbstrakteKlasse}

\umlnote[x=5,y=3,width=3cm]{AbstrakteKlasse}{
  + primitiveMethode1()\\
  + primitiveMethode2()\\
}
\end{tikzpicture}

%-----------------------------------------------------------------------
%
%-----------------------------------------------------------------------

\section{Allgemeines Code-Beispiel}

\def\TmpCode#1{\liJavaDatei[firstline=3]{entwurfsmuster/schablone/allgemein/#1}}

\TmpCode{AbstrakteKlasse}
\TmpCode{KonkreteKlasse}

\literatur

\end{document}
