\documentclass{lehramt-informatik-haupt}
\liLadePakete{syntax}

\begin{document}

%%%%%%%%%%%%%%%%%%%%%%%%%%%%%%%%%%%%%%%%%%%%%%%%%%%%%%%%%%%%%%%%%%%%%%%%
% Theorie-Teil
%%%%%%%%%%%%%%%%%%%%%%%%%%%%%%%%%%%%%%%%%%%%%%%%%%%%%%%%%%%%%%%%%%%%%%%%

\chapter{Abstrakte Fabrik (Abstract factory)}

\begin{quellen}
\item \cite{wiki:abstrakte-fabrik}
\item \url{https://www.philipphauer.de/study/se/design-pattern/abstract-factory.php}
% \item \cite{gof}
\item \cite[Kapitel 8.3.2, Seite 250-252]{schatten}
\item \cite[Kapitel 1.3, Seite 25-29]{eilebrecht}
\item \cite[Kapitel 11, Seite 127-145]{siebler}
\end{quellen}

\section{Zweck}

Es wird eine Schnittstelle bereitgestellt, um \memph{Familien
verbundener oder abhängiger Objekte} zu erstellen, ohne die konkreten
Klassen zu spezifizieren.
\footcite[Seite 25]{eilebrecht}

%-----------------------------------------------------------------------
%
%-----------------------------------------------------------------------

\section{Allgemeines Code-Beispiel}

\def\TmpCode#1{\liJavaDatei[firstline=3]{entwurfsmuster/abstrakte_fabrik/allgemein/#1}}

\TmpCode{Produkte}
\TmpCode{AbstrakteFabrik}
\TmpCode{Klient}

\literatur

\end{document}
