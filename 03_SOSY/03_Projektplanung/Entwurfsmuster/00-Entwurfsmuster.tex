\documentclass{lehramt-informatik-haupt}
\usepackage{tabularx}

\begin{document}

%%%%%%%%%%%%%%%%%%%%%%%%%%%%%%%%%%%%%%%%%%%%%%%%%%%%%%%%%%%%%%%%%%%%%%%%
% Theorie-Teil
%%%%%%%%%%%%%%%%%%%%%%%%%%%%%%%%%%%%%%%%%%%%%%%%%%%%%%%%%%%%%%%%%%%%%%%%

\chapter{Entwurfsmuster / Design pattern}

\begin{liQuellen}
\item \cite{wiki:entwurfsmuster}
\item \cite[Seite 39-43]{sosy:fs:3}
\end{liQuellen}

\section{Verwendete Überschriften}

\begin{enumerate}
\item Zweck
\item Szenario
\item UML-Diagramm
\item Akteure
\item Allgemeines Code-Beispiel
\end{enumerate}

Wiederkehrende, geprüfte, bewährte Lösungsschablonen für typische Probleme\footcite[Seite 39]{sosy:fs:3}

Wiederverwendung

Kein Garant für gutes Design – können dieses aber gut unterstützen

\begin{itemize}
\item Erzeugungsmuster (Creational Patterns)

\begin{itemize}
\item Einzelstück (Singleton)
\item Abstrakte Fabrik (Abstract Factory)
\end{itemize}

\item Strukturmuster (Structural Patterns)

\begin{itemize}
\item Adapter
\item Dekorierer (Decorator)
\item Kompositum (Composite)
\item Stellvertreter (Proxy)
\end{itemize}

\item Verhaltensmuster (Behavioral Patterns)

\begin{itemize}
\item Beobachter (Observer)
\item Wiederholer (Iterator)
\item Schablone (Template)
\item Zustand (State)
\end{itemize}

\item Sonstige

\begin{itemize}
\item Modell-Präsentation-Steuerung (Model-View-Controller)
\end{itemize}
\end{itemize}

\footcite[Seite 39]{sosy:fs:3}

\literatur

\end{document}
