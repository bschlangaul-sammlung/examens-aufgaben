\documentclass{lehramt-informatik-aufgabe}
\liLadePakete{petri,mathe}
\begin{document}
\liAufgabenTitel{Rechnen}

\section{Aufgabe 3: Rechnen
\index{Petri-Netz}
footcite[Seite 2]{sosy:ab:4}}

Gegeben sei die Darstellungsmatrix $A$ und der Belegungsvektor $v$ eines
Petri-Netzes:

\begin{displaymath}
A =
\begin{blockarray}{ccccc}
  & t_1 & t_2 & t_3 & t_4 \\
  \begin{block}{c(cccc)}
  p_1 & -1 & 0 & 0 & 1 \\
  p_2 & 1 & -1 & 1 & 0 \\
  p_3 & 0 & 1 & 0 & -1 \\
  p_4 & 0 & -1 & 1 & 0 \\
  p_5 & 0 & 0 & -1 & 1 \\
  \end{block}
\end{blockarray}
\enspace
,
\enspace
v =
\left(
  \begin{array}{c}
  0 \\
  0 \\
  1 \\
  0 \\
  1
  \end{array}
\right)
\end{displaymath}
\begin{enumerate}

%%
% (a)
%%

\item Skizzieren Sie das zugehörige Petri-Netz.

\begin{antwort}
\begin{center}
\begin{tikzpicture}
\node[place,tokens=0,label=south:$p_1$] at (2,4) (p1) {};
\node[place,tokens=0,label=east:$p_2$] at (6,4) (p2) {};
\node[place,tokens=1,label=$p_3$] at (10,2) (p3) {};
\node[place,tokens=0,label=$p_4$] at (6,0) (p4) {};
\node[place,tokens=1,label=$p_5$] at (2,0) (p5) {};

\node[transition,label=$t_1$] at (4,4) {}
  edge[pre] (p1)
  edge[post] (p2);
\node[transition,label=$t_2$] at (8,2) {}
  edge[pre] (p2)
  edge[pre] (p4)
  edge[post] (p3);
\node[transition,label=$t_3$] at (4,0) {}
  edge[pre] (p5)
  edge[post] (p2)
  edge[post] (p4);
\node[transition,label=east:$t_4$] at (0,2) (t4) {}
  edge[post] (p1)
  edge[post] (p5);

\draw[pre,bend left=75] (t4) -- (0,-1) -- (10,-1) -- (p3);
\end{tikzpicture}
\end{center}
\end{antwort}

%%
% (b)
%%

\item Berechnen Sie mithilfe der Darstellungsmatrix $A$ und zum
Belegungsvektor $v$, die Belegung nach Schaltung von $t_3 \rightarrow t_2
\rightarrow t_4$.

\begin{antwort}
\begin{displaymath}
v_\text{neu} =
v +
A \cdot \left(\begin{array}{c}0\\0\\1\\0\end{array}\right) +
A \cdot \left(\begin{array}{c}0\\1\\0\\0\end{array}\right) +
A \cdot \left(\begin{array}{c}0\\0\\0\\1\end{array}\right) =
\left(\begin{array}{c}1\\0\\1\\0\\1\end{array}\right)
\end{displaymath}
\end{antwort}

\end{enumerate}

\section{Aufgabe 4: Erreichbarkeitsgraph\footcite[Seite 3]{sosy:ab:4}}

Gegeben ist das folgende Petri-Netz:

\begin{center}
\begin{tikzpicture}
\node[place,tokens=3,label=$p_1$] at (0,2) (p1) {};
\node[place,label=$p_2$] at (4,4) (p2) {};
\node[place,label=east:$p_3$] at (4,0) (p3) {};

\node[transition,label=$t_1$] at (2,4) {}
  edge[pre] (p1)
  edge[post] node[auto]{2} (p2);
\node[transition,label=$t_2$] at (2,2) {}
  edge[pre] node[auto]{2} (p1)
  edge[post] (p2)
  edge[post] node[auto]{3} (p3);
\node[transition,label=$t_3$] at (2,0) {}
  edge[pre] node[auto]{4} (p3)
  edge[post] node[auto]{3} (p1);
\node[transition,label=east:$t_4$] at (4,2) {}
  edge[pre] node[auto]{3} (p2)
  edge[post] (p3);
\end{tikzpicture}
\end{center}

\begin{enumerate}

%%
% (a)
%%

\item Geben Sie die dazugehörige Darstellungsmatrix sowie den
Belegungsvektor an.

\begin{antwort}
\begin{displaymath}
A =
\begin{blockarray}{ccccc}
      & t_1 & t_2 & t_3 & t_4 \\
  \begin{block}{c(cccc)}
  p_1 & -1  & -2  & 3   & 0 \\
  p_2 & 2   & 1   & 0   & -3 \\
  p_3 & 0   & 3   & -4  & 1 \\
  \end{block}
\end{blockarray}
\enspace
,
\enspace
v =
\left(
  \begin{array}{c}
  3 \\
  0 \\
  0
  \end{array}
\right)
\end{displaymath}
\end{antwort}

%%
% (b)
%%

\item Skizzieren Sie den Erreichbarkeitsgraphen des Petri-Netzes.

\begin{antwort}

\end{antwort}

%%
% (c)
%%

\item Begründen Sie anhand des Erreichbarkeitsgraphen, ob das Petri-Netz
verklemmungsfrei ist oder nicht.

\begin{antwort}

\end{antwort}
\end{enumerate}
\end{document}
