\documentclass{lehramt-informatik-aufgabe}
\liLadePakete{petri,mathe}
\begin{document}
\liAufgabenTitel{Erreichbarkeitsgraph}

\section{Aufgabe 4: \footcite[Seite 3, Aufgabe 5]{sosy:ab:4}
\index{Petri-Netz}}

Gegeben ist das folgende Petri-Netz:

\begin{center}
\begin{tikzpicture}[li petri]
\node[place,tokens=3,label=$p_1$] at (0,2) (p1) {};
\node[place,label=$p_2$] at (4,4) (p2) {};
\node[place,label=east:$p_3$] at (4,0) (p3) {};

\node[transition,label=$t_1$] at (2,4) {}
  edge[pre] (p1)
  edge[post] node[auto]{2} (p2);
\node[transition,label=$t_2$] at (2,2) {}
  edge[pre] node[auto]{2} (p1)
  edge[post] (p2)
  edge[post] node[auto]{3} (p3);
\node[transition,label=$t_3$] at (2,0) {}
  edge[pre] node[auto]{4} (p3)
  edge[post] node[auto]{3} (p1);
\node[transition,label=east:$t_4$] at (4,2) {}
  edge[pre] node[auto]{3} (p2)
  edge[post] (p3);
\end{tikzpicture}
\end{center}

\begin{enumerate}

%%
% (a)
%%

\item Geben Sie die dazugehörige Darstellungsmatrix sowie den
Belegungsvektor an.

\begin{liAntwort}
\begin{displaymath}
A =
\begin{blockarray}{ccccc}
      & t_1 & t_2 & t_3 & t_4 \\
  \begin{block}{c(cccc)}
  p_1 & -1  & -2  & 3   & 0 \\
  p_2 & 2   & 1   & 0   & -3 \\
  p_3 & 0   & 3   & -4  & 1 \\
  \end{block}
\end{blockarray}
\enspace
,
\enspace
v =
\left(
  \begin{array}{c}
  3 \\
  0 \\
  0
  \end{array}
\right)
\end{displaymath}
\end{liAntwort}

%%
% (b)
%%

\item Skizzieren Sie den Erreichbarkeitsgraphen des Petri-Netzes.
\index{Erreichbarkeitsgraph}

\begin{liAntwort}

\end{liAntwort}

%%
% (c)
%%

\item Begründen Sie anhand des Erreichbarkeitsgraphen, ob das Petri-Netz
verklemmungsfrei ist oder nicht.

\begin{liAntwort}
Durch Schalten von $t_1 \rightarrow t_1 \rightarrow t_1 \rightarrow t_4
\rightarrow t_4$ wird beispielsweise eine Verklemmung erreicht. Das
Petri-Netz ist also nicht verklemmungsfrei. Am Erreichbarkeitsgraphen
erkennt man das anhand der Senke im Knoten [0,0,2].
\end{liAntwort}
\end{enumerate}
\end{document}
