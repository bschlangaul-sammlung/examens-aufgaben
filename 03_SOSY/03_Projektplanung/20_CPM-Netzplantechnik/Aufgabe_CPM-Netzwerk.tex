\documentclass{lehramt-informatik-aufgabe}
\liLadePakete{cpm,mathe,checkbox}
\begin{document}
\liAufgabenTitel{CPM-Netzwerk}

\section{CPM-Netzwerk
\index{CPM-Netzplantechnik}
\footcite{sosy:e-klausur}
}

\begin{center}
\begin{tikzpicture}
\ereignis{1}(0,0)
\ereignis{2}(1,1)
\ereignis{3}(1,-1)
\ereignis{4}(2,0)
\ereignis{5}(3,1)
\ereignis{6}(4,0)
\ereignis{7}(5,1)
\ereignis{8}(5,-1)
\ereignis{9}(6,0)

\vorgang(1>2){3}
\vorgang(1>3){4}
\vorgang(2>5){1}
\vorgang(3>4){3}
\vorgang(3>8){5}
\vorgang(5>6){2}
\vorgang(6>7){1}
\vorgang(6>8){3}
\vorgang(7>9){1}
\vorgang(8>9){2}

\scheinvorgang(2>4){}
\scheinvorgang(4>6){}
\scheinvorgang(6>9){}
\scheinvorgang(5>7){}
\end{tikzpicture}
\end{center}

\begin{enumerate}

%%
%
%%

\item Welche Scheinvorgänge könnten aus dem Netzwerk entfernt werden,
ohne dass Informationen verloren gehen?

\begin{itemize}
\liRichtig 2 $\rightarrow$ 4
\liFalsch 4 $\rightarrow$ 6
\liRichtig 5 $\rightarrow$ 7
\liRichtig 6 $\rightarrow$ 9
\end{itemize}

%%
%
%%

\item Berechnen Sie für jedes Ereignis den frühesten Termin, wobei
angenommen wird, dass das Projekt zum Zeitpunkt 0 startet.

\begin{antwort}
\begin{center}
\begin{tabular}{|l|r|r|}
\hline
$i$ & Nebenrechnung & $\text{FZ}_i$\\
\hline\hline
1             &                                                & 0 \\\hline
2             & \f$0 + \t{3}(1-2) = 3$                         & 3 \\\hline
3             & \f$0 + \t{4}(1-3) = 4$                 & 4 \\\hline
4             & \f$\t{3}(1-2) + \t{0}(2-4) = 3$                & \\
              & \f$\t{4}(1-3) + \t{3}(3-4) = 7$                & \\
              & \f$\max(3,7)$                                  & 7 \\\hline
5             & \f$\t{3}(1-2) + \t{1}(2-5) = 4$                 & 4 \\\hline
6             & \f$\max(7 + 0, 4 + 2)$  & 7 \\\hline
7             & \f$\max(4 + 0, 7 + 1)$  & 8 \\\hline
8             & \f$\max(4 + 5, 7 + 3)$  & 10 \\\hline
9             & \f$\max(8 + 1, 7 + 0, 10 + 2)$  & 12 \\\hline
\end{tabular}
\end{center}
\end{antwort}

%%
%
%%

\item Berechnen Sie für jedes Ereignis auch die spätesten Zeiten, indem
Sie für das letzte Ereignis den frühesten Termin als spätesten Termin
ansetzen.

\begin{antwort}
\begin{center}
\begin{tabular}{|l|r|r|}
\hline
$i$ & Nebenrechnung & $\text{SZ}_i$\\
\hline\hline
1             & \f$\min(4 - 3, 4 - 4)$  & 0 \\
2             & \f$\min(5 - 1, 7 - 0)$ & 4 \\
3             & \f$\min(10 - 5, 7 - 3)$ & 4 \\
4             & \f$7 - 0$ & 7 \\
5             & \f$\min(11 - 0, 7 - 2)$ & 5 \\
6             & \f$\min(12 - 0, 11 - 1, 10 - 3)$ & 7 \\
7             & \f$12 - 1$ & 11 \\
8             & \f$12 - 2$ & 10 \\
9             & \f{}siehe $\text{FZ}_9$  & 12 \\\hline
\end{tabular}
\end{center}
\end{antwort}

%%
%
%%

\item Geben Sie nun die Pufferzeiten der Ereignisse an.

\begin{antwort}
\begin{center}
\begin{tabular}{|l|l|l|l|l|l|l|l|l|l|}
\hline
Ereignis         & 1 & 2 & 3 & 4 & 5 & 6 & 7  & 8  & 9 \\\hline\hline
frühester Termin & 0 & 3 & 4 & 7 & 4 & 7 & 8  & 10 & 12 \\\hline
spätester Termin & 0 & 4 & 4 & 7 & 5 & 7 & 11 & 10 & 12 \\\hline
Puffer           & 0 & 1 & 0 & 0 & 1 & 0 & 3  & 0  & 0 \\\hline
\end{tabular}
\end{center}
\end{antwort}

\item Wie verläuft der kritische Pfad durch das Netzwerk?

\begin{antwort}
1 3 4 6 8 9
\end{antwort}

\end{enumerate}

\end{document}
