\documentclass{lehramt-informatik-aufgabe}
\liLadePakete{cpm,gantt,mathe}
\begin{document}
\liAufgabenTitel{CPM und Gantt}

\section{CPM und Gantt
\index{CPM-Netzplantechnik}
\footcite[Aufgabe 4 (Check-Up), Seite 2]{sosy:ab:5}}
% korrigiert 8.9.2020

\begin{enumerate}

%%
% (a)
%%

\item Gegeben ist folgender (unvollständiger) CPM-Netzplan, sowie die
frühesten und spätesten Termine und die Pufferzeiten aller Ereignisse:

\begin{minipage}{4cm}
\begin{tikzpicture}
\liCpmEreignis{1}(0,1)
\liCpmEreignis{2}(1,1)
\liCpmEreignis{3}(2,2)
\liCpmEreignis{4}(2,0)
\liCpmEreignis{5}(3,1)

\liCpmVorgang(1>2){}
\liCpmVorgang(2>3){}
\liCpmVorgang(2>4){}
\liCpmVorgang(4>5){}
\liCpmVorgang(3>5){}

\liCpmScheinvorgang(3>4){}
\end{tikzpicture}
\end{minipage}
%
\begin{minipage}{5cm}
\begin{tabular}{|l|l|l|l|l|l|}
\hline
Ereignis         & 1 & 2 & 3 & 4 & 5 \\\hline\hline
frühester Termin & 0 & 1 & 2 & 4 & 8 \\\hline
spätester Termin & 0 & 1 & 2 & 5 & 8 \\\hline
Puffer           & 0 & 0 & 0 & 1 & 0 \\\hline
\end{tabular}
\end{minipage}

Vervollständigen Sie den CPM-Netzplan, indem Sie mit Hilfe obiger
Tabelle die Zeiten der Vorgänge berechnen.

\begin{liAntwort}
\begin{tikzpicture}
\liCpmEreignis{1}(0,1)
\liCpmEreignis{2}(1,1)
\liCpmEreignis{3}(2,2)
\liCpmEreignis{4}(2,0)
\liCpmEreignis{5}(3,1)

\liCpmVorgang(1>2){1}
\liCpmVorgang(2>3){1}
\liCpmVorgang(2>4){3}
\liCpmVorgang(4>5){3}
\liCpmVorgang(3>5){6}

\liCpmScheinvorgang(3>4){}
\end{tikzpicture}

%%
%
%%

\liPseudoUeberschrift{Frühester Termin/Zeitpunkt}

\begin{tabular}{|l|r|r|}
\hline
$i$ & Nebenrechnung & $\text{FZ}_i$\\
\hline\hline
1             &                                                & 0 \\\hline
2             & \f$0 + \t{1}(1-2) = 1$                         & 1 \\\hline
3             & \f$\t{1}(1-2) + \t{1}(2-3) = 2$                & 2 \\\hline
4             & \f$\t{1}(1-2) + \t{3}(2-4) = 4$                & \\
              & \f$\t{1}(1-2) + \t{1}(2-3) + \t{0}(3-4) = 2$   & \\
              & \f$\max(4,2)$                                  & 4 \\\hline
5             & \f$\f\t{1}(1-2) + \t{1}(2-3) + \t{6}(3-5) = 8$ & \\
              & \f$\t{1}(1-2) + \t{3}(2-4) + \t{3}(4-5) = 7$   & \\
              & \f$\max(8,7)$                                  & 8 \\\hline
\end{tabular}

%%
%
%%

\liPseudoUeberschrift{Spätester Termin/Zeitpunkt}

\begin{tabular}{|l|r|r|}
\hline
$i$ & Nebenrechnung & $\text{SZ}_i$\\
\hline\hline
1             & \f$8 - \t{6}(3-5) - \t{1}(2-3) - \t{1}(1-2) = 0$ & \\
              & \f$8 - \t{3}(4-5) - \t{3}(2-4) - \t{1}(1-2) = 1$ & \\
              & \f$\min(0,1)$ & 0 \\\hline

2             & \f$8 - \t{6}(3-5) - \t{1}(2-3) = 1$ & \\
              & \f$8 - \t{3}(4-5) - \t{3}(2-4) = 2$ & \\
              & \f$\min(1,2)$ & 1 \\\hline

3             & \f$8 - \t{6}(3-5) = 2$ & 2  \\\hline

4             & \f$8 - \t{3}(4-5) = 5$      & 3 \\\hline
5             & \f{}siehe $\text{FZ}_5$  & 8 \\\hline
\end{tabular}
\end{liAntwort}

%%
% (b)
%%

\item Bestimmen Sie zum nachfolgenden CPM-Netzplan für jedes Ereignis
den \emph{frühesten Termin}, den \emph{spätesten Termin} sowie die
\emph{Gesamtpufferzeit}. Geben Sie außerdem den \emph{kritischen Pfad}
an.

\begin{center}
\begin{tikzpicture}
\liCpmEreignis{1}(0,2)
\liCpmEreignis{2}(2,4)
\liCpmEreignis{3}(2,0)
\liCpmEreignis{4}(4,2)
\liCpmEreignis{5}(6,4)
\liCpmEreignis{6}(6,0)
\liCpmEreignis{7}(8,2)

\liCpmVorgang(1>2){4}
\liCpmVorgang(1>3){8}
\liCpmVorgang(2>3){5}
\liCpmVorgang(2>4){7}
\liCpmVorgang(2>5){8}
\liCpmVorgang(3>4){1}
\liCpmVorgang(3>6){6}
\liCpmVorgang(4>5){2}
\liCpmVorgang(4>6){5}
\liCpmVorgang(5>6){6}
\liCpmVorgang(5>7){7}
\liCpmVorgang(6>7){2}
\end{tikzpicture}
\end{center}

\begin{liAntwort}
\begin{tabular}{|l|l|l|l|l|l|l|l|}
\hline
i             & 1 & 2 & 3  & 4 & 5  & 6  & 7  \\\hline\hline
$\text{FZ}_i$ & 0 & 4 & 9  & 11 & 13 & 19 & 21  \\\hline
$\text{SZ}_i$ & 0 & 4 & 10 & 11 & 13 & 19 & 21  \\\hline
GP            & 0 & 0 & 1  & 0  & 0  & 0  & 0  \\\hline
\end{tabular}

%%
%
%%

\liPseudoUeberschrift{Frühester Termin/Zeitpunkt}

\begin{tabular}{|l|r|r|}
\hline
$\text{FZ}_i$ & Nebenrechnung & \\
\hline\hline
1 &                                                                      & 0 \\\hline
2 &                                                                      & 4 \\\hline
3 & \f$\max(8, \v{4}(2) + 5) = \max(8, 9)$                                      & 9 \\\hline
4 & \f$\max(\v{9}(3) + 1, \v{4}(2) + 7) = \max(10, 11)$                    & 11 \\\hline
5 & \f$\max(\v{4}(2) + 8, \v{11}(4) + 2) = \max(12, 13)$                   & 13 \\\hline
6 & \f$\max(\v{13}(5) + 6, \v{11}(4) + 5, \v{9}(3) + 6) = \max(19, 16, 15)$ & 19 \\\hline
7 & \f$\max(\v{13}(5) + 7, \v{19}(6) + 2) = \max(20, 21)$                  & 21 \\\hline
\end{tabular}

%%
%
%%

\liPseudoUeberschrift{Spätester Termin/Zeitpunkt}

\begin{tabular}{|l|r|r|}
\hline
$\text{SZ}_i$ & Nebenrechnung & \\
\hline\hline
1 & \f$\min(\v{4}(2) - 4, \v{10}(3) - 8) = \min(0, 2)$                    & 0 \\\hline
2 & \f$\min(\v{13}(5) - 8, \v{11}(4) - 7, \v{10}(3) - 5) = \min(5, 4, 5)$ & 4\\\hline
3 & \f$\min(\v{11}(4) - 1, \v{19}(6) - 6) = \min(10, 13)$                 & 10 \\\hline
4 & \f$\min(\v{13}(5) - 2, \v{19}(6) - 5) = \min(11, 14)$                 & 11 \\\hline
5 & \f$\min(\v{21}(7) - 7, \v{19}(6) - 6) = \min(14, 13)$                 & 13 \\\hline
6 & \f$\v{21}(7) - 2$                                                     & 19 \\\hline
7 & \f{}siehe $\text{FZ}_7$                                               & 21 \\\hline
\end{tabular}

%%
%
%%

\liPseudoUeberschrift{Kritischer Pfad}

$1 \rightarrow 2 \rightarrow 4 \rightarrow 5 \rightarrow 6 \rightarrow 7$

$\t{4}(1-2) + \t{7}(2-4) + \t{2}(4-5) + \t{6}(5-6) + \t{2}(6-7) = 21$

\begin{center}
\begin{tikzpicture}[scale=0.8,transform shape]
\liCpmEreignis{1}(0,2)
\liCpmEreignis{2}(2,4)
\liCpmEreignis{3}(2,0)
\liCpmEreignis{4}(4,2)
\liCpmEreignis{5}(6,4)
\liCpmEreignis{6}(6,0)
\liCpmEreignis{7}(8,2)

\liCpmVORGANG(1>2){4}
\liCpmVorgang(1>3){8}
\liCpmVorgang(2>3){5}
\liCpmVORGANG(2>4){7}
\liCpmVorgang(2>5){8}
\liCpmVorgang(3>4){1}
\liCpmVorgang(3>6){6}
\liCpmVORGANG(4>5){2}
\liCpmVorgang(4>6){5}
\liCpmVORGANG(5>6){6}
\liCpmVorgang(5>7){7}
\liCpmVORGANG(6>7){2}
\end{tikzpicture}
\end{center}
\end{liAntwort}

%%
% (c)
%%

\item Konvertieren Sie das nachfolgende Gantt-Diagramm in ein
CPM-Netzwerk. Als Hilfestellung ist die Anordnung der Ereignisse bereits
vorgegeben.

\begin{center}
\begin{ganttchart}[x unit=1cm, y unit chart=0.8cm]{0}{9}
\gantttitlelist{0,...,9}{1} \\
\ganttbar[name=A]{A}{0}{2} \\
\ganttbar[name=B]{B}{1}{4} \\
\ganttbar[name=C]{C}{3}{4} \\
\ganttbar[name=D]{D}{6}{8}

\node at (A) {3};
\node at (B) {4};
\node at (C) {2};
\node at (D) {3};

\ganttlink[link type=s-s]{A}{B}
\ganttlink[link type=f-s]{A}{C}
\ganttlink[link type=f-f]{B}{D}
\ganttlink[link type=s-f]{B}{C}
\ganttlink[link type=s-s]{C}{D}
\end{ganttchart}
\end{center}

\begin{liAntwort}
\begin{center}
\begin{tikzpicture}[scale=0.8,transform shape]
\liCpmEreignis{SP}(-1.5,2)

\liCpmEreignis{A1}(0,2)
\liCpmEreignis{A2}(1.5,4)
\liCpmEreignis{B1}(1.5,0)
\liCpmEreignis{B2}(6,0)
\liCpmEreignis{C1}(3,4)
\liCpmEreignis{C2}(5.5,1.5)
\liCpmEreignis{D1}(6,4)
\liCpmEreignis{D2}(7.5,2.5)

\liCpmEreignis{EP}(9,2)

\liCpmVorgang(A1>A2){3}
\liCpmVorgang(A1>B1){1}
\liCpmVorgang(B1>B2){4}
\liCpmVorgang(B1>C2){4}
\liCpmVorgang(C1>C2){2}
\liCpmVorgang(C1>D1){3}
\liCpmVorgang(D1>D2){3}

\liCpmScheinvorgang(A2>C1){}
\liCpmScheinvorgang(B2>D2){}
\liCpmScheinvorgang(D2>EP){}
\liCpmScheinvorgang(C2>EP){}
\liCpmScheinvorgang(SP>A1){}
\end{tikzpicture}
\end{center}
\end{liAntwort}
\end{enumerate}
\end{document}
