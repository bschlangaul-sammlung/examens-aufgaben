\documentclass{lehramt-informatik-haupt}
\liLadePakete{gantt,cpm}
\begin{document}

%%%%%%%%%%%%%%%%%%%%%%%%%%%%%%%%%%%%%%%%%%%%%%%%%%%%%%%%%%%%%%%%%%%%%%%%
% Theorie-Teil
%%%%%%%%%%%%%%%%%%%%%%%%%%%%%%%%%%%%%%%%%%%%%%%%%%%%%%%%%%%%%%%%%%%%%%%%

\chapter{Gantt-Diagramm}

\begin{liQuellen}
\item \cite[Seite 194]{schatten}
\item \cite{wiki:gantt-diagramm}
\end{liQuellen}

Ein Gantt-Diagramm ist ein Instrument des Projektmanagements, das die
zeitliche Abfolge von Aktivitäten graphisch in Form von Balken auf einer
Zeitachse darstellt.

\section{Anordnungsbeziehungen im Gantt Diagramm\footcite[Seite 34]{sosy:fs:3}}

\begin{description}
\item[Normalfolge EA:] end-to-start relationship Anordnungsbeziehung vom
Ende eines Vorgangs zum Anfang seines Nachfolgers.

\begin{center}
\begin{ganttchart}{1}{9}
\ganttbar{}{2}{4} \\
\ganttbar{}{6}{8}
\ganttlink[link type=f-s]{elem0}{elem1}
\end{ganttchart}
\end{center}

\item[Anfangsfolge AA:] start-to-start relationship Anordnungsbeziehung
vom Anfang eines Vorgangs zum Anfang seines Nachfolgers.

\begin{center}
\begin{ganttchart}{1}{9}
\ganttbar{}{2}{4} \\
\ganttbar{}{6}{8}
\ganttlink[link type=s-s]{elem0}{elem1}
\end{ganttchart}
\end{center}

\item[Endefolge EE:] finish-to-finish relationship Anordnungsbeziehung
vom Ende eines Vorgangs zum Ende seines Nachfolgers.

\begin{center}
\begin{ganttchart}{1}{9}
\ganttbar{}{2}{4} \\
\ganttbar{}{6}{8}
\ganttlink[link type=f-f]{elem0}{elem1}
\end{ganttchart}
\end{center}

\item[Sprungfolge AE:] start-to-finish relationship Anordnungsbeziehung
vom Anfang eines Vorgangs zum Ende seines Nachfolgers

\begin{center}
\begin{ganttchart}{1}{9}
\ganttbar{}{2}{4} \\
\ganttbar{}{6}{8}
\ganttlink[link type=s-f]{elem0}{elem1}
\end{ganttchart}
\end{center}

\end{description}

\section{Von Gantt nach CPM}

\begin{enumerate}

\item Zeichne (falls gefordert bzw. nötig) für das Projekt einen Start-
und Endknoten

\item Zeichen für jede Aktivität (= Balken im Gantt Diagramm) einen
Start- und einen Endknoten

\item Zeichne zwischen Start- und Endknoten einen Vorgang ein. Seine
Dauer entspricht der Länge des Balkens (falls Pufferzeiten in Klammern
angegeben sind werden sie von der Balkenlänge abgezogen).

\item Zeichne für jede Anordnungsbeziehung die im Gantt Diagramm
eingezeichnet ist einen Vorgang ein. Ist der zeitliche Versatz zwischen
dem Balken der Aktivität und dem der Nachfolge-Aktivität

\begin{itemize}
\item positiv, so entspricht die Richtung des Pfeils der im Gantt
Diagramm ist.

\item negativ so wird die Richtung des Pfeils im CPM Netz umgekehrt.

\item null, so zeichne einen Scheinvorgang ein.
\end{itemize}

Der Betrag des zeitlichen Versatzes der Balken ist die Dauer des
Vorgangs (also immer positiv).
\end{enumerate}

\footcite[Seite 35]{sosy:fs:3}

\section{Von CPM nach Gantt}

\begin{enumerate}
\item Zeichne für jeden Vorgang einen Balken ein. Beginne beim FAZ und
ende beim SEZ des Vorgangs. Optional: Trage die Bezeichnung des Vorgangs
ein.

\item Optional (falls gefordert): Trage Dauer und Puffer ein.

\item Pfeile im CPM die zwischen Anfangs- und Endpunkt verschiedener
Vorgänge verlaufen werden zu Anordnungsbeziehungen (AOBs):

\begin{itemize}
\item identifiziere im CPM denjenigen Vorgang, der der Nachfolgevorgang
ist (z.B. Entwurf ist Nachfolger der Anforderungsanalyse).

\item identifiziere damit die Art der AOB (EA, AA, EE, AE)

\item Zeichne sie im Gantt Diagramm ein

\item Optional (falls gefordert): beschrifte sie mit der identifizierten
Art (EA, AA, EE, AE) und ggf. der

\begin{itemize}
\item Verzögerung = positiver Wert: Pfeil zeigt im CPM zum
Nachfolgevorgang

\item Überlappung = negativer Wert: Pfeil geht im CPM vom
Nachfolgevorgang aus
\end{itemize}
\end{itemize}
\end{enumerate}

\footcite[Seite 36]{sosy:fs:3}

\literatur

\end{document}
