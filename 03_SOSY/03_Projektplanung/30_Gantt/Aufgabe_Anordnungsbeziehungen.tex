\documentclass{lehramt-informatik-aufgabe}
\liLadePakete{}
\begin{document}
\liAufgabenTitel{Anordnungsbeziehungen}

\section{Aufgabe 2: Gantt
\index{Gantt-Diagramm}
\footcite[Seite 1]{sosy:ab:5}}

In Gantt-Diagrammen unterscheidet man vier Anordnungsbeziehungen:
Normalfolge (EA), Anfangsfolge (AA), Endfolge (EE) und Sprungfolge
(AE). Ordnen Sie folgende Beispielen den Anordnungsbeziehungen (EA, AA,
EE, AE) zu.

\begin{enumerate}

%%
% (a)
%%

\item Tests durchführen und Dokumentation erstellen

\begin{antwort}
EE (Das Testen muss abgeschlossen sein bevor die Erstellung der
Dokumentation beendet werden kann.)
\end{antwort}

%%
% (b)
%%

\item Systementwurf und Implementierung

\begin{antwort}
AA (Der Beginn des Systementwurfs ist Voraussetzung für den Beginn der
Implementierung.)
\end{antwort}

%%
% (c)
%%

\item neue Anwendung in Betrieb nehmen und alte Anwendung abschalten

\begin{antwort}
AE (Die alte Anwendung kann erst abgeschaltet werden, wenn die neue
Anwendung in Betrieb genommen wurde.)
\end{antwort}

%%
% (d)
%%

\item Implementierung und Dokumentation

\begin{antwort}
EE (Die Implementierung muss abgeschlossen sein bevor die Dokumentation
beendet werden kann.)
\end{antwort}

%%
% (e)
%%

\item Abitur schreiben und Studieren

\begin{antwort}
EA (Das Abitur muss abgeschlossen sein bevor mit dem Studium begonnen
werden kann.)
\end{antwort}

%%
% (f)
%%

\item Führerschein machen und selbstständiges Autofahren

\begin{antwort}
EA (Der Führerschein muss bestanden sein bevor mit dem selbstständigen
Autofahren begonnen werden kann.)
\end{antwort}

%%
% (g)
%%

\item Studieren und Buch aus Uni-Bibliothek ausleihen

\begin{antwort}
AA (Der Beginn des Studiums ist Voraussetzung für das Ausleihen eines
Buches aus der Uni-Bibliothek.)
\end{antwort}
\end{enumerate}
\end{document}
