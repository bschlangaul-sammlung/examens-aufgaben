\documentclass{lehramt-informatik-aufgabe}
\liLadePakete{}
\begin{document}
\liAufgabenTitel{DVD-Automat}

\section{Zustands-\index{Zustandsdiagramm} und einem Aktivitätsdiagramm\index{Aktivitätsdiagramm}
\footcite{sosy:ab:3}}

\begin{enumerate}
\item Beschreiben Sie den Unterschied zwischen einem Zustands- und einem
Aktivitätsdiagramm.

\item Der DVD-Automat für die Filmauswahl aus Blatt 2, Aufgabe 1 soll
als Zustandsdiagramm modelliert werden. Beachten Sie dabei die
angegebenen Funktionalitäten des Automaten.

\begin{enumerate}
\item Geben Sie die Ein- und Ausgaben des Automaten für die Filmauswahl
und für die Aus- und Rückgabe von Filmen an.

\item Geben Sie alle Zustandsattribute an, die für die Modellierung der
Automaten notwendig sind und beschreiben Sie deren Verwendungszweck.

\item Identifizieren Sie anhand der Zustandsattribute die Zustände der
obigen Automaten und geben Sie eine Charakterisierung der Zustände durch
Angabe der möglichen Wertebereiche der Zustandsattribute an. Welcher der
Zustände ist der Anfangszustand?

\item Zeichnen Sie die Zustandsübergangsdiagramme. Verwenden Sie hierzu
die Syntax mit Ein- und Ausgabe, Vor- und Nachbedingungen.
\end{enumerate}

\item Betrachten wir nun einen gewöhnlichen Video- und DVD-Verleih.
Beschreiben Sie das Szenario des Ausleihs eines Videos mit Hilfe eines
Aktivitätsdiagramms. Bei dem Videoverleih gelte:

\begin{itemize}
\item Der Kunde identifiziert sich beim Ausleihen mit seiner Kundenkarte
oder seinem Passwort. Hat der Kunde noch keine Karte, so muss der
Mitarbeiter ihn registrieren und ihm eine Kundenkarte ausstellen.

\item Filme können wie beim Automaten per Internet bis zu zwei Stunden
im Voraus reserviert werden.

\item Der Kunde hat kein Gehaltskonto, sondern bezahlt seine Gebühren
bei der Rückgabe des Videos in bar oder per Karte.

\item Ansonsten gelten die gleichen Bedingungen wie beim Automaten.
\end{itemize}
\end{enumerate}

\end{document}
