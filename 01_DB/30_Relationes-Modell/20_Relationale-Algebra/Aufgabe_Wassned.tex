\documentclass{lehramt-informatik-aufgabe}
\liLadePakete{syntax,mathe}
\begin{document}
\liAufgabenTitel{Wassned}

\section{Aufgabe 4: Relationale Algebra Einstieg
\index{Relationale Algebra}
\footcite{db:pu:2}}

Gegeben ist folgende Datenbank-Anfrage:

\begin{math}
\pi_{\text{Bezeichnung}}(
  \sigma_{\text{SWS} = 2 \land\neg(\text{Name} = \mlq \text{Wassned} \mrq)}(
    \text{Vorlesung} \bowtie \text{Professor}
  )
)
\end{math}

\begin{enumerate}

%%
% (a)
%%

\item Geben Sie eine umgangssprachliche Formulierung der Anfrage an!

\begin{antwort}[muster]
Eine Liste mit den Bezeichnungen der Vorlesungen, die 2 Semester
Wochenstunden dauern und die nicht vom dem Professor „Wassned“ gelesen
werden.
\end{antwort}

%%
% (b)
%%

\item Geben Sie die Ergebnistabelle an!

\begin{antwort}[muster]

\begin{tabular}{|l|}
\hline
\textbf{Bezeichnung}\\
Japanische Malerei\\
Chinesische Schrift\\
\hline
\end{tabular}
\end{antwort}
\end{enumerate}
\end{document}
