\documentclass{lehramt-informatik-haupt}
\liLadePakete{mathe,normalformen}

\begin{document}
\let\ah=\liAttributHuelle
\let\m=\liAttributMenge

%-----------------------------------------------------------------------
%
%-----------------------------------------------------------------------

\chapter{Schlüssel}

\begin{description}
\item[Superschlüssel] (gelegentlich auch Oberschlüssel genannt) Attribut
oder Attributkombination, von der alle Attribute einer Relation
funktional abhängen
\footcite[Seite 181 Kapitel 6.2 „Superschlüssel“]{kemper}

\item[Schlüsselkandidat] (auch Kandidatenschlüssel oder
Alternativschlüssel genannt) \textbf{Minimaler} Superschlüssel (Keine
Teilmenge dieses Superschlüssels ist ebenfalls Superschlüssel)

\item[Primärschlüssel] Unter allen Schlüsselkandidaten einer
Relation wird ein sogenannter \textbf{Primärschlüssel} ausgewählt.

\item[Schlüsselattribut] Attribut, das Teil eines
Schlüsselkandidaten ist

\item[Nicht-Schlüsselattribut] Attribut, das an keinem der
Schlüsselkandidaten beteiligt ist
\end{description}

%-----------------------------------------------------------------------
%
%-----------------------------------------------------------------------

\subsection{Bestimmung von Schlüsselkandidaten ohne Algorithmus}

\begin{itemize}
\item Attribute, die auf \textbf{keiner rechten Seite} einer FD
vorkommen.

\item Attribute, die in \textbf{keiner FD} vorkommen.
\end{itemize}

%-----------------------------------------------------------------------
%
%-----------------------------------------------------------------------

\subsection{Algorithmus zur Bestimmung von Schlüsselkandidaten}

\href{http://www.ict.griffith.edu.au/normalization_tools/normalization/ind.php}
{Normalization Tool of the Griffith Univerity}

Foliensätze Seite 8

\begin{itemize}
\item
$Erg = \m{}$,
$Test = \m{\m{\textit{alle Attribute der Relation}}}$,
$K = \m{\textit{alle Attribute der Relation}}$

\begin{itemize}
\item Wähle ein Attribut aus $K$, ist
$\ah{F,K \SLASH \m{A}} = R$?
(Also die Menge $K$ ohne das Attribut $A$)

\item ja $\Rightarrow$ ändere $\textit{Test}$:
streiche $K$ aus $\textit{Test}$, füge $K \SLASH \m{A}$ in $\textit{Test}$ ein.

\item $K$ selbst bleibt unverändert!

\item entferne ein anderes Attribut aus $K$, so lange, bis alle
Attribute reihum untersucht wurden.
\end{itemize}

\item wenn kein $K\m{A}$ Superschlüssel $\Rightarrow$
$K$ ist Schlüsselkandidat!
Füge $K$ zu $Erg$ hinzu und lösche $K$ aus $\textit{Test}$.

\item mache dasselbe mit allen Mengen, die jetzt in $\textit{Test}$ sind,
bis $\textit{Test}$ leer ist.

\end{itemize}

\subsection{Am Beispiel einer Aufgabe}

Seite 12

Gegeben sei ein Relationenschema $R$ mit Attributen $A, B, C, D$. Für
dieses Relationenschema seien die folgenden Mengen an funktionalen
Abhängigkeiten (FDs) gegeben:

\begin{multline*}
F = \{ \\
  A \rightarrow B,\\
  B \rightarrow C,\\
  A \rightarrow D,\\
\}
\end{multline*}

%%
%
%%

\subsubsection{Abkürzung}

$A$ kommt auf keiner rechten Seite der FD’s vor.
Man kann es über FD's nicht erreichen. $A$ muss also Teil des
Schlüsselkandidaten sein.

$\ah{F, \m{A}} = R = \textit{Superschlüssel}$

A ist minimal Schlüsselkandidat.
%
Jeder weitere Schlüsselkandidat muss ebenfalls minimal sein und zudem
A enthalten $\rightarrow$ A einziger Schlüsselkandidat.

%%
%
%%

\subsubsection{Mit Hilfe des Algorithmus:}

$Test = \m{\m{A, B, C, D}}$ $Erg = \m{}$

\begin{enumerate}

%%
% 1.
%%

\item $K = \m{A, B, C, D}$

$K \SLASH A: \ah{F, \m{B, C, D}} = \m{B, C, D}$ !\\

$K \SLASH B: \ah{F, \m{A, C, D}} = R$\\
$\rightarrow Test = \m{\m{A, C, D}}$ \\

$K \SLASH C: \ah{F, \m{A, B, D}} = R$\\
$\rightarrow Test = \m{\m{A, C, D}, \m{A, B, D}}$ \\

$K \SLASH D: \ah{F, \m{A, B, C}} = R$\\
$\rightarrow Test = \m{\m{A, C, D}, \m{A, B, D}, \m{A, B, C}}$ \\

%%
% 2.
%%

\item $K = \m{A, C, D}$

$K \SLASH A: \ah{F, \m{C, D}} = \m{C, D}$ !\\

$K \SLASH C: \ah{F, \m{A, D}} = R$\\
$\rightarrow Test = \m{\m{A, D}, \m{A, C, D}, \m{A, B, D}, \m{A, B, C}}$ \\

$K \SLASH C: \ah{F, \m{A, C}} = R$\\
$\rightarrow Test = \m{\m{A, C}, \m{A, D}, \m{A, C, D}, \m{A, B, D}, \m{A, B, C}}$ \\

%%
% 3.
%%

\item $K = \m{A, C}$

$K \SLASH A: \ah{F, \m{C}} = \m{C}$ !\\

$K \SLASH C: \ah{F, \m{A}} = R$\\
$\rightarrow Test = \m{\m{A}, \m{A, D}, \m{A, C, D}, \m{A, B, D}, \m{A, B, C}}$ \\

%%
% 4.
%%

\item $K = \m{A}$

$K \SLASH A$: ! $\rightarrow$ kein Superschlüssel ohne A mehr möglich\\
$\rightarrow$ dieses K wandert in Ergebnis und wird in Test gelöscht

$\rightarrow Test = \m{\m{A, D}, \m{A, C, D}, \m{A, B, D}, \m{A, B, C}}$ \\
$\rightarrow Erg = \m{\m{A}}$ \\
\end{enumerate}

analog verfahren wir mit den übrigen Mengen in Test, wie man bereits
sieht bleibt $\m{A}$ einziger Schlüsselkandidat.

\literatur

\end{document}
