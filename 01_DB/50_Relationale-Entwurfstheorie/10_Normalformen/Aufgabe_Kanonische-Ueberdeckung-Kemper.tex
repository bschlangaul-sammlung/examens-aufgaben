\documentclass{lehramt-informatik-aufgabe}
\liLadePakete{normalformen}
\begin{document}
\liAufgabenTitel{Kanonische Überdeckung (Kemper)}

\section{Kanonische Überdeckung (kleines Beispiel aus Kemper)
\index{Kanonische Überdeckung}
\footcite[Seite 186]{kemper}
}

$F = \{\lifa A > B, B \rightarrow C, AB \rightarrow C\}$

\begin{compactitem}
\item \lifa A > B
\item \lifa B > C
\item \lifa  AB > C
\end{compactitem}

\begin{enumerate}
\item Kanonische Überdeckung

\begin{enumerate}
\item Linksreduktion

$\liAH{F, A} = \{A, B, C\}$

\begin{compactitem}
\item $A \rightarrow B$
\item $B \rightarrow C$
\item $A \rightarrow C$
\end{compactitem}

\item Rechtsreduktion

\begin{compactitem}
\item $A \rightarrow B$
\item $B \rightarrow C$
\item $A \rightarrow \emptyset$
\end{compactitem}

\item Leere Klauseln leeren

\begin{compactitem}
\item $A \rightarrow B$
\item $B \rightarrow C$
\end{compactitem}

\item Vereinigung

\item Neues Relationenschema

\begin{compactitem}
\item $R1(A, B)$
\item $R2(B, C)$
\end{compactitem}

\item Hinzufügen einer Relation
\item Entfernen überflüssiger Teilschemata

\end{enumerate}
\end{enumerate}

\end{document}
