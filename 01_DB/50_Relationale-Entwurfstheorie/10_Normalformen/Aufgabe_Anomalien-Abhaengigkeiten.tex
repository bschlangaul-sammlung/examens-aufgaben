\documentclass{lehramt-informatik-aufgabe}
\liLadePakete{normalformen}

\begin{document}
\let\fa=\liFunktionaleAbhaengigkeit
\let\FA=\liFunktionaleAbhaengigkeiten
\let\ah=\liAttributHuelle
\let\m=\liAttributMenge

\liAufgabenTitel{Anomalien Abhängigkeiten}

\section{Anomalien und Abhängigkeiten
\footcite[Seite 1, Aufgabe 1: Anomalien und Abhängigkeiten]{db:ab:5}
}

Gegeben ist die Relation \emph{Abteilungsmitarbeiter}, repräsentiert
durch folgende Tabelle. Es sei angenommen, dass innerhalb einer
Abteilung keine Mitarbeiter mit identischem Namen existieren. Die
Abteilungsnummer ist eindeutig, es kann aber durchaus sein, dass mehrere
Abteilungen die gleiche Bezeichnung tragen.

\begin{tabular}{lllll}
Name           & Straße       & Ort     & AbtNr & Bezeichnung \\
Schweizer      & Hauptstraße  & Zürich  & A3    & Finanzen    \\
Deutscher      & Lindenstraße & Passau  & A4    & Informatik  \\
Österreicher   & Nebenstraße  & Wien    & A4    & Informatik
\end{tabular}

\begin{enumerate}

%%
% (a)
%%

\item Geben Sie - orientiert an der obigen Tabelle - ein Beispiel für
eine mögliche Änderungsanomalie an!

\begin{liAntwort}
\begin{description}
\item[Update-Anomalie]
\index{Update-Anomalie}

Die Abteilung A4 wird umbenannt, beispielsweise in Softwareabteilung.
Die Änderung wird aus Versehen nicht in allen Tupeln mit AbtNr = A4
vollzogen.

\item[Delete-Anomalie]
\index{Delete-Anomalie}

Herr Schweizer (aus der Abteilung A4) verlässt die Firma und wird aus
der Datenbank gelöscht. Damit gehen auch die Daten über die Abteilung A3
verloren.

\item[Insert-Anomalie]
\index{Insert-Anomalie}

Es wird eine neue Abteilung A5 (Hardwareabteilung) geschaffen, der aber
noch keine Mitarbeiter zugeteilt sind. Damit müsste ein Tupel (NULL,
NULL, NULL, A5, Hardwareabteilung) in die Datenbank eingefügt werden. Da
aber das Attribut Name sicher in jedem Schlüsselkandidaten enthalten
sein muss, kann der Wert von Name keinen Nullwert enthalten. Das Tupel
kann nicht eingefügt werden.

\end{description}
\end{liAntwort}

%%
% (b)
%%

\item Bestimmen Sie eine Menge $F$ der funktionalen Abhängigkeiten, die
sich aus Ihrer Analyse des Anwendungsbereiches ergeben. (Triviale
Abhängigkeiten brauchen nicht angegeben werden.) Begründen Sie Ihre
Entscheidung kurz.
\index{Funktionale Abhängigkeiten}

\begin{liAntwort}
\begin{itemize}
\item \fa{AbtNr -> Bezeichnung} \\
Die Abteilungsnummer ist eindeutig (als „künstliches“
Unterscheidungsmerkmal für Abteilungen) und legt damit die Abteilung
eindeutig fest.

\item \fa{Name, AbtNr -> Strasse, Ort} \\
Da der Name innerhalb der Abteilung eindeutig ist, ist damit der
Mitarbeiter und folglich auch die Adressdaten eindeutig festgelegt. Da
es sich bei dieser Attributkombination um den Primärschlüssel handelt,
bestimmt diese Attributkombination auch das Attribut „Bezeichnung“,
allerdings darf es nicht in diese FA aufgenommen werden, da die
Abteilungsbezeichnung nicht von der Kombination aus Name \& AbtNr
abhängig, sondern nur von der AbtNr allein, somit muss dies als
einzelne FA formuliert werden und kann hier nicht aufgenommen werden

$\rightarrow$ der Rückschluss daraus wäre nämlich, dass sich die
Bezeichnung der Abteilung nur aus der Kombination von Mitarbeiter und
AbtNr erkennen lässt und nicht allein aus der AbtNr und das wäre ja
nicht korrekt. Grundsätzlich gilt: Primärschlüssel und FA’s müssen
getrennt betrachtet werden!

\FA[F]{
  Name, AbtNr -> Strasse, Ort;
  AbtNr -> Bezeichnung;
}
\end{itemize}

\end{liAntwort}

%%
% (c)
%%

\item Bestimmen Sie z.B. mit Hilfe des Attributhüllen-Algorithmus die
Attributhülle
\index{Attributhüllen-Algorithmus}\index{Attributhülle}

\begin{enumerate}

%%
% 1.
%%

\item $\ah{F, \m{Name, Bezeichnung}}$

\begin{liAntwort}
$\ah{F, \m{Name, Bezeichnung}} = \m{Name, Bezeichnung}$
\end{liAntwort}

%%
% 2.
%%

\item $\ah{F, \m{Name, AbtNr}}$

\begin{liAntwort}
$\ah{F, \m{Name, AbtNr}} = \m{Name, AbtNr, Strasse, Ort, Bezeichnung}$
\end{liAntwort}
\end{enumerate}

%%
% (d)
%%

\item Ist \m{Name, Bezeichnung} (bzw. \m{Name, AbtNr} ein Superschlüssel
der Relation Abteilungsmitarbeiter? Kurze Begründung!
\index{Superschlüssel}

\begin{liAntwort}
${Name, AbtNr}$
\end{liAntwort}

\end{enumerate}
\end{document}
