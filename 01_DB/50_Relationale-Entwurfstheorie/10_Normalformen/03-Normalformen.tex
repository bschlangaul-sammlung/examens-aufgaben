\documentclass{lehramt-informatik-haupt}
\liLadePakete{mathe,normalformen}
\usepackage{tabularx}
\usepackage{paralist}

\begin{document}
\let\ah=\liAttributHuelle
\let\m=\liMenge

%%%%%%%%%%%%%%%%%%%%%%%%%%%%%%%%%%%%%%%%%%%%%%%%%%%%%%%%%%%%%%%%%%%%%%%%
% Theorie-Teil
%%%%%%%%%%%%%%%%%%%%%%%%%%%%%%%%%%%%%%%%%%%%%%%%%%%%%%%%%%%%%%%%%%%%%%%%

\chapter{Normalformen}

\begin{lernkartei}{Normalformen}
\begin{description}
\item[1NF] alle Attribute atomar

\item[2NF] 1NF $+$ kein Nichtprimärattribut hängt funktional von einer
echten Teilmenge eines Schlüsselkandidaten ab

\item[3NF] 2NF $+$ keine transitiven Abhängigkeiten über
Nichtschlüsselkandidaten

\item[BCNF] 3NF $+$ jede Determinante ein Schlüsselkandidat\footcite[Seite 179]{kemper}
\end{description}
\end{lernkartei}

%-----------------------------------------------------------------------
%
%-----------------------------------------------------------------------

\section{Anomalien}

\begin{lernkartei}{Anomalien}
\begin{itemize}
\item Update-Anomalie
\item Einfüge- oder Insert-Anomalie
\item Lösch- oder Delete-Anomalie
\end{itemize}
\end{lernkartei}

%-----------------------------------------------------------------------
%
%-----------------------------------------------------------------------

\section{Attributhülle}

\begin{lernkartei}{Bestimmung der Attributhülle}
\begin{flalign*}
\textit{Hülle}(F,\alpha )\\
    \alpha^{+}=\alpha\\
   \textbf{while} (\text{Änderung} an \alpha^{+}) do\\
      \textbf{foreach}(\textit{Abhängigkeit} \beta \rightarrow \gamma \in F) do\\
        \textbf{if} (\beta \subseteq \alpha^{+} ) \textbf{then} \alpha^{+}=\alpha^{+}\cup \gamma
\end{flalign*}
\end{lernkartei}

Die Attributhülle $\alpha ^{+}$ eines bestimmten Attributs $\alpha$
(müsste eigentlich heißen: einer bestimmten Menge von Attributen) ist
eine Liste aller Attribute, die von $\alpha$ funktional
abhängen.\footcite[Attributhülle]{wiki:funktionale-abhängigkeit}

Durch die Bestimmung der Attributhülle kann angegeben werden, welche anderen
Attribute in einer Relation durch die \memph{gegebenen} Attribute
bestimmt werden können.\footcite[Seite 8]{db:fs:4}\footcite[Seite 179]{winter}

%-----------------------------------------------------------------------
%
%-----------------------------------------------------------------------

\subsection{Attributhüllenalgorithmus\footcite[Seite 179]{winter}}

Beispiel:\footcite[Seite 8]{db:fs:4}

$F$:

$PQ \rightarrow R$

$PR \rightarrow Q$

$Q \rightarrow S$

$PS \rightarrow T$

$QR \rightarrow P$

$T \rightarrow U$

Gesucht

$\ah{F, \m{P,Q}}$

\noindent
\begin{tabularx}{\linewidth}{lX}
$\m{P, Q}$ &
Initialisierung von Erg mit den vorgegebenen Attributwerten P und Q
\\\hline

$\m{P, Q} \cup \m{R}$  &
Die linke Seite der FD $P Q \rightarrow R$ ist bereits in Erg enthalten.
Also werden die Attribute der rechten Seite zur Menge Erg hinzugefügt.
\\\hline

$\m{P, Q, R}$ &
Wegen der FD $PR \rightarrow Q$ käme Q hinzu, ist aber schon in Erg
enthalten. Also bleibt Erg unverändert.
\\\hline

$\m{P, Q, R} \cup \m{S}$ &
Die linke Seite der FD $Q \rightarrow S$ ist bereits in Erg enthalten.
Also werden die Attribute der rechten Seite zur Menge Erg hinzugefügt.
\\\hline

$\m{P, Q, R, S} \cup \m{T}$ &
Die linke Seite der FD $PS \rightarrow T$ ist bereits in Erg enthalten.
Also werden die Attribute der rechten Seite zur Menge Erg hinzugefügt.
\\\hline

$\m{P, Q, R, S, T}$ &
Wegen der FD $QR \rightarrow P$ käme P hinzu, ist aber schon in Erg
enthalten. Also bleibt Erg unverändert.
\\\hline

$\m{P, Q, R, S, T} \cup \m{U}$ &
Die linke Seite der FD $T \rightarrow U$ ist bereits in Erg enthalten.
Also werden die Attribute der rechten Seite zur Menge Erg hinzugefügt.
\\\hline

$\m{P, Q, R, S, T, U}$ &
Es ist also bereits die gesamte Attributmenge erreicht. Weitere
Betrachtungen erübrigen sich hiermit.
\\\hline
\end{tabularx}

\section{Die Normalformen}

%-----------------------------------------------------------------------
%
%-----------------------------------------------------------------------

\subsection{Erste Normalform\footcite[Seite 195]{winter}}

Ein Relationenschema R ist in \textbf{erster Normalform (1NF)}, wenn
alle Attribute atomar sind.

%-----------------------------------------------------------------------
%
%-----------------------------------------------------------------------

\subsection{Zweite Normalform}

Eine Relation ist genau dann in der zweiten Normalform, wenn die erste
Normalform vorliegt und kein Nichtprimärattribut (Attribut, das nicht
Teil eines Schlüsselkandidaten ist) funktional von einer echten
Teilmenge eines Schlüsselkandidaten abhängt.

Anders gesagt: Jedes nicht-primäre Attribut (nicht Teil eines
Schlüssels) ist jeweils von \textbf{allen ganzen Schlüsseln} abhängig,
nicht nur von einem Teil eines Schlüssels. Wichtig ist hierbei, dass die
Nichtschlüsselattribute wirklich von allen Schlüsseln vollständig
abhängen.\footcite[Zweite Normalform (2NF)]{wiki:normalisierung}

%-----------------------------------------------------------------------
%
%-----------------------------------------------------------------------

\subsection{Dritte Normalform\footcite[Seite 201]{winter}}

aufgrund von transitiven Abhängigkeiten über Nichtschlüsselkandidaten
Falls A nicht prim ist, muss A funktional von jedem Superschlüssel
abhängen

Die dritte Normalform ist genau dann erreicht, wenn sich das
Relationenschema in der 2NF befindet, und kein Nichtschlüsselattribut
(hellgraue Zellen in der Tabelle) von einem Schlüsselkandidaten
transitiv abhängt.\footcite[Dritte Normalform (3NF)]{wiki:normalisierung}

%-----------------------------------------------------------------------
%
%-----------------------------------------------------------------------

\subsection{Boyce-Codd-Normalform}

Sei $R$ eine Relationenschema in erster Normalform. Sei $F$ eine Menge
einfacher nichttrivialer funktionaler Abhängigkeiten für $R$. $R$ ist in
Boyce-Codd-Normalform (BCNF) bzgl. $F$, falls für jede FD $X \rightarrow
A$ aus $F$ gilt: $X$ ist ein Superschlüssel von $R$.
\footcite[Boyce-Codd-Normalform (BCNF)]{wiki:normalisierung}

Ein Relationenschema ist in der Boyce-Codd-Normalform, wenn es in der
3NF ist und jede Determinante (Attributmenge, von der andere Attribute
funktional abhängen) ein Schlüsselkandidat ist (oder die Abhängigkeit
ist trivial).

Die BCNF (nach Raymond F. Boyce und Edgar F. Codd) verhindert, dass
Teile zweier aus mehreren Feldern zusammengesetzten Schlüsselkandidaten
voneinander abhängig sind.

Die Überführung in die BCNF ist zwar immer verlustfrei möglich, aber
nicht immer abhängigkeitserhaltend. Die Boyce-Codd-Normalform war
ursprünglich als Vereinfachung der 3NF gedacht, führte aber zu einer
neuen Normalform, die diese verschärft: Eine Relation ist automatisch
frei von transitiven Abhängigkeiten, wenn alle Determinanten
Schlüsselkandidaten sind.

\subsubsection{Beispiel}

Gegeben ist das Relationenschema
$Aufnahmepruefung\m{PersNr, Schuelername, Fach, Note}$
und die Menge von FDs
$F = \{
  PersNr Schuelername \rightarrow Note,
  Schuelername Fach \rightarrow Note,
  PersNr \rightarrow Fach,
  Fach \rightarrow PersNr
\}$.

Die funktionalen Abhängigkeiten implizieren, dass es zwei
Schlüsselkandidaten gibt, nämlich
$\m{PersNr, Schuelername}$ und
$\m{Schuelername, Fach}$.

Das Relationenschema befindet sich in 3NF, da gilt:

\begin{itemize}
\item PersNr, Fach und Schülername sind prime Attribute.

\item Bei allen FDs, auf deren rechten Seite das (einzige) nichtprime
Attribut Note vorkommt, ist die Attributmenge auf der linken Seite ein
Superschlüssel.
\end{itemize}

Trotzdem kann es zu Redundanzen kommen, da die Information, in welchem
Fach eine Lehrkraft prüft, mehrfach abgespeichert wird.

%%%%%%%%%%%%%%%%%%%%%%%%%%%%%%%%%%%%%%%%%%%%%%%%%%%%%%%%%%%%%%%%%%%%%%%%
%
%%%%%%%%%%%%%%%%%%%%%%%%%%%%%%%%%%%%%%%%%%%%%%%%%%%%%%%%%%%%%%%%%%%%%%%%

\chapter{Übungen}

%-----------------------------------------------------------------------
%
%-----------------------------------------------------------------------

\section{Abstaktes R\footcite[Seite 1, Aufgabe 1]{db:ab:6}}

Gegeben sei das Relationenschema $R\m{A, B, C, D, E, G}$ mit

\begin{multline}
F = \{ \\
  E \rightarrow D, \\
  C \rightarrow B, \\
  CE \rightarrow G, \\
  B \rightarrow A \\
\}
\end{multline}

\begin{enumerate}

%%
% (a)
%%

\item Zeigen Sie: ${C, E}$ ist der einzige Schlüsselkandidat von $R$.

\begin{liAntwort}
A, G, D kommen auf keiner linken Seite vor

$\ah{F, \m{C, E}} = R$

$\ah{F, \m{C}} = \m{B} != R$

$\ah{F, \m{E}} = \m{E, D} != R$
\end{liAntwort}

\begin{liAntwort}

$C$ und $E$ kommen auf keiner rechten Seite der FDs aus $F$ vor, d.h.
$C$ und $E$ müssen Teil jedes Schlüsselkandidaten sein.

Außerdem gilt: $\ah{F, \m{C, E}} = \m{A, B, C, D, E, G} = R$

$\m{C, E}$ ist somit Superschlüssel von $R$. Zudem ist $\m{C, E}$
minimal, da beide Attribute Teil jedes SK sein müssen.

$\Rightarrow$ $\m{C, E}$ ist damit der einzige Schlüsselkandidat von $R$
(da kein SK ohne $C$ und $E$ möglich ist).

Anmerkung:

\begin{itemize}
\item Man könnte hier auch einen Algorithmus zur Bestimmung der
Schlüsselkandidaten verwenden, dessen einziges Ergebnis wäre dann $\{C,
E\}$. In diesem Fall lässt sich die Schlüsselkandidateneigenschaft
jedoch einfacher zeigen, sodass man den Algorithmus und somit Zeit
sparen kann.

\item Achtung! $\m{C, E}$ ist zwar der einzige Schlüsselkandidat, aber
nicht der einzige Superschlüssel, auch \m{A, B, C, D, E, G} wäre ein
Superschlüssel!
\end{itemize}

\end{liAntwort}

%%
% (b)
%%

\item Ist $R$ in 2NF?

\begin{liAntwort}
Ist nicht in der 2NF, denn $D$ hängt von $E$ ab, also von einer
echten Teilmenge des Schlüsselkandidaten $\m{C, E}$. Das gleiche gilt
für die FD $C \rightarrow B$.
\end{liAntwort}

\begin{liAntwort}
$R$ ist nicht in 2NF, denn:

Betrachte $E \rightarrow D$: $D$ ist ein Nicht-Schlüsselattribut und $E$
ist echt Teilmenge des Schlüsselkandidaten $\m{C, E}$.
%
Ebenso ist $B$ nicht voll funktional abhängig vom Schlüsselkandidaten,
sondern nur von einer echten Teilmenge des Schlüsselkandidaten, nämlich
$C$.

Anmerkung:

\begin{itemize}
\item Ob alle Attributwerte atomar sind, können wir in einem abstrakten
Schema wie diesem nicht wirklich sagen, daher kann dies Annahme in der
Regel nicht getroffen werden.

\item Dass $A$ von $B$ abhängig ist, spielt bei der Entscheidung über
die 2. NF keine Rolle, da $B$ selbst (genauso wie $A$) ein
Nicht-Schlüsselattribut ist. Wichtig ist nur, ob es Abhängigkeiten
zwischen einem Teil der Schlüsselkandidaten (also einem
Schlüsselattribut) und einem Nicht-Schlüsselattribut gibt.

\item Um der 2NF zu genügen, müsste in folgenden Relationen aufgeteilt
werden:
$R1(C, E, G)$,
$R2(C, B, A)$,
$R3(E, D)$
\end{itemize}

\end{liAntwort}

%%
% (c)
%%

\item Ist $F$ minimal?
\begin{multline}
F = \{ \\
  E \rightarrow D, \\
  C \rightarrow B, \\
  CE \rightarrow G, \\
  B \rightarrow A \\
\}
\end{multline}

\begin{liAntwort}
Kanonische Überdeckung

\begin{enumerate}
\item Linksreduktion\\
$AttrHul\m{F, \m{C}} = \m{C, B}$ $\rightarrow G$ nicht enthalten\\
$AttrHul\m{F, \m{E}} = \m{E, D}$ $\rightarrow G$ nicht enthalten

\item Rechtsreduktion\\
Kein Attribut auf einer rechten Seite ist redundant: Da das einzelne
Attribut, das die rechte Seite einer FD aus F bildet, bei keiner anderen
FD auf der rechten Seite auftritt, kann die rechte Seite einer FD nicht
unter ausschließlicher Verwendung der restlichen FD aus der
entsprechenden linken Seite abgeleitet werden.
\end{enumerate}

\end{liAntwort}

\begin{liAntwort}
Vorgehen: Entsprechen die hier abgebildeten Funktionalen Abhängigkeiten
bereits einer kanonischen Überdeckung von F oder nicht?

\begin{itemize}
\item Eliminierung redundanter Attribute auf der linken Seite: Die
Attributmenge auf den linken Seiten der FDs sind bereits bis auf $CE
\rightarrow G$ einelementig. Bei $CE \rightarrow G$ ist \m{CE} der
Schlüsselkandidat, also kann kein redundantes Attribut vorliegen.

\item Eliminierung redundanter Attribute auf der rechten Seite (hier müssen auch alle
einelementigen FA’s betrachtet werden)

\begin{itemize}
\item $E \rightarrow D$: $\ah{F - \m{E \rightarrow D}, \m{E}} = \m{E}$, d.h.
$D \notin \ah{F - \m{E \rightarrow D}, \m{E}}$

\item $C \rightarrow B$: $\ah{F - \m{C \rightarrow B}, \m{C}} = \m{C}$, d.h.
$B \notin \ah{F - \m{C \rightarrow B}, \m{E}}$

\item $CE \rightarrow G$: $\ah{F - \m{CE \rightarrow G}, \m{C, E}} = \m{A, B, C, D, E}$, d.h.
$G \notin \ah{F - \m{CE \rightarrow G}, \m{E}}$
$\Rightarrow$ $CE \rightarrow G$ ist nicht redundant

\item $B \rightarrow A$: $\ah{F - \m{B \rightarrow A}, \m{B}} = \m{B}$, d.h.

$A \notin \ah{F - \m{B \rightarrow A}, \m{E}}$
$\Rightarrow$ $B \rightarrow A$ ist nicht redundant
\end{itemize}

F ist bereits minimal.
\end{itemize}

\end{liAntwort}

\end{enumerate}

%-----------------------------------------------------------------------
%
%-----------------------------------------------------------------------

\section{Anomalien und Abhängigkeiten
\footcite[Seite 1, Aufgabe 1: Anomalien und Abhängigkeiten]{db:ab:5}
}

Gegeben ist die Relation \emph{Abteilungsmitarbeiter}, repräsentiert
durch folgende Tabelle. Es sei angenommen, dass innerhalb einer
Abteilung keine Mitarbeiter mit identischem Namen existieren. Die
Abteilungsnummer ist eindeutig, es kann aber durchaus sein, dass mehrere
Abteilungen die gleiche Bezeichnung tragen.

\begin{tabular}{lllll}
Name           & Straße       & Ort     & AbtNr & Bezeichnung \\
Schweizer      & Hauptstraße  & Zürich  & A3    & Finanzen    \\
Deutscher      & Lindenstraße & Passau  & A4    & Informatik  \\
Österreicher   & Nebenstraße  & Wien    & A4    & Informatik
\end{tabular}

\renewcommand{\labelenumi}{(\alph{enumi})}
\begin{enumerate}

%%
% (a)
%%

\item Geben Sie - orientiert an der obigen Tabelle - ein Beispiel für
eine mögliche Änderungsanomalie an!

\begin{liAntwort}
\begin{description}
\item[Update-Anomalie] Die Abteilung A4 wird umbenannt, beispielsweise
in Softwareabteilung. Die Änderung wird aus Versehen nicht in allen
Tupeln mit AbtNr = A4 vollzogen.

\item[Delete-Anomalie] Herr Schweizer (aus der Abteilung A4) verlässt
die Firma und wird aus der Datenbank gelöscht. Damit gehen auch die
Daten über die Abteilung A3 verloren.

\item[Insert-Anomalie] Es wird eine neue Abteilung A5
(Hardwareabteilung) geschaffen, der aber noch keine Mitarbeiter
zugeteilt sind. Damit müsste ein Tupel (NULL, NULL, NULL, A5,
Hardwareabteilung) in die Datenbank eingefügt werden. Da aber das
Attribut Name sicher in jedem Schlüsselkandidaten enthalten sein muss,
kann der Wert von Name keinen Nullwert enthalten. Das Tupel kann nicht
eingefügt werden.
\end{description}
\end{liAntwort}

%%
% (b)
%%

\item Bestimmen Sie eine Menge F der funktionalen Abhängigkeiten, die
sich aus Ihrer Analyse des Anwendungsbereiches ergeben. (Triviale
Abhängigkeiten brauchen nicht angegeben werden.) Begründen Sie Ihre
Entscheidung kurz.

\begin{liAntwort}
\begin{itemize}
\item $AbtNr \rightarrow Bezeichnung$ \\
Die Abteilungsnummer ist eindeutig (als „künstliches“
Unterscheidungsmerkmal für Abteilungen) und legt damit die Abteilung
eindeutig fest.

\item $Name, AbtNr \rightarrow Strasse, Ort$ \\
Da der Name innerhalb der Abteilung eindeutig ist, ist damit der
Mitarbeiter und folglich auch die Adressdaten eindeutig festgelegt. Da
es sich bei dieser Attributkombination um den Primärschlüssel handelt,
bestimmt diese Attributkombination auch das Attribut „Bezeichnung“,
allerdings darf es nicht in diese FA aufgenommen werden, da die
Abteilungsbezeichnung nicht von der Kombination aus Name \& AbtNr
abhängig, sondern nur von der AbtNr allein, somit muss dies als
einzelne FA formuliert werden und kann hier nicht aufgenommen werden

$\rightarrow$ der Rückschluss daraus wäre nämlich, dass sich die
Bezeichnung der Abteilung nur aus der Kombination von Mitarbeiter und
AbtNr erkennen lässt und nicht allein aus der AbtNr und das wäre ja
nicht korrekt. Grundsätzlich gilt: Primärschlüssel und FA’s müssen
getrennt betrachtet werden!

$F = \{
  Name\ AbtNr \rightarrow Strasse\ Ort,
  AbtNr \rightarrow Bezeichnung
\}$
\end{itemize}

\end{liAntwort}

%%
% (c)
%%

\item Bestimmen Sie z.B. mit Hilfe des Attributhüllen-Algorithmus die
Attributhülle

\renewcommand{\labelenumii}{\arabic{enumii}.}
\begin{enumerate}

%%
% 1.
%%

\item $\ah{F, \m{Name, Bezeichnung}}$

\begin{liAntwort}
$\ah{F, \m{Name, Bezeichnung}} = \m{Name, Bezeichnung}$
\end{liAntwort}

%%
% 2.
%%

\item $\ah{F, \m{Name, AbtNr}}$

\begin{liAntwort}
$\ah{F, \m{Name, AbtNr}} = \m{Name, AbtNr, Strasse, Ort, Bezeichnung}$
\end{liAntwort}
\end{enumerate}

%%
% (d)
%%

\item Ist ${Name, Bezeichnung}$ (bzw. ${Name, AbtNr}$ ein Superschlüssel
der Relation Abteilungsmitarbeiter? Kurze Begründung!

\begin{liAntwort}
${Name, AbtNr}$
\end{liAntwort}

\end{enumerate}

%-----------------------------------------------------------------------
%
%-----------------------------------------------------------------------

\section{Minimale Überdeckung\footcite[Seite 1, Aufgabe 2]{db:ab:5}}

Gegeben ist die Menge $F = \{A \rightarrow BC, C \rightarrow DA, E
\rightarrow AC, CD \rightarrow BE\}$. Bestimmen Sie eine minimale
Überdeckung von F.

\begin{multline*}
F = \{ \\
  A \rightarrow BC, \\
  C \rightarrow DA, \\
  E \rightarrow AC, \\
  CD \rightarrow BE \\
\}
\end{multline*}

\begin{liAntwort}

\begin{enumerate}
\item Linksreduktion

$\ah{F, \m{D}} = \m{D}$ \\
$\ah{F, \m{C}} = \m{C, D, A, B, E}$ \\

\begin{multline*}
F' = \{ \\
  A \rightarrow BC, \\
  C \rightarrow DA, \\
  E \rightarrow AC, \\
  C \rightarrow BE \\
\}
\end{multline*}

\item Rechtsreduktion

$\ah{F - \m{A \rightarrow BC}, \m{A}} = \m{A}$ \\
$\ah{F - \m{C \rightarrow DA}, \m{C}} = \m{C, B, E, A}$ \\
$\ah{F - \m{E \rightarrow AC}, \m{E}} = \m{E}$ \\
$\ah{F - \m{C \rightarrow BE}, \m{C}} = \m{C, D, A, B}$ \\
Keine Linksreduktion möglich

schon minimal

\end{enumerate}

\end{liAntwort}

%-----------------------------------------------------------------------
%
%-----------------------------------------------------------------------

\section{Schlüsselkandidaten von Relation \emph{Abstrakt}
\footcite[Seite 1, Aufgabe 3]{db:ab:5}
}

Gegeben sei die Relation \emph{Abstrakt} mit dem Schema $Abstrakt\{A, B, C, D,
E\}$ und die Menge der funktionalen Abhängigkeiten $F = \{A \rightarrow
BC, CD \rightarrow E, AC \rightarrow E, B \rightarrow D\}$.

\noindent
Bestimmen Sie die Schlüsselkandidaten von Abstrakt!

\begin{multline*}
F = \{ \\
  A \rightarrow BC, \\
  CD \rightarrow E, \\
  AC \rightarrow E, \\
  B \rightarrow D \\
\}
\end{multline*}

%-----------------------------------------------------------------------
%
%-----------------------------------------------------------------------

\literatur

\end{document}
