\documentclass{lehramt-informatik-haupt}
\liLadePakete{mathe,normalformen}
\usepackage{tabularx}
\usepackage{paralist}

\begin{document}
\let\ah=\liAttributHuelle
\let\m=\liAttributMenge
\let\FA=\liFunktionaleAbhaengigkeiten
\let\fa=\liFunktionaleAbhaengigkeit

%%%%%%%%%%%%%%%%%%%%%%%%%%%%%%%%%%%%%%%%%%%%%%%%%%%%%%%%%%%%%%%%%%%%%%%%
% Theorie-Teil
%%%%%%%%%%%%%%%%%%%%%%%%%%%%%%%%%%%%%%%%%%%%%%%%%%%%%%%%%%%%%%%%%%%%%%%%

\chapter{Normalformen}

\begin{liLernkartei}{Normalformen}
\begin{description}
\item[1NF] alle Attribute atomar

\item[2NF] 1NF $+$ kein Nichtprimärattribut hängt funktional von einer
echten Teilmenge eines Schlüsselkandidaten ab

\item[3NF] 2NF $+$ keine transitiven Abhängigkeiten über
Nichtschlüsselkandidaten

\item[BCNF] 3NF $+$ jede Determinante ein Schlüsselkandidat\footcite[Seite 179]{kemper}
\end{description}
\end{liLernkartei}

%-----------------------------------------------------------------------
%
%-----------------------------------------------------------------------

\section{Anomalien}

\begin{liLernkartei}{Anomalien}
\begin{itemize}
\item Update-Anomalie
\item Einfüge- oder Insert-Anomalie
\item Lösch- oder Delete-Anomalie
\end{itemize}
\end{liLernkartei}

%-----------------------------------------------------------------------
%
%-----------------------------------------------------------------------

\section{Attributhülle}

\begin{liLernkartei}{Bestimmung der Attributhülle}
\begin{flalign*}
\textit{Hülle}(F,\alpha )\\
    \alpha^{+}=\alpha\\
   \textbf{while} (\text{Änderung} an \alpha^{+}) do\\
      \textbf{foreach}(\textit{Abhängigkeit} \beta \rightarrow \gamma \in F) do\\
        \textbf{if} (\beta \subseteq \alpha^{+} ) \textbf{then} \alpha^{+}=\alpha^{+}\cup \gamma
\end{flalign*}
\end{liLernkartei}

Die Attributhülle $\alpha ^{+}$ eines bestimmten Attributs $\alpha$
(müsste eigentlich heißen: einer bestimmten Menge von Attributen) ist
eine Liste aller Attribute, die von $\alpha$ funktional
abhängen.\footcite[Attributhülle]{wiki:funktionale-abhängigkeit}

Durch die Bestimmung der Attributhülle kann angegeben werden, welche
anderen Attribute in einer Relation durch die \memph{gegebenen}
Attribute bestimmt werden können.\footcite[Seite
8]{db:fs:4}\footcite[Seite 179]{winter}

%-----------------------------------------------------------------------
%
%-----------------------------------------------------------------------

\subsection{Attributhüllenalgorithmus\footcite[Seite 179]{winter}}

Beispiel:\footcite[Seite 8]{db:fs:4}

\FA{P, Q -> R;
P, R -> Q;
Q -> S;
P, S -> T;
Q, R -> P;
T -> U;}

Gesucht

$\ah{F, \m{P,Q}}$

\noindent
\begin{tabularx}{\linewidth}{lX}
$\m{P, Q}$ &
Initialisierung von Erg mit den vorgegebenen Attributwerten P und Q
\\\hline

$\m{P, Q} \cup \m{R}$  &
Die linke Seite der FD $P Q \rightarrow R$ ist bereits in Erg enthalten.
Also werden die Attribute der rechten Seite zur Menge Erg hinzugefügt.
\\\hline

$\m{P, Q, R}$ &
Wegen der FD $PR \rightarrow Q$ käme Q hinzu, ist aber schon in Erg
enthalten. Also bleibt Erg unverändert.
\\\hline

$\m{P, Q, R} \cup \m{S}$ &
Die linke Seite der FD $Q \rightarrow S$ ist bereits in Erg enthalten.
Also werden die Attribute der rechten Seite zur Menge Erg hinzugefügt.
\\\hline

$\m{P, Q, R, S} \cup \m{T}$ &
Die linke Seite der FD $PS \rightarrow T$ ist bereits in Erg enthalten.
Also werden die Attribute der rechten Seite zur Menge Erg hinzugefügt.
\\\hline

$\m{P, Q, R, S, T}$ &
Wegen der FD $QR \rightarrow P$ käme P hinzu, ist aber schon in Erg
enthalten. Also bleibt Erg unverändert.
\\\hline

$\m{P, Q, R, S, T} \cup \m{U}$ &
Die linke Seite der FD $T \rightarrow U$ ist bereits in Erg enthalten.
Also werden die Attribute der rechten Seite zur Menge Erg hinzugefügt.
\\\hline

$\m{P, Q, R, S, T, U}$ &
Es ist also bereits die gesamte Attributmenge erreicht. Weitere
Betrachtungen erübrigen sich hiermit.
\\\hline
\end{tabularx}

\section{Die Normalformen}

%-----------------------------------------------------------------------
%
%-----------------------------------------------------------------------

\subsection{Erste Normalform\footcite[Seite 195]{winter}}

Ein Relationenschema R ist in \textbf{erster Normalform (1NF)}, wenn
alle Attribute \memph{atomar} sind.

%-----------------------------------------------------------------------
%
%-----------------------------------------------------------------------

\subsection{Zweite Normalform}

Eine Relation ist genau dann in der zweiten Normalform, wenn die erste
Normalform vorliegt und \memph{kein Nichtprimärattribut} (Attribut, das
nicht Teil eines Schlüsselkandidaten ist) funktional von einer
\memph{echten Teilmenge eines Schlüsselkandidaten abhängt}.

Anders gesagt: Jedes nicht-primäre Attribut (nicht Teil eines
Schlüssels) ist jeweils von \textbf{allen ganzen Schlüsseln} abhängig,
nicht nur von einem Teil eines Schlüssels. Wichtig ist hierbei, dass die
Nichtschlüsselattribute wirklich von allen Schlüsseln vollständig
abhängen.\footcite[Zweite Normalform (2NF)]{wiki:normalisierung}

%-----------------------------------------------------------------------
%
%-----------------------------------------------------------------------

\subsection{Dritte Normalform\footcite[Seite 201]{winter}}

aufgrund von transitiven Abhängigkeiten über Nichtschlüsselkandidaten
Falls A nicht prim ist, muss A funktional von jedem Superschlüssel
abhängen

Die dritte Normalform ist genau dann erreicht, wenn sich das
Relationenschema in der 2NF befindet, und kein Nichtschlüsselattribut
von einem Schlüsselkandidaten \memph{transitiv abhängt.}\footcite[Dritte
Normalform (3NF)]{wiki:normalisierung}

%-----------------------------------------------------------------------
%
%-----------------------------------------------------------------------

\subsection{Boyce-Codd-Normalform}

Sei $R$ eine Relationenschema in erster Normalform. Sei $F$ eine Menge
einfacher nichttrivialer funktionaler Abhängigkeiten für $R$. $R$ ist in
Boyce-Codd-Normalform (BCNF) bzgl. $F$, falls für jede FD $X \rightarrow
A$ aus $F$ gilt: $X$ ist ein Superschlüssel von $R$.
\footcite[Boyce-Codd-Normalform (BCNF)]{wiki:normalisierung}

Ein Relationenschema ist in der Boyce-Codd-Normalform, wenn es in der
3NF ist und jede Determinante (Attributmenge, von der andere Attribute
funktional abhängen) ein Schlüsselkandidat ist (oder die Abhängigkeit
ist trivial).

Die BCNF (nach Raymond F. Boyce und Edgar F. Codd) verhindert, dass
Teile zweier aus mehreren Feldern zusammengesetzten Schlüsselkandidaten
voneinander abhängig sind.

Die Überführung in die BCNF ist zwar immer verlustfrei möglich, aber
nicht immer abhängigkeitserhaltend. Die Boyce-Codd-Normalform war
ursprünglich als Vereinfachung der 3NF gedacht, führte aber zu einer
neuen Normalform, die diese verschärft: Eine Relation ist automatisch
frei von transitiven Abhängigkeiten, wenn alle Determinanten
Schlüsselkandidaten sind.

\subsubsection{Beispiel}

Gegeben ist das Relationenschema
$Aufnahmepruefung\m{PersNr, Schuelername, Fach, Note}$
und die Menge von FDs
$F = \{
  PersNr Schuelername \rightarrow Note,
  Schuelername Fach \rightarrow Note,
  PersNr \rightarrow Fach,
  Fach \rightarrow PersNr
\}$.

Die funktionalen Abhängigkeiten implizieren, dass es zwei
Schlüsselkandidaten gibt, nämlich
$\m{PersNr, Schuelername}$ und
$\m{Schuelername, Fach}$.

Das Relationenschema befindet sich in 3NF, da gilt:

\begin{itemize}
\item PersNr, Fach und Schülername sind prime Attribute.

\item Bei allen FDs, auf deren rechten Seite das (einzige) nichtprime
Attribut Note vorkommt, ist die Attributmenge auf der linken Seite ein
Superschlüssel.
\end{itemize}

Trotzdem kann es zu Redundanzen kommen, da die Information, in welchem
Fach eine Lehrkraft prüft, mehrfach abgespeichert wird.

\literatur

\end{document}
