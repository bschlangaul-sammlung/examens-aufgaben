\documentclass{lehramt-informatik-aufgabe}
\liLadePakete{mathe,normalformen}
\begin{document}
\liAufgabenTitel{Relation-MNVTPPN}

\section{Zusatzaufgabe 1 (wird nicht in der Übung
besprochen)
\index{Synthese-Algorithmus}
\footcite{db:pdf:tum:uebung-08}}

Betrachten Sie ein abstraktes Relationenschema $R = \{M, N, V, T, P,
PN\}$ mit den FDs

\begin{compactitem}
\item $M \rightarrow M$
\item $M \rightarrow N$
\item $V \rightarrow T, P, PN$
\item $P \rightarrow PN$
\end{compactitem}

\begin{enumerate}
\item Bestimmen Sie alle Kandidatenschlüssel.

\begin{antwort}
V kommt auf keiner rechten Seite der FDs vor.

$\text{AttrHuell}(R, \liK{V}) = \{V, T, P, PN\} \neq  R$

$\text{AttrHuell}(R, \liK{V, M}) = \{V, M, N, T, P, PN\} = R$

$\text{AttrHuell}(R, \liK{V, P}) = \{V, P, T, PN\} \neq R$

$V, M$ ist Schlüsselkandidat
\end{antwort}

\item In welcher Normalform befindet sich die Relation?

\begin{antwort}
1NF weil nichtprimäre Attribute von einer echten Teilmenge des
Schlüsselkandidaten abhängen (z. B. $M \rightarrow N$).
\end{antwort}

\item Bestimmen Sie zu den gebenen FDs die kanonische Überdeckung.

\begin{antwort}

\begin{enumerate}
\item Linkreduktion bleibt aus

\item Rechtsreduktion: PN ist doppelt

$\text{AttrHuell}(R - (V \rightarrow T, P, PN) \cup (V \rightarrow T, P), \{V\}) = \{V, T, P, PN\}$

\begin{compactitem}
\item $M \rightarrow M$
\item $M \rightarrow N$
\item $V \rightarrow T, P$
\item $P \rightarrow PN$
\end{compactitem}
\item Leere Klausel streichen

\item Vereinigung

\begin{compactitem}
\item $M \rightarrow N$
\item $V \rightarrow T, P$
\item $P \rightarrow PN$
\end{compactitem}

\end{enumerate}

\end{antwort}

\item Falls nötig, überführen Sie die Relation verlustfrei und
abhängigkeitsbewahrend in die dritte Normalform.
\end{enumerate}
\end{document}
