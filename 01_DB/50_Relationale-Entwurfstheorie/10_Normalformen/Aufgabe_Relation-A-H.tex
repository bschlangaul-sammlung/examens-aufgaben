\documentclass{lehramt-informatik-aufgabe}
\liLadePakete{normalformen}
\begin{document}
\liAufgabenTitel{Relation A-H}

\section{Synthesealgorithmus
\index{Synthese-Algorithmus}
\footcite[Seite 1, Aufgabe 2: Synthesealgorithmus]{db:pu:4}
}

Überführen Sie das Relationenschema mit Hilfe des Synthesealgorithmus in
die 3. Normalform!

$R(A, B, C, D, E, F, G, H)$

\begin{compactitem}
\item $F \rightarrow E$
\item $A \rightarrow B, D$
\item $AE \rightarrow D$
\item $A \rightarrow E, F$
\item $AG \rightarrow H$
\end{compactitem}

\begin{liAntwort}

\begin{enumerate}

\item Kanonische Überdeckung

\begin{enumerate}

%%
%
%%

\item 1. Linksreduktion:

Wir betrachten nur die zusammengesetzten Attribute:

\begin{itemize}
\item $AE \rightarrow D$:

$\liAH{F, \{A\}} = \{A, E, F, B, \textbf{D}\}$ \\
$\textit{AttrHüll}(F, \{E\}) = \{E\}$

\item $AG \rightarrow H$:

$\textit{AttrHüll}(F, \{A\}) = \{A, E, F, B, D\}$ \\
$\textit{AttrHüll}(F, \{G\}) = \{G\}$
\end{itemize}

\textbf{FDs}

\begin{compactitem}
\item $F \rightarrow E$
\item $A \rightarrow B, D$
\item $A \rightarrow D$
\item $A \rightarrow E, F$
\item $AG \rightarrow H$
\end{compactitem}

%%
%
%%

\item 2. Rechtsreduktion:

Nur die Attribute betrachten, die rechts doppelt vorkommen:

$E$:

$\textit{AttrHüll}(F - \{F \rightarrow E\}, \{F\}) = \{F\}$ \\
$\textit{AttrHüll}(F - \{A \rightarrow E\}, \{A\}) = \{A, B, D, F, \textbf{E}\}$

$D$:

$\textit{AttrHüll}(F - \{A \rightarrow D\}, \{A\}) = \{A, B, \textbf{D}, F, E\}$

$A \rightarrow D$ kann wegen der Armstrongschen Dekompositionsregel
weggelassen werden. Wenn gilt $A \rightarrow B, D$, dann gilt auch $A
\rightarrow B$ und $A \rightarrow D$

\textbf{FDs}

\begin{compactitem}
\item $F \rightarrow E$
\item $A \rightarrow B, D$
\item $A \rightarrow \emptyset$
\item $A \rightarrow F$
\item $AG \rightarrow H$
\end{compactitem}

\item 3. Leere Klauseln streichen:

\begin{compactitem}
\item $F \rightarrow E$
\item $A \rightarrow B, D$
\item $A \rightarrow F$
\item $AG \rightarrow H$
\end{compactitem}

\item 4. Vereinigung

\begin{compactitem}
\item $F \rightarrow E$
\item $A \rightarrow B, D, F$
\item $AG \rightarrow H$
\end{compactitem}

Jetzt die weiteren Hauptschritte:

\end{enumerate}

\item Neues Relationenschema

\begin{compactitem}
\item $R1(F, E)$
\item $R2(A, B, D, F)$
\item $R3(A, G, H)$
\end{compactitem}

\item Hinzufügen einer Relation

Schlüsselkandidaten hinzufügen, falls nicht vorhanden:
$R4(A, C, G)$

\begin{compactitem}
\item $R1(F, E)$
\item $R2(A, B, D, F)$
\item $R3(A, G, H)$
\item $R4(A, C, G)$
\end{compactitem}

\item Entfernen überflüssiger Teilschemata

nichts zu tun

\end{enumerate}
\end{liAntwort}

\end{document}
