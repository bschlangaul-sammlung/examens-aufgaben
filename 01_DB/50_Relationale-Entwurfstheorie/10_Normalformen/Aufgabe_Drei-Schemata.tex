\documentclass{lehramt-informatik-aufgabe}
\liLadePakete{mathe,normalformen}
\begin{document}
\liAufgabenTitel{Drei-Schemata}

\section{Normalformen Einstieg
\footcite[Seite 1, Aufgabe 1: Normalformen Einstieg]{db:pu:4}
}

Es seien folgende Relationenschemata mit den jeweiligen Mengen
funktionaler Abhängigkeiten gegeben:

$S_1 \{P, Q, R\}$ mit
$F_1 = \{P Q \rightarrow R, P R \rightarrow Q, QR \rightarrow P \}$

$S_2 \{P, R, S, T \}$ mit
$F_2 = \{P S \rightarrow T \}$

$S_3 \{P, S, U \}$ mit
$F_3 = \{\}$

\begin{enumerate}

%%
% (a)
%%

\item Welche der drei Schemata sind in
BCNF\index{Boyce-Codd-Normalform}, welche in 3NF\index{Dritte
Normalform}, welche in 2NF\index{Zweite Normalform}? Begründe!

\begin{antwort}
\begin{itemize}
\item $S_1$: BCNF

\item $S_2$: 1NF aber nicht 2NF

\item $S_3$: BCNF
\end{itemize}
\end{antwort}

%%
% (b)
%%

\item Wenden Sie auf ($S_2$, $F_2$) den
Synthesealgorithmus\index{Synthese-Algorithmus} an, und bestimmen Sie
auch die Mengen aller nichttrivialen einfachen funktionalen
Abhängigkeiten, die über den erhaltenen Teilrelationen gelten. Ihr
Lösungsweg muss nachvollziehbar sein.

\begin{antwort}
PS -> T (ist schon kanonische Überdeckung)

2. Schritt

R1 (P, S, T)

3. Schritt

R1 (P, S, T) mit F21 = {PS->}
R2 (P, S, R) mit F22 = {}

\end{antwort}

\end{enumerate}
\end{document}
