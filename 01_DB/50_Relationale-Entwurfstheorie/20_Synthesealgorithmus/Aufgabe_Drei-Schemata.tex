\documentclass{lehramt-informatik-aufgabe}
\liLadePakete{normalformen,synthese-algorithmus}
\begin{document}
\let\FA=\liFunktionaleAbhaengigkeiten
\let\schrittE=\liSyntheseSchrittUeberschriftErklaerung

\liAufgabenTitel{Drei-Schemata}

\section{Normalformen Einstieg
\footcite[Seite 1, Aufgabe 1: Normalformen Einstieg]{db:pu:4}
}

Es seien folgende Relationenschemata mit den jeweiligen Mengen
funktionaler Abhängigkeiten gegeben:

\begin{description}
\item \liRelation[$S_1$]{P, Q, R} mit

\FA[$F_1$]{P, Q -> R; P, R -> Q; Q, R -> P}

\item \liRelation[$S_2$]{P, R, S, T} mit

\FA[$F_2$]{P, S -> T}

\item \liRelation[$S_3$]{P, S, U}  mit

\FA[$F_3$]{}
\end{description}

\begin{enumerate}

%%
% (a)
%%

\item Welche der drei Schemata sind in
BCNF\index{Boyce-Codd-Normalform}, welche in 3NF\index{Dritte
Normalform}, welche in 2NF\index{Zweite Normalform}? Begründe!

\begin{liAntwort}
\begin{description}
\item[$S_1$] BCNF

\item[$S_2$] 1NF aber nicht 2NF

\item[$S_3$] BCNF
\end{description}

($S_1$, $F_1$) und ($S_3$, $F_3$) sind offenbar in BCNF und
daher auch in 3NF und 2NF. ($S_2$ , $F_2$) ist offenbar nicht in
2NF, da der Schlüsselkandidat PRS ist und T von einem Teil dieser
Schlüsselkandidaten, nämlich PS, abhängig ist und daher auch nicht in
3NF oder BCNF.
\end{liAntwort}

%%
% (b)
%%

\item Wenden Sie auf ($S_2$, $F_2$) den
Synthesealgorithmus\index{Synthese-Algorithmus} an, und bestimmen Sie
auch die Mengen aller nichttrivialen einfachen funktionalen
Abhängigkeiten, die über den erhaltenen Teilrelationen gelten. Ihr
Lösungsweg muss nachvollziehbar sein.

\begin{liAntwort}
Achtung! Die Aufteilung in die einzelnen Schritte stimmt nicht.

\begin{enumerate}
\item \schrittE{1}

PS -> T (ist schon kanonische Überdeckung)

\item \schrittE{2}

Enthält keines der so entstandenen Relationenschemata einen
Schlüsselkandidaten von R, so wird ein weiteres Relationenschema
erzeugt, das als Attribute die Attribute eines (beliebigen)
Schlüsselkandidaten von R hat.

R1 (P, S, T)

\item \schrittE{3}

Attribute von R, die in keinem der in Schritt 1 und 2 erzeugten
Relationenschemata auftauchen, werden in einem zusätzlichen
Relationenschema zusammengefasst

\liRelation[$S_{21}$]{P, S, T} mit \FA[$F_{21}$]{P S-> T}

\bigskip

\liRelation[$S_{22}$]{P, S, R} mit \FA[$F_{22}$]{}

\item \schrittE{4}

Von den in Schritt 1 bis 3 entstandenen Schemata werden die Schemata
wieder verworfen, deren Attributmenge eine Teilmenge eines anderen in
Schritt 1 bis 3 entstandenen Schemas ist und solche mit gleichem
Schlüsselkandidaten zusammengefasst.

\end{enumerate}
\end{liAntwort}

\end{enumerate}
\end{document}
