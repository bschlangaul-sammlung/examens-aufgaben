\documentclass{lehramt-informatik-aufgabe}
\liLadePakete{normalformen}
\begin{document}
\let\fa=\liFunktionaleAbhaengigkeit
\let\FA=\liFunktionaleAbhaengigkeiten
\let\ah=\liAttributHuelle
\let\m=\liMenge

\liAufgabenTitel{Relation A-H}

\section{Synthesealgorithmus
\index{Synthese-Algorithmus}
\footcite[Seite 1, Aufgabe 2: Synthesealgorithmus]{db:pu:4}
}

Überführen Sie das Relationenschema mit Hilfe des Synthesealgorithmus in
die 3. Normalform!

$R(A, B, C, D, E, F, G, H)$

\FA{
F -> E;
A -> B, D;
AE -> D;
A -> E, F;
AG -> H;
}

\begin{liAntwort}

\begin{enumerate}

\item Kanonische Überdeckung

\begin{enumerate}

%%
%
%%

\item 1. Linksreduktion:

Wir betrachten nur die zusammengesetzten Attribute:

\begin{itemize}
\item \fa{AE -> D}:

$\ah{F, \m{A}} = \m{A, E, F, B, \textbf{D}}$ \\
$\ah{(F, \m{E}} = \m{E}$

\item \fa{AG -> H}:

$\ah{F, \m{A}} = \m{A, E, F, B, D}$ \\
$\ah{F, \m{G}} = \m{G}$
\end{itemize}

\textbf{FDs}

\FA{F -> E;
A -> B, D;
A -> D;
A -> E, F;
AG -> H;
}

%%
%
%%

\item 2. Rechtsreduktion:

Nur die Attribute betrachten, die rechts doppelt vorkommen:

$E$:

$\ah{F - \m{F \rightarrow E}, \m{F}} = \m{F}$ \\
$\ah{F - \m{A \rightarrow E}, \m{A}} = \m{A, B, D, F, \textbf{E}}$

$D$:

$\ah{F - \m{A \rightarrow D}, \m{A}} = \m{A, B, \textbf{D}, F, E}$

$A \rightarrow D$ kann wegen der Armstrongschen Dekompositionsregel
weggelassen werden. Wenn gilt $A \rightarrow B, D$, dann gilt auch $A
\rightarrow B$ und $A \rightarrow D$

\textbf{FDs}

\FA{
F -> E;
A -> B, D;
A -> nichts;
A -> F;
AG -> H;
}

\item 3. Leere Klauseln streichen:

\FA{
F -> E;
A -> B, D;
A -> F;
AG -> H;
}

\item 4. Vereinigung

\FA{
F -> E;
A -> B, D, F;
AG -> H;
}

Jetzt die weiteren Hauptschritte:

\end{enumerate}

\item Neues Relationenschema

\begin{compactitem}
\item $R1(F, E)$
\item $R2(A, B, D, F)$
\item $R3(A, G, H)$
\end{compactitem}

\item Hinzufügen einer Relation

Schlüsselkandidaten hinzufügen, falls nicht vorhanden:
$R4(A, C, G)$

\begin{compactitem}
\item $R1(F, E)$
\item $R2(A, B, D, F)$
\item $R3(A, G, H)$
\item $R4(A, C, G)$
\end{compactitem}

\item Entfernen überflüssiger Teilschemata

nichts zu tun

\end{enumerate}
\end{liAntwort}

\end{document}
