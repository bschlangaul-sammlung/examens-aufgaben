\documentclass{lehramt-informatik-aufgabe}
\liLadePakete{mathe,normalformen,synthese-algorithmus}
\begin{document}
\let\a=\liAttributHuelleOhneMathe
\let\m=\liAttributMenge
\let\fa=\liFunktionaleAbhaengigkeit
\let\FA=\liFunktionaleAbhaengigkeiten
\let\schrittE=\liSyntheseSchrittUeberschriftErklaerung

\liAufgabenTitel{Relation-MNVTPPN}

\section{Zusatzaufgabe 1 (wird nicht in der Übung
besprochen)
\index{Synthese-Algorithmus}
\footcite{db:pdf:tum:uebung-08}}

Betrachten Sie ein abstraktes Relationenschema $R = \{M, N, V, T, P,
PN\}$ mit den FDs

\FA{
  M -> M;
  M -> N;
  V -> T, P, PN;
  P -> PN;
}

\begin{enumerate}
\item Bestimmen Sie alle Kandidatenschlüssel.

\begin{liAntwort}
V kommt auf keiner rechten Seite der FDs vor.

$\a{R, \m{V}} = \m{V, T, P, PN} \neq  R$

$\a{R, \m{V, M}} = \m{V, M, N, T, P, PN} = R$

$\a{R, \m{V, P}} = \m{V, P, T, PN} \neq R$

$V, M$ ist Schlüsselkandidat
\end{liAntwort}

\item In welcher Normalform befindet sich die Relation?

\begin{liAntwort}
1NF weil nichtprimäre Attribute von einer echten Teilmenge des
Schlüsselkandidaten abhängen (z. B. \fa{M -> N}).
\end{liAntwort}

\item Bestimmen Sie zu den gebenen FDs die kanonische Überdeckung.

\begin{liAntwort}

\begin{enumerate}
\item \schrittE{1-1}

Linkreduktion bleibt aus

\item \schrittE{1-2}

PN ist doppelt

$\a{R - (V \rightarrow T, P, PN) \cup (V \rightarrow T, P), \m{V}} = \m{V, T, P, PN}$

\FA{
  M -> M;
  M -> N;
  V -> T, P;
  P -> PN;
}

\item \schrittE{1-3}

\item \schrittE{1-4}

\FA{
  M -> N;
  V -> T, P;
  P -> PN;
}

\end{enumerate}
\end{liAntwort}

\item Falls nötig, überführen Sie die Relation verlustfrei und
abhängigkeitsbewahrend in die dritte Normalform.
\end{enumerate}
\end{document}
