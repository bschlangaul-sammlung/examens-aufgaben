\documentclass{lehramt-informatik-aufgabe}
\liLadePakete{syntax}
\begin{document}
\liAufgabenTitel{Kaufhaus zum letzten Mal!}

\section{Aufgabe 3: Kaufhaus zum letzten Mal!
\index{SQL}
\footcite{db:ab:7}}

Gegeben ist wieder die Kaufhaus-Datenbank. Wie lauten die SQL-Befehle
für folgende Sichten?

\begin{enumerate}

%%
% (a)
%%

\item Sicht \verb|view1|: Gesucht sind alle Informationen zu Artikeln,
an denen das Kaufhaus mehr als 35\% verdient.

\begin{antwort}[falsch]
Da stimmt irgendwas nicht mit er Mathematik
\begin{minted}{sql}
CREATE VIEW view1 AS
SELECT *
FROM Artikel
WHERE 1 - Einkaufspreis / Verkaufspreis >= 0.35;
\end{minted}
\end{antwort}

\begin{antwort}[muster]
\begin{minted}{sql}
CREATE VIEW view1 AS
SELECT *
FROM Artikel
WHERE Verkaufspreis > 1.35 * Einkaufspreis;
\end{minted}
\end{antwort}

%%
% (b)
%%

\item Sicht \verb|view2|: Gesucht sind alle Informationen zu Artikeln,
an denen das Kaufhaus mehr als 35\% verdient und die für höchstens 50 €
verkauft werden.

\begin{antwort}[richtig]
Wahrscheinlich soll mit der View weitergemacht werden.
\begin{minted}{sql}
CREATE VIEW view2 AS
SELECT *
FROM Artikel
WHERE Verkaufspreis > 1.35 * Einkaufspreis AND Verkaufspreis <= 50;
\end{minted}
\end{antwort}

\begin{antwort}[muster]
\begin{minted}{sql}
CREATE VIEW view2 AS
SELECT *
FROM view1
WHERE Verkaufspreis <= 50;
\end{minted}
\end{antwort}
\end{enumerate}

\end{document}
