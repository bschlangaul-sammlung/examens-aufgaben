\documentclass{lehramt-informatik-aufgabe}
\liLadePakete{syntax}
\begin{document}
\liAufgabenTitel{noch mal Kaufhaus...}

\section{Aufgabe 2: noch mal Kaufhaus...
\index{SQL}
\footcite{db:ab:7}}

Gegeben ist wiederum die Kaufhaus-Datenbank (Tabellen siehe oben).
Formulieren Sie nachfolgende Anfragen in SQL!

\begin{enumerate}

%%
% (a)
%%

\item Welche Artikel (Artikelnummer, Abteilungsname) werden in den
Abteilungen angeboten? Die Ausgabe soll absteigend nach der
Artikelnummer sortiert werden. Bei gleicher Artikelnummer sollen die
betroffenen Abteilungen alphabetisch aufgelistet werden.

\begin{antwort}[richtig]
Viel zu kompliziert. Geht ohne Joins
\begin{minted}{sql}
SELECT Artikel.ArtNr AS Artikelnummer, Abteilungsname
FROM Artikel, Bestand WHERE
  Bestand.ArtNr = Artikel.ArtNr
ORDER BY Artikel.ArtNr DESC, Abteilungsname ASC;
\end{minted}
\end{antwort}

\begin{antwort}[muster]
\begin{minted}{sql}
SELECT ArtNr, Abteilungsname
FROM Bestand
ORDER BY ArtNr DESC, Abteilungsname;
\end{minted}
\end{antwort}
%%
% (b)
%%

\item Wie viele verschiedene Waren werden in der Lebensmittelabteilung
verkauft?

\begin{antwort}[muster]
\begin{minted}{sql}
SELECT COUNT(*)
FROM Bestand
WHERE Abteilungsname = 'Lebensmittel';
\end{minted}
\end{antwort}

%%
% (c)
%%

\item Wie viele verschiedene Waren werden in den einzelnen Abteilungen
verkauft?

\begin{antwort}[muster]
\begin{minted}{sql}
SELECT Abteilungsname, COUNT(*)
FROM Bestand
GROUP BY Abteilungsname;
\end{minted}
\end{antwort}

%%
% (d)
%%

\item Wie viel kostet der billigste, wie viel der teuerste Artikel?

\begin{antwort}[muster]
\begin{minted}{sql}
SELECT MIN(Verkaufspreis), MAX(Verkaufspreis)
FROM Artikel;
\end{minted}
\end{antwort}

%%
% (e)
%%

\item Gib die Namen aller Abteilungen aus, deren Gesamtvorrat an Artikel
kleiner als 100 ist!

\begin{antwort}[muster]
\begin{minted}{sql}
SELECT Abteilungsname
FROM Bestand
GROUP BY Abteilungsname
HAVING COUNT(Vorrat) < 100;
\end{minted}
\end{antwort}

%%
% (f)
%%

\item Gesucht sind Bezeichnung und Verkaufspreis aller in der Datenbank
gespeicherten Artikel. Die Ausgabe soll alphabetisch aufgelistet werden.
Bei gleicher Bezeichnung sollen die teureren Artikel zuerst aufgelistet
werden.

\begin{antwort}[muster]
\begin{minted}{sql}
SELECT Bezeichnung, Verkaufspreis
FROM Artikel
ORDER BY Bezeichnung, Verkaufspreis DESC;
\end{minted}
\end{antwort}

%%
% (g)
%%

\item Gib für alle Artikel, von denen (unabhängig von der Abteilung)
noch mindestens 130 Exemplare vorrätig sind, die Artikelnummer und den
aktuellen Vorrat aus!

\begin{antwort}[falsch]
Derselbe Artikel kann in mehreren Abteilungen angeboten werden.
\begin{minted}{sql}
SELECT ArtNr, Vorrat
FROM Bestand
WHERE Vorrat >= 130;
\end{minted}
\end{antwort}

\begin{antwort}[muster]
\begin{minted}{sql}
SELECT ArtNr, SUM(Vorrat)
FROM Bestand
GROUP BY ArtNr
HAVING SUM(Vorrat) >= 130;
\end{minted}
\end{antwort}
\end{enumerate}
\end{document}
