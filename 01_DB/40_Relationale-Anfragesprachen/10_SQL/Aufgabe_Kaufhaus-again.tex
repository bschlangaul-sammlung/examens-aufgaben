\documentclass{lehramt-informatik-aufgabe}
\liLadePakete{syntax}
\begin{document}
\liAufgabenTitel{Kaufhaus again}

\section{Aufgabe 1: Kaufhaus again...
\index{SQL}
\footcite{db:ab:7}}

Gegeben ist wiederum die Kaufhausdatenbank

\liPseudoUeberschrift{Artikel}

\begin{tabular}{llll}
\textbf{ArtNr} & \textbf{Bezeichnung}  & \textbf{Verkaufspreis} & \textbf{Einkaufspreis} \\
95    & Kamm         & 1.25          & 0.80          \\
97    & Kamm         & 0.99          & 0.75          \\
507   & Seife        & 3.93          & 2.45          \\
1056  & Zwieback     & 1.20          & 0.90          \\
1401  & Räucherlachs & 4.90          & 3.60          \\
2045  & Herrenhose   & 37.25         & 24.45         \\
2046  & Herrenhose   & 20.00         & 17.00         \\
2340  & Sommerkleid  & 94.60         & 71.50
\end{tabular}

\liPseudoUeberschrift{Abteilung}

\begin{tabular}{lll}
\textbf{Abteilungsname} & \textbf{Stockwerk} & \textbf{Abteilungsleiter} \\
Lebensmittel   & I         & Josef Kunz       \\
Lebensmittel   & EG        & Monika Stiehl    \\
Textilien      & II        & Monika Stiehl
\end{tabular}

\liPseudoUeberschrift{Bestand}

\begin{tabular}{lll}
\textbf{Abteilungsname}  & \textbf{ArtNr} & \textbf{Vorrat} \\
Lebensmittel    & 1056  & 129    \\
Lebensmittel    & 1401  & 200    \\
Textilien       & 2045  & 14
\end{tabular}

\begin{enumerate}

%%
% (a)
%%

\item Formulieren Sie nachfolgende Anfragen in SQL mit Hilfe von Joins!

\begin{itemize}

%%
% Punkt 1
%%

\item Wie viele Packungen Zwieback sind noch vorrätig?

\begin{antwort}[richtig]
\begin{minted}{sql}
SELECT b.Vorrat
FROM Bestand b JOIN Artikel a ON b.ArtNr = a.ArtNr
WHERE a.Bezeichnung = 'Zwieback'
\end{minted}
\end{antwort}

\begin{antwort}[muster]
Hinweis: In obigem Lösungsansatz wird berücksichtigt, dass ein Artikel,
hier der Zwieback, in mehreren Abteilungen verkauft werden kann. Geht
man davon aus, dass Zwieback nur in einer Abteilung verkauft wird, kann
man die Aggregatfunktion SUM weglassen.
\begin{minted}{sql}
SELECT SUM(b.Vorrat)
FROM Bestand b, Artikel a
WHERE b.ArtNr = a.ArtNr AND a.Bezeichnung = 'Zwieback';
\end{minted}
\end{antwort}

%%
% Punkt 2
%%

\item In welchem Stockwerk wird Räucherlachs verkauft?

\begin{antwort}[muster]
\begin{minted}{sql}
SELECT Abteilung.Stockwerk
FROM Artikel, Abteilung, Bestand
WHERE Artikel.ArtNr = Bestand.ArtNr AND
Bestand.Abteilungsname = Abteilung.Abteilungsname AND
Artikel.Bezeichnung = 'Räucherlachs';
\end{minted}
\end{antwort}

\end{itemize}

%%
% (b)
%%

\item Formulieren Sie folgende Anfragen an die Kaufhaus-Datenbank unter
Verwendung von geschachtelten SELECT-Anweisungen!

\begin{itemize}

%%
% Punkt 1
%%

\item Gib die Bezeichnungen und die Artikelnummern aller Artikel aus,
die nicht mehr als der Artikel mit der Artikelnummer 1401 kosten!

\begin{antwort}[muster]
Hinweis: Durch Hinzufügen der Bedingung NOT(ArtNr=1401) wird der Artikel
mit der Nummer 1401 in der Ergebnistabelle nicht aufgeführt
\begin{minted}{sql}
SELECT Bezeichnung, ArtNr AS Artikelnummer
FROM Artikel
WHERE Verkaufspreis <= (
  SELECT Verkaufspreis FROM Artikel WHERE ArtNr = 1401
);
\end{minted}
\end{antwort}

%%
% Punkt 2
%%

\item Gesucht sind Bezeichnung und Verkaufspreis aller Artikel, die in
der Textilienabteilung verkauft werden!

\begin{antwort}[muster]
\begin{minted}{sql}
SELECT Bezeichnung, Verkaufspreis
FROM Artikel WHERE ArtNr in (
  SELECT ArtNr FROM Bestand WHERE Abteilungsname = 'Textilien'
);
\end{minted}
\end{antwort}

%%
% Punkt 3
%%

\item Welche Produkte (Angabe der Bezeichnung) werden im Erdgeschoss
verkauft?

\begin{antwort}[falsch]
Fragestellung: Es sollen geschachtelte SQL-Anfragen verwendet werden.
\verb|DISTINCT| vergessen
\begin{minted}{sql}
SELECT Bezeichnung
FROM Artikel
WHERE ArtNr in (
  SELECT ArtNr
  FROM Abteilung, Bestand
  WHERE Abteilung.Abteilungsname = Bestand.Abteilungsname AND
    Abteilung.Stockwerk = 'EG'
);
\end{minted}
\end{antwort}

\begin{antwort}[muster]
\begin{minted}{sql}
SELECT DISTINCT Bezeichnung
FROM Artikel
WHERE ArtNr in (
  SELECT ArtNr
  FROM Bestand
  WHERE Abteilungsname IN (
    SELECT Abteilungsname
    FROM Abteilung
    WHERE Stockwerk = 'EG'
  )
);
\end{minted}
\end{antwort}

%%
% Punkt 4
%%

\item Gib die Namen aller Abteilungsleiter aus, in deren Abteilungen von
jedem Artikel weniger als 100 Exemplare vorrätig sind!

\begin{antwort}[richtig]
\begin{minted}{sql}
SELECT DISTINCT Abteilungsleiter
FROM Abteilung
WHERE Abteilungsname IN (
  SELECT Abteilungsname
  FROM Bestand
  GROUP BY Abteilungsname
  HAVING MAX(Vorrat) < 100
);
\end{minted}
\end{antwort}

\begin{antwort}[richtig]
\begin{minted}{sql}
SELECT DISTINCT Abteilungsleiter
FROM Abteilung
WHERE Abteilungsname NOT IN (
  SELECT Abteilungsname
  FROM Bestand
  WHERE Vorrat >= 100
);
\end{minted}
\end{antwort}

oder:

\begin{antwort}[muster]
\begin{minted}{sql}
SELECT DISTINCT Abteilungsleiter
FROM Abteilung
WHERE NOT EXISTS (
  SELECT *
  FROM Bestand
  WHERE (Abteilung.Abteilungsname =
  Bestand.Abteilungsname) AND Vorrat >= 100
);
\end{minted}
\end{antwort}
\end{itemize}

%%
% (c)
%%

\item Lösen Sie die Aufgabe 1b) Punkt 1 ohne Verwendung einer
geschachtelten SQL Anfrage! (Gib die Bezeichnungen und die
Artikelnummern aller Artikel aus, die nicht mehr als der Artikel mit der
Artikelnummer 1401 kosten!)
\begin{antwort}[falsch]
Irgendwie habe ich über Bezeichnung gejoint, was nicht nötig ist.
\begin{minted}{sql}
SELECT a1.Bezeichnung, a1.ArtNr as Artikelnummer
FROM Artikel a1, Artikel a2, Bestand b
WHERE
  b.ArtNr = a1.ArtNr AND
  a2.ArtNr = a1.ArtNr AND
  a2.ArtNr = 1401 AND
  a1.Verkaufspreis <= a2.Verkaufspreis;
\end{minted}
\end{antwort}

\begin{antwort}[muster]
Irgendwie habe ich über Bezeichnung gejoint, was nicht nötig ist.
\begin{minted}{sql}
SELECT a.Bezeichnung, a.ArtNr as Artikelnummer
FROM Artikel a, Artikel b
WHERE
  a.Verkaufspreis <= b.Verkaufspreis AND
  b.ArtNr = 1401;
\end{minted}
\end{antwort}

%%
% (d)
%%

\item Formulieren Sie nachfolgende Anfragen mit Mengenoperatoren!

\begin{itemize}

%%
% Punkt 1
%%

\item Gibt es registrierte Artikel, die noch nicht im Bestand aufgeführt
sind?

\begin{antwort}[muster]
\begin{minted}{sql}
SELECT ArtNr FROM Artikel
EXCEPT
SELECT ArtNr FROM Bestand;
\end{minted}
\end{antwort}

%%
% Punkt 2
%%

\item Welche Artikel (Artikelnummer) sind registriert und bereits im
Bestand aufgeführt?

\begin{antwort}[muster]
\begin{minted}{sql}
SELECT ArtNr FROM Artikel
INTERSECT
SELECT ArtNr FROM Bestand;
\end{minted}
\end{antwort}

%%
% Punkt 3
%%

\item Welche Artikel (Bezeichnung und Artikelnummer) sind bereits
registriert und im Bestand aufgeführt?

\begin{antwort}[muster]
\begin{minted}{sql}
SELECT Bezeichnung, ArtNr FROM Artikel WHERE ArtNr IN (
  SELECT ArtNr FROM Artikel
  INTERSECT
  SELECT ArtNr FROM Bestand
);
\end{minted}
\end{antwort}
\end{itemize}

%%
% (d)
%%

\item Formulieren Sie folgende Anfragen in SQL:

\begin{itemize}

%%
% Punkt 1
%%

\item Welche Artikel mit dem Anfangsbuchstaben “S” gibt es?

\begin{antwort}[muster]
\begin{minted}{sql}
SELECT Bezeichnung FROM Artikel WHERE Bezeichnung LIKE 'S%';
\end{minted}
\end{antwort}

%%
% Punkt 2
%%

\item Welche Artikel haben an der 3. Stelle ein “i”?

\begin{antwort}[muster]
\begin{minted}{sql}
SELECT Bezeichnung FROM Artikel WHERE Bezeichnung LIKE '__i%';
\end{minted}
\end{antwort}

%%
% Punkt 3
%%

\item Heißt der Artikel “Zwieback” oder “Zweiback”?

\begin{antwort}[muster]
\begin{minted}{sql}
SELECT Bezeichnung FROM Artikel WHERE Bezeichnung LIKE 'Zw__back';
\end{minted}
\end{antwort}
\end{itemize}
\end{enumerate}
\end{document}
