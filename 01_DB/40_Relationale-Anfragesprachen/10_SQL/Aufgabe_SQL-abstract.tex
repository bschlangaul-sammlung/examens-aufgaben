\documentclass{lehramt-informatik-aufgabe}
\liLadePakete{syntax}
\begin{document}
\liAufgabenTitel{SQL abstrakt}

\section{Aufgabe 3: SQL abstrakt...
\index{SQL}
\footcite{db:ab:3}
}

Gegeben sind die drei Tabellen TAB1, TAB2 und TAB3:

\liPseudoUeberschrift{TAB1}

\begin{tabular}{lll}
A&B&C\\
1&2&3\\
4&5&6\\
7&8&9\\
\end{tabular}

\liPseudoUeberschrift{TAB2}

\begin{tabular}{ll}
A&B\\
1&4\\
3&7\\
\end{tabular}

\liPseudoUeberschrift{TAB3}

\begin{tabular}{ll}
A&B\\
1&2\\
9&6\\
3&3\\
9&2\\
2&7\\
7&4\\
9&4\\
3&7\\
\end{tabular}

\bigskip

Geben Sie die Ergebnistabellen der folgenden Aussagen an:

\begin{enumerate}

%%
% (a)
%%

\item

\begin{minted}{sql}
SELECT A FROM TAB3 WHERE B = 4 OR B = 7 ORDER BY A;
\end{minted}

\begin{antwort}[muster]
\begin{tabular}{l}
A\\
2\\
7\\
9\\
3\\
\end{tabular}
\end{antwort}

%%
% (b)
%%

\item

\begin{minted}{sql}
SELECT * FROM TAB3 WHERE NOT (B = 7) ORDER BY A ASC, B
DESC;
\end{minted}

\begin{antwort}[muster]
\begin{tabular}{ll}
A&B\\
1&2\\
3&3\\
7&4\\
9&6\\
9&4\\
9&2\\
\end{tabular}
\end{antwort}

%%
% (c)
%%

\item

\begin{minted}{sql}
SELECT COUNT(DISTINCT A) FROM TAB3 WHERE B >= 3;
\end{minted}

\begin{antwort}[muster]
4
\end{antwort}

%%
% (d)
%%

\item

\begin{minted}{sql}
SELECT A, COUNT(*), SUM (B), MAX(A), AVG(B) FROM TAB3
GROUP BY A;
\end{minted}

\begin{antwort}
\begin{tabular}{lllll}
A&Count&SumB&MaxA&AvgB\\
1&1&2&1&2\\
2&1&7&2&7\\
3&2&10&3&5\\
7&1&4&7&4\\
9&3&12&9&4\\
\end{tabular}
\end{antwort}

%%
% (e)
%%

\item

\begin{minted}{sql}
SELECT TAB1.*, TAB2.* FROM TAB1, TAB2;
\end{minted}

\begin{antwort}
\begin{tabular}{lllll}
TAB1.A&TAB1.B&TAB1.C&TAB2.A&TAB2.B\\
1&2&3&1&4\\
4&5&6&1&4\\
7&8&9&1&4\\
1&2&3&3&7\\
4&5&6&3&7\\
7&8&9&3&7\\
\end{tabular}
\end{antwort}

%%
% (f)
%%

\item

\begin{minted}{sql}
SELECT TAB1.*, TAB2.* FROM TAB1, TAB2 WHERE TAB1.A =
TAB2.B;
\end{minted}

\begin{antwort}
\begin{tabular}{lllll}
TAB1.A&TAB1.B&TAB1.C&TAB2.A&TAB2.B\\
4&5&6&1&4\\
7&8&9&3&7\\
\end{tabular}
\end{antwort}

%%
% (g)
%%

\item

\begin{minted}{sql}
SELECT TAB1.*, TAB2.* FROM TAB1, TAB2 WHERE TAB1.A =
TAB2.B AND TAB2.A = 3;
\end{minted}

\begin{antwort}
\begin{tabular}{lllll}
TAB1.A&TAB1.B&TAB1.C&TAB2.A&TAB2.B\\
7&8&9&3&7\\
\end{tabular}
\end{antwort}

%%
% (h)
%%

\item

\begin{minted}{sql}
SELECT TAB1.A, TAB1.C, TAB2.A FROM TAB1, TAB2 WHERE
TAB1.A = TAB2.B AND TAB2.A = 3;
\end{minted}

\begin{antwort}
\begin{tabular}{lll}
TAB1.A&TAB1.C&TAB2.A\\
7&9&3\\
\end{tabular}
\end{antwort}

\end{enumerate}
\end{document}
