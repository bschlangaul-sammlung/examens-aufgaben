\documentclass{lehramt-informatik-haupt}

\begin{document}

%%%%%%%%%%%%%%%%%%%%%%%%%%%%%%%%%%%%%%%%%%%%%%%%%%%%%%%%%%%%%%%%%%%%%%%%
% Theorie-Teil
%%%%%%%%%%%%%%%%%%%%%%%%%%%%%%%%%%%%%%%%%%%%%%%%%%%%%%%%%%%%%%%%%%%%%%%%

\chapter{Einführung in relationale Datenbanksysteme \&
Datenmodellierung}

%-----------------------------------------------------------------------
%
%-----------------------------------------------------------------------

\section{Grundlagen von Datenbanksystemen}

%%
%
%%

\subsection{Begriffsklärung\footcite[Seite 13]{winter}}

\begin{itemize}
\item DB = Datenbank
\item Menge der gespeicherten Daten
\item DBMS = Datenbankmanagementsystem
\item Gesamtheit aller Programme für den Umgang mit den Daten
\item DBS = Datenbanksystem
\item System zur Beschreibung, Verwaltung, Speicherung und Wiedergewinnung
großer Datenmengen
\item Möglichkeit der gleichzeitigen Nutzung durch mehrer
Anwendungsprogramme
\item Informationssystem
\item Bereitstellung von Informationen zur Steuerung eines Aufgabenbereichs
\item Zugriff auf elektronisch gespeicherte Daten
\item realisiert durch DBMS
\end{itemize}

%%
%
%%

\subsection{Begriffshierachie}

\begin{itemize}
\item Informationssystem
\item DBS = Datenbanksystem

\begin{itemize}
\item DB = Datenbank
\item DBMS = Datenbankmanagementsystem
\end{itemize}
\end{itemize}

%%
%
%%

\subsection{Aufgaben des Datenbankmanagementsystems
\footcite[Seite 13]{winter}}

\begin{itemize}
\item Persistenz
\item Datenaustausch
\item Zugriffskontrolle
\item Effizienz
\end{itemize}

%%
%
%%

\subsection{grundlegendes Prinzip bei Datenbanksystemen
(DBS)}

\begin{itemize}
\item strikte Trennung von Daten und Datenbearbeitung
\item Benutzer unabhängig von der eigentlichen Organisation der Daten
\item Zurverfügungstellung einer Datenschnittstelle
\end{itemize}

%%
%
%%

\subsection{Datenbanken in der Praxis
\footcite[Seite 13]{winter}}

\begin{itemize}
\item Einzelbenutzersystem: Datenbestand von einem Anwender eingetragen und
gepflegt

\item Mehrbenutzersystem: gleichzeitiger Zugriff von unterschiedlichen
Benutzergruppen, Arbeitsumgebung für einzelnen Benutzer wie
Einzelbenutzersystem

\item Verteilte Datenbankensysteme: DB und DBS auf verschiedene Rechner
verteilt, durch Netzwerk erscheint sie für den Benutzer als zentrale,
einheitliche Datenbank z. B. Client-Server-Datenbanksysteme

\item Multidatenbanksystem
\end{itemize}

%%%%%%%%%%%%%%%%%%%%%%%%%%%%%%%%%%%%%%%%%%%%%%%%%%%%%%%%%%%%%%%%%%%%%%%%
% Aufgaben
%%%%%%%%%%%%%%%%%%%%%%%%%%%%%%%%%%%%%%%%%%%%%%%%%%%%%%%%%%%%%%%%%%%%%%%%

\chapter{Aufgaben}

%-----------------------------------------------------------------------
% Aufgabe 1
%-----------------------------------------------------------------------

\section{Aufgabe 1: Terminologie\footcite[Seite 1]{db:ab:1}}

\noindent
Beschreiben die Begriffe \emph{„Datenbank“} und \emph{„Datenbanksystem“}
das Gleiche?
Kurze Begründung!
\footcite[13]{winter}

\begin{antwort}
Nein! \emph{Datenbanksystem} ist der Oberbegriff. Zu
\emph{Datenbankensystem} (DBS) gehört die \emph{Datenbank} (DB) und das
\emph{Datenbankmanagementsystem} (DBMS). Unter dem Begriff
\emph{Datenbank} versteht man die Menge der gespeicherten Daten.
(Die zweite
Komponente eines Datenbanksystems ist das Datenbankmanagementsystems,
mit Hilfe dessen die Daten in der Datenbank verwaltet werden können.)
\end{antwort}

%-----------------------------------------------------------------------
% Aufgabe 2
%-----------------------------------------------------------------------

\section{Aufgabe 2: ddi.cs.fau.de\footcite[Seite 1]{db:ab:1}}

Zur Speicherung von Daten kann ein Dateisystem verwendet werden.
Betrachten Sie die Informationsseiten der Didaktik der Informatik an der
FAU im Internet (https: //ddi.cs.fau.de/). Gehen Sie davon aus, dass für
jede Seite, die Sie betrachten, eine Datei existiert, in der die auf der
Seite sichtbaren Informationen gespeichert sind.
\footcite[Seite 19-20]{winter}

\begin{enumerate}

%%
% (a)
%%

\item Warum ist diese Art der Datenabspeicherung nicht besonders
günstig?

\begin{antwort}
Redundanz:

In dieser Art zu speichern sind viele Redundanzen enthalten. So muss
beispielsweise das Menü mit den HTML-Links auf jeder Seite gespeichert
sein, um durch die Seite navigieren zu können.

Beschränkte Zugriffsmöglichkeit:

Die Daten können nur
schlecht maschinell abgefragt werden. Sie können nur einzeln im Browser
aufgerufen werden.

Beschränkte Zugriffskontrolle:

Die HTML-Seiten können entweder als ganze Seite unter einen
Passwortschutz gestellt werden oder vollkommen öffentlich ins Netz
gestellt werden. So können einzelene Informationen auf den Seite, wie
zum Beispiel die Raumnummer nicht einzelen in ihrer Sichtbarkeit
beeinflusst werden.
\end{antwort}

\begin{antwort}[muster]
Die gleiche Information kommt auf sehr vielen verschiedenen Seiten vor!
Damit muss dieselbe Information in unterschiedlichen Dateien
abgespeichert werden, d. h. ein Großteil der Information ist redundant
gespeichert.
\end{antwort}

%%
% (b)
%%

\item Es wird angenommen, dass folgende Aktualisierungen durchgeführt
werden müssen:

\begin{itemize}
\item Die Mitarbeiter haben sich geändert.
\item Das Projekt „AMI – Agile Methoden im Informatikunterricht“ ist
abgeschlossen. Alle diesbezüglichen Informationen sollen deshalb
gelöscht werden.
\end{itemize}

Zu welchem Problemen kann es dabei (aufgrund der Datenspeicherung)
kommen?

\begin{antwort}
Nicht aus alle Seiten werden die Links zur der Unterseite Projekt „AMI“
entfernt. Es kann zur Inkonsistenz kommen.
\end{antwort}

\begin{antwort}[muster]
Mitarbeiterinformationen wie auch die Daten des Kurses sind redundant
gespeichert. Änderungen oder Löschungen müssen deshalb in allen Dateien
erfolgen, in denen die entsprechenden Informationen gespeichert sind.
Bei vielen Dateien hat man aber in der Regel keinen Überblick, wo eine
bestimmte Information überall abgespeichert ist. Dies kann sehr leicht
zu einem inkonsistenten, d. h. nicht stimmigen, Datenbestand führen.
Konkret können beispielsweise folgende Probleme auftreten:

\begin{itemize}
\item Die Änderung der Mitarbeiter wird an einer Stelle vergessen, d. h.
dass z. B. Dr. X noch Anfragen und E-Mails an die Adresse X@fau.de
bekommt, obwohl er schon längst ausgeschieden ist und die o. g.
E-Mail-Adresse nicht mehr existiert.

\item Zum Löschen der AMI-Seite müssen die zentrale Projektseite und
alle Links gelöscht werden. Die Löschung eines Link auf die zentrale
Projektseite wird vergessen, d. h. man bekommt einen „toten“ Link.
\end{itemize}
\end{antwort}
\end{enumerate}

\literatur

\end{document}
