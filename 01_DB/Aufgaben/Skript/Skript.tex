\documentclass{lehramt-informatik-haupt}
\liLadePakete{normalformen,syntax}

\begin{document}
\let\ah=\liAttributHuelle
\let\m=\liMenge

%-----------------------------------------------------------------------
%
%-----------------------------------------------------------------------

\section{Aufgabe 5.1 (Seite 79)}

\begin{enumerate}
\item Gegeben sind die Domänen Klassenstufe = \m{5, 11} und
Raumnummer = \m{101, 102, 202}\}. Bestimmen Sie das kartesische Produkt
Klassenstufe x Raumnummer.

\begin{liAntwort}
\m{(5, 101), (5, 102), (5, 202), (11, 101), (11, 102), (11,202)}
\end{liAntwort}

\item Es gilt: Klasse $\subseteq$ Klassenstufe $\times$ Raumnummer.
Bestimmen Sie diese Relation bezüglich der dem Material
zugrundeliegenden Schulverwaltungsdaten.

\begin{liAntwort}
Klasse = {(5, 101), (11, 202)}
\end{liAntwort}

\end{enumerate}

%-----------------------------------------------------------------------
%
%-----------------------------------------------------------------------

\section{Aufgabe 5.2 (Seite 81)}

\begin{enumerate}

%%
%
%%

\item Die Objekte Schüler lassen sich durch eine Relation mit
dem Relationschema Schueler (Eintrittsjahr, Nr, Name, Konfession)
darstellen. Geben Sie die Extension der Relation an.

Lösung: \m{(2017, 1, Max Mustermann, rk), (2018, 2, Sabine Kracker, ev)}

%%
%
%%

\item Klassen lassen sich durch Angabe von Name und
Klassenzimmer beschreiben. Geben Sie ein mögliches Relationenschema
Klasse an.

\begin{liAntwort}
Klasse \m{[Name: STRING, Klassenzimmer: STRING]}
\end{liAntwort}

\end{enumerate}

%-----------------------------------------------------------------------
%
%-----------------------------------------------------------------------

\section{Aufgabe 7.1 (Seite 161)}

\begin{enumerate}

%%
% 1.
%%

\item Betrachten Sie das Datenbankschema der Schulverwaltungsdatenbank
und geben Sie eine mögliche Reihenfolge der Tabellenerzeugung an!

\begin{minted}{md}
ist_Fachlehrkraft_von{Klasse: INTEGER, Fach: STRING, Lehrkraft: INTEGER, Stundenzahl: INTEGER}
Lehrkraft{PersNr: INTEGER, Name: STRING, Geschlecht: {'m', 'w'}, Wohnort: STRING, Geburtsjahr: INTEGER}
Klasse{Name: INTEGER, Klassenzimmer: INTEGER, Klassenleitung: INTEGER}
Schueler{Eintrittsjahr: INTEGER, Nr: INTEGER, Name: STRING, Konfession: {'ev', 'rk', 'isl', 'bk'}, gehoert_zu: INTEGER}
hat_Lehrbefaehigung_in{Lehrkraft: INTEGER, Fach: STRING}
hat_Fachbetreuung_in{Lehrkraft: INTEGER , Fach: STRING, Wertigkeit: {1, 2}}
Fach{Name:STRING, Pflichtfach: {'ja', 'nein'}}
\end{minted}

\noindent
\verb|PersNr| wurde umbenannt in \verb|Klassenleitung| und ist ein
Fremdschlüssel von \verb|Lehrkraft|

\noindent
\verb|gehoert_zu| ist das umbenannte Attribut \verb|Klasse.Name|, also
ein Fremdschlüssel von \verb|Klasse|.

\begin{minted}{md}
Lehrkraft
Fach
Klasse (Lehrkraft.Klassenleitung)
Schueler (Klasse.Name)
hat_Lehrbefaehigung_in (Lehrkraft.PersNr, Fach.Name)
ist_Fachlehrkraft_von (Klasse.Name, Fach.Name, Lehrkraft.PersNr)
hat_Fachbetreuung_in (Lehrkraft.PersNr, Fach.Name)
\end{minted}

%%
% 2.
%%

\item Geben Sie nun die zugehörigen \verb|CREATE TABLE|-Befehle an
(Angabe Seite 134, Lösung Seite 161).

\begin{minted}{sql}
CREATE TABLE Lehrkraft (
  PersNr INTEGER PRIMARY KEY,
  Name VARCHAR(60),
  Geschlecht CHAR(1),
  Wohnort VARCHAR(60),
  Geburtsjahr INTEGER,
  CHECK (Geschlecht in ('m', 'w'))
);

CREATE TABLE Fach (
  Name VARCHAR(60) PRIMARY KEY,
  Pflichtfach INTEGER,
  CHECK (Pflichtfach in (0, 1))
);

CREATE TABLE Klasse (
  Name INTEGER,
  Klassenzimmer INTEGER,
  Klassenleitung INTEGER,
  PRIMARY KEY (Name, Klassenzimmer),
  FOREIGN KEY (Klassenleitung) REFERENCES Lehrkraft(PersNr)
);

CREATE TABLE Schueler (
  Eintrittsjahr INTEGER,
  Nr INTEGER PRIMARY KEY,
  Name VARCHAR(60),
  Konfession VARCHAR(2),
  gehoert_zu INTEGER,
  CHECK (Konfession in ('ev', 'rk', 'isl', 'bk')),
  FOREIGN KEY (gehoert_zu) REFERENCES Klasse(Name)
);

CREATE TABLE hat_Lehrbefaehigung_in (
  Lehrkraft INTEGER NOT NULL,
  Fach VARCHAR(20) NOT NULL,
  PRIMARY KEY (Lehrkraft, Fach),
  FOREIGN KEY (Lehrkraft) REFERENCES Lehrkraft(PersNr),
  FOREIGN KEY (Fach) REFERENCES Fach(Name)
);

CREATE TABLE ist_Fachlehrkraft_von (
  Klasse INTEGER NOT NULL,
  Fach VARCHAR(20) NOT NULL,
  Lehrkraft INTEGER NOT NULL,
  Stundenzahl INTEGER NOT NULL,
  PRIMARY KEY (Klasse, Fach, Lehrkraft),
  FOREIGN KEY (Klasse) REFERENCES Klasse(Name),
  FOREIGN KEY (Fach) REFERENCES Fach(Name),
  FOREIGN KEY (Lehrkraft) REFERENCES Lehrkraft(PersNr)
);

CREATE TABLE hat_Fachbetreuung_in (
  Lehrkraft INTEGER NOT NULL,
  Fach VARCHAR(20) NOT NULL,
  Wertigkeit INTEGER NOT NULL,
  CHECK (Wertigkeit in (1, 2)),
  PRIMARY KEY (Lehrkraft, Fach),
  FOREIGN KEY (Lehrkraft) REFERENCES Lehrkraft(PersNr),
  FOREIGN KEY (Fach) REFERENCES Fach(Name)
);
\end{minted}

\end{enumerate}

%-----------------------------------------------------------------------
%
%-----------------------------------------------------------------------

\section{Aufgabe 7.2 (Seite 134)}

\begin{enumerate}
\item Geben Sie eine mögliche Reihenfolge für das Löschen der Tabellen
der Schulverwaltungsdatenbank an! (Angabe Seite 134, Lösung Seite 162)

Umgekehrte Reiheinfolge wie die Erstellung.

\begin{minted}{md}
hat_Fachbetreuung_in (Lehrkraft.PersNr, Fach.Name)
ist_Fachlehrkraft_von (Klasse.Name, Fach.Name, Lehrkraft.PersNr)
hat_Lehrbefaehigung_in (Lehrkraft.PersNr, Fach.Name)
Schueler (Klasse.Name)
Klasse (Lehrkraft.Klassenleitung)
Fach
Lehrkraft
\end{minted}
\end{enumerate}

%-----------------------------------------------------------------------
%
%-----------------------------------------------------------------------

\section{Aufgabe 7.3 (Seite 135)}

\begin{enumerate}

%%
% 1.
%%

\item Zur Tabelle Lehrkraft soll die Spalte Amtsbezeichnung hinzugefügt
werden. Möglich sind die Bezeichnungen StR, OStR, StD und OStD. (Angabe
Seite 135)

\begin{minted}{sql}
ALTER TABLE Lehrkraft
ADD COLUMN Amtsbezeichnung VARCHAR(10)
CHECK (Amtsbezeichnung in ('StR', 'OStR', 'StD', 'OStD'));
\end{minted}

%%
% 2.
%%

\item In die Tabelle Schueler soll die Spalte Konfession gelöscht
werden. (Angabe Seite 135)

\begin{minted}{sql}
ALTER TABLE Schueler
DROP COLUMN Konfession;
\end{minted}

\end{enumerate}

%-----------------------------------------------------------------------
%
%-----------------------------------------------------------------------

\section{Aufgabe 7.4 (Seite 136)}

\begin{enumerate}

%%
% 1.
%%

\item Falbala (Konfession: bk) tritt im Schuljahr 2000 als erste
Schülerin in die Schule ein. Sie kommt in die Klasse 11. (Angabe Seite
136, Lösung 162)

\begin{minted}{sql}
INSERT INTO Schueler (Name, Konfession, Eintrittsjahr, Nr, gehoert_zu)
VALUES ('Falbala', 'bk', 2000, 1, 11);
\end{minted}

%%
% 2.
%%

\item Das Fach Informatik soll in den Fächertabelle aufgenommen werden.
Es ist aber noch nicht klar, ob das Fach ein Pflichtfach ist oder nicht.
(Angabe Seite 136, Lösung 162)

\begin{minted}{sql}
INSERT INTO Fach (Name) VALUES (Informatik);
\end{minted}

%%
% 3.
%%

\item Die Lehrerin Graf qualifiziert sich im Fach Informatik nach und
hat damit die Lehrbefähigung in diesem Fach. (Angabe Seite 137, Lösung
163)

wahrscheinlich falsch wegen verschachtelten \verb|SELECT|.

\begin{minted}{sql}
INSERT INTO hat_Lehrbefaehigung_in (Lehrkraft, Fach)
VALUES((SELECT PersNr FROM Lehrkraft WHERE Name = 'Graf'), 'Informatik');
\end{minted}

%%
% 4.
%%

\item Die Klasse 7 wird in die Datenbank aufgenommen. Die Klassenleitung
hat Frau Rinser, das Klassenzimmer steht noch nicht fest. (Angabe Seite
137, Lösung 163)

wahrscheinlich falsch wegen verschachtelten \verb|SELECT|.

\begin{minted}{sql}
INSERT INTO Klasse (Name, Klassenleitung)
VALUES (7, (SELECT PersNr FROM Lehrkraft WHERE Name = 'Rinser'));
\end{minted}
\end{enumerate}

%-----------------------------------------------------------------------
% 7.5
%-----------------------------------------------------------------------

\section{Aufgabe 7.5 (Seite 138)}

\begin{enumerate}

%%
% 1.
%%

\item Der Schüler Obelix (Eintrittsjahr 1999/2) ist aus der Schule
ausgetreten und soll deswegen aus der Datenbank gelöscht werden. (Angabe
Seite 138, Lösung Seite)

\begin{minted}{sql}
DELETE FROM Klasse
WHERE Name = 'Obelix' AND Eintrittsjahr = 1999 AND Nr = 2;
\end{minted}

%%
% 2.
%%

\item aufgrund sinkender Schülerzahl ist für das Fach Mathematik nur
noch ein Fachbetreuer zulässig. Herr Gauß muss deshalb das Amt aufgeben.
(Angabe Seite 138, Lösung Seite 163)

falsch: in \verb|hat_Fachbetreuung_in| steht keine Lehrername nur die ID

\begin{minted}{sql}
DELETE FROM hat_Fachbetreuung_in
WHERE Lehrkraft = 'Gauß' and Fach = 'Mathematik';
\end{minted}

Musterlösung:

\begin{minted}{sql}
DELETE FROM hat_Fachbetreuung_in
WHERE Lehrkraft IN (
  SELECT PersNr
  FROM Lehrkraft
  WHERE Name = 'Gauss'
);
\end{minted}

%%
% 3.
%%

\item Alle Lehrkräfte der Jahrgänge 1941 und früher, werden pensioniert
und aus der Datenbank entfernt. (Angabe Seite 138, Lösung Seite 163)

\begin{minted}{sql}
DELETE FROM Lehrkraft WHERE Geburstjahr <= 1941;
\end{minted}

%%
% 4.
%%

\item Ist das Löschen von Daten wie in Übung 1 und 3 in der Realität
adäquat? Was gibt es für Alternativen? (Angabe Seite 138, Lösung Seite
163)

\begin{liAntwort}
Man könnte eine neue Spalte einfügen, \zB bei Lehrer pensioniert.
\end{liAntwort}

\end{enumerate}

%-----------------------------------------------------------------------
% 7.6
%-----------------------------------------------------------------------

\section{Aufgabe 7.6 (Seite 163)}

\begin{enumerate}

%%
% 1.
%%

\item Die Klasse 11 erhält als neues Klassenzimmer den Raum 303. (Angabe
Seite 139 Lösung Seite 163)

\begin{minted}{sql}
UPDATE Klasse SET Klassenzimmer = 303 WHERE Name = 11;
\end{minted}

%%
% 2.
%%

\item Der Schüler Asterix (Eintrittsjahr 1999, Nummer 1) hat seinem
Namen geändert und heißt nun Idefix. Außerdem ist er zum evangelischen
Glauben konvertiert. (Angabe Seite 139 Lösung Seite 163)

\begin{minted}{sql}
UPDATE Schueler
SET Name = 'Idefix', Konfession = 'ev'
WHERE Eintrittsjahr = 1999 AND Nummer = 1;
\end{minted}

%%
% 3.
%%

\item Das Fach Informatik ist nun Pflichtfach. (Angabe Seite 139 Lösung
Seite 163)

\begin{minted}{sql}
UPDATE Fach
SET Pflichtfach = 1
WHERE Name = 'Informatik';
\end{minted}

\end{enumerate}

%-----------------------------------------------------------------------
% 7.7
%-----------------------------------------------------------------------

\section{Aufgabe 7.7 (Seite 143)}

\begin{enumerate}

%%
% 1.
%%

\item Welche Lehrkräfte - unter Angabe von Personalnummer, Name und
Geschlecht - gibt es im Kollegium? (Angabe Seite 143)

\begin{minted}{sql}
SELECT PersNr, Name, Geschlecht FROM Lehrkraft;
\end{minted}

%%
% 2.
%%

\item Welche Schülerinnen und Schüler - unter Angabe von Eintrittsjahr,
Nummer und Name - sind an der Schule? (Angabe Seite 143)

\begin{minted}{sql}
SELECT Eintrittsjahr, Nr, Name FROM Schueler;
\end{minted}

%%
% 3.
%%

\item Welche Fächer werden an der Schule unterrichtet? Die Ausgabespalte
soll den Namen “Angebotene Faecher” erhalten. (Angabe Seite 143)

falsch: Spaltenname darf kein Leerzeichen haben.

\begin{minted}{sql}
SELECT Name AS "Angebotene_Faecher" FROM Fach;
\end{minted}

%%
% 4.
%%

\item Welche Schülernamen existieren an der Schule? Duplikate sollen
nicht ausgegeben werden. (Angabe Seite 143)

\begin{minted}{sql}
SELECT DISTINCT Name FROM Schueler;
\end{minted}

\end{enumerate}

%-----------------------------------------------------------------------
%
%-----------------------------------------------------------------------

\section{Aufgabe 7.8 (Seite 146)}

\begin{enumerate}

%%
% 1.
%%

\item Welche Lehrer haben eine dreistellige Personalnummer?

\begin{minted}{sql}
SELECT * FROM Lehrkraft WHERE Personalnummer > 99;
\end{minted}

%%
% 2.
%%

\item Welche Lehrerinnen - unter Angabe von Personalnummer und Name -
wohnen in Berlin oder Muenchen?

\begin{minted}{sql}
SELECT PersNr, Name
FROM Lehrkraft
WHERE
  Geschlecht = 'w' AND
  Wohnort IN ('Berlin', 'Muenchen');
\end{minted}

%%
% 3.
%%

\item Welche Schülernamen gibt es bei den Bekenntnislosen in der fünften
Klasse?

\begin{minted}{sql}
SELECT DISTINCT Name
FROM Schueler
WHERE Konfession = 'bk';
\end{minted}

\item Welche Ergebnisse liefern folgende Anfragen?

\begin{liAntwort}
Alle männliches Lehrkräfte oder LehrerInnen die in Passau wohnen und
nach 1944 geboren sind.

Alle Lehrkräft nach 1944 geboren, die entweder männlich sind oder in
Passau wohnen.
\end{liAntwort}

\end{enumerate}

%-----------------------------------------------------------------------
% Aufgabe 7.9
%-----------------------------------------------------------------------

\section{Aufgabe 7.9 (Seite 147)}

\begin{enumerate}
\item Welche Fächer, alphabetisch sortiert, gibt es an der Schule?

\begin{minted}{sql}
SELECT Name
FROM Fach
ORDER BY Name;
\end{minted}

\item Gewünscht ist Klasse, Eintrittsjahr und Name aller katholischen
Schülerinnen und Schüler, sortiert nach der Klasse und innerhalb der
Klasse nach dem Eintrittsjahr.

\begin{minted}{sql}
SELECT gehoert_zu AS Klasse, Eintrittsjahr, Name
FROM Schueler
WHERE Konfesstion = 'rk'
ORDER BY gehoert_zu, Eintrittsjahr;
\end{minted}

\item Gesucht sind Personalnummer und Name der Lehrerinnen, die in
Passau wohnen, absteigend sortiert nach dem Geburtsjahr.

\begin{minted}{sql}
SELECT PersNr, Name FROM Lehrkraft
WHERE Geschlecht = 'w' AND Wohnort 'Passau'
ORDER BY Geburstjahr DESC;
\end{minted}

\end{enumerate}

%-----------------------------------------------------------------------
% 7.10
%-----------------------------------------------------------------------

\section{Aufgabe 7.10 (Seite 149)}

\begin{minted}{sql}
CREATE TABLE Lehrkraft (
PersNr INTEGER PRIMARY KEY ,
Name VARCHAR(20) NOT NULL,
Geschlecht VARCHAR(1) CHECK (Geschlecht IN ('m', 'w')),
Wohnort VARCHAR(20) ,
Geburtsjahr INTEGER);
\end{minted}

\begin{minted}{sql}
CREATE TABLE Schueler (
Eintrittsjahr INTEGER,
Nr INTEGER,
Name VARCHAR(20) NOT NULL,
Konfession VARCHAR(3),
gehoert_zu INTEGER,
PRIMARY KEY (Eintrittsjahr, Nr),
CHECK (Konfession IN ('rk', 'ev', 'bk', 'isl')),
FOREIGN KEY (gehoert_zu) REFERENCES Klasse (Name) ) ;
\end{minted}

\begin{enumerate}
\item Wie viele Lehrerinnen bzw. Lehrer aus Passau gibt es?

falsch:

\begin{minted}{sql}
SELECT COUNT(*) AS Anzahl
FROM Lehrkraft
WHERE Wohnort = 'Passau'
GROUP BY Wohnort;
\end{minted}

Musterlösung:

\begin{minted}{sql}
SELECT Geschlecht COUNT(Geschlecht) AS Anzahl
FROM Lehrkraft
WHERE Wohnort = 'Passau'
GROUP BY Geschlecht;
\end{minted}

\item Wie viele Schülerinnen und Schüler pro Konfession gibt es in der
elften Klasse?

war richtig:

\begin{minted}{sql}
SELECT Konfession, COUNT(*) AS Anzahl
FROM Schueler
WHERE gehoert_zu = 11
GROUP BY Konfession;
\end{minted}

\item Wie groß sind die Klassenstärken?

\verb|AS Klasse| vergessen

\begin{minted}{sql}
SELECT gehoert_zu AS Klasse, COUNT(*) AS Anzahl
FROM Schueler
GROUP BY gehoert_zu;
\end{minted}

\item Zu wie vielen Fächern gibt es eine Fachbetreuung?

falsch

\begin{minted}{sql}
SELECT Fach
FROM hat_Fachbetreuung_in
GROUP BY Fach
HAVING COUNT(*) = 1;
\end{minted}

Musterlösung

\begin{minted}{sql}
SELECT COUNT (DISTINCT Fach)
FROM hat_Fachbetreuung_in;
\end{minted}

\item Gesucht ist die höchste laufende Nummer innerhalb eines
Eintrittsjahres! Zusätzlich ausgegeben werden soll dazu das jeweilige
Eintrittsjahr.

\begin{minted}{sql}
SELECT Eintrittsjahr, MAX(Nr)
FROM Schueler
GROUP BY Eintrittsjahr;
\end{minted}

\item Wie viele Mathematikstunden sind von Lehrkräften zu halten, wenn
der Stundenansatz pro Klasse um 2 erhöht wird?

\begin{minted}{sql}
?
\end{minted}
\end{enumerate}

%-----------------------------------------------------------------------
%
%-----------------------------------------------------------------------

\section{Aufgabe 7.11 (Seite 150)}

\begin{enumerate}

%%
% 1.
%%

\item Zu welchem Fach gibt es nur eine Lehrkraft?

\begin{minted}{sql}
SELECT Fach
FROM hat_Lehrbefaehigung_in
GROUP BY Fach
HAVING COUNT(*) = 1;
\end{minted}

%%
% 2.
%%

\item Welche Klassen haben derzeit mehr als 10 Stunden Unterricht?
Auszugeben sind die Klasse und deren Gesamtstundenzahl. (Lösung Seite
166)

\begin{minted}{sql}
SELECT Klasse, SUM(Stundenzahl)
FROM ist_Fachlehrkraft_von
GROUP BY Klasse
HAVING SUM(Stundenzahl) > 10;
\end{minted}

%%
% 3.
%%

\item Ist folgende Anfrage vernünftig?  (Lösung Seite 166)
\end{enumerate}

\begin{minted}{sql}
SELECT Konfession
FROM Schueler
WHERE gehoert_zu = 11
GROUP BY Konfession
HAVING COUNT (*) = 0 ;
\end{minted}

\begin{liAntwort}
Ist nicht vernünftig, da Konfession gecheckt wird auf ev rk etc und
also Konfession immer einen Wert hat.  (Lösung Seite 166)
\end{liAntwort}

%-----------------------------------------------------------------------
%
%-----------------------------------------------------------------------

\section{Aufgabe 7.12 (Seite 153)}

\begin{enumerate}

%%
% 1.
%%

\item Welche Fächer dürfen die einzelnen Lehrkräfte unterrichten?
Gewünscht ist die Ausgabe der Namen von Lehrkraft und Fach.

\begin{minted}{sql}
SELECT l.Name, f.Fach
FROM Lehrkraft l, hat_Lehrbefaehingung_in f
WHERE l.PersNr = f.Lehrkraft;
\end{minted}

%%
% 2.
%%

\item Welche Lehrerin bzw. welcher Lehrer (Angabe des Namens) ist
Fachbetreuer in Deutsch?

\begin{minted}{sql}
SELECT l.Name
FROM Lehrkraft l, hat_Fachbetreuung_in f
WHERE l.PersNr = f.Lehrkraft AND f.Fach = 'Deutsch';
\end{minted}

%%
% 3.
%%

\item Gesucht sind die Namen der Lehrkräfte, die die Schülerin Falbala
unterrichten?

\begin{minted}{sql}
SELECT l.Name
FROM Lehrkraft l, ist_Fachlehrkraft_von f, Schueler s
WHERE
  s.Name = 'Falbala' AND
  s.gehoert_zu = f.Klasse AND
  f.Lehrkraft = l.PersNr;
\end{minted}

%%
% 4.
%%

\item Gibt es Lehrkräfte (Angabe der PersNr), die mehr als ein Fach in
derselben Klasse unterrichten?

\begin{minted}{sql}
SELECT f1.Lehrkraft AS PersNr
FROM ist_Fachlehrkraft_von f1, ist_Fachlehrkraft_von f1
WHERE
  f1.Lehrkraft = f2.Lehrkraft AND
  f1.Klasse = f2.Klasse AND
  f1.Fach <> f2.Fach;
\end{minted}

\end{enumerate}

%-----------------------------------------------------------------------
% 7.13
%-----------------------------------------------------------------------

\section{Aufgabe 7.13 (Seite 154)}
\begin{enumerate}
\item Welche Lehrkraft ist Fachbetreuer in Sport?

\begin{minted}{sql}
SELECT Name FROM Lehrkraft WHERE PersNr = (
  SELECT Lehrkraft FROM hat_Fachbetreuung_in WHERE Fach = 'Sport'
);
\end{minted}

\item Welche Lehrkräfte sind älter als der Durchschnitt des Kollegiums?

\begin{minted}{sql}
SELECT Name FROM Lehrkraft WHERE Geburstjahr < (
  SELECT AVG(Geburtsjahr) FROM Lehrkraft
);
\end{minted}

\end{enumerate}

%-----------------------------------------------------------------------
% 7.14
%-----------------------------------------------------------------------

\section{Aufgabe 7.14 (Seite 156)}

\begin{enumerate}
\item Welche Lehrerinnen sind 1950, 1952, 1956 oder 1957 geboren?
(Lösung Seite 167)

\begin{minted}{sql}
SELECT Name
FROM Lehrkraft
WHERE
  Geschlecht = 'w' AND
  Geburtsjahr IN (1950, 1952, 1956, 1957);
\end{minted}

\item Welche Lehrkräfte haben die Fachbetreuung in Mathematik? (Lösung
Seite 167)

\begin{minted}{sql}
SELECT Name FROM Lehrkraft WHERE PersNr IN (
  SELECT Lehrkraft FROM hat_Fachbetreuung_in WHERE Fach = 'Mathematik'
);
\end{minted}

\item Welchen Lehrkräften, alphabetisch absteigend sortiert, wurden
bereits Klassen zugeteilt? (Lösung Seite 167)

falsch: \verb|ist_Fachlehrkraft_von| nicht verwendet. Es geht nicht
um die Klassenleitung.

\begin{minted}{sql}
SELECT Name FROM LEHRKRAFT WHERE PersNr IN (
  SELECT Klassenleitung FROM Klasse
)
ORDER BY Name DESC;
\end{minted}

\end{enumerate}

%-----------------------------------------------------------------------
% 7.15
%-----------------------------------------------------------------------

\section{Aufgabe 7.15 (Seite 157)}
\begin{enumerate}
\item Welche Lehrkräfte haben die Lehrbefähigung im Fach Informatik?

\begin{minted}{sql}
SELECT (
  SELECT Name FROM Lehrkraft WHERE PersNr = Lehrkraft
)
FROM hat_Lehrbefaehigung_in
WHERE Fach = 'Informatik';
\end{minted}

\item Welche Lehrkräfte haben noch keine Unterrichtstunden?

nicht sicher ob diese Lösung auch stimmt.
V
\begin{minted}{sql}
SELECT Name
FROM Lehrkraft
WHERE PersNr NOT IN (SELECT DISTINCT Lehrkraft FROM ist_Fachlehrkraft_von);
\end{minted}

\item Gesucht sind die Personalnummern der Lehrkräfte, die gleichzeitig
in Fachbetreuung und Klassenleitung eingebunden sind.

nicht sicher ob diese Lösung auch stimmt.

\begin{minted}{sql}
SELECT FROM Klasse k, hat_Fachbetreuung_in f WHERE k.Klassenleitung = f.Lehrkraft;
\end{minted}

\end{enumerate}

%-----------------------------------------------------------------------
% 7.16
%-----------------------------------------------------------------------

\section{Aufgabe 7.16 (Seite 158)}

\begin{enumerate}
\item Gesucht sind alle Schülerinnen und Schüler mit Ausnahme der
Bekenntnislosen. (Lösung Seite 168)

\begin{minted}{sql}
SELECT Name FROM Schueler
EXCEPT
SELECT Name FROM Schueler WHERE Konfession = 'bk';
\end{minted}

\item Gesucht sind die Namen des Lehrerkollegiums und der Schülerschaft.
(Lösung Seite 168)

\begin{minted}{sql}
SELECT Name FROM Lehrkraft
UNION
SELECT Name FROM Schueler;
\end{minted}

\item Welchen Lehrkräften (Angabe der Personalnummer) wurden bereits
Klassen zugeteilt? (Lösung Seite 168)

\begin{minted}{sql}
SELECT PersNr FROM Lehrkraft
INTERSECT
SELECT Lehrkraft FROM ist_Fachlehrkraft_von;
\end{minted}

\end{enumerate}

%-----------------------------------------------------------------------
%
%-----------------------------------------------------------------------

\section{Aufgabe 7.17 (Seite 158)}
\begin{enumerate}
\item Welche Fächer mit dem Anfangsbuchstaben M gibt es? (Lösung Seite
168)

\begin{minted}{sql}
SELECT Name
FROM Fach
WHERE Name LIKE 'M%';
\end{minted}

\item Welche Lehrernamen haben an der zweiten Stelle den Buchstaben u?
(Lösung Seite 168)

\begin{minted}{sql}
SELECT Name FROM Lehrkraft
WHERE Name LIKE '_u%';
\end{minted}

\item Heißt die Schülerin aus der 11. Klasse Fabala oder Falbala?
(Lösung Seite 169)

\begin{minted}{sql}
SELECT Name
FROM Schueler
WHERE Name LIKE 'Fa%bala';
\end{minted}

\end{enumerate}

%-----------------------------------------------------------------------
%
%-----------------------------------------------------------------------

\section{Aufgabe 7.18 (Seite 160)}

\begin{enumerate}
\item Es soll eine View mit dem Namen \verb|Klasse_5| erstellt werden,
die die Namen aller Schüler der fünften Klasse zeigt. (Lösung Seite 169)

\begin{minted}{sql}
CREATE VIEW Klasse_5 AS
SELECT Name FROM Schueler WHERE gehoert_zu = 5;
\end{minted}

\item Gibt es in der fünften Klasse einen Schüler mit dem Namen Asterix?
Erstellen Sie eine View Schuelersuche unter Ausnutzung der Lösung der
Vorgängeraufgabe. (Lösung Seite 169)

\begin{minted}{sql}
CREATE VIEW Schuelersuche AS
SELECT Name FROM Klasse_5 WHERE Name = 'Asterix';
\end{minted}

\item Gewünscht ist eine Sicht mit dem Namen \verb|Lehrkraefte_5|, die
eine Liste mit den Lehrkräften, die in der Klasse 5 unterrichten, und
dem entsprechenden Fach berechnet. (Lösung Seite 169)

\begin{minted}{sql}
CREATE VIEW Lehrkraefte_5 AS
SELECT l.NAME
FROM ist_Fachlehrkraft_von f, Lehrkraft l
WHERE f.Klasse = 5 AND f.Lehrkraft = l.PersNr;
\end{minted}

\end{enumerate}

%-----------------------------------------------------------------------
%
%-----------------------------------------------------------------------

\section{Aufgabe 8.1 (Seite 174)}

Gegeben sei die Relation Artikel mit dem Schema
\texttt{Artikel\m{Lieferant, Adresse, Artikelname, Preis}}. Die
Relation wird durch folgende Tabelle repräsentiert:

Geben Sie Beispiele für mögliche Anomalien an!

\begin{liAntwort}
Lösung Seite 203
\begin{description}
\item[Update-Anomalie:] Die Firma Meier verlegt ihren Firmensitz nach
Regensburg. Die Adressänderungen werden nicht konsequent bei allen
betroffenen Tupeln vorgenommen.

\item[Insert-Anomalie:] In die Datenbank soll ein neuer Lieferant
eingetragen werden, der aber vorerst keine Artikel liefert. Da das
Attribut \texttt{Artikel} zum Primärschlüssel gehört, darf dessen Wert
keine Nullwert sein. Damit kann ein Lieferant öhneÄrtikel nicht in die
Datenbank eingetragen werden.

\item[Delete-Anomalie:] Der Lieferant kann kurzfristig weder Seezunge
noch Forelle liefern. Die entsprechenden Einträge müssen gelöscht
werden. Damit geht aber auch die Information über die Adresse des
Lieferanten verloren.
\end{description}
\end{liAntwort}

%-----------------------------------------------------------------------
%
%-----------------------------------------------------------------------

\section{Aufgabe 8.2 (Seite 176)}

Gegeben sei das Relationenschema $Artikel\{ Lieferant, Adresse,
Artikelname, Preis\}$ und die Instanz

Angenommen wird, dass es keine zwei Lieferanten mit demselben Namen
gibt, jeder Lieferant nur eine Adresse hat und kein Lieferant zwei
verschiedene Artikel mit demselben Artikelnamen liefert. Welche FDs
gelten? Begründen Sie Ihre Entscheidung!

Lösung 204

\begin{liAntwort}
\begin{itemize}
\item $Lieferant \rightarrow Adresse$

Über den Lieferantennamen kommt man zur Adresse

\item ${Lieferant, Artikelname \rightarrow Preis}$

Über den Lieferantennamen und den Artikelnamen kommt man zur Preis. Es
gibt zwei verschiedene Arten von Forellen
\end{itemize}
\end{liAntwort}

\begin{liAntwort}
Alle trivialen FDs gelten.
\begin{itemize}
\item $Lieferant \rightarrow Adresse$ gilt, da es keine zwei Lieferanten
mit gleichem Namen geben kann und jeder Lieferant eine eindeutige
Adresse hat. Damit ist durch die Angabe des Lieferanten die
entsprechende Adresse festgelegt.

\item $Artikelname \rightarrow  Preis$ gilt nicht, da es mehrere gleiche
Artikelnamen (die dann zu verschiedenen Lieferanten gehören würden)
geben kann und dadurch nicht eindeutig auf den Preis geschlos- sen
werden kann.

\item $Lieferant \rightarrow Artikel$ gilt nicht, da ein Lieferant
mehrere Artikel anbieten kann.

\item $Lieferant \rightarrow Preis$ gilt nicht, da ein Lieferant mehrere
Artikel anbieten kann und damit keine eindeutige Zuordnung möglich ist.

\item $Artikelname \rightarrow Lieferant$ gilt nicht, da es durchaus
mehrere Artikel gleichen Namens geben kann, die dann aber von
verschiedenen Lieferanten stammen.

\item $Lieferant, Artikelname \rightarrow Preis$ gilt, da durch die
Angabe von Lieferant und Artikelname der
Preis eindeutig bestimmt werden kann.
\end{itemize}

\end{liAntwort}

%-----------------------------------------------------------------------
%
%-----------------------------------------------------------------------

\section{Aufgabe 8.3 (Seite 181)}

\begin{enumerate}

%%
%
%%

\item Gegeben sei das Relationenschema
$Artikel\m{Lieferant, Adresse, Artikelname, Preis}$ , die folgende
Menge $F$ funktionaler Abhängigkeiten für Artikel
$F := \{
  Lieferant \rightarrow Adresse,
  Lieferant Artikelname \rightarrow Preis
\}$
und die Menge
$X = \m{Lieferant, Artikelname}$.
Bestimmen Sie $\ah{F, X}$.

\begin{liAntwort}
$F^+ = \m{Lieferant, Artikelname}$

1. Durchlauf

$Lieferant$ ja $F^+ = \m{Lieferant, Artikelname, Adresse}$
$Lieferant Artikelname$ ja $F^+$ $F^+ = \m{Lieferant, Artikelname, Adresse, Preis}$

2. Durchlauf

keine Veränderungen mehr

$F^+ = \m{Lieferant, Artikelname, Adresse, Preis}$
\end{liAntwort}

%%
%
%%

\item Gegeben sei das Relationenschema $Abstrakt\m{A, B, C, D, E}$ , und
$F := \{
  A D \rightarrow C,
  B \rightarrow D,
  E \rightarrow A
\}$. Bestimmen Sie $\ah{F, \m{A, B}}$.
\end{enumerate}

\begin{liAntwort}
$F^+ = A, B$

1. Durchlauf

A D nichts
B ja $F^+ \cup D$ $F^+ = A B D$
E nichts

2. Durchlauf

A D ja $F^+ \cup A B D C$
\end{liAntwort}

%-----------------------------------------------------------------------
%
%-----------------------------------------------------------------------

\section{Aufgabe 8.4 (Seite 184)}

\begin{enumerate}

%%
%
%%

\item Gegeben sei das Relationenschema
\texttt{Professor\m{PersNr, Dienststelle, Zimmer, Telefon}}
und
\texttt{F = \{
  PersNr Dienststelle $\rightarrow$ Zimmer,
  Zimmer $\rightarrow$ Telefon
\}}.
Ist F bereits minimal?

\begin{liAntwort}

\begin{enumerate}
\item Überprüfung der FDs auf Einfachheit: F enthält nur einfache FDs.

\item Überprüfung, ob redundante Attribute vorliegen:

\begin{itemize}
\item \texttt{Zimmer $\rightarrow$ Telefon}: \texttt{\m{Zimmer}} ist
bereits einelementig, d.h. es gibt kein redundantes Attribut

\item \texttt{PersNr Dienststelle $\rightarrow$ Zimmer}

\begin{itemize}
\item Wir berechnen:
\texttt{\ah{F, \m{PersNr}} = \m{PersNr}}, aber
\texttt{Zimmer $\notin$ \m{PersNr}}.

Dienststelle ist nicht redundant.

\item Wir berechnen:
\texttt{\ah{F, \m{Dienststelle}} = \m{Dienststelle}},
aber
\texttt{Zimmer $\notin$ \m{Dienststelle}}.
PersNr ist nicht redundant.
\end{itemize}

Es gibt keine redundanten Attribute.
\end{itemize}

\item Überprüfung, ob redundante FDs vorliegen:

\begin{itemize}
\item \texttt{Zimmer $\rightarrow$ Telefon}:
Wir berechnen:
\texttt{\ah{F - \m{Zimmer $\rightarrow$ Telefon}, \m{Zimmer}} = \m{Zimmer}},
aber
\texttt{Telefon $\notin$ \m{Zimmer}}.
\texttt{Zimmer $\rightarrow$ Telefon} ist nicht redundant.

\item \texttt{PersNr Dienststelle $\rightarrow$ Zimmer}:
Wir berechnen:
\texttt{\ah{F - \m{PersNr Dienststelle $\rightarrow$ Zimmer}, \m{PersNr, Dienststelle}} =
\m{PersNr, Dienststelle}},
aber
\texttt{Zimmer $\notin$ \m{PersNr, Dienststelle}}.
\texttt{PersNr Dienststelle $\rightarrow$ Zimmer} ist nicht redundant.
\end{itemize}

\end{enumerate}
F ist bereits minimal.
\end{liAntwort}

%%
%
%%

\item Gegeben ist die Menge von FDs \texttt{F = \{%
  PersNr $\rightarrow$ Name Wohnort,
  PersNr Name $\rightarrow$ Geburtsjahr,
  Name $\rightarrow$ PersNr%
\}}.
Bestimmen Sie die minimale Überdeckung von \texttt{F}!

\begin{liAntwort}
\texttt{F = \{%
  PersNr $\rightarrow$ Name,
  PersNr $\rightarrow$ Wohnort,
  PersNr Name $\rightarrow$ Geburtsjahr,
  Name $\rightarrow$ PersNr%
\}}

\begin{itemize}
\item \texttt{PersNr Name $\rightarrow$ Geburtsjahr} ohne \texttt{Name}:

\texttt{Geburtsjahr} $\in$ \texttt{AttrHüll\m{F, \m{PersNr}} = \m{PersNr, Name, Wohnort, Geburtsjahr}}

\item \texttt{PersNr Name $\rightarrow$ Geburtsjahr} ohne \texttt{PersNr}:

\texttt{Geburtsjahr} $\in$ \texttt{AttrHüll\m{F, \m{Name}} = \m{Name, PersNr, Wohnort, Geburtsjahr}}

\texttt{F = \{%
  PersNr $\rightarrow$ Name, Wohnort, Geburtsjahr,
  Name $\rightarrow$ PersNr%
\}}
\end{itemize}

\end{liAntwort}

\end{enumerate}

%-----------------------------------------------------------------------
%
%-----------------------------------------------------------------------

\section{Aufgabe 8.5 (Seite 186])}

Gegeben sei das Relationenschema \texttt{Professor\m{PersNr, Dienststelle, Zimmer, Telefon}}.

\begin{itemize}
\item Bestimmen Sie eine sinnvolle Menge \texttt{F} von FDs!

\begin{liAntwort}
Es ist sicher sinnvoll anzunehmen, dass ein Professor in jeder
Dienststelle höchstens ein Zimmer hat. Ferner hat ein Professorenzimmer
höchstens einen Telefonanschluss. Damit kann beispielsweise folgende
Menge von funktionalen Abhängigkeiten festgelegt werden:

\begin{itemize}
\item \texttt{PersNr Dienststelle $\rightarrow$ Zimmer}
\item \texttt{Zimmer $\rightarrow$ Telefon}
\end{itemize}
\end{liAntwort}

\item Prüfen Sie, ob {PersNr, Dienststelle} ein Superschlüssel ist.

\begin{liAntwort}
\footnotesize
\begin{tabular}{ll}
Ergebnismenge \texttt{Erg} &
Erklärung\\

\texttt{\m{PersNr, Dienststelle}} &
\texttt{Erg} wird initialisiert\\

\texttt{\m{PersNr, Dienststelle} $\cup$ \m{Zimmer}} &
wegen FD \texttt{PersNr Dienststelle $\rightarrow$ Zimmer}\\

\texttt{\m{PersNr, Dienststelle, Zimmer} $\cup$ \m{Telefon}} &
wegen FD \texttt{Zimmer $\rightarrow$ Telefon}\\
\end{tabular}

\end{liAntwort}

\end{itemize}

%-----------------------------------------------------------------------
%
%-----------------------------------------------------------------------

\section{Aufgabe 8.7 (Seite 198)}

\begin{enumerate}
\item Gegeben sei das Relationenschema
$Artikel\m{\textrm{Lieferant, Adresse, Artikelname, Preis}}$
und die Menge
$F = \m{\textrm{Lieferant} \rightarrow \textrm{Adresse}, \textrm{Lieferant Artikelname} \rightarrow \textrm{Preis} }$.
Zeigen Sie, dass Artikel nicht 2NF ist!

\item Gegeben ist das Relationenschema
$Lehrkraft\m{\textrm{PersNr, Name, Geburtsdatum, Wohnort}}$
mit
$F = \m{PersNr \rightarrow Name, PersNr \rightarrow Geburtsdatum, PersNr \rightarrow Wohnort, Name Geburtsdatum \rightarrow PersNr}$.
Ist das Relationenschema in 2NF?

\end{enumerate}

\end{document}
