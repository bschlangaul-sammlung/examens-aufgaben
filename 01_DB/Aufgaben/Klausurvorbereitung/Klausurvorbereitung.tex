\documentclass{lehramt-informatik-haupt}
\liLadePakete{mathe,syntax,er}
\usepackage[normalem]{ulem}
\begin{document}

\chapter{Aufgabenblatt 7: Klausurvorbereitung}

%-----------------------------------------------------------------------
%
%-----------------------------------------------------------------------

\section{Aufgabe 1: Wiederholung ER-Modell und Relationenmodell\footcite{db:ab:7}}

Für einen Flughafen soll eine Datenbank für folgendes Szenario
entwickelt werden: Von den \textbf{Fluggesellschaften} sollen
\emph{Name} und \emph{Hauptsitz} abgespeichert werden. Der
Gesellschaftsname ist dabei eindeutig. Die Gesellschaften sind
Eigentümer von Flugzeugen. Wichtig ist, seit wann das Flugzeug für die
Gesellschaft im Einsatz ist und wie viele Flugzeuge die Gesellschaft
insgesamt besitzt.

Die \textbf{Flugzeuge} tragen eine eindeutige \emph{FZ-Nr}.
Weiterhin soll auch der \textbf{Typ} des Flugzeuges mit
\emph{Sitzplatzanzahl} und \emph{Größe der Besatzung} abrufbar sein.

Für einen \textbf{Flug} setzt eine Fluggesellschaft ein Flugzeug ein.
Dabei muss dieses Flugzeug aber nicht Eigentum der Fluggesellschaft
sein. Der Flug hat eine \emph{Flug-Nr}. Bezüglich eines Fluges sind
\emph{Flug-Nr}, \emph{Abflugzeit} und \emph{Zielflughafen} abzuspeichern.

Ein \textbf{Passagier} kann Flüge buchen. Von den Passagieren müssen
\emph{Name} und \emph{Gebdat} bekannt sein. Dabei ist aber davon
auszugehen, dass es Passagiere mit gleichem Namen und gleichem
Gebdat geben kann.

Bei der \textbf{Buchung} wird dem Passagier eine \emph{Sitzplatz-Nr}
zugeteilt. Für jeden Flug muss die Anzahl der gebuchten Plätze
feststellbar sein.

\begin{enumerate}

%%
% (a)
%%

\item Erstellen Sie ein ER-Diagramm! Verarbeiten Sie dabei nur die
unbedingt notwendigen Informationen. Geben Sie an, wie die nicht im
ER-Modell „auftauchenden“ Informationen bestimmt werden können.

\begin{antwort}
\begin{center}
\begin{tikzpicture}[er2,scale=0.7,transform shape]
\node[entity] (Fluggesellschaft) {Fluggesellschaft};
\node[attribute, left of=Fluggesellschaft] {Name} edge (Fluggesellschaft);
\node[attribute, above of=Fluggesellschaft] {Hauptsitz} edge (Fluggesellschaft);

\node[relationship,right of=Fluggesellschaft] (besitzt) {besitzt} edge (Fluggesellschaft);

\node[entity,right of=besitzt] (Flugzeug) {Flugzeug} edge (besitzt);
\node[attribute, above left of=Flugzeug] {FZ-Nr} edge (Flugzeug);
\node[attribute, above of=Flugzeug] {Typ} edge (Flugzeug);
\node[attribute, above right of=Flugzeug] {Sitzplatzanzahl} edge (Flugzeug);
\node[attribute, right of=Flugzeug] {Besatzung} edge (Flugzeug);

\node[entity,below of=Flugzeug] (Flug) {Flug};

\node[relationship,below of=besitzt] (setztEin) {setztEin}
  edge (Flug)
  edge (Flugzeug)
  edge (Fluggesellschaft);

\node[relationship,right of=Flug] (bucht) {bucht}
  edge (Flug);

\node[entity,below of=Flug] (Passagier) {Passagier}
  edge (bucht);
\node[attribute, left of=Passagier] {Name} edge (Passagier);
\node[attribute, below left of=Passagier] {Gebdat} edge (Passagier);
\node[attribute, below of=Passagier] {PasID} edge (Passagier);
\end{tikzpicture}
\end{center}

%\includegraphics[width=\linewidth]{Aufgabe-1_muster}

Wie viele Flugzeuge eine Fluggesellschaft besitzt ergibt sich nicht
direkt aus dem ER-Modell, sondern kann später durch eine einfache
SQL-Abfrage ermittelt werden:

\begin{minted}{sql}
SELECT FluggesellschaftName, COUNT (*) AS Anzahl
FROM Eigentuemer_von
GROUP BY FluggesellschaftName;
\end{minted}

Auch die Anzahl der gebuchten Sitze pro Flug lässt sich durch eine
SQL-Abfrage ermit- teln:

\begin{minted}{sql}
SELECT Flug-Nr, COUNT (*) AS Anzahl
FROM bucht
GROUP BY Flug-Nr;
\end{minted}
\end{antwort}

%%
% (b)
%%

\item Legen Sie die Primärschlüssel fest. Begründen Sie Ihre
Entscheidung, falls Sie zusätzliche künstliche Schlüssel einfügen.

\begin{antwort}[richtig]
Pass-Nr weil Geburtstdatum und Namen zusammen nicht
eindeutig ist.
\end{antwort}

\begin{antwort}[muster]
Von einem Passagier werden nur Name und Gebdat gespeichert. Da
diese beiden nicht eindeutig sind, können sie nicht als Primärschlüssel
verwendet werden. Folglich muss ein künstlicher Schlüssel eingeführt
werden, die \verb|Passagier-Nr|. \verb|Typ-Nr| des Flugzeugtyps sollte
auch gespeichert werden, da Personalzahl und Sitzplatzzahl nicht
eindeutig sind.
\end{antwort}

%%
% (c)
%%

\item Geben Sie die Funktionalitäten an!

%%
% (d)
%%

\item Erstellen Sie nun ein zu dieser Modellierung passendes
Relationenschema! Markieren Sie Schlüssel und Fremdschlüssel.

\begin{antwort}[richtig]
\verb|angehoeren| wird aufgelöst: \verb|Typ-Nr[Flugzeugtyp]| nach
\verb|Flugzeug|

\verb|besitzen| wird aufgelöst: \verb|FGName[Fluggesellschaft]| nach
\verb|Flugzeug|

\verb|fliegen| wird aufgelöst: \verb|FGName[Fluggesellschaft]|
\verb|FZ-Nr[Flugzeug]| nach \verb|Flug|

\bigskip

{
\footnotesize\ttfamily
Fluggesellschaft(\underline{FGName}, Hauptsitz)

Flug(\underline{Flug-Nr}, Abflugzeit, Zielflughafen,
\dashuline{FGName[Fluggesellschaft]},
\dashuline{FZ-Nr[Flugzeug]})

Flugzeug(\underline{FZ-Nr},
\dashuline{FGName[Fluggesellschaft]},
DatumErsterEinsatz,
\dashuline{Typ-Nr[Flugzeugtyp]})

Flugzeugtyp(\underline{Typ-Nr}, SitzplatzAnzahl, GroesseBesatzung)

Passagier(\underline{Pass-Nr}, Name, Gebdat)

buchen(\dashuline{Pass-Nr[Passagier]},
\dashuline{Flug-Nr[Flug]}, Sitzplatz-Nr)
}
\end{antwort}

%%
% (e)
%%

\item Finden Sie nun jeweils eine Datenbank-Anfrage (in SQL und in
relationaler Algebra) zur Lösung der folgenden Problemstellungen:

\renewcommand{\labelenumii}{\alph{enumii}.}
\begin{enumerate}

%%
% a.
%%

\item Die Fluggesellschaft „Never-Come-Back-Airlines“ (NCA) will wissen,
ob (und wenn ja bei welchem Flug) heute Abend Passagiere (Name,
Sitzplatz-Nr) mit einem ihrer Flugzeuge unterwegs sind, die heute
Geburtstag haben (GebDat = TODAY).

\begin{antwort}[richtig]
$
\pi_{\text{Name,Sitzplatz-Nr}}(
  \sigma_{\text{GebDat} = \text{TODAY}}(\text{Passagier})
  \bowtie
  \text{buchen}\\
  \bowtie
  \sigma_{\text{Name} = \mlq \text{Never-Come-Back-Airlines} \mrq \land \text{Abflugzeit} > \text{18.00}}(\text{Flug})
)
$

\begin{minted}{sql}
SELECT p.Name, p.Sitzplatz-Nr
FROM Passagier p, Flug f, buchen b
WHERE
  f.Name = 'Never-Come-Back-Airlines' AND
  f.Abflugzeit > 18.00 AND
  p.Gebdat = TODAY AND
  f.Flug-Nr = b.Flug-Nr AND
  b.Pass-Nr = p.Pass-Nr;
\end{minted}
\end{antwort}

%%
% b.
%%

\item Ein Passagier möchte erfahren, welcher Flug (Flug-Nr, FZ-Nr,
Abflugzeit, FGName) derjenige mit dem „modernsten“ Flugzeug ist, der
nach „London“ geht.

\begin{minted}{sql}
SELECT f.Flug-Nr, f.FZ-Nr, f.Abflugzeit, f.FGName
FROM Flug f, Flugzeug fz
WHERE
  f.FZ-Nr = fz.FZ-Nr AND
  f.Zielflughafen = 'London' AND
  fz.DatumErsterEinsatz = (
    SELECT MAX(Flugzeug.DatumErsterEinsatz)
    FROM Flug, Flugzeug
    WHERE
      Flug.Zielflughafen = 'London' AND
      Flug.FZ-Nr = Flugzeug.FZ-Nr
  )
\end{minted}

\end{enumerate}

\end{enumerate}

%-----------------------------------------------------------------------
%
%-----------------------------------------------------------------------
\section{Aufgabe 2\footcite{db:ab:7}}

Überführen Sie folgendes ER-Diagramm in ein (verfeinertes)
Relationenschema!

%-----------------------------------------------------------------------
%
%-----------------------------------------------------------------------

\ExamensAufgabeA 66111 / 1996 / 09 : Aufgabe 4

%-----------------------------------------------------------------------
%
%-----------------------------------------------------------------------

\ExamensAufgabeA 66111 / 1997 / 09 : Aufgabe 3

%-----------------------------------------------------------------------
%
%-----------------------------------------------------------------------

\section{Aufgabe 6: Bus-Unternehmen\footcite{db:ab:7}}

Konvertieren Sie das folgende ER-Modell in ein relationales DB-Schema.
Hinweis: Die Angabe der Domänen ist nicht notwendig!

%-----------------------------------------------------------------------
%
%-----------------------------------------------------------------------

\section{Aufgabe 7: IS-A\footcite{db:ab:7}}

Konvertieren Sie folgenden Ausschnitt aus einem ER-Modell in ein Relationenschema.

%-----------------------------------------------------------------------
%
%-----------------------------------------------------------------------

\section{Aufgabe 8: Süße Produktion?\footcite{db:ab:7}}

Vorgegeben sind die Domänen Abteilung = Verwaltung, Lager, Produktion
und Mitarbeiter = Fent, Süß, Dobler.

\begin{enumerate}

%%
% (a)
%%

\item Bestimmen Sie Mitarbeiter x Abteilung! Welche Aussage liefert
dieses kartesische Produkt?

%%
% (b)
%%

\item Geben Sie – orientiert an Aufgabe a) - eine mögliche
Interpretation der Menge (Fent,Produktion), (Süß, Lager) an.

\end{enumerate}

%-----------------------------------------------------------------------
%
%-----------------------------------------------------------------------

\section{Aufgabe 9: Miniwelt\footcite{db:ab:7}}

Modellieren Sie folgende Miniwelt: Eine Firma mit Firmenname und
Firmensitz besteht aus Abteilungen (Abteilungsname, Abteilungsnummer).
Die Abteilungen haben Mitarbeiter, von denen Personalnummer, Büronummer
und die dienstliche Telefonnummer gespeichert werden sollen.

%-----------------------------------------------------------------------
%
%-----------------------------------------------------------------------

\section{Aufgabe 10: Nachteile der Normalisierung (Staatsexamen Herbst
2003, Thema II, Aufgabe 1)\footcite{db:ab:7}}

Erläutern Sie, inwiefern sich eine vollständige Normalisierung
nachteilig auf die Geschwindigkeit der Anfragebearbeitung auswirken kann
und wie darauf reagiert werden kann!

%-----------------------------------------------------------------------
%
%-----------------------------------------------------------------------

\section{Aufgabe 11: Schicht auf Schicht\footcite{db:ab:7}}

Das Drei-Schichten-Modell trägt den verschiedenen Sichten auf eine
Datenbank Rechnung. Geben Sie zu den unten (unter a bis e) genannten
Vorgängen jeweils an, welche
der folgenden Aussagen zutreffen:

\renewcommand{\labelenumi}{(\arabic{enumi})}
\begin{enumerate}
\item Änderungen in bestehenden Anwendungsprogrammen notwendig
\item Änderungen im externen Schema notwendig
\item Änderungen im konzeptionellen Schema notwendig
\item Änderungen im internen Schema notwendig
\end{enumerate}

\subsection{Vorgänge}

\renewcommand{\labelenumi}{(\alph{enumi})}
\begin{enumerate}

%%
% (a)
%%

\item Ein neues Anwendungsprogramm wird geschrieben, das bestehende
Daten nutzt.

%%
% (b)
%%

\item Der Datentyp eines Attributs wird geändert, z.B. wird statt
varchar(20) varchar(30) verwendet.

%%
% (c)
%%

\item Ein neues Anwendungsprogramm wird entwickelt, das neue
(zusätzliche) Datenstrukturen benötigt.

%%
% (d)
%%

\item Es werden neue Daten eingespeichert bzw. bestehende gelöscht.

%%
% (e)
%%

\item Der Zugriff auf die Daten wird optimiert.
\end{enumerate}

\noindent
Sie könnten dazu folgende Tabelle ausfüllen: Kreuzen Sie das
entsprechende Feld an, wenn die Aussage diesbezüglich wahr ist, oder
lassen Sie es andernfalls leer. Ist die Aussage nicht eindeutig und
situationsbedingt wahr, setzen sie ein eingeklammertes Kreuz (x) ein!

\begin{antwort}[muster]
\begin{tabular}{lllll}
  &  1  &  2  & 3   & 4 \\
a &     & (x) &     &   \\
b & (x) & (x) & (x) & x \\
c & (x) & (x) &  x  & x \\
d &     &     &     &   \\
e &     &     &     & x
\end{tabular}

\subsection{Hinweis}

\begin{itemize}
\item Zu b) Bei der vorgegebenen Änderung des Datentyps ist sicher das
interne Schema betroffen, da sich die Speicherstruktur ändert. Wird der
Datentyp „stark“ geändert, d.h. beispielsweise char durch integer
ersetzt, kann das auch Änderungen bei (1), (2) und (3) nach sich ziehen.

\item Zu d) Schemata beschreiben Strukturen. Das Speichern bzw. Löschen
von Daten kann damit keinen Einfluss auf ein Schema haben.
\end{itemize}

\end{antwort}

%-----------------------------------------------------------------------
%
%-----------------------------------------------------------------------

\ExamensAufgabeTTA 46116 / 2015 / 09 : Thema 1 Teilaufgabe 2 Aufgabe 1

\ExamensAufgabeTTA 46116 / 2012 / 03 : Thema 1 Teilaufgabe 1 Aufgabe 3

\literatur

\end{document}
