\documentclass{lehramt-informatik-haupt}
\liLadePakete{syntax}

\begin{document}

%-----------------------------------------------------------------------
%
%-----------------------------------------------------------------------

\section{Top-N-Query\footcite[Seite 8]{db:fs:3}}

Ermittle die TOP(10) der Spitzenverdiener für jede Abteilung und gib
den Rang und den Namen aus. (Alle Angestellten haben
unterschiedliches Gehalt)

\begin{minted}{sql}
SELECT P.AbtNr, COUNT(*) AS Rang, P.Nachname
FROM Angestellte P, Angestellte A
WHERE P.AbtNr = A.AbtNr AND P.Gehalt <= A.Gehalt
GROUP BY P.PersNr, P.Nachname, P.AbtNr
HAVING COUNT(*) <= 10;
\end{minted}

%-----------------------------------------------------------------------
%
%-----------------------------------------------------------------------

\section{Rekursive SQL-Abfragen}

\begin{quellen}
\item \cite[Seite 136]{kemper}
\item \cite[Seite 27-31]{db:fs:3}
\item \cite{db:html:jaxenter:rekursives-sql}
\end{quellen}

\begin{minted}{sql}
WITH RECURSIVE t(v) AS (
  SELECT 1     -- Seed Row
  UNION ALL
  SELECT v + 1 -- Recursion
  FROM t
)
SELECT v
FROM t
LIMIT 5
\end{minted}
\footcite{db:html:jaxenter:rekursives-sql}

Was ist Rekursion? Grundsätzliche Idee: Bestimmte Vorgänge werden auf
ein Produkt, das sie bereits hervorgebracht haben, von neuem angewandt.
Dabei entstehen unendliche Abläufe, die mit Hilfe bestimmter Bedingungen
abgebrochen werden.

Wann ist das erforderlich?

Typisches Beispiel: In einem Unternehmen sollen Hierarchien erfasst
werden. Wir möchten für einen Chef alle Untergebenen ermitteln, aber
eben nicht nur auf der Hierarchieebene direkt unter ihm, sondern ALLE.
Dabei scheitert eine klassische Anweisung!

Basisaufbau:

\begin{minted}{sql}
WITH tabelle (spaltenliste) AS
( UrsprungsSELECT
UNION ALL
RekursionsSELECT )
SELECT spaltenliste FROM tabelle WHERE ...
\end{minted}

Hilfreiche Fragen zur Erstellung:

Welche Spalten sollen in meiner Ergebnismenge auftauchen und/oder werden
für die Rekursionsbedingung benötigt?

Wie lautet der SELECT für den Satz, von dem die Rekursion ausgehen soll?

Wie lautet die Rekursionsbedingung?

\literatur
\end{document}
