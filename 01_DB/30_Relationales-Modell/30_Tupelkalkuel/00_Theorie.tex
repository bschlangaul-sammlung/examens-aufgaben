\documentclass{lehramt-informatik-haupt}
\liLadePakete{syntax,mathe}

\begin{document}

%%%%%%%%%%%%%%%%%%%%%%%%%%%%%%%%%%%%%%%%%%%%%%%%%%%%%%%%%%%%%%%%%%%%%%%%
% Theorie-Teil
%%%%%%%%%%%%%%%%%%%%%%%%%%%%%%%%%%%%%%%%%%%%%%%%%%%%%%%%%%%%%%%%%%%%%%%%

\chapter{Tupelkalkül
\footcite[Seite 100]{kemper}}

%-----------------------------------------------------------------------
%
%-----------------------------------------------------------------------

\section{Symbole}

\begin{tabular}{l|l|l|l}
\textbf{Name} & \textbf{Symbol} & \textbf{LaTeX} & \textbf{Interpretation}\\

\hline
\hline

\multicolumn{4}{l}{Mengenrelationen} \\

\hline
\hline

% \in
% https://de.wikipedia.org/wiki/Liste_mathematischer_Symbole#Mengenrelationen
Element von & $\in$ & \verb|\in| & das Element $a$  a ist in der Menge $A$ enthalten \\

\hline

\multicolumn{4}{l}{Junktoren} \\

\hline
\hline

% ∧
% https://de.wikipedia.org/wiki/Liste_mathematischer_Symbole#Junktoren
Und & $\land$ & \verb|\land| & Aussage $A$ und Aussage $B$ \\

% ∨
% https://de.wikipedia.org/wiki/Liste_mathematischer_Symbole#Junktoren
Oder & $\lor$ & \verb|\lor| & Aussage $A$ und Aussage $B$ \\

% ⇒
% https://de.wikipedia.org/wiki/Liste_mathematischer_Symbole#Junktoren
folgt & $\Rightarrow$ & \verb|\Rightarrow| & aus Aussage $A$ folgt Aussage $B$ \\

% ¬
% https://de.wikipedia.org/wiki/Liste_mathematischer_Symbole#Junktoren
Negation & $\neg$ & \verb|\neg| & nicht Aussage $A$\\

\hline

\multicolumn{4}{l}{Quantoren} \\

\hline
\hline

% ∀
% https://de.wikipedia.org/wiki/Liste_mathematischer_Symbole#Quantoren
All-Quantor & $\forall$ & \verb|\forall| & für alle Elemente $x$ \\

% \exists
% https://de.wikipedia.org/wiki/Liste_mathematischer_Symbole#Quantoren
Existenzquantor & $\exists$ & \verb|\exists| & es existiert mindestens ein Element $x$\\

\end{tabular}

%-----------------------------------------------------------------------
%
%-----------------------------------------------------------------------

\section{Definitionen}

%%
%
%%

\subsection{Kalkül}

In formalen Wissenschaften wie Logik und Mathematik ein formales System
von Regeln, mit denen sich aus gegebenen Aussagen (Axiomen) weitere
Aussagen ableiten lassen.
\footcite{wiki:kalkuel}

%%
%
%%

\subsection{Kalkül (Datenbank)}

Für die theoretische Betrachtung und die semantisch genaue Definition
von Anfragesprachen für Datenbanken werden Kalkülausdrücke genutzt,
speziell der Tupelkalkül (englisch tuple calculus) und der
Bereichskalkül (auch Domänen-Kalkül, englisch domain calculus).
\footcite{wiki:kalkuel-datenbank}

%%
%
%%

\subsection{Prädikat}

Prädikat (von lateinisch praedicare ‚zusprechen‘) nennt man in der
modernen Prädikatenlogik den Teil einer atomaren Aussage.
\footcite{wiki:praedikat-logik}

Ein Prädikat ist eine Folge von Wörtern mit Leerstellen, die zu einer
wahren oder falschen Aussage wird, wenn in jede Leerstelle ein Eigenname
eingesetzt wird. Zum Beispiel ist die Wortfolge „… ist ein Mensch“ ein
Prädikat, weil durch Einsetzen eines Eigennamens – etwa „Sokrates“ – ein
Aussagesatz, zum Beispiel „Sokrates ist ein Mensch“, entsteht.
\footcite{wiki:praedikatenlogik}

%-----------------------------------------------------------------------
%
%-----------------------------------------------------------------------

\section{Grundlagen Relationales Tupelkalkül}

\begin{itemize}
\item Verwandt mit der Relationalen Algebra, gleich mächtig, aber
deklarativ ausgerichtet

\item Basiert auf dem
\href{https://de.wikipedia.org/wiki/Pr%C3%A4dikatenlogik_erster_Stufe}
{mathematischen Prädikatenkalkül erster Stufe}.

\item Form: $\{t | P(t)\}$

\begin{itemize}
\item $t$: Tupelvariable

\item $|$: bedeutet soviel wie „mit der Eigenschaft, dass...“

\item $P$: Prädikat, das erfüllt sein muss, damit $t$ ins Ergebnis
aufgenommen wird

\item Datenbankprädikate werden als $P(t)$ oder $t \in P$ geschrieben,
auf einzelne Elemente des Tupels (Attribute) greift man durch
Punktnotation mit Angabe des Attributnamens zu: $t.A$ (bzw. durch
Zugriff über den Index $t[i]$)

\item $t$ ist freie Variable, d.h. $t$ darf nicht durch Existenz- oder
Allquantor quantifiziert sein

\item Aufbau nicht existierender Tupel: durch den Tupelkonstruktor:
$\{[t_1.A_1,...,t_n.A_n] | P(t_1,...,t_n)\}$

\end{itemize}
\end{itemize}

%-----------------------------------------------------------------------
%
%-----------------------------------------------------------------------

\section{Beispiel Kunden}

\begin{mdframed}
\noindent
$KUNDE (kdnr, kname, adresse, ort)$

\noindent
$AUFTRAG (auftragsnr, kdnr, warennr, menge)$

\noindent
$WARE (warennr, wname, wpreis)$
\end{mdframed}

\begin{description}
\item[Orte, in denen es Kunden gibt]
$\{t.ort | KUNDE(t)\}$ bzw. $\{t.ort | t \in KUNDE \}$

\item[Alle Kunden aus Bremen]
$\{t | KUNDE(t) \land t.ort=\mlq Bremen \mrq \}$

\item[Kunden mit Bestellung]
$\{t | KUNDE(t) \land \exists s(AUFTRAG(s) \land s.kdnr=t.kdnr)\}$

\item[Waren ohne Bestellung]
$\{t | WARE(t) \land \neg\exists s(AUFTRAG(s) \land s.warennr=t.warennr)\}$
\end{description}

\section{Beispiel Englischlehrer}

Weiteres Beispiel für eine einfache Abfrage: Gesucht sind der Name
und die Schülerzahl aller Klassen, deren Englischlehrer die
Personalnummer 42 hat.

$\{[k.KName, k.Schuelerzahl] | k \in Klasse \land k.Englischlehrer = 42\}$

%-----------------------------------------------------------------------
%
%-----------------------------------------------------------------------

\section{Beispiel Professoren
\footcite{net:html:uni-osnabrueck:dbs-skript}}

\begin{description}
\item[Alle C4-Professoren]
$\{p | p \in \text{Professoren} \land \text{p.Rang} = 'C4'\}$

\item[Alle Professorennamen zusammen mit den Personalnummern ihrer
Assistenten]

$\{[\text{p.Name}, \text{a.PersNr}] | p \in \text{Professoren} \land
a \in \text{Assistenten} \land \text{p.PersNr} = \text{a.Boss}\}$

\item[Alle diejenigen Studenten, die sämtliche vierstündigen Vorlesungen
gehört haben]

$\{s | s \in \text{Studenten} \land \forall
v \in \text{Vorlesungen}(\text{v.SWS} = 4 \Rightarrow
\exists h \in \text{hören}(\text{h.VorlNr} = \text{v.VorlNr} \land
\text{h.MatrNr} = \text{s.MatrNr}))\}$

\item[sichere Ausdrücke]

Für die Äquivalenz des Tupelkalküls mit der Relationenalgebra ist es
wichtig, sich auf sogenannte sichere Ausdrücke zu beschränken, d.h.
Ausdrücke, deren Ergebnis wiederum eine Teilmenge der Domäne ist. Zum
Beispiel ist der Ausdruck

$\{n | \neg (n \in \text{Professoren})\}$

\end{description}

%%%%%%%%%%%%%%%%%%%%%%%%%%%%%%%%%%%%%%%%%%%%%%%%%%%%%%%%%%%%%%%%%%%%%%%%
% Aufgaben
%%%%%%%%%%%%%%%%%%%%%%%%%%%%%%%%%%%%%%%%%%%%%%%%%%%%%%%%%%%%%%%%%%%%%%%%

\chapter{Aufgaben}

%-----------------------------------------------------------------------
%
%-----------------------------------------------------------------------


%-----------------------------------------------------------------------
%
%-----------------------------------------------------------------------

\ExamensAufgabeTTA 66116 / 2018 / 03 : Thema 2 Teilaufgabe 1 Aufgabe 4

%-----------------------------------------------------------------------
%
%-----------------------------------------------------------------------

\ExamensAufgabeTTA 46116 / 2013 / 03 : Thema 1 Teilaufgabe 2 Aufgabe 2

\literatur

\end{document}
