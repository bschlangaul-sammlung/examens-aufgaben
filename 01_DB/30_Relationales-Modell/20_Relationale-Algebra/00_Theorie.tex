\documentclass{lehramt-informatik-haupt}

\liLadePakete{mathe,syntax,relationale-algebra}

\begin{document}

%%%%%%%%%%%%%%%%%%%%%%%%%%%%%%%%%%%%%%%%%%%%%%%%%%%%%%%%%%%%%%%%%%%%%%%%
% Theorie-Teil
%%%%%%%%%%%%%%%%%%%%%%%%%%%%%%%%%%%%%%%%%%%%%%%%%%%%%%%%%%%%%%%%%%%%%%%%

\chapter{Relationale Algebra}

\begin{liQuellen}
\item \cite{net:html:uni-innsbruck:relax}
\end{liQuellen}

\section{Symbole}
% https://dbai.tuwien.ac.at/education/dm/resources/symbols.html

\begin{tabular}{l|l|l|l}
\textbf{Name} & \textbf{Symbol} & \textbf{LaTeX} & \textbf{Alternativtext}\\\hline\hline
% σ
Selektion & $\sigma$ & \verb|\sigma| & SEL\\
% π
Projektion & $\pi$ & \verb|\pi| & PR\\
% ∪
Vereinigung & $\cup$ & \verb|\cup| & UNION\\
% ∩
Durchschnitt & $\cap$ & \verb|\cap| & INTERSECTION\\
% −
Mengendifferenz & $-$ & - & -\\
% ✕
kartesisches Produkt & $\times$ & \verb|\times| & X\\
% ρ ←
Umbenennung (+Zuweisung) & $\rho \leftarrow$ & \verb|\rho \leftarrow| & RENAME\\
% ÷
Division & $\div$ & \verb|\div| & DIV\\

% △
Symmetrische Differenz & $\bigtriangleup$ & \verb|\bigtriangleup| & \\

\hline

% ⋈
Join & $\bowtie$ & \verb|\bowtie| & JOIN\\

% ⟕
%Left Outer Join & {\tiny \textifsym{d|><|}} & \verb|{\tiny \textifsym{d|><|}}| & LOJOIN\\

% Right Outer Join & ⟖ & {\tiny \textifsym{|><|d}} & ROJOIN\\

% Full Outer Join & ⟗ & {\tiny \textifsym{d|><|d}} & FOJOIN\\

% ⋉ needs \usepackage{amssymb}
Left Semi Join & $\ltimes$ & \verb|\ltimes| & LSJOIN\\

% ⋊ needs \usepackage{amssymb}
Right Semi Join & $\rtimes$ & \verb|\rtimes| & RSJOIN\\

\hline

% ∧
Und & $\land$ & \verb|\land| & AND\\

% ∨
Oder & $\lor$ & \verb|\lor| & OR\\

% ¬
Negation & $\neg$ & \verb|\neg| & -\\

% ≥
Größer-gleich & $\geq$ & \verb|\geq| & >=\\

% ≤
Kleiner-gleich & $\leq$ & \verb|\leq| & <=\\

% ≠
Ungleich & $\neq$ & \verb|\neq| & =/=\\

% ≡
Äquivalenz & $\equiv$ & \verb|\equiv| & EQ\\

% ∃
Existenzquantor & $\exists$ & \verb|\exists| & EXISTS\\

% ∀
All-Quantor & $\forall$ & \verb|\forall| & FORALL\\
\end{tabular}

%-----------------------------------------------------------------------
%
%-----------------------------------------------------------------------

\section{Operationen der Relationen Algebra}

\subsection{Mengenoperation}

%%
%
%%

\subsubsection{Vereinigung}

\begin{description}
\item[Symbol-Schreibweise] $R \cup S$
\item[SQL] UNION
\end{description}

%%
%
%%

\subsubsection{Mengendifferenz}

\begin{description}
\item[Symbol-Schreibweise] $R - S$
\item[SQL] EXCEPT
\end{description}

%%
%
%%

\subsubsection{Mengendurchschnitt (Schnittmenge/Intersection)}

\begin{description}
\item[Symbol-Schreibweise] $R \cap S$
\item[SQL] INTERSECT
\end{description}

%%
%
%%

\subsubsection{Symmetrische Differenz}

\begin{description}
\item[Symbol-Schreibweise] $R \bigtriangleup S$
\item[SQL] INTERSECT
\end{description}

%-----------------------------------------------------------------------
%
%-----------------------------------------------------------------------

\subsection{Selektion}

\begin{description}
\item[Symbol-Schreibweise] $\sigma_{Ausdruck}(R)$
\item[lineare Schreibweise] $R[Ausdruck]$
\item[SQL] WHERE
\end{description}

%-----------------------------------------------------------------------
%
%-----------------------------------------------------------------------

\subsection{Projektion}

\begin{description}
\item[Symbol-Schreibweise] $\pi_{\beta}(R)$
\item[lineare Schreibweise] $R[\beta]$
\item[SQL] SELECT
\end{description}

%-----------------------------------------------------------------------
%
%-----------------------------------------------------------------------

\subsection{Kartesisches Produkt (Kreuzprodukt)}

\begin{description}
\item[Symbol-Schreibweise] $R \times S$
\item[lineare Schreibweise] $R~x~S$
\item[SQL] CROSS JOIN
\end{description}

%-----------------------------------------------------------------------
%
%-----------------------------------------------------------------------

\subsection{Umbenennung}

\begin{description}
\item[Symbol-Schreibweise] $\rho_{[{neu}\leftarrow alt]} (R)$
\item[lineare Schreibweise] $R[alt\rightarrow neu]$
\end{description}

%-----------------------------------------------------------------------
%
%-----------------------------------------------------------------------

\subsection{Division\footcite[Division]{relationale-algebra}}

\begin{description}
\item[Symbol-Schreibweise] $R \div S$
\end{description}

Die Division ist dann definiert durch:

$R\div S:=\pi_{{R'}}(R)-\pi_{{R'}}((\pi_{{R'}}(R)\times S)-R)$

\begin{minted}{sql}
SELECT distinct MatrNr
FROM hoert
WHERE MatrNr NOT IN(
  SELECT R.MatrNr
  FROM hoert R, Professor P, Vorlesung V
  WHERE P.Name = 'Sokrates'
  AND P.PersNr=V.gelesenVon
  AND (R.MatrNr, V.VorlNr) NOT IN (
    SELECT MatrNr, VorlNr
    FROM hoert
  )
);
\end{minted}

%-----------------------------------------------------------------------
%
%-----------------------------------------------------------------------

\section{Die 5 Grundoperationen der Relationalen Algebra
\footcite{net:pdf:lmu:dbs}
}

Mit diesen Grundoperationen lassen sich weitere Operationen (z. B. die
Schnittmenge) nachbilden.

\begin{itemize}
\item Verenigung $R = S \cup T$
\item Differenz $R = S - T$
\item Kartesisches Prokukt (Kreuzprodukt) $R = S \times T$
\item Selection $R = \sigma_F(S)$
\item Projektion $R = \pi_{A,B,...}(S)$
\end{itemize}

\literatur

\end{document}
