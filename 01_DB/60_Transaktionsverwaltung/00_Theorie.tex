\documentclass{lehramt-informatik-haupt}

\begin{document}

\chapter{Transaktionsverwaltung}

\section{Definition Transaktion}

Unter einer Transaktion versteht man die \memph{„Bündelung“ mehrerer
Datenbankoperationen}, die in einem Mehrbenutzersystem ohne unerwünschte
Einflüsse durch andere Transaktionen als \memph{Einheit fehlerfrei
ausgeführt} werden sollen.

%-----------------------------------------------------------------------
%
%-----------------------------------------------------------------------

\section{ACID-Prinzip}

\begin{liQuellen}
\cite[Kapitel 9.5 „Eigenschaften von Transaktionen“, Seite 305]{kemper}
\cite[Seite 1]{db:fs:5}
\cite{wiki:acid}
\end{liQuellen}

\begin{description}
\item[Atomicity]

Eine Transaktion ist atomar, d.\,h. von den vorgesehenen
Änderungsoperationen auf die Datenbank haben entweder alle oder keine
eine Wirkung auf die Datenbank.

\item[Consistency]

Eine Transaktion überführt einen korrekten (konsistenten)
Datenbankzustand wieder in einen korrekten (konsistenten)
Datenbankzustand.

\item[Isolation]

Eine Transaktion bemerkt das Vorhandensein anderer (parallel
ablaufender) Transaktionen nicht und beeinflusst auch andere
Transaktionen nicht.

\item[Durability]

Die durch eine erfolgreiche Transaktion vorgenommenen Änderungen sind
dauerhaft (persistent).
\end{description}

%-----------------------------------------------------------------------
%
%-----------------------------------------------------------------------

\section{Operationen auf Transaktions-Ebene}

\begin{liQuellen}
\item \cite[Seite 5]{db:fs:5}
\item \cite[Seite 211]{winter}
\item \cite[Kapitel 9.3, Seite 302-302]{kemper}

\end{liQuellen}

\begin{description}
\item[read \& write]

als Operationen zur Ausführung von Änderungen

\item[begin of transaction (BOT)]

Mit diesem Befehl wird der Beginn einer eine Transaktion darstellende
Befehlsfolge gekennzeichnet.

\item[commit]

Hierdurch wird die Beendigung der Transaktion eingeleitet. Alle
Änderungen der Datenbasis werden durch diesen Befehl festgeschrieben,
d.h. sie werden dauerhaft in die Datenbank eingebaut.

\item[abort]

Dieser Befehl führt zu einem \memph{Selbstabbruch} der Transaktion. Das
Datenbanksystem muss sicherstellen, dass die Datenbasis wieder in den
Zustand zurückgesetzt wird, der vor Beginn der Transaktionsausführung
existierte.
\end{description}

\noindent
Für den Abschluss einer Transaktion gibt es 2 Möglichkeiten:

\begin{itemize}
\item Den erfolgreichen Abschluss mit \texttt{commit}.
\item Den erfolglosen Abschluss mit \texttt{abort}
\end{itemize}

%-----------------------------------------------------------------------
%
%-----------------------------------------------------------------------

\section{Zustandsübergänge einer Transaktion}

\begin{liQuellen}
\item \cite[Kapitel 9.7, Seite 301]{kemper}
\item \cite[Seite 5]{db:fs:5}
\end{liQuellen}

\begin{description}
\item[potentiell]

Die Transaktion ist codiert und „wartet darauf", in den Zustand aktiv zu
wechseln. Diesen Übergang nennen wir inkarnieren.

\item[aktiv]

Die aktiven (d.h. derzeit rechnenden) Transaktionen konkurrieren
untereinander um die Betriebsmittel, wie \zB Hauptspeicher, Rechnerkern
zur Ausführung von Operationen, etc.

\item[wartend]

Bei einer Überlast des Systems: (\zB thrashing (Seitenflattern) des
Puffers) kann die Transaktionsverwaltung einige aktive Transaktionen in
den Zustand wartend verdrängen. Nach Behebung der Überlast werden diese
wartenden Transaktionen sukzessive wieder eingebracht, d.h. wieder
aktiviert.

\item[abgeschlossen]

Durch den commit-Befehl wird eine aktive Transaktion beendet. Die
Wirkung abgeschlossener TAS kann aber nicht gleich in der Datenbank
festgeschrieben werden. Vorher müssen noch möglicherweise verletzte
Konsistenzbedingungen überprüft werden.

\item[persistent]

Die Wirkungen abgeschlossener Transaktionen werden — wenn die
Konsistenzerhaltung sichergestellt ist — durch festschreiben dauerhaft
in die Datenbasis eingebracht. Damit ist die Transaktion persistent.
Dies ist einer von zwei möglichen Endzuständen einer
Transaktionsverarbeitung.

\item[gescheitert]

Transaktionen können aufgrund vielfältiger Ereignisse scheitern. \ZB
kann der Benutzer selbst durch ein abort eine aktive Transaktion
abbrechen. Weiterhin können Systemfehler zum Scheitern aktiver oder
wartender Transaktionen führen. Bei abgeschlossenen Transaktionen können
auch Konsistenzverletzungen festgestellt werden, die ein Scheitern
veranlassen.

\item[wiederholbar]

Einmal gescheiterte Transaktionen sind u.U. wiederholbar. Dazu muss
deren Wirkung auf die Datenbasis zunächst zurückgesetzt werden. Danach
können sie durch Neustarten wiederum aktiviert werden.

\item[aufgegeben]

Eine gescheiterte Transaktion kann sich aber auch als „hoffnungslos"
herausstellen. In diesem Fall wird ihre Wirkung zurückgesetzt und die
Transaktionsverarbeitung geht in den Endzustand aufgegeben über.
\end{description}

%-----------------------------------------------------------------------
%
%-----------------------------------------------------------------------

\section{Fehler}

\section{„Einfacher” Fehlerfall}

Änderungen durch nicht zu Ende geführte Transaktionen müssen rückgängig
gemacht werden. Diesen Vorgang nennt man \emph{Rollback}. Dazu müssen
gegebenenfalls die zuletzt gültigen Werte der geänderten
Datenbankobjekte in die Datenbank geschrieben werden.

Es muss sichergestellt werden, dass die Änderungen erfolgreich zu Ende
geführter Transaktionen tatsächlich permanent gemacht werden. Dazu
müssen gegebenenfalls die neuen Werte der Datenbankobjekte erneut auf
die Datenbank geschrieben werden.

%%
%
%%

\subsection{Das Lost-Update-Problem\footcite{wiki:verlorenes-update}}

Änderungen in einer Datenbank, die durch eine Transaktion vorgenommen
wurden, gehen aufgrund unkontrollierter Parallelausführung verloren.

\emph{„Überschreiben“}

%%
%
%%

\subsection{Das Dirty-Read-Problem\footcite{wiki:schreib-lese-konflikt}}

Der Wert eines Datenobjekts, der noch nicht permanent gespeichert wurde,
wird gelesen.

\emph{„Zwischenstände lesen“}

%%
%
%%

\subsection{Das Unrepeatable-Read-Problem\footcite{wiki:nichtwiederholbares-lesen}}

Während einer Transaktion wird der Wert eines Datenobjekts zweimal
gelesen, wobei dieser Wert aber in der Zwischenzeit von einer anderen
Transaktion geändert wurde.

\emph{„Verschiedene Zwischenstände lesen“}

%-----------------------------------------------------------------------
%
%-----------------------------------------------------------------------

\section{Zweiphasige Abarbeitung mit Sperren\footcite[Seite 15]{db:fs:5}}

Die Abarbeitung von Transaktionen erfolgt zweiphasig oder genügt dem
2-Phasen-Sperrprotokoll, wenn keine Transaktion eine Sperre freigibt,
bevor sie alle benötigten Sperren angefordert hat.

\begin{itemize}
\item Ein strenges 2-Phasen-Sperrprotokoll wird dadurch realisiert, dass

\begin{itemize}
\item eine Transaktion sukzessive alle benötigten Sperren anfordert
(alle Elemente, die eine Transaktion lesen oder ändern möchte, werden
von dieser Transaktion reserviert und dürfen nicht von anderen
Transaktionen abgegriffen werden),

\item alle Lesesperren (\textbf{S}hared oder \textbf{R}ead) bis zum
Ende, d.\,h. bis zur Aktion commit, hält und

\item alle Schreibsperren (e\textbf{X}clusive oder \textbf{W}rite) bis
nach \texttt{commit} hält.
\end{itemize}

\item Eine bereits gesetzte Lesesperre (durch Angabe von \texttt{rlock})
kann durch \texttt{xlock} in eine \emph{Schreibsperre umgewandelt}
werden, ohne dass vorher die Lesesperre aufgehoben werden muss.

\item Reine \emph{Lesesperren} auf ein Element können \emph{mehrere}
Transaktionen gleichzeitig besitzen, dann ist jedoch \emph{keine
Schreibsperre} mehr möglich.
\end{itemize}

%%
%
%%

\subsection{2-Phasen-Sperrprotokoll\footcite[Seite 16]{db:fs:5}}

Die Serialisierbarkeit ist bei Einhaltung des folgenden
Zwei-Phasen-Sperrprotokolls durch den Scheduler gewährleistet. Bezogen
auf individuelle Transaktion wird folgendes verlangt:\footcite{wiki:sperrverfahren}

\begin{enumerate}
\item Jedes Objekt, das von einer Transaktion benutzt werden soll, muss
vorher entsprechend gesperrt werden.

\item Eine Transaktion fordert eine Sperre, die sie schon besitzt, nicht
erneut an.

\item Eine Transaktion muss die Sperren anderer Transaktionen auf dem
von ihr benötigten Objekt gemäß der Verträglichkeitstabelle beachten.
Wenn die Sperre nicht gewährt werden kann, wird die Transaktion in eine
entsprechende Warteschlange eingereiht — bis die Sperre gewährt werden
kann.

\item Jede Transaktion durchläuft zwei Phasen:

\begin{itemize}
\item Eine Wachstumsphase, in der sie Sperren anfordern, aber keine
freigeben darf und

\item Eine Schrumpfungsphase, in der sie ihre bisher erworbenen
Sperren freigibt, aber keine weiteren anfordern darf.
\end{itemize}

\item Bei \texttt{EOT} (Transaktionsende) muss eine Transaktion alle
ihre Sperren zurückgeben
\end{enumerate}

\section{Spezialfälle des 2-Phasensperrprotokolls}

Es gibt zwei Spezialfälle von 2PL:

\begin{description}

\item[Konservatives 2-Phasen-Sperrprotokoll]

Das konservative 2-Phasen-Sperrprotokoll (Preclaiming), bei welchem zu
Beginn der Transaktion alle benötigten Sperren auf einmal gesetzt
werden. Dies verhindert in jedem Fall Deadlocks, führt aber auch zu
einem hohen Verlust an Parallelität, da eine Transaktion ihre erste
Operation erst dann ausführen kann, wenn sie alle Sperren erhalten hat.

\item[Striktes 2-Phasen-Sperrprotokoll]

Das strikte 2-Phasen-Sperrprotokoll, bei welchem alle gesetzten
Write-Locks erst am Ende der Transaktion (nach der letzten Operation)
freigegeben werden. Dieses Vorgehen verhindert den Schneeballeffekt,
also das kaskadierende Zurücksetzen von sich gegenseitig beeinflussenden
Transaktionen. Der Nachteil ist, dass Sperren häufig viel länger
gehalten werden als nötig und sich somit die Wartezeit von blockierten
Transaktionen verlängert. Die Read-Locks werden entsprechend dem
Standard-2PL-Verfahren entfernt.
\footcite{wiki:sperrverfahren}
\end{description}

%%
%
%%

\subsection{Deadlock (bei Transaktionen)}

Deadlock bei den sperrbasierten Synchronisationsmethoden:

\begin{itemize}
\item Beide Transaktionen können zunächst lesen

\item Beim Umwandeln der Lesesperre in eine Schreibsperre in Schritt 9
kommt es zu einem Sperrkonflikt, da T2 noch die Lesesperre hält.

\item Die Transaktion T1 wird vom System verzögert.

\item In Schritt 10 möchte T2 ihre Lesesperre in eine Schreibsperre
umwandeln.

\item Es kommt erneut zu einem Sperrkonflikt, der wiederum zur
Verzögerung der zweiten Transaktion führt.
\end{itemize}

\begin{itemize}
\item Deadlock-Situationen müssen entweder vermieden oder vom DatenbankMS
erkannt und aufgelöst werden.

\item Das DatenbankMS kann in diesem Fall eine der beiden Transaktionen
abbrechen und diese von Neuem starten.
\end{itemize}

%-----------------------------------------------------------------------
%
%-----------------------------------------------------------------------

\section{Recovery-Klassen: Wiederherstellungsmechanismen des DatenbankS nach
Fehlern}

\begin{description}

%%
%
%%

\item[Partial Undo (partielles Zurücksetzen / R1-Recovery)]

Nach Transaktionsfehler

\begin{itemize}
\item Isoliertes und vollständiges Zurücksetzen der veränderten Daten in
den Zustand zu Beginn der Transaktion

\item Beeinflusst andere Transaktionen nicht!
\end{itemize}

%%
%
%%

\item[Partial Redo (partielles Wiederholen / R2-Recovery)]

Nach Systemfehler (mit Verlust des Hauptspeicherinhalts)

\begin{itemize}
\item Wiederholung aller verlorengegangenen Änderungen (waren nur im
Puffer) von abgeschlossenen Transaktionen
\end{itemize}

%%
%
%%

\item[Global Undo (vollständiges Zurücksetzen / R3-Recovery)]

Nach Systemfehler (mit Verlust des Hauptspeicherinhalts), \zB
Stromausfall

\begin{itemize}
\item Zurücksetzen aller durch den Ausfall abgebrochenen Transaktionen
\end{itemize}

%%
%
%%

\item[Global Redo (vollständiges Wiederholen / R4-Recovery)]

Nach Gerätefehler

\begin{itemize}
\item Einspielen einer Archivkopie auf neuen Datenträger und
Nachvollziehen aller beendeten Transaktionen, die nach der letzten
beendeten Transaktion auf der Archivkopie noch ausgeführt wurden
\end{itemize}

\end{description}

%-----------------------------------------------------------------------
%
%-----------------------------------------------------------------------

\section{Puffer-Verwaltung}

\cite[Kapitel 10.2.1 Ersetzung von Puffer-Seiten Seite 311]{kemper}

Wann werden Änderungen, die von Transaktionen ausgelöst wurden, vom
temporären Puffer in die Datenbank geschrieben (permanent)?

%-----------------------------------------------------------------------
%
%-----------------------------------------------------------------------

\section{Verdrängungsstrategien\footcite[Kapitel 10.2.1 Ersetzung von
Puffer-Seiten Seite 311-312]{kemper}}

Ersetzung „schmutziger“ Seiten im Puffer

\begin{description}

\item [No-Steal]

Schmutzige Seiten dürfen nicht aus dem Puffer entfernt und in die
Datenbank übertragen werden, solange die Transaktion noch aktiv ist. Die
Datenbank enthält keine Änderungen nicht-erfolgreicher Transaktionen.
Eine UNDO-Recovery ist nicht erforderlich. langen
Änderungs-Transaktionen können zu Problemen führen, da große Teile des
Puffers blockiert werden

\item [Steal]

Schmutzige Seiten dürfen jederzeit ersetzt und in die Datenbank
eingebracht werden. Die Datenbank kann unbestätigte Änderungen
enthalten. Eine UNDO-Recovery ist erforderlich. Es handelt sich um eine
effektivere Puffernutzung bei langen Transaktionen mit vielen
Änderungen.
\end{description}

%%
%
%%

\subsection{Ausschreibestrategien (EOT-Behandlung)\footcite[Kapitel
10.2.2 Einbringen von Änderungen einer Transaktion Seite
312-313]{kemper}}

\begin{description}

\item [Force]

Alle geänderten Seiten werden spätestens bei EOT (vor COMMIT) in die
Datenbank geschrieben. Bei einem Systemfehler ist keine REDO-Recovery
erforderlich. Die Force-Strategie benötigt einen hohen I/O-Aufwand, da
Änderungen jeder Transaktion einzeln geschrieben werden. Die Vielzahl an
Schreibvorgängen führt zu schlechteren Antwortzeiten, länger gehaltenen
Sperren und damit zu mehr Sperrkonflikten. Große Datenbank-Puffer werden
schlecht genutzt.

\item [No-Force]

Änderungen können auch erst nach dem COMMIT in die Datenbank geschrieben
werden. Die Änderungen durch mehrere Transaktionen werden „gesammelt“.
Beim COMMIT werden lediglich REDO-Informationen in die Log-Datei
geschrieben. Bei einem Systemfehler ist eine REDO-Recovery erforderlich.
Die Änderungen auf einer Seite über mehrere Transaktionen hinweg können
gesammelt werden.\footcite[Seite 25]{db:fs:5}
\end{description}

%-----------------------------------------------------------------------
%
%-----------------------------------------------------------------------

\section{Das WAL-Prinzip\footcite[Kapitel 10.3.5 Seite 318]{kemper}}

Wir hatten gerade bemerkt, dass die Log-Einträge spätestens dann
geschrieben werden müssen, wenn sich der zur Verfügung stehende
Ringpuffer gefüllt hat. Bei der von uns zugrundegelegten
Systemkonfiguration (force, steal, und update-in-place) ist aber
zusätzlich unabdingbar, das sogenannte WAL-Prinzip (Write Ahead Log)
einzuhalten. Dafür gibt es zwei Regeln, die beide befolgt werden müssen:

\begin{enumerate}
\item Bevor eine Transaktion festgeschrieben (committed) wird, müssen
alle „zu ihr gehörenden“ Log-Einträge ausgeschrieben werden.

\item Bevor eine modifizierte Seite ausgelagert werden darf, müssen alle
Log-Einträge, die zu dieser Seite gehören, in das temporäre und das
Log-Archiv ausgeschrieben werden.
\end{enumerate}

Die erste Regel des WAL-Prinzips ist notwendig, um erfolgreich
abgeschlossene Transaktionen nach einem Fehler nachvollziehen (Redo) zu
können. Die zweite Regel wird benötigt, um im Fehlerfall die Änderungen
nicht abgeschlossener Transaktionen aus den modifizierten Seiten der
materialisierten Datenbasis entfernen zu können.

Beim WAL-Prinzip schreibt man natürlich alle Log-Einträge bis zu dem
letzten notwendigen aus — d.\,h. man übergeht keine Log-Einträge, die von
Regel 1. und 2. nicht erfasst sind. Dies ist essentiell, um die
chronologische Reihenfolge der LogEinträge im Ringpuffer zu wahren.

\literatur

\end{document}
