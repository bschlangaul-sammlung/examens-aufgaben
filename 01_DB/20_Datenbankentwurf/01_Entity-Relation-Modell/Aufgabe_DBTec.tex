\documentclass{lehramt-informatik-aufgabe}
\liLadePakete{er}
\begin{document}
\liAufgabenTitel{DBTec}

\section{Aufgabe 3: DBTec
\index{Entity-Relation-Modell}
\footcite{db:ab:1}}

Die Firma \emph{DBTec} fertigt verschiedene Geräte. Für die betriebliche
Organisation dieser Firma soll eine relationale Datenbank eingesetzt
werden. Dabei gilt folgendes:
% Entity: Bauteil
Jedes \mpEntity{Bauteil}, das verwendet wird, hat eine eindeutige
\mpAttribute{Nummer} und eine \mpAttribute{Bezeichnung}, die allerdings
für mehrere verschiedene Bauteile gleich sein kann. Von jedem Teil
werden außerdem der \mpAttribute{Name des Herstellers}, der
\mpAttribute{Einkaufspreis} pro Stück und der am Lager vorhandene
\mpAttribute{Vorrat} gespeichert.

% Entity: Gerät
Jedes herzustellende \mpEntity{Gerät} hat eine eindeutige
\mpAttribute{Bezeichnung}. Auch von jedem schon gefertigten Gerätetyp
soll der aktuelle \mpAttribute{Lagerbestand} gespeichert werden, ebenso
wie der \mpAttribute{Verkaufspreis} des Gerätes. In unserem fiktiven
Betrieb gilt die Regelung, dass Maschinen, die mehr als 1000,- EUR
kosten, unentgeltlich an die Kunden ausgeliefert werden; für Geräte, die
weniger kosten, ist zusätzlich zum Preis eine gerätespezifische
\mpAttribute{Anliefergebühr} zu entrichten. In der Datenbank ist
ebenfalls zu speichern, welche Bauteile für welche Geräte
\mpRelationship{benötigt} werden. Es gibt Bauteile, die für mehrere
Geräte verwendet werden.

% Entity: Kunde
Von jedem \mpEntity{Kunden} werden der \mpAttribute{Name}, die
\mpAttribute{Adresse} und die \mpAttribute{Branche} gespeichert. Es kann
verschiedene Kunden mit demselben Namen oder derselben Adresse geben.
% Relationship: betreutKunde
Außerdem ist zu jedem Kunden vermerkt, wer aus unserer Firma für die
entsprechende \mpRelationship{Kundenbetreuung zuständig} ist.
% Relationship: beliefert
Natürlich ist auch zu speichern, welche Kunden mit welchen Geräten
\mpRelationship{beliefert} werden. Es kann sein, dass gewissen Kunden
für bestimmte Geräte \mpAttribute{Sonderkonditionen} eingeräumt worden
sind, dies soll ggf. ebenfalls in der Datenbank vermerkt werden.

\begin{enumerate}

%%
% (a)
%%

\item Bestimmen Sie die Entity- und die Relationship-Typen mit ihren
Attributen und zeichnen Sie ein mögliches Entity-Relationship-Diagramm!

%%
% (b)
%%

\item Bestimmen Sie zu allen Entity-Typen einen Primärschlüssel und
tragen Sie diese in das Modell ein.

%%
% (c)
%%

\item Bestimmen Sie die Funktionalitäten (1:1, 1:n, n:m) der
Relationship-Typen und tragen Sie diese in das Modell ein.

%%
% (d)
%%

\item In der Firma wird ein neues Betreuungssystem eingeführt. Jeder
\mpEntity{Kundenbetreuer} ist für die Kunden eines festgelegten Bezirks
\mpRelationship{zuständig}. Die \mpEntity{Bezirke} sind
\mpAttribute{durchnummeriert}. Für jeden Bezirk existiert eine
\mpAttribute{Beschreibung}, die nicht näher festgelegt ist. Erweitern
Sie Ihr ER-Modell aus Teilaufgabe a) entsprechend. Bezirke werden nur
festgelegt, wenn es dazu auch Kunden gibt.

\end{enumerate}

\end{document}
